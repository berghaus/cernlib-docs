\chapter[Transition radiation]{Transition radiation}

\section{Method}


Let us consider the mean number of XTR (X-ray transition radiation) quanta 
$d^2\bar{N}/d\omega d\Omega$ generated by relativistic charged particle 
intersecting  the interface between two media 1 and 2 ($1 \rightarrow 2$) 
into forward 
hemisphere in unit energy $\omega$ and  solid angle $\Omega$ . In the 
X-ray region, when one could neglect of refraction and reflection 
of the XTR quanta this value is given by the following expression [Bolot82] :

\begin{equation}
\frac{d^2\bar{N}}{d\omega d\Omega} = 
\frac{\alpha}{4\pi^2}\beta^2
\frac{\sqrt{\varepsilon_2}\sin^2\theta}{\omega}
\left|
\frac{1}{1 - \beta\sqrt{\varepsilon_1 - \varepsilon_2\sin^2\theta}} - 
\frac{1}{1 - \beta\sqrt{\varepsilon_2}\cos\theta}
\right|^2  ,
\end{equation}

where $\alpha$ is the fine structure constant, $\beta$ is the ratio of 
the particle velocity $v$ to the velocity of light $c$ , $\varepsilon_1$  
and  $\varepsilon_2$  are the complex dielectric constants in the medium 1 
and 2  respectively ($d\Omega = 2\pi\sin\theta d\theta$) . In the X-ray 
region $\omega > 1$ keV  the imaginary part of the dielectric constant is 
much less than the real part which in turn is practically equal to the 
well known approximation $(1 - \omega^2/\omega_p^2)$  , where $\omega_p$  
is the plasma energy of the medium . In the ultrarelativistic case the 
radiation is concentrated in the region of small angles 
$\theta \sim 1/\gamma \ll 1$ 
($\gamma$ is the Lorenz factor of the particle). Then the relation (1.1) 
can be transformed to the well known form [Avak75], [Ginzb78], [Dolg93]:

\begin{equation}
\frac{d^2\bar{N}}{d\omega d\xi} = 
\frac{\alpha}{\pi}
\frac{\xi}{\omega}
\left|
\frac{1}{\frac{1}{\gamma^2} + \eta^{(1)} + \xi} - 
\frac{1}{\frac{1}{\gamma^2} + \eta^{(2)} + \xi}
\right|^2  ,
\end{equation}

where $\eta^{(i)} = (\omega_p^{(i)}/\omega)^2$ , and  $\omega_p^{(i)}$  
is the plasma energy in  the i-th medium  ($i = 1,2$). We introduced in 
addition, suitable for the following calculations, value 
$\xi = 2(1 - \cos\theta) \sim \theta^2 \ll 1$, where the angle $\theta$   
is counted  from the particle direction. Generally, the trivial numerical 
integration of (1.2) allows us to simulate the energy and angular distributions 
of the XTR quanta generated. The relation (1.2) however has a sharp peak when 
the denominators are of the order of $1/\gamma^2$  . The proper calculation 
of the narrow region of the maximum  forces one to decrease the integration 
step resulting in the accumulation of  round-off errors and numerical 
unstability of the results. It is more suitable therefore to use 
analytical expressions [Bag98].

Let us derive firstly the expressions for the angular density of XTR 
quanta $d\bar{N}/d\xi$ . Introducing the auxiliary coefficients: 
$x = 1/\omega$; $c = (1/\omega_p^{(1)})^2$ ; $d = (1/\omega_p^{(2)})^2$; 
$\sigma(\gamma,\xi) = \xi + 1/\gamma^2$; 
$a^2 = \sigma c$; $b^2 = \sigma d$, we have

\begin{equation}
\frac{d\bar{N}}{d\xi} = 
\int_{\omega_1}^{\omega_2}\frac{d^2\bar{N}}{d\omega d\xi}d\omega =
\frac{\alpha}{\pi}\xi\left[A + B + C\right] ,
\end{equation}

\[
A = \frac{c^2}{2}\left[\frac{1}{a^2(x^2 + a^2)} + 
\frac{1}{a^4}\ln\frac{x^2}{x^2 + a^2}\right]_{x_2}^{x_1} ,
\]

\[
B = -\frac{cd}{a^2 - b^2}\left[\frac{1}{b^2}\ln\frac{x^2}{x^2 + b^2} -
\frac{1}{a^2}\ln\frac{x^2}{x^2 + a^2}\right]_{x_2}^{x_1} ,
\]

\[
C = \frac{d^2}{2}\left[\frac{1}{b^2(x^2 + b^2)} + 
\frac{1}{b^4}\ln\frac{x^2}{x^2 + b^2}\right]_{x_2}^{x_1} ,
\]

where $x_i = 1/\omega_i,  (i = 1,2)$ .   The expression (1.3)  is sufficiently 
smooth to get the integral spectral distribution of  XTR quanta $\bar{N}(\xi)$ 
by  numerical integration:

\[
\bar{N}(\xi) = \int_{\xi}^{\xi_{max}}\frac{d\bar{N}}{d\xi}d\xi  ,
\]

where  $\xi_{max} \sim (100/\gamma)$  is the upper boundary of the 
angular region where the main part of the XTR quanta is generated.     

The spectral density of XTR quanta $d\bar{N}/\omega$ is given by:

\begin{equation}
\frac{d\bar{N}}{d\omega} = 
\int_{0}^{\xi_{max}}\frac{d^2\bar{N}}{d\omega d\xi}d\xi = 
\frac{\alpha}{\pi\omega}
\left[
\frac{\tilde{a} + \tilde{b}}{\tilde{a} - \tilde{b}}
\ln\frac{\omega + \tilde{b}}{\omega + \tilde{a}} + 
\frac{\tilde{a}}{\omega + \tilde{a}} + 
\frac{\tilde{b}}{\omega + \tilde{b}}
\right]_{0}^{\xi_{max}} ,
\end{equation}

where $\tilde{a} = 1/\gamma^2 + \eta^{(1)}$, 
$\tilde{b} = 1/\gamma^2 + \eta^{(2)}$ . From here for, 
$\omega < \omega_{c}^{(2)} << \omega_{c}^{(1)}$,
($\omega_{c}^{(i)} = \gamma\omega_{p}^{(i)}, (i = 1,2)$ is the characteristic 
energy of XTR in the i-th medium), we have with logarithmic accuracy

\begin{eqnarray}
\bar{N}(\omega)& = &
\int_{\omega}^{\omega_{c}^{(1)}}\frac{d\bar{N}}{d\omega}d\omega \nonumber\\
               & = &
\frac{\alpha}{\pi}
\left[
2\ln\frac{\omega_{c}^{(1)}}{\omega}
\left(\ln\frac{\omega_{c}^{(1)}}{\omega_{c}^{(2)}} - 1\right) - 
\ln^2\frac{\omega_{c}^{(1)}}{\omega_{c}^{(2)}} + 2\ln2 -\frac{\pi^2}{24}
\right].
\end{eqnarray}

One can see from the expression (1.5), that the total number of XTR quanta 
generated by ultrarelativistic particle intersecting the interface between 
two media is small and of the order of some units of $\alpha$ . For example, 
for $\omega = 1$ keV, $\gamma = 10^4$ about $0.034 \approx 5\alpha$    
is generated from the interface between polypropylene and air.

\section{Simulation of the X-ray transition radiation production}

The simulation of XTR quanta generated by relativistic charged particle 
intersecting the interface between two media can be done according to the 
following algorithm. We start with the estimation of the total number of XTR 
quanta based on the Poisson distribution with the mean number according the 
expression (1.5). In those rare events when the number of XTR quanta is not 
equal to zero the energy and angle of each XTR quantum are simulated based 
on the integral distributions obtained by the numerical integration of the 
expressions (1.4) and (1.3) respectively. 

\section{Status of this document}
14.10.98 created by V.Grichine

\begin{thebibliography}{99}

\bibitem[Avak75]{avak} Avakian A.L., Garibian G.M., and,Yang C.
{\em Nucl. Instr. and Meth.  \underline{129}, 303} (1975)

\bibitem[Bag98]{bag} Bagulya A.V. and Grichine V.M.
{Bulletin of Lebedev Institute, no.1-2, 12 } (1998)

\bibitem[Bolot82]{bolot}  Bolotovsky B.M.
{\em Lebedev Institute Proceedings, vol. 140, 95} (1982)

\bibitem[Ginzb78]{ginzb} Ginzburg V.L.and Cytovitch V.N.
{\em Soviet Physics (Uspekhi), vol. 126, 553 } (1978)

\bibitem[Dolg93] {dolg} Dolgoshein B.  
{\em Nucl. Instr. and Meth.  A326, 434} (1993).

\end{thebibliography}
