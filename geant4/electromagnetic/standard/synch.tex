\chapter[Synchrotron radiation]{Synchrotron radiation}

\section{Method}

The spectral distribution of the mean photon number of synchrotron radiation   
$d\bar{N}/d\omega$ produced by an ultrarelativistic electron moving in a 
constant uniform magnetic field along a trajectory with length $L$ can be 
expressed by definition in terms of the spectrum of the mean energy loss of 
the radiation $d\bar{\Delta}/d\omega$  by the following expression \cite{maier}:

\begin{equation}
\frac{d\bar{N}}{d\omega} = \frac{1}{\omega}\frac{d\bar{\Delta}}{d\omega} = 
\frac{\sqrt{3}}{2\pi}\alpha\left(\frac{L\gamma}{R}\right)\frac{1}{\omega_{c}}
\int_{\omega/\omega_{c}}^{\infty}K_{5/3}(\eta)d\eta   .
\end{equation}

Here:    
\begin{eqnarray*}
\omega    &  &  \mbox{photon energy}\\   
\alpha    &  &  \mbox{fine structure constant}\\   
R         &  &  \mbox{instantaneous radius of curvature of the trajectory}\\ 
K         &  &  \mbox{Macdonald function}\\ 
\omega_c  &  &  \mbox{characteristic energy of synchrotron radiation}\\
\end{eqnarray*} 
$\omega_c = 1.5\beta(\hbar c/R)\gamma^3$, $\hbar$ is the Planck constant, 
and $\beta = v/c$  is the ratio of the electron velocity $v$  to the speed of 
light $c$). In the SI system of units: $R(m) = P(GeV/c)/0.3B_{\bot}(T)$ , 
where  $B_{\bot}$ is the component of magnetic flux density perpendicular to 
the electron velocity, and $P$  is the electron momentum (for the case 
considered $R \approx const$   along the whole trajectory) . For the 
simulation of the energy spectrum of synchrotron radiation by the Monte-Carlo 
method we need to calculate the integral distribution of the mean number of 
photons with the energy more than the given one: $\bar{N}_{>\omega}$  . 
Transforming  $d\bar{N}/d\omega$ by the integral representation of 
the Macdonald function \cite{abram}, we have :

\begin{eqnarray}
\bar{N}_{>\omega}& = &
\int_{\omega}^{\infty}\frac{d\bar{N}}{d\omega'}d\omega' \nonumber\\
& = &
\frac{\sqrt{3}}{2\pi}\alpha\left(\frac{L\gamma}{R}\right)
\int_0^{\infty}\frac{\cosh\left(\frac{5}{3}t\right)}{\cosh^2(t)}
\exp\left[-\frac{\omega}{\omega_c}\cosh(t)\right]dt .
\end{eqnarray}

The latter integral is calculated numerically by the quadrature Laguerre 
formula \cite{korn} . The calculations show that taking into account 
about 50 roots of 
the Laguerre polynomials results in an accuracy of the integral estimation 
better than $10^{-4}$ [Bag98] . The mean number of synchrotron radiation photons
$\bar{N}$  (= $\bar{N}_{>0}$) produced by an ultrarelativistic electron 
moving in a constant magnetic field along a trajectory with the length $L$
  , is given by :

\begin{eqnarray}
\bar{N} = \bar{N}_{>0}& = & 
\frac{\sqrt{3}}{2\pi}\alpha\left(\frac{L\gamma}{R}\right)
\int_0^{\infty}\frac{\cosh\left(\frac{5}{3}t\right)}{\cosh^2(t)}dt \nonumber\\
& = &
\frac{5}{2\sqrt{3}}\alpha\left(\frac{L\gamma}{R}\right) \approx 
10^{-2}\left(\frac{L\gamma}{R}\right) .
\end{eqnarray}

Qualitatively this result can be manipulated using the fact that the mean 
number of photons produced along the formation zone length 
$z \approx R/\gamma$  is proportional to $\alpha$  .
 Then for the length $L$ , $\bar{N} \approx \alpha L/(R/\gamma)$ . Note, that 
for the ultrarelativistic case $\gamma\gg1$   , when $R \sim \gamma$ , 
$\bar{N}$ does not depend on the electron energy but is defined by the 
values of $L$ and $B_{\bot}$ only. The mean energy loss for synchrotron 
radiation $\bar\Delta$ , corresponding to a trajectory with the length $L$, 
experiences the essential relativistic rise instead :

\begin{equation}
\bar{\Delta} = \int_0^{\infty}\omega\frac{d\bar{N}}{d\omega}d\omega =
\frac{2}{3}\alpha\hbar c\left(\frac{L\gamma^2}{R^2}\right)\beta\gamma^2 =   
\frac{8\bar{N}}{15\sqrt{3}}\omega_{c} 
\approx 0.31\bar{N}\omega_{c} \sim \gamma^2
\end{equation}

The angular distribution of synchrotron radiation produced by ultrarelativistic 
electron shows the clear 'searchlight' effect. The main number of photons is 
radiated in the angle limit of the order of $1/\gamma$   around the electron 
trajectory direction. In the interesting region of $\gamma > 10^3$ 
the angular resolution of 
X-ray and gamma detectors usually does not allow us to measure the details 
of the angular distribution of synchrotron radiation. Therefore, the angular 
distribution is roughly set to be flat in the range $0 - 1/\gamma$ .

\section{Simulation of synchrotron radiation}

The distance $x$ along electron/positron trajectory to the next point of 
creation of synchrotron radiation 
photon is simulated according to exponential distribution,
$exp(-x\bar{N}/L)$. The energy $\omega$ of synchrotron radiation photon 
is simulated
according the distribution $\bar{N}_{>\omega}/\bar{N}$. The angles $\theta$ 
and $\varphi$ counted from the electron trajectory (the electron momentum 
is considered to be parallel to the local z-axis) are rouhgly set to be 
distributed randomly in the ranges
$0 - 1/\gamma$ and $0 - 2\pi$ respectively

\section{Status of this document}
15.10.98 created by V.Grichine

\begin{thebibliography}{99}

\bibitem[Abram64]{abram} Edited by M. Abramovwitz and I.A. Stegan.
{\em Handbook of Mathemaatical Functions,
 NBS Applied Mathematics Series 55} (1964)

\bibitem[Bag98]{bag} Bagulya A.V. and Grichine V.M.
{Bulletin of Lebedev Institute, no.9-10, 7 } (1998)

\bibitem[Korn61]{korn}G.A. Korn and T.M. Korn. 
{\em Mathematical Handbook for scientists and 
    engineers, McGRAW-HILL BOOK COMPANY, INC} (1961)

\bibitem[Maier91]{maier}  R. Maier.
{\em Synchrotron Radiation, CERN Report 91-04, 97-115} (1991)

\end{thebibliography}




