
\chapter[$e^+ e^-$ annihilation]{$e^+ e^-$ annihilation}
\section{Cross section and Mean free path}
\subsection{Cross section per atom}
For the annihilation in fly of $e^+ e^-$ into two photons 
the cross-section formula of Heitler is used \cite{heitler,egs4}: 

\begin{eqnarray*}
\sigma(Z,E) & = & \frac{Z \pi r^2_0}{\gamma +1}
                  \left[\frac{\gamma^2 + 4 \gamma +1}{\gamma^2 -1}
                  \ln \left(\gamma +\sqrt{\gamma^2 -1} \right)  -\frac
                  {\gamma +3}{\sqrt{\gamma^2 -1}} \right]  \\
E      & = & \mbox{total energy of the incident positron} \\
\gamma & = & E/m_e c^2 \\
r_0    & = & \mbox{classical electron radius}      
\end{eqnarray*}

\subsection{Mean free path}
\begin{itemize}
\item[*]
         In a simple material the number of atoms per volume is:
         \[n  = \frac{\mathcal{N}\rho}{A}\]
         where:
         \begin{eqnarray*}
          \mathcal{N} &  & \mbox{Avogadro's number} \\
          \rho        &  & \mbox{density of the medium} \\
          A           &  & \mbox{mass of mole} 
         \end{eqnarray*}
\item[*]
         In a compound material the number of atoms of Element elm per volume is:
         \[n_{elm}  = \frac{\mathcal{N}\rho w_{elm}}{A_{elm}}\]
         where:
         \begin{eqnarray*}
          \mathcal{N} &  & \mbox{Avogadro's number} \\
          \rho        &  & \mbox{density of the medium} \\
          w_{elm}     &  & \mbox{proportion by mass of the Element elm}\\
          A_{elm}     &  & \mbox{mass of mole of the Element elm} 
         \end{eqnarray*} 
\item[*] 
         The mean free path, $\lambda$, for a positron to be annihilated with
         an electron is given by
         \[
           \lambda(E_{\gamma}) \equiv \frac{1}{\Sigma (E_{\gamma})} 
             = \frac{1}{\sum_{elm}{\lbrack n_{elm} \sigma(Z_{elm},E_{\gamma})\rbrack}}
         \]
         where $\sum_{elm}$ runs over all Elements the material is made of.
\end{itemize}
 
\section {Sampling the final state}
\subsection{kinematical limits}
\[e^+ \; e^- \to \gamma_a \; \gamma_b \]
\noindent 
The incident $e^+$ has a kinetic energy $T$, a total energy: $E_e = T + mc^2$ \\
and a momentum: $Pc = \sqrt{T(T+2mc^2)}$ \\
The total available energy is: $E_{tot} = E_e + mc^2 = E_a + E_b$  \\
The momentum conservation is: $ \vec{P} = \vec{P}_{\gamma_a} + \vec{P}_{\gamma_b}$ 


\noindent
Lets define the ratio of energy transfered to one photon (says $\gamma_a$):
\[ \epsilon = \frac{E_a}{E_{tot}} \equiv \frac{E_a}{T+2mc^2} \]
\\
The energy transfered to $\gamma_a$ is maximum when $\gamma_a$ is emitted in the
direction of the incident $e^+$. Then $E_a \mathsf{max} = (E_{tot}+Pc)/2$
\\
The energy transfered to $\gamma_a$ is minimum when $\gamma_a$ is emitted in the opposite
direction of the incident $e^+$. Then $E_a \mathsf{min} = (E_{tot}-Pc)/2$

\noindent Therefore:
\begin {eqnarray*}
  \epsilon_{max} &=& \frac{E_a \mathsf{max}}{E_{tot}} =
  \frac{1}{2} \left\lbrack 1+ \sqrt{\frac{\gamma - 1}{\gamma+1}} \right\rbrack \\ 
  \epsilon_{min} &=& \frac{E_a \mathsf{min}}{E_{tot}} =
  \frac{1}{2} \left\lbrack 1- \sqrt{\frac{\gamma - 1}{\gamma+1}} 
  \right\rbrack
\end {eqnarray*}
\\
where $\qquad \gamma = (T + mc^2)/mc^2$


\noindent  
Therefore the range of $\epsilon$ is: 
 $\quad \lbrack \epsilon_{min} \; ; \; \epsilon_{max} \rbrack $
 $\qquad ( \equiv \lbrack \epsilon_{min} \; ; \; 1-\epsilon_{min} \rbrack) $
      
\subsection{sample the gamma energy}
The differential cross-section of the two-photon
positron-electron annihilation can be written as 
\cite{heitler,egs4}:
\[
   \frac{d \sigma (Z, \epsilon)} {d \epsilon} =
   \frac{Z \pi r_0^2}{\gamma - 1} \left( \frac{1}{\epsilon} \right)
   \left[
   1+\frac{2\gamma}{(\gamma+1)^2}-\epsilon-\frac{1}{(\gamma+1)^2}\frac{1}{\epsilon}
   \right]
\]
where
$Z$ is the atomic number of the material, $r_{0}$ the classical electron 
radius, and $\epsilon \in [ \epsilon_{min} \; ; \; \epsilon_{max} ]$

\noindent
The differential
cross-section can be decomposed as:
\[
   \frac{d \sigma (Z, \epsilon)} {d \epsilon} =
   \frac{Z \pi r_0^2}{\gamma - 1} 
   \alpha f(\epsilon) g(\epsilon)
\]
where: 
\begin{eqnarray*}
\alpha      &=& \ln (\epsilon_{max}/\epsilon_{min}); \qquad  
f(\epsilon)  = 1/(\alpha \epsilon) \\
g(\epsilon) &=& \left[
    1+\frac{2\gamma}{(\gamma+1)^2}-\epsilon-\frac{1}{(\gamma+1)^2}\frac{1}{\epsilon}
    \right]  \equiv  
    1-\epsilon+\frac{2 \gamma \epsilon -1}{\epsilon (\gamma +1)^2}
\end{eqnarray*}
Given 2 random numbers $r, r' \in ]0,1[$, 
the photon energies are sampled as follow:
\begin{enumerate}
\item 
  sample $\epsilon$ from $f(\epsilon):
  \epsilon =\epsilon_{min} \left( \frac{\epsilon_{max}}{\epsilon_{min}} \right)^r$
\item
  test the rejection function: if $g(\epsilon) \geq r'$ accept
  $\epsilon$, otherwise return to step 1.
\end{enumerate}

\noindent
Then the photon energies are:
 $E_a = \epsilon E_{tot} \qquad E_b = (1-\epsilon) E_{tot}$

\subsection{compute the final kinematic}
Lets be $\theta$ the angle between the incident $e^+$ and $\gamma_a$. From the
energy-momentum conservation we have:
\[
\cos \theta = \frac{1}{Pc} \left[ T+mc^2 \frac{2\epsilon -1}{\epsilon} \right]
 = \frac{\epsilon(\gamma +1) - 1}{\epsilon \sqrt{\gamma^2 -1}}
\] \\
The azimuthal angle, $\phi$, is generated isotropically and 
the momentum vector of the photons, $\vec{P_{\gamma_a}}$ and $\vec{P_{\gamma_b}}$
are computed from the energy-momentum conservation and
transformed into the {\tt World} coordinate system. 

\subsection{annihilation at rest} 
The method \textbf{AtRestDoIt} treates the special case when a positron comes 
at rest before annihilating.
It generates two photons with energy $k=m$.
The angular distribution is isotropic.

\section{Restrictions}
\begin{enumerate}
\item
The annihilation processes producing one or three or more photons are
ignored, because these processes are negligible compared to the 
annihilation into two photons\cite{egs4,messel};
\item
In calculating the process it is assumed that the atomic electron initially
is free and at rest. This is the usual assumption used in shower programs
\cite{egs4};
\end {enumerate}
\section{Status of this document}
 9.10.98  created by M.Maire.
   
\begin{thebibliography}{99}
\bibitem[Heit54]{heitler} W. Heitler.
   {\em The Quantum Theory of Radiation, Clarendon Press, Oxford} (1954)
\bibitem[Mess70]{messel} H. Messel and D. Crawford.
   {\em Electron-Photon shower distribution, Pergamon Press} (1970)
\bibitem[egs4]{egs4} R. Ford and W. Nelson.
   {\em SLAC-265, UC-32} (1985)      
\end{thebibliography}

