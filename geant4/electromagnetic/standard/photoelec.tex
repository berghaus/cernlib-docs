
\chapter[PhotoElectric effect]{PhotoElectric effect}
\section{Cross section and Mean free path}

\subsection{binding energy of the shells} 
The binding energy of the inner shells have been parameterised as:
\[
   B_i (Z) = Z^2 (a_i + b_i Z + c_i Z^2 + d_i Z^3 )  
\]
where $ i = K, L_1, L_2 $, and the constants $a_i, b_i, c_i,d_i$
are tabulated inside dedicated functions.

\subsection{Cross section per atom}
Lets $E_{\gamma} =$ incident gamma energy, and $\epsilon = E_{\gamma}/m_{e}c^2 $ \\
The photoelectric total cross-section per atom has been parameterised as:
\[
   \sigma(Z,\epsilon)=\frac{Z^{\alpha}} {\epsilon^{\beta}} F(Z,\epsilon)
\]
\\ 
where $\alpha$ and $\beta$ are results of a fit, and $F(Z,\epsilon)$ is:
\[
  \begin{array}{lll}
  \mbox{for\quad} E_{\gamma} > B_K: & F=  p_{1K }/Z  + p_{2K }/\epsilon  + p_{3K } 
                           & + p_{4K }Z + p_{5K }\epsilon +p_{6K } Z^2 \\
                       & & + p_{7K }Z\epsilon + p_{8K }\epsilon^2 + p_{9K }Z^3 \\                          
       & & + p_{10K} Z^2\epsilon + p_{11K} Z/\epsilon^2 + p_{12K}\epsilon^3 \\
  \mbox{for\quad} E_{\gamma} \in ]B_{L1}, B_K]: &
                                   F= p_{1L1}/Z + p_{2L1}/\epsilon + p_{3L1} \\
  \mbox{for\quad} E_{\gamma} \in ]B_{L2}, B_{L1}]: &                                    
                                   F= p_{1L2}/Z + p_{2L2}/\epsilon + p_{3L2} \\
  \mbox{for\quad} E_{\gamma} \leq B_{L2}: & F= p_{1M}
  \end{array}
\]
\\ 
The fit was made over 301 data points chosen between:
\[5 \leq Z \leq 100\quad \mbox{and} \quad 10 \; keV \leq E \leq 50\; MeV \]
The values of the parameters are defined within the method which computes the
cross section per atom. \\
The accuracy of the fit is estimated to be:
\begin{displaymath}
 \frac{\Delta \sigma}{\sigma} \leq 
    \left\{
        \begin{array}{ll} 25\% & \mbox{near to the peaks} \\
                          10\% & \mbox{elsewhere.}
         \end{array}
    \right.
\end{displaymath}

\subsection{Mean free path}

\begin{itemize}
\item[*]
         In a simple material the number of atoms per volume is:
         \[n  = \frac{\mathcal{N}\rho}{A}\]
         where:
         \begin{eqnarray*}
          \mathcal{N} &  & \mbox{Avogadro's number} \\
          \rho        &  & \mbox{density of the medium} \\
          A           &  & \mbox{mass of mole} 
         \end{eqnarray*}
\item[*]
         In a compound material the number of atoms of Element elm per volume is:
         \[n_{elm}  = \frac{\mathcal{N}\rho w_{elm}}{A_{elm}}\]
         where:
         \begin{eqnarray*}
          \mathcal{N} &  & \mbox{Avogadro's number} \\
          \rho        &  & \mbox{density of the medium} \\
          w_{elm}     &  & \mbox{proportion by mass of the Element elm}\\
          A_{elm}     &  & \mbox{mass of mole of the Element elm} 
         \end{eqnarray*} 
\item[*] 
         The mean free path, $\lambda$, for a photon to interact via photo electric
         effect is given by
         \[
           \lambda(E_{\gamma}) \equiv \frac{1}{\Sigma (E_{\gamma})} 
             = \frac{1}{\sum_{elm}{\lbrack n_{elm} \sigma(Z_{elm},E_{\gamma})\rbrack}}
         \]
         where $\sum_{elm}$ runs over all Elements the material is made of.
\end{itemize}

\section{final state}
\subsection{choose an Element}
The binding energy of the shells depend of the atomic number $Z_{elm}$. \\
In a compound material one choose randomly an Element on the basis of the
probability:
\[
  Prob(Z_{elm},E_{\gamma}) = 
                      \frac{n_{elm} \sigma(Z_{elm},E_{\gamma})}{\Sigma (E_{\gamma})}
\]
\subsection{final state}
The simulation is presently rather crude.             \\
A quanta can be absorbed if $E_{\gamma} > B_{shell}$.
The photoelectron is emitted with kinetic energy:
\[T_{photoelectron} = E_{\gamma}-B_{shell}(Z_{elm})\]
\\ 
The electron has the same direction as the incident gamma.
\section{Status of this document}
 9.10.98  created by M.Maire.
