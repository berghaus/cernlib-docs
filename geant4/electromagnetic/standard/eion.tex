\chapter[Ionisation]{Ionisation}
 

 The class G4eIonisation calculates the continuous energy loss
  due to ionisation and
 simulates the 'discrete' part of the ionisation , the Moller and Bhabha
scattering or delta ray production by $e^-/e^+$ . The formulae used here
 are the same than those in GEANT3 (\cite{eion.geant3}).

\section{Method}

Let: \[\frac{d\sigma(E,T)}{dT}\]
be the differential cross-section
for the ejection of an electron with kinetic energy $T$ by an incident
${e^{\pm}}$ of total energy $E$ moving in a medium of density $\rho$.
 Let us denote the value of the {\em kinetic energy cut-off} or
 {\em production threshold for $\delta$-rays} by  $T_{cut}$.
Below this threshold the soft
electrons ejected are simulated as continuous energy loss by the incident
$e^+/e^-$, and above it they are explicitly generated.

The mean value of the energy lost by the incident ${\rm e^{\pm}}$ to
the soft $\delta$-rays is:
\begin{equation}
E_{soft}(E,T_{cut})
= \int_{0}^{T_{cut}} \frac{d \sigma (E,T)}{dT} T \: dT  \label{eion.a}
\end{equation}
whereas the total cross-section for the ejection of
an electron of energy \linebreak $T > T_{cut}$ is:
\begin{equation}
\sigma (E,T_{cut})
= \int_{T_{cut}}^{T_{Max}}\frac{d \sigma (E,T)} {dT} \: dT \label{eion.b}
\end{equation}
where $T_{Max}$ is the maximum energy transferable to the free electron:
\begin{equation}
\label{eion.c}
T_{Max} = \left\{ \begin{array}{ll}
             E-m & {for \hspace{.2cm} e^+}  \\
             (E-m)/2 & {for \hspace{.2cm} e^- ,} \\
              \end{array} \right .
\end{equation}
where $m$ is the electron mass.
The method of calculation of the continuous energy loss
and the total cross-section are
explained below. 

\section{Continuous energy loss}
The integration of \ref{eion.a} leads to the Berger-Seltzer
formulae ( see e.g in 
   \cite{eion.messel}) :

\begin{equation}
\label{eion.d}
\frac {dE}{dx} = \frac{2 \pi r_0^ 2 mn }{\beta^2}
       \left [\ln \frac{2(\tau + 2)} {(I/m)^2}+ F^{\pm} (\tau , \Delta )
- \delta \right ],
\end{equation}
 where

\[
\begin{array}{ll}
\gamma           & \frac{E}{m}                           \\
\beta^2          & 1-\frac{1}{\gamma^2}                  \\
\tau             & \gamma-1                              \\
\tau_c           & \frac{T_{cut}}{m}           \\
 T_{cut}         &  \mbox{energy cut for} \: e^{\pm}      \\
\tau_{max}       & \mbox{maximum possible energy transfer in $e^-$ mass:
  $\tau$ for $e^+$, $\tau/2$ for $e^-$}  \\
\Delta           & \min(\tau_c,\tau_{max})              \\
n                & \mbox{electron density of the medium}        \\
I                & \mbox{average mean ionisation energy}        \\
\delta           & \mbox{density effect correction}.
\end{array}
\]

The functions $ F^{\pm}$  are given by
\begin{equation}
\label{eion.e}
F^+ (\tau,\Delta) =\ln(\tau\Delta ) -
\frac{\Delta^2}{\tau}\left[\tau + 2 \Delta -
\frac{3\Delta^2 y } {2} -\left(\Delta - \frac{\Delta^3 }{3} \right) y^2 
- \left (\frac{\Delta^2}{2} - \tau
       \frac{\Delta^3}{3} + \frac{\Delta^4 } {4} \right)
          y^3  \right]  
\end{equation}

\begin{equation}
\label{eion.f}
F^- (\tau,\Delta ) = -1 -\beta^2 +\ln \left [(\tau - \Delta)
\Delta \right ] + \frac{\tau}{\tau -\Delta}+\frac{\left [
\frac{\Delta^2}{2} + ( 2\tau +1) \ln
\left (1- \frac{\Delta}{\tau} \right ) \right ]}{\gamma^2}
\end{equation}

where $ y = 1/(\gamma+1) $.

The density effect correction is calculated according to the
 formalism of Sternheimer ( \cite{eion.sternheimer}). 

\section{Total cross-sections}
The integration of formula \ref{eion.b} gives the total cross-section
 (\cite{eion.messel}) for M\"{o}ller scattering ($e^- e^-$):
\begin{equation}
\label{eion.g}
\sigma ( Z,E,T_{cut} ) =\frac {2 \pi r_0^2 mZ}{\beta^2(E-m)}
 \left[\frac{(\gamma-1)} {\gamma^2}\left(\frac{1}{x}-1\right)
             +\frac{1}{x}-\frac{1}{1-x}-\frac{2\gamma-1}{\gamma^2}\ln
\frac{1-x}{x}\right]
\end{equation}
and for Bhabha scattering ($e^+ e^-$) :
\begin{equation}
\label{eion.h}
\sigma (Z,E,T_{cut}) =\frac{ 2 \pi r^2_0  mZ }
      {(E-m)}\left [\frac {1 }{\beta^2}  \left(\frac{1}{x}-1\right)
  + B_1 x + B_2 (1-x) -
 \frac {B_3 } {2} ( 1-x^2 ) +\frac{B_4}{3}(1-x^3)\right]
\end{equation}
where

\[
\begin{array}{ll}
\gamma = \frac{E}{m}       & \beta^2 = 1-\frac{1}{\gamma^2} \\ [.2cm]
 x     =\frac {T_{cut}}{E-m}  & \gamma=\frac{1}{\gamma + 1} \\ [.2cm]
 B_1=2-y^2                 &  B_2=(1-2y)(3+y^2)              \\ [.2cm]
B_3=(1-2y)^2+(1-2y)^3      &  B_4=(1-2y)^3
\end{array}
\]

The formulae [\ref{eion.g}] and [\ref{eion.h}]
give the total cross-section of the scattering
above the threshold energies

\begin{equation}
T_{\rm Moller}^{\rm thr} =2T_{cut}  \mbox{\hspace{2cm}and\hspace{2cm}}
T_{\rm Bhabha}^{\rm thr} = T_{cut}
\end{equation}
The interaction length for the production of $\delta$-rays is calculated
during initialisation .

\section{Simulation of the $\delta -ray$ production}
\subsection{Differential cross section}

The differential cross-section of the $\delta$-ray production can
be written as in equation \ref{eion.i} , \ref{eion.j}
 (\cite{eion.messel}). For the
electron/electron (M\"{o}ller) scattering we have:
\begin{equation}
\label{eion.i}
\frac{d\sigma }{d \epsilon }=\frac{2 \pi Z r^2 _0 m }{\beta^2 (E-m)}
\left[ \frac{(\gamma -1 )^2}  {\gamma^2 }+\frac{1}{\epsilon}
\left(\frac{1}{\epsilon}-\frac{2 \gamma -1 } {\gamma^2 } \right) +
\frac{1}{1- \epsilon}\left(\frac{1} {1- \epsilon} \frac{2 \gamma - 1}
{\gamma^2 }\right)  \right]
\end{equation}
and for the positron-electron (Bhabha) scattering:
\begin{equation}
\label{eion.j}
\frac{d \sigma}{d \epsilon}=\frac{2 \pi Z r^2_0 m }{(E-m)}\left[
\frac{1} {\beta^2 \epsilon^2}-\frac{B_1}{\epsilon}+B_2 - B_3 \epsilon
+ B_4 \epsilon^2\right]
 \end{equation}

where

\[
\begin{array}{lcllcl}
Z        & = & \mbox{atomic number of the medium}   &
E        & = & \mbox{energy of the incident particle} \\
M        & = & \mbox{rest mass of the incident particle} &
\gamma   & = & \frac{E}{M}   \\ [.2cm]
\beta^2  & = & 1- \frac{1} {\gamma^2 } &
 y       & = & \frac{1} {\gamma + 1}               \\ [.2cm]
 B_1     & = & 2-y^2                   &
 B_2     & = & (1-2y)(3+y^2 )                       \\
 B_3     & = & (1-2y)^2+(1-2y)^3 &
 B_4     & = & (1-2y)^3  \\
\epsilon & = & \frac{T} {E-m}
\end{array}
\]

with $T$ the kinematic energy of the scattered electron (of the lower
energy in the case of $e^+ e^-$ scattering).

The kinematical limits for the variable $\epsilon$ are:

\[
\epsilon_0 = \frac{T_{cut}}{E-m} \leq \epsilon \leq \frac{1}{2}
\mbox{\hspace{.2cm} for $e^- e^-$} \hspace{2cm}
\epsilon_0 = \frac{T_{cut}}{E-m} \leq \epsilon \leq 1
\mbox{\hspace{.2cm} for $e^+ e^-$}
\]

\subsection{Sampling}

Apart from the normalisation, the cross-section can be written as
\[
\frac{d\sigma}{d\epsilon}=f(\epsilon) g(\epsilon),
\]
where, for $e^- e^-$ scattering,
\begin{eqnarray*}
f(\epsilon)&=&\frac{1}{\epsilon^2} \frac{\epsilon_0 }{1- 2\epsilon_0} \\
g(\epsilon)&=&\frac{4}{9\gamma^2 - 10 \gamma + 5}\left[(\gamma -1)^2
\epsilon^2 - (2 \gamma^2 +2\gamma -1) \frac{\epsilon} {1- \epsilon }+
\frac{\gamma^2}{(1- \epsilon )^2 }\right]
\end{eqnarray*}
and for $e^+ e^-$ scattering
\begin{eqnarray*}
  f(\epsilon)&=&\frac{1}{\epsilon^2} \frac{\epsilon_0}{1- \epsilon_0 } \\
  g(\epsilon)&=&\frac{B_0 -B_1 \epsilon +B_2 \epsilon^2
     -B_3 \epsilon^3 +B_4 \epsilon ^4}{B_ 0-B_1\epsilon_0
+B_2\epsilon^2_0
    -B_3 \epsilon^3_0 +B_4 \epsilon^4_0 }
\end{eqnarray*}
Here $ B_0=\gamma^2/(\gamma^2-1)$ and
all the other quantities have been defined above.

The variable $\epsilon$ is sampled by:
\begin{enumerate}
\item sample $\epsilon$ from $f(\epsilon)$
\item calculate the rejection function $g(\epsilon)$ and accept the
sampled $\epsilon$ with a probability of $g(\epsilon)$.
\end{enumerate}

After the successful sampling of $\epsilon$,  the direction
 of the scattered electron is generated with respect to the direction of the
incident particle. The azimuthal angle $\phi$ is generated isotropically;
the polar angle
$\theta$ is calculated from the energy momentum conservation.
This information
is used to calculate the energy and momentum of both scattered
particles and to transform them into the {\em global} coordinate system.

\section{Status of this document}

 9.10.98     created by L. Urb\'an.

\begin{thebibliography}{99}

\bibitem[GEANT3]{eion.geant3}
  GEANT3 manual ,CERN Program Library Long Writeup W5013 (October 1994).
\bibitem[Mess70]{eion.messel}
  H.Messel and D.F.Crawford. Pergamon Press,Oxford,1970.
\bibitem[Ster71]{eion.sternheimer}
  R.M.Sternheimer. Phys.Rev. B3 (1971) 3681.

\end{thebibliography}

