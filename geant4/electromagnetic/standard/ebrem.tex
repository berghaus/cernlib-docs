\chapter[Bremsstrahlung]{Bremsstrahlung}
 

 The class G4eBremsstrahlung calculates the continuous energy loss
  due to {\em sub-cutoff} photons and
 simulates the 'discrete' part of the interaction ,
the bremsstrahlung  by $e^-/e^+$ . The formulae used here
 are the very similar to those used in GEANT3 (\cite{ebrem.geant3}).
 The only {\em important} difference is that in GEANT4 both of the LPM 
effect and dielectric supression of the bremsstrahlung 
(\cite{ebrem.galitsky},\cite{ebrem.anthony}) have been implemented,
while GEANT3 contains the latter only (using the name Migdal correction
for it) .

\section{Cross section and energy loss}

Let's call $d\sigma(Z,T,k)/dk$ the differential cross-section for
production of a photon of energy $k$ by an electron of kinetic energy
$T$ in the field of an atom of charge $Z$, and
$k_c$ the energy cut-off below which the soft photons are treated
as continuous energy loss .
Then the mean value of the energy lost by the electron due
to soft photons is
\begin{equation}
  E_{Loss}^{brem} (Z,T,k_c ) =
\int_{0}^{k_ c}k\frac{d \sigma (Z,T,k)}{dk}dk
\end{equation}
whereas the total cross-section for the emission of
a photon of energy larger than $k_c$ is
\begin{equation}
 \sigma_{brem} (Z,T,k_c ) = \int_{k_c}^{T}\frac{d \sigma (Z,T,k)}{dk} dk
\end{equation}
Many theories of the bremsstrahlung process exist, each with its own
limitations and regions of applicability. Perhaps the best synthesis
of these theories can be found in the paper of S.M. Seltzer and M.J.
Berger \cite{ebrem.seltzer}. The authors give a tabulation of the
bremsstrahlung cross-section $d\sigma/dk$ differential in the photon
energy $k$, for electrons with kinetic energies $T$ from 1 keV to 10
GeV. For electron energies above 10 GeV the screened Bethe-Heitler
differential cross-section can be used
(\cite{ebrem.egs4}  
together with the dielectric supression
correction (\cite{ebrem.messel} , \cite{ebrem.migdal} ) .
  This correction      
 decreases the differential cross-section at photon
energies below a certain fraction of the incident electron energy
($d\sigma/dk$ decreases significantly if $ k/T \leq 10^{-4}$.)

\subsection{Parameterisation of energy loss and total cross-section}
Using the tabulated cross-section values of Seltzer and Berger together
with the corrected Bethe-Heitler formula (which so includes the 
 dilelectric supression,too) we have computed
$\sigma(Z,T,k_c ) $
and we have used these computed values as ``data points''
in the fitting procedure.
Calculating the ``low energy'' ($T\leq 10$ GeV) data we have applied the
dielectric supression to the results of Seltzer and Berger.
We have chosen the parameterisations:
\begin{equation}
\label{ebrem.a}
\sigma (Z,T,k_c ) =\frac{Z(Z+\xi_{\sigma} )(T+m)^2}
     {T(T+2m)}[ \ln (T/k_c )]^\alpha F_{\sigma} (Z,X,Y)
\mbox{\quad(barn)}\\
\end{equation}
and
\begin{equation}
\label{ebrem.b}
E_{Loss}^{brem} (Z,T,k_c ) =\frac{Z(Z+ \xi_l)(T+m)^2 }
      {(T+2m)}\left[\frac{k_c C_ M }{T}\right]^\beta
      F_{l} (Z,X,Y) \mbox{\quad(GeV barn)}
\end{equation}
where  $m$ is the mass of the electron,
\[
\begin{array}{lll}
X = \ln(E/m), & Y = \ln (v_{\sigma} E/k_c) &
\mbox{\quad for the total cross-section }\sigma \\
X = \ln(T/m), & Y = \ln(k_c/v_lE)       &
\mbox{\quad for the energy loss }  E_{Loss}^{brem}  \\
\end{array}
\]
with $E=T+m$.
The constants $\xi_{\sigma},\ \xi_l,\ \alpha,\ \beta,\ v_{\sigma},\ v_l$
are parameters to be fitted.
\begin{eqnarray*}
C_M &=&\frac{1}{1+\frac{n r_0 \lambda r^2  (T+m)^2}{\pi k^2_c}}
\end{eqnarray*}
is the Midgal correction factor, with
\begin{description}
\item $r_o$  classical electron radius;
\item $\lambda$  reduced electron Compton wavelength;
\item $n$  electron density in the medium.
\end{description}
The factors $(T+m)^2/T(T+2m)$ and $(T+m)^2/(T+2m)$ come from
the scaled cross-section computed by Seltzer and Berger:
\[
f(k/T) = \frac{\beta^2}{Z^2 }k\frac{d \sigma}{dk}
       = \frac{T(T+2m)} {(T+m)^2 Z^2}  k \frac{d \sigma}{dk}
\]

The functions $F_i(Z,X,Y)$ ($ i=\sigma,l $) have the form
\begin{equation}
 F_i(Z,X,Y) = F_{i0}(X,Y)+ZF_{i1}(X,Y)
\end{equation}
where $F_{ij}(X,Y)$ are polynomials of the variables $X,Y$
\begin{eqnarray}
F_{i0}(X,Y) & = &  (C_1+C_2X+\ldots+C_6X^5)+(C_7+C_8X+\ldots+C_{12}X^5)Y
                                                             \nonumber \\
          &   & +(C_{13}+C_{14}X+\ldots + C_{18}X^5)Y^2+\ldots \nonumber \\
          &   & +(C_{31}+C_{32}X+\ldots+C_{36}X^5)Y^5        \nonumber \\
          &   & \mbox{\hspace{11cm}}  Y\leq 0      \nonumber \\
          & = &  (C_1+C_2X+\ldots+C_6X^5)+(C_7+C_8X+\ldots+C_{12}X^5)
                                                             \nonumber \\
          &   & +(C_{37}+C_{38}X+\ldots+C_{42}X^5)Y^2+\ldots+ \nonumber \\
          &   &  (C_{55}+C_{56}X+\ldots+C_{60}X^5)Y^5        \nonumber \\
          &   & \mbox{\hspace{11cm}}  Y > 0        \nonumber \\
F_{i1}(X,Y) & = &  (C_{61}+C_{62}X+\ldots+C_{65}X^4)
                +(C_{66}+C_{67}X+\ldots+C_{70}X^4)Y          \nonumber \\
          &   & +(C_{71}+C_{72}X+\ldots+C_{75}X^4)Y^2+\ldots \nonumber \\
          &   & +(C_{81}+C_{82}X+\ldots+C_{85}X^4) Y^4       \nonumber \\
          &   &   \mbox{\hspace{11cm}}  Y\leq 0    \nonumber \\
          & = &  (C_{61}+C_{62}X+\ldots+C_{65}X^4)
                +(C_{66}+C_{67}X+\ldots+C_{70}X^4)Y          \nonumber \\
          &   & +(C_{86}+C_{87}X+\ldots+C_{90}X^4)Y^2+\ldots \nonumber \\
          &   & +(C_{96}+C_{97}X+\ldots+C_{100}X^4)Y^4       \nonumber  \\
          &   &  \mbox{\hspace{11cm}}  Y>0         \nonumber \\
           \nonumber
\end{eqnarray} 
$F_{ij}(X,Y)$ denotes in fact a function constructed from two
polynomials
\[
F_{ij}(X,Y) = \left \{
\begin{array}{ll}
F_{ij}^{neg}(X,Y) & \mbox{for\hspace{0.5cm}} Y\leq 0 \\
F_{ij}^{pos}(X,Y) & \mbox{for\hspace{0.5cm}} Y>0
\end{array} \right .
\]
where the polynomials $F_{ij}$ fulfil the conditions
\[
F_{ij}^{neg}(X,Y)_{Y=0} = F_{ij}^{pos}(X,Y)_{Y=0}   \hspace{2cm}
\left ( \frac{\partial F_{ij}^{neg}}{\partial Y} \right )_{Y=0} =
\left ( \frac{\partial P_{ij}^{pos}}{\partial Y} \right )_{Y=0}
\]
We have computed 4000 ``data points'' in the range
\[
Z = 6  ;  13  ;  29  ;  47  ;  74  ;  92  \hspace{1cm}
10 \: \mbox{keV} \leq T \leq 10 \: \mbox{TeV}  \hspace{1cm}
10 \: \mbox{keV} \leq k_c \leq T
\]
and we have performed a least-squares fit to determine the parameters.

The values of the parameters ($\xi_{\sigma}$, $\alpha$,
$v_{\sigma}$, $C_i $ for $\sigma$ and
$\xi_l$, $\beta$, $V_l$, $C$ for $E_{Loss}^{brem}$)
can be found in the code of the class itself.

The errors of the parameterisations (\ref{ebrem.a}) and (\ref{ebrem.b})
can be estimated as

\begin{eqnarray*}
\frac{\Delta\sigma} {\sigma} & = & \left \{
\begin{array}{llr}
        12-15\%    & \mbox{for    } & T \leq 1 MeV \\
        \leq 5-6\% & \mbox{for    } & 1 MeV < T \leq 10 TeV
\end{array}
\right . \\[1cm]
\frac{\Delta E_{Loss}^{brem}}
     {E_{Loss}^{brem}} & = & \left \{
\begin{array}{llr}
        10-15\%    & \mbox{for    } & T \leq1 MeV  \\
        5-6\%      & \mbox{for    } & 1 MeV < T \leq 100 GeV\\
        10\%       & \mbox{for    } & 100 GeV < T \leq10 TeV
\end{array}
\right .
\end{eqnarray*}


The energy loss due to soft photon bremsstrahlung is tabulated at
initialisation time as a function of the medium and of the energy.  

The mean free path for discrete bremsstrahlung is tabuled at initialisation
time as a function
of the medium and of the energy, too.  

\subsection{Corrections for $e^+ e^-$ differences}

The radiative energy loss for electrons or positrons is:

\begin{eqnarray*}
-\frac{1}{\rho} \left ( \frac{dE}{dx} \right )_{rad}^{\pm} & = &
\frac{N_{Av} \alpha r_e^2}{A} (T+m) Z^2 \Phi_{rad}^{\pm}(Z,T) \\
\Phi^{\pm}_{rad}(Z,T) & = & \frac{1}{\alpha r_{e}^2 Z^2 (T+m)}
\int^{T}_{0}{k\frac{d\sigma^{\pm}}{dk}dk}
\end{eqnarray*}

Reference \cite{ebrem.kim} says that: \\
{\it ``The differences between the radiative loss of positrons
and electrons are considerable and cannot be disregarded.

[...] The ratio of the radiative energy loss for positrons
to that for electrons obeys a simple scaling law, [...] is a
function only of the quantity $T/Z^2$''}

In other words:

\begin{eqnarray*}
\eta & = & \frac{\Phi_{rad}^{+}(Z,T)}{\Phi_{rad}^{-}(Z,T)} =
\eta \left (\frac{T}{Z^2}\right )
\end{eqnarray*}

The authors have calculated this function in the range $10^{-7}
\leq \frac{T}{Z^2} \leq 0.5$ (here the kinetic energy T is
expressed in MeV). Their {\it data} can be fairly accurately
reproduced using a parametrisation:

\begin{eqnarray*}
\eta & = & \left \{
\begin{array}{llr}
0 & \mbox{if   } & x \leq -8 \\
\frac{1}{2} + \frac{1}{\pi} \arctan \left( a_1 x + a_3 x^3
+ a_5 x^5 \right ) & \mbox{if  } & -8 < x < 9 \\
1 & \mbox{if   } & x \geq 9
\end{array}
\right .
\end{eqnarray*}

where:

\begin{eqnarray*}
x & = & \log \left ( C \frac{T}{Z^2} \right ) \mbox{(T in GeV)} \\
C & = & 7.5221 \times 10^{6} \\
a_1 & = & 0.415 \\
a_3 & = & 0.0021 \\
a_5 & = & 0.00054
\end{eqnarray*}

This $e^+ e^-$ energy loss difference is not a pure low-energy
phenomenon (at least for high Z), as it can be seen from
Tables~\ref{ebrem.c}.

\begin{table}[hbt]
\begin{centering}
\begin{tabular}{rr|r|r} \hline
\multicolumn{1}{c}{$\frac{T}{Z^2} (GeV)$}
& \multicolumn{1}{c|}{T}
& \multicolumn{1}{c|}{$\eta$}
& \multicolumn{1}{c}{$\left ( \frac{rad. \ loss}{total \ loss}
\right )_{e^-}$} \\[3mm] \hline
$10^{-9}$ & $\sim 7 keV$ & $\sim 0.1$ & $\sim 0\%$ \\
$10^{-8}$ & $67 keV $ & $\sim 0.2$ & $\sim 1\%$ \\
$2 \times 10^{-7}$ & $1.35 MeV$ & $\sim 0.5$ & $\sim 15\%$ \\
$2 \times 10^{-6}$ & $13.5 MeV$ & $\sim 0.8$ & $\sim 60\%$ \\
$2 \times 10^{-5}$ & $135. MeV$ & $\sim 0.95$ & $> 90\%$ \\ \hline
\end{tabular}
\caption{ratio of the $e^+ e^-$ radiative energy loss in lead
(Z=82).}
\label{ebrem.c}
\end{centering}
\end{table}

The scaling holds for the ratio of the total radiative energy
losses, but it is significantly broken for the photon spectrum
in the screened case.
In case of a point Coulomb charge the
scaling would hold also for the spectrum.
The scaling can be expressed by:

\begin{eqnarray*}
\frac{\Phi^+}{\Phi^-} = \eta \left ( \frac{T}{Z^2} \right )
& \hspace{3cm} &
\frac{\frac{d\sigma^+}{dk}}{\frac{d\sigma^-}{dk}} =
\mbox{does not scale}
\end{eqnarray*}

If we consider the photon spectrum from bremsstrahlung reported
in \cite{ebrem.kim} we see that:

\begin{eqnarray*}
\frac{d\sigma^{\pm}}{dk} = S^{\pm} \left( \frac{k}{T} \right )
\hspace{2cm}
\frac{S^{+}(k)}{S^{-}(k)} \leq 1 & \hspace{1cm} & S^{+}(1) = 0
\hspace{2cm} S^{-}(1)  >  0
\end{eqnarray*}

We further assume that:

\begin{eqnarray}
\frac{d\sigma^+}{dk} = f(\epsilon) \frac{d\sigma^-}{dk}
& \hspace{2cm} &
\epsilon = \frac{k}{T}
\label{ebrem.d}
\end{eqnarray}

In order to satisfy approximately the scaling law for the ratio
of the total radiative energy loss, we require for $f(\epsilon)$:

\begin{eqnarray}
\int^{1}_{0}{f(\epsilon)d\epsilon} & = & \eta
\label{ebrem.e}
\end{eqnarray}

From the photon spectra we require:

\begin{eqnarray}
\left .
\begin{array}{l}
f(0) = 1 \\
f(1) = 0
\end{array}
\right \} \hspace{2cm} \mbox{for all $Z,T$}
\label{ebrem.f}
\end{eqnarray}

We have chosen a simple function $f$:

\begin{eqnarray}
f(\epsilon) & = & C (1-\epsilon)^{\alpha} \hspace{3cm} C,\alpha > 0
\label{ebrem.g}
\end{eqnarray}

from the conditions (\ref{ebrem.e}), (\ref{ebrem.f}) we get:

\begin{eqnarray*}
C & = & 1 \\
\alpha & = & \frac{1}{\eta} - 1 \hspace{2cm}
\mbox{($\alpha > 0$ because $\eta < 1$)} \\
f(\epsilon) & = & (1-\epsilon)^{\frac{1}{\eta}-1}
\end{eqnarray*}

We have defined weight factors $F_{l}$ and $F_{\sigma}$ for the
positron continuous energy loss and discrete bremsstrahlung cross
section:

\begin{eqnarray}
F_{l} = \frac{1}{\epsilon_{0}} \int^{\epsilon_{0}}_{0}
{f(\epsilon)d\epsilon} & \hspace{3cm} &
F_{\sigma} = \frac{1}{1-\epsilon_{0}} \int^{1}_{\epsilon_{0}}
{f(\epsilon)d\epsilon}
\label{ebrem.h}
\end{eqnarray}

where $\epsilon_{0} = \frac{k_c}{T}$ and $k_c$ is the photon
cut . In this scheme the positron energy loss and
discrete bremsstrahlung can be calculated as:

\begin{eqnarray*}
\left ( - \frac{dE}{dx} \right )^{+} = F_{l}
\left ( - \frac{dE}{dx} \right )^{-} & \hspace{2cm} &
\sigma^{+}_{brems} = F_{\sigma} \sigma^{-}_{brems}
\end{eqnarray*}

As in this approximation the photon spectra are identical, the
same sampling is used for generating $e^- e^+$ bremsstrahlung.
The following relations hold:

\begin{eqnarray*}
F_{\sigma} & = & \eta (1-\epsilon_{0})^{\frac{1}{\eta}-1}
< \eta \\
\epsilon_{0} F_{l} + (1-\epsilon_{0}) F_{\sigma} & = & \eta
\hspace{6cm} \mbox{from the def (\ref{ebrem.h})} \\
\Rightarrow F_{l} & = & \eta \frac{1-(1-\epsilon_{0})^{\frac{1}
{\eta}})}{\epsilon_{0}} > \eta \frac{1-(1-\epsilon_{0})}
{\epsilon_{0}} = \eta  \hspace{1cm}
\Rightarrow   \left \{
\begin{array}{l}
F_{l} > \eta \\
F_{\sigma} < \eta
\end{array} \right .
\end{eqnarray*}

which is consistent with the spectra.

The effect of this $e^- e^+$ bremsstrahlung difference can be also
seen in e.m. shower development, when the primary energy is not too
high. 

\subsection{LPM effect}

The LPM effect (see e.g in \cite{ebrem.galitsky},\cite{ebrem.anthony} )
 is the supression of the photon production due to the
multiple scattering of the electrons.If an electron multiple scatters
while traversing the co called formation zone, the bremsstrahlung
amplitude from before and after the scattering can interfere, reducing the
amplitude for bremsstrahlung photon emission (similar supression occurs 
for pair production,too). The supression becomes significant for photon
energies below a certain value, given by
\begin{equation}
\label{ebrem.k}
 \frac{k}{E} < \frac{E}{E_{LPM}}
\end{equation}
 , where
\[
\begin{array}{ll}
k    & \mbox{photon energy} \\
E    & \mbox{electron energy} \\
E_{LPM} & \mbox{characteristic energy for LPM effect (depend on the medium).} 
\end{array}
\]

The value of the LPM characteristic energy can be written as

\begin{equation}
\label{ebrem.l}
  E_{LPM} = \frac{\alpha m^2 X_0}{4 h c}
\end{equation}

 , where
\[
\begin{array}{ll}
\alpha  & \mbox{fine structure constant} \\
m       & \mbox{electron mass} \\
X_0     & \mbox{radiation length in the material} \\
h       & \mbox{Planck constant} \\
c       & \mbox{velocity of light in vacuum.}
\end{array} 
\]

 The LPM supression of the photon spectrum is given by the formula

\begin{equation}
\label{ebrem.m}
  S_{LPM} = \sqrt{\frac{E_{LPM} \cdot k}{E^2}}
\end{equation}

, while the dielectric supression (included already in the parametrisations) 
can be written as

\begin{equation}
\label{ebrem.n}
  S_p = \frac{k^2}{k^2 + C_p \cdot E^2}
\end{equation}

 , where the quantity $C_p$ is given by

\begin{equation}
\label{ebrem.o}
   C_p = \frac{r^2_0 \lambda^2_e n}{\pi}
\end{equation}

In eq. \ref{ebrem.o} the parameters are
\[
\begin{array}{ll}
r_0     & \mbox{classical electron radius} \\
\lambda_e & \mbox{electron Compton wawelength} \\
n       & \mbox{electrondensity in the material.}
\end{array}
\]

 Both of the supression effects reduce the effective formation length
of the photon , so the supressions {\em do not simply multiply.} For the
total supression S the following equation holds (see \cite{ebrem.galitsky})
\begin{equation}
\label{ebrem.p}
  \frac{1}{S} = 1 + \frac{1}{S_p} + \frac{S}{S^2_{LPM}}
\end{equation}
 which can be solved easily for $S$
\begin{equation}
\label{ebrem.q}
  S = \frac{\sqrt{S^4_{LPM}\cdot (1 + \frac{1}{S_p})^2 + 4 \cdot S^2_{LPM}}
      -S^2_{LPM} \cdot (1 + \frac{1}{S_p})}{2}
\end{equation}
 
 The implementation of the LPM effect has been done in the following way:
The cross section and energy loss should be modified by applying the
 (energy and material dependent ) factor $\frac{S}{S_p}$ . This is done
at initialisation time for the energy loss by computing the correction
factor
\begin{equation}
\label{ebrem.r}
   f_c = \frac{\int_0^{k_cut} n_\gamma (k) \cdot \frac{S}{S_p} dk}
              {\int_0^{k_cut} n_\gamma (k) dk}
\end{equation}
 , wherea $n_\gamma (k)$ is the photon specrum .
 The total cross section given by the parametrization \ref{ebrem.a} has
not been modified , the algorithm of the photon generation takes care
of the LPM effect.

\section{Simulation of the discrete bremsstrahlung}

The photon energy is sampled according to the Seltzer and Berger
bremsstrahlung spectrum~\cite{ebrem.seltzer}.
 Seltzer and Berger have calculated the
spectra for materials with atomic numbers Z = 6,13,29,47,74,92
in the electron (kinetic) energy range  1 keV - 10 GeV. Their tabulated
results have been used as input in a parametrising-fitting procedure.
The functional form of the parameterisation for the quantity:

\[
S(x) = C k \frac{d \sigma}{d k}
\]

can be written as
\begin{equation}
\label{eq:phys341-1}
S(x) = \left \{
\begin{array}{ll}
(1-a_{h} \epsilon )F_{1}(\delta) + b_{h} \epsilon^{2} F_{2} (\delta)
& T \geq 1 MeV \\
1 + a_{l} x + b_{l} x^{2} & T < 1 MeV
\end{array} \right .
\end{equation}
where:
\[
\begin{array}{lcllcl} 
C & & \mbox{normalisation constant} &
k & & \mbox{photon energy} \\ [1mm]
T, E & & \mbox{kinetic and total energy of the primary electron} &
x & = & \frac{k}{T} \\ [2mm]
\epsilon & = & \frac{k}{E} = x \frac{T}{E}
\end{array}
\]

The $F_{i}(\delta)$ screening functions depend on the screening variable:
\[
\begin{array}{lcll}
\delta & = & \frac{136 m_{e}}{Z^{1/3} E} \frac{\epsilon}{1-\epsilon} \\
F_{1}(\delta) & = & F_{0} (42.392 - 7.796 \delta +1.961 \delta^{2} - F)
& \delta \leq 1 \\
F_{2}(\delta) & = & F_{0} (41.734 - 6.484 \delta +1.250 \delta^{2} - F)
& \delta \leq 1 \\
F_{1}(\delta) & = & F_{2}(\delta) =
F_{0} (42.24 - 8.368 \ln(\delta + 0.952) -F) & \delta > 1 \\
F_{0} & = & \frac{1}{42.392-F} \\
F & = & 4 \ln Z - 0.55 (\ln Z)^{2}
\end{array}
\]

$a_{h,l}$ and $b_{h,l}$ are parameters to be fitted.

The `high energy' (T $>$  1 MeV) formula comes from the
Coulomb-corrected, sceened Bethe-Heitler formula (see e.g.
\cite{ebrem.williams,ebrem.butcher,ebrem.egs4}). However, there are two things in eq.
(\ref{eq:phys341-1}) which make a difference:

\begin{enumerate}
\item $a_{h}, b_{h}$ depend on T and on the atomic number Z ( in the case
of the Bethe-Heitler spectrum $a_{h} = 1$, $b_{h} =0.75$);
\item the function $F$ is not the same than that in the Bethe-Heitler
cross-section, this function gives a better behaviour in the
high frequency limit, i.e. when $k \rightarrow T$  ($x \rightarrow 1$).
\end{enumerate}

The T and Z dependence of the parameters are described by the equations:

\begin{eqnarray*}
a_{h} & = & 1 + \frac{a_{h1}}{u}+\frac{a_{h2}}{u^{2}}+\frac{a_{h3}}{u^{3}} \\
b_{h} & = & 0.75+\frac{b_{h1}}{u}+\frac{b_{h2}}{u^{2}}+\frac{b_{h3}}{u^{3}} \\
a_{l} & = & a_{l0} + a_{l1} u + a_{l2} u^{2} \\
b_{l} & = & b_{l0} + b_{l1} u + b_{l2} u^{2} \\
\mbox{with} \\
u & = & \ln \left ( \frac{T}{m_{e}} \right )
\end{eqnarray*}

the $a_{hi}, b_{hi}, a_{li}, b_{li}$ parameters are polynomials of second order
in the variable:

\[
v = [Z (Z+1)]^{1/3}
\]

It can be seen relatively easily that for the limiting case $T \rightarrow
\infty$, $a_{h} \rightarrow 1, b_{h} \rightarrow 0.75$,
so eq. (\ref{eq:phys341-1}) gives the Bethe-Heitler cross section.

There are altogether 36 linear parameter in the formulae , their
values are given in the code. The parameterisation reproduces
the Seltzer-Berger tables within a few \% (2-3 \% on average,
the maximum error being less than 10-12 \%), the tables, on the other hand,
agree well with the experimental data and theoretical (low- and high-energy)
results (less than 10 \% below 50 MeV, less than 5 \% above 50 MeV).

Apart from the normalisation the cross section differential in photon
energy can be written as:
\[
\frac{d \sigma}{d k} = \frac{1}{\ln \frac{1}{x_{c}}} \frac{1}{x}
g(x) = \frac{1}{\ln \frac{1}{x_{c}}} \frac{1}{x} \frac{S(x)}{S_{max}}
\]

where $x_{c} = k_{c}/T$, $k_{c}$ is the photon cut-off energy below
which the bremsstrahlung is treated as a continuous energy loss
. Using this decomposition of
the cross section and two random numbers $r_{1}$, $r_{2}$ uniformly
distributed in $]0,1[$, the sampling of $x$ is done as follows:
\begin{enumerate}
\item sample $x$ from
\[
\frac{1}{\ln \frac{1}{x_{c}}} \frac{1}{x} \mbox{\hspace{1cm}setting\hspace{1cm}}
x = e^{r_{1} \ln x_{c}}
\]

\item calculate the rejection function $g(x)$ and:
\begin{itemize}
\item if $r_{2} > g(x)$ reject $x$ and go back to 1;
\item if $r_{2} \leq g(x)$ accept $x$.
\end{itemize}
\end{enumerate}

To apply the dielectric supression  \cite{ebrem.migdal}
  all it has to be done is to
multiply the rejection function by the supression factor , which is 
equivalent to the expression \ref{ebrem.n}:

\[
C_M (\epsilon)  =\frac{1 + C_0 / \epsilon_c^2}
               {1 + C_0 / \epsilon^2}
\]
 where
\[
C_0 =\frac{nr_0 \lambda^2 }{\pi}, \hspace{1cm} \epsilon_c = \frac{k_{c}}{E}
\]
\begin{itemize}
\item[$n$]           electron density in the medium
\item[$r_0$]         classical electron radius
\item[$\lambda$]    reduced Compton wavelength of the electron.
\end{itemize}
This correction decreases the cross-section for low photon energy.

 After having $\epsilon$ sampled , the supression factor $f_{LPM}=\frac{S}{S_p}$ is
used as a rejection function in order to take into account the LPM effect , too.
Here the supression factor is compared to a random number $r$ uniformly
distributed in the interval $(0,1)$. If  $f_{LPM} \geq r$ the simulation
is continuing otherwise the bremsstrahlung process is finished 
{\em without photon production } .It can be seen easily that this procedure
takes into account the LPM supression correctly.


After the successful sampling of $\epsilon$,  the
polar angles of the radiated photon are generated with respect to the parent
electron's momentum. It is difficult to find in the literature
simple formulas for this angle. For example the double differential
cross section reported by Tsai~\cite{ebrem.tsai1,ebrem.tsai2} is the
following:
\begin{eqnarray*}
\frac{d \sigma}{dkd \Omega}
& = & \frac{2 \alpha^{2}e^{2}}{\pi k m^{4}}
  \left\{ \left[ \frac{2\epsilon-2}{(1+u^2)^2}+
\frac{12u^2(1-\epsilon)}{(1+u^2)^4}\right]
      Z(Z+1)  \right. \\
&   & \mbox{} + \left. \left[ \frac{2-2\epsilon-\epsilon^{2}}{(1+u^2)^2}-
      \frac{4u^2(1-\epsilon)}{(1+u^2)^4}
      \right]
      \left[ X-2Z^{2}f_{c}((\alpha Z)^{2})\right]
      \right\} \\
u & = & \frac{E \theta}{m} \\
X & = & \int_{t_{min}}^{m^{2}(1+u^{2})^{2}}
{\left [ G_{Z}^{el}(t) + G_{Z}^{in}(t) \right ] \frac{t-t_{min}}
{t^{2}} dt} \\
G_{Z}^{el, in}(t) & & \mbox{atomic form factors} \\
t_{min} & = & \left [ \frac{k m^{2} (1+u^{2})}{2 E (E-k)} \right ] ^{2}
 = \left [ \frac{\epsilon m^{2} (1+u^{2})}{2 E (1-\epsilon)} \right ] ^{2}
\end{eqnarray*}

This distribution is complicated to sample, and it is anyway only an
approximation to within few percent, if nothing else, due to
the presence of the atomic form-factors.
The angular dependence is contained in the
variable $u = E \theta m^{-1}$. For a given value
of $u$ the dependence of the shape of the function on $Z$, $E$,
$\epsilon = k/E$ is very weak.
Thus, the distribution can be approximated by a function
\begin{equation}
f(u) = C \left( u e^{-au} + d u e^{-3au} \right)
\end{equation}
where
\[
C = \frac{9a^{2}}{9 + d} \hspace{1cm} a = 0.625 \hspace{1cm}
d = 0.13 \: \left ( 0.8+\frac{1.3}{Z} \right ) \left (100+\frac{1}{E} \right )
(1+\epsilon)
\]
where $E$ is in GeV. While this approximation is good at high energies,
it becomes less accurate around few MeV. However in that region the
ionisation losses dominate over the radiative losses.

The sampling of the function $f(u)$ can be done in the following way
($r_{i},\: i=1,2,3$ are uniformly distributed random numbers
in [0,1]):
\begin{enumerate}
\item Choose between $u e^{-au}$ and $d u e^{-3au}$:
\[
b = \left \{ \begin{array}{ll}
a & \mbox{if\hspace{0.5cm}}r_{1} < 9/(9+d) \\
3a & \mbox{if\hspace{0.5cm}}r_{1} \geq 9/(9+d)
\end{array} \right .
\]
\item Sample $u e^{-bu}$:
\[
u=-\frac{\log ( r_{2} r_{3}) }{b}
\]
\item check that:
\[
u \leq u_{max} = \frac{E \pi}{m}
\]
otherwise go back to 1.
\end{enumerate}

The probability of failing in the last test is reported in
table~\ref{tb:phys341-1}.

\begin{table}
\begin{centering}
\begin{tabular}{|l|l|}
\multicolumn{2}{c}{$\displaystyle
P = \int^{\infty}_{u_{max}}{f(u) \: du} \hfill $} \\ [0.5cm]
\hline
E (MeV) & P(\%) \\ \hline
0.511 & 3.4 \\
0.6 &  2.2 \\
0.8 & 1.2 \\
1.0 & 0.7 \\
2.0 & $<$ 0.1 \\ \hline
\end{tabular}
\caption{Angular sampling efficiency}
\label{tb:phys341-1}
\end{centering}
\end{table}


The function $f(u)$ can be used also to describe
the angular distribution of the photon in $\mu$ bremsstrahlung and to
describe the angular distribution in photon pair production.

The azimuthal angle, $\Phi$, is generated isotropically.
This information is used to calculate the momentum vector of the radiated
photon, to transform it to the global  coordinate system. 
Also, the momentum of the parent electron is updated.

\section{Status of this document}

   9.10.98  created by L. Urb\'an.
  
\begin{thebibliography}{99}

\bibitem[GEANT3]{ebrem.geant3}
  GEANT3 manual ,CERN Program Library Long Writeup W5013 (October 1994).
\bibitem[Gal64]{ebrem.galitsky}
  V.M.Galitsky and I.I.Gurevich. Nuovo Cimento 32 (1964) 1820.
\bibitem[Ant97]{ebrem.anthony}
  P.L. Anthony et al. SLAC-PUB-7413/LBNL-40054 (February 1997)
\bibitem[Sel85]{ebrem.seltzer}
  S.M.Seltzer and M.J.Berger. Nucl.Inst.Meth. 80 (1985) 12.
\bibitem[Mess70]{ebrem.messel}
  H.Messel and D.F.Crawford. Pergamon Press,Oxford,1970.
\bibitem[EGS4]{ebrem.egs4} W.R. Nelson et al.:The EGS4 Code System.
   {\em SLAC-Report-265 , December 1985 }
\bibitem[Mig56]{ebrem.migdal}
   A.B. Migdal. Phys.Rev. 103. (1956) 1811.
\bibitem[Kim86]{ebrem.kim} 
   L. Kim et al. Phys. Rev. A33 (1986) 3002.
\bibitem[Tsai74]{ebrem.tsai1}
   Y-S. Tsai . Rev. Mod. Phys. 46. (1974) 815.
\bibitem[Tsai77]{ebrem.tsai2}
   Y-S. Tsai . Rev. Mod. Phys. 49. (1977) 421.
\bibitem[Will60]{ebrem.williams}
   R. W. Williams . Fundamental Formulas of Physics, vol.2. Dover Pubs.Inc,1960.
\bibitem[Butc60]{ebrem.butcher}
   J. C. Butcher and H. Messel. Nucl.Phys. 20. (1960) 15.

\end{thebibliography}

