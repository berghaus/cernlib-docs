 \chapter[Multiple scattering]{Multiple scattering}

 The G4MultipleScattering class simulates the multiple scattering of
charged particles  in material.It uses a new multiple scattering (MSC)
model which  does not use the
Moliere formalism(\cite{msc.moliere}).This MSC model simulates the scattering of the 
particle after a given step , computes the mean path length 
correction and the mean lateral displacement as well.

 Let us define a few notation first.

 The true path length ('t' path length) is the total length travelled
by the particle. All the physical processes restrict this 't' step.

 The geometrical ( or 'z') path length is the straight distance between
the starting and endpoint of the step , if there is no magnetic field.
The geometry gives a constraint for this 'z' step. It should be noted,
that the geometrical step length is meaningful in the case of magnetic
field, too, but in this case it is a distance along a curved 
trajectory.

 The mean properties of the multiple scattering process are determined
by the transport mean free path , \(\lambda\) , which is a function of the 
energy in a given material.Some of the mean properties - the mean lateral
 displacement and the second moment of cos(theta) - depend on the second
 transport mean free path, too. (The transport mean free path is called
 first transport mean free path as well.)

  The 't'\(\Rightarrow\)'z' (true path length -- geometrical path length) transformation is given by the simple equation

   \begin{equation}
         z = \lambda*(1.-exp(-t/\lambda))                \label{msc.a}
   \end{equation}

 which is an exact result for the mean values of z , if
the differential cross section has an axial symmetry and the energy loss
can be neglected .
  This formula and some other expressions for the first moments of the spatial
distribution after a given 'true' path length t have been taken from the excellent
paper of Fernandez-Varea et al. \cite{msc.fernandez}, but the expressions have been
calculated originally by Goudsmit and Saunderson \cite{msc.goudsmit} and Lewis
\cite{msc.lewis}.
  Inverting eq. \ref{msc.a} the 'z'\(\Rightarrow\)'t' transformation can be written as

   \begin{equation}
        t = -\lambda*ln(1.-z/\lambda)                     \label{msc.b}
   \end{equation}

 where \(z < \lambda\) should be required (this condition is fulfilled
 if z has been computed from eq. \ref{msc.a}).

  The mean value of \(cos(\theta)\) - \(\theta\) is the scattering angle after a
true step length t - is

   \begin{equation}
          <cos(\theta)> = exp(-t/\lambda)               \label{msc.c}  
   \end{equation}

  The transport mean free path values have been calculated by Liljequist et al.
\cite{msc.liljequist2},\cite{msc.liljequist1} for electrons and positrons in the kinetic
energy range \(0.1 keV -- 20 MeV\) in 15 materials . The MSC model uses these
values with an appropriate interpolation or extrapolation in the atomic number
\(Z\) and in the velocity of the particle \(\beta\) , when it is necessary.
  
 The quantity \(cos(\theta)\) is sampled in the MSC model according to a model function
 \(f(cos(\theta))\). The shape of this function has been choosen in such a way,
that\(f(cos(\theta))\) reproduces the results of the direct simulation ot the particle
transport rather well and eq. \ref{msc.c} is satisfied.
 The functional form of this model function is

  \begin{equation}
      f(x) = p \frac{(a + 1)^2 (a - 1)^2}{2 a} \frac{1}{(a-x)^3}
            + (1-p) \frac{1}{2}                            \label{msc.d}
  \end{equation}

 where  \( x= cos(\theta)\) , \( 0 \leq p \leq 1\) and \( a > 1\) . The model
 parameters \(p\) and \(a\) depend on the path length t , the energy of the
 particle and the material.They are not independent parameters , they should
 satisfy the constraint

  \begin{equation}
        \frac{p}{a} = exp(-\frac{t}{\lambda})              \label{msc.e}
  \end{equation} 

 which follows from eq. \ref{msc.c} .

  The mean lateral displacement is given by a more complicated formula
(see the paper \cite{msc.fernandez} ), but this quantity also can be calculated
 relatively easily and accurately.
 
  It is worth to note that in this MSC model there is no step limitation
 originated from the multiple scattering process. Another important feature
 of this model
 that the total 'true' path length of the particle does not depend the
 length of the steps . Most of the algorithms used in simulations do not have
 these properties.
  
 In the case of heavy charged particles ( \(\mu,\pi,proton,etc.\) ) the
 mean transport free path is calculated from the \(e+/e-\)  \(\lambda\) values
 with a 'scaling'.

 In its present form the model computes and uses {\em mean}  path length
 corrections and lateral displacements, the only {\em random} quantity is
 the scattering angle \(\theta\) which is sampled according to the model
 function \( f \).   

  The G4MultipleScattering process has  'AlongStep' and  'PostStep'
parts.

  The AlongStepGetPhysicalInteractionLength function performs the\linebreak
 \mbox{'t' step \(\Rightarrow\) 'z' step} transformation . It should be called after the
other physics GetPhysicalInteractionLength functions but before
the GetPhysicalInteractionLength of the transportation process.The
reason for this restriction is the following: The physics processes
'feel' the true path length travelled by the particle , the geometry
(transport) uses the 'z' step length.If we want to compare the minimum
step size coming from the physics with the constraint of the geometry,
we have make the transformation.

  The AlongStepDoIt function of the process performs the inverse,
 'z'\(\Rightarrow\)'t' transformation.This function should be called after the 
AlongStepDoIt of the transportation process , i.e. after the particle
relocation determined by the geometrical step length, but before applying
any other (physics) AlongStepDoIt.

  The PostStepGetPhysicalInteractionLength part of the multiple
scattering process is very simple , it sets the force flag to 'Forced'
in order to ensure the call of the PostStepDoIt in every step and
returns a big value as interaction length (that means that the multiple
scattering process does not restrict the step size).

\section{Status of this document}
  9.10.98  created by L. Urb\'an.

\begin{thebibliography}{99}
\bibitem[Mol48]{msc.moliere}
   {\em Z. Naturforsch. 3a (1948) 78. }
\bibitem[Fer93]{msc.fernandez}J. M. Fernandez-Varea et al.
   {\em NIM B73 (1993) 447.}
\bibitem[Goud40]{msc.goudsmit}S. Goudsmit and J. L. Saunderson.
   {\em Phys. Rev. 57 (1940) 24. }
\bibitem[Lew50]{msc.lewis} H. W. Lewis. 
   {\em Phys. Rev. 78 (1950) 526. }
\bibitem[Lil87]{msc.liljequist1} D. Liljequist and M. Ismail.
   {\em J.Appl.Phys. 62 (1987) 342. }
\bibitem[Lil90]{msc.liljequist2} D. Liljequist et al.
   {\em J.Appl.Phys. 68 (1990) 3061. }
\end{thebibliography}
