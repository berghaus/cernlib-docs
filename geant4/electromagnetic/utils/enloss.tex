\chapter{Compute the energy loss}
 
 The energy loss processes are very similar for \(e+/e-\) , \(\mu+/\mu-\) and
charged hadrons , so it is quite natural to have a common description for them
and we have chosen this way.

\section{Energy loss of electrons/positrons}

 The G4eEnergyLoss class computes the continuous energy loss \linebreak
  of electrons/positrons.
 The continuous energy loss is calculated as
a sum of the contribution of the different processes.
 At present there are two processes
contributing to the continuous energy loss , they are:

            the ionisation process (class G4eIonisation ) and
            the bremsstrahlung process (class G4eBremsstrahlung).

 The class G4eIonisation calculates the contribution due to ionisation and
it simulates the 'discrete' part of the ionisation , the Moller and Bhabha
scattering or delta ray production by e+/e-.

 The class G4eBremsstrahlung computes the energy loss contribution due to
soft bremsstrahlung and it simulates the 'discrete' or hard bremsstrahlung.
 
 The formulae used to compute the continuous energy loss are the same than
those in GEANT3, but the two programs are not the same. 
 The main difference between GEANT3 and GEANT4 is that while in GEANT3 the
continuous energy loss, delta ray production and bremsstrahlung are three
independent processes , in GEANT4 these processes are closely connected,
they form a unified scheme.

 The G4eEnergyLoss class constructs dE/dx and range tables for every material.
 First the dE/dx tables are costructed and filled , the energy loss class
  simply summes up of the contributions computed by the ionisation and 
 bremsstrahlung processes. After this step G4eEnergyLoss creates range tables
 of \(e+/e-\) for every material and computes the inverse of these range 
 tables as well. All the tables are constructed 
 at the beginning of the GEANT4 run , at initialisation time. Later, during
the simulation the energy loss process performs two tasks : it imposes a
limit on the stepsize of the particle and computes the energy loss after
a step travelled by the particle. 
 
 
 \subsection{Step limit due to the continuous energy loss}

 The continuous energy loss imposes a limit on the stepsize. 
 The reason of this is the
energy dependence of the cross sections. It is generally assumed in MC programs that
the cross sections of the particles are approximately constant during a step , i.e.
 the step size should be so small that the relative difference of the cross sections
at the beginning of the step and at the end should be small enough.In principle one
has to use very small steps in order to have an accurate simulation , but the computing
time increases if the stepsize decreases. As a good compromise the stepsize is
limited here by the requirement that the stopping range of the particle can
decrease by not more than 20 \% during the step. This condition works fine for
a particle of kinetic energy \(> 0.5 MeV - 1. MeV\) , but for low energy it 
gives very short step sizes.
 To cure this problem a lower limitation on the stepsize is also introduced.
The choice of this lower limit is quite natural , the stepsize can not be
smaller than the {\em cut in range} parameter of the program. The stepsize 
limit varies smoothly with decreasing energy from the value given by the
condition \(\Delta range/range=0.20\) to the lowest possible value
 {\em cut in range} .
It is worthwile to mention , that the lower limit used here is much more
natural and physical , than the one is used in GEANT3 at the automatic
calculation of the tracking parameters. 

 \subsection{Computation of the energy loss}

 The computation of the {\em mean energy loss} after a given step is done
by using the range and inverse range tables. The mean loss \(\Delta T\) 
 can be written as

  \begin{equation}
       \Delta T = T_0 - f_T(r_0-step)
  \end{equation}

 where \(T_0\) is the kinetic energy,
       \(r_0\)    the range           at the beginning of the step \(step\),
        the function \(f_T(r)\) is the inverse of the range table (i.e. it
        gives the kinetic energy of the particle for a range value of r) .
       
 After the mean energy loss has been calculated the process computes the
 {\em actual} energy loss , i.e. the loss with fluctuation. The fluctuation
 is computed from a fluctuation model , from the model {\em GLANDZ} of
 the GEANT3 code \cite{eloss.geant3} , \cite{eloss.kati} .

 \section{Energy loss of muons}

  The energy loss of muons is computed by the class G4MuEnergyLoss. The
 scheme of the computation is the same as in the case of \(e+/e-\),
 the only difference is that now there are three processes contributing
 to the total continuous energy loss , these are

   the ionisation process (class G4MuIonisation), 
   the bremsstrahlung process (class G4MuBremsstrahlung) and
   the direct pair production of muons (class G4MuPairProduction). 
  
  The G4MuIonisation class computes the contribution to the continuous energy
 loss due to ionisation and simulates the corresponding 'discrete' process,
 the knock-on electron or \(\delta\)-ray production by muons.

  The G4MuBremsstrahlung class calculates the continuous loss due to the 
 {\em soft} bremsstrahlung and simulates the 'discrete', {\em hard}
 bremsstrahlung.

  The G4MuPairProduction class gives the contribution to the continuous 
 energy loss due to {\em sub-cutoff} \(e+/e-\) pairs and performs the
 simulation of the pair production.

 \section{Energy loss of charged hadrons}

 The continuous energy loss of the charged hadrons are calculated by the class
 G4hEnergyLoss . The algorithm is the same than in the case of \(e+/e-\)
 energy loss. Here there is only one process , the ionisation (class G4hIonisation)
 which  contributes to the 
 continuous energy loss .

 The G4hIonisation class computes the continuous energy loss and simulates 
 the \(\delta\)-ray producion by hadrons.

  In the case of energy loss of the hadrons two dE/dx , range and inverse range
 tables are constructed only , these are the tables for {\em proton} and
 {\em antiproton} . The energy loss for othe charged hadrons are computed
 from these tables at the  {\em scaled kinetic energy} \(T_{scaled}\)

   \begin{equation}
     T_{scaled} = \frac{ M_{proton} T}{ M_{particle}}
   \end{equation}

  where \(T\) is the kinetic energy of the particle, $M_{proton}$ and
  $M_{particle}$ are the masses of the proton and particle, respectively.

  There is an important constraint for this process:
  {\em the cut in range must be the same for all the charged hadrons} .
  This condition is meaningful physically and it makes possible to use
  the proton/antiproton tables only when computing the energy loss.
\section{Status of this document}
 9.10.98  created by L. Urb\'an.
 
\begin{thebibliography}{99}

\bibitem[GEANT3]{eloss.geant3}
  GEANT3 manual ,CERN Program Library Long Writeup W5013 (October 1994).
\bibitem[La95]{eloss.kati}
  K. Lassila-Perini and L. Urban , NIM A362 (1995),416.

\end{thebibliography}

