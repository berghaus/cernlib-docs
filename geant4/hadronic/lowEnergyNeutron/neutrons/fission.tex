For neutron induced fission, we take first chance, second chance, third chance 
and forth chance fission into account. 

Neutron yields are tabulated
as a function of both the incoming and outgoing neutron energy.
The neutron angular distributions are either tabulated, or represented in terms
of an expansion in legendre polynomials, similar to the angular distributions
for neutron elastic scattering. In case no data are available on the angular
distribution, isotropic emission in the centre of mass system of the collision
is assumed.

There are six different possibilities implemented to represent the neutron 
energy distributions. The energy distribution of the fission neutrons 
$f(E\rightarrow E')$ 
can be tabulated as a normalised 
function of the incoming and outgoing neutron energy, again using the ENDF/B-VI
interpolation schemes to minimise data volume and maximise precision. 

The energy distribution can also be represented 
as a general evaporation spectrum,
$$f(E\rightarrow E')~=~f\left(E'/\Theta(E)\right).$$
Here $E$ is the energy of the incoming neutron, $E'$ is the energy of a fission
neutron, and $\Theta(E)$ is effective temperature used to characterise the
secondary neutron energy distribution. Both the effective temperature and the
functional behaviour of the energy distribution are taken from tabulations.

Alternatively energy distribution can be represented 
as a Maxwell spectrum, $$f(E\rightarrow E')~\propto~\sqrt{E'}{\rm e}^{E'/\Theta(E)},$$
or a evaporation spectrum
$$f(E\rightarrow E')~\propto~E'{\rm e}^{E'/\Theta(E)}.$$
In both these cases, the temperature is tabulated as a function of the incoming
neutron energy.

The last two options are the energy dependent Watt spectrum, and the Madland
Nix spectrum. For the energy dependent Watt spectrum, the energy distribution
is represented as 
$$f(E\rightarrow E')~\propto~{\rm e}^{-E'/a(E)}\sinh{\sqrt{b(E)E'}}.$$
Here both the parameters a, and b are used from tabulation as function of the
incoming neutron energy.
In the case of the Madland Nix spectrum, the energy distribution is described
as
$$f(E\rightarrow E')~=~{1\over 2}\left[g(E',<K_l>)~+~g(E',<K_h>)\right].$$
Here 
$$g(E',<K>)~=~ {1\over 3\sqrt{<K>\Theta}}\left[u_2^{3/2}E_1(u_2)-u_1^{3/2}E_1(u_1)
+\gamma(3/2, u_2) - \gamma(3/2, u_1)\right],$$
$$ u_1(E',<K>) = {(\sqrt{E'}-\sqrt{<K>})^2 \over \Theta},~{\rm and}$$
$$ u_2(E',<K>) = {(\sqrt{E'}+\sqrt{<K>})^2 \over \Theta}.$$
Here $K_l$ is the kinetic energy of light fragments and $K_h$ the kinetic energy
of heavy fragments, $E_1(x)$ is the exponential integral, and $\gamma(x)$ is the
incomplete gamma function. The mean kinetic energies for light and heavy
fragments are assumed to be energy independent.
The temperature $\Theta$ is tabulated as a function of the kinetic
energy of the incoming neutron.

Fission photons are describes in analogy to capture photons, where evaluated
data are available. The measured nuclear excitation levels and transition
probabilities are used otherwise, if available.

As an example of the results is shown in figure\ref{fission} the energy
distribution of the fission neutrons in third chance fission
of 15~MeV neutrons on Uranium ($^{238}$U). This distribution contains two
evaporation spectra and one Watt spectrum.
Similar comparisons for neutron yields, energy and angular distributions, and
well as fission photon yields, energy and angular distributions have
been performed for
${\rm^{238}U}$, 
${\rm^{235}U}$, 
${\rm^{234}U}$, and 
${\rm^{241}Am}$
for a set of incoming neutron energies.
In all cases the agreement between evaluated data and Monte Carlo is very good.

\begin{figure}[b!] % fig 1
\centerline{\epsfig{file=hadronic/lowEnergyNeutron/neutrons/plots/fissionu238.tc.15mev.energy.epsi,height=3.5in,width=3.5in}}
\vspace{10pt}
\caption{Comparison of data and Monte Carlo for fission neutron energy
distributions for induced fission by 15~MeV neutrons on Uranium ($^{238}U$). 
The points are evaluated data, the histogram is
the Monte Carlo prediction.}
\label{fission}
\end{figure}


