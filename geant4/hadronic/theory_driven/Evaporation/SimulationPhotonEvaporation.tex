\section{Simulation of photon evaporation.}

\hspace{1.0em}The photon evaporation channel 
is taken into account as a competitor for fragment evaporation
and fission channels. For this case we consider only the giant 
dipole resonance photon 
evaporation.


\subsection{The total probability of $\gamma$ evaporation.}
\hspace{1.0em}As the first approximation we have neglected the nucleus 
angular momenta and 
 assume that 
 dipole $E1$--transitions is the main source of $\gamma$--quanta from 
highly--excited nuclei \cite{Iljinov92}.
The probability to evaporate $\gamma$
 in the energy interval $(\epsilon_{\gamma}, \epsilon_{\gamma}+d\epsilon_{\gamma})$ 
 per unit of time is given
\begin{equation}
\label{SPE7}W_{\gamma}(\epsilon_{\gamma}) =
 \frac{1}{(\pi \hbar c)^2}\sigma_{\gamma}(\epsilon_{\gamma})
\frac{\rho_{\gamma}(E^{*}-\epsilon_{\gamma})}
{\rho_c(U_c)}\epsilon^2_{\gamma},
\end{equation}
where $\sigma_{\gamma}(\epsilon_{\gamma})$ is the inverse (absorption of  $\gamma$)
reaction cross section, 
$\rho_{\gamma}(U-\epsilon_{\gamma})$ and $\rho(U)$ are level densities of nucleus after and
before $\gamma$ evaporation, respectively.

The photoabsorption reaction cross section is given by the expression
\begin{equation}
\label{SPE8} 
\sigma _{\gamma}(\epsilon_{\gamma}) = 
\frac{\sigma_0 \epsilon^2_{\gamma} \Gamma^2_{R}}
{(\epsilon^2_{\gamma} - E_{GDP}^2)^2 + \Gamma^2_R\epsilon^2_{\gamma}},
\end{equation}
where $\sigma_0=2.5A$ mb, $\Gamma_R=0.3E_{GDP}$ and $E_{GDP}= 40.3 A^{-1/5}$ MeV are 
empirical parameters of the giant dipole resonance \cite{Iljinov92}.
The total radiation probability is
\begin{equation}
\label{SPE9}W_{\gamma} =\frac{3}{(\pi \hbar c)^2}\int_{0}^{E^{*}}
 \sigma_{\gamma}(\epsilon_{\gamma})
\frac{\rho_{\gamma}(E^{*}-\epsilon_{\gamma})}
{\rho_c(U_c)}\epsilon^2_{\gamma}d\epsilon_{\gamma}.
\end{equation}
The integration is performed numericaly.

\subsection{Energy of evaporated photon.}
\hspace{1.0em}The energy of $\gamma$-quantum is sampled 
according to the Eq. $(\ref{SPE7})$
distribution.

\subsection{Discrete photon evaporation.}

\hspace{1.0em} The last step of evaporation cascade consists of evaporation of 
photons with discrete energies. The competition between photons and 
fragmrnts as well as giant resonance photons is neglected at this step. 

If we want simulate evaporation of photons with 
different multipolarities $L$ ( e.g. electric and magnetic dipole $E1$ and 
$M1$ as well as the electric quadrupole $E2$ photons),
 we need more careful consideration of  
the angular momenta and parities for the compound and residual nuclei. 
For this case we should rewrite the 
probability to evaporate  particle
$b$ in the energy interval $(T_b, T_b+dT_b)$ per unit of time. 
This probability
 is given by \cite{IKP94}  
\begin{equation}
\label{SPE1}W_b (J_c, \pi_c,J_b,\pi_b,T_b)=\frac 1\hbar
\frac{\rho_b (U_b-Q_b-T_b,J_b,\pi_b)}{\rho_c (U_c,J_c, \pi_c)%
}\sum_{S=|J_b-s|}^{|J_b+s|}\sum_{l=|J_c-S|}^{|J_c+S|}T_l(T_b) 
\end{equation}
where $J_c$, $\pi_c$ and $J_b$, $\pi_b$ are the angular momenta and 
parities of the compound nucleus and 
 residual
nucleus, respectively and $s$ is fragment spin. 
The parity 
conservation should be taken into account in the summation over $l$.
   The transmission coefficients
of fragments $T_l(T_b)$ can be derived 
from the inverse absorption cross sections 
\cite{IKP94} 
\begin{equation}
\label{SPE2} \frac{d\sigma_b(T_b)}{d\bf{l}}=\lambda^2
T_l(T_b)\delta(\bf{n},\bf{l}),
\end{equation}
where $\lambda$ is the de Broglie wavelength. The 
delta function takes into 
account the fact that the orbital momentum $\bf{l}$ is perpendicular 
to the direction $\bf{n}$ of motion of fragment.

Similarly for the photon evaporation:
\begin{equation}
\label{SPE3}W_\gamma ^L(J_c,J_{\gamma}, \epsilon_{\gamma})=
\frac 1\hbar\frac{\rho (E^{*}-\epsilon_{\gamma},J_{\gamma},\pi_{\gamma})}{\rho
(U_c,J_c,\pi_c)}f_L(\epsilon _\gamma ) 
\end{equation}
where $f_L(\epsilon _\gamma )$ are the strength functions, 
 $L$ is the multipolarity of the $\gamma$-transition and 
$\epsilon _{\gamma}$ is the photon energy. 
The strength functions for the photon evaporation 
can be derived from the photoabsorption cross 
sections. This approach requires the knowledge of the resonance 
parameters for nuclei (see above). Another simplified 
approach to obtain the strength 
functions is based on the single particle transition estimates:
\begin{equation}
\label{SPE4}f_L(\epsilon _\gamma )=\chi_{L}\epsilon_{\gamma}^{2L+1},
\end{equation}  
where  $\chi_{L}$ are constants can be found in 
 \cite{Wilkinson60} or in  \cite{Endt81}.
 
The widths of different decays:
\begin{equation}
\label{SPE4}\Gamma_b (J_c)=\hbar\int_{V_b}^{U-Q_b}\sum_{J_b}
W_b (J_c,J_b,T_b)dT_b
\end{equation}
\begin{equation}
\label{SPE5}\Gamma _{\gamma}
^L(J_c)=\hbar
\sum_{J_{\gamma}=|J_c-L|}^{|J_c+L|}\int_0^{E^{*}}
\Gamma_\gamma ^L(J_c,J_{\gamma}, \epsilon_{\gamma})d\epsilon _{\gamma}. 
\end{equation}

As we already discussed (see Eq. $(\ref{SFE22})$) 
one can approximate level density by 
\begin{equation}
\label{SPE6}\rho (U,J)=\frac{\sqrt{a}(2J+1)\hbar^3}{48\sqrt{2\theta ^3}U^2}\exp (2 
\sqrt{aU}). 
\end{equation}

However, so far we consider the discrete E1, M1 and E2 photon evaporation 
from only tabulated isotopes. 
There are large number of isotopes  \cite{ENSDF} with the experimentally
 measured exited level energies, angular momenta and relative transitions 
 probabilities. This information is 
 implemented in the code.
