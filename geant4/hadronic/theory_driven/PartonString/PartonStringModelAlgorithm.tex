\section{Parton string model algorithm.}

\hspace{1.0em} 
The parton string model algorithm can be considered as a set of 
steps should be performed:
\begin{enumerate}
\item Create two vectors (projectile and target) of particles: assign initial 
projectile and target particle types, their coordinates and momenta.
In the case of hadron-nucleus (or nucleus-nucleus) interaction 
one should perform
target nucleus (or projectile and target nuclei) initial state simulation
 and sample impact parameter;
\item Sample 
collision participants and separated them into diffractive and non-diffractive.
Store the total interaction four momentum of participants;
\item For non-diffractive inelastic collisions sample the number 
of the soft longitudinal and hard kinky string can be produced.
\item Excite and reexcite colliding particles in the case of diffractive collisions and 
create diffractive longitudinal strings, if particles are not participate in 
further soft or hard collisions;
\item Perform longitudinal and kinky string excitations;
\item Perform string decay simulation;
\item Correct energies and momenta of produced particles, if it is needed.
\end{enumerate}
