\section{Longitudinal string decay.}

\subsection{Hadron production by string fragmentation.}
\hspace{1.0em}A string is stretched between flying away constituents:
quark and antiquark or quark and diquark or diquark and antidiquark or
antiquark and antidiquark.  From knowledge of the constituents
longitudinal $p_{3i}=p_{zi}$ and transversal $p_{1i}=p_{xi}$,
$p_{2i}=p_{yi}$ momenta as well as their energies $p_{0i} = E_{i}$,
where $i=1,2$, we can calculate string mass squared:
\begin{equation}
\label{LSD1} M^2_S=p^{\mu}p_{\mu}=p^2_0-p^2_1-p^2_2-p^2_3,
\end{equation}
where $p_{\mu} = p_{\mu 1}+p_{\mu 2}$ is the string four momentum and 
 $\mu = 0,1,2,3$. 
 
The fragmentation of a string follows an iterative scheme:
\begin{equation}
string \Rightarrow hadron + new \ string,
\end{equation}
\textit{i. e.} a quark-antiquark (or diquark-antidiquark) pair is
created and placed between leading quark-antiquark (or diquark-quark or
diquark-antidiquark or antiquark-antidiquark) pair.

The values of the strangeness suppression and diquark suppression
factors are
\begin{equation}
\label{LSD2} u:d:s:qq = 1:1:0.35:0.1.
\end{equation}

A hadron is formed randomly on one of the end-points of the string. The
quark content of the hadrons determines its species and charge.  In the
chosen fragmentation scheme we can produce not only the groundstates of
baryons and mesons, but also their lower excited states.  If for baryons the
quark-content does not determine whether the state belongs to the lowest
octet or to the lowest decuplet, then octet or decuplet are choosen with
probability which defined by spin and isospin of diquark (antidiquark).
 In the case
of mesons the multiplet must also be determined before a type of hadron
can be assigned. The probability of choosing a certain multiplet depends
on the spin of the multiplet.

In case of resonances the mass $m$ is determined according to
Breit-Wigner distribution:
\begin{equation}
\label{LSD3} F(m) = 2\pi\frac{\Gamma(m_R)}{(m_R - m)^2 + \Gamma(m_R)^2/4},
\end{equation}
All pole masses $m_R$ and total decay widths $\Gamma(m_R)$ are taken
from the Review of Particle Properties \cite{PDG96}. 

 The zero transverse
momentum of created quark-antiquark (or diquark-antidiquark) pair is
defined by the sum of an equal and opposite directed transverse momenta
of quark and antiquark.

The transverse momentum of created quark is randomly sampled according
to probability ($\ref{LSE4}$) 
with the parameter $a = 0.55$ GeV$^{-2}$. Then a hadron transverse momentum
${\bf p_t}$ is determined by the sum of the transverse momenta of its
constituents.

The fragmentation function $f^h(z,p_t)$ represents the probability
distribution for hadrons with the transverse momenta $\bf{p_t}$ to
aquire the light cone momentum fraction 
$z=z^{\pm}=(E^h \pm p^h_z/(E^q \pm p^q_z)$, where $E^h$ and $E^q$ are 
the hadron and fragmented quark energies, respectively 
and $p^h_z$ and $p^q_z$ are hadron and fragmented quark longitudinal momenta, 
respectively, 
and $z^{\pm}_{min} \leq z^{\pm} \leq z^{\pm}_{max}$, from the
fragmenting string. The values of $z^{\pm}_{min,max}$ are determined by
hadron $m_h$ and constituent transverse masses and the available string
mass. One of the most common fragmentation function is  used in
the LUND model \cite{LUND83}:
\begin{equation}
\label{LSD5} f^h (z, p_t) \sim \frac{1}{z}(1-z)^a
\exp{[-\frac{b(m_h^2+p_t^2)}{z}]}.
\end{equation}
One can use this fragmentation function for the decay of the excited 
string.

One can use also the fragmentation functions are derived in \cite{Kai87}:
\begin{equation}
\label{LSD6} f^{h}_q (z, p_t)=[1+\alpha^h_q(<p_t>)] (1-z)^{\alpha^h_q(<p_t>)}.
\end{equation}
The advantage of these functions as compared to the LUND fragmentation
function is that they have correct three--reggeon behaviour at
$z\rightarrow 1$ \cite{Kai87}.

\subsection{The hadron formation time and coordinate.}

\hspace{1.0em}To calculate produced hadron formation times and longitudinal 
coordinates we consider the $(1+1)$-string  with mass $M_S$ and string
tension $\kappa$, which decays into hadrons at string rest frame.  The
$i$-th produced hadron has energy $E_i$ and its longitudinal momentum
$p_{zi}$, respectively. Introducing light cone variables $p^{\pm}_{i}=
E_{i} \pm p_{iz}$ and numbering string breaking points consecutively
from right to left we obtain $p^{+}_{0} = M_{S}$, $p_{i}^{+}=\kappa
(z^{+}_{i-1}-z_i^{+})$ and $p_{i}^{-} = \kappa x^{-}_i$.

We can identify the hadron formation point coordinate and time as the
point in space-time, where the quark lines of the quark-antiquark pair
forming the hadron meet for the first time (the so-called 'yo-yo'
formation point \cite{LUND83}): 
\begin{equation}
\label{LSD7}t_i = \frac{1}{2\kappa}[M_S - 2 \sum_{j=1}^{i-1}p_{zj} + E_i -
p_{zi}]
\end{equation}
and coordinate 
\begin{equation}
\label{LSD8}
z_i = \frac{1}{2\kappa}[M_S - 2 \sum_{j=1}^{i-1}E_{j} + p_{zi}- E_i].
\end{equation}

\subsection{Lorentz boost and rotation of string.}

\hspace{1.0em}The simulation of string decay is considered in the
rest of string frame, with string end quarks are moving along $z$-axis.
We can perform Lorentz transformation to the c.m. of string of the
constituent momenta:
\begin{equation}
\label{LSD9} p_{\mu 1,\mu 2}\rightarrow L_{\mu}p_{\mu 1,\mu 2},
\end{equation}
where 
\begin{equation}
\label{LSD10} L_{0}=\beta^{\nu}p_{\nu}
\end{equation}
and 
\begin{equation}
\label{LSD11} L_k=\beta_kp_0 + \sum_{l=1}^{3}(\delta_{lk} +
\frac{\beta_k\beta_l}{1+\beta_0})p_l.
\end{equation}
 $\beta_{\mu}$ is defined as follows:
\begin{equation}
\label{LSD12} \beta_{\mu} = \frac{p_{\mu}}{M_S}.
\end{equation}
The string orientation relatively $z$-axis is determined by two Euler
angles $\alpha$ and $\beta$, which can be calculated according to
\begin{equation}
\label{LSD13} \cos{\alpha}= \frac{p_{32}}{\sqrt{p_{22}^2 + p_{32}^2}}
\end{equation}
and
\begin{equation}
\label{LSD14} \cos{\beta}= \frac{\sqrt{p_{22}^2 + p_{32}^2}}
{\sqrt{p_{12}^2 + p^2_{22} + p_{32}^2}}.
\end{equation}
Then,  by string rotation:
\begin{equation}
\label{LSD15} p_{k1,k2} \rightarrow R_{kl}p_{l1,l2}, 
\end{equation}
we can obtain the motion of constituents along the $z$-axis.  Here the
matrix $R_{kl}$ is given by
\begin{equation}
\label{LSD16} R_{kl}=
\begin{array}{|ccc|}
\cos{\beta} & -\sin{\alpha}\cos{\beta}&
-\cos{\alpha}\sin{\beta}\\
0 & \cos{\alpha}&
-\sin{\alpha}\\
\sin{\beta} & -\sin{\alpha}\cos{\beta}&
-\cos{\alpha}\cos{\beta}
\end{array}
\end{equation}
and $k,l =1,2,3$.

Finaly after string decay, using 
Eq. ($\ref{LSD9}$) and Eq. ($\ref{LSD15}$) we have to
perform the backward Lorentz boost with $-\beta_{\mu}$ of hadron
4-momenta and 4-coordinates and the backward rotation of hadron
3-momenta and 3-coordinates with $R^{-1}_{kl}$.
