\section{Stopped particle absorption simulation.}

\subsection{Mechanism of the stopped particle absorption by a nucleus.}
\hspace{1.0em} 
An absorption of a stopped $\pi^{-}$-meson, $K^{-}$-meson and $\bar{p}$
 by a nucleus procceeds in several
 steps \cite{IKP94}:
\begin{enumerate}
\item A particle is captured by the Coulomb fiels of a nucleus forming a pionic
or a kaonic or $\bar{p}$-atom;
\item Such atom de-excites through the emission of Auger-electrons and $X$-rays;
\item A stopped particle from the atomic orbit is captured by nucleus (
 by a pair or more of intranuclear nucleons in the case of a stopped pion or 
 by reaction on a quasifree nucleon producing a pion and $\Lambda$ or
$\Sigma$ hyperon in the case of a stopped kaon or
 by annihilation on a quasifree nucleon in the case of $\bar{p}$-capture);
\item Rescatterings of fast nucleons and pions produced in a stopped 
particle absorption (hadron
kinetics);
\item Decay of excited residual nucleus (nucleus deexcitation).
\end{enumerate}

Thus the absorption processes for the stopped pion, kaon and antiproton are
similar. However, there are some absorption 
peculiarities for each type of particles.
   
\subsection{Absorption of stopped $\pi^{-}$ by nucleus.}
\hspace{1.0em} It is simulated  by the kinetic model.
As follows from calculations within the framework of the optical
model \cite{INC76} with the Kisslinger potential \cite{Kiss55}
 the capture a pion from an orbit of atom takes place at
radius $r$ in the
nuclear surface and absorption probability $P_{abs}(r)$ can be approximated by 
\begin{equation}
\label{SAS1} P_{abs}(r) = P_0 \exp{[-0.5(\frac{r-R_{\pi}}{D_{\pi}})^2]},
\end{equation}
where parameters of the Gaussian distribution $R_{\pi} \approx R_{1/2}$, 
where $R_{1/2}$ is the half-density radius, and $D_{\pi}$
 for different nuclei can be found
in \cite{IKP94}.

The absorption of the  pion is considered as the $s$-wave (non-resonant) 
absorption mainly by the the simplest cluster consisting of two nucleon 
$(np)$ or $(pp)$. 

Once a pion has been absorbed by a nucleon pair, the pion mass is converted
into kinetic energy of nucleon. Each nucleon has the energy $E_N = m_{\pi}/2$ 
in the center of mass pair. In the center of mass nucleons flay away in
opposite direction isotropically.
The inital momentum of pair is taken as a sum of nucleon Fermi momenta.

\subsection{Absorption of stopped $K^{-}$ by nucleus.}
\hspace{1.0em} It is simulated in the kinetic model framework.
In this case the absorption probability was choosen the same as in
annhihilation of the stopped antiprotons.

\subsection{Annihilation of stopped $\protect\bar{p}$ by nucleus.}
In this case the absorption probability was also given by equation of 
(\ref{SAS1}) with the values of $R_{\bar{p}} = R_{\pi}$ and dispertion $D^2 = 
1$\ fm$^2$ \cite{INC82}.

The annhihilation of antiproton on a quasifree nucleon is modelled via the
annihilation of a diquark-antidiquark with subsequent fragmentation of the 
meson string as it was done in the parton string model. 
