\section{Selection of particle collisions and particle decays.}

\subsection{Particle collision criterion and collision time.}

\hspace{1.0em}For each pair $(i,j)$ of particles their total interaction
cross section $\sigma_{tot}(s_{ij})$ is calculated. This cross section
depends on the quantum numbers of the particles. It is also a function of
the total particle c.m. energy squared $s_{ij} = (p_i + p_j)^2)$, where
$p_i = (E_i, {\bf p_i})$ and $p_j=(E_j, {\bf p_j})$ are four momenta of
particle $i$ with mass $m_i = \sqrt{E^2_i - {\bf p_i}^2}$ and particle
$j$ with mass $m_j = \sqrt{E^2_j - {\bf p_j}^2}$, respectively.  The
cross section is interpreted geometrically as an interaction area with
radius
\begin{equation}
\label{SCD1}b_{ij} = \sqrt{\frac{\sigma_{tot}(s_{ij})}{\pi}}.
\end{equation}
It was assumed that two particles will collide, if their transverse
(with regards to the relative velocity vector of the particles) distance
$d_{ij}$ fulfils the condition:
\begin{equation}
\label{SCD2} d_{ij} \leq b_{ij}.
\end{equation}
For the distance $d_{ij}$ we choose
 the relative distance between two particles at
the time of closest approach calculated in the c.m. frame of two
particles with coordinates ${\bf x^{*}_i}$ and ${\bf x^{*}_j}$ and
momenta ${\bf p^{*}_i}$ and ${\bf p^{*}_j}$, respectively. It is
determined by
\begin{equation}
\label{SCD3} d^2_{ij} = ({\bf x^{*}_i} - {\bf x^{*}_j})^2 - 
\frac{[({\bf x^{*}_i} - {\bf x^{*}_j})({\bf p^{*}_i} - {\bf p^{*}_j})]^2}
{({\bf p^{*}_i} - {\bf p^{*}_j})^2}.
\end{equation}
This distance also can be    
written as the Lorentz invariant expression.

Because the particles are distance $d_{ij}$ apart, in an reference frame
the collision event will have two different times $t_{ij}$ and $t_{ji}$.
If we will consider more than two particles, each particle will have a
set of possible collision times. Therefore the time ordering of the
individual binary collisions strongly varies with the respective
reference frame. The final results does depend on this criterion
\cite{KDCDN84}, \cite{KBHMP95}. 

We order collisions in the reference frame of the hadron-nucleus or
nucleus-nucleus interaction.
Thus using the particle locations ${\bf x}$ and the particle
momenta ${\bf p}$ in the reference frame of the hadron-nucleus or
nucleus-nucleus interaction one can obtain for the time of closest
approach of the two colliding particles:
\begin{equation}
\label{SCD4} \tau_{coll}=t_{ij}=t_{ji}=
-\frac{({\bf x_i} - {\bf x_j})  ({\bf p_i}/E_i - {\bf
p_j}/E_j)}{({\bf p_i}/E_i - {\bf p_j}/E_j)^2}.
\end{equation} 
  
\subsection{Hadron resonance lifetime.}

\hspace{1.0em}The lifetime  $t$ for each particle with mass $M_R$ was
sampled according to probability:
\begin{equation}
\label{SCD5} P(t) = 1 -\exp{(-\frac{t\Gamma_{tot}(m)}{\gamma \hbar}}, 
\end{equation}
where $\gamma = E/m$ is the Lorentz factor, defined through particle
total energy $E$. The particle (resonance) width $\Gamma_{tot}(m)$
depends from particle mass $m$.

The full decay width $\Gamma_{tot}(m)$ of a resonance can be defined as
the sum of all partial decay widths and depends on the mass $m$ of the
excited resonance:
\begin{equation}
\label{SCD6} \Gamma_{tot}(m) = \sum_{br =\{i,j\}}^{N_{br}} \Gamma_{i,j}(m),
\end{equation}
where the partial decay widths $\Gamma_{i,j}(m)$ for the decay into the
exit channel with particles $i$ and $j$ is given by \cite{URQMD97}
\begin{equation}
\label{SCD7}\Gamma_{i,j}(m)=\Gamma_{i,j}(m_R)\frac{m_R}{m}(\frac{p_{i,j}(m)}
{p_{i,j}(m_R)})^{2l+1}\frac{1.2}{1+0.2(\frac{p_{i,j}(m)}
{p_{i,j}(m_R)})^{2l}},
\end{equation} 
where $m_R$ is the pole mass of the resonance, $\Gamma_{i,j}(m_R)$ its
partial width into the channel $i$ and $j$ at the pole and $l$ the decay
angular momentum of the exit channel and
\begin{equation}
\label{SCD8} p_{i,j}(m) = p_{c.m.}(m, m_i, m_j)= 
\frac{1}{2m}\sqrt{(m^2 - (m_i + m_j)^2)(m^2 - (m_i -
m_j)^2)}. 
\end{equation}
 All pole masses and partial
decay widths are taken from the Review of Particle Properties
\cite{PDG96}.
