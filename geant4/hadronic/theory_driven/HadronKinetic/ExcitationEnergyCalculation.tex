\section{Calculation of a residual nucleus excitation energy.}

\hspace{1.0em}The excitation energy $U$ of the residual nucleus is
defined as the energy above the ground state energy (mass)
$M_{res}(A_{res},Z_{res})$ of residual nucleus with mass number
$A_{res}$ and charge $Z_{res}$.
For the precompound model we need to the number of excitons, which 
consist from 
$N_p$ is the number of "particles" and $N_{h}$ is the number of
"holes".  The located at $r$ particle and hole energies are 
calculated from the Fermi
energy $T_F(r)$.
All cascade nucleons with kinetic energies above $T_{F}$ and less than
$T<T_F+B+V_C+E_{cut}$, where $B(A,Z)$ is binding energy, $V_C$ is
Coulomb energy for proton and $E_{cut}$ is cut energy (it is a model
parameter, $E_{cut}\approx 15$ MeV), are called "particles". The "holes"
are results of cascade interactions, when particles stroke nucleons from
nucleus and they occupy states below $T_{F}$.


Thus in hadron-nucleus and nucleus-nucleus collision cases the number
 of holes can be calculated for 
 target nucleus and for projectile nucleus since we know which (projectile or 
target) nucleon was stroked and 
know the relative (from centrum of nucleus) position of a nucleon. In 
this case we are able to calculate difference between scattered 
nucleon (the mass of 
this nucleon is a table mass) kinetic energy (defined in a nucleus rest 
frame) and nucleon Fermi energy at given 
relative (from centrum of a nucleus) position of this nucleon. If this 
difference is less than the nucleon binding energy in a given nucleus plus 
some cut energy, then the nucleon is considered as a particle 
absorbed in a nucleus. Due to internuclear potentials other 
particles can be also absorbed by a nucleus.

We know which particle leaves a nucleus. We can reduce the leaving 
particle energy by the effective potential ("recoil" energy of nucleus after 
left particle \cite{FRRS96} 
  and rescale (taking into account "recoil" momentum of left particle)
   its momentum.
The nucleus excitation energy 
can be calculated by summing of 4-momenta of 
particles, which belong to a nucleus. Then excitation energy of 
a nucleus can be calculated by adding a "recoil" energy and 
subtracting of the ground state nucleus energy 
(calculated using a liquid drop model)$M_{res}(A_{res},Z_{res}$ 
from the c.m. total energy $E_{res}(A_{res},Z_{res})$ of the 
rest nucleus system:
\begin{equation}
\label{EEC1}U=E_{res}(A_{res},Z_{res})+ E_{recoil} -M_{res}(A_{res},Z_{res}).
\end{equation}

