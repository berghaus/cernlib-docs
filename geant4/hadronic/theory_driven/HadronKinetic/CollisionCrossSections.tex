\section{Hadron interaction cross sections.}

\hspace{1.0em}The hadron interaction cross sections are a function of
the incoming and outgoing particle types and their c. m. energy
$\sqrt{s_{i,j}} = \sqrt{(p_i + p_j)^2}$, where $p_i = (E_i, {\bf p_i})$
and $p_j=(E_j, {\bf p_j})$ are denoting the four momenta of the incoming
particle $i$ with mass $m_i = \sqrt{E^2_i - {\bf p_i}^2}$ and the
incoming particle $j$ with mass $m_j = \sqrt{E^2_j - {\bf p_j}^2}$,
respectively.  The hadron interaction cross sections were either be
tabulated, parametrized according to algebraic functions 
 or extracted from other cross
sections via general principles, such as isotopical invariance,
detailed balance and additive quark model.

\subsection{Hadron interaction tabulated and parametrized cross sections.}

\hspace{1.0em}The total  and elastic $pp$, $pn$, $\bar{p}p$,
$\pi^{\pm}p$ and $K^{\pm}p$ 
cross sections are well known \cite{PDG96}. 
 We tabulated $pp$ and $pn$ total and 
elastic cross sections at low energies $\sqrt{s_{i,j}} < 5$ \ GeV and at 
higher energies we use the CERN/HERA parametrizations \cite{PDG96}. 
The CERN/HERA
parametrizations \cite{PDG96} are used to obtain the $\bar{p}p$ 
and $K^{+}p$ as well as the high energy
 non-resonance $\pi^{\pm}p$ and $K^{-}p$ 
total and elastic cross sections.
The functional
dependence of the total $\pi^{\pm}p$ and $K^{-}p$ cross sections 
 on the total particle c.m. energy $\sqrt{s_{i,j}}$ shows at
low energies the resonance behaviours. These cross sections are calculated.
We calculate also nucleon-nucleon cross sections for resonance excitation 
processes.
The calculation of other cross
sections using general principles, such as isotopical invariance,
detailed balance and additive quark model is explained below.

\subsection{The isotopical invariance.}

\hspace{1.0em}As follows from experimental observations of charge or
isotopical invariance of nuclear forces the hadron interaction
scattering amplitudes depend only on the total isospin $T$ but not on
its projection $T^z$.  It gives a possibility to separate cross sections
according to isospin and find relations between different hadron
interaction cross sections.

Each hadrons $h$ can be characterized by isospin $T$ and its projection
$T^z$, i. e.  using Dirac's notations, $|h> = |T, T^z>$. Thus,
e. g. nucleon and pion states can be defined as follows: $|p> = |1/2,
1/2>$, $|n> = |1/2, -1/2>$, $|\pi^{+}> = |1, 1>$, $|\pi^{-}> = |1, -1>$
and $|\pi^{0}> = |1, 0>$.  The $|np>$, $|pp>$, $|nn>$ and different
pion-nucleon states are given by
\begin{equation}
\label{CCS1} |h_1h_2> = \sum_{|T_{h_1} - T_{h_2}|}^{T_{h_1} + T_{h_2}}|T_{h_1h_2}
T_{h_1h_2}^z>
<T_{h_1}T_{h_1}^{z}T_{h_2}T_{h_2}^{z}||T_{h_1h_2}T_{h_1h_2}^z>,
\end{equation}
where $<T_{h_1}T_{h_1}^{z}T_{h_2}T_{h_2}^{z}||T_{h_1h_2}T_{h_1h_2}^z> $
are Clebsch-Gordan coefficients, $T^{h_1}$,$T^{h_1}_z$ and
$T^{h_2}$,$T^{h_2}_z$ are isospin and its projection for hadron $h_1$
and hadron $h_2$, respectively and $T^{h_1h_2}$, $T_{h_1h_2}^z =
T_{h_1}^z + T_{h_2}^z$ are the total isospin and its projection of the
$h_1, h_2$ system.  The Clebsch-Gordan coefficient squared gives the
probability to find hadron system in a state $|T_{h_1h_2}T_{h_1h_2}^z>$.
In the nucleon-nucleon ($T_{NN} = 0$ or $T_{NN} = 1$) or in the
pion-nucleon ($T_{\pi N} = 1/2$ or $T_{\pi N} = 3/2$) system only two
values of isospin are allowed. If we consider reactions
\begin{equation}
\label{CCS2}
\begin{array}{ccc}
p + p \rightarrow p + p,\\
p + n \rightarrow p + n,\\
n + n \rightarrow n + n,
\end{array}
\end{equation}
then corresponding amplitudes are related to the isospin amplitudes:
\begin{equation}
\label{CCS3}
\begin{array}{ccc}
A_{p + p \rightarrow p + p} = <1, 1|1,1> =A_1,\\
A_{p + n \rightarrow p + n} =  1/2<1, 0|1,0>+1/2<0,0|0,0> = 
1/2A_1 + 1/2 A_0,\\
A_{n + n \rightarrow n + n}= <1,-1|1, -1> = A_1,
\end{array}
\end{equation}
where $A_1$ and $A_0$ are the $1\rightarrow 1$ and $0\rightarrow 0$ isospin 
amplitudes.
If we consider reactions
\begin{equation}
\label{CCS4}
\begin{array}{ccc}
\pi^{+} + p \rightarrow \pi^{+} + p,\\
\pi^{-} + p \rightarrow \pi^{0} + n,\\
\pi^{0} + p \rightarrow \pi^{0} + p,
\end{array}
\end{equation}
then corresponding scattering amplitudes are also related to the isospin
amplitudes:
\begin{equation}
\label{CCS5}
\begin{array}{ccccccc}
A_{\pi^{+} + p \rightarrow \pi^{+} + p}= 
<3/2, 3/2|3/2, 3/2> = A_{3/2},\\
A_{\pi^{-} + p \rightarrow \pi^{0} + n} = \\
=\sqrt{2}/3<3/2, -1/2|3/2, -1/2>  - 
\sqrt{2}/3<1/2, -1/2|1/2, -1/2>= \\
=\sqrt{2}/3A_{3/2} -\sqrt{2}/3A_{1/2},\\
A_{\pi^{0} + p \rightarrow \pi^{0} + p}= \\
=2/3<3/2, 1/2|3/2, 1/2> + 1/3<1/2, 1/2|1/2, 1/2> = \\
=2/3A_{3/2}+1/3A_{1/2},
\end{array}
\end{equation}
where $A_{3/2}$ and $A_{1/2}$ are the $3/2\rightarrow 3/2$ and
$1/2\rightarrow 1/2$ isospin amplitudes. The optical theorem links the
total particle interaction cross section $\sigma^{ij}_{tot}(s_{ij})$ and
the imaginary part of the scattering amplitude $A_{ij}(0)$ at zero
scattering angle:
\begin{equation}
\label{CCS6} \sigma_{tot}(s_{ij}) = \frac{1}{s_{ij}}Im A_{ij}(0).
\end{equation}
Thus, e. g. for the nucleon-nucleon system isotopical invariance leads
to $\sigma^{nn}_{tot,el} = \sigma^{pp}_{tot,el}$ and for the
pion-nucleon system one has to measure only the first two of the above
considered elastic and charge-exchange cross sections, the third one and
all other possible elastic and charge-exchange cross sections can be
determined by them. The similar consideration can be done for
kaon-nucleon and other systems.

\subsection{The additive quark model cross sections.}

\hspace{1.0em}
In the additive quark model (AQM) cross section depends only
from the quark content of the colliding hadrons \cite{Goulianos83}:
\begin{equation}
\label{CCS7}\sigma^{AQM}_{tot}=40 (\frac{2}{3})^{n_{M}}(1 - 0.4 x^s_1)(1-0.4 
x^s_2)
\end{equation}
and
\begin{equation}
\label{CCS8}\sigma^{AQM}_{el}=0.039\sigma_{tot}^{\frac{2}{3}},
\end{equation}
where $n_{M}$ is the number of colliding mesons and $x^s_i$ is the 
ratio of strange quarks to non-strange quarks in the $i$-th hadron.
Thus we can use the AQM  to obtain unknown cross section by the 
 "scaling" of known cross sections (see below an example).

\subsection{Detailed balance cross sections for resonance-nucleon 
interactions.}

\hspace{1.0em}The principle of detailed balance is based on the
time-reversal invariance of the matrix element of the reaction. It gives
a possibility to determine the differential and total resonance-nucleon
and resonance-resonance cross sections, e. g. cross sections for the
processes: $\Delta_{1232}N\rightarrow NN$ and
$\Delta_{1232}\Delta_{1232}\rightarrow NN$.  It should be mentioned that
inelastic baryon-resonance deexcitations are very important for nuclear
absorption of mesons, which are "bound" in resonances.  According to
the detailed balance the cross sections of inverse $f\rightarrow i$ and
direct $i\rightarrow f$ reactions, where $i$ and $f$ denote initial and
final states, respectively, are related each other as follows:
\begin{equation}
\label{CCS9} \sigma_{f\rightarrow i}=\frac{{\bf p^2_i}}{{\bf p^2_f}}\frac{g_i}
{g_f}\sigma_{i\rightarrow f},
\end{equation}
where ${\bf p^2_{i,f}}$ are the c. m. momenta of the initial and final
states,
$g_{i,f}=(2S_{i1,f1}+1)(2S_{i2,f2}+1)(2T_{i1,f1}+1)(2T_{i2,f2}+1)$ are
denoting the spin-isospin degeneracy factors.

However, the Eq. ($\ref{CCS9}$) is exacly valid in the case of stable
particles with well-defined masses. For the case of one incoming
resonance this equation should be  modified as it has been
derived in \cite{DB91}. 
 
\subsection{Baryon annihilation cross sections.} 

\hspace{1.0em}The baryon annihilation cross sections are a function of
the incoming and outgoing particle types and their c. m. energy
$\sqrt{s_{i,j}} = \sqrt{(p_i + p_j)^2}$, where $p_i = (E_i, {\bf p_i})$
and $p_j=(E_j, {\bf p_j})$ are denoting the four momenta of the incoming
particle $i$ with mass $m_i = \sqrt{E^2_i - {\bf p_i}^2}$ and the
incoming particle $j$ with mass $m_j = \sqrt{E^2_j - {\bf p_j}^2}$,
respectively.  We follow the approach \cite{URQMD97} to calculate 
 baryon-antibaryon annihilation cross sections. In this approach the same 
 initial energy dependence is ised for all baryon-antibaryon cross sections 
 and different baryon quark content is taken into account by scaling 
 factor obtained from the additive quark model (AQM)\cite{Goulianos83} 
 (in the additive quark model cross section depends only
from the quark content of the colliding hadrons): 
\begin{equation}
\label{CCS10}\sigma_{\bar{B}B}(\sqrt{s})=\frac{\sigma^{AQM}_{BB}}
{\sigma^{AQM}_{NN}}
\sigma_{\bar{N}N}(\sqrt{s}),
\end{equation} 
where antiproton-proton annihilation cross section $\sigma^{ann}_{\bar{p}p}$ 
is parametrized as \cite{KD89}
\begin{equation}
\label{CCS11}\sigma^{ann}_{\bar{p}p}=\sigma_{0}\frac{s_0}{s}[\frac{A^2s_0}
{(s-s_0)^2+A^2s_0}+B]
\end{equation}
and antiproton-neutron cross annihilation cross section is treated identicaly.
The values of parameters are $\sigma_{0}=120$\ mb, $s_0=4m^2_{N}$, 
$A=50$\ MeV and $B=0.6$. 

\subsection{The interaction cross sections for 
meson-baryon collisions with quark annihilation.}

\hspace{1.0em}
The meson-baryon
interactions going through quark annihilation
 are dominated by the formation of an intermediate
resonance up to c.m. energies of $2.2$ \ GeV. 
 Thus the total meson-baryon cross section with quark annihilation 
 can be approximated as follows \cite{URQMD97}
\begin{equation}
\begin{array}{cc}
\label{CCS14} \sigma^{MB}_{tot}(\sqrt{s_{MB}}) = \\
=\sum_{R=\Delta ,N^{*}}<j_M,m_M,j_B, m_B||J_R,
M_R>\frac{2T_R+1}{(2T_M+1)(2T_B+1)} \times \\
\times \frac{\pi}{p^2_{c.m.}} 
\frac{\Gamma_{tot}\Gamma_{R\rightarrow MB}}
{(m_R - \sqrt{s_{MB}})^2 +1/4\Gamma^2_{tot}}
\end{array}
\end{equation}
with the total and partial $\sqrt{s}$-dependent decay widths
$\Gamma_{tot}$ and $\Gamma_{R\rightarrow MB}$. In Eq. ($\ref{CCS14}$)
$<j_M,m_M,j_B, m_B||J_R,M_R>$ denotes the Clebsch-Gordan coefficient
determined by meson $j_M$, baryon $j_B$ and resonance $J_R$ angular
momenta and their projections $m_M$, $m_B$ and $M_R$,
respectively. $T_M$, $T_B$ and $T_R$ are meson, baryon and resonance
isospins, respectively, and $p_{c.m.}$ (see Eq. ($\ref{SCD8}$) 
is c.m. momentum of incoming 
particles. Similarly, we able to calculate the meson-meson total cross 
section interaction going through quark annihilation \cite{URQMD97}.


\subsection{Nucleon-nucleon resonance interaction cross sections.}

\hspace{1.0em}
Up to incident beam energies of $4-5$ \ GeV/nucleon particle
production in nucleon-nucleon collisions is dominated by resonance decays.  
To obtain resonance excitation cross sections we 
 employ a practial approach of \cite{URQMD97}. 
In this approach the resonance excitation cross sections are
determined as
\begin{equation}
\label{CCS15} \sigma_{i,j\rightarrow k,l}(\sqrt{s_{i,j}}) =
(2S_k + 1)(2S_l
+1)\frac{p_{k,l}}{p_{i,j}}\frac{1}{s_{i,j}}|M(\sqrt{s_{i,j}},m_k,m_l)|^2,
\end{equation}
where $S_k$ and $S_l$ are spins of outgoing particles and 
$p_{k,l}$ and $p_{i,j}$ (see Eq. ($\ref{SCD8}$) 
are ingoing and outgoing c.m. momenta. The matrix
element $|M|^2$ is assumed to have no
spin-dependence with free parameters are tuned \cite{URQMD97} to 
experimental measurements. 

\subsection{The cross section of $\pi$-absorption by a quasi-deutron pair.}

\hspace{1.0em}
We consider two mechanisms of the 
pion absorption. For low energy the $s$-wave (non-resonant) 
absorption is dominated. 
The $s$-wave absorption is also important  
  for stopped negative pions.
 The two-body $s$-wave absorption cross section of pion with energy $E_{\pi}$
and momentum $p$ in the laboratory frame is derived from the optical
model \cite{Ericson88} as 
\begin{equation}
\label{CCS12} \sigma^A_{s}(E_{\pi}) = \frac{4\pi}{p}[1 + 
\frac{E_{\pi}}{2m} Im B_0(E_{\pi})]\rho(r),
\end{equation}
where $\rho(r)$ is nuclear density and $Im B_0=0.14$ fm$^4$. From Eq.
($\ref{CCS12}$) we can find the absorption mean free path $\lambda_{abs}(p,r) 
= 1/(\sigma^A_{s}(E_{\pi})\rho(r))$. We use it to sample pion absorption point 
$r$.
The $s$-wave (non-resonant)  pion absorption is determined 
 mainly by the the simplest cluster consisting of two nucleon 
$(np)$ or $(pp)$. 

To sample absorption event  
it is assumed that the charge state of two nucleons is determined by the number 
of pairs of a given type, e. g. the probability $P_{np}$ 
of absorption by $np$-pairs is
\begin{equation}
\label{CCS13} P_{np} = \frac{NZ}{NZ+Z(Z-1)/2}.
\end{equation}






