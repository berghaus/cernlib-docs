\section{Pauli blocking simulation.}

\hspace{1.0em}A free particle interaction 
cross section is reduced to an effective cross section
by the Pauli-blocking due to Fermi statistics. For each 
collision the phase-space densities $f_i$, where $i$ means fermion, in
the final states should be checked in order to assure that the final
distribution in phase space is in agreement with the Pauli principle,
which rules out the posibility of finding more than one fermion in a
single quantum state. There are two different Pauli blocking procedures: 
the cascade Pauli blocking procedure, which can be applied for final state 
nucleons in case of hadron--nucleus interaction and the Quantum Molecular 
Dynamics (QMD) Pauli blocking procedure, which can be applied for any 
final state fermions.

\subsection{The cascade Pauli blocking procedure.}

\hspace{1.0em}In this procedure, a  nucleus with atomic number $A$ and
charge $Z$ is treated as an ideal local completely degenerate Fermi gas
of nucleons with coordinates ${\bf r}$, momenta ${\bf p}$.  The nucleon
phase-space density is approximated by
\begin{equation}
\label{PBS1} f_{i}({\bf r}, {\bf p}) = \Theta 
({\bf p}_{i}^F({\bf r}) - {\bf p}).
\end{equation}
Because all states below Fermi-level are already occupyied, after each
interaction one should check that the momenta ${\bf p^{\prime}}_i$ of
all secondary nucleons are above the Fermi-level, i. e.
\begin{equation}
\label{PBS2} p^{\prime}_i > p_{i}^F(r).
\end{equation}
If among the secondary nucleons there is a nucleon with momentum lower
the the Fermi-level, then this collision is considered as prohibited
(Pauli-blocked).


\subsection{ The QMD Pauli blocking procedure.} 

\hspace{1.0em}We consider nucleons (and other fermions)
are not points in phase space. They are represented by Gaussian shaped
density distributions \cite{URQMD97}:
\begin{equation}
\label{PBS3}\phi({\bf x_i}, t) = (\frac{2\alpha}{\pi})^{3/4}
\exp{\{-\alpha ({\bf x_i}
- {\bf r_i} (t))^2 + \frac{i}{\hbar}{\bf p_i} (t) {\bf x_i}\}}, 
\end{equation}
where $\alpha=0.25$ \ fm$^{-2}$ is a model parameter and $\hbar =
197.327$ \ MeVfm is the conversion constant. .  The total wave function
is assumed to be a direct product of these functions.  The phase-space
density can be obtained by the Wigner transform of the wave function:
\begin{equation}
\label{PBS4} f({\bf r}, {\bf p})=\sum_{i}f_i({\bf r}, {\bf p}),
\end{equation}
where
\begin{equation}
\label{PBS5} f_i({\bf r}, {\bf p})= \frac{1}{(\pi\hbar)^3}\exp{\{-2\alpha ({\bf r}
- {\bf r}_i(t) )^2 - \frac{1}{2\alpha {\hbar}^2}({\bf p}-{\bf p}_i(t))^2 \}}
\end{equation} 
with normalization
\begin{equation}
\label{PBS6} \int d{\bf r}d{\bf p}f_i({\bf r}, {\bf p})=1.
\end{equation}
The normalised on the number of particles density is
\begin{equation}
\label{PBS7} \rho({\bf r})=\sum_{i}\rho_i({\bf r}),
\end{equation}
where
\begin{equation}
\label{PBS8} \rho_i({\bf r})=\int \frac{d{\bf p}}{(\pi\hbar)^3}f_i({\bf r},{\bf p})=
(\frac{\pi}{2\alpha})^{-3/2}\exp{\{-2\alpha ({\bf r}-{\bf r}_i)^2\}}.
\end{equation}

The normalised on the number of particles momentum density is
\begin{equation}
\label{PBS9} g({\bf p})=\sum_{i}g_i({\bf p}),
\end{equation}
where
\begin{equation}
\label{PBS10} g_i({\bf p})=\int \frac{d{\bf r}}{(\pi\hbar)^3}f_i({\bf r},{\bf p})=
\hbar^{-3}(2\pi\alpha )^{-3/2}\exp{\{-\frac{1}{2\alpha \hbar^2} ({\bf p}-{\bf p}_i)^2\}}.
\end{equation}

The overlap phase-space density $f^{ovp}_i$ and particle density
$\rho^{ovp}_i$ of particle $i$ with other particles are given by
\begin{equation}
\begin{array}{c}
\label{PBS11} f^{ovp}_i = \sum_{j\neq i}
\int d{\bf r}d{\bf p} f_i({\bf r},{\bf p})f_j({\bf r},{\bf p})= \\
= \frac{1}{8(\pi \hbar)^{3}}\sum_{j\neq i}
\exp{\{-\alpha ({\bf r}_i-{\bf r}_j)^2-\frac{1}{4\alpha \hbar^2}({\bf p}_i-{\bf p}_j)^2\}}
\end{array}
\end{equation}
and
\begin{equation}
\label{PBS12} \rho^{ovp}_i = \sum_{j\neq i}
\int d{\bf r} \rho_i({\bf r})\rho_j({\bf r})
= \sum_{j\neq i}(\frac{\pi}{\alpha})^{-3/2}
\exp{\{-\alpha ({\bf r}_i-{\bf r}_j)^2\}}.
\end{equation}

Thus the phase-space fermion overlaping densities $f^{ovp}_i$ at the
final states can be directly calculated and used for simulation of
Pauli-blocking.
For two indistinguishable nucleons $i$ and $j$ the  function
\begin{equation}
\begin{array}{c}
\label{PBS13}
F^{block}_{i}=\sum_{j\neq i} 8(\pi \hbar)^3 \delta_{\sigma_i\sigma_j}
\delta_{\tau_i\tau_j}
\int d{\bf r}d{\bf p} f_i({\bf r},{\bf p})f_j({\bf r},{\bf p})]= \\
=\delta_{\sigma_i\sigma_j}\delta_{\tau_i\tau_j}
\exp{\{-\alpha ({\bf r}_i-{\bf r}_j)^2-\frac{1}{4\alpha \hbar^2}
({\bf p}_i-{\bf p}_j)^2\}}
\end{array}
\end{equation}
can be interpreted as the Pauli-blocking probability.
  Here
$\sigma_{i,j}=\pm 1$ and $\tau_{i,j}=\pm 1$ denote spin and isospin
indices of nucleons, respectively.
For example the Pauli-blocking of the two-body collisions can be checked by the
blocking-probability $1- (1-F^{block}_i)(1-F^{block}_j)$.

\subsection{The QMD Pauli-blocking algorithm.}
\hspace{1.0em}For each produced baryon, which is located at position
${\bf r}$ and has momentum ${\bf p}$, the value of
$F^{block}_i$
and at the same time the value of 
\begin{equation}
\label{PBS14}
d_i=\sum_{j\neq i}\exp{\{-2\alpha ({\bf r}-{\bf r}_i)^2\}}
\end{equation}
are calculated.  As was found in \cite{Konopka96}, there is an
approximately straight line dependence:
\begin{equation}
\label{PBS15}
F^{block}_i = a_{fit} + b_{fit}d_i,
\end{equation}
where $a_{fit}=1.49641$ and $b_{fit}=0.208736$, which divides
$(F^{block}_i-d_i)$-plane into two the Pauli-blocked and the
Pauli-allowed domains.

Then a collision is only allowed, if computed values fulfill the
conditions:
\begin{equation}
\label{PBS16} F^{block}_i \leq a_{fit} + b_{fit}d_i
\end{equation}  
for each outgoing baryon $i$.
