\section{The intermediate energy $\gamma$-nucleon and $\gamma$-nucleus interactions.}

\hspace{1.0em}
We use the hadron kinetic model to perform simulation of $\gamma$-nucleus
inelastic collisions at intermediate energies. In this particular model
interaction of a particle with target nucleus is reduced to the interaction of 
a particle with intranuclear nucleons.
In the hadron kinetic model we realized several different mechanisms of the
$\gamma$-nucleon interaction.
 
The first one is the absorption of $\gamma$-quantum by a quasideutron pair
\begin{equation}
\label{IEGI1} \gamma + (np)\rightarrow np.
\end{equation}
The photoabsorption cross section calculated according to quasi-deutron model 
\cite{CS66}, \cite{BGIJT74} 
\begin{equation}
\label{IEGI2}\sigma_{A}=kZ(1-Z/A)\sigma_{D},
\end{equation}
where $\sigma_{D}$ is the cross section for deutron photodisintegration and
$A$, $Z$ are the mass and charge numbers of the nucleus in question,
respectively. The model parameter $k=10$ was taken.

The second one is the excitation of resonance and production of mesons:
\begin{equation}
\label{IEGI3} \gamma  N\rightarrow \Delta(1232),
 \gamma N \rightarrow N^{*}(1535) 
\end{equation}

At high energy the Regge approach is used to calculate $\gamma$-nucleon 
cross sections. The inelastic $\gamma$-nucleon and $\gamma$-nucleus
processes are simulated using the parton string
model.

At high energy the $\gamma$-nucleon interaction simulation is performed by the 
parton string model.
