\section{Nuclear fission cross section.}

\hspace{1.0em}The probability $P_{n}^{fis}$ that fission occurs at any
step of evaporation chain with $n$ evaporated fragments can be defined
as 
\begin{equation}
\label{FCS1}P_{n}^{fis} = 1-P_{n}, 
\end{equation} 
where $P_{n}$ is the probability of a transition from an excited state
to the ground state for the nucleus only by evaporation of $n$
fragments. The probability $P_{n}$ can be calculated using equation:
\begin{equation}
\label{FCS2}P_{n}=\prod_{i=1}^{n}[1-W_{fis}(U_i,A_i,Z_i)/W_{tot}(U_i,A_i,Z_i)],
\end{equation}
where $W_{fis}$ fission probability (per unit time) in the Bohr and
Wheeler theory of fission \cite{BW39}. It is assumed to be proportional
to the level density $\rho_{fis}(T)$ at the saddle point:
\begin{equation}
\label{FCS3}W_{fis}=\frac{1}{2\pi h \rho(U)}\int_{0}^{U-B_{fis}}
\rho_{fis}(U-B_{fis}-T)dT.
\end{equation}
In Eq. ($\ref{FCS3}$) $B_{fis}$ is the fission barrier height.  $W_{tot}$
is total decay probability (per unit time) of a nucleus:
\begin{equation}
\label{FCS4} W_{tot}=W_{fis}+\sum_{b=1}^{6}W_{b}
\end{equation}
and $W_{b}$ is the probability to evaporate fragment of type $b$.  In
the Weisskopf and Ewing theory of particle evaporation \cite{WE40}:
\begin{equation}
\label{FCS5}W_{b}=\frac{(2s_b + 1)m_b}{\pi^2 h^3 \rho(U)}
\int_{V_b}^{U-Q_b} 
\sigma_b(T_b)\rho(E^{*})T_bdT_b,
\end{equation}
where $V_b$ is the Coulomb barrier for fragment $b$ evaporation,
$\sigma_{b}(T_b)$ is the inverse (absorption of fragment $b$ with its kinetic 
energy $T_b$) reaction
cross section, $s_b$ and $m_b$ are fragment spin and mass respectively,
$Q_b$ is the fragment $b$ binding energy.
$\rho(E^{*})$ and $\rho(U)$ are level densities of nucleus after, when
excitation energy of nucleus will be $E^{*}=U-Q_b-T_b$, and before
fragment evaporation, respectively.  

Eq. ($\ref{FCS1}$) gives us a possibility to calculate numericaly (MC
-method) the so-called fissility of nucleus $P_{fis} =
\sigma_{fis}/\sigma_{in}$ (see e.g. \cite{ICC80}), where $\sigma_{in}$
is the inelastic nuclear reaction cross section and hence the fission
cross section $\sigma_{fis}$ E.g.
\begin{equation}
\label{FCS2} \sigma_{fis}=\sigma_{in}P_{fis}=\sigma_{in}\frac{1}{N_{ch}}
\sum_{n=1}^{N_{ch}}P^{fis}_{n},
\end{equation}
where $N_{ch}$ is the number of fragment evaporation chains, which is
used for averaging.
 
As one can see from Eq. ($\ref{FCS3}$) the fission barrier height
$B_{fis}$ and the parameter of the level density of a nucleus $a_{fis}$
at saddle point are the basic ingredients of model, which are necessary
for the calculation of fission cross section.
