\subsection {Pion absorption at rest}
The absorption of stopped negative pions in nuclei is interpreted 
(references...) as starting with the absorption of the pion by two or
more correlated nucleons; the total energy of the pion is transferred to
the absorbing nucleons, which then may leave the nucleus directly, or
undergo final-state interactions with the residual nucleus. The remaining
nucleus de-excites by evaporation of low energetic particles.

G4PiMinusAbsorptionAtRest generates the primary absorption component of 
the process through
the parameterisation of existing experimental data;
the primary absorption component is handled by class G4PiMinusStopAbsorption.
In the current implementation only absorption on a nucleon pair is considered,
while contributions from absorption on nucleon clusters are neglected; 
this approximation is supported by experimental 
results (references...), showing that it is the dominating contribution.

Several features of stopped pion absorption are known from experimental
measurements on various materials:
\begin{itemize}
\item the average number of nucleons emitted, as resulting from the
primary absorption process;
\item the ratio of nn vs np as nucleon pairs involved in the absorption
process;
\item the energy spectrum of the resulting nucleons emitted and their
opening angle distribution.
\end{itemize}
The corresponding final state products and related distributions are
generated according to a parameterisation
of the available experimental measurements listed above. The dependence on
the material is handled by a strategy (reference...) pattern:
the features pertaining to material for which experimental data are available 
are treated in G4PiMinusStopX classes (where X represents an element),
inheriting from G4StopMaterial base class. In case of absorption on an element
for which experimental data are not available, the experimental distributions
for the elements closest in Z are used. 

The excitation energy of the residual nucleus is calculated by difference
between the initial energy and the energy of the final state products of
the primary absorption process.

Another strategy handles the nucleus deexcitation; the current default
implementation consists in handling the deexcitatoin component of the process
through the evaporation model described in (reference...).



%%% Local Variables: 
%%% mode: latex
%%% TeX-master: t
%%% End: 
