%%%%%%%%%%%%%%%%%%%%%%%%%%%%%%%%%%%%%%%%%%%%%%%%%%%%%%%%%%%%%%%%%%%%%
% Elementary Particle Naming Scheme                                 %
% Part of paper ``Scientific Text Processing''                      %
%   To be published in International Journal of Modern Physics      %
%   Part C, Physics and Computers                                   %
% Michel Goossens and Eric van Herwijnen                            %
% Latex version 1.06                                                %
% Last Mod  1 Feb 1995 (mg)                                         %
%%%%%%%%%%%%%%%%%%%%%%%%%%%%%%%%%%%%%%%%%%%%%%%%%%%%%%%%%%%%%%%%%%%%%
\documentclass{article}
\usepackage{longtable,array,pennames,times}
\usepackage{penmath}% special version of psmath with upright small Greek
\topmargin 0 pt    %   Nominal distance from top of paper to top of page
\textheight 53\baselineskip
\advance\textheight by \topskip
\usepackage{html}
\oddsidemargin0mm
\evensidemargin0mm
\textwidth16cm
\setlongtables
%\driver{dvips}
%\setlength{\changebarsep}{1mm}
%\LTchunksize=10
\newcommand{\bs}{{\ttfamily\char'134}}% backslash
\begin{document}
\title{The Elementary Particle Entity Notation (PEN) Scheme}
\author{Michel Goossens and Eric van Herwijnen\\
\it CERN, CH-1211 Geneva 23, Switzerland}
\maketitle
\begin{abstract}
In this article an Elementary Particle Entity Notation (PEN)
scheme is
proposed for use with \TeX{} and SGML. This scheme not only assures
the typographic correctness of the printed symbols, but also
eases the automatic extraction of information about the
article by the recognition of the entity names.
\par\vspace{\baselineskip}
{\it Keywords\/}:
Text-processing; SGML; \TeX; Elementary particles; PEN
\end{abstract}
 
\section[]{Typographical rules for scientific texts}
 
In scientific texts the printed  form of a symbol
often implies a meaning which is not easily captured by generic markup.
Therefore authors using some form
of generic coding (like \LaTeX{} or SGML) need to know about
typographical conventions.
The following is a brief summary of the most important
rules for composing scientific texts\cite{IUPAP,LOWE}.
 
\begin{enumerate}
\item The most important rule is
\emph{consistency}: a symbol should always be
the same, whether it appears in a formula or in the text, on the
main line or as a superscript or
subscript.
I.e. in \TeX, once you have used a symbol inside mathematics
mode ('\$'), always use it inside mathematics mode.
Inside math mode, \TeX\ by default prints characters in {\it italics}.
\par
For scientific work, however, quite a few symbols must be set in
\emph{roman} (upright) characters\footnote{With \LaTeX{} roman type in maths
mode can be achieved by the {\tt\bs mathrm} command.}.
This is the case for the following families of symbols, which represent
the names of:
\begin{itemize}
\item units, such as g, cm, s, keV.
Note that physical constants are usually in italics,
so units involving constants
are mixed roman-italics, e.g.
GeV/{\it c} (where the {\em c} is italic
because it symbolizes the speed
of light, a constant);
\item particles, for example p, K, q, H. For elementary particles
the PEN (Particle Entity Notation)
scheme is proposed (see the next section);
\item standard mathematical functions (sin, det, cos, tan, Re, Im, etc.).
Use the built-in \LaTeX\ functions for these (\verb+\sin+ etc.);
\item chemical elements, for example Ne, O, Cu;
\item numbers;
\item names of waves or states (p-wave) and covariant couplings
(A for axial, V for vector), names of monopoles (E for electric,
M for magnetic);
\item abbreviations that are initials of bits of words
(exp, for experimental; min, for minimum);
\item the 'd' in integrands (e.g. d{\it p}).
 
\end{itemize}
 
In all cases, following these rules will help the reader
understand at first glance what one is talking about.
Some instances in which it is important to use the correct symbol,
in the correct type, are shown in the table below:
 
\begin{center}
\begin{tabular}{|lp{45mm}|lp{45mm}|}
\hline
\multicolumn{2}{|c}{\itshape roman type } &
\multicolumn{2}{|c|}{\itshape italic type} \\
\hline
A & ampere (electric unit)      & $A$ & atomic number (variable)       \\
e & electron (particle name)    & $e$ & electron charge (constant)     \\
g & gluon (particle name)       & $g$ & gravitational constant         \\
l & litre (volume unit)         & $l$ & length (variable)              \\
m & metre (length unit)         & $m$ & mass (variable)                \\
p & proton (particle name)      & $p$ & momentum (variable)            \\
q & quark (particle name)       & $q$ & electric charge (variable)     \\
s & second (time unit)          & $s$ & c.m. energy squared (variable) \\
t & tonne (weight unit)         & $t$ & time (variable)                \\
V & volt (electric unit)        & $V$ & volume (variable)              \\
Z & Z boson (particle name)     & $Z$ & atomic charge (variable)       \\
\hline
\end{tabular}
\end{center}
 
\item Let your word processor do as much work as it can. Do not try to
change your system's defaults too much; this will decrease the
portability and maintainability of your documents.
\TeX\ implements part of the rules mentioned above
by default in math mode.
\item Do not add blanks at random to make formulae look ``nicer'`.
\item Restrain from using specific page layout commands (like
\verb+\break+ with \TeX).
You will forget that you put them in your text and later wonder
why some text is badly adjusted or starts a new line.
 
\end{enumerate}
 
 
\section[]{Entity definitions for elementary particles}
 
In texts on high energy physics frequently
re-occurring strings are the names of elementary particles.
For example, the $\mbox{Z}^{0}$
particle can be coded in various
different ways with \LaTeX: \verb+$\mbox{Z}^0$+,
\verb+$\mathrm{Z^0}$+ and  \verb+Z$^0$+
all achieve the same typographical effect, a roman
Z with a superscript 0. In the interest of standardization and
typing convenience, we propose below an ``entity'' naming scheme,
which will not only relieve the user from having to worry about the
correctness of what he types, but also will allow an automatic extraction
of the particle names from the input file, so that it will be easy to enter
data about an article using this convention into a database of abstracts.
 
The naming scheme uses a notation which takes the
following constraints into consideration:
\begin{enumerate}
\item The notation should be able to describe
all particles in the particle data summary tables from the
``Review of Particle Properties''\cite{PPD} and
any future extension to these.
\item The names should not exceed
eight characters. This is the maximum length for entities in the
SGML reference concrete syntax\cite{ERIC1}. Staying within this
limit means that the notation can be used with most SGML applications.
\item Common particles such as protons and electrons
should have short and simple names.
\item Items that are indicated by superscripts are indicated before
items that are indicated by subscripts.
 
\end{enumerate}
 
Due to the eight character limitation the mass could not be added to
the name. This means that in general an entity on its own is not
adequate to unambiguously identify a particle, c.f.
$\eta $(549) and $\eta $(1300) are both referred to as \texttt{Pgh}.
Including mass dependences into the names is not a good idea anyway,
since the mass can change with time
when more precise measurements become available.
The ambiguity was solved by adding a letter to the end
of the name where a mass appears in the name in the particle data
summary tables. Thus $\eta $(549) is referred to as \texttt{Pgh}
while $\eta $(1300) is referred to as \texttt{Pgha}. Higher letters
correspond to higher masses, in the order given in the tables.
 
The PEN scheme is independent of any text processing system.
We have implemented it in \TeX\ (in such a way that it may
be used in all macro packages, e.g. \LaTeX) and SGML.
The \TeX\ implementation will print particle masses, which will
be regularly updated according to the Review of Particle Properties
publication.
It is constructed so that the PEN name can be used both in
mathematics and text mode.
 
 
\subsection[]{Principles of the Particle Entity Notation (PEN)}
 
Starting at the left, a name is built from the following characters:
 
\begin{enumerate}
\item Start the entity with a recognized string (in the following
this was chosen as \texttt{uppercase P}). This is necessary to
uniquely identify entities as following the PEN convention.
\item The following letters act as an escape to signal a special
interpretation of the string. Present escape sequences are:
\begin{itemize}
\item \texttt{a} for anti particle (normally visually
represented with a
bar over the particle's name)
\item \texttt{b} for bottom particle
\item \texttt{c} for charmed particle
\item \texttt{g} for indicating the subsequent letter is Greek.
The correspondence between Latin and Greek letters is based on the
notation for mathematical Greek characters used by the AAP mathematical
formula application\cite{AAP}:
\begin{verbatim}
<!NOTATION greek2 PUBLIC "+//ISBN 1-880124::NISO//NOTATION GREEK-2//EN">
\end{verbatim}
This one-letter correspondence is
as follows:
\tabcolsep 3pt
\begin{center}
\begin{tabular}{*{4}{|cl>{\ttfamily}c}|}
\hline
\bf Greek & \bf name & \bf code & \bf Greek & \bf name & \bf code  &
\bf Greek & \bf name & \bf code & \bf Greek & \bf name & \bf code  \\
\hline
$\alpha$  & alpha    & a    & A         & Alpha    & A     &
$\beta$   & beta     & b    & B         & Beta     & B     \\
$\gamma$  & gamma    & g    & $\Gamma$  & Gamma    & G     &
$\delta$  & delta    & d    & $\Delta$  & Delta    & D     \\
$\epsilon$& epsilon  & e    & E         & Epsilon  & E     &
$\zeta$   & zeta     & z    & Z         & Zeta     & Z     \\
$\eta$    & eta      & h    & H         & Eta      & H     &
$\theta$  & theta    & q    & $\Theta$  & Theta    & Q     \\
$\iota$   & iota     & i    & I         & Iota     & I     &
$\kappa$  & kappa    & k    & K         & Kappa    & K     \\
$\lambda$ & lambda   & l    & $\Lambda$ & Lambda   & L     &
$\mu$     & mu       & m    & $M$       & Mu       & M     \\
$\nu$     & nu       & n    & N         & Nu       & N     &
$\xi$     & xi       & x    & $\Xi$     & Xi       & X     \\
o         & omicron  & o    & O         & Omicron  & O     &
$\pi$     & pi       & p    & $\Pi$     & Pi       & P     \\
$\rho$    & rho      & r    & R         & Rho      & R     &
$\sigma$  & sigma    & s    & $\Sigma$  & Sigma    & S     \\
$\tau$    & tau      & t    & T         & Tau      & T     &
$\upsilon$& upsilon  & u    & $\Upsilon$& Upsilon  & U     \\
$\phi$    & phi      & f    & $\Phi$    & Phi      & F     &
$\chi$    & chi      & c    & X         & Chi      & C     \\
$\psi$    & psi      & y    & $\Psi$    & Psi      & Y     &
$\omega$  & omega    & w    & $\Omega$  & Omega    & W     \\
\hline
\end{tabular}
\end{center}
\item \texttt{q} for quark particle
\item \texttt{s} for strange particle
\item \texttt{S} for supersymmetric particle
\item \texttt{t} for top particle
\end{itemize}
 
\item The one-letter name of the particle
\item Optionally followed by other information
\begin{itemize}
\item \texttt{z} for zero, \texttt{i} for one, \texttt{ii} for two,
\texttt{iii} for three, \texttt{iv} for four
\item \texttt{m} for minus, \texttt{p} for plus, \texttt{pm} for plus/minus
\item \texttt{pr} for prime
\item \texttt{st} for asterisk (star)
\item \texttt{L} for left handed, \texttt{R} for right handed
\item any one-letter particle name
\end{itemize}
\end{enumerate}
 
\subsection[]{Particle encodings according to the PEN Scheme}
 
In table  \ref{TPNSEXA} we show how to encode the particles from
the summary tables of particle
properties in the ``Review of Particle Properties''\cite{PPD}
using the PEN convention.
In the rightmost column we give the computer name of the particle,
as defined by ``A guide to Experimental Elementary Particle
Physics Literature (1985-1989)''\cite{EPPLIT}. This
is the name to be used when
searching the Particle Data group's databases. Notice that these
names cannot be used either for \TeX\ or SGML, as they
do not satisfy the constraints of the PEN scheme as defined above.
When a name is marked as ``\texttt{not available}'', sometimes a charged
or neutral version exists (not given in the table).
 
The \TeX\ implementation is available as a style file
\texttt{pennames.sty} which should be input in the usual way
at the start of the document for \TeX{} or specified as a
minor option on the {\tt\bs documentstyle} command for \LaTeX.
To obtain the symbol
required, prefix the PEN name by a backslash (``\bs'').
 
The SGML implementation exists as a public entity set, that can
be included in SGML documents with the following entity definition:
\begin{verbatim}
<!ENTITY % PEN PUBLIC
   "+//ISBN 92-9083-041-7::CERN//ENTITIES Particle Entity Names//EN">
\end{verbatim}
Refer to a particle entity by prefixing its name by an ampersand (``\&'')
and suffixing it with a semi-colon (``;'').
 
\begin{thebibliography}{1}
 
\bibitem{IUPAP}
International Union of pure and applied Physics.
\newblock {\em Symbols, Units, Nomenclature and fundamental Constants
in Physics}.
\newblock Physica, 146A:1--67, 1987.
\bibitem{LOWE}
D.E. Lowe.
\newblock {\em A Guide to international recommendations on names and symbols
  for quantities and on units of measurements}.
\newblock World Health Organization, Geneva, 1975.
\bibitem{PPD}
Particle~Data Group.
\newblock {\em Review of particle properties.}
\newblock Physics Review D, 50, Part 1, pages 1173-1826, August 1994.
\bibitem{ERIC1}
E.~van~Herwijnen.
\newblock {\em Practical SGML}.
\newblock Wolters-Kluwer Academic Publishers, Boston, 1990.
\bibitem{AAP}
American National Standards Institute.
\newblock {\em American National Standard for Electronic Manuscript Preparation
  and Markup}
\newblock ANSI/NISO Z39.59-1988, 1988.
\bibitem{EPPLIT}
Particle Data Group.
\newblock {\em A Guide to Experimental Elementary Particle Physics Literature
  (1985-1989)}.
\newblock Lawrence Berkeley Laboratory, LBL-90 Revised, UC-414, November 1990.
 
\end{thebibliography}
 
\small\noindent
\framebox[\textwidth][t]{\hfill\begin{minipage}{0.92\textwidth}%
\vspace*{2mm}
%\parskip\baselineskip
The files \texttt{pennames.sty}, containing the definitions
for the particle names with \TeX{},
\texttt{pennames.entities}, the SGML entity definitions
and \texttt{pennames.ps}, the PostScript source
of this document can be obtained via \texttt{anonymous} ftp as follows
(commands to be typed by the user are underlined):
 
\begin{tabular}{@{\hspace{2cm}}>{\tt}l}
\underline{ftp asisftp.cern.ch}\\
Trying 128.141.8.104...\\
Connected to asis01.cern.ch.\\
220 asis01 FTP server (SunOS 4.1) ready.\\
Name (asis01:username): \underline{anonymous}\\
Password: \underline{username.node}\\
ftp> \underline{cd cernlib/doc/tex.dir/pennames}\\
ftp> \underline{get pennames.sty}\\
ftp> \underline{get pennames.entities}\\
ftp> \underline{get pennames.ps}\\
ftp> \underline{quit}\\
\end{tabular}
 
Please send comments or suggestions to \texttt{goossens@cern.ch}.
\vspace*{2mm}
\end{minipage}\hfill}%end of minipage in framebox
 
\begin{latexonly}
\newpage
%\renewcommand{\arraystretch}{0.960}
\begin{longtable}{|>{\ttfamily}l|l|p{.35\linewidth}|>{\ttfamily}l|}
\caption[]{PEN names for elementary particles in PDG list}
\label{TPNSEXA}
\endfirsthead
\caption[]{PEN names (\emph{continued})}                      \\ \hline
\multicolumn{1}{|c|}{\bf PEN}                             &
\multicolumn{1}{c|}{\bf symbol}                           &
\multicolumn{1}{c|}{\bf conventional name}                &
\multicolumn{1}{c|}{\bf computer name}                        \\ \hline
\endhead
\hline
\endfoot
\hline
\multicolumn{1}{|c|}{\bf PEN}                             &
\multicolumn{1}{c|}{\bf symbol}                           &
\multicolumn{1}{c|}{\bf conventional name}                &
\multicolumn{1}{c|}{\bf computer name}                        \\ \hline
\multicolumn{4}{|c|}{\bf\boldmath Gauge and Higgs bosons}             \\ \hline
Pgg      & \Pgg    & gamma                    & GAMMA          \\
PW       & \PW     & W boson                  & W              \\
PWp      & \PWp    & W plus                   & W+             \\
PWm      & \PWm    & W minus                  & W-             \\
PZz      & \PZz    & Z zero                   & Z              \\
PHz      & \PHz    & Higgs zero               & not available  \\
PHpm     & \PHpm   & Higgs plus/minus         & HIGGS+-        \\
PWR      & \PWR    & right-handed W           & not available  \\
PWpr     & \PWpr   & W prime                  & WPRIME         \\
PZLR     & \PZLR   & left-right handed Z      & not available  \\
PZgc     & \PZgc   & Z chi                    & not available  \\
PZgy     & \PZgy   & Z psi                    & not available  \\
PZge     & \PZge   & Z eta                    & not available  \\
PZi      & \PZi    & Z one                    & not available  \\
PAz      & \PAz    & axion                    & AXION          \\ \hline
\multicolumn{4}{|c|}{\bf\boldmath Leptons}                            \\ \hline
Pgne     & \Pgne   & electron neutrino        & NUE            \\
Pagne    & \Pagne  & anti electron neutrino   & NUEBAR         \\
Pgngm    & \Pgngm  & muon neutrino            & NUMU           \\
Pagngm   & \Pagngm & anti muon neutrino       & NUMUBAR        \\
Pgngt    & \Pgngt  & tau neutrino             & NUTAU          \\
Pagngt   & \Pagngt & anti tau neutrino        & NUTAUBAR       \\
Pe       & \Pe     & electron                 & not available  \\
Pep      & \Pep    & positron                 & E+             \\
Pem      & \Pem    & e minus                  & E-             \\
Pgm      & \Pgm    & muon                     & not available  \\
Pgmm     & \Pgmm   & mu minus                 & MU-            \\
Pgmp     & \Pgmp   & mu plus                  & MU+            \\
Pgt      & \Pgt    & tau                      & not available  \\
PLpm     & \PLpm   & charged lepton           & LEPTON+-       \\
PLz      & \PLz    & stable neutral heavy lepton   & not available\\
PEz      & \PEz    & neutral para- or ortho-lepton & not available\\ \hline
\multicolumn{4}{|c|}{\bf\boldmath Light Unflavored Mesons (S=C=B=0)}  \\ \hline
Pgp      & \Pgp    & pion                     & PI             \\
Pgpm     & \Pgpm   & pi minus                 & PI-            \\
Pgpp     & \Pgpp   & pi plus                  & PI+            \\
Pgppm    & \Pgppm  & pi plus/minus            & PI+-           \\
Pgpz     & \Pgpz   & pi zero                  & PI0            \\
Pgh      & \Pgh    & eta                      & ETA            \\
Pgr      & \Pgr    & rho                      & RHO(770)       \\
Pgo      & \Pgo    & omega                    & OMEGA(783)     \\
Pghpr    & \Pghpr  & eta prime                & ETAPRIME(958)  \\
Pfz      & \Pfz    & f zero                   & F0(975)        \\
Paz      & \Paz    & a zero                   & A0(980)        \\
Pgf      & \Pgf    & phi                      & PHI(1020)      \\
Phia     & \Phia   & h one                    & H1(1170)       \\
Pbi      & \Pbi    & b one                    & not available  \\
Pai      & \Pai    & a one                    & A1(1260)       \\
Pfii     & \Pfii   & f two                    & F2(1270)       \\
Pfi      & \Pfi    & f one                    & F1(1285)       \\
Pgha     & \Pgha   & eta 1295                 & ETA(1295)      \\
Pgpa     & \Pgpa   & pion 1300                & not available  \\
Paii     & \Paii   & a two                    & A2(1320)       \\
Pgoa     & \Pgoa   & omega 1390               & not available  \\
Pfza     & \Pfza   & f zero 1400              & F0(1400)       \\
Pfia     & \Pfia   & f one 1420               & F1(1420)       \\
Pghb     & \Pghb   & eta 1440                 & ETA(1440)      \\
Pgra     & \Pgra   & rho 1450                 & not available  \\
Pfib     & \Pfib   & f one 1510               & F1(1510)       \\
Pfiipr   & \Pfiipr & f two prime              & F2PRIME(1525)  \\
Pfzb     & \Pfzb   & f zero 1590              & F0(1590)       \\
Pgob     & \Pgob   & omega 1600               & not available  \\
Pgoiii   & \Pgoiii & omega three              & OMEGA3(1670)   \\
Pgpii    & \Pgpii  & pi two                   & PI2(1670)      \\
Pgfa     & \Pgfa   & phi 1680                 & PHI(1680)      \\
Pgriii   & \Pgriii & rho three                & not available  \\
Pgrb     & \Pgrb   & rho 1700                 & RHO(1700)      \\
Pfiia    & \Pfiia  & f two 1720               & F2(1720)       \\
Pgfiii   & \Pgfiii & phi three                & PHI3(1850)     \\
Pfiib    & \Pfiib  & f two 2010               & F2(2010)       \\
Pfiv     & \Pfiv   & f four                   & F4(2050)       \\
Pfiic    & \Pfiic  & f two 2300               & F2(2300)       \\
Pfiid    & \Pfiid  & f two 2340               & F2(2340)       \\ \hline
\multicolumn{4}{|c|}{\bf\boldmath Strange Mesons (S=$\pm1$, C=B=0)} \\ \hline
PK       & \PK     & kaon                     & K              \\
PKpm     & \PKpm   & K plus minus             & K+-            \\
PKp      & \PKp    & K plus                   & K+             \\
PKm      & \PKm    & K minus                  & K-             \\
PKz      & \PKz    & K zero                   & K0             \\
PaKz     & \PaKz   & anti K-zero              & KBAR0          \\
PKgmiii  & \PKgmiii  & K mu three             & not available  \\
PKeiii   & \PKeiii   & K e three              & not available  \\
PKzS     & \PKzS     & K zero short           & not available  \\
PKzL     & \PKzL     & K zero long            & not available  \\
PKzgmiii & \PKzgmiii & K zero mu three        & not available  \\
PKzeiii  & \PKzeiii  & K zero e three         & not available  \\
PKst     & \PKst     & K star                 & not available  \\
PKi      & \PKi      & K one                  & K1(1270)       \\
PKsta    & \PKsta    & K star (1370)          & not available  \\
PKia     & \PKia     & K one (1400)           & not available  \\
PKstz    & \PKstz    & K star zero (1430)     & not available  \\
PKstii   & \PKstii   & K star two (1430)      & not available  \\
PKstb    & \PKstb    & K star (1680)          & not available  \\
PKii     & \PKii     & K two (1770)           & not available  \\
PKstiii  & \PKstiii  & K star three           & not available  \\
PKstiv   & \PKstiv   & K star four            & not available  \\ \hline
\multicolumn{4}{|c|}{\bf\boldmath Charmed Mesons (C=$\pm1$)}          \\ \hline
PD       & \PD       & D                      & D              \\
PaD      & \PaD      & anti D                 & DBAR           \\
PDpm     & \PDpm     & D plus/minus           & D+-            \\
PDm      & \PDm      & D minus                & D-             \\
PDp      & \PDp      & D plus                 & D+             \\
PDz      & \PDz      & D zero                 & D0             \\
PaDz     & \PaDz     & anti D zero            & DBAR0          \\
PDstpm   & \PDstpm   & D star plus/minus      & D*(2010)+-     \\
PDstz    & \PDstz    & D star zero            & D*(2010)0      \\
PDiz     & \PDiz     & D one zero             & D1(2420)0      \\
PDstiiz  & \PDstiiz  & D star two zero        & D2*(2460)0     \\
\newpage
\multicolumn{4}{|c|}{\bf\boldmath Charmed Strange Mesons (C=S=$\pm1$)}\\ \hline
PsDp     & \PsDp     & D s plus               & D/S+           \\
PsDm     & \PsDm     & D s minus              & D/S-           \\
PsDst    & \PsDst    & D s star               & D/S*           \\
PsDipm   & \PsDipm   & D s one plus/minus     & not available  \\ \hline
\multicolumn{4}{|c|}{\bf\boldmath Bottom Mesons (B=$\pm1$)}           \\ \hline
PB       & \PB       & B                      & B              \\
PBp      & \PBp      & B plus                 & B+             \\
PBm      & \PBm      & B minus                & B-             \\
PBpm     & \PBpm     & B plus/minus           & B+-            \\
PBz      & \PBz      & B zero                 & B0             \\
PaB      & \PaB      & anti B                 & BBAR           \\
PaBz     & \PaBz     & anti B zero            & BBAR0          \\
Pcgh     & \Pcgh     & eta c                  & ETA/C(1S)      \\
PJgy     & \PJgy     & J psi                  & J/PSI(1S)      \\
Pcgcz    & \Pcgcz    & chi c zero             & CHI/C0(1P)     \\
Pcgci    & \Pcgci    & chi c one              & CHI/C1(1P)     \\
Pcgcii   & \Pcgcii   & chi c two              & CHI/C2(1P)     \\
Pgy      & \Pgy      & psi                    & PSI(2S)        \\
Pgya     & \Pgya     & psi 3770               & PSI(3770)      \\
Pgyb     & \Pgyb     & psi 4040               & PSI(4040)      \\
Pgyc     & \Pgyc     & psi 4160               & PSI(4160)      \\
Pgyd     & \Pgyd     & psi 4415               & PSI(4415)      \\
PgU      & \PgU      & Upsilon                & not available  \\
Pbgcz    & \Pbgcz    & chi b zero             & CHI/B0(1P)     \\
Pbgci    & \Pbgci    & chi b one              & CHI/B1(1P)     \\
Pbgcii   & \Pbgcii   & chi b two              & CHI/B2(1P)     \\
PgUa     & \PgUa     & Upsilon (2S)           & UPSI(2S)       \\
Pbgcza   & \Pbgcza   & chi b zero (2P)        & CHI/B0(2P)     \\
Pbgcia   & \Pbgcia   & chi b one (2P)         & CHI/B1(2P)     \\
Pbgciia  & \Pbgciia  & chi b two (2P)         & CHI/B2(2P)     \\
PgUb     & \PgUb     & Upsilon (3S)           & UPSI(3S)       \\
PgUc     & \PgUc     & Upsilon (4S)           & UPSI(4S)       \\
PgUd     & \PgUd     & Upsilon (10860)        & UPSI(10860)    \\
PgUe     & \PgUe     & Upsilon (11020)        & UPSI(11020)    \\ \hline
\multicolumn{4}{|c|}{\bf\boldmath quarks}                      \\ \hline
Pq       & \Pq       & quark                  & QUARK          \\
Paq      & \Paq      & anti-quark             & QUARKBAR       \\
Pqd      & \Pqd      & down quark             & DQ             \\
Paqd     & \Paqd     & anti down quark        & DQBAR          \\
Pqu      & \Pqu      & up   quark             & UQ             \\
Paqu     & \Paqu     & anti up   quark        & UQBAR          \\
Pqs      & \Pqs      & strange quark          & SQ             \\
Paqs     & \Paqs     & anti strange quark     & SQBAR          \\
Pqc      & \Pqc      & charmed quark          & CQ             \\
Paqc     & \Paqc     & anti charmed quark     & CQBAR          \\
Pqb      & \Pqb      & bottom  quark          & BQ             \\
Paqb     & \Paqb     & anti bottom  quark     & BQBAR          \\
Pqt      & \Pqt      & top     quark          & TQ             \\
Paqt     & \Paqt     & anti top     quark     & TQBAR          \\ \hline
\multicolumn{4}{|c|}{\bf\boldmath N Baryons (S=0, I=1/2)}      \\ \hline
Pp       & \Pp       & proton                 & P              \\
Pap      & \Pap      & anti-proton            & PBAR           \\
Pn       & \Pn       & neutron                & N              \\
PNa      & \PNa      & N (1440) P 11          & N(1440P11)     \\
PNb      & \PNb      & N (1520) D 13          & not available  \\
PNc      & \PNc      & N (1535) S 11          & not available  \\
PNd      & \PNd      & N (1650) S 11          & not available  \\
PNe      & \PNe      & N (1675) D 15          & not available  \\
PNf      & \PNf      & N (1680) F 15          & not available  \\
PNg      & \PNg      & N (1700) D 13          & not available  \\
PNh      & \PNh      & N (1710) P 11          & not available  \\
PNi      & \PNi      & N (1720) P 13          & not available  \\
PNj      & \PNj      & N (2190) G 17          & not available  \\
PNk      & \PNk      & N (2220) H 19          & not available  \\
PNl      & \PNl      & N (2250) G 19          & not available  \\
PNm      & \PNm      & N (2600) I 1,11        & not available  \\ \hline
\multicolumn{4}{|c|}{\bf\boldmath $\Delta$ Baryons (S=0, I=3/2)}\\ \hline
PgDa     & \PgDa     & Delta (1232) P 33      & DELTA(1232P33) \\
PgDb     & \PgDb     & Delta (1620) S 31      & not available  \\
PgDc     & \PgDc     & Delta (1700) D 33      & not available  \\
PgDd     & \PgDd     & Delta (1900) S 31      & not available  \\
PgDe     & \PgDe     & Delta (1905) F 35      & not available  \\
PgDf     & \PgDf     & Delta (1910) P 31      & not available  \\
PgDh     & \PgDh     & Delta (1920) P 33      & not available  \\
PgDi     & \PgDi     & Delta (1930) D 35      & not available  \\
PgDj     & \PgDj     & Delta (1950) F 37      & not available  \\
PgDk     & \PgDk     & Delta (2420) H 3,11    & not available  \\ \hline
\multicolumn{4}{|c|}{\bf\boldmath $\Lambda$ Baryons (S=$-1$, I=0)}\\ \hline
PgL      & \PgL      & Lambda                 & LAMBDA         \\
PagL     & \PagL     & anti Lambda            & LAMBDABAR      \\
PgLa     & \PgLa     & Lambda (1405) S 01     & LAMBDA(1405S01)\\
PgLb     & \PgLb     & Lambda (1520) D 03     & LAMBDA(1520D03)\\
PgLc     & \PgLc     & Lambda (1600) P 01     & not available  \\
PgLd     & \PgLd     & Lambda (1670) S 01     & not available  \\
PgLe     & \PgLe     & Lambda (1690) D 03     & not available  \\
PgLf     & \PgLf     & Lambda (1800) S 01     & not available  \\
PgLg     & \PgLg     & Lambda (1810) P 01     & not available  \\
PgLh     & \PgLh     & Lambda (1820) F 05     & not available  \\
PgLi     & \PgLi     & Lambda (1830) D 05     & not available  \\
PgLj     & \PgLj     & Lambda (1890) P 03     & not available  \\
PgLk     & \PgLk     & Lambda (2100) G 07     & not available  \\
PgLl     & \PgLl     & Lambda (2110) F 05     & not available  \\
PgLm     & \PgLm     & Lambda (2350) H 09     & not available  \\ \hline
\multicolumn{4}{|c|}{\bf\boldmath $\Sigma$ Baryons (S=$-1$, I=1)}\\ \hline
PgSm     & \PgSm     & Sigma minus            & SIGMA-         \\
PgSp     & \PgSp     & Sigma plus             & SIGMA+         \\
PagSm    & \PagSm    & anti Sigma minus       & SIGMABAR-      \\
PagSp    & \PagSp    & anti Sigma plus        & SIGMABAR+      \\
PgSz     & \PgSz     & Sigma zero             & SIGMA0         \\
PagSz    & \PagSz    & anti Sigma zero        & SIGMABAR0      \\
PgSa     & \PgSa     & Sigma (1385) P 13      & not available  \\
PgSb     & \PgSb     & Sigma (1660) P 11      & not available  \\
PgSc     & \PgSc     & Sigma (1670) D 13      & not available  \\
PgSd     & \PgSd     & Sigma (1750) S 11      & not available  \\
PgSe     & \PgSe     & Sigma (1775) D 15      & not available  \\
PgSf     & \PgSf     & Sigma (1915) F 15      & not available  \\
PgSg     & \PgSg     & Sigma (1940) D 13      & not available  \\
PgSh     & \PgSh     & Sigma (2030) F 17      & not available  \\
PgSi     & \PgSi     & Sigma (2250)           & not available  \\ 
\newpage
\multicolumn{4}{|c|}{\bf\boldmath $\Xi$ Baryons (S=$-2$, I=1/2)}      \\ \hline
PgXz     & \PgXz     & Xi zero                & XI0            \\
PagXz    & \PagXz    & anti Xi zero           & XIBAR0         \\
PgXm     & \PgXm     & Xi minus               & XI-            \\
PagXp    & \PagXp    & anti Xi plus           & XIBAR+         \\
PgXa     & \PgXa     & Xi (1530) P 13         & not available  \\
PgXb     & \PgXb     & Xi (1690)              & not available  \\
PgXc     & \PgXc     & Xi (1820) D 13         & not available  \\
PgXd     & \PgXd     & Xi (1950)              & not available  \\
PgXe     & \PgXe     & Xi (2030)              & not available  \\ \hline
\multicolumn{4}{|c|}{\bf\boldmath $\Omega$ Baryons (S=$-3$, I=0)}     \\ \hline
PgOm     & \PgOm     & Omega minus            & OMEGA-         \\
PagOp    & \PagOp    & anti Omega plus        & OMEGABAR+      \\
PgOma    & \PgOma    & Omega (2250) minus     & OMEGA(2250)-   \\ \hline
\multicolumn{4}{|c|}{\bf\boldmath Charmed Baryons (C=$+1$)}           \\ \hline
PcgLp    & \PcgLp    & charmed Lambda plus    & LAMBDA/C+      \\
PcgXz    & \PcgXz    & charmed Xi zero        & not available  \\
PcgXp    & \PcgXp    & charmed Xi plus        & not available  \\
PcgS     & \PcgS     & charmed Sigma (2455)   & not available  \\ \hline
\multicolumn{4}{|c|}{\bf\boldmath Supersymmetric Particles}           \\ \hline
PSgg     & \PSgg     & photino                & PHOTINO        \\
PSgxz    & \PSgxz    & neutralino             & NEUTRALINO     \\
PSZz     & \PSZz     & supersymmetric Z zero  & ZINO           \\
PSHz     & \PSHz     & Higgsino               & HIGGSINO       \\
PSgxpm   & \PSgxpm   & chargino               & CHARGINO       \\
PSWpm    & \PSWpm    & supersymmetric W +-    & not available  \\
PSHpm    & \PSHpm    & charged Higgsino       & not available  \\
PSgn     & \PSgn     & scalar neutrino        & not available  \\
PSe      & \PSe      & scalar electron        & not available  \\
PSgm     & \PSgm     & scalar muon            & not available  \\
PSgt     & \PSgt     & scalar tau             & not available  \\
PSq      & \PSq      & scalar quark           & SQUARK         \\
PaSq     & \PaSq     & scalar anti quark      & SQUARKBAR      \\
PSg      & \PSg      & gluino                 & GLUINO         \\
\end{longtable}
\renewcommand{\arraystretch}{1.0}
\end{latexonly}
\begin{htmlonly}
\begin{tabular}{|l|l|l|l|}
\caption{PEN names for elementary particles in PDG list}
\label{TPNSEXA}
\multicolumn{1}{|c|}{\bf PEN}                             &
\multicolumn{1}{c|}{\bf symbol}                           &
\multicolumn{1}{c|}{\bf conventional name}                &
\multicolumn{1}{c|}{\bf computer name}                        \\ \hline
\multicolumn{4}{|c|}{\bf\boldmath Gauge and Higgs bosons}             \\ \hline
Pgg      & \Pgg    & gamma                    & GAMMA          \\
PW       & \PW     & W boson                  & W              \\
PWp      & \PWp    & W plus                   & W+             \\
PWm      & \PWm    & W minus                  & W-             \\
PZz      & \PZz    & Z zero                   & Z              \\
PHz      & \PHz    & Higgs zero               & not available  \\
PHpm     & \PHpm   & Higgs plus/minus         & HIGGS+-        \\
PWR      & \PWR    & right-handed W           & not available  \\
PWpr     & \PWpr   & W prime                  & WPRIME         \\
PZLR     & \PZLR   & left-right handed Z      & not available  \\
PZgc     & \PZgc   & Z chi                    & not available  \\
PZgy     & \PZgy   & Z psi                    & not available  \\
PZge     & \PZge   & Z eta                    & not available  \\
PZi      & \PZi    & Z one                    & not available  \\
PAz      & \PAz    & axion                    & AXION          \\ \hline
\multicolumn{4}{|c|}{\bf\boldmath Leptons}                            \\ \hline
Pgne     & \Pgne   & electron neutrino        & NUE            \\
Pagne    & \Pagne  & anti electron neutrino   & NUEBAR         \\
Pgngm    & \Pgngm  & muon neutrino            & NUMU           \\
Pagngm   & \Pagngm & anti muon neutrino       & NUMUBAR        \\
Pgngt    & \Pgngt  & tau neutrino             & NUTAU          \\
Pagngt   & \Pagngt & anti tau neutrino        & NUTAUBAR       \\
Pe       & \Pe     & electron                 & not available  \\
Pep      & \Pep    & positron                 & E+             \\
Pem      & \Pem    & e minus                  & E-             \\
Pgm      & \Pgm    & muon                     & not available  \\
Pgmm     & \Pgmm   & mu minus                 & MU-            \\
Pgmp     & \Pgmp   & mu plus                  & MU+            \\
Pgt      & \Pgt    & tau                      & not available  \\
PLpm     & \PLpm   & charged lepton           & LEPTON+-       \\
PLz      & \PLz    & stable neutral heavy lepton   & not available\\
PEz      & \PEz    & neutral para- or ortho-lepton & not available\\ \hline
\multicolumn{4}{|c|}{\bf\boldmath Light Unflavored Mesons (S=C=B=0)}  \\ \hline
Pgp      & \Pgp    & pion                     & PI             \\
Pgpm     & \Pgpm   & pi minus                 & PI-            \\
Pgpp     & \Pgpp   & pi plus                  & PI+            \\
Pgppm    & \Pgppm  & pi plus/minus            & PI+-           \\
Pgpz     & \Pgpz   & pi zero                  & PI0            \\
Pgh      & \Pgh    & eta                      & ETA            \\
Pgr      & \Pgr    & rho                      & RHO(770)       \\
Pgo      & \Pgo    & omega                    & OMEGA(783)     \\
Pghpr    & \Pghpr  & eta prime                & ETAPRIME(958)  \\
Pfz      & \Pfz    & f zero                   & F0(975)        \\
Paz      & \Paz    & a zero                   & A0(980)        \\
Pgf      & \Pgf    & phi                      & PHI(1020)      \\
Phia     & \Phia   & h one                    & H1(1170)       \\
Pbi      & \Pbi    & b one                    & not available  \\
Pai      & \Pai    & a one                    & A1(1260)       \\
Pfii     & \Pfii   & f two                    & F2(1270)       \\
Pfi      & \Pfi    & f one                    & F1(1285)       \\
Pgha     & \Pgha   & eta 1295                 & ETA(1295)      \\
Pgpa     & \Pgpa   & pion 1300                & not available  \\
Paii     & \Paii   & a two                    & A2(1320)       \\
Pgoa     & \Pgoa   & omega 1390               & not available  \\
Pfza     & \Pfza   & f zero 1400              & F0(1400)       \\
Pfia     & \Pfia   & f one 1420               & F1(1420)       \\
Pghb     & \Pghb   & eta 1440                 & ETA(1440)      \\
Pgra     & \Pgra   & rho 1450                 & not available  \\
Pfib     & \Pfib   & f one 1510               & F1(1510)       \\
Pfiipr   & \Pfiipr & f two prime              & F2PRIME(1525)  \\
Pfzb     & \Pfzb   & f zero 1590              & F0(1590)       \\
Pgob     & \Pgob   & omega 1600               & not available  \\
Pgoiii   & \Pgoiii & omega three              & OMEGA3(1670)   \\
Pgpii    & \Pgpii  & pi two                   & PI2(1670)      \\
Pgfa     & \Pgfa   & phi 1680                 & PHI(1680)      \\
Pgriii   & \Pgriii & rho three                & not available  \\
Pgrb     & \Pgrb   & rho 1700                 & RHO(1700)      \\
Pfiia    & \Pfiia  & f two 1720               & F2(1720)       \\
Pgfiii   & \Pgfiii & phi three                & PHI3(1850)     \\
Pfiib    & \Pfiib  & f two 2010               & F2(2010)       \\
Pfiv     & \Pfiv   & f four                   & F4(2050)       \\
Pfiic    & \Pfiic  & f two 2300               & F2(2300)       \\
Pfiid    & \Pfiid  & f two 2340               & F2(2340)       \\ \hline
\multicolumn{4}{|c|}{\bf\boldmath Strange Mesons (S=$\pm1$, C=B=0)} \\ \hline
PK       & \PK     & kaon                     & K              \\
PKpm     & \PKpm   & K plus minus             & K+-            \\
PKp      & \PKp    & K plus                   & K+             \\
PKm      & \PKm    & K minus                  & K-             \\
PKz      & \PKz    & K zero                   & K0             \\
PaKz     & \PaKz   & anti K-zero              & KBAR0          \\
PKgmiii  & \PKgmiii  & K mu three             & not available  \\
PKeiii   & \PKeiii   & K e three              & not available  \\
PKzS     & \PKzS     & K zero short           & not available  \\
PKzL     & \PKzL     & K zero long            & not available  \\
PKzgmiii & \PKzgmiii & K zero mu three        & not available  \\
PKzeiii  & \PKzeiii  & K zero e three         & not available  \\
PKst     & \PKst     & K star                 & not available  \\
PKi      & \PKi      & K one                  & K1(1270)       \\
PKsta    & \PKsta    & K star (1370)          & not available  \\
PKia     & \PKia     & K one (1400)           & not available  \\
PKstz    & \PKstz    & K star zero (1430)     & not available  \\
PKstii   & \PKstii   & K star two (1430)      & not available  \\
PKstb    & \PKstb    & K star (1680)          & not available  \\
PKii     & \PKii     & K two (1770)           & not available  \\
PKstiii  & \PKstiii  & K star three           & not available  \\
PKstiv   & \PKstiv   & K star four            & not available  \\ \hline
\multicolumn{4}{|c|}{\bf\boldmath Charmed Mesons (C=$\pm1$)}          \\ \hline
PD       & \PD       & D                      & D              \\
PaD      & \PaD      & anti D                 & DBAR           \\
PDpm     & \PDpm     & D plus/minus           & D+-            \\
PDm      & \PDm      & D minus                & D-             \\
PDp      & \PDp      & D plus                 & D+             \\
PDz      & \PDz      & D zero                 & D0             \\
PaDz     & \PaDz     & anti D zero            & DBAR0          \\
PDstpm   & \PDstpm   & D star plus/minus      & D*(2010)+-     \\
PDstz    & \PDstz    & D star zero            & D*(2010)0      \\
PDiz     & \PDiz     & D one zero             & D1(2420)0      \\
PDstiiz  & \PDstiiz  & D star two zero        & D2*(2460)0     \\
\newpage
\multicolumn{4}{|c|}{\bf\boldmath Charmed Strange Mesons (C=S=$\pm1$)}\\ \hline
PsDp     & \PsDp     & D s plus               & D/S+           \\
PsDm     & \PsDm     & D s minus              & D/S-           \\
PsDst    & \PsDst    & D s star               & D/S*           \\
PsDipm   & \PsDipm   & D s one plus/minus     & not available  \\ \hline
\multicolumn{4}{|c|}{\bf\boldmath Bottom Mesons (B=$\pm1$)}           \\ \hline
PB       & \PB       & B                      & B              \\
PBp      & \PBp      & B plus                 & B+             \\
PBm      & \PBm      & B minus                & B-             \\
PBpm     & \PBpm     & B plus/minus           & B+-            \\
PBz      & \PBz      & B zero                 & B0             \\
PaB      & \PaB      & anti B                 & BBAR           \\
PaBz     & \PaBz     & anti B zero            & BBAR0          \\
Pcgh     & \Pcgh     & eta c                  & ETA/C(1S)      \\
PJgy     & \PJgy     & J psi                  & J/PSI(1S)      \\
Pcgcz    & \Pcgcz    & chi c zero             & CHI/C0(1P)     \\
Pcgci    & \Pcgci    & chi c one              & CHI/C1(1P)     \\
Pcgcii   & \Pcgcii   & chi c two              & CHI/C2(1P)     \\
Pgy      & \Pgy      & psi                    & PSI(2S)        \\
Pgya     & \Pgya     & psi 3770               & PSI(3770)      \\
Pgyb     & \Pgyb     & psi 4040               & PSI(4040)      \\
Pgyc     & \Pgyc     & psi 4160               & PSI(4160)      \\
Pgyd     & \Pgyd     & psi 4415               & PSI(4415)      \\
PgU      & \PgU      & Upsilon                & not available  \\
Pbgcz    & \Pbgcz    & chi b zero             & CHI/B0(1P)     \\
Pbgci    & \Pbgci    & chi b one              & CHI/B1(1P)     \\
Pbgcii   & \Pbgcii   & chi b two              & CHI/B2(1P)     \\
PgUa     & \PgUa     & Upsilon (2S)           & UPSI(2S)       \\
Pbgcza   & \Pbgcza   & chi b zero (2P)        & CHI/B0(2P)     \\
Pbgcia   & \Pbgcia   & chi b one (2P)         & CHI/B1(2P)     \\
Pbgciia  & \Pbgciia  & chi b two (2P)         & CHI/B2(2P)     \\
PgUb     & \PgUb     & Upsilon (3S)           & UPSI(3S)       \\
PgUc     & \PgUc     & Upsilon (4S)           & UPSI(4S)       \\
PgUd     & \PgUd     & Upsilon (10860)        & UPSI(10860)    \\
PgUe     & \PgUe     & Upsilon (11020)        & UPSI(11020)    \\ \hline
\multicolumn{4}{|c|}{\bf\boldmath quarks}                      \\ \hline
Pq       & \Pq       & quark                  & QUARK          \\
Paq      & \Paq      & anti-quark             & QUARKBAR       \\
Pqd      & \Pqd      & down quark             & DQ             \\
Paqd     & \Paqd     & anti down quark        & DQBAR          \\
Pqu      & \Pqu      & up   quark             & UQ             \\
Paqu     & \Paqu     & anti up   quark        & UQBAR          \\
Pqs      & \Pqs      & strange quark          & SQ             \\
Paqs     & \Paqs     & anti strange quark     & SQBAR          \\
Pqc      & \Pqc      & charmed quark          & CQ             \\
Paqc     & \Paqc     & anti charmed quark     & CQBAR          \\
Pqb      & \Pqb      & bottom  quark          & BQ             \\
Paqb     & \Paqb     & anti bottom  quark     & BQBAR          \\
Pqt      & \Pqt      & top     quark          & TQ             \\
Paqt     & \Paqt     & anti top     quark     & TQBAR          \\ \hline
\multicolumn{4}{|c|}{\bf\boldmath N Baryons (S=0, I=1/2)}      \\ \hline
Pp       & \Pp       & proton                 & P              \\
Pap      & \Pap      & anti-proton            & PBAR           \\
Pn       & \Pn       & neutron                & N              \\
PNa      & \PNa      & N (1440) P 11          & N(1440P11)     \\
PNb      & \PNb      & N (1520) D 13          & not available  \\
PNc      & \PNc      & N (1535) S 11          & not available  \\
PNd      & \PNd      & N (1650) S 11          & not available  \\
PNe      & \PNe      & N (1675) D 15          & not available  \\
PNf      & \PNf      & N (1680) F 15          & not available  \\
PNg      & \PNg      & N (1700) D 13          & not available  \\
PNh      & \PNh      & N (1710) P 11          & not available  \\
PNi      & \PNi      & N (1720) P 13          & not available  \\
PNj      & \PNj      & N (2190) G 17          & not available  \\
PNk      & \PNk      & N (2220) H 19          & not available  \\
PNl      & \PNl      & N (2250) G 19          & not available  \\
PNm      & \PNm      & N (2600) I 1,11        & not available  \\ \hline
\multicolumn{4}{|c|}{\bf\boldmath $\Delta$ Baryons (S=0, I=3/2)}\\ \hline
PgDa     & \PgDa     & Delta (1232) P 33      & DELTA(1232P33) \\
PgDb     & \PgDb     & Delta (1620) S 31      & not available  \\
PgDc     & \PgDc     & Delta (1700) D 33      & not available  \\
PgDd     & \PgDd     & Delta (1900) S 31      & not available  \\
PgDe     & \PgDe     & Delta (1905) F 35      & not available  \\
PgDf     & \PgDf     & Delta (1910) P 31      & not available  \\
PgDh     & \PgDh     & Delta (1920) P 33      & not available  \\
PgDi     & \PgDi     & Delta (1930) D 35      & not available  \\
PgDj     & \PgDj     & Delta (1950) F 37      & not available  \\
PgDk     & \PgDk     & Delta (2420) H 3,11    & not available  \\ \hline
\multicolumn{4}{|c|}{\bf\boldmath $\Lambda$ Baryons (S=$-1$, I=0)}\\ \hline
PgL      & \PgL      & Lambda                 & LAMBDA         \\
PagL     & \PagL     & anti Lambda            & LAMBDABAR      \\
PgLa     & \PgLa     & Lambda (1405) S 01     & LAMBDA(1405S01)\\
PgLb     & \PgLb     & Lambda (1520) D 03     & LAMBDA(1520D03)\\
PgLc     & \PgLc     & Lambda (1600) P 01     & not available  \\
PgLd     & \PgLd     & Lambda (1670) S 01     & not available  \\
PgLe     & \PgLe     & Lambda (1690) D 03     & not available  \\
PgLf     & \PgLf     & Lambda (1800) S 01     & not available  \\
PgLg     & \PgLg     & Lambda (1810) P 01     & not available  \\
PgLh     & \PgLh     & Lambda (1820) F 05     & not available  \\
PgLi     & \PgLi     & Lambda (1830) D 05     & not available  \\
PgLj     & \PgLj     & Lambda (1890) P 03     & not available  \\
PgLk     & \PgLk     & Lambda (2100) G 07     & not available  \\
PgLl     & \PgLl     & Lambda (2110) F 05     & not available  \\
PgLm     & \PgLm     & Lambda (2350) H 09     & not available  \\ \hline
\multicolumn{4}{|c|}{\bf\boldmath $\Sigma$ Baryons (S=$-1$, I=1)}\\ \hline
PgSm     & \PgSm     & Sigma minus            & SIGMA-         \\
PgSp     & \PgSp     & Sigma plus             & SIGMA+         \\
PagSm    & \PagSm    & anti Sigma minus       & SIGMABAR-      \\
PagSp    & \PagSp    & anti Sigma plus        & SIGMABAR+      \\
PgSz     & \PgSz     & Sigma zero             & SIGMA0         \\
PagSz    & \PagSz    & anti Sigma zero        & SIGMABAR0      \\
PgSa     & \PgSa     & Sigma (1385) P 13      & not available  \\
PgSb     & \PgSb     & Sigma (1660) P 11      & not available  \\
PgSc     & \PgSc     & Sigma (1670) D 13      & not available  \\
PgSd     & \PgSd     & Sigma (1750) S 11      & not available  \\
PgSe     & \PgSe     & Sigma (1775) D 15      & not available  \\
PgSf     & \PgSf     & Sigma (1915) F 15      & not available  \\
PgSg     & \PgSg     & Sigma (1940) D 13      & not available  \\
PgSh     & \PgSh     & Sigma (2030) F 17      & not available  \\
PgSi     & \PgSi     & Sigma (2250)           & not available  \\ 
\newpage
\multicolumn{4}{|c|}{\bf\boldmath $\Xi$ Baryons (S=$-2$, I=1/2)}      \\ \hline
PgXz     & \PgXz     & Xi zero                & XI0            \\
PagXz    & \PagXz    & anti Xi zero           & XIBAR0         \\
PgXm     & \PgXm     & Xi minus               & XI-            \\
PagXp    & \PagXp    & anti Xi plus           & XIBAR+         \\
PgXa     & \PgXa     & Xi (1530) P 13         & not available  \\
PgXb     & \PgXb     & Xi (1690)              & not available  \\
PgXc     & \PgXc     & Xi (1820) D 13         & not available  \\
PgXd     & \PgXd     & Xi (1950)              & not available  \\
PgXe     & \PgXe     & Xi (2030)              & not available  \\ \hline
\multicolumn{4}{|c|}{\bf\boldmath $\Omega$ Baryons (S=$-3$, I=0)}     \\ \hline
PgOm     & \PgOm     & Omega minus            & OMEGA-         \\
PagOp    & \PagOp    & anti Omega plus        & OMEGABAR+      \\
PgOma    & \PgOma    & Omega (2250) minus     & OMEGA(2250)-   \\ \hline
\multicolumn{4}{|c|}{\bf\boldmath Charmed Baryons (C=$+1$)}           \\ \hline
PcgLp    & \PcgLp    & charmed Lambda plus    & LAMBDA/C+      \\
PcgXz    & \PcgXz    & charmed Xi zero        & not available  \\
PcgXp    & \PcgXp    & charmed Xi plus        & not available  \\
PcgS     & \PcgS     & charmed Sigma (2455)   & not available  \\ \hline
\multicolumn{4}{|c|}{\bf\boldmath Supersymmetric Particles}           \\ \hline
PSgg     & \PSgg     & photino                & PHOTINO        \\
PSgxz    & \PSgxz    & neutralino             & NEUTRALINO     \\
PSZz     & \PSZz     & supersymmetric Z zero  & ZINO           \\
PSHz     & \PSHz     & Higgsino               & HIGGSINO       \\
PSgxpm   & \PSgxpm   & chargino               & CHARGINO       \\
PSWpm    & \PSWpm    & supersymmetric W +-    & not available  \\
PSHpm    & \PSHpm    & charged Higgsino       & not available  \\
PSgn     & \PSgn     & scalar neutrino        & not available  \\
PSe      & \PSe      & scalar electron        & not available  \\
PSgm     & \PSgm     & scalar muon            & not available  \\
PSgt     & \PSgt     & scalar tau             & not available  \\
PSq      & \PSq      & scalar quark           & SQUARK         \\
PaSq     & \PaSq     & scalar anti quark      & SQUARKBAR      \\
PSg      & \PSg      & gluino                 & GLUINO         \\
\end{tabular}
\end{htmlonly}
 
\end{document}
