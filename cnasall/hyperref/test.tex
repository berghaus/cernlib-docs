\errorcontextlines=10
\documentclass[]{book}
\makeatletter
\def\listoffigures{\@restonecolfalse\if@twocolumn\@restonecoltrue\onecolumn\fi
\chapter*{List of Figures}
{\let\\ \ \markboth{Title}{List of Figures}}
 \addcontentsline{toc}{chapter}{\protect
 \numberline{LIST OF FIGURES\string\hss}}
\@starttoc{lof}\if@restonecol \twocolumn\fi}
\makeatother
\usepackage{epsfig,longtable}
\usepackage{times,makeidx}
\usepackage{url}
\RequirePackage[hyperindex,colorlinks,backref]{hyperref}
\usepackage{varioref}
\def\boldindex#1{\textbf{\hyperpage{#1}}}
\makeindex
\newcommand{\eqnref}[1]{{equation~$\ref{#1}$}}
\begin{document}
\pdfbookmark{Title}{tit}
\title{Testing Hyperref}
\author{Sebastian Rahtz}
\date{May 1997}
\setlongtables
\maketitle
\tableofcontents
\listoffigures
\newtheorem{Argument}{Here we go}[section]
\def\thesubsection{\Roman{section} -- \arabic{subsection}}
\setcounter{tocdepth}{1}
\chapter[First part]{First part, leading to \protect\ref{horrid} next}
\section[Test section]{Our \protect\LaTeX\ test section (leading to \ref{One})
 for 100\% of \AE horrid $X[Y]Z$ 
 things, like $42$\label{horrid}}
and so see section \ref{horrid}.
\section{Section One --- cats}\label{One}
see section \vref{Three} about cats\index{animals!cats} and cite 
\cite{Barcelo:1992:caa}
\subsection{one.1}
some text with a footnote\footnote{WISH UPON A STAR}
and a reference to a long table\index{tables!long!longtables}, \ref{mylongtab}
\subsection{one.2}
dogs
\newpage
\section{Section Two --- \TeX\ is a dog}
\subsection{two.1}
\subsection{two.2}
\newpage
cite \cite{Barcelo:1992:caa} again.
\chapter{Second part}
\section{Section Three --- Camels}\label{Three}
see section \ref{One}
\subsection{three.1}
some text with a footnote\footnote{OVER THE RAINBOW}
\index{rainbows}
\subsection{three.2}

\newpage
\section[Section Four --- Butterflies]{Section Four --- Butterflies and so on}
\subsection{four.1}
\subsection{four.2}
camels
Refer to \hyperref{}{test}{test1}{with these words}
\newpage
\section{Introduction}\label{sec1}
\subsection{subsec}
\newpage
\subsection{subsec}\label{subsec1.2}
Define a marker \hyperdef{}{test}{test1}{here} while this one is a
PostScript picture acting as marker:
\index{PS pictures}
\hyperlink{testpiccy}{\epsfig{figure=picture,height=1in}}
\newpage
\section{two}\label{sec2}
\subsection{Subsection 2}
\subsection{Subsection 3}
\newpage
\section{three}
This is a reference to section 1 (\ref{sec1}), subsection 1.2 (\ref{subsec1.2})
and section 2 (\ref{sec2}). References to \cite{Barcelo:1992:caa,Dallas:aia}.

\begin{figure}

xxxx

\hypertarget{testpiccy}{Test picture}

xxxxx

xxxxx
\caption{{A cat}}
\label{fig1}
\end{figure}

\texttt{<<where is \eqnref{eq1}>>}

and does a link to a url like \url{http://www.tug.org} work?

\begin{equation}
  \label{eq1}
  zzzz + b
\end{equation}
\newpage
\begin{equation}
  \label{eq2}
  d - e
\end{equation}

\begin{eqnarray}
  \label{eq3}
  y &=&z\\
  g &=&h\\
\end{eqnarray}
We need some lists:
\begin{enumerate}
\item oranges\index{oranges|boldindex}
\item lemons\index{lemons|boldindex}
\item beer\index{beer|boldindex}
  \begin{enumerate}
  \item Samuel Smiths
  \item Labatts
  \end{enumerate}
\end{enumerate}

Lets look at labels in lists:
\begin{enumerate}
\item oranges\label{oranges}
\item lemons\label{lemons}
\item beer\label{beer}
  \begin{enumerate}
  \item Samuel Smiths\label{smiths}
  \item Labatts\label{labatts}
  \end{enumerate}
\end{enumerate}
\clearpage

from which see \ref{oranges}, \ref{lemons}, \ref{smiths} and
\ref{labatts}


see
sec1: \ref{sec1}
sec2: \ref{sec2}
eq1: \ref{eq1}
fig1: \ref{fig1}
and cite \cite{Barcelo:1992:caa} again.
\onecolumn
\begin{longtable}{lll}
\caption{A test long table}\label{mylongtab}\\
a & b & c \\a
a & b & c \\a
a & b & c \\a
a & b & c \\a
a & b & c \\a
a & b & c \\a
a & b & c \\a
a & b & c \\a
a & b & c \\a
a & b & c \\a
a & b & c \\a
a & b & c \\a
a & b & c \\a
a & b & c \\a
a & b & c \\a
a & b & c \\a
a & b & c \\a
a & b & c \\a
a & b & c \\a
a & b & c \\a
a & b & c \\a
a & b & c \\a
a & b & c \\a
a & b & c \\a
a & b & c \\a
a & b & c \\a
a & b & c \\a
a & b & c \\a
a & b & c \\a
a & b & c \\a
a & b & c \\a
a & b & c \\a
a & b & c \\a
a & b & c \\a
a & b & c \\a
a & b & c \\a
a & b & c \\a
a & b & c \\a
a & b & c \\a
a & b & c \\a
a & b & c \\a
a & b & c \\a
a & b & c \\a
a & b & c \\a
a & b & c \\a
a & b & c \\a
a & b & c \\a
a & b & c \\a
a & b & c \\a
a & b & c \\a
a & b & c \\a
a & b & c \\a
a & b & c \\a
a & b & c \\a
\end{longtable}
Does \hyperref{}{equation}{2.5.1}{this} point to the second equation?
Does anything point to the eqnarray (\ref{eq3})?
\index{cats}

\begin{thebibliography}{99}
\bibitem{Barcelo:1992:caa}
{Barcel\'o, J.} 1992.
\newblock Programming an intelligent database in archaeology. In \emph{Computer
  Applications and Quantitative Methods in Archaeology 1991}, {Lock, G. \&
  J.~Moffett} (eds),   21--28, Oxford: British Archaeological Reports.

\bibitem[Dallas 1992]{Dallas:aia}
{Dallas, C.~J.} 1992.
\newblock Syntax and semantics of figurative art: a formal approach. In
  \emph{Archaeology and the Information Age}, {Reilly, P. \& S.~Rahtz} (eds),
  chapter~16, London: Routledge.

\bibitem[Stankovic 1988]{stankovic:88}
J.~Stankovic, ``Misconceptions about real-time computing: a serious problem for
  next-generation systems,'' {\em Computer}, vol.~21, no.~10, pp.~10--19, Oct.
  1988.

\end{thebibliography}
\clearpage
An index entry for gnus\index{gnus}
\clearpage
An index entry for gnus\index{gnus}
\clearpage
An index entry for gnus\index{gnus}
\clearpage
An index entry for gnus\index{gnus}
\clearpage
An index entry for gnus\index{gnus}
\clearpage
An index entry for gnus\index{gnus}
\chapter*{An appendix --- the Index}
\printindex
\end{document}
