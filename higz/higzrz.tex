%%%%%%%%%%%%%%%%%%%%%%%%%%%%%%%%%%%%%%%%%%%%%%%%%%%%%%%%%%%%%%%%%%%%%%%%%%%%%%%%
%                                                                              %
%   HIGZ  User Guide -- LaTeX Source                                           %
%                                                                              %
%   Chapter: Interface routines with RZ                                        %
%                                                                              %
%   This document needs the following external EPS files:                      %
%     none                                                                     %
%                                                                              %
%   Editor: Michel Goossens / AS-MI                                            %
%   Last Mod.:  9 July 1993 oc                                                 %
%                                                                              %
%%%%%%%%%%%%%%%%%%%%%%%%%%%%%%%%%%%%%%%%%%%%%%%%%%%%%%%%%%%%%%%%%%%%%%%%%%%%%%%%
\Filename{H1Storingpictures}
\chapter{Storing pictures on ZEBRA/RZ direct access files}
\index{interface with RZ}

The routines described in this chapter allow the \HIGZ{} user to store pictures
on disk and retrieve them. The pictures created on disk by a given \HIGZ{} 
program can be used by other \HIGZ{} application programs. Facilities to list 
the contents of a RZ directory, to purge old cycles, create subdirectories, 
etc. are available in the \ZEBRA/RZ package.


\Filename{H2Interfaceroutines}
\section{Interface routines}
\index{interface routines}
\Shubr{IZFILE}{(LUN,CHDIR,CHOPT)}
\Action
This routine declares a pre-open direct acces file to be \ZEBRA/RZ file.
\Pdesc
\begin{DLtt}{1234567}
\item[LUN]   Logical unit number.
\item[CHDIR] {\tt CHARACTER} variable specifying the name of the top directory.
\item[CHOPT] {\tt CHARACTER} variable specifying the option(s) desired:
\begin{DLtt}{12345}
   \item['N'] Creates a {\bf N}ew RZ file with top directory name {\tt CHDIR}
   \item[' '] Open an existing RZ file with read only access.
   \item['U'] Open an existing RZ file in {\bf U}pdate mode.
   \item['A'] Pictures are {\bf A}utomatically saved on disk.
\end{DLtt}
\end{DLtt}
When option {\tt'A'} is given or when option \Sind{AURZ} is activated by 
\Rind{IGSET}, pictures are automatically saved into the RZ file. In this case,
there is only one picture in memory (the current picture). The last current 
picture is written on disk when \Rind{IGEND} is called.
 

\Shubr{IZOPEN}{(LUN,CHDIR,CFNAME,CHOPT,*LRECL*,ISTAT*)}
\Action
Open a HIGZ/RZ picture file. This routine open a direct access file and
call \Rind{IZFILE}. For more details see the description of the \ZEBRA{}
routine \Rind{RZOPEN} in the \ZEBRA{} manual.
\Pdesc
\begin{DLtt}{1234567}
\item[LUN]    Logical unit number.
\item[CHDIR]  {\tt CHARACTER} variable specifying the name of
              the top directory.
\item[CFNAME] File name.
\item[CHOPT]  {\tt CHARACTER} variable specifying the option(s) desired:
\begin{DLtt}{12345}
   \item['N'] Creates a {\bf N}ew RZ file with top directory name {\tt CHDIR}
   \item[' '] Open an existing RZ file with read only access.
   \item['U'] Open an existing RZ file in {\bf U}pdate mode.
   \item['A'] Pictures are {\bf A}utomatically saved on disk.
\end{DLtt}
\item[LRECL]  Integer variable specifying the record length of the file in 
              machine words. If a value of zero (0) is specified \Rind{IZOPEN}
              will attempt to obtain the correct record length from the file 
              itself. A value of zero must not be specified for new files.
\item[ISTAT]  Integer variable in which the status code is returned. 
\end{DLtt}
When option {\tt'A'} is given or when option \Sind{AURZ} is activated by 
\Rind{IGSET}, pictures are automatically saved into the RZ file. In this case,
there is only one picture in memory (the current picture). The last current 
picture is written on disk when \Rind{IGEND} is called.


\Shubr{IZIN}{(PNAME,ICYCLE)}
\Action
This routine reads a picture from an RZ data file and puts it in memory.
\Pdesc
\begin{DLtt}{1234567}
\item[PNAME] {\tt CHARACTER} variable specifying the name of picture to be read.
\item[ICYCLE] Cycle number of the picture to be read.
If {\tt ICYCLE} is greater than the highest existing cycle number on the RZ 
file, then the picture with the highest cycle number is read.
\end{DLtt}


\Shubr{IZOUT}{(PNAME,ICYCLE*)}
\Action
This routine writes on an RZ data file a memory resident picture.
\Pdesc
\begin{DLtt}{12345678}
\item[PNAME] {\tt CHARACTER} variable specifying the name of picture to be
             written.
\item[ICYCLE*] Cycle number of the picture written. If a picture with name 
               {\tt PNAME} does not already exist on the output file, then a 
               value for {\tt ICYCLE} of {\tt 1} is returned, otherwise a value
               one higher than the (previous) highest cycle number on the file.
\end{DLtt}


\Shubr{IZSCR}{(PNAME,ICYCLE)}
\Action
This routine deletes (scratches) a picture from an RZ data file.
\Pdesc
\begin{DLtt}{1234567}
\item[PNAME] {\tt CHARACTER} variable specifying the name of
picture to be deleted.
\item[ICYCLE] Cycle number of the picture to be deleted.
\end{DLtt}
