%%%%%%%%%%%%%%%%%%%%%%%%%%%%%%%%%%%%%%%%%%%%%%%%%%%%%%%%%%%%%%%%%%%%%%%%%%%%%%%%
%                                                                              %
%   HIGZ/HPLOT User Guide -- LaTeX Source                                      %
%                                                                              %
%   Front material (Title page, copyright, foreword,...)                       %
%                                                                              %
%   Last Mod. 13 Sept 1993 16:20 mg                                            %
%                                                                              %
%%%%%%%%%%%%%%%%%%%%%%%%%%%%%%%%%%%%%%%%%%%%%%%%%%%%%%%%%%%%%%%%%%%%%%%%%%%%%%%%
 
%%%%%%%%%%%%%%%%%%%%%%%%%%%%%%%%%%%%%%%%%%%%%%%%%%%%%%%%%%%%%%%%%%%%%%%%%%%%%%%%
%    Tile page                                                                 %
%%%%%%%%%%%%%%%%%%%%%%%%%%%%%%%%%%%%%%%%%%%%%%%%%%%%%%%%%%%%%%%%%%%%%%%%%%%%%%%%
\notHTML{\def\Ptitle#1{\special{ps: /Printstring (#1) def}
                       \epsfbox{/user/goossens/cnasall/cnastit.eps}}}
\HTML{\def\Ptitle#1{<B>#1</B>}}
 
\begin{titlepage}
\notHTML{\vspace*{-23mm}}%
\notHTML{\mbox{\epsfig{file=/usr/local/lib/tex/ps/cern15.eps,height=30mm}}}%
\HTML{<IMG SRC="../cernlogo.gif">}%
\hfill
\raise8mm\hbox{\Large\bf CERN Program Library Long Writeups Q120 and Y251}
\hfill\mbox{}
\HTML{<P>}
\begin{center}
\mbox{}\\[10mm]
\mbox{\Ptitle{HIGZ}}\\[0.5cm]
{\LARGE High Level Interface to Graphics and Zebra}\\[6mm]
{\LARGE User's Guide}\\[15mm]
\mbox{\Ptitle{HPLOT}}\\[0.5cm]
{\LARGE User's Guide}\\[25mm]
\HTML{<P>\\}
{\Large Application Software Group}\\[0.5cm]
{\Large Computing and Networks Division}\\[0.5cm]
\end{center}
\HTML{<P>}
\notHTML{\vfill}%
\begin{center}\Large CERN Geneva, Switzerland\end{center}
\end{titlepage}

\Filename{H1Preface}
\HTML{<H1>Preface</H1>}

%%%%%%%%%%%%%%%%%%%%%%%%%%%%%%%%%%%%%%%%%%%%%%%%%%%%%%%%%%%%%%%%%%%%
%    Copyright  page                                               %
%%%%%%%%%%%%%%%%%%%%%%%%%%%%%%%%%%%%%%%%%%%%%%%%%%%%%%%%%%%%%%%%%%%%
\HTML{\PRE}
\thispagestyle{empty}
\framebox[\textwidth][t]{\hfill\begin{minipage}{0.96\textwidth}%
\vspace*{3mm}\begin{center}Copyright Notice\end{center}
\parskip\baselineskip
CERN Program Library entries {\bf Q120} and {\bf Y251}
 
{\bf HIGZ -- High level Interface to Graphics and Zebra}
 
{\bf HPLOT -- User's Guide}
 
\copyright{} Copyright CERN, Geneva 1993
 
Copyright and any other appropriate legal protection of these
computer programs and associated documentation reserved in all
countries of the world.
 
These programs or documentation may not be reproduced by any
method without prior written consent of the Director-General
of CERN or his delegate.
 
Permission for the usage of any programs described herein is
granted apriori to those scientific institutes associated with
the CERN experimental program or with whom CERN has concluded
a scientific collaboration agreement.
 
Requests for information should be addressed to:
\vspace*{-.5\baselineskip}
\begin{center}
\tt\begin{tabular}{l}
CERN Program Library Office              \\
CERN-CN Division                         \\
CH-1211 Geneva 23                        \\
Switzerland                              \\
Tel.      +41 22 767 4951                \\
Fax.      +41 22 767 7155                \\
Bitnet:   CERNLIB@CERNVM                 \\
DECnet:   VXCERN::CERNLIB (node 22.190)  \\
Internet: CERNLIB@CERNVM.CERN.CH
\end{tabular}
\end{center}
\vspace*{2mm}
\end{minipage}\hfill}%end of minipage in framebox
\vspace{6mm}
\HTML{<P>}
 
{\bf Trademark notice: All trademarks appearing in this guide are acknowledged as such.}
\vfill
\HTML{<P>}
\begin{tabular}{l@{\quad}l@{\quad}>{\tt}l}
{\em Contact Person\/}:        & Olivier Couet /CN   & (COUET\atsign CERNVM.CERN.CH)\\[1mm]
{\em Technical Realization\/}: & Michel Goossens /CN & (GOOSSENS\atsign CERNVM.CERN.CH)\\[1cm]
{\em Edition -- October 1993}
\end{tabular}
\HTML{\ePRE}%
\newpage

%%%%%%%%%%%%%%%%%%%%%%%%%%%%%%%%%%%%%%%%%%%%%%%%%%%%%%%%%%%%%%%%%%%%%%%%%%%%%%%%
%    Introductory material                                                     %
%%%%%%%%%%%%%%%%%%%%%%%%%%%%%%%%%%%%%%%%%%%%%%%%%%%%%%%%%%%%%%%%%%%%%%%%%%%%%%%%
\pagenumbering{roman}
\setcounter{page}{1}
\Filename{H2HIGZPreliminary-remarks}
\section*{Preliminary remarks}

This guide conbines the user documentation for both the
HIGZ (Part I) and HPLOT (Part II) packages.
They are implemented on various mainframes (e.g. IBM~VM/CMS, Cray and VAX/VMS)
and Unix workstations (e.g. HP, Apollo, Ultrix, IBM RS6000, Silicon Graphics and 
Sun).
\index{IBM!VM/CMS}\index{VM/CMS!IBM system}\index{VAX/VMS}\index{Cray}
\index{Unix}\index{workstation}\index{mainframe}
\index{Hewlett Packard}\index{Apollo}\index{Ultrix}\index{IBM!RS6000}
\index{Silicon Graphics}\index{Sun}
 
\HIGZ~has been designed to provide basic graphics functions similar to \GKS. 
\HPLOT{} is a histogram plotting and editing system closely linked to 
\HBOOK.

\subsection*{notation}
\index{notation}

Throughout this manual, all the \GKS~like functions are indicated as follows:

\Shubr
[GKS]{GKSLIKE}{(parameters)}
 
Type of the subroutine parameters is defined by their
initial letter following the usual \FORTRAN~conventions:

\begin{ULc}
\item parameters starting with the letter {\tt I} through {\tt N} are 
      {\tt INTEGER}.
\item parameters starting with the letter {\tt A} through {\tt H} and {\tt O}
      through {\tt Z} are {\tt REAL}.
\item in addition to the above, parameters starting with the
      sequence {\tt CH} are of type {\tt CHARACTER}.
\end{ULc}

In the description of the routines a \Lit{*} following
the name of a parameter indicates that this is an {\bf output} parameter
(e.g. \Lit{OUTPAR*}).
If another \Lit{*} precedes a parameter in the calling sequence, the
parameter in question is both an {\bf input} and {\bf output} parameter
(e.g. \Lit{*IOPAR*}).

Examples are in {\tt monotype face} and strings to be input by the user 
are {\tt\underline{underlined}}.
In the index the page where a routine is defined is in {\bf bold},
page numbers where a routine is referenced are in normal type.

This document has been produced using \LaTeX~\cite{bib-LATEX}
with the \Lit{cernman} style option, developed at CERN. 
A compressed PostScript file \Lit{higz.ps}, containing a complete printable version
of this manual, can be obtained from any CERN machine
by anonymous ftp as follows
(commands to be typed by the user are underlined):

\vspace*{3mm} 
\begin{tabular}{@{\hspace{12mm}}>{\tt}l}
\underline{ftp asis01.cern.ch}\\
Trying 128.141.201.136...\\
Connected to asis01.cern.ch.\\
220 asis01 FTP server (SunOS 4.1) ready.\\
Name (asis01:username): \underline{anonymous}\\
Password: \underline{your\_{}mailaddress}\\
ftp> \underline{cd cernlib/doc/ps.dir}\\
ftp> \underline{binary}\\
ftp> \underline{get higz.ps.Z}\\
ftp> \underline{quit}\\
\end{tabular}
\ding{32}% space in Zapfdingbat font (to have zapf font loaded)
 
%%%%%%%%%%%%%%%%%%%%%%%%%%%%%%%%%%%%%%%%%%%%%%%%%%%%%%%%%%%%%%%%%%%%%%%%%%%%%%%%
%    Tables of contents ...                                                    %
%%%%%%%%%%%%%%%%%%%%%%%%%%%%%%%%%%%%%%%%%%%%%%%%%%%%%%%%%%%%%%%%%%%%%%%%%%%%%%%%
\newpage
\tableofcontents
\newpage
\listoffigures
\listoftables
 
