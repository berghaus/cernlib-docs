\batchmode

\documentstyle[11pt,epsfig,longtable,changebar,makeidx,lscape]{cernman}
\makeatletter
\makeindex
\newcommand{\FZfile}{FZ~file\index{FZ!Sequential input/output}\index{input/output!FZ}}
\newcommand{\RZfile}{RZ~file\index{RZ!Random input/output}\index{input/output!RZ}}
\newcommand{\IQUEST}{\Lit{IQUEST}\index{IQUEST@{\tt IQUEST}!user communication vector in common {\tt QUEST}}\index{IQUEST@{\tt IQUEST}!error reporting}\index{error reporting!{\tt IQUEST}}\index{QUEST@{\tt QUEST}!user communication common}}
\newcommand{\QUEST}{\Lit{QUEST}\index{IQUEST@{\tt IQUEST}!user communication vector in common {\tt QUEST}}\index{IQUEST@{\tt IQUEST}!error reporting}\index{error reporting!{\tt IQUEST}}\index{QUEST@{\tt QUEST}!user communication common}}
\renewcommand{\ZEBRA}{\textsc{ZEBRA}}
\renewcommand{\Copt}[1]{\texttt{#1}}
\renewcommand{\Ropt}[1]{\texttt{#1}}
\renewcommand{\Rarg}[1]{\texttt{#1}}
\def\condbreak#1{}
\driver{DVIPS}
\setlongtables
\makeindex
\romanfont{times}
\PScommands\setcounter{secnumdepth}{3}
\setcounter{tocdepth}{2}
\newenvironment{landscapebody}{\begin{landscape}}{\end{landscape}}
\makeatletter
\def\LS@rot{\setbox\@outputbox=\vbox{\@rotr\@outputbox}}
\makeatother
\long\def\NODOC#1{#1}
\renewcommand{\ZEBRA}{\textsc{ZEBRA}}\renewcommand{\Copt}[1]{\texttt{#1}}\renewcommand{\Ropt}[1]{\texttt{#1}}\renewcommand{\Rarg}[1]{\texttt{#1}}\def\condbreak#1{}\def\LS@rot{\setbox\@outputbox=\vbox{\@rotr\@outputbox}}\def\NODOC#1{#1}
\makeatother
\newenvironment{tex2html_wrap}{}{}
\usepackage{screen}
\begin{document}
\pagestyle{empty}
\stepcounter{chapter}
\stepcounter{section}
\stepcounter{subsection}
\newpage

{\samepage \clearpage \begin{landscape}\mbox{}\vspace*{1cm}
\centerline{\textbf{{Example of (part of) output generated by \Rind{DZSHOW} (\Ropt{D} Down option)}}}<tex2html_verbatim_mark>XMPt2
\newpage
\mbox{}\vspace*{1cm}
\centerline{\textbf{{Example of (part of) output generated by \Rind{DZSHOW} (\Ropt{S} Side option)}}}<tex2html_verbatim_mark>XMPt3
\end{landscape}
}


\stepcounter{subsection}
\stepcounter{subsection}
\newpage

{\samepage \clearpage \begin{landscape}\mbox{}\vspace*{1cm}
\centerline{\textbf{{Example of the use of \Rind{DZSTOR}}}}<tex2html_verbatim_mark>XMPt4
\end{landscape}
}


\stepcounter{subsection}
\stepcounter{subsection}
\newpage

{\samepage \clearpage \begin{UL}\item The cumulative number of words occupied by all banks so far
\item The total number of words occupied by all banks at this level
\item The length of the longest bank at this level
\item The number of banks at this level (any identifier)
\item Structural relation
\item Bank identifier(s)
\end{UL}
}


\stepcounter{section}
\stepcounter{subsection}
\stepcounter{subsubsection}
\stepcounter{subsubsection}
\newpage

{\samepage \clearpage \begin{OLc}\item The 4 character Hollerith bank identifier preceded by a \Lit{(}
if the bank has been dropped.
\item The bank numeric identifier
\item The address of the bank (status word) relative to the beginning of
the store and as an absolute address (in octal or hexadecimal)
\item The contents of the system and user part of the status word of the
of the bank (bits \Lit{19-32} and \Lit{1-18}) and of its I/O characteristic.
\item Number of links (\Lit{NL})/ of structural links
(\Lit{NS})/ of data words (\Lit{ND})
\item The contents of the next (\Lit{N})/up (\Lit{U})/and origin (\Lit{O})
\index{link!next}
links of the bank,
as well as of the contents of the address pointed to by the origin link
\index{link!origin}
(\Lit{@O}), which should contain
the address of the bank itself (hence allowing an easy cross-check).
When an inconsistency is detected the
faulty address is preceded by a minus sign (\Lit{-}).
\end{OLc}
}


\newpage

{\samepage \clearpage \begin{OL}\item a two character flag:
\begin{DLttc}{12}
\item[**]the bank is dropped (also signaled by a left parenthesis '('
on the first line)
\item[.]the bank is active, all non-zero links are printed
\item[+]the bank is active, not all non-zero links are printed
\item[F]in position 2 flags a bank with potentially dangerous
contents in the links printed. This could be either:
\begin{ULc}
\item illegal link content
\item dropped bank supporting an active bank (not via \Lit{NX} link)
\item active bank pointing to a dropped bank
\end{ULc}
\end{DLttc}
\item links \Lit{1,2....N} are printed in this order with \Lit{N} the smaller of the
the following 2 numbers:
\begin{DLttc}{12}
\item[N1] the last non-zero link of this bank;
\item[N2] the number of links which can be printed on one line
(typically 9)
\end{DLttc}
If the link points to a correct bank-address, the \Lit{ID} of that
bank is also printed, preceded by \Lit{(} if this bank has been dropped.
If the link does not point to a status word, then a \Lit{-} or
\Lit{****} is printed against it for legal or illegal link content.
\end{OL}
}


\newpage

{\samepage \clearpage \begin{landscape}\mbox{}\vspace*{1cm}
\centerline{\textbf{{Example of the output of \Rind{DZSNAP}}}}<tex2html_verbatim_mark>XMPt8
\end{landscape}
}


\stepcounter{subsection}
\stepcounter{section}
\stepcounter{subsection}
\stepcounter{subsection}
\stepcounter{chapter}

\end{document}