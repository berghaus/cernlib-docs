%%%%%%%%%%%%%%%%%%%%%%%%%%%%%%%%%%%%%%%%%%%%%%%%%%%%%%%%%%%%%%%%%%%
%                                                                 %
%   ZEBRA User Guide -- LaTeX Source                              %
%                                                                 %
%   Chapter FZ (Sequential input/output)                          %
%                                                                 %
%   The following external EPS files are referenced:              %
%   none                                                          %
%                                                                 %
%   Editor: Michel Goossens / AS-MI                               %
%   Last Mod.:  8 Dec. 1990   mg                                  %
%                                                                 %
%%%%%%%%%%%%%%%%%%%%%%%%%%%%%%%%%%%%%%%%%%%%%%%%%%%%%%%%%%%%%%%%%%%
\chapter{FZ: The sequential Input-Output Package}
\section{Principles}
\par The aim of the FZ package
is to transfer ZEBRA data structures from and to a sequential file,
i.e. disk or tape, or to another part of memory.
It should fit both online and offline processing requirements.
Speed is important in both cases,
but is more critical in the online context.
For this it must be possible to write large areas of memory
to a file, without the need to insert control information
during the transfer; all control information needed
should be grouped at the begining.
Also, since ZEBRA may not be running on the online machine,
the format should be such that it is easy to program in other systems.
\subsection{Native and exchange I/O modes}
\index{FZ!mode!native}
\index{FZ!mode!exchange}
\par Files written in {\bf native} mode can be read only on machines
of the same kind as the writing machine.
The logical records are written with ordinary unformatted {\tt WRITE}
statements,
directly transferring the memory areas to the file.
Hence the representation of the data in the file is the same as
the internal representation of the machine;
the blocking of the logical records and the addition of
system control words
is done by the operating system,
and is of no concern to ZEBRA.
\par {\bf Exchange} mode enables transfer of files between different
machines. The input/output routines, of which there is
one set for each machine, have to handle two things:
one is the blocking of the logical records into physical records,
the other is the translation of every word from the internal
to the exchange representation,
according to the nature of the word
(floating-point, integer, Hollerith, bit pattern).
\subsection{Logical records structure}
\par The unit of information to be written or read is one data structure,
which in the simplest case may consist of zero or one bank.
A data structure on a file is represented by two,
or possibly several,
{\bf logical records.}
\index{FZ!record!logical}
\footnote{The format of logical records and of the files written
are described in the ZEBRA reference manual.}
The first record, called the ``pilot record'',
\index{FZ!record!pilot}
contains context information about the data structure which follows,
such that the program can set up everything necessary before
initiating a read to receive the data structure
at the right place in memory.
Amongst other things,
this specifies the amount of memory required to hold
the data structure and the table to drive the relocation
of all the links in the structure.
The pilot record also contains a user-defined identification
and characteristic of the data structure, the so-called
{\bf user header vector}.
It permits selective reading without expanding unwanted
data structures into memory.
\par The second logical record (and subsequent ones, if any)
contain the dump of the memory area occupied by the banks
making up the data structure;
these records are called the {\bf bank material records}.
\index{FZ!record!bank material}
\par Banks of a data structure on a recording medium
are interconnected by links,
and the relocation table contained in the pilot record provides
the description of how to convert these links to their new values
in memory.
\par Data structures follow each other on the recording medium
in a sequential manner.
A series of data structures is often called an {\bf event}.
Events are seperated by {\bf start-of-event} and
{\bf end-of-event} system markers.
\index{FZ!event}
\index{FZ!marker!start-of-event}
\index{FZ!marker!end-of-event}
 
Events can be grouped in a {\bf run}. On the file these are
separated by {\bf start-of-run} and {\bf end-of-run}
marker records, which can also contain user information.
\index{FZ!marker!end-of-run}
\index{FZ!marker!start-of-run}
 
To avoid problems with hardware end-of-files,
a software simulation is provided by ZEBRA {\bf end-of-file}
records, which can carry user information.
\index{FZ!marker!end-of-file}
\section{File operations}
\Subr{CALL FZFILE (LUN,LREC,CHOPT)}
This routine should be called to declare a file to ZEBRA.
\index{FZ!input!declare file}
\index{FZ!output!declare file}
\Idesc
\begin{DL}{MMM}
\item[LUN]Logical unit number
\item[LREC]Record length (in {\bf words})\\
{\bf Native} file format - maximum logical record length
\begin{DL}{MM}
\item[=0:] Standard length: 2440 words
\item[>0:] User defined length
\end{DL}
{\bf Exchange} file format - physical record length
\begin{DL}{MM}
\item[=0:] Standard length: 900 words
\item[>0:] User defined length (must be a multiple of 90 words).
\end{DL}
\item[CHOPT]Character variable specifying the selected options.
\begin{DL}{MMMMMM}
\item[medium]{\tt' '} Disk (default),\\
{\tt'M'} Memory,                    \\
{\tt'T'} Magnetic tape
\item[file format]{\tt' '} Native (default),\\
{\tt'A'} Exchange format, ASCII mapping    \\
{\tt'X'} Exchange format, binary
\item[data format]for medium {\bf disk or tape:}
\begin{DL}{MM}
\item[' ']default - {\it 'native'} for {\it native} file format\\
or    {\it 'exchange'} for {\bf exchange} file format
\item['N']Native data format
\end{DL}
for medium {\bf memory}
\begin{DL}{MM}
\item[' ']default - 'native'
\item['X']exchange data format
\end{DL}
\item[processing]default - Input only
\item[direction]
\begin{DL}{MM}
\item['I']Input enabled
\item['O']Output enabled
\item['IO']Input/output enabled
\end{DL}
\item[EoF handling]
\begin{DL}{MM}
\item[' ']default - Chosen dependent on file and machine
\item['0']No hardware file marks
\item['1']Hardware file mark only for level 2 EoF.
\item['2']Hardware file marks both for level 1 and 2 EoFs.
\end{DL}
\item[various]
\begin{DL}{MM}
\item['R']Initial rewind
\item['Q']Quiet, print error messages only
(log level -2 with \Rind{FZLOGL}, see below).
\item['P']Permissive, enable error return
(valid only if the user code checks the error status).
\end{DL}
\end{DL}
\end{DL}
\subsection{FZFILE return code}
\index{QUEST!IQUEST}
\begin{DL}{MMMMMM}
\item[IQUEST(1)=0]Everything is all right
\item[IQUEST(1)=1]The file has already been initialized with \Rind{FZFILE}
\item[IQUEST(1)=2]LUN invalid
\item[IQUEST(1)=3]The requested format is not yet available on the given machine
\end{DL}
\subsection{Notes:}
Non-zero error returns are only enabled with the 'P' option,
otherwise control goes to \Rind{ZFATAL}. If the 'P' option is selected
the status must imperatively be checked by the user.
\par \Rind{FZFILE} does {\bf not} itself execute a FORTRAN
{\tt OPEN} statement in
order to leave the latter under the control of the user.
\subsection{Set the log level for an I/O unit number}
The logging level
(i.e. the verboseness of the messages of the ZEBRA system) can be
controlled for each input/output unit by a call to \Rind{FZLOGL}.
\Subr{CALL FZLOGL (LUN,LOGLEV)}
\index{FZ!logging level}
\index{logging level}
\Idesc
\begin{DL}{MMMM}
\item[LUN]Number of the input/output unit
for which the logging level has to be set
\item[LOGLEV]Logging level
\begin{DL}{MM}
\item[-3]Suppress all messages
\item[-2]Print error messages only
\item[-1]Terse mode
\item[ 0]Normal mode
\item[ 1]Normal mode and details of conversion problems
\item[ 2]Print to monitor calls to the FZ routines
\item[ 3]Short diagnostics
(Short dumps to debug user-written output routines).
\item[ 4]Full diagnostics
(Full dumps to debug user-written output routines)
\end{DL}
\end{DL}
\par
As indicated above, each declaration of an input/output file via
\Rind{FZFILE}
associates a logging level equal to the system-wide logging level
set be \Rind{MZEBRA}.
At any point in a program the logging level can be reset to a new
value by calling \Rind{FZLOGL} with the appropriate parameters.
\subsection{Limit the size of an output file}
The size of an output file can be limited by issuing a call to \Rind{FZLIMI}.
\par
The reason for having this facility is the fact that ``end-of-tape''
is a problem which cannot be solved satisfactorily in general.
To assist the user in controling reel switching,
ZEBRA counts the total number of words written,
and checks after every data structure written out
(but not for start-of-run, end-of-run, end-of-file)
whether the limit specified has been reached.
If so, it returns the ``pseudo end-of-tape'' condition (cf. \Rind{FZOUT})
for every data structure output until a new request is made
with {\tt ALIMIT=0}. Thus the user can switch tape, call
{\tt FZLIMI(LUN,0.)},
and continue to write another tape,
again waiting for the ``end-of-tape'' signal.
\Subr{CALL FZLIMI (LUN,ALIMIT)}
\index{FZ!limit output}
\Idesc
\begin{DL}{MMMM}
\item[LUN]Logical unit number
\item[ALIMIT]Floating point number giving the limit in mega-words
\begin{DL}{MM}
\item[>0]User defined limit
\item[=0]Increase the limit by the amount specified in a previous call
with {\tt ALIMIT > 0}
\item[<0]Unlimited (as initialized by \Rind{FZFILE})
\end{DL}
\end{DL}
\subsection{Examples:}
\par Set the file-size to 12.75 Mwords for unit 21
\begin{verbatim}
      CALL FZLIMI (21,12.75)
\end{verbatim}
Set the file-size to be the current data-volume plus 12.75 Mwords
\begin{verbatim}
      CALL FZLIMI (21,0.)
\end{verbatim}
\Subr{CALL FZRUN (LUN,NRUN,NUH,IUHEAD)}
\par writes a start-of-run or end-of-run record.
\index{FZ!start-of-run}
\index{FZ!end-of-run}
\Idesc
\begin{DL}{MMMM}
\item[LUN]Logical unit number
\item[NRUN]{\tt>0:} The run number\\
{\tt=0:} Increment the run number by 1 \\
{\tt<0:} End-of-run record
(ignored if the run is already terminated)
\item[NUH]Length of the user information vector (length can be zero)
\item[IUHEAD]Vector of length {\tt NUH}
containing the user information (integers only).
\end{DL}
\Subr{CALL FZENDI (LUN,CHOPT)}
\par
Terminates one or several input files.
\index{FZ!input!close file}
\Idesc
\begin{DL}{MMMM}
\item[LUN]Logical unit number.
If zero: All FZ input files
\item[CHOPT]Character variable specifying the selected options.
\begin{DL}{MM}
\item['O']Switch to output.
Permit write after read    \\
Needed in case the file
has been positioned for output by reading
\item['Q']Quiet, suppress printing of file statistics
\item['R']Final rewind
\item['T']Terminate, drop control-bank for this file
and print file statistics
\item['U']Unload file
\end{DL}
\end{DL}
\Subr{CALL FZENDO (LUN,CHOPT)}
\par Terminates one or several output files and by default
will write an end-of-run, will flush the output buffer in
exchange mode, and will leave the file positioned for further output.
\index{FZ!output!close file}
\Idesc
\begin{DL}{MMMM}
\item[LUN]Logical unit number.
If zero: All FZ output files
\item[CHOPT]Character variable specifying the selected options.
\begin{DL}{MM}
\item['E']Write end-of-file (unless done)
\item['E2']Write end-of-data (unless done)
\item['F']Flush the buffer only
\item['I']Switch to input.\\
Write end-of-data and rewind (if not yet done),
cancel the output permission.\\
Useful for reading back a file just written.
\item['Q']Quiet, suppress printing of file statistics
\item['R']Rewind (unless done)
\item['T']Terminate: write end-of-run (if not yet done).\\
Print the file statistics (unless done) and
drop the control-bank for this file.
\item['U']Unload file
\end{DL}
\end{DL}
\section{Output a data structure}
\Subr{CALL FZOUT (LUN,IXDIV,LENTRY,IEVENT,CHOPT,IOD,NUH,IUHEAD)}
Writes one data structure.
\index{FZ!output!data structure}
\Idesc
\begin{DL}{MMMMM}
\item[LUN]Logical unit number
\item[IXDIV]Index of the store/division containing the data structure.
May be a compound index (see \Rind{MZIXCO})
on page~\pageref{SR_MZIXCO}
for more details) if the 'D' option is selected.
\item[LENTRY]Entry address of the data structure.
It may be zero if the 'Z' option is selected
\item[IEVENT]Start-of-event flag. Possible values are
$0$ for event continued and $1$ for new event.
\item[CHOPT]Character variable specifying the selected options.
\begin{DL}{MM}
\item[' ']The data structure supported by the bank at {\tt LENTRY} is written
out (the next link is not followed)
\item['D']Complete division(s)\\
default: Dropped banks are squeezed out\\
\phantom{default: }(slower but maybe more economic than 'DI')
\item['DI']Immediate dump of divisions with dropped banks included
\item['L']Write the data structure supported by the linear structure at LENTRY
(the next link is followed)
\item['N']No links, i.e. linkless handling.
By default links are significant.
\item['S']Single bank at {\tt LENTRY}
\item['Z']Zero banks, i.e. empty data structure, header only
\end{DL}
\item[IOD]Description of the nature of the user header vector.
Either immediate type descriptors (see Table~\ref{MZFORMS})
or an I/O characteristic obtained by \Rind{MZIOCH}
\index{I/O characteristic}
(see Page~\pageref{MZIOCH})
\item[NUH]Number of words in the user header vector (may be zero)
\item[IUHEAD]The user header vector
\end{DL}
\subsection{FZOUT return codes}
\Rind{FZOUT} returns the write status in the QUEST vector as follows:
\index{QUEST!IQUEST}
\begin{DL}{MMMMMM}
\item[IQUEST(1)]Operation status code
\begin{DL}{MM}
\item[+1]``pseudo end-of-tape'' (cf. \Rind{FZLIMI})
\item[ 0]normal completion
\item[-1]first attempt to write after end-of-data
\item[-2]error return
\end{DL}
\item[IQUEST(11)]NWBK, number of words of bank material
\item[IQUEST(12)]NWTB, size of the relocation tables
\item[IQUEST(13)]number of pilot records written so far
\item[IQUEST(14)]number of words written (in millions)
\item[IQUEST(15)]number of words (up to 1 million) written so far
\item[IQUEST(16)]number of logical records written so far
\item[IQUEST(17)]Number of physical records written so far (exchange mode only)
\end{DL}
\section{Input a data structure}
To read the next data structure one calls \Rind{FZIN}.
The return code in the user communication vector
\index{QUEST!IQUEST}
{\tt IQUEST(1)} will tell the caller whether
the {\tt READ} operation was free of error, and whether the
object read was a data structure, or a start-of-run,
or an end-of-run,
or an end-of-file signal.
\Rind{FZIN} may be asked to skip to and then read the next start-of-event
data structure or the next start-of-run record.
\subsection{Read a single data structure}
\par In the simplest case (opt = '.' or blank) \Rind{FZIN} will read the next
data structure into the division indicated by the parameter {\tt IXDIV},
at the same time delivering the user header vector to the
parameter {\tt IQHEAD}.
\par The selective read has been provided to rapidly skip unwanted
data structures without expansion into memory and without relocation
of the links:
calling \Rind{FZIN} with opt='S' causes reading of the next pilot record
only, returning to the user the header vector for taking a
decision to read or skip the ``pending data structure''.
Skipping is done by asking for the next data structure;
accepting is done by calling \Rind{FZIN} with opt='A'.
\par {\bf Remember
that every call to FZIN has to be checked for the success of
the operation by testing on IQUEST(1).}
\subsection{Read a data structure by segment}
\par In the cases described so far the complete data structure is
read and is deposited into one particular division.
It is however possible to steer individual data segments of
the data structure into particular divisions,
or to cause them to be ignored.
(see the ZEBRA reference manual for more details)
\par The parameters {\tt LSUP} and {\tt JBIAS} in the calling sequence below
describe how the data structure should be inserted into
an already existing data structure. Their meaning
is similar to that for routine \Rind{MZLIFT} on page~\pageref{SR_MZLIFT}.
\Subr{CALL FZIN (LUN,IXDIV,*LSUP*,JBIAS,CHOPT,*NUH*,IUHEAD*)}
\index{FZ!FZIN}
\index{FZ!input!data structure}
\Idesc
\begin{DL}{MMMM}
\item[LUN]Logical unit number
\item[IXDIV]Index of the division to receive the data structure.\\
{\tt IXDIV = 0} means division 2 of the primary store
\item[*LSUP*]
\item[JBIAS]{\tt JBIAS < 1: LSUP} is the supporting bank
and {\tt JBIAS} is the link bias specifying where the data structure has to be
introduced into this bank, i.e. the data structure will be connected
to {\tt LQ(LSUP+JBIAS)}.\\
{\tt JBIAS = 1: LSUP} is the supporting link, i.e. the data structure
is connected to LSUP (top level data structure)\\
{\tt JBIAS = 2:} Stand alone data structure, no connection.
\item[CHOPT]Character variable specifying the selected options.
\begin{DL}{MMMM}
\item[Event]
\begin{DL}{MM}
\item[' ']default - go to next data structure
\item['E']go for next start-of-event data structure
\item['R']go for next start-of-run record
\end{DL}
\item[Select]
\begin{DL}{MM}
\item[' ']default - Read the next header and its data structure
(may mean: skip pending data structure or current event)
\item['A']Accept, read the pending data structure
({\tt NUH} and {\tt IUHEAD} are not used)
\item['S']Select, read the next header only
(may mean: skip pending data structure or current event)
({\tt LSUP} and {\tt JBIAS} are not used)
\end{DL}
\end{DL}
\item[*NUH*]Maximum size available for header vector {\tt IUHEAD}
\end{DL}
\Odesc
\begin{DL}{MMMM}
\item[*LSUP*]For {\tt JBIAS = 1 or 2, LSUP} receives
the entry address to the data structure
In any case {\tt IQUEST(13)} returns the entry address
into the data structure
\item[*NUH*]Useful size of the header vector stored in {\tt IUHEAD}
\item[IUHEAD*]User header vector
\end{DL}
\subsection{FZIN return codes}
\par \Rind{FZIN} returns the read status, either normal or error completion,
in the {\tt QUEST} vector as follows:
\index{QUEST!IQUEST}
\subsubsection{Normal read status returns are:}
\begin{DL}{MMMMMM}
\item[IQUEST(1)]Operation status code
\begin{DL}{MM}
\item[<0]error, see below
\item[ 0]normal completion
\item[ 1]start-of-run record
\item[ 2]end-of-run record
\item[ 3]ZEBRA end-of-file
\item[ 4]system end-of-file  (level 1 EoF)
\item[ 5]system end-of-data  (level 2 EoF)
\item[ 6]first attempt to read beyond EoD
\end{DL}
\item[IQUEST(2)]number of logical records read so far
\item[IQUEST(3)]number of physical records read so far (exchange mode)
\par
\item[IQUEST(11)]for {\tt IQUEST(1)=0:1 or 0} for yes/no start
new event \\
for {\tt IQUEST(1)=1:} run number for start-of-run
\item[IQUEST(12)]zero
\item[IQUEST(13)]{\tt LENTRY}, the entry address into the data structure.\\
Zero means: empty data structure.\\
(Not yet a valid address if return from 'S' option)
\item[IQUEST(14)]{\tt NWBK},
the number of words occupied by the data structure in memory.\\
Zero means: empty data structure
\item[IQUEST(20)]{\tt NWIOCH}, size of the IO characteristic
\item[IQUEST(21)...]{\tt NWIOCH} words of IO characteristic for the user
header vector
\end{DL}
\subsubsection{Error status returns are:}
\begin{DL}{MMMMMM}
\item[IQUEST(1)]Error status
\begin{DL}{MM}
\item[-8]  . . .
\item[-7]for 3 consecutive errors
\item[-6]for 2 consecutive errors
\item[-5]read error
\item[-4]bad constructs, probably not a file written by \Rind{FZOUT}
\item[-3]bad data
\item[-2]break in block sequence number (exchange mode only)
\item[-1]faulty call: A option given, but no pending data structure
\end{DL}
\item[IQUEST(2)]number of logical records read so far
\item[IQUEST(3)]number of physical records read so far (exchange mode)
\end{DL}
