%  $Header: /afs/.cern.ch/project/cnas_doc/sources/zebra/RCS/zebfz3.tex,v 1.1 1993/11/13 15:25:14 goossens Exp $

\Filename{H1-FZ-Usage-exchange-mode}
\chapter{Usage of FZ files in exchange mode}
\label{sec:H1FZ-exchange-mode}

In the examples of this chapter the default record size
for physical records is used, i.e. 900 words or 3600 bytes.
To mark this, the second parameter to \Rind{FZFILE} is given explicitely
as 900 (where zero would be enough).
One will probably want to use a different value,
especially for tape files,
in which case one has to change 900 and 3600 to the appropriate values.

The suggestions of this chapter are preliminary as it was not
possible to test all the cases individually.
People are kindly asked to mail their corrections for
this chapter to \Lit{zoll@cernapo.cern.ch}

\Filename{H2-FZ-Exch-format-representation}
\section{Exchange file format representation}

A true exchange-mode file consists of a stream of fixed-length
records without any system control-words;
such a file can be shipped between machines using 'ftp'
in binary mode.

Unfortunately, the Fortran implementations of several UNIX
machines cannot read or write such a file in sequential mode,
for this mode they insist on having sytem control-words
with every record.

On these machines,
such as Apollo, DECstation, HP Unix, Silicon Graphics, Sun,
one should use the direct-access mode, or possibly the C-Library mode,
selecting the \Ropt{D} or the \Ropt{L} option with \Rind{FZFILE}.

There is another possibility:
if on these machines one creates a Zebra file using sequential
Fortran WRITE, one gets a file of fixed-length records,
but with system control-words.
Such a file one can re-read with sequential READ, of course,
and one can ship it to another machine using the CERN utility \Rind{ZFTP},
which can produce the target copy with or without system-control words.
This is fine for sequential use of the file;
the problem remains that one cannot then read the same file
sometimes sequentially, sometimes with direct-access.

The preferred solution for theses machines is to write and read it
in direct-access mode for disk files,
in C Library mode for tape files.

And generally: use ZFTP rather than FTP, if you have it,
to ship files around, particularly if the target machine is VAX.

\Filename{H2-FZ-Tape-file-Fortran}
\section{Tape file, Fortran}

Tapes to be sent off-site should be \Lit{UNLABELLED},
because labels create nothing but trouble to the receiver.

Exchange-mode tape files cannot be handled with Fortran I/O
on several UNIX machines.
For these machines one has to use the \Ropt{L} mode,
reading through the C Library interface, see the next paragraph.

\subsubsection*{ALLIANT}

Assuming that the name of the tape drive is \Lit{/dev/rxt00h}:

Open the file and initialize FZ:

\begin{verbatim}
      OPEN (Lun, FILE='/dev/rxt00h', RECORDTYPE='FIXED'
     +,          RECL=3600, BLOCKSIZE=3600, FORM='UNFORMATTED')

      CALL FZFILE (Lun, 900, 'TX')     for input
   or CALL FZFILE (Lun, 900, 'TXO')    for output
\end{verbatim}

\subsubsection{CONVEX}

Assuming that the name of the tape drive is \Lit{/dev/mt12}:

Open the file and initialize FZ:

\begin{verbatim}
      OPEN (Lun, FILE='/dev/mt12', RECORDTYPE='FIXED'
     +,          RECL=3600, BLOCKSIZE=3600, FORM='UNFORMATTED')

      CALL FZFILE (Lun, 900, 'TX')     for input
   or CALL FZFILE (Lun, 900, 'TXO')    for output
\end{verbatim}

\subsubsection*{Apollo Aegis}

One may stage a file to or from disk with:

\begin{verbatim}
 tape to disk:  RWMT -R -UNLAB -RAW -F 1 -RL 3600 -BL 3600 pathname

 disk to tape:  RWMT -W -UNLAB -RAW -F 1 -RL 3600 -BL 3600 pathname
\end{verbatim}

If one has an on-line tape unit, one may connect the tape
to a \Lit{pathname} with

\begin{verbatim}
 EDMTDESC pathname -C -S LAB NO RF F BL 3600 RL 3600 ASCNL NO
\end{verbatim}

\subsubsection*{IBM MVS, input}

If the file is read with IOPACK on 'unit' 24:

To inform the system of the intention to use a tape drive
one should give right at the beginning of the JCL:

\begin{verbatim}
      /*UNIT   T6250=1       (or T1600)
\end{verbatim}

JCL for the file, if unlabelled:

\begin{verbatim}
      //G.IOFILE24 DD DSN=dsname,DISP=(SHR,KEEP),
      //            DCB=(RECFM=U,BLKSIZE=3600),
      //            UNIT=T6250,LABEL=(1,NL,,IN),VOL=SER=tapvsn
\end{verbatim}

Initialize FZ for this file:

\begin{verbatim}
      CALL FZFILE  (24, 900, 'TXY')
\end{verbatim}

\subsubsection*{IBM MVS, output}

To deprotect the tape start the JCL with:

\begin{verbatim}
      /*UNIT   T6250=1       (or T1600)
      //       EXEC  RING,TAPE=tapvsn
\end{verbatim}

JCL for the file, if unlabelled:

\begin{verbatim}
      //G.IOFILE24 DD DSN=dsname,DISP=(NEW,KEEP),
      //            DCB=(RECFM=U,BLKSIZE=3600),
      //            UNIT=T6250,LABEL=(1,NL,,OUT),VOL=SER=tapvsn
\end{verbatim}

Initialize FZ for this file:

\begin{verbatim}
      CALL FZFILE  (24, 900, 'TXYO')
\end{verbatim}

\subsubsection*{IBM VM/CMS}

To inform the system of the intention to use a tape drive
one should give right at the beginning of the JCL:

\begin{verbatim}
input:       SETUP    TAPE 181 tapevsn NL 6250 (END
           or SETUP    TAPE 181 tapevsn NL 38K  (END

output:      SETUP    TAPE 181 tapevsn NL 6250 RING (END
           or SETUP    TAPE 181 tapevsn NL 38K  RING (END
\end{verbatim}

Fortran:  JCL for file 24, say, if unlabelled:

\begin{verbatim}
   FILEDEF  24 TAP1 NL 1 (RECFM U BLKSIZE  3600 PERM
\end{verbatim}

Initialize FZ for this file:

\begin{verbatim}
      CALL FZFILE  (24, 900, 'TXF')    for input
  or  CALL FZFILE  (24, 900, 'TXFO')   for output
\end{verbatim}

\Rind{IOPACK}:  JCL for file 24, say, if unlabelled:

\begin{verbatim}
   FILEDEF  IOFILE24 TAP1 NL 1 (RECFM U BLKSIZE  3600 PERM
\end{verbatim}

Initialize FZ for this file:

\begin{verbatim}
      CALL FZFILE  (24, 900, 'TXY')    for input
 or   CALL FZFILE  (24, 900, 'TXYO')   for output
\end{verbatim}

\subsubsection*{VAX VMS, input}

Take out the write ring, mount the tape and give it a logical name with:

\begin{verbatim}
   normally:

     $ ASSIGN MTA0: zname
     $ ALLOC zname
     $ MOUNT zname/FOREIGN/DENS=6250/BLOCKSIZE=3600/RECORDSIZE=3600

   at CERN cluster VXCERN:

     $ SETUP/BLOCK=3600/RECORD=3600/NOLABEL  tapid vid zname
\end{verbatim}

Open the file and initialize FZ:

\begin{verbatim}
      OPEN (Lun, FILE='zname', STATUS='OLD'
     +,          FORM='UNFORMATTED', READONLY)

      CALL FZFILE (Lun, 900, 'TX')
\end{verbatim}

The specifications  \Lit{RECORDTYPE='FIXED',RECL=900,BLOCKSIZE=3600}
could be given,
but they are not needed as they are taken from the MOUNT.
(They must be specified on the MOUNT,
or else the file will not be read correctly.)

\subsubsection*{VAX VMS, output}

Put a write ring; assign and mount as for input:

\begin{verbatim}
   normally:

     $ ASSIGN MTA0: zname
     $ ALLOC zname
     $ MOUNT zname/FOREIGN/DENS=6250/BLOCK=3600/RECORD=3600

   at CERN cluster VXCERN:

     $ SETUP/WRITE/BLOCK=3600/RECORD=3600/NOLABEL  tapid vid zname
\end{verbatim}

Open the file and initialize FZ:

\begin{verbatim}
      OPEN (Lun, FILE='zname', STATUS='NEW', RECORDTYPE='FIXED'
     +,          RECL=900, BLOCKSIZE=3600, FORM='UNFORMATTED')

      CALL FZFILE (Lun, 900, 'TXO')
\end{verbatim}

\Filename{H2-FZ-Tape-file-CLibrary}
\section{Tape file, C Library}

This mode is available only on machines running under UNIX,
and only on machines where the CF package of KERNLIB has
been implemented.

Assume that the name of the tape drive is \Lit{/dev/mt12}.

\subsection*{Input}

Open the file and initialize FZ:

\begin{verbatim}
      CALL CFOPEN (IQUEST(1), 1, 900, 'r ', 0, '/dev/mt12 ', ISTAT)
      CALL FZFILE (Lun, 900, 'TL')
\end{verbatim}

\subsection*{Output}

Open the file and initialize FZ:

\begin{verbatim}
      CALL CFOPEN (IQUEST(1), 1, 900, 'w ', 0, '/dev/mt12 ', ISTAT)
      CALL FZFILE (Lun, 900, 'TLO')
\end{verbatim}

The record-length is given as the number of machine words per
record, thus '900' is for 32-bit machines;
on 64-bit machines this would be '450'.

Note the passing of the file-pointer returned from \Rind{CFOPEN}
to \Rind{FZFILE} via \Lit{IQUEST(1)}.

If you are running ZEBRA version 3.66 with KERNLIB constructed
from KERNFOR 4.26, note the following problem:

The CF routines delivered with KERNFOR 4.26 do not work
correctly for on-line tapes; they have been re-written
and version KERNFOR 4.27 has been released.

\Filename{H2-FZ-DF-Fortran-sequential}
\section{Disk file, Fortran sequential}

True exchange-mode disk files cannot be handled with
sequential Fortran I/O on several Unix machines.
For these machines one should use the D mode,
for Fortran direct-access, see the next paragraph.

\subsection*{ALLIANT}

For input, open the file and initialize FZ:

\begin{verbatim}
      OPEN (Lun, FILE='zname', STATUS='OLD', FORM='UNFORMATTED'
     +,          RECORDTYPE='FIXED', RECL=3600, BLOCKSIZE=3600)

      CALL FZFILE (Lun, 900, 'X')
\end{verbatim}

For output, open the file and initialize FZ:

\begin{verbatim}
      OPEN (Lun, FILE='zname', STATUS='UNKNOWN', FORM='UNFORMATTED'
     +,          RECORDTYPE='FIXED', RECL=3600, BLOCKSIZE=3600)

      CALL FZFILE (Lun, 900, 'XO')
\end{verbatim}

\subsection*{CONVEX}

For input, open the file and initialize FZ:

\begin{verbatim}
      OPEN (Lun, FILE='zname', STATUS='OLD', FORM='UNFORMATTED'
     +,          READONLY
     +,          RECORDTYPE='FIXED', RECL=3600, BLOCKSIZE=3600)

      CALL FZFILE (Lun, 900, 'X')
\end{verbatim}

For output, open the file and initialize FZ:

\begin{verbatim}
      OPEN (Lun, FILE='zname', STATUS='UNKNOWN', FORM='UNFORMATTED'
     +,          RECORDTYPE='FIXED', RECL=3600, BLOCKSIZE=3600)

      CALL FZFILE (Lun, 900, 'XO')
\end{verbatim}

\subsection*{IBM MVS, input}

If the file is handled with \Rind{IOPACK} on 'unit' 24:

JCL for the file:

\begin{verbatim}
//G.IOFILE24 DD DISP=SHR,DSN=gg.uuu.name
\end{verbatim}

Initialize FZ for this file:

\begin{verbatim}
      CALL FZFILE  (24, 900, 'XY')
\end{verbatim}

\subsection*{IBM MVS, output}

JCL for the file:

\begin{verbatim}
//G.IOFILE24 DD DSN=uu.ggg.name,DISP=(NEW,CATLG),
//            DCB=(RECFM=U,BLKSIZE=3600),
//            SPACE=(3600,800,RLSE),UNIT=SYSDA
\end{verbatim}

Initialize FZ for this file:

\begin{verbatim}
  CALL FZFILE  (24, 900, 'XYO')
\end{verbatim}

\subsection*{IBM VM/CMS}


To handle with Fortran, JCL for the file:

\begin{verbatim}
      FI 24 DISK fname ftype fmode (RECFM U LRECL 3600 BLKSIZE 3600 PERM
\end{verbatim}

Initialize FZ for this file:

\begin{verbatim}
      CALL FZFILE  (24, 900, 'XF')     for input
 or   CALL FZFILE  (24, 900, 'XFO')    for output
\end{verbatim}


To handle with \Rind{IOPACK}, JCL for the file:

\begin{verbatim}
   FILEDEF  IOFILE24 DISK fname ftype fmode (RECFM U BLKSIZE 3600 PERM
\end{verbatim}

Initialize FZ for this file:

\begin{verbatim}
      CALL FZFILE  (24, 900, 'XY')     for input
 or   CALL FZFILE  (24, 900, 'XYO')    for output
\end{verbatim}

The file mode of a Zebra exchange file should be 1,
thus one might give A1 for the 'fmode' parameter.

\subsection*{VAX VMS, input}

Open the file and initialize FZ:

\begin{verbatim}
      OPEN (Lun, FILE='zname', STATUS='OLD'
     +,          FORM='UNFORMATTED', READONLY)

      CALL FZFILE (Lun, 900, 'X')
\end{verbatim}

\subsection*{VAX VMS, output}

Open the file and initialize FZ:

\begin{verbatim}
      OPEN (Lun, FILE='zname', STATUS='NEW', RECORDTYPE='FIXED'
     +,          RECL=900, BLOCKSIZE=3600, FORM='UNFORMATTED')

      CALL FZFILE (Lun, 900, 'XO')
\end{verbatim}

Such a file created on the VAX has these properties:

\begin{verbatim}
   VxCrnA$ DIR FZXVAX.DAT.* /FULL

   Directory disk:[uuu]

   FZXVAX.DAT;1                  File ID:  (19177,30,0)
   Size:          240/240        Owner:    [L3_1,uuu]
   Created:  31-MAY-1988 12:03   Revised:  31-MAY-1988 12:11 (2)
   Expires:   <None specified>   Backup:    6-JUN-1988 07:18
   File organization:  Sequential
   File attributes:    Allocation: 240, Extend: 0, Global buffer count: 0
                       No version limit
 ! Record format:      Fixed length 3600 byte records
 ! Record attributes:  None
   Journaling enabled: None
   File protection:    System:RWED, Owner:RWED, Group:RE, World:RE
   Access Cntrl List:  None
\end{verbatim}

Note the parameters marked by '!' on the left margin.

If a file acquired with FTP on a VAX does not have these properties,
one could fix this with this little COM file:

\begin{verbatim}
      $ SET NOVERIFY     !  RESIZE.COM   900724 12.00
      $ ON ERROR     THEN $ GOTO EXIT
      $ ON CONTROL_Y THEN $ GOTO EXIT
      $!
      $!   COM-file to re-size FTP files
      $!
      $ IF (P1 .EQS. "") THEN
      $   INQUIRE P1 "Enter UNIX file name"
      $   INQUIRE P2 "Enter VMS  file name"
      $   INQUIRE P3 "Give record size in bytes(<CR>=3600)"
      $   IF (P3 .EQS. "") THEN P3 = 3600
      $ OPEN/WRITE OUTP EXCHQZZZ.DAT
      $ WRITE OUTP    "RECORD"
      $ WRITE OUTP    "BLOCK_SPAN              yes"
      $ WRITE OUTP    "CARRIAGE_CONTROL        none"
      $ WRITE OUTP    "FORMAT                  fixed"
      $ WRITE OUTP    "SIZE                    ''P3'"
      $ CLOSE OUTP
      $ EXCHANGE/NETWORK 'P1 'P2 -
              /TRANSFERT=BLOCK -
              /FDL=EXCHQZZZ.DAT
      $EXIT:
      $ DELETE/NOCONF/NOLOG EXCHQZZZ.DAT.*
      $ EXIT
\end{verbatim}

\Filename{H2-FZ-DF-Fortran-d-a}
\section{Disk file, Fortran direct-access}

This mode works on all machines, except IBM VM.

Note that one may create a true exchange-format file with
sequential WRITE on some machine, and read it with
direct-access on this or another machine.

\subsection*{Input}

Open the file and initialize FZ:

\begin{verbatim}
      OPEN (Lun, FILE='zname', STATUS='OLD', FORM='UNFORMATTED'
     +,          ACCESS='DIRECT', RECL=3600)

      CALL FZFILE (Lun, 900, 'D')
\end{verbatim}

Key-word \Lit{READONLY} should be given on CONVEX, VAX, DECstation.

\subsection*{Output}

Open the file and initialize FZ:

\begin{verbatim}
      OPEN (Lun, FILE='zname', STATUS='NEW', FORM='UNFORMATTED'
     +,          ACCESS='DIRECT', RECL=3600)

      CALL FZFILE (Lun, 900, 'DO')
\end{verbatim}

On most machines the record-length is given in bytes.

On VAX and DECstation one must specify words: \Lit{RECL=900}

\Filename{H2-FZ-Disk-file-CLibrary}
\section{Disk file, C Library}

This mode is available only on machines running under UNIX,
and only on machines where the CF package of KERNLIB has
been implemented.

\subsection*{Input}

Open the file and initialize FZ:

\begin{verbatim}
      CALL CFOPEN (IQUEST(1), 0, 900, 'r ', 0, 'zname ', ISTAT)
      CALL FZFILE (Lun, 900, 'L')
\end{verbatim}

\subsection{Output}

Open the file and initialize FZ:

\begin{verbatim}
      CALL CFOPEN (IQUEST(1), 0, 900, 'w ', 0, 'zname ', ISTAT)
      CALL FZFILE (Lun, 900, 'LO')
\end{verbatim}

The record-length is given as the number of machine words per
record, thus '900' is for 32-bit machines;
on 64-bit machines this would be '450'.

Note the passing of the file-pointer returned from \Rind{CFOPEN}
to \Rind{FZFILE} via \Lit{IQUEST(1)}.
