% 13 November 1993 mg
\Filename{H1-MZ-special-applications}
\chapter{MZ routines for special applications}
\label{sec:H1-MZ-routines-for-special-applications}

\Filename{H2-MZCOPY-copy-d-s}
\section{MZCOPY - copy a data-structure}

\Rind{MZCOPY} allows one to copy a data-structure from memory to memory.
\begin{verbatim}
The source memory may be

  either (case 1) one or several divisions in some store F;

      or (case 2) ordinary stand-alone memory,
                  called 'flat memory' below.

The target memory may be

  either (case 3) one particular division in a store T, which may
                  or may not be the same as store F;

      or (case 4) ordinary stand-alone memory.
\end{verbatim} 

For case 1, the data-structure may be identified in the usual way
as the collection of banks which depend structurally
from the header bank.
This (option \Ropt{D} not selected) requires a logical walk through
the structure,
marking each bank which belongs to it by setting a system bit
in the status word.
A sequential scan over the memory between the first and the last bank
involved constructs then the table of the material to be copied.
(This way of identifying the d/s must not be used with
a read-only store.)

Alternatively, the d/s may be identified as the contents
of one or several divisions,
in which case the user has to be certain that the complete contents
of this division(s) belong to one single d/s.
This (option \Ropt{D} selected) does not require the logical walk,
and is by that much faster.
Moreover, the copy may or may not be immediate (option I):
if yes (option \Ropt{DI}) dead banks present in the division(s) are
also copied, if not (option \Ropt{D} without \Ropt{I}) a sequential scan
over the division(s) concerned constructs a more detailed
table of all the sectors of contiguous live banks in the division(s).
Thus, if the user knows a priori that the amount of dead material
is negligeable he should select option \Ropt{DI} for speed,
saving the sequential scan.
However, if the amount of dead material is significant,
he should select option \Ropt{D} only,
again for speed saving the time to copy the useless material.

Cases 2 and 4 (flat memory) are intendend for use in communication
between separate processes running on the same computer
through shareable Common blocks:
the sending process places a d/s into a flat memory area,
the receiving process copies it from there into its Zebra store.
Again, the data must all belong to one single d/s.

If the No-link option is selected the copied result
is a simple collection of banks,
with all links except link 0 reset to zero,
connected to be a linear structure;
the entry address returned is the address of
the physically first bank.


\subsection*{Copy from and to Zebra storage:}

\Shubr{MZCOPY}{(IXDVFR,!LENTRY, IXDVTO,*!LSUP*,JBIAS, chOPT)}

with
\begin{XMP}
     IXDVFR:  index of division(s) to be copied from
               may be zero [or IXSTOR] if the D option is not selected
               may be a compound index if the D option is selected

    !LENTRY:  entry address of the d/s

     IXDVTO:  index of the division to be copied to;
               this must be the index of one particular division

            The d/s copied is linked into a pre-existing d/s
            as directed by LSUP and JBIAS, which have the same
            significance as for MZLIFT:

      !LSUP:  if JBIAS < 1: LSUP is the supporting bank,
      JBIAS:                 connection to link LQ(LSUP-JBIAS)

               if JBIAS = 1: LSUP is the supporting link,
                              connection to LSUP (top-level d/s)
                              LSUP* returns the entry adr to the d/s

               if JBIAS = 2: stand-alone d/s, no connection
                              LSUP* returns the entry adr to the d/s

      chOPT:  character string,
               individual characters select options:

            d/s:    by default the d/s supported by the bank at
                     LENTRY is copied (link 0 not followed)

                   L copy the d/s supported by the linear structure
                     at LENTRY (link 0 followed)

                   M  copy the banks pre-marked by the user,
                      see section{sec:MZFLAG} for details
                      (also: section~\ref{sec:FZmarking}, User marking ...)

                   D  copy complete division(s)
                      default: dead banks are squeezed out
                               (slower but maybe more economic than DI)
                   DI immediate dump of division(s),
                      dead banks, if any, are also copied

                   S copy the single bank at LENTRY

                    see separate descriptions below for

                   F   copy from flat memory
                   T   copy   to flat memory

         others:   N no links, ie. linkless handling,
                     default: links are significant

                   P permit error returns
                     default: error exit is with CALL ZTELL (15,1)

                   Z zero all links pointing outside the d/s
                     (this is implied if store T is not store F)
\end{XMP} 

\subsection*{Copy from flat memory:}

\begin{verbatim}
      IQUEST(1) = NWFR:    number of words
      IQUEST(2) = JOFFFR:  link relocation off-set, usually zero
\end{verbatim} 

\Shubrz{MZCOPY}{(MEMFR,!LENTRY, IXDVTO,*!LSUP*,JBIAS, chOPT)}

with
\begin{verbatim}
      MEMFR:   the words MEMFR(1:NWFR) hold the d/s to be copied,
               links in the d/s, if any, assume addressing such that
               MEMFR(L) corresponds to LQ(L+JOFFFR)

      In this case options L, D, I, S are irrelevant.
\end{verbatim} 

\subsection*{Copy to flat memory:}

\begin{verbatim}
      IQUEST(3) = NWMXTO:  maximum # of words, ie. capacity of MEMTO
      IQUEST(4) = JOFFTO:  link relocation off-set, usually zero
\end{verbatim} 

\Shubrz{MZCOPY}{(IXDVFR,!LENTRY, MEMTO*,LENTO*,NWTO*, chOPT)}

\begin{verbatim}
with
     MEMTO*:   the words MEMTO(1:NWTO) receive the d/s copied,
               NWTO is not allowed to exceed NWMXTO,
               links in the d/s, if any, will be relocated such that
               MEMTO(L) corresponds to LQ(L+JOFFTO)
     !LENTO*   is the entry address into the copied d/s.
\end{verbatim} 

\subsection*{Error return}

\Lit{IQUEST(1/2)} will signal on return the status of the copy:
\begin{verbatim}
        = 0:  all is well,  IQUEST(2) = number of words
   and if option P given:
        = 1:  LENTRY invalid
        = 2:  bank chaining clobbered found in the input d/s
        = 3:  not enough space to construct the relocation table
        = 4:  d/s larger than the target space
        = 5:  d/s to be copied is empty
        = 6:  bank chaining clobbered found in the copied d/s
\end{verbatim} 

\Filename{H2-MZNEED}
\section{MZNEED - calculate free space in division}

This routine measures the free space available in a particular
division, either the space available immediately
or the space after garbage collection.

\Shubr{MZNEED}{(IXDIV, NEED, chOPT)}

with
\begin{verbatim}
      IXDIV     index of the division

      NEED      number of words needed, trigger garbage collection
                if G option selected and less than NEED words available

      chOPT     character string of options:

                G  execute garbage collection
                   if less than NEED words available

On return /QUEST/IQUEST(100) contains the following information
about the division IXDIV:

   IQUEST(11)   NEX, the number of free words beyond NEED,
                negative if less than NEED words available,
                ie. the total available is NEED+IQUEST(11);

   IQUEST(12)   the space occupied by the banks, including dead banks
                if any, of the division;

   IQUEST(13)   the maximum space allowed for this division,
                as given to MZDIV.
\end{verbatim} 

\Example

Suppose \Rarg{IXDIV} is a forward division, and the user wants to
take all its space into one single maximum size bank,
which will be filled by a variable amount of data, but less
than \Rarg{NEED} words; when the filling is complete the bank
will be reduced to its true size:

\begin{verbatim}
      NEED = 8000
      CALL MZNEED (IXDIV, NEED, 'G')
      IF (IQUEST(11).LT.0)              GO TO no luck

      ND = NEED + IQUEST(11) - 19       (7 links, 12 system words)
      CALL MZBOOK (IXDIV,L,LSUP,JB, 'name',7,2,ND,2, -1)

      fill words IQ(L+1) TO IQ(L+NU)

      CALL MZPUSH (IXDIV,L,0, NU-ND, 'R')
\end{verbatim}

\Filename{H2-MZINQS}
\section{MZINQS - inquiry to get the parameters of a store}

This routine returns the parameters of the selected store in \IQUEST:

\Shubr{MZINQS}{(IXSTOR)}

\begin{verbatim}
with     IXSTOR     index of the store

On return /QUEST/IQUEST(100) contains the following information:

    IQUEST(1):  opmode   operation mode of the store, = zero normally
    IQUEST(2):  NQSTRU   number of structural permanent links
    IQUEST(3):  NQREF    number of permanent links
    IQUEST(4):  NQLINK   number of permanent + working space links
    IQUEST(5):  LSTA1    first word after the working space
    IQUEST(6):  LSTA21   first word after the store
    IQUEST(7):  NQMINR   minimum size of the reserve area after div. 1
    IQUEST(8):  LQ2END   minimum value of LQEND(2)
    IQUEST(9):  JQDVLL   last low division
   IQUEST(10):  JQDVSY   system division, also: first high division
   IQUEST(11):  NQLOGL   current log level for this store
   IQUEST(12):  NQSNAM   chars 1:4 of the printing name of the store
   IQUEST(13):           chars 5:8
   IQUEST(14):  lowORG   lowest  possible origin-adr for the store
   IQUEST(15):  highORG  highest possible origin-adr for the store
   IQUEST(16):  initalc  initial allocation to last division created
   IQUEST(17):  toffset  offset of the system tables
\end{verbatim} 
The permanent links occupy  LQ(1) to LQ(NQREF).
\begin{verbatim}
The working space links go from LQ(NQREF+1)  to LQ(NQLINK),
the working space data  go from LQ(NQLINK+1) to LQ(LSTA1-1).
\end{verbatim} 
Divsion 1 starts at \Lit{LSTA1}.

Bank space goes from \Lit{LQ(LSTA1)} to \Lit{LQ(LSTA21-1)}.

There must always be a gap of at least \Lit{NQMINR} words between divisions 1 and 2.

\Lit{LQ2END} is the lowest possible address for the end of division 2.

The existing short range divisions have numbers 1 to \Lit{JQDVLL}.

The system division has the number \Lit{JQDVSY} at the moment,
this will change if \Rind{MZDIV}is called to create a new long-range division.
If \Lit{JQDVSY} is less than 20 the user's long range divisions have numbers 
\Lit{JQDVSY+1} to 20.

No 'origin link' may have a value less than \Lit{lowORG} or larger than \Lit{highORG},
both numbers could be negative; they depend on where the link-areas are
physically in absolute memory, relative to the store.

For a secondary store the system tables are at the end of
the memory declared to \Rind{MZSTOR}, they start in \Lit{LQ(toffset)}.
For the primary store \Lit{toffset=0} is returned.

At log-level 2 the information returned is also printed by \Lit{MZINQS}.

\Filename{H2-MZINQD}
\section{MZINQD - inquiry to get the parameters of a division}

This routine returns the parameters of the selected division in \IQUEST:

\Shubr{MZINQD}{(IXDIV)}

\begin{verbatim}
with     IXDIV     index of the division

On return /QUEST/IQUEST(100) contains the following information
about the division IXDIV:

    IQUEST(1):  mode = 0 forward, 1 reverse division
    IQUEST(2):  kind = 1 short range, 2 long, 4 package, 8 system
    IQUEST(3):  start adr LSTA
    IQUEST(4):  end   adr LEND
    IQUEST(5):  LEXTR highest or lowest possible value of LEND or LSTA
    IQUEST(6):  initial size declared to MZDIV
    IQUEST(7):  maximum size declared to MZDIV
    IQUEST(8):  char 1:4 of the printing name of the division
    IQUEST(9):  char 5:8
   IQUEST(10):  JDVSH, if non-zero: sharing neighbour division

This means:

The memory actually occupied by the division goes
from LQ(LSTA) to LQ(LEND-1), ie. LEND-LSTA words.

The reserve space immediately available to the division would go
   from  LQ(LEND)  to LQ(LEXTR-1) for a forward division,
or from  LQ(LEXTR) to LQ(LSTA-1)  for a reverse division (mode=1).
\end{verbatim} 

If \Lit{JDVSH} is non-zero this division shares the reserve area with
the division designated by \Lit{IXDIV}.

At log-level 2 the information returned is also printed by \Rind{MZINQD}.

\Filename{H2-MZDRED}
\section{MZDRED - reduce the space reserved to a division}

The situation can arise that one needs a lot of space in some
long-range division during the initialization phase,
which is then abandoned.
Since divisions can only grow, but never shrink, this space
could be blocked for the whole run.

To reduce this space of a particular division
to the initial allocation for for this division
(or to its present occupation, whichever is larger),
one should first 
\Lit{CALL \Rind{MZGARB} (IXDIV)}, to collect dead banks,
followed by

\Shubr{MZDRED}{(IXDIV)}

\begin{verbatim}
with  IXDIV     index of the division
\end{verbatim} 

This will shift all divisions below division \Lit{IXDIV} upwards
by the amount of space liberated.
