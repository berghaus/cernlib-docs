%  $Header: /afs/.cern.ch/project/cnas_doc/sources/zebra/RCS/zebfz4.tex,v 1.3 1995/02/02 08:53:11 goossens Exp $
\Filename{H1-FZ-Format-specifications}
\chapter{Format specifications}
\label{sec:H1-FZ-format-specs}

\Filename{H2-FZ-native-mode}
\section{Native mode}

In native mode
a data-structure is represented by a series of logical records (LR).
These records are written and read by Fortran \Lit{WRITE} and \Lit{READ}
statements.

Since indefinitly large records would cause problems in
auxiliary programs,
\Rind{FZOUT} will make sure that no record exceeds the maximum
logical record size declared for the file with \Rind{FZFILE},
breaking a chunk of information into several records if necessary.
Independently, the pilot record may not exceed 1024 words.

A data-structure starts with the 'pilot' record (type 2 or 3),
\index{pilot record}\index{record!pilot}%
which may be continued with 'pilot continuation' records (type 4)
\index{pilot continuation record}\index{record!pilot!continuation}%
if necessary; this is followed by as many 'bank material' records
\index{bank material record}\index{record!bank material}%
as needed (type 7 for all except the last which has type 8).

\subsection*{Native mode, logical records: general format}

\begin{verbatim}
word -1:  NWLR, the number of words in this record,
            excluding the first two (!) words
            (except for padding records in exchange file format)

 word 0:  logical record type LRTYP:

            LRTYP= 1: record 'start or end of run'

                   2: pilot record for d/s 'start of event'

                   3: pilot record for d/s 'event continued'

                   4: pilot continuation record

              5 or 6: padding records, ignored on input,
                       (not used in native mode)

                   7: bank material record, with more to come

                   8: last bank material record of a d/s

                   9: emergency-stop record

words 1 -> NWLR:  data words of the logical record
\end{verbatim}

\subsection{Native mode, logical records: start-of-run and end-of-run}

\begin{verbatim}
word -1:  NWLR, the record length

      0:  LRTYP=1, the record type

      1:  NRUN, the run number, a positive integer
          NRUN=0 signals 'end-of-run'

      2 -> NWLR:  user information, integers only
\end{verbatim}

A record \Lit{LRTYP=1} with \Lit{NRUN} positive may be used to mark the beginning
of a run; this is not obligatory.

A record \Lit{LRTYP=1} with \Lit{NRUN=0} may be used to signal the end of a run.

\subsection*{Native mode, logical records: the pilot record}

\begin{verbatim}
word -1:  NWLR, record length limited to NWLR < 1020

      0:  LRTYP = 2 or 3, the record type

      1:  the floating point number 12345.0 as a check word

      2:  Zebra version number, integer = 10000. * QVERSIO
           zero if written by a user routine

      3:  processing option bits

      4:  zero (reserve word)

      5:  NWTX, number of words in the text-vector, may be zero

      6:  NWSEG, number of words in the segment table,
           = 3 * number of table entries

             if NWSEG=0: no table, but not necessarily empty d/s

      7:  NWTAB, number of words in the relocation table,
           = 2 * number of table entries

             if NWTAB=0:  linkless data-structure

      8:  NWBK, total number of words occupied by the bank material,
           both on memory and on the file, may be zero if empty d/s

      9:  LENTRY, entry address into the d/s

     10:  NWUHIO = NWUH + NWIO, number of words in the user header
           vector plus its I/O characteristic, zero if no header

           11: control-word for the I/O characteristic
                either: immediate, NWIO=1,
                         the whole vector is of the same type
                         = 1  all bits
                           2  all integer
                           3  all floating
                           7  self-decriptive

                    or: complex, start of the I/O characteristic
                         as delivered by MZIOCH, NWIO words in all

           11 + NWIO: start of the user header vector,
                       NWUH < 401 words

     11+NWUHIO:  start of the segment table, NWSEG words

     11+NWUHIO+NWSEG:  start of the text-vector, NWTX words

     11+NWUHIO+NWSEG+NWTX: start of the relocation table, NWTAB words
\end{verbatim}

\subsection{Native mode, logical records: the pilot continuation records}

If the text vector is too large to fit into the pilot record
it is written out as a separate \Lit{LRTYP=4} continuation record:

\begin{verbatim}
       word -1:  NWTX
             0:  LRTYP=4
             1:  start of the text vector, NWTX words
\end{verbatim}

The relocation table is written as a separate \Lit{LRTYP=4} continuation
record (or even several such records in the very unlikely case
that the table is larger than the maximum record length) if:

\begin{itemize}
\item text vector did not fit into the pilot record
\item table is too large to join the pilot record
\item table is longer than 40 words (because the buffer space for
      the 'early table words' in the control-bank associated to the
      file has this size) as follows:
\begin{verbatim}
      word -1:  NWTAB
            0:  LRTYP=4
            1:  start of the relocation table, NWTAB words
\end{verbatim}
\end{itemize}

\subsection*{Native mode, logical records: the bank material records}

\begin{verbatim}
word -1:  NWLR, the number of words in this record,
            excluding the first two words

      0:  LRTYP = 7 or 8, record type

      1 -> NWLR:  the bank material
\end{verbatim}

The bank material is dumped directly in sequential order
into either one or several records;
all but the last record are given the type \Lit{LRTYP=7},
the last has \Lit{LRTYP=8}.

\subsection*{Pilot: The segment table}

Each entry has 3 words:

\begin{itemize}
\item    2 words are the Hollerith name of the division of origin,
\item    1 word indicates the number of words of bank material for 
         this segment.
\end{itemize}

The table on the file gives first the names of the divisions
for the segments, followed by the size indicators;
in both cases in the order in which the segments occur on the file,
which is also the order in which they where in memory at \Rind{FZOUT} time.

For example: suppose data from two divisions,
called \Lit{'BGO'} and \Lit{'CALORIM'}, where written.
Division \Lit{BGO} was physically lower in the store at output time
and delivered 21469 words, whilst division \Lit{CALORIM} gave 9199 words.
The segment table for this would consist of six words:

\begin{verbatim}
         4HBGO , 4H    , 4HCALO, 4HRIM , 21469, 9199
\end{verbatim}

If all the bank material comes from one single division only
the segment table is suppressed.

\subsection*{Pilot: The text vector}

This is stored as a series of lines,
each line is preceded by a word-count indicating the number of
4-character Hollerith words to follow (this corresponds to the
\Rind{MZIOCH} format \Lit{'/*H'}).

\subsection*{Pilot: The relocation table}

There is one entry for each contiguous section of memory,
containing an integral number of banks, which has been written
to the file. The entry consists of the original start address
of the section, followed by the end+1 address.
The entries follow each other in the table, thus:

\begin{verbatim}
       word 1:  start adr of first storage region
            2:  end+1 adr of first storage region
            3:  start adr of second storage region
                 ...           ...
        NWTAB:  end+1 adr of last storage region
\end{verbatim}


\Filename{H2-FZ-exchange-mode}
\section{Exchange mode}

The exchange mode provides an interface between computers of
different architecture, through which data-structures can travel.
This interface has two distinct components:

\begin{itemize}
 \item The {\bf exchange data format} defines the representation
       of the data for the data-types allowed with ZEBRA:
       bit pattern, integer, floating, double-precision, Hollerith.
 \item The {\bf exchange file format} defines the organisation of
       the data into logical records,
       and their blocking into physical records.
\end{itemize}

Normal blocking methods (e.g., IBM's VBS)
use control-information associated to every phy\-si\-cal record;
for good reasons, because this provides safe and easy recovery to the
next logical record in case a physical record is lost.
In ZEBRA this could not be used because the intention was to provide
a format which would allow data-taking computers to dump rapidly
large amounts of data directly from memory to the file,
without copying through a buffer and without distroying the data
in the store,
which implies the use of physical records without
any control-information.
Such physical records we call 'fast blocks'.
We do however need a different kind of physical record,
which does provide control information,
and which can be used for recovery.
Such blocks are called 'steering blocks'.
Any data-structure starts on a steering block and may continue on as
many fast blocks as needed,
their number being recorded in the control-information of the
steering block.

To permit recovery, the steering blocks carry a particular bit
pattern on the first 4 words,
i.e. a pattern of 128 bits.
This reduces the a priori chance of a fast block
simulating a steering block to 1 in \(4.*10^{38}\).
This is further reduced by contraints on the contents of the control
information for the steering block.
(To keep this probability low the user must not deliberately
use elements of this pattern as signals,
such as end markers.)

Some copy utilities copy a magnetic tape with suppression of faulty
blocks.
If a fast block of a particular data-structure is lost in this way,
\Rind{FZIN} would read the copy without getting an error signal
for the fast block,
and it would try to absorb the first block after the d/s,
a steering block, using it as data.
To prevent this, \Rind{FZIN} is programmed never to accept as fast
a block which looks like a steering block.
If a d/s contains a genuine fast block simulating a steering block,
it will be lost.
The chance of this happening is sufficiently low to be tolerable,
as shown above.

A file to be read by \Rind{FZIN} is not necessarily the result of \Rind{FZOUT},
it may have been produced by a user program,
and this may not yet be fully debugged.
FZIN has been programmed to catch hopefully all logical errors
possible in the user's output routines,
and to give full diagnostics with logging level 4.

Whilst files in native mode can be written and read by identical
code on all machines,
specific code is provided for each type of machine
to read and write files in exchange mode.
Idiosyncracies of any given machine (like the ones of the VAX)
must remain local to the specific code of that machine.
This is an important principle,
as it avoids the combinatorial problem of every machine
having to know about all other machines.
Also,
it provides for a clean solution to the advent of a new machine,
or the disappearance of an obsolete machine.

\Filename{H2-FZ-Exchange-Data-format}
\section{Exchange Data format}

The unit of information is a word of 32 bits,
(double-precision numbers require two words each).
A word may contain:

\begin{verbatim}
 - positive integer:  true binary
 - negative integer:  two's complement of the absolute value
 - floating point:    IEEE standard as shown below
 - Hollerith:         4 character 8-bit ASCII, stored left to right
 - bit pattern:       32 bits of zero or one (no conversion)
\end{verbatim}

The following reduction may occur when transforming
to the exchange representation on machines with a word-size
larger than 32 bits:

\begin{verbatim}
 - integer:     'bad conversion' if more than 31 significant bits
 - floating:    reduced precision, but correctly rounded
 - Hollerith:   only the first 4 characters are taken
                 (trailing characters should be blank anyway)
 - bit pattern: only the right-hand 32 bits are taken
\end{verbatim}

The following expansions may occur for the inverse transformation:

\begin{verbatim}
 - Hollerith:   blank fill
 - bit pattern: zero-fill of the leading bits
\end{verbatim}

The IEEE representation of
single and double precision floating-point numbers is:

\begin{verbatim}
Single precision:  1  bit  32       the sign-bit 's'
                   8  bits 31 - 24  the characteristic 'c'
                  23  bits 23 -  1  the mantissa 'm'

Double precision:  1  bit  64       the sign-bit 's'
                  11  bits 63 - 53  the characteristic 'c'
                  52  bits 52 -  1  the mantissa 'm'
\end{verbatim}

A normal floating point number is represented by 3 fields
on one or two 32-bit words:

\Lit{'m'} is the fractional part of the mantissa M;
the binary point is assumed to be immediately to the left
of the most significant bit of m,
preceded by an implied 1 (hidden bit), for example:

\begin{verbatim}
         m             M mantissa       decimal value

         0000...0      1.0000...0       1.
         1000...0      1.1000...0       1.5
         1100...0      1.1100...0       1.75
\end{verbatim}

\Lit{'c'} is the characteristic, being the binary exponent with a bias of
127 or 1023 added, for example:

\begin{verbatim}
         c             exponent        c              exp

         011...101     -2              100...000      1
         011...110     -1              100...001      2
         011...111      0              100...011      3
\end{verbatim}

A normal floating-point number does not have a c of all ones
or all zeros, such characteristics are used to represent Exceptions.

\Lit{'s'} is the sign-bit, the number is negative if this bit is on.
The representation of floating-point numbers is of the
sign/magnitude form, ie. only the sign-bit is complemented
to change the sign of the number.

The Exceptions are:
\begin{verbatim}
   c=0, m=0            floating-point zero
   c=0, m= not 0       'de-normalized number',
                       value:   +/-  0.m  *  2**(-126 or -1022)
   c=all 1, m=0        signed infinity  (c=255 or 2047)
   c=all 1, m= not 0   'not a number (NaN)'
\end{verbatim}

\Filename{H2-FZ-Exchange-File-Format}
\section{Exchange File Format}

The logical records of the exchange file format look much like those
of the native mode,
except that a data-structure may be represented by a single LR
of type 2 or 3.
LR type 4 is used for bank material continuation,
LR types 7 and 8 do not exist.

\subsection*{Exchange File Format, logical records: general format}

\begin{verbatim}
word -1:  NWLR, the number of words in this record,
          excluding the first two (!) words, except:
          excluding the first word only for padding records
          for NWLR = 0  the LR type 5 is implied

 word 0:  logical record type LRTYP:

            LRTYP= 1: record 'start or end of run'

                   2: (start of) data-structure 'start of event'

                   3: (start of) data-structure 'event continued'

                   4: data-structure continued

              5 or 6: padding records, ignored on input,

words 1 -> NWLR:  data words of the logical record
\end{verbatim}

\subsection*{Exchange File Format, logical records: start-of-run and end-of-run}

\begin{verbatim}
word -1:  NWLR, the record length

      0:  LRTYP=1, the record type

      1:  NRUN, the run number, a positive integer
          NRUN = 0  signals 'end-of-run'
          NRUN =-1  signals 'Zebra end-of-file'

      2 -> NWLR:  user information, integers only
\end{verbatim}

A record \Lit{LRTYP=1} with \Lit{NRUN} positive may be used to mark the beginning
of a run; this is not obligatory.

A record \Lit{LRTYP=1} with \Lit{NRUN=0} may be used to signal the end of a run.

\subsection*{Exchange File Format, logical records: (start of) data-structure}

\begin{verbatim}
word -1:  length NWLR = 10 + NWIO + NWUH + NWSEG + NWTX + NWTAB
                           + NWBKST  (bank material in this record)

      0:  LRTYP = 2 or 3, the record type

      1:  the floating point number 12345.0 as a check word

      2:  Zebra version number, integer = 10000. * QVERSIO
           zero if written by a user routine

      3:  processing option bits
           = 0  normally
             1  Direct-access-Table record
             2  DaT forward reference record

      4:  zero (reserve word)

      5:  NWTX, number of words in the text-vector, may be zero

      6:  NWSEG, number of words in the segment table,
           = 3 * number of table entries

      7:  NWTAB, number of words in the relocation table,
                  = 2 * number of table entries,
           if = 0:  linkless data-structure

      8:  NWBK, total number of words occupied by the bank material,
           both on memory and on the file, may be zero if empty d/s

      9:  LENTRY, entry address into the d/s

word 10:  NWUHIO = NWUH + NWIO, number of words in the user header
           vector plus its I/O characteristic, zero if no header

           11: control-word for the I/O characteristic
               either: immediate, NWIO=1,
                         the whole vector is of the same type
                         = 1  all bits
                           2  all integer
                           3  all floating
                           7  self-decriptive

                    or: complex, start of the I/O characteristic
                         as delivered by MZIOCH, NWIO words in all

\end{verbatim}

The 10 words of basic pilot information are followed by
these data sectors:

\begin{DLtt}{123456}
\item[NWIO]   words of the I/O characteristic for the user header vector
\item[NWUH]   words of the user header vector
\item[NWSEG]  words of the segment table
\item[NWTX]   words of the text-vector
\item[NWTAB]  words of the relocation table
\item[NWBKST] words of bank material in this record, at most = \Lit{NWBK}
\end{DLtt}

A decription of the contents of the segment table, the text vector,
and the relocation table is found in the format decription of the
native mode.
The bank material is a direct copy
(but normally transformed to exchange data format)
of the data in the dynamic store.

\subsection*{Exchange File Format, logical records: data-structure continued}

\begin{verbatim}
   word -1:  NWLR, the record length

         0:  LRTYP=4, the record type

         1 -> NWLR:  further bank material
\end{verbatim}

\subsection*{Exchange File Format, logical records: padding records}

Padding records must be used to mark unused areas of
physical records.

\begin{verbatim}
   word  0:  NWLR, the record length (excluding the first word only!)

         1:  LRTYP = 5 or 6, the record type

         2 -> NWLR:  words to be ignored on input
\end{verbatim}

The total size of a padding record is \Lit{NWLR+1},
including the LR length and the \Lit{LR} type;
thus a padding record with \Lit{NWRL=1} occupies two words.
To pad exactly one word, \Lit{NWRL=0}  should be stored,
in this case the \Lit{LR} type 5 is implied
without being present in the data.

The maximum size of a padding record is the size of a
physical record.

There must not be more than 4 consecutive padding records.

\subsection*{Exchange File Format: physical records}

A file in exchange file format is supported by physical records of
identical length (fixed-length records), this being the only
format which can safely be read on all machines.
A physical record consists of \Lit{NWPHR} words;
the word size is 32 bits for the Exchange Data Format,
for the Native Data Format it is the word size of the machine.

The physical record size \Lit{NWPHR} is 900 words standard,
but a different size may be specified at OPEN time.
For the Exchange Data format, \Lit{NWPHR} must be a multiple of 90 words,
ie. 360 bytes = 64x5x9 bits,
this being the smallest size compatible with all machines.

The following table shows, for the Exchange Data Format,
some possible values of \Lit{NWPHR}, being multiples of 90 words.
On machines with a word size of more than 32 bits the apparent
record size, which may be needed in JCL, will be less;
this is also shown, supposing that packing is dense.

\begin{verbatim}
        n  NWPHR   bytes  is words of  36    48    60    64  bits

        1     90     360               80    60    48    45
       10    900    3600              800   600   480   450
       20   1800    7200             1600  1200   960   900
       40   3600   14400             3200  2400  1920  1800
       50   4500   18000             4000  3000  2400  2250
       64   5760   23040 = 512*45    5120  3840  3072  2880
       80   7200   28800             6400  4800  3840  3600
\end{verbatim}

{\bf Note}: with Zebra version 3.61
the '90 word rule' has been relaxed such that the size of
physical recods is required to be a multiple of 30 words only.
One should not take advantage of this if one has 36-bit machines
in the experiment.

There are two kinds of physical records:
the {\bf steering} blocks start with 8 words of control information;
the {\bf fast} blocks have no control information at all.

Several fast blocks in succession are called a {\bf burst};
each burst of fast blocks is preceded by a steering block,
whose control information indicates the number of fast blocks
to be expected.

The format of a steering block is as follows:

\begin{verbatim}
L+0  word 1:  hex  0123CDEF
 +1  word 2:       80708070
 +2  word 3:       4321ABCD
 +3  word 4:       80618061

 +4  word 5:  bits  1 -> 24: NWPHR, the physical record length
                               (in 32-bit words)

              bits 25 -> 32: flag bits:
                               32 emergency stop block
                               31 end-of-run
                               30 start-of-run in this block

 +5  word 6:  physical record counter (fast blocks do not count)
               to permit checking for blocks lost by read errors;
               if zero: no checking

 +6  word 7:  NWTOLR, the off-set to the first logical record
               starting in this block:

               unless NWTOLR=0, the first new LR in this block starts
               at word NWTOLR+1, this word being its record size.

               if NWTOLR = 0: the whole block continues data
                               of the current logical record.

 +7  word 8:  NFAST, the number of fast blocks to follow just behind
               the current steering block

 +8  words 9 -> NWPHR:  the data body of the block
\end{verbatim}

The decimal values of the steering block stamp,
either as 16-bit half-words or as 32-bit words, unsigned or signed,
are:

\begin{verbatim}
    0123CDEF =  (  291 | 52719) =     19 123 695
    80708070 =  (32880 | 32880) =  2 154 856 560  or  -2 140 110 736
    4321ABCD =  (17185 | 43981) =  1 126 280 141
    80618061 =  (32865 | 32865) =  2 153 873 505  or  -2 141 093 791

    0123CDEF =  001 1074 6757  octal
    80708070 =  200 3410 0160
    4321ABCD =  103 1032 5715
    80618061 =  200 3030 0141
\end{verbatim}

\Filename{H2-FZ-Example-dedicated-online}
\section{Example for coding dedicated on-line output}

To see the implications of this format, let us sketch what
an output routine would have to do which has \Lit{NWD} words
in a vector \Lit{MDATA} in a common \Lit{/DATA/} already in exchange data format,
say they are all integers,
to be written out as a Zebra file in exchange file format.

It has to preceed this vector by four kinds of control information,
going backward in this order:

\begin{itemize}
\item add the bank system words to make it into a Zebra bank
\item add the pilot information
\item add two words to make it into a logical record
\item add eigth control words to make the first physical record,
      a steering block
\end{itemize}

Most of this information can be set up once by initialization,
and only a few holes must be filled in for each data-structure.
In particular, the user header vector which identifies the
structure must be filled,
in the example we chose a header of 4 words.
Remember that all numbers must be in exchange representation;
but most numbers are positive integers, and hence are no problem.
Numbers where one has to be careful are flagged by \Lit{!!},
numbers which must be re-filled are marked by \Lit{*}.

\begin{verbatim}
   COMMON /DATA/MPR(8),MLR(2),MPILI(15),MBK(10),MDATA(100000)
   DIMENSION     MREC(100035)
   EQUIVALENCE  (MREC(1),MPR(1))

      PARAMETER (NWPHR=900)
\end{verbatim}

\subsubsection*{Initialize}

Physical record control words

\begin{verbatim}
      MPR(1) = hex 0123CDEF  (markers)
      MPR(2) =     80708070
      MPR(3) =     4321ABCD
      MPR(4) =     80618061
      MPR(5) = NWPHR         (block size)
    * MPR(6) = 0             (block number)
      MPR(7) = 8             (8 words before start of LR)
    * MPR(8) no init         (no. of fast blocks to follow)
\end{verbatim}

Logical record control words

\begin{verbatim}
    * MLR(1) no init         (record size)
      MLR(2)   = 3           (record type, d/s)
\end{verbatim}

Pilot information

\begin{verbatim}
   !! MPILI(1) = 12345.0     (check word, floating point)
                              IEEE:  hex 4640E400
      MPILI(2) = 0           (Zebra version, user origin)
      MPILI(3) = 0           (processing option)
      MPILI(4) = 0           (reserve)
      MPILI(5) = 0           (no text vector)
      MPILI(6) = 0           (no segment table)
      MPILI(7) = 0           (no relocation table, linkless d/s)
    * MPILI(8) no init       (total bank space)
      MPILI(9) = 0           (no entry link)
      MPILI(10)= 5           (length of the user header, =NWIO+NWUH)
      MPILI(11)= 2           (I/O characteristic: all integer)
    * MPILI(12->15)          (user header vector)
\end{verbatim}

Bank system words

\begin{verbatim}
      MBK(1) bits 17-32:  2      (I/O char., all integer, say,
                                              cf book MZ, para 1.15)
                   1-16: 12      (= NIO + NL + 12)
      MBK(2) = 0             (next pointer)
      MBK(3) = 0             (  up pointer)
      MBK(4) = 0             (orig pointer)
      MBK(5) = 0             (Numeric   bank ID)
   !! MBK(6) = 4HDATA        (Hollerith bank ID)
      MBK(7) = 0             (number of links)
      MBK(8) = 0             (number of structural links)
    * MBK(9) no init         (number of data words)
      MBK(10)= 0             (status word)
\end{verbatim}

\subsubsection*{Event by event}

Suppose now \Lit{NWD} words are received into the vector \Lit{MDATA},
just behind \Lit{MBK(10)}, in exchange data format, and the lot is to be
written out. We first have to fill the variable part in the control
information for this data structure:

\begin{verbatim}
      MBK(9)   = NWD         (size of the data part of the bank)
      MPILI(8) = NWD+10      (amount of bank space required)
      MLR(1)   = NWD+25      (logical record size)
      MPILI(12)= run number         (user header vector)
      MPILI(13)= event number
      MPILI(14)= date
      MPILI(15)= time

      NWT  = NWD + 35           (total number of words to be written)
      NREC = (NWT-1)/NWPHR + 1  (number of physical records)
      MPR(8) = NREC -1          (number of fast blocks)

      NWPAD = NWPHR*NREC - NWT  (no. of unused words in last block)
      MDATA(NWD+1) = NWPAD-1    (logical record length)
      MDATA(NWD+2) = 5          (padding record)
\end{verbatim}

and now we can dump the lot with

\begin{verbatim}
      JD = 0
      DO  19  JREC=1,NREC
      write physical record, words (MREC(J+JD),J=1,NWPHR)
   19 JD = JD + NWPHR
      MPR(6) = MPR(6) + 1      (bump the block number)
\end{verbatim}
