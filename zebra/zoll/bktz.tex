% file : bktz.tex
\thispagestyle{empty}     % empty head
\markboth{}{25 Sept 91}   % empty foot
\vspace*{-20mm}
\begin{center} {\large\bf Zebra Reference Manual} \vspace{8mm} \end{center}
\begin{center} {\LARGE\bf book TZ} \vspace{2mm} \end{center}
\begin{center} {\Large\bf Title Handling} \vspace{4mm} \end{center}
\begin{center}
  {\large\bf Zebra version 3.67 \vspace{2mm} \\
             June 1991          \vspace{2mm} \\
             J. Zoll}
\end{center}
\vspace*{20pt}
\begin{description}
  \in{30mm}
  \item[Chapter 1] User specifications for the TZ package
  \begin{itemize}
     \in{30mm}
     \item[1.1] TZFIND -- access to title banks
     \item[1.2] TZINIT -- creating title banks from a text file
     \item[1.3] TZUSER -- editing title banks during input
     \item[1.4] TZSHUN -- insert banks into a title structure
     \item[1.5] TZINQ  -- inquire about the title d/s
  \end{itemize}
  \item[Chapter 2] Formats for the text input to TZINIT
  \begin{itemize}
     \in{30mm}
     \item[2.1] Out-line
     \item[2.2] Control-lines
     \item[2.3] Title header lines
     \item[2.4] Free-field input
     \item[2.5] Fortran formatted input
  \end{itemize}
\end{description}
\newpage
\vspace*{40pt}
{\large\bf Acknowledgement} \vspace{4mm}

This package is derived from the HYDRA package TQ

V.Frammery, U.Herbst-Berthon, J.Zoll,\\
HYDRA Topical Manual, book TQ, CERN Program Library
\cleardoublepage  % continue on next odd page
\markboth{Principles}{Principles}
\vspace*{20pt}
\lile{10mm}
\begin{center} {\large\bf{Principles}} \end{center}

The TZ linear data-structure of easy-creation and easy-access
banks has been provided primarily to satisfy the need of
programs for run-dependent control-information
(called 'titles').

There may be one title structure for each Zebra store.
The title stucture normally resides in some
long-range division of the user.

To give a means of creating into the TZ structure banks which
can easily be changed from run to run,
the routine TZINIT is provided.
This reads a text file provided by the user,
containing the instructions of which title banks to create
and also the data to be placed into these banks.

Access by the user program to a given bank in the TZ structure
in store designated by IXSTOR is obtained with :

\verb"      CALL TZFIND (IXSTOR, L, IDH, IDN, IFLAG)"

This finds the title bank with the Hollerith bank identifier IDH
and the numeric identifier IDN (if non-zero);
it returns in L the pointer to the title bank found.
The last parameter controls the action to be taken
in case of missing title.

Some applications need different versions of a given title bank
as a function of the data they are processing.
TZVERS has been provided for this problem; it receives a selection
integer as one of its parameters, and it will scan the different
versions of the title bank until it finds one whose validity range
matches the selector.
The validity range must be specified on the first two words of
all the versions of all title banks to be handled with TZVERS.

\vspace*{20pt}
\begin{center} {\large\bf Conventions} \end{center}

In this manual we flag output parameters in subroutine calls,
i.e. parameters which return values to the caller,
by an asterisk "*".

If the input value of such a parameter is also significant
this is marked by prefixing a second asterisk "*".

A parameter which is a link is marked by an exclamation mark "!".

The types of variables follow from the Fortran default typing
convention, except that variables beginning with the letters
"{\tt ch}" are of type {\tt CHARACTER}.

\lile{-10mm}
\chapter{Using the TZ package}

\section{TZFIND - access to title banks}
\smark{TZFIND - TZVERS}

The banks in the linear TZ structure are identified by their
normal Hollerith and numeric bank identifiers.
To find a particular title bank, one calls :

\ROUTA{CALL TZFIND (IXSTOR, !L*, IDH, IDN, IFLAG)}
\bvb

with

    IXSTOR  index of the store holding the title-structure
            (or index of any division in this store);
            = zero for the primary store

       !L*  output parameter to contain the link pointing
            to the title bank found; = zero if not found

       IDH  the Hollerith identifier,  either
            an integer variable of the form 4Hxxxx, or
            a literal character string 'xxxx' of 4 characters exactly

       IDN  the numeric identifier,
            if zero the first bank matching IDH is returned,
            otherwise the search continues to find the bank IDH
            which has the numeric identifier IDN;

     IFLAG  indicating the action to be taken for a missing title:

            IFLAG = 0 :  return with L=0

            IFLAG > 0 :  TZFIND shall take the error-exit
                         CALL ZTELL (n,1)  with n :

                         n = 61    for IFLAG=1,
                         n = IFLAG for IFLAG > 99.
\end{verbatim} \lspa

TZFIND returns the first bank with IDH / IDN in the linear structure.
If there are other banks with the same IDH / IDN further down,
they cannot be reached with TZFIND;
but they could be reached with TZVERS
or with LZFIND (cf. book MZ).

\cleardoublepage  % continue on next odd page
\vspace*{8mm}
TZVERS is similar to TZFIND, but it allows one also to select
a particular version from a set of title banks all having the
same IDH (and IDN, maybe).

The use of TZVERS requires that the first two words of the title banks
to be selected contain two integers specifying the validity range
of the bank.

\ROUTA{CALL TZVERS (IXSTOR, !LR*, IDH, IDN, ISELECT, IFLAG)}
\bvb

with

    IXSTOR  index of the store holding the title-structure
            (or index of any division in this store);
            = zero for the primary store

      !LR*  output parameter to contain the link pointing
            to the title data; = zero if not found

       IDH  the Hollerith identifier,  either
            an integer variable of the form 4Hxxxx, or
            a literal character string 'xxxx' of 4 characters exactly

       IDN  the numeric identifier,
            if zero the first bank matching IDH is returned,
            otherwise the search continues to find the bank IDH
            which has the numeric identifier IDN;

   ISELECT  selector to find the bank whose validity range
            matches this integer inclusively

     IFLAG  indicating the action to be taken for a missing title:

            IFLAG = 0 :  return with LR=0

            IFLAG > 0 :  TZVERS shall take the error-exit
                         CALL ZTELL (n,1)  with n :

                         n = 62    for IFLAG=1,
                         n = IFLAG for IFLAG > 99.
\end{verbatim} \lspa
\newpage
\section{TZINIT - creating title banks from a text file}
\smark{TZINIT}

The data to be read by TZINIT are prepared on a formatted file,
where they can easily be changed.
Typically normal free-field format titles on this file
look like this :

\bvb
      *DO MEDI
          3                    #. this is a trailing comment
          1.      217.5993     #. MEDIUM 1 : AIR PATH
          1.5259    7.496      #.        2 : VAC TANK WINDOW
          1.5324   17.0278     #.        3 : MAIN WINDOW
          1.1005   50.54       #. HYDROGEN

      *DO FZO 21 -ACW
          4  0  :TLO3  :/dev/mt12    #. FZ option and file name
\end{verbatim} \lspa

The data for the title banks are given one after the other.
Each title starts with the 'title header line',
marked by *DO in col. 1,
giving the Hollerith identifier (like MEDI or FZO) and
possibly the numeric identifier (like 21),
possibly followed by options valid for this title (like {\tt -ACW}),
selecting the way in which the data are to read.

Full specifications for the formats are given in chapter 2.

With these data TZINIT will create in the title structure a bank
with the Hollerith ID 'MEDI' and with 9 data words,
and also a bank FZO / 21.
If this resides in the primary store, the program may
{\tt CALL~TZFIND~(0,L,'MEDI',0,1)} to get access to the bank MEDI
with the following contents :
\bvb
         IQ(L+1)     3
          Q(L+2)     1.
          Q(L+3)     217.5993
          Q(L+4)     1.5259
            ...
          Q(L+9)     50.54
\end{verbatim} \lspa

If, after digesting the data, the program no longer needed
the bank it could remove it with  {\tt CALL MZDROP (0,L,'.')}

\newpage
To read the title data from a file connected to the logical
unit number LUN one calls :

\ROUTA{CALL TZINIT (LUN, IXDIV)}
\bvb
with    LUN :  logical unit number,
               if =  0 :  Zebra 'card reader' IQREAD
                  = -1 :  Zebra 'terminal input' IQTTIN

      IXDIV :  division into which the title structure is to be stored;
               if = IXSTOR :    division 2 of store IXSTOR
               if = zero :      division 2 of the primary store
               if = IXSTOR+24 : system division of store IXSTOR
               if = 24 :        system division of the primary store
\end{verbatim} \lspa

TZINIT may be called several times in succession for different files,
thus building up the title structure from different sources;
on the second or later calls for the same store the division
part of IXDIV is ignored.

New title banks are always linked by TZINIT to the end
of the title structure;
thus this linear structure reflects the chronological order
in which the banks have been read.
This allows to over-ride a title bank from one file
by a title bank from a file read earlier.
Take this common example :
\bvb
      CALL TZINIT (0,IXTITL)
      CALL TZINIT (7,IXTITL)
\end{verbatim} \lspa
Giving the same title with different contents on the 'card reader'
over-rides the one from LUN=7.

Note the following problem:
as explained in book MZ, section {\tt MZXREF},
Zebra assumes by default that banks in any division do not have
links pointing into the system division.
Thus, if one puts the title structure into the system division
one must not also make links in banks point to title banks,
because such a link would not be updated for a garbage collection
in the system division.
The same is true if the user sends the titles into
a 'package' division (cf. book MZ, section {\tt MZDIV}).
This problem does not exist for links in link areas.
\newpage
\section{TZUSER - editing title banks during input}
\smark{TZUSER}

Sometimes it is inconvenient to have to prepare
a given title on the input file in just the way in which
it is expected by the program.
Therefore TZINIT allows for a transformation of the data read,
to be done by the user routine TZUSER
at the level of individual titles.

This transformation by TZUSER is requested either globally
for all titles by the control-line  *USER  (cf. para.~2.2)
or individually for particular titles by giving
option --U or --Un on the header line for the title bank.
This causes TZINIT to transfer control to TZUSER as soon
as the input of the data for the particular title is complete,
and before starting to handle the next title.

Communication between TZINIT and TZUSER is as follows :

\ROUTA{CALL TZUSER (!LOLD)}
\bvb
with  COMMON /TZUC/ JSTOR,IXTITL, NPARA, LNEW,NWOCC, NAME(20)

      !LOLD :  address of the original bank filled with the input data

      JSTOR :  the number of the store = 0, 1, 2, ... currently being
               used, in case TZUSER has to cope with several structures

     IXTITL :  the index of the title division for convenience

      NPARA :  the value "n" from the option -Un
               = zero  if n not given

      LNEW* :  adr of the replacement bank lifted by the user, if any,
               initialized to zero by TZINIT;
               set to  -1 :  skip this title bank
                      -99 :  kill this run

    *NWOCC* : the number of useful words available in the bank at LOLD,
              this may be reduced or increased by TZUSER, but not
              above NAME(4).  (TZINIT lifted the bank with
              NAME(4) words and has read NWOCC words into it.)

       NAME :  the 'name' used by TZINIT to lift the bank at LOLD :

               NAME(1)    Hollerith identifier
               NAME(2/3)  NL / NS - number of links = 0
               NAME(4)    ND - maximum number of data words
               NAME(5->)  I/O characteristic
\end{verbatim} \lspa
\newpage
By programming TZUSER, the user may do one of 4 things :
\bvb
      - change the information and the size of the original bank;
      - create a new bank to replace the old one;
      - simply suppress this title altogether by setting LNEW=-1;
      - kill the run by setting LNEW=-99.
\end{verbatim} \lspa
\vspace*{4mm}
{\large\bf Change the information}

Modify the data as necessary, set NWOCC to the number of useful
words in the bank if this has changed (without exceeding NAME(4)!),
and return.

When control returns to TZINIT it will take the appropriate action
according to the content of LNEW and maybe NWOCC.

\vspace*{4mm}
{\large\bf Lift a replacement bank}

To avoid problems in case of garbage collection,
get the address of the original bank as follows :
\bvb
      SUBROUTINE TZUSER (LPARA)
      DIMENSION    LPARA(9)
      LOLD = LPARA(1)
\end{verbatim} \lspa
Set NAME(4) to the wanted size of the replacement bank,
and maybe also NAME(5-) for the I/O characteristic,
and call :
\bvb
      CALL MZLIFT (IXTITL,LNEW,LPARA,0,NAME,-1)
      LOLD = LPARA(1)
\end{verbatim} \lspa
This second statement re-loads the local variable LOLD with
the address of the old bank, which might have changed.
Copy the information, possibly modified, from the bank at LOLD
into the bank at LNEW, and return.

TZINIT relies on the replacement bank having been
linked as the 'next' bank to the original.

\vspace*{4mm}
{\large\bf Kill execution}

If TZUSER discovers problems it can signal to TZINIT
that the job should be killed by setting --99 into LNEW.
TZINIT will still read to the end of the file to find other
problems, maybe, and then call ZFATAL.

\vspace*{4mm}
{\large\bf Loading TZUSER}

As always with user routines called from a general library
routine, there is the problem of getting the right TZUSER loaded.
If it is sent through the compiler to the linker there is no
problem. But if it is compiled and then put onto a user
library one must force its loading from this library,
either by loader directives, if available, or more simply
by having a {\tt CALL TZUSER} in one's MAIN program.
In this latter case this must be a conditional
call which will never be executed.

For the applications which do not need TZUSER there is a dummy
version on the Zebra library. It will cause the job to fail
in case it is reached by accident.
\cleardoublepage  % continue on next odd page
\section{TZSHUN - insert banks into a title structure}
\smark{TZSHUN}

Although most commonly the banks in the TZ structure
are created by TZINIT,
it may sometimes be handier
to set-up some titles directly in the program,
rather than taking them from an external text file.
The main advantage of introducing the titles via TZINIT is
that they are easily changed without re-compilation of the program.
But for a title which never changes one can lift a bank
in the right division and hand it to TZSHUN to re-link
it into the TZ structure :

\ROUTA{CALL TZSHUN (IXSTOR, !L, IFLAG)}
\bvb
with  IXSTOR :  index of the store holding the title structure
                (or index of any division in this store);

          !L :  address of the (linear structure of) new bank(s)

       IFLAG :  if = 1 :  insert at the start,
                   = 0 :  insert at the end of the structure.
\end{verbatim} \lspa

Example : create a default title bank TRAN in the system division
of store IXSTOR if it does not already exist:

\bvb
      DIMENSION    NAME(5)
      DATA NAME /4HTRAN, 0, 0, nd, 3/

      CALL TZFIND (IXSTOR,L,NAME,0,0)
      IF (L.EQ.0)  THEN
          CALL MZLIFT (IXSTOR+24, L, 0,2, NAME)
          Q(L+1) =  title word 1
          Q(L+2) =  title word 2
           ...       ...
          Q(L+nd)=  title word nd
          CALL TZSHUN (IXSTOR,L,0)
        ENDIF
\end{verbatim} \lspa

Note: the system division of store IXSTOR is specified
symbolically as IXSTOR+24.

Note: if L points in fact to a linear structure  all banks are taken.
\newpage
\section{TZINQ - inquire about the title d/s}
\smark{TZINQ}

The title d/s is supported by a system link not directly accessible
to the user.
To get access to this structure as such, not to a particular bank,
one can use this routine.

\ROUTA{CALL TZINQ (IXSTOR, IXTITL*, !L*, IFLAG)}
\bvb
with  IXSTOR :  index of the store holding the title structure
                (or index of any division in this store);

     IXTITL* :  returns the index of the division holding
                the title structure

         !L* :  returns the address of the first or the last bank
                = zero if there are no titles

       IFLAG :  if = 1 :  get the adr of the first bank
                   = 0 :  get the adr of the  last bank
                other values are reserved for future extensions
\end{verbatim} \lspa

{\large\bf Example}

Suppose for a particular program one has a considerable volume of titles,
such that one spends too much time in TZINIT for every individual
execution of this program.
One could have a separate job to translate most of the text titles
into a binary FZ file whenever they have been updated.
The job which needs these titles reads them from the FZ file
and appends them to any titles which it has already read with TZINIT.

{\bf Job to translate text titles to binary}
\bvc
      PROGRAM TTXTFZ

      CHARACTER    NAME*(*)
      PARAMETER   (NAME = 'cba1990')

      CHARACTER    FIN*(*), FOUT*(*)
      PARAMETER   (FIN = NAME // '.ttx',  FOUT = NAME // '.tfz')

      PARAMETER   (LIM2Q=60000, NNQ=80000)
      DIMENSION    LQ(NNQ), IQ(NNQ), Q(NNQ)
      EQUIVALENCE (Q(1),IQ(1),LQ(9))
      COMMON //    FENCE(4), LQ, LASTQ

      PRINT 9001, FIN, FOUT
 9001 FORMAT (/' TTX to TFZ executing'
     F/' with  input file = ',A /'      output file = ',A)

      CALL MZEBRA (-1)
      CALL MZVERS
      CALL MZSTOR (IXST,'/DYN/','.',FENCE,LQ,LQ,LQ,LQ(LIM2Q),LASTQ)

      OPEN (11, FILE=FIN, FORM='FORMATTED', STATUS='OLD')
      CALL TZINIT (11, 1)          { read into forward division 1
      CLOSE (11)

      OPEN (11, FILE=FOUT, FORM='UNFORMATTED', STATUS='UNKNOWN')
      CALL FZFILE (11, 0, 'O')

      CALL TZINQ  (0, IXTITL, LGO, 1)
      CALL FZOUT  (11, 1, LGO, 0, 'LDI', 0,0,0)
      CALL FZENDO (0, 'T')
      CALL MZEND
      END
\end{verbatim}
{\bf Piece of code loading the binary titles}
\bvc
      CHARACTER    FIN*(*)
      PARAMETER   (FIN  = 'cba1990.tfz')

C--       first read the variable titles from the 'card reader'

      IXTITL = set the title division
      CALL TZINIT (0,IXTITL)

C--       then append the binary titles

      OPEN (49, FILE=FIN, FORM='UNFORMATTED',STATUS='OLD')
      CALL FZFILE (49, 0, '.')

      CALL TZINQ  (IXSTOR, IXTT, LTT, 0)
      IF (LTT.EQ.0)  THEN
          L     = 0
          JBIAS = 1
        ELSE
          L     = LTT
          JBIAS = 0
        ENDIF

      CALL FZIN   (49, IXTITL,L,JBIAS, '.', 0,0)
      IF (IQUEST(1).NE.0)          GO TO trouble
      IF (LTT.EQ.0)  CALL TZSHUN (IXTITL,L,0)
      CALL FZENDI (49, 'TXQ')
\end{verbatim} \lspa
\chapter{Formats for the text input to TZINIT}
\section{Out-line}
\smark{Format out-line}

The input to TZINIT is presented on a formatted text file
of 80-column card-images,
giving the titles one after the other,
as shown by the example on the next page.

Each title starts with the 'title header line'
{\tt  *DO ...} (see para.~2.3)
which selects mode and options for the reading of its
associated 'title data'.
These have to come immediately and completely after the
title header line.

Normal mode is 'free-field format' where the data are read line-by-line
and the text is interpreted.
The mode of each data word is taken from the way it is written :
integer data have neither the decimal point nor an exponent,
floating data have either or both,
Hollerith data start with : or ' or ",
octal data are pre-fixed with \#O;
see para.~2.4 for full details.

With 'Fortran format' the data are read by a single
Fortran READ statement with the FORMAT taken from the --F option on
the title header line.
This is useful for computer generated titles.

'Control lines' to TZINIT may be given interspersed
with the titles.
Global options are selected with control lines;
other examples are the comment line with  {\tt *--}  in column 1,
and the input terminator *FINISH;
see para.~2.2 for details.

Control lines to TZINIT, including title-header lines starting
with *DO, are handled case-insensitive, ie. they are converted
to upper case before processing.

\newpage
\bvc
*-------            TITLE VERSION A3
*PRINT
*.-------1---------2---------3---------4---------5---------6---------7
*DO  RTBC  -iF                   #. 810127     CERN  ROLL 103
13.       1100.
            .99999  +.00042  +.00302
           -.00042   .99993  -.0023
           -.00302  +.0023    .99993     0.     +.015     1.963

*DO  CAM1  -e -n27 -c11 -iF      #. 810121 17.27  ROLL 103
 1 CAM       3.  12.    9.067    11.8516   75.6561     25.    -75.6561
 1 MED       2.        1.51      1.5       1.458     2.382     1.57
             4.06      1.0884
 1 DIST       -.00169     -.00004      .015        .0011      -.0001
               .00394     -.00245      .00378      .051        .0023
               .033        .09

*DO  FID1  -c11 -iF              #. 810121 17.27  ROLL 103  ERASME
            12.
 1 4   1     1.  -17.2475  -18.6369
 1 4   2     2.  -17.2595   -4.2708
 1 4   3     3.    -.7761   -4.3177
 1 1   1    11.  -13.8081  -15.9118
 1 1   2    12.  -13.7343   -7.5253
 1 1   3    13.   -4.8669   -7.5898
 1 1   4    14.   -4.8931  -15.9473
 1 2   1    21.  -13.0353  -15.6000
 1 2   2    22.  -12.8979   -7.3464
 1 2   4    24.   -3.7895  -15.6767
 1 3   3    33.   -2.6674  -17.2822
 1 3   4    34.   -3.1636  -17.8522

*DO  CAM2  -e -n27 -c11 -iF      #. 810121 17.27  ROLL 103
 2 CAM       3.  12.    9.1488  -10.3208   75.673      25.    -75.673
 2 MED       2.        1.51      1.5       1.458     2.382     1.57
             4.06      1.0884
 2 DIST       -.0017       .00005      .015       -.0011      -.0001
               .00043     -.00443     -.00278      .0065      -.0024
               .033        .05

*DO  FID2  -c11/40 -iF           #. 810121 17.27  ROLL 103  ERASME
            13.
 2 4   1     1.  -17.4350    3.9107     examples of other representations
 2 4   2     2.  -17.4786   18.2503     integer    15  -24  0
 2 4   3     3.    -.8612   18.2306     floating   -.123E-7  1E-12
 2 1   1    11.  -13.8891    7.4810     octal      #0731244600
 2 1   2    12.  -13.8296   15.8810     hex        #xffff
 2 1   3    13.   -4.9569   15.7810     Hollerith  :PISA :BARI
 2 1   4    14.   -4.9724    7.4133                :K*(1430)
 2 2   1    21.  -13.1122    7.3163                'text with blanks'
 2 2   3    23.   -3.7338   15.3468                "text with ' and blank"
 2 2   4    24.   -3.8710    7.1994
 2 3   1    31.  -15.4499    4.5935
 2 3   2    32.  -15.8947    5.0865
 2 3   4    34.   -3.2437    4.4470
*.-------1---------2---------3---------4---------5---------6---------7
\end{verbatim} \lspa
\newpage
\section{Control-lines}
\smark{Control-lines to TZINIT}

Control-lines carry special instructions to TZINIT;
they may only be given between titles,
and not within the data-body of a particular title.
An option selected by a control line is switched on or off
at the moment the line is read, and stays in effect until
changed by the inverse selection,
but the effect does not carry from one call to TZINIT to the next.

\bvb
   *---        normal comment line
   *.--        non-printing comment line
               blank lines are allowed and ignored

   *LOG        logging information to be printed by TZINIT;
   *LOG  OFF   switch off  (default : according to the log level of
                                      the store, normally 'on')

                  If LOG is 'on' TZINIT will echo on the log-file
                  each control line and each title header line read.

   *PRInt      echo data lines for free-field format titles;
   *PRInt OFF  switch off  (default : 'off')

                  *PRINT implies *LOG

   *USer       call TZUSER for every title bank;
   *USer OFF   switch off  (default : 'off')

   *KIll       kill the run also for slight trouble;
   *KIll OFF   switch off  (default : 'off')

   *ANYWAY     TZINIT is to return to the caller even for fatal errors,
               IQUEST(1) = number of errors, =0 normally

   *FINish     end of input data
\end{verbatim} \lspa

One gives *FINISH only to terminate titles on
the 'standard input file' (the 'card reader') in order to
separate from further input data.
For stand-alone files this is not necessary, as the EOF will
terminate correctly.
On the IBM it could even be harmful with concatenated title files.
\newpage
\section{Title header lines}
\smark{Title header lines}

A title-header line signals the start of data for a new title bank;
it specifies the ID of the title bank,
the way the data are to be read, and further options.
Its format is :
\bvb
   *DO  <idh> [idn]  [-<opt><option text>]   [#. comment ]

      *DO    in cols 1/4 marks the header line;
      <idh>  is the 4 character Hollerith bank ID;
      <idn>  is the numeric bank ID, integer, if any;
      <opt>  is a single option letter,
      <otx>  is the option text, if needed.

The following options are available :

   -F(format)  read the data with a single formatted Fortran
               READ statement, using the FORMAT given, which must
               not contain blanks, the brackets must be present.

   -Itext  set the I/O characteristic for the bank, with :
           text = B or I or F or D  if the bank is all bits, integer,
                                       floating, or double precision;
           text = (string) with "string" the I/O characteristic,
                           cf. book MZ, section MZFORM et al.
           default : type undefined

    -Nn  expect n data words, for -F this must be exact; for free-field
         format this may be omitted if less than 2000 words are to be
         read and if the size is not to be changed by TZUSER.
         A bank will initially be lifted to accomodate <n> data words to
         be read and will be shortened when input processing is complete.

    -U[n] call TZUSER for this bank, passing n as NPARA, zero if absent.

   For free-field format only :

     -C[a][/e]  use columns a/e only on each line,
                defaults :  1 for a,   80 for e

                Ex.:  -C11/72   use only the text on columns 11 to 72
                      on each line, except: '*' in col. 1 is forbidden

         -E[n]  expect exactly <n> words [as given by -Nn].
                If less or more data words are found, TZINIT will
                print a message and go to ZFATAL on end of input.

         -S[n]  true size of the bank is <n> words [as given by -Nn].
                The bank is not reduced to the number of data words
                actually read.

   -A[n][C][W]  initialize Hollerith handling as though the control-item
                #A[n][C][W]  had been read; see next paragraph.
\end{verbatim} \lspa
\newpage
\section{Free-field input}
\smark{Free-field input}

In free-field format the data-items are given one after the other,
separated by blanks.
The mode of each data-item is recognized from the way it
has been written by the user;
thus 12 is an integer,
\#012 is an octal integer (of value 10),
12. is a floating-point number,
and :ABC is Hollerith.
The rules for writing the data items try to strike a balance
between freedom and catching mistakes; they are the following :

{\bf Integer} :  a string of digits, maybe preceded by + or --
\bvb
      Examples:    123  +123  -123  1  0
\end{verbatim}
{\bf Floating} : a number containing a decimal dot and/or an exponent
\bvc
      Examples:       12.     +12.34  -.34   1.  .1  -12.34
                      12E     +12.34E -.34E  1E  .1E -12.34E
                     -12.34E5  12.E5  12E5  12E+5  12E-5
                     -12.34+5  12.+5  12+5  12+5   12-5
\end{verbatim}
{\bf Double precision} : a number containing an exponent letter D
\bvb
      Examples:       12D     +12.34D -.34D  1D  .1D -12.34D
                     -12.34D5  12.D5  12D5  12D+5  12D-5
\end{verbatim} \lspa
{\bf Binary -- Octal -- Hexadecimal} :  a number preceded
by \#B or \#b,
or by \#O or \#o or \#0,
or by \#X or \#x,
which will be stored right-justified with zero-fill;
bits beyond the capacity of a computer word are truncated on the left.
\bvc
      Examples :  #B10001  #b11101  #b1111111000111000011111111
                  #017777  #O17777  #o33211234567 (32 bits!)
                  #X177AB  #x17cde  #xffffFFFF    (32 bits!)
\end{verbatim} \lspa
{\bf BCD} :  must start with  ' or " or : thereby
selecting the delimiting character, followed by the Hollerith string,
terminated with the delimiter (which is "blank" for ":").
\bvb
      Examples :    :ABCD    :PI+   :K*   :K(1430)   :RATHERLONGSTRING
                    'AB CD'  :it's        "'ab' 'cd'"
\end{verbatim} \lspa
{\bf Repeat count} :  indicates the number of times which the data item which
follows immediately is to be stored.
It is an unsigned integer larger than 1 followed by '*'.
\bvb
      Examples :  100*0   4* -.0379   3*:PI-   2* :K(1430)  2*:LONGSTRING
\end{verbatim} \lspa
{\bf Comment} :    can be either trailing or interspersed.
A comment starts with  \#.  and stops with the next  \#  or the end-of-line.
\bvb
      Example :   12  #. any text, but not a number-sign  #  13
\end{verbatim} \lspa
\newpage
{\bf Hollerith handling}

Hollerith text is by default stored in A4 format,
thus the data-item {\tt ":ABCDEF"} gives on all machines 2 words
containing {\tt "ABCD"} and \verb'"EF  "' with blank-fill.
If this is not what is needed it can be changed either with
the 'control item' \#A or with the title header line option --A,
which have the same syntax.

Variable length strings could be awkward to handle,
therefore one can ask TZINIT to store the Hollerith string preceded
by a word count as an extra integer word,
optionally preceded in turn by a character count :
\bvb
Header-line option :   -A[n][C][W]
      Control item :   #A[n][C][W]

   with   n :  if given it selects An representation,
               if absent or if n is larger than the maximum number
               of characters per word, the data are stored continuous
               without trailing blanks in each word.

          C :  if given an extra word is stored specifying the
               number of characters in the string, not counting
               trailing blanks in any word.

          W :  if given an extra word is stored specifying the
               number of words occupied by the BCD string,
               excluding itself and the character count, if present.

Example:  #A4CW  :LONGSTRING

          would give 5 words :
          IQ(L)   = 10       character count
          IQ(L+1) = 3        word count
          IQ(L+2) = 4HLONG
          IQ(L+3) = 4HSTRI
          IQ(L+4) = 4HNG
\end{verbatim} \lspa
{\bf Other control items}

{\bf \#Double } instructs TZINIT to read and store all floating-point
numbers that follow as double-precision numbers.

{\bf \#Normal } reverts to normal, cancelling the effect
of a previous \#D for the numbers that follow the \#N.

Control-items must be blank-terminated like data items
and may be freely mixed with data items.
They must not occur between a multiplier and its data item.

The option selected by a control item applies to all following
data until changed again by a new control item.
\newpage
\bvb
Example of usage :  suppose we have this title bank :

      *DO FZO 21  -i(3I *H 1I *H)
      #. mode nwrec  opt    name #   #ACW
            4     0  :TLO3  :/dev/mt12

giving the parameters for a file to be used by FZ of Zebra.
(The I/O characteristic is only useful if one wanted Zebra
 to print this bank, which is unlikely in this case.)

The program could digest this as follows :

      CHARACTER    CHOPT*8, CHNAM*80

      LUN = 21
      CALL TZFIND (0, LT, 'FZO ', LUN, 0)
      IF (LT.EQ.0)                 GO TO no output
      MODE   = IQ(LT+1)
      NWREC  = IQ(LT+2)
      LOPT   = LT+3
      LNAM   = LOPT + IQ(LOPT+1) + 2
      NCHOPT = IQ(LOPT)
      NCHNAM = IQ(LNAM)
      CALL UHTOC (IQ(LOPT+2),99, CHOPT, NCHOPT)
      CALL UHTOC (IQ(LNAM+2),99, CHNAM, NCHNAM)
      CALL MZDROP (0,LT,'.')

      IF (MODE ...
               ...
        ELSEIF (MODE.EQ.4)  THEN
          IF (NWREC.EQ.0)  NWREC = 5760
          CALL CFOPEN (IQUEST(1),1,NWREC,'w',0,CHNAME(1:NCHNAM),ISTAT)
          CALL FZFILE (LUN,NWREC,CHOPT(1:NCHOPT))
        ELSEIF (MODE ...
                     ...
        ENDIF
\end{verbatim} \lspa
\newpage
\section{Fortran formatted input}
\smark{Fortran formatted input}

With Fortran formatted input all the data for the complete title bank
are read with a single READ statement using the format given
on the header line.
For this to be possible the exact number of words to be read
must be specified with the --N option.

\bvb
   Example :

       *DO  FIDU 1 -IF -N37 -F(10X,F10.0/(10X,3F10.0))
                   12.
        1 4   1     1.     -17.2475  -18.6369
        1 4   2     2.     -17.2595   -4.2708
        1 4   3     3.       -.7761   -4.3177
        1 1   1    11.     -13.8081  -15.9118
        1 1   2    12.     -13.7343   -7.5253
        1 1   3    13.      -4.8669   -7.5898
        1 1   4    14.      -4.8931  -15.9473
        1 2   1    21.     -13.0353  -15.6000
        1 2   2    22.     -12.8979   -7.3464
        1 2   4    24.      -3.7895  -15.6767
        1 3   3    33.      -2.6674  -17.2822
        1 3   4    34.      -3.1636  -17.8522
       *.-------1---------2---------3---------4---------5
\end{verbatim} \lspa

The option {\tt -IF} specifying the I/O characteristic
is important in this case.
It causes the execution of a Fortran READ statement with an I/O list
of type REAL;
on some machines Fortran is not able to read acting solely under
FORMAT control, ignoring the type of the I/O list.
