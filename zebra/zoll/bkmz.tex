\Filename{H1-MZ-memory-management}
\chapter{Memory management}
\label{sec:H1-MZ-memory-management}

The memory management package MZ of ZEBRA is fundamental
to all other ZEBRA packages.

The ZEBRA package MZ derives from its predecessors:

\begin{itemize}
\item the ZEBRA data structure derives from the HYDRA data structure~\cite{bib-HYDRAMZ};
\item the multi-store strategy derives from ZBOOK~\cite{bib-ZBOOK}
\end{itemize}

\Filename{H2-MZ-ZEBRA-bank-format}
\section{ZEBRA bank format}

\begin{verbatim}
    LQ(L-NL-NIO-1) IOcb/NOFF    IOcb: I/O control byte (16 bits)
                                NOFF = NIO + NL + 12 (16 bits)

    LQ(L-NL-NIO)   IOW (1)      first extra I/O descriptor word (if any)
                   ...
    LQ(L-NL-1)     IOW (NIO)    last  extra I/O descriptor word

    LQ(L-NL)       link NL      last reference link
                   ...
    LQ(L-NS-1)     link NS+1    first reference link
    LQ(L-NS)       link NS      last structural link
                   ...
    LQ(L-1)        link 1       first structural link

    LQ(L)          next         adr of the next bank in the linear str.
    LQ(L+1)        up           adr of the supporting bank
    LQ(L+2)        origin       adr of the supporting link

    IQ(L-5)        IDN          numeric   bank ID
    IQ(L-4)        IDH          Hollerith bank ID  (4 characters)
    IQ(L-3)        NL           number of links
    IQ(L-2)        NS           number of structural links
    IQ(L-1)        ND           number of data words

    IQ(L)          status       bits  1-18 user
                                     19-32 system   25: drop bit
                                     19-22 NIO      26: mark bit
    IQ(L+1)        first data word
                   ...
    IQ(L+ND)       last  data word

            This layout requires:

                  DIMENSION    LQ(999), IQ(999), Q(999)
                  EQUIVALENCE (LQ(9),IQ(1),Q(1))


 ---- Format of a short dead region ----

    word 1:  bits  1-16: NW, # of words, with  0 < NW < 12
                   17-24: NW again for redundancy
                      25: drop-bit set
                  26-end: zero
     2->NW:  dead words, content irrelevant
\end{verbatim} 

\Filename{H2-MZEBRA-initialize-ZEBRA-system}
\section{MZEBRA - initialize the ZEBRA system}

To initialize the ZEBRA system Common blocks the user must call
\Rind{MZEBRA}, before any other request to ZEBRA.

In particular, the following COMMON variables of interest to
the user are initialized:

\begin{verbatim}
logical unit numbers:
      COMMON /ZUNIT/ IQREAD,IQPRNT,IQPR2,IQLOG,IQPNCH,IQTTIN,IQTYPE

default logging level:
      NQLOGD in /ZSTATE/

where
      IQREAD   lun for standard system input ('card reader')

      IQPRNT   lun for standard  user print output
      IQPR2    lun for secondary user print output
               initialization: IQPRNT=IQPR2=IQLOG

      IQLOG    lun for standard system output ('line printer')
               this is used for all ZEBRA system printing

      IQPNCH   standard system coded output to be read back by
               program ('card punch')

      IQTTIN   standard on-line terminal input  (zero if none)
      IQTYPE   standard on-line terminal output (zero if none)

      NQLOGD   system-wide default logging level, see next para.
               standard default initialization to zero
\end{verbatim} 

On any particular machine ZEBRA knows the correct values for
the logical unit numbers, for example it knows that 6 and 7
are the right values for IQREAD and IQLOG on the IBM,
similarly  L"INPUT" and L"OUTPUT"  on the CDC Cybers.
It is of advantage to the user to direct all his print output
through the unit numbers provided by ZEBRA in /ZUNIT/.
After the call to MZEBRA he is free to change any particular
value in /ZUNIT/, and with the call to MZEBRA he may also
modify the initialization, in particular he may re-direct
the standard print output to the on-line terminal.

The parameter in the call to MZEBRA may select initialization
options either with an integer constant or with a list:

\ROUTA{CALL MZEBRA (LIST)}
\begin{verbatim}
 --- short form of options:

   with  LIST = 0:  standard defaults
               -1:  preset IQLOG = IQTYPE, ie. output to terminal
               -2:  preset NQLOGD= -2,     ie. only error messages
               -3:  preset NQLOGD= -2 and IQLOG=IQTYPE

\end{verbatim}

\begin{verbatim}
 --- options specified by a list:

   with  LIST(1)   N elements in the list to follow (excluding itself)
                   if zero or -ve: standard options

             (2)   -> NQLOGD, the system-wide default log level
                              (see MZLOGL on the next page)

                   for example: LIST(2)= -3: suppress all messages
                   standard default: NQLOGD = 0

             (3)   -> IQLOG,  lun for standard log printing
                              unless absent or zero
                              if = -1: set IQLOG = IQTYPE

             (4)   -> IQPRNT, lun for standard output printing
                              if absent/zero: set IQPRNT = IQLOG
                              if = -1:        set IQPRNT = IQTYPE
\end{verbatim} 
\ROUTA{CALL MZVERS}
 prints at logging level -1 or above an initialisation message
on unit IQLOG, showing amongst other things,
the version number of the current ZEBRA system.
In IQUEST(1) it returns as an integer
the Zebra MZ version number multiplied by 10000;
thus version 3.66 /7 would give 36607.

\ul{Examples}:
\begin{verbatim}
   set normal logging, set printer output to terminal:

      CALL MZEBRA (-1)
      CALL MZVERS

   set monitor logging, set unit 4 as print file:

      DIMENSION    LIST(3)
      DATA  LIST   / 2,  2, 4 /
      CALL MZEBRA (LIST)
      CALL MZVERS
\end{verbatim} 

MZEBRA only initializes the general ZEBRA system commons,
it does not initialize the dynamic store.
Before any request to the ZEBRA system involving the store,
the user must initialize it by calling MZSTOR.

At the end of execution the user should
\ROUTA{CALL MZEND}
to obtain the statistics of usage of the dynamic store.

\Filename{H2-MZLOGL-change-MZ-logging-level}
\section{MZLOGL - change the MZ logging level}

Various parts of the ZEBRA system write log messages
to the standard system output,
and occasionally also to the on-line terminal, if any.
Examples of messages provided for are:
\begin{verbatim}
   a) messages for recoverable errors:
      read errors, data errors

   b) intialization messages for:
      stores, divisions, link areas, files

   c) termination messages giving statistics of usage of
      various facilities like memory, files

   d) operation messages:
      change in program phase, end-of-file

   e) watch messages for hopefully rare, expensive events:
      garbage collection, MZPUSH with relocation

   f) monitor messages to help the user debug his program
\end{verbatim} 
To control the amount of information thus provided to the user,
a log level is defined and can be set and reset by the user
at execution time.
The default log level zero enables the messages which one would
usually like to see for record in a production run.
The user may reduce the log level to cut out most or all messages;
he may increase the level to watch the running of his program,
or even to debug his data or his input files.

Separate ZEBRA entities, such as dynamic stores or files,
each have their own attached log level,
which may be changed by the user at any time.
By default they inherit the global system-wide default log-level
set by MZEBRA, whose own default is zero.

A somewhat similar system has been used
for the debugging of the ZEBRA system itself;
the corresponding WRITE statements are still present in the code
on the PAM files, although not on the object libraries,
and could be activated after re-compilation
by setting a special log level.

(The code for generating the logging messages is conditional and can
be de-selected at generation time of the ZEBRA binary library.
This is controlled by PATCHY conditionals:
\begin{verbatim}
   +USE, QPRINT, T=INH.  deselects all messages
   +USE, QDEBUG, T=INH.  deselects all messages at or above level 2
   +USE, QDEVZE.         selects the messages for debugging Zebra. )
\end{verbatim} 

The log level attached to a particular dynamic store is initialized
by MZSTOR, normally to the global default log level.
The user may change and re-change it at any time with:

\ROUTA{CALL MZLOGL (IXSTOR,LOGL)}
\begin{verbatim}
with
      IXSTOR  index of the store,
              zero for the primary store

        LOGL  the desired log level, as shown in the following
              table, which also shows which MZ routines print
              at this (or higher level):

     level  -3:  no log messages at all

            -2:  error messages        ZFATAL, ZPHASE

            -1:  terse logging         MZEBRA, MZSTOR, ZPHASE

             0:  normal logging        MZDIV, MZLINK

            +1:  log to watch          MZLINT, MZGARB, MZPUSH

            +2:  log to monitor calls to ZEBRA
                  MZLINT, MZWORK, MZBOOK, MZLIFT, MZDROP,
                  MZPUSH, MZREPL, MZGARB, MZLOGL

   ( Messages to debug the ZEBRA system itself:

      giving  LOGL = 100+n  sets the log-level to MIN(n,2)
                            and the debug print level to "n"

                   this is not normally available !          )
\end{verbatim} 

\Filename{H2-MZSTOR-initialize-dynamic-store}
\section{MZSTOR - initialize a dynamic store}

A call to \Rind{MZSTOR} is required to initialize the dynamic store
before any operation using this store.

ZEBRA can handle up to 16 different dynamic stores. Each such store
must reside in a Common block, not in a local vector.
Each store must be intialized by calling MZSTOR once,
and once only for this store.
The first store initialized is the primary store,
its store-index IXSTOR will be zero.
Further secondary stores may be initialized,
their IXSTOR will be allocated non-zero values by MZSTOR.

In the call to MZSTOR the user specifies the first and the last
word of his dynamic store, he indicates the number and kinds
of permanent links contained at the beginning of the store,
and he communicates the FORTRAN name of the common block he
is using for this store for printing purposes.
A "fence" region of at least one word must be reserved preceding
the store to allow catching errors due to using L=0.

MZSTOR initializes the store with 3 divisions:
the forward "working space" division~1
followed by the reverse division 2, which is the default division
in many instances, and the reverse "system" division
at the end of the store.

For each secondary store, the system needs an area of
about 400 words to hold the system tables for this store.
This area is allocated by MZSTOR on the last words
of the dynamic store itself.

The use of several dynamic stores introduces an execution time
overhead proportional to the number of times that ZEBRA has to
operate in a store other than the "current" one.
All ZEBRA routines check on the current store being the right one;
if not, a call to MZSDIV changes the "environment".
A normal application uses only one store, the primary store.
Apart from allowing to point from any data to any other,
this also saves having to carry the store index,
which is simply zero.

\lile{-4mm} %!!!

A given dynamic store is initialized by

\ROUTBM{ CALL MZSTOR (IXSTOR*,chNAME,chOPT,FENCE}
       {+,            LQ(1),LQ(LR),LQ(LW),LQ(LIM2),LQ(LAST))}
\begin{verbatim}
with

     IXSTOR*  returns the index of the store, to be used when specifying
              this store in subsequent calls to the ZEBRA system.
              The indices for the divisions 1 + 2 created by MZSTOR are:
              for division 1:  IXDIV = IXSTOR + 1
                           2:  IXDIV = IXSTOR + 2

              IXSTOR will be set to zero for the primary store.

      chNAME  name of the store for printing purposes,
              8 characters maximum

      chOPT  character string of options:

            log:     set log level to the default set up by MZEBRA
                   Q  quiet, set log level to -2

       FENCE  safety area immediately in front of the store to
              protect against illegal use of LQ(0), LQ(-1), etc.
              The fence extends from FENCE(1) to LQ(0).

       LQ(1)  first word of the dynamic store

      LQ(LR)  first permanent reference link, if any

      LQ(LW)  first word in the store after the permanent links,
              this is the first word available to the working space,
              or to division 1. The following words are allocated
              as permanent user links, if any:

                LQ(1)  to LQ(LR-1)  structural links, if any
                LQ(LR) to LQ(LW-1)   reference links, if any

    LQ(LIM2)  lowest position of the end of division 2,
              to protect divisions 1 and 2 from being squeezed out
              of existence by other divisions created later.

    LQ(LAST)  last word of the dynamic store.

Required:

  -  the fence must have one word at least, but at most 1000 words.
  -  the data region of the store (ie. the total store minus
     the permanent links) must not be less than 2000 words.
\end{verbatim} 

The store is allocated by MZSTOR as follows:

\begin{verbatim}
 _________________________________________________________________ _ _ _ _
|       |        permanent       | div |         | div | division |
| fence | structural | reference |  1  | reserve |  2  |  system  |
|       |          links         | --> |   area  | <-- |   <----  | [table]
|_______|____________|___________|_____|_________|_____|__________|_ _ _ _
|       |(1)         |(LR)       |(LW) |         |     |    (LAST)|
 FENCE   LQ                                                       [or LAST]
\end{verbatim} 

The fence region is preset to contain IQNIL in every word,
this must never be changed;
the debug aids will check for over-writing.
The permanent links are cleared to zero; the rest of the store is not
touched.

The use of division 1 is somewhat special as explained
in para.~1.07 for MZWORK.

\ul{Examples}:
\begin{verbatim}
for a normal primary store:

      PARAMETER   (LIM2Q=40000)
      PARAMETER   (NNQ=120000)
      DIMENSION    LQ(NNQ), IQ(NNQ), Q(NNQ)
      EQUIVALENCE (LSAFE,LQ(1)), (Q(1),IQ(1),LQ(9))

      COMMON //    IXDIV1, ...            division indices
     +,            FENCE(16),LSAFE(10)    ten unused links for safety
     +,            LMAIN, ...             more structural links
     +,            L1, ...                more reference  links
     +,            DIV12(99)

      CALL MZSTOR (0,'//','.',FENCE,LQ,L1,DIV12,Q(LIM2Q),Q(NNQ))

for a secondary store without permanent links:

      DIMENSION    LZ(40000), IZ(40000), Z(40000)
      EQUIVALENCE (Z(1),IZ(1),LZ(9))

      COMMON /ZDYN/IXSTZ,IXDV1,IXDV2,IXHIT
     +,            FENDZ(16),LZ,LASTZ

      CALL MZSTOR (IXSTZ,'/ZDYN/','.',FENDZ,LZ,LZ,LZ,LZ(30000),LASTZ)
      IXDV1 = IXSTZ + 1
      IXDV2 = IXSTZ + 2
\end{verbatim} 
{\bf Note:}  To ease the use of double-precision variables
in the working space,
the number of words in the fence,
more generally: the number of words in the Common block
preceding the dynamic store,
as well as the number of permanent links,
should be {\em even}.

MZSTOR prints at log level -1 or above
an initialization message on unit IQLOG,
showing the whereabouts and the sizes of the store.

\Filename{H2-MZDIV-create-new-division}
\section{MZDIV  - create a new division}

MZDIV may trigger relocation.

A dynamic store is physically subdivided into 'divisions'.
Up to 20 divisions are allowed, which permits 17 divisions
created by the user with MZDIV, beyond the 3 divisions
created initially by MZSTOR.

Dividing the store into divisions is a device for keeping
different data-structures in physically separate parts
of the dynamic store.
In principle the user does not need to care where and in what
order his banks are kept physically in the store,
since all logical relations are described by links,
not by arrangement.
The reason for using divisions in spite of this general principle
is exclusively the possible gain in efficiency,
when deleting a whole data-structure, for example,
or with the output of a data-structure to tape or disk.
This may be seen by comparing the operations necessary to output
either a data-structure whose banks co-exist intermixed with
the banks of other structures, or a data-structure which has
the exclusive use of a division:
the first case requires a logical walk through the data-structure
to identify all the banks which belong to it,
plus the construction of a table to indicate their where-abouts,
and the write-out of the memory according to that table;
in the second case a contiguous chunk of memory
can be written out directly.

Mode of a division:  depending on whether a division grows
at its higher or at its lower end,
a division is said to be of mode 'forward' or 'reverse'
(MODE = 0 or 1).
Reverse mode is selected by the R option in the call to MZDIV,
else forward mode is assumed.
By arranging a forward division to be followed by a reverse
division (of the same kind),
the 'reserve areas', ie. the reserved space for these divisions
not currently allocated to banks,
of the two divisions is contiguous and is hence available
to either division,
thereby reducing the total memory requirement in general.

Kind of a division:  Depending on its usage,
we distinguish three kinds of divisions (apart from the
system division, which is used by the ZEBRA system itself):

 -- a 'user short-range' division is a division which is
wiped clean after every event;

 -- a 'user long-range' division is a division which carries
data living beyond several events,
or many events,
or until the termination phase;

 -- a 'package' division is a division used by some service
packages,
and whose contents are not normally of any concern to the user.

Links, including structural links,
may in general point from one division to any other division
of the same store,
except that the link 'next-of-same' may not,
ie. a linear stucture must be contained within one division.
To reduce the load on relocation,
the user has the possibility to indicate which division points
to which other divisions,
as explained in the para.~1.14 for MXREF.
By default ZEBRA assumes that all user divisions may point
to all other user divisions.

User short-range divisions are allocated one after the other
from left to right, starting with division 1;
long-range and package divisions
are allocated one before the other from right to left,
starting with division 20, pushing the system division downwards.

A given division in a particular store is identified by its 'index';
thus if a bank is to be created in this division,
its index has to be specified to MZLIFT.
The division index carries the store-number and the division-number
encoded onto a word of 32 bits;
when a new division is created by a call to MZDIV the next free
division-number is allocated to it,
and the encoded division index is returned to the user as
an output parameter.
The store index, constructed by MZSTOR, is formally a division index
for the (non-existing) division zero.
The format of the division index permits the simultaneous selection
of several divisions by a 'compound index',
see para.~1.15 (MZIXCO) for the specifications.

A division is created with:

\ROUTA{CALL MZDIV  (IXSTOR,IXDIV*,chNAME,NW,NWMAX,chOPT)}

\begin{verbatim}
 with
      IXSTOR  index of the store for creation,
              zero for the primary store

      IXDIV*  returns the index of the division created,
              to be used when specifying this division
              in subsequent calls to the ZEBRA system.

      chNAME  name of the division for printing purposes,
              8 characters maximum

          NW  number of words to be allocated to the division initially,
              during execution later the division may grow, but not beyond
       NWMAX  maximum size of the division

       chOPT  character string of options:

             mode:     forward mode is default    (gives MODE = 0
                     R  reverse division                  MODE = 1
                     M  match the neighbour division      MODE = 0 or 1

             kind:     user short-range is default       KIND = 1
                     L  user long-range division          KIND = 2
                     P  package division                  KIND = 3 )
                        (P implies C, over-rules L)

             xref:     by default all user divisions point to all other
                                   user divisions (cf. para. 1.14)
                     C  division contained, ie. no links point outside

Required:     NW at least 100 words,  NWMAX at least NW words

\end{verbatim}

\ul{Examples}
\begin{verbatim}
      CALL MZDIV (0,IXHITS,'HITS',10000,20000,'.')

           creates a user short-range division HITS in forward mode.

      CALL MZDIV (0,IXCALI,'CALIB',8000,8000,'RL')

           creates a user long-range division CALIB in reverse mode.
\end{verbatim} 

If the user creates several divisions it is economic to create
pairs of forward/reverse divisions.
With Zebra version 3.67 the mode option M has been introduced
to request automatic pairing with the divisions already existing.

\ul{Example}
\begin{verbatim}
      CALL MZSTOR (0, '//', ...
      CALL MZDIV  (0, IXABC, 'ABC', NW3,  NWMAX3, 'M')
      CALL MZDIV  (0, IXDEF, 'DEF', NW4,  NWMAX4, 'M')
      CALL MZDIV  (0, IXXYZ, 'XYZ', NW20, NWMX20, 'LM')
      CALL MZDIV  (0, IXUVW, 'UVW', NW19, NWMX19, 'LM')
\end{verbatim} 

This will give a primary store with this lay-out:

\begin{verbatim}
   | div 1      div 2 | div 3      div 4 |   div    | div 19      div 20 |
   |                  |                  |  system  |                    |
|  | ----->    <----- | ----->    <----- |     <--  | ------>    <------ |
|__|__________________|__________________|__________|____________________|
\end{verbatim} 

The forward divisions 3 and 19 are followed by the reverse
divisions 4 and 20.
The divisions of each pair share the same memory region,
originally of NW3+NW4 and NW19+NW20 words.
Thus the occupied space of one division can be large
(even larger than its own declared maximum) at a particular
moment during execution,
if the space occupied by the other division of the pair is small
enough to keep the sum below the maximum.

{\large\bf Remember}: the higher logical entity above the bank
is the 'data-structure' and {\large\em not} the division;
the division is a physical concept entirely different
from the logical concept of the data-structure,
and the two must not be confused.

MZDIV prints at log level 0 or above
an initialization message on unit IQLOG.

\Filename{H2-MZLINK-MZLINT-initialize-link-area}
\section{MZLINK / MZLINT - initialize a link area}

MZLINK / MZLINT may trigger garbage collection.

A link area is a vector of links outside any dynamic store,
with all its links pointing to one particular store,
consisting of NS structural links followed by NR reference links.
Either NS or NR may be zero.

We distinguish two kinds of link areas:

\subsection*{Permanent link area}

A permanent link area is initialized once at the steering level
and stays alive for the whole program;
it consists of just the vector of links.

\subsection*{Temporary link area}

A temporary link area is requested and de-activated
by lower-level code any number of times.
Each such area has two words pre-fixed to the link-vector
for efficiency:

\begin{DLtt}{123456}
\item[word 1] is a flag indicating whether the area is active or not;
              if this word is zero, the contents of the area will
              not be updated in case of a memory move.
              The user may reset this word to zero to de-activate the area.
\item[word 2] is a key allowing the system to easily find the
              where-abouts of this area in the system tables without searching.
              This word must never be touched by the user.
\end{DLtt}

A link area must be in COMMON storage;
if it is in local storage there is a danger that FORTRAN
optimization causes a register to preserve the old value of a link
across a relocation operation,
for garbage collection,
but also for simple updating with MZDROP, MZSHUNT, etc.

As for links in banks, a structural link may only contain zero or the
valid address of a bank; it must never contain garbage.

To initialize a permanent link area, one calls once,
and once only for this area:

\ROUTA{CALL MZLINK (IXSTOR,chNAME,!LAREA,!LREF,!LREFL)}
\begin{verbatim}
with

      IXSTOR  index of the store into which the links will point,
              (IXDIV of any division in this store allowed)
              zero for the primary store

      chNAME  name of the FORTRAN Common in which the link area resides,
              for printing purposes, 8 characters maximum

      !LAREA  first word of the link area,
              being also the first link of this area

       !LREF  first reference link, if any;
              last structural link, if no reference links

      !LREFL  last reference link, if any,
              if none: give LAREA in this parameter position
\end{verbatim} 

MZLINK will clear the links to zero.

\ul{Examples}
\begin{verbatim}
mixed link area:

      COMMON /LAMIX/ LS1,...,LSN, LR1,...,LRN

      CALL MZLINK (0,'/LAMIX/',LS1,LR1,LRN)

structural link area:

      COMMON /LASTR/ LS1, ..., LSN

      CALL MZLINK (0,'/LASTR/',LS1,LSN,LS1)

reference link area:

      COMMON /LAREF/ LR1, ..., LRN

      CALL MZLINK (0,'/LAREF/',LR1,LR1,LRN)
\end{verbatim} 
Note that in a permanent link area with exactly one link
this link cannot be a reference link.

MZLINK prints at log level zero (or above) an initialization
message on unit IQLOG.

To activate a temporary link area, one calls with:

\ROUTA{CALL MZLINT (IXSTOR,chNAME,LAREA,!LREF,!LREFL)}
\begin{verbatim}
with

      IXSTOR  index of the store, as for MZLINK

      chNAME  name of the link area, as for MZLINK

       LAREA  first word of the link area,
              with: LAREA(1) the user flag word
                     LAREA(2) system word
                     LAREA(3) the first link of the area

       !LREF  first reference link, if any, as for MZLINK

      !LREFL  last reference link, if any, as for MZLINK
\end{verbatim} 

MZLINT will clear the links to zero,
set the flag-word LAREA(1) to be non-zero,
and set the system-word LAREA(2) on first contact.

To de-activate a temporary link-area the user sets LAREA(1)=0.
From then on the links in this area are no longer relocated,
and hence will be meaningless.
To re-activate the area the user could set LAREA(1)=1,
but he must then clear the contents of the links himself;
it is safer to call MZLINT, which will do the necessary.

\ul{Examples}
\begin{verbatim}
mixed link area:

      COMMON /LAMIX/ LAMIX(2), LS1,...,LSN, LR1,...,LRN

      CALL MZLINT (0,'/LAMIX/',LAMIX,LR1,LRN)

structural link area:

      COMMON /LASTR/ LASTR(2), LS1, ..., LSN

      CALL MZLINT (0,'/LASTR/',LASTR,LSN,LASTR)

reference link area:

      COMMON /LAREF/ LAREF(2), LR1, ..., LRN

      CALL MZLINT (0,'/LAREF/',LAREF,LR1,LRN)
\end{verbatim} 
MZLINT prints a log message at level 1 for initialization
and at level 2 for re-initialization.

\Filename{H2-MZWORK-allocate-working-space}
\section{MZWORK - allocate working space}

MZWORK may wipe division~1 and it may trigger garbage collection.

The region at the beginning of the dynamic store just after
the permanent links may be used as working space,
consisting of a set of reference links followed by
a set of data words.

The user requests reservation of the working space with

\ROUTA{CALL MZWORK (IXSTOR,DFIRST,DLAST,IFLAG)}
\begin{verbatim}
with  IXSTOR  index of the store or of any of its divisions,
              zero for the primary store

      DFIRST  first data word of the working space,
              the preceding words are taken as reference links;
              this parameter is ignored if IFLAG is =2 or =-1

       DLAST  last data word,
              this parameter is ignored if IFLAG is =-1 or =3 or =4

       IFLAG  = 0  define a new working space,
                   previous contents are not to be retained

              = 1  vary the limits of the working space,
                   keeping intact the links which are common to
                   both the old and the new working space.

              = 2  vary only the DLAST limit of the working space,
                   keeping alive all links and the data words
                   which are common to the old and the new working space.

              = 3  keep the DLAST limit unchanged, keep division 1,
                   re-define the DFIRST limit for new links,    
                   ie. clear the links to zero.             

              = 4  keep the DLAST limit unchanged, keep division 1,
                   re-define the DFIRST limit keeping intact the links
                   which are common to the old and the new working space.

              =-1  reset the working space to null,
                   ie. to zero links, zero data words
\end{verbatim} 
Staring with Zebra version 3.67, the links of the working space
are cleared to zero by \\
MZWORK;
if IFLAG= 1 or 4 only the new links are cleared.


A call to MZWORK with ${\tt IFLAG < 3}$ destroys division 1 of the store
{\em without} a relocation pass to reset links pointing into division 1.
This is done in this way for efficiency,
hence normally only very temporary data should be kept in division 1,
and only working space links should point to them.
To say it differently: division 1 is logically part of the
working space,
its time of existence is the same as that of the working space,
and it is good practice to maintain links into division 1
only in the working space.
If however it is necessary to hold such links elsewhere,
one should either reset them explicitly or wipe the division with
{\tt CALL~MZWIPE~(1)} before calling MZWORK
[ or with {\tt CALL~MZWIPE~(IXSTOR+1)} for a secondary store ].

\ul{Example}
\begin{verbatim}
+CDE, Q.
     +,         LR1, ...     working space reference links
     +,         DFIRST, ...  working space data words
     +,         DLAST

      CALL MZWORK (0,DFIRST,DLAST,0)
\end{verbatim} 
will give this layout of the store:

\begin{verbatim}
    _____________________________________________________________
   |           |       |             |               |           |
   | permanent |  wsp  |   w. space  | div 1   div 2 | other     |
   |   links   | links |     data    | --->     <--- | divisions |
   |___________|_______|_____________|_______________|___________|
   |LQ         |LR1    |DFIRST  DLAST|                           |
\end{verbatim} 

MZWORK prints a monitor log message at level 2.

\Filename{H2-MZLIFT-MZBOOK-bank-creation}
\section{MZLIFT / MZBOOK - bank creation}

MZBOOK / MZLIFT may cause garbage collection.

A bank may be lifted with either

\ROUTBM{CALL MZBOOK (IXDIV,!L*,!LSUP,JB, chID,NL,NS,ND,IOD,NZERO)}
       {CALL MZLIFT (IXDIV,!L*,!LSUP,JB, NAME, NZERO)}
\begin{verbatim}
with
        IXDIV   index of the division in which the bank is to be lifted
                 = 0       use default in   primary store
                [= IXSTOR  use default in secondary store ]
                see below for details

          !L*   returns the address of the bank created

        !LSUP   if JB < 0:  address of the supporting up bank
                if JB = 0:  address of the supporting previous bank
                if JB = 1:  supporting link*
                if JB = 2:  LSUP not used

           JB   if JB < 1:  link bias in the supporting bank
                             (must be a structural link)
                if JB = 1:  create top-level bank
                if JB = 2:  create stand-alone bank

         NAME   specifies the properties of the bank:

 NAME(1) = ID:  4-character bank identifier, alphanumeric,
                  MZBOOK: variable of type CHARACTER
                  MZLIFT: Hollerith string of the form 4Hxxxx
     (2) = NL:  total number of links, NL < 64k
     (3) = NS:  number of structural links, NS <= NL
                  (not counting the 3 links next, up, origin)
     (4) = ND:  number of data words, ND < 1 Mwords

     (5) = IOD:  I/O descriptor for the data words, which may be:
                 - result from MZFORM, 1 word, I/O index
                 - result from MZIOBK, 1 or more words, I/O characteristic
                 - immediate, the whole bank is of the same type:
                    IOD = 0  undefined, bank cannot be transported
                          1  32-bit unsigned integer
                          2  signed integer
                          3  floating         4  double-precision
                          5  Hollerith        7  self-descriptive
                 - requests:
                    IOD = 9  retrieve I/O characteristic by IDH,
                             stored in the system by MZFORM
                         11  take the I/O characteristic from any
                             bank in the target linear structure,
                             if this is empty act as for 9
\end{verbatim}

\begin{verbatim}
        NZERO   controls whether and how much of the data-part
                of the bank is preset to zero:
                   N = -1:  no presetting
                   N = 0:  the whole bank is cleared
                   N > 0:  the first N words are cleared
                The links of the bank are always cleared to zero.
\end{verbatim} 
In what follows we will need the parameter {\bf LNEXT},
which is the address of the successor bank to the bank to be lifted;
its address will be stored into link 0 of the new bank,
ie. into its 'next' link.
LNEXT is obtained as follows:
\begin{verbatim}
        if  JB < 1:   LNEXT = LQ(LSUP+JB)
            JB = 1:   LNEXT = LSUP
            JB = 2:   LNEXT = 0
\end{verbatim} 
The {\em division} [ and the store ] into which the bank
is to be lifted is selected by the parameter IXDIV.
If a definite division is specified, this is used,
but it must be compatible with LNEXT;
a linear structure must be contained within one and the same
division !

If 'default division' is specified,
by giving zero [ or IXSTOR ] for IXDIV,
the bank will be created in the division holding the logically
nearest bank of the data structure into which the new bank
will be inserted, in this order of priority:

\in{4mm}
\begin{verbatim}
  -  if the bank is added to an existing, non-empty, linear structure,
     i.e. if LNEXT is non-zero, or if JB=0,
     use the division holding this linear structure;
  -  if JB < 0 use the division holding the supporting up bank at LSUP;
  -  use division 2.

\end{verbatim} 
\in{-4mm}
The {\em numeric bank identifier} IDN serves to readily identify
a particular bank within a linear structure for interactive work.
Normally the Hollerith identifiers IDH in a linear structure are
all identical;
it is desirable that the IDN's are all different.
In principle the user can set any positive integer into IQ(L-5) after
the bank has been created, but MZLIFT/MZBOOK will store a default
value in this order of priority:

\in{4mm}
\begin{verbatim}
  -  if LNEXT is non-zero: use IDN+1 of the bank at LNEXT;

  -  if JB = 0: use IDN+1 of the bank at LSUP;

  -  if JB < 0: use the value IDN = -JB;

  -  IDN = 1 will be set.
\end{verbatim} 
\in{-4mm}

The bank being lifted is
{\em inserted}
into an existing data-structure according to LSUP
under control of the parameter JB:

\begin{verbatim}
--- case 1: JB < 0       insert as dependent of the bank at LSUP
                          ex.:  CALL MZLIFT (0,L,LSUP,-4,NAME,0)
            _______
           |       |
         __|  SUP  |  <-------------------------.------ ...
        |  |       |  <---------.               |
  link  |  |_______|            |               |
JB= -4  |                       |               |
        |                       |               |
        |                    up |            up |
        |                  ________        ________
        |                 |        |      |        |
        `---------------> |  lift  | ---> |  NEXT  | --->  ...
                          |________|      |________|

  SUP   is the supporting bank whose address is LSUP
  lift  is the new bank being lifted
  NEXT  is the first bank of the linear structure in front of which
        the bank 'lift' is inserted, ie. the initial content of
        link JB in bank SUP was LNEXT = LQ(LSUP+JB).

The up-link and the origin-link in the bank lifted are set as

                  up-link  <--  LSUP
              origin-link  <--  LSUP+JB


--- case 2: JB = 0       insert inside a linear structure
                          ex.:  CALL MZLIFT (0,L,LSUP,0,NAME,0)
     ______
    |      |
  __|  UP  |  <------------------------------------------.---- ...
 |  |      |  <------------.                             |
 |  |______|               |                             |
 |                         |<-------------.              |
 |                         |              |              |
 |                      up |           up |           up |
 |      ____          ________       ________       ________
 |     |    |        |        |     |        |     |        |
 `---> |    | -...-> |  SUP   | --> |  lift  | --> |  NEXT  | --> ...
       |____|        |________|     |________|     |________|


The up-link and the origin-link in the bank lifted are set as

                  up-link  <--  copy of the up-link in bank SUP
              origin-link  <--  LSUP
\end{verbatim} 
SUP, lift, NEXT have the same significance as in case 1.
UP is the bank pointed to by the up-link in bank SUP,
which link might be zero.
If bank NEXT does not exist, ie. if LNEXT = LQ(LSUP) is zero,
the bank 'lift' is added at the end of the linear structure.

\begin{verbatim}
--- case 3: JB = +1      create new top-level bank
                          ex.:  CALL MZLIFT (0,L,LSUP,1,NAME,0)

              usually: zero  <------------------.
                                                 |
                                                 |
                                  -------------->|
                                 |               |
                              up |            up |
                            ________        ________
                           |        |      |        |
 result  LSUP and L  ----> |  lift  | ---> |  NEXT  | ---> ...
                           |________| .--> |________|
                                      |
  input  LSUP  -----------------------^
\end{verbatim} 
For  JB=+1  the origin-link and the up-link are taken from
the bank NEXT pointed to by LSUP on input;
the first time round, when NEXT does not exist,
ie. when LSUP=0,
the up-link is set to zero and the origin-link is made to point
to the location containing LSUP.

The location pointed to by the origin-link is filled with the address
of the bank created, unless the origin-link is zero.
\begin{verbatim}

--- case 4: JB = +2      no inserting at all
                          ex.:  CALL MZLIFT (0,L,0,2,NAME,0)
\end{verbatim} 
For  JB=2  the bank is lifted without being linked into a structure;
both the origin-link and the up-link in the bank are set to zero. \vspace{8mm}

{\large\bf Note}: because structural links are used to connect banks as
illustrated by these examples, and are hence manipulated by the system,
they must never contain anything but valid bank addresses, or zero.

MZLIFT and MZBOOK print a monitor log message at level 2.

\Filename{H2-MZPUSH-alter-size-of-bank}
\section{MZPUSH - alter the size of a bank}

MZPUSH (increasing) may cause garbage collection.

To increase or decrease the size of a bank, one may

\ROUTA{CALL MZPUSH (IXDIV,*!L*,INCNL,INCND,chOPT)}
\begin{verbatim}
with

       IXDIV  index of the division; zero [or IXSTOR] allowed,
              in which case MZPUSH will find the division

        *!L*  address of the bank to be pushed,
              on return L contains the new address of the bank
              (cannot be a link in a bank !)

       INCNL  number of additional links,
              zero for no change, negative for decrease.
              Additional links will be given type 'reference',
              unless the original bank has only structural links.

       INCND  number of additional data words,
              zero for no change, negative for decrease.

       chOPT  character string of options:

                 default: any link may point to the bank at L,

              R  but not into the abandoned bank region
                        (in case of bank reduction)

              I  isolated: only the inherent structural links
                           point to this bank
\end{verbatim} 
For 'isolated' the user is sure that no link other than the
supporting structural link, and the link passed in the parameter L,
and the reverse links in the first level dependents,
point to this bank.

The I/O descriptor of the old bank must also be valid for the
new bank; if this would not be the case one should use MZREPL instead
of MZPUSH.

New link words and new data words acquired by the bank
are cleared to zero.

Except for special cases,
increasing the size of a bank is an expensive operation,
and should be avoided.
The correct solutions to the problems of indefinite number of objects
or of indefinite size are either the use of a linear structure
or the lifting of a bank of maximum size,
which is reduced to the final size once the contents are complete.

The increase of a bank is handled in general as follows:
a new bank is lifted with the new dimensions,
the contents of the original bank are copied over,
and the old bank is marked as dropped.
Any link pointing to the original bank must be re-routed
to the replacement bank by MZPUSH.
In full generality this can only be done by a relocation pass
over all memory (ie. all link areas and all divisions pointing
to the division in which the bank resides.)

The expensive part in this operation is the relocation pass,
and this can be saved under special circumstances:

 a) Increasing the data part of the last bank in a forward division,
or the link part in a reverse division,
can be done in situ without the lifting of a new bank.
Hence any link pointing to the bank remains valid.
(However note: this is not a good reason for creating an extra
division, because the lifting of a maximum size bank is
a better solution.)

 b) If there are no links pointing to the bank (except the
standard structural support links)
there is no point to scan all links,
because the relocation pass would have no effect.
However, ZEBRA could not detect this situation without actually
performing the relocation pass,
but the user may know the situation a priori,
in which case he can signal it to MZPUSH with the I option.

Reducing the size of a bank is less onerous (if
the R option is given).
The parameters of the original bank are adjusted to the
new dimension in situ and the abandoned memory is marked as dead.
Links which point into the abandoned region, if any,
can only be reference links and must be reset to zero.
To have links pointing into a bank, rather than to the status-word,
is relatively rare.
Again, the relocation pass can be saved,
and for reducing a bank this is the normal situation,
if the user knows the situation a priori and signals it to
MZPUSH with the R option (or I which is stronger).

MZPUSH prints a monitor log message at level 2,
but if the operation involves a relocation pass a log message
will be given at level 1.

MZPUSH accumulates statistics of the number of times that an
expensive operation was performed;
this is printed by MZEND.

\Filename{H2-MZREPL-connect-replacement-banks}
\section{MZREPL - connect replacement banks}

If one wants to replace a given 'old' bank in a data-structure
by a 'new' bank,
and if there are links elsewhere in the data-structure(s)
or in link-areas pointing to the old bank,
this is a non-trivial operation because these links
have to be relocated to point to the new bank.
Normally this can be done with MZPUSH,
but not if the I/O descriptor has to change.
For these cases MZREPL is provided.

The user is supposed to lift and fill the 'new' bank and also a
tiny 'index' bank with 2 (structural) links and 1 data word,
recording the address of the old bank in link 1 and the address
of the new bank in link 2.
This index bank is handed to MZREPL which will then execute
the necessary relocation, using the data word as working storage.

If several such operations are needed one should not loop
over calls to MZREPL, but construct a linear structure of
index banks for MZREPL, which will then do all the replacements
with one single relocation pass.
Because of this feature MZREPL can be more economical than
MZPUSH for multiple replacements.

Both the old and the new banks are required to be in one single
division, but the index banks may be in a different division.

\ROUTA{CALL CALL MZREPL (IXDIV, *!LIX*, chOPT)}
\begin{verbatim}
with
       IXDIV  index of the division; zero [or IXSTOR] allowed,
              in which case MZREPL will find the division

      *!LIX*  the address of the linear structure of the index banks.

       chOPT  character string of options:

                 default: drop the old and the index banks

              K  keep the old and the index banks

              I  isolated, as for MZPUSH
\end{verbatim} 

If one needs to keep the old banks beyond the call to MZREPL,
note this: really all links pointing to the 'old' banks are
relocated, and therefore one would loose access to the old banks.
To overcome this the K option selects a special processing:
before going into the relocation the link pointing to the
old bank is saved into the data-word of the index bank
(this works because there is no garbage collection involved);
just before returning to the user all index banks are changed
to have only 1 structural link restored to support the old bank,
and link 2 continues to point to the corresponding new bank,
but now as a reference link.

Each old bank is changed to have zero structural links,
the 'next' link is reset to zero, and the bank is made to
depend from its index bank.

When one has finished with the old banks one can drop the
structure of index banks, thereby also dropping the old banks.

The linear structure of index banks is re-ordered by MZREPL for
increasing address of the old banks;
LIX returns the address of the new first bank.


\ul{Example}
\begin{verbatim}

   Replace all the banks of a linear structure.

   LIX, LOLD, LNEW  are temporary links,
   either in the working space or in a link area.

   IXDIVA is the division index of the banks;
   IXDIVX is the index of some short range division to hold the
          index banks, division 1 by preference.

      LIX = 0    clear LIX to start (reference link allowed)

      LOLD = get the adr of the first old bank

   24 CALL MZBOOK (IXDIVX, L, LIX,1, 'IXIX',2,2,1,0, -1)
      LQ(LIX-1) = LOLD

      CALL MZLIFT (IXDIVA, LNEW, LIX,-2, ...

          fill the new bank at LNEW
               ie. links LQ(LNEW-NL) to LQ(LNEW-1)
                    data IQ(LNEW+1)  to IQ(LNEW+ND)
               but not 'next', 'up', 'origin'

      LOLD = LQ(LOLD)
      IF (LOLD.NE.0)         GO TO 24

      CALL MZREPL (IXDIVA, LIX, '.')
\end{verbatim} 

MZREPL prints a monitor log message at level 2.

\Filename{H2-MZDROP-drop-bank-and-dependents}
\section{MZDROP - drop a bank and its dependents}

With MZDROP one may either drop the bank at L or,
if the L option is given, the linear structure at L.
Dropping a bank implies dropping also the whole partial
data-structure which depends on this bank downwards.

Dropped banks stay in memory with the IQDROP status-bit set,
links pointing to them continue to do so,
except for the immediate structural link indicated via the
origin-link of the bank at L, which is bridged or reset to zero,
whichever is appropriate.

To drop one gives

\ROUTA{CALL MZDROP (IXSTOR, !L, chOPT)}
\begin{verbatim}
with

      IXSTOR  index of the store or of any division in this store,
              zero for the primary store

          !L  the address of the bank or linear structure to be dropped

       chOPT  character string of options:

                  default: drop the bank at L only,
                  ie. the 'next' link of this bank is not followed

               L  drop the linear structure pointed to by L
                  ie. the 'next' link of the bank at L is followed

               V  drop only the partial data-structure
                  dependent vertically downwards from the bank at L,
                  but not the bank itself.
\end{verbatim} 
MZDROP prints a monitor log message at level 2.

The parameter L in the call to MZDROP is not as such changed
on return,
but if it is the structural link supporting the d/s it will
in fact be up-dated for the removal of the first bank
in this structure.

Suppose this artifical sequence of operations (with LS=0 initially):
\begin{verbatim}
      CALL MZLIFT (0,L,LS,1,...)          create bank '3'
      CALL MZLIFT (0,L,LS,1,...)          create bank '2'
                                  LS now points to bank 2

      CALL MZLIFT (0,L,LS,1,...)          create bank '1'
                                  LS now points to bank 1

      CALL MZDROP (0,LS,'.')              drop bank 1
                                  LS now points again to bank 2

if however the dropping is done with

      CALL MZDROP (0,L,'.')               drop bank 1

         then LS will be made to point to bank 2,
         but L will continue to point to the (dead) bank 1.
\end{verbatim}
Since the parameter L in the call to MZDROP is not changed,
selective dropping of banks in a linear structure can be done
with a loop like this:
\begin{verbatim}
      L = start adr of the linear structure
   12 IF (L.NE.0)  THEN
          IF (bank not wanted)  CALL MZDROP (0,L,'.')
          L = LQ(L)
          GO TO 12
        ENDIF
\end{verbatim}

\Filename{H2-MZWIPE-reset-complete-divisions-empty}
\section{MZWIPE - reset complete divisions to empty}

MZWIPE causes relocation normally.

By calling MZWIPE the user can delete the contents of
whole divisions:

\ROUTA{CALL MZWIPE (IXWIPE)}

with the parameter IXWIPE indicating the divisions
of a particular store to be wiped out.
IXWIPE may be any of the three possible forms of a division index:
\begin{verbatim}
  a) specific division index, as returned by MZDIV

  b) generic division index, [ IXSTOR + ] n, where

        n = 21:  all user short range divisions
            22:  all user long  range divisions
            23:  all package divisions

  c) compound division index, as created by MZIXCO, see there for details.
\end{verbatim} 

IXWIPE = 0  is taken to mean IXWIPE = 21,
ie. all short range divisions in the primary store;
similarily IXSTOR+21 can be used.

Wiping divisions resets the divisions to be empty,
but without reducing the space reserved for them,
followed by a relocation pass to reset to zero all links
pointing into the wiped divisions.
Included into this pass are the links of all link areas,
of the working space and of all divisions which are declared
to point to the divisions in question
(all of this for one particular store only, of course).

If several divisions are to be wiped out,
this must be done by {\em one} call to MZWIPE,
and not by several calls in succession,
to save the time of multiple relocation passes,
each of which would take longer than the single pass.

\ul{Examples} (for the primary store)
\begin{verbatim}
  1) Wipe the last 'event':      CALL MZWIPE (0)

  2) Wipe division 1:            CALL MZWIPE (1)

  3) Wipe all user's divisions:  IX = MZIXCO (21,22,0,0)
                                  CALL MZWIPE (IX)

  4) Wipe divisions IX1, IX2, IX3, IX4, and IX5:

                   IX = MZIXCO (IX1,IX2,IX3,IX4)
                   IX = MZIXCO (IX,IX5,0,0)
                   CALL MZWIPE (IX)
\end{verbatim} 

MZWIPE operates by calling MZGARB, which will print a monitor
log message at level 2.

\Filename{H2-MZGARB-garbage-collection-requested-by-user}
\section{MZGARB - garbage collection requested by the user}

Garbage collection is triggered by the system if there is not
enough space to satisfy an MZWORK or MZLIFT request
(hopefully this does not normally occur).
Thus the user does not have to worry about initiating
garbage collection to win space in the dynamic store.
To remove the last event from the store,
the user calls MZWIPE which is much more efficient
than dropping the banks of the event followed by garbage collection.

He may however occasionally want to force a garbage collection
to tidy up his data,
particularly during the initialization phase of his program.
Again, as in MZWIPE, if there are several divisions to be tidied up,
this must be done by {\em one} call to MZGARB.
Also,
if one or several divisions are to be wiped out at the same moment,
this should be included into the call to MZGARB;
one and the same relocation pass can handle wiping
and garbage collection simultaneously.

The calling sequence is:

\ROUTA{CALL MZGARB (IXGARB,IXWIPE)}
\begin{verbatim}
with

      IXGARB  index of the divisions where garbage
              is to be collected (none if =0)

      IXWIPE  index of the divisions to be wiped out
              = 0:  no divisions to be wiped
\end{verbatim} 

Both IXGARB or IXWIPE may be any of the three possible forms
of a division index:
\begin{verbatim}
  a) specific division index, as returned by MZDIV

  b) generic division index, [ IXSTOR + ] n, where

        n = 21:  all user short range divisions
            22:  all user long  range divisions
            23:  all package divisions

  c) compound division index, as created by MZIXCO, see there for details.
\end{verbatim} 

MZGARB prints a monitor log message at level 2,
but if the operation involves a garbage collection
a message is given at level 1.

\Filename{H2-MZXREF-set-cross-reference-between-divisions}
\section{MZXREF - set cross-reference between divisions}

To save time when wiping a given division (or divisions),
and also on garbage collection,
ZEBRA will relocate the links of only those divisions
which reference the division(s) being changed.
To know which division may have links pointing
to which other division,
ZEBRA keeps internally a cross-reference matrix;
the entry for a given division is intialized by MZDIV
and this may be modified by the user calling MZXREF:

\ROUTA{CALL MZXREF (IXFROM,IXTO,chOPT)}
\begin{verbatim}
with

      IXFROM  index of the division which contains links pointing
              to the divisions indicated by IXTO;
              this must be the index of one particular division.

        IXTO  index of the division(s) which are referenced

       chOPT  character string of options:

            none  set reference(s), ie. overwrite the previous
                  content of the matrix entry
               A  add reference(s), ie. add to the matrix entry,
                  keeping what was there before
               R  remove reference(s), ie. take the references
                  out from the matrix entry, but keep the others
                  (R over-rules A)
               C  contained division, ie. clear the matrix entry
                  (C over-rules A and R)

      IXTO may be any of the three possible forms of a division index:

        a) specific division index, as returned by MZDIV

        b) generic division index, [ IXSTOR + ] n, where

              n = 21:  all user short range divisions
                  22:  all user long  range divisions
                  23:  all package divisions
                  24:  the system division

        c) compound division index as created by MZIXCO.
\end{verbatim} 

MZDIV creates a division with its matrix row initialized
(unless C option) as follows:
\begin{verbatim}
      user division:  references all other user divisions
      package div.:  no references at all
\end{verbatim} 

Note: if division FR contains a bank which supports structurally
banks in division TO,
then the forward links point from division FR to division TO,
but there are also the reverse links in the supported banks
which point from division TO to division FR.
As a result one would need both
\begin{verbatim}
          CALL MZXREF (IXFR,IXTO,'A')
     and  CALL MZXREF (IXTO,IXFR,'A')

\end{verbatim} 
\ul{Examples} \hspace*{4mm} (for the primary store)

\begin{verbatim}
  1) User's division IXTHIS may reference all other user divisions:

         nothing needs to be done, this is the default assumption

  2) Division IXTHIS references only banks in division 2:

                   CALL MZXREF (IXTHIS, 2, '.')

  3) Division IXTHIS references only, but maybe all,
     the short-range divisions:

                   CALL MZXREF (IXTHIS, 21, '.')

  4) Division IXTHIS references all short-range divisions
     and also the long-range division IXLONG:

                   CALL MZXREF (IXTHIS, 21, '.')
                   CALL MZXREF (IXTHIS, IXLONG, 'A')

  5) Division IXTHIS to reference all short-range divisions
     except the division IXSH:

                   CALL MZXREF (IXTHIS, 21, '.')
                   CALL MZXREF (IXTHIS, IXSH, 'R')
\end{verbatim} 
\vspace{4mm}
To print the matrix entry for one particular division,
or the complete cross-reference matrix of a store, one may

\ROUTA{CALL MZXRPR (IXDIV)}
\begin{verbatim}
with

       IXDIV  1) index of the division to be shown
              2) = IXSTOR, index of the store to be shown
\end{verbatim} 

\Filename{H2-MZIXCO-create-compound-division-index}
\section{MZIXCO - create compound division index}

A compound division index permits to indicate several divisions
of the same store in one single word,
as used with MZWIPE for example.

MZIXCO joins up to four division indices into a compound:

\ROUTA{IXCO = MZIXCO (IX1,IX2,IX3,IX4)}

If less than 4 indices are to be joined trailing zeros
should be given.
If more than 4 indices are to be joined this is done by
repeated calls.

Any input parameter IXn may take any one of the three possible
forms of a division index:
\begin{verbatim}
  a) specific division index, as returned by MZDIV

  b) generic division index, [ IXSTOR + ] n, where

        n = 21:  all user short range divisions
            22:  all user long  range divisions
            23:  all package divisions
            24:  the system division

  c) compound division index, created by a previous call to MZIXCO.

\end{verbatim} 
\ul{Examples} \hspace*{2mm}  (for the primary store)
\begin{verbatim}
  1) Compound to give divisions 1 and 2:

                   IXCO = MZIXCO (1,2,0,0)

  2) Compound to give divisions 1, 2, and IXHITS:

                   IXCO = MZIXCO (1,2,IXHITS,0)

  3) Compound to give all short-range divisions and IXCUMU:

                   IXCO = MZIXCO (21,IXCUMU,0,0)

[ 4) Compound to give all user divisions of the store
     whose index is IXSTOR:

                   IXCO = MZIXCO (IXSTOR+21,IXSTOR+22,0,0) ]
\end{verbatim} 

{\large\bf Format of the division index}
\begin{verbatim}

      bits 25, 26, 32    non-zero is illegal

      bit 31  zero:     simple index
              one:     compound index

      bits 27-30 :     the number of the store,
                         zero for the primary store

Division JDIV is selected if

     simple index:      the value of bits 1-24 is JDIV
   compound index:      bit JDIV is present

JDIV may be 1 to 20 for that particular division,
         or it may be generic with:

          JDIV = 21:  user's short range divisions
                 22:  user's  long range divisions
                 23:  package divisions
                 24:  system division
\end{verbatim} 

\Filename{H2-MZFORM--handle-IO-characteristic}
\section{MZFORM {\it et al.} - handle the I/O characteristic}

MZFORM may cause garbage collection.

The nature of the contents of any bank which is to be transported
from one computer to another one has to be indicated to ZEBRA,
such that it can do the necessary tranformations.
In the simplest case that all the data words of a bank are
of the same type, this is easily indicated in the parameters
to MZLIFT or MZBOOK.
For anything more complicated the user specifies the "format"
of the bank by calling MZIOBK or MZFORM which encode the format
into a variable number of words to be included into each bank
in the system part as the "I/O characteristic".

Thus the content description is carried by each bank;
this avoids complicated logistics of finding bank descriptors
elsewhere than in the bank itself.
Complex bank contents require a relatively large number
of extra system words.
This could represent a substantial overhead on memory
or file space occupation,
which the user can avoid in the design of his bank contents.
Anyway, the number of these extra descriptor words is limited to 16,
and any descriptor which would need more is refused.
Thus ZEBRA will not handle any arbitrary bank contents
via this basic procedure,
but by using the concept of the "self-describing" sector (see below)
the user can indeed store any kind of information mix,
decided at execution time, into a bank and have it travel
from one computer to another one.

{\large\bf Sectors}

The basic element for setting up an I/O characteristic is the sector,
which is a number of words consecutive in the bank which are
all of the same type.
A sector is described in the "format" parameter to MZFORM {\it et al.}
as a combination of its word-count "c" and its type "t" as  "ct".
For example:
24F is a sector of 24 single-precision floating-point numbers,
24D is a sector of 24 words holding 12 double-precision numbers,
and  1I is a sector of one integer word.

The possible values for "t" are:
\begin{verbatim}
      t =  B   bit string of 32 bits, right justified
           I   integer
           F   floating-point
           D   double precision
           H   4-character Hollerith
           S   self-describing sector (see below)
\end{verbatim} 
A "static" sector is a sector of a fixed number of words,
such as the 24F of the example above.

An "indefinite-length" sector is a sector whose end is defined
by the end of the bank.
This is written as  --t, for example  --F  signals that the rest
of the bank is all floating-point words.

A "dynamic" sector is a sector which is preceded in the bank
by a single positive integer word indicating the sector length;
if this number is zero this means that the rest of
the bank is currently unused.
This is written as  *t, for example  *F  indicates
a dynamic sector of type floating.


Thus the word-count "c" in the sector specification is written as:
\begin{verbatim}
      c =  n   numeric, n words:       static length sector
           -   all remaining words:    indefinite length sector
           *   dynamic length sector
\end{verbatim} 
A "self-describing" sector is a dynamic sector whose type
is also encoded into the one word preceding the sector as
\begin{verbatim}
   word 0  =  16*NW + IT

      with  NW = length of the sector
            IT = numeric representation of the type
                 =  1  B bit string    2  I integer
                    3  F floating      4  D double precision
                    5  H Hollerith
                    6  (reserve)       7  (special)
\end{verbatim} 
The form "nS" is meaningless;
the form "*S" indicates one particular sector;
the form "--S" is {\em special} in that it indicates
that the rest of the bank is filled by self-describing sectors,
as many as there may be.
(Thus the forms, for example, '4I 5F / *S' and '4I 5F --S' are
equivalent, but the second form is more economic;
the user may give either, the internal result will be the
second form.)

\vspace{4mm}
{\large\bf Complete Characteristics}

Looking now at the bank as a whole,
we divide it into a "leading part" and a "trailing part",
either of which may be empty.

The leading part consists of one region of maybe several sectors,
occurring once at the beginning of the bank.
This leading region may end with an indefinite-length sector,
in which case the trailing part is empty.

The trailing part of the bank may be empty or it may consist
of an indefinite number of regions which
{\em all have the same structure},
such that the same format description is valid for all of them.

The symbol "/" marks the break between the leading region
and the trailing regions in the format parameter to MZFORM {\it et al.}

\ul{Examples}
\begin{verbatim}
trailing part empty:

      '-F'         the whole bank is floating
      '3I -F'      the first 3 words are integer, the rest is F
      '*I -F'      the first word n=IQ(L+1) is a positive integer,
                   words 2 to n+1 are integers, the rest is F
      '3B *I -F'   the first sector consists of 3 words bit-string,
                   the second sector is dynamic of type integer,
                   the rest of the bank is floating
      '3I *F'      the first 3 words are integer, followed by a
                   dynamic sector of type F, the rest (if any) of
                   the bank is currently unused
\end{verbatim} 

\begin{verbatim}

both parts present

      '3B 7I / 2I 4F 16D'  the leading region has 3 B and 7 I words,
                           each trailing region consists of 2 integer
                           words, followed by 4 F words, followed
                           by 16 D words, ie. 8 double-precision numbers
      '4I / *H'    the bank starts with 4 integer words,
                   the rest is filled with dynamic Hollerith sectors
      '*I / 2I *F' the leading region is one dynamic I sector,
                   each trailing region consists of 2 integers
                   followed by a dynamic F sector
                   (ie. 3 integers plus a number of floating words
                    this number being indicated by the 3rd integer)

leading part empty

      '/ *H'       the bank is filled with dynamic Hollerith sectors
      '/ 4I 29F'   4 integers and 29 floating numbers alternate
\end{verbatim} 
\cleardoublepage  % continue on next odd page
\ul{Economic formats}

It is in the interest of the user to design his bank contents
such that the I/O characteristic is as simple as possible,
because the number of system words in any bank increases
with the complexity of the lay-out.
"Simple" means: as few sectors to be described as possible.
\begin{verbatim}
For example:  '2B 2I 2B 2I 2B 2I -F' is much less simple then
               '6B 6I -F'.
\end{verbatim} 
Moreover, if the integers described by this format are sure to
be positive integers, then one can use the even simpler form
{\tt '12B -F'}.

In the following we give an exhaustive list of the most
economic bank formats, those requiring zero or one extra system
word in the banks.

\ul{Zero extra I/O words}

These bank formats can be described by the 16 bits of the
I/O control-byte alone:
\begin{verbatim}
   (0)  '-t' or '*t'
        'ct -t'           if c < 64   (c=* is represented as c=0,
          or 'ct *t'                   hence  '*t -t'  is a sub-case)
   (1)  '*t *t -t'
   (2)  '*t *t *t'
   (3)  'ct / *t'         if c < 64
        '/ ct *t'         if c < 64
        '/ ct'            this is useful only if c=*
                          else the form '-t' is used
   (4)  '*t / *t *t'
        '*t *t / *t'
   (5)  '/ *t *t *t'
\end{verbatim} 
\ul{One extra I/O word}

These bank formats can be described by the 16 bits of the
I/O control-byte plus the 32 bits of one extra I/O word:
\begin{verbatim}
   (1)  'ct -t'
        'ct ct -t'        if c < 65536
        'ct ct ct -t'     if c < 1024
   (2)  'ct *t'
        'ct ct *t'        if c < 65536
        'ct ct ct *t'     if c < 1024
   (4)  'ct / ct'         if c < 65536
        'ct / ct ct'      if c < 1024
        'ct / ct ct ct'   if c < 256
        'ct ct / ct'      if c < 1024
        'ct ct / ct ct'   if c < 256
   (5)  '/ ct ct'         if c < 65536
        '/ ct ct ct'      if c < 1024        (remember:
        '/ ct ct ct ct'   if c < 256          c=0 means c=*)
\end{verbatim} 


Three routines are provided to mediate between the user
specifying the bank format in a readable form
and the highly encoded I/O characteristic
to be included into any bank at creation time.

{\large\bf MZIOCH}

analyses the format chFORM to convert and pack it into
the output vector IOWDS.
This is the basic routine,
but it is usually called by the user only to specify formats
of objects other than banks, like the user header vector for FZOUT.

To specify bank formats the following two routines serve more
conveniently:

{\large\bf MZIOBK}

is provided for the context of MZLIFT;
like MZIOCH it analyses the format chFORM,
but it stores the result as part of the bank-description vector
NAME for MZLIFT.

{\large\bf MZFORM}

again analyses the format chFORM,
but it does not return the result to the user.
Instead,
it remembers the I/O characteristic in a system data-structure,
returning to the user only the index to the characteristic
in the system.
The user may then either pass this index to MZBOOK (or MZLIFT)
at bank creation time,
or alternatively he may request MZBOOK (or MZLIFT) to search
the system data-structure for the I/O characteristic
associated to the Hollerith identifier IDH of the bank to be
created.

The first word of the I/O characteristic delivered by MZIOCH or
MZIOBK has the following redundant format:
\begin{verbatim}
          |       16       |    5    |     5     | 6 bits|
          |----------------------------------------------|
          |  control-byte  |  NWXIO  |  NWXIO+1  |     1 |
          |----------------|---------|-----------|-------|

\end{verbatim}
The I/O index delivered by MZFORM has the following format:
\begin{verbatim}
          |       16       |    5    |     5     | 6 bits|
          |----------------------------------------------|
          |      index     |    0    |  NWXIO+1  |     2 |
          |----------------|---------|-----------|-------|
\end{verbatim} 
where NWXIO is the number of extra I/O words,
ie. the total length of the characteristic is NWXIO+1.

\ul{Typing rules for chFORM}

The format should be typed giving the "ct" for each sector,
in the order in which they occur in the bank, as shown in the
examples.
Leading, interspersed, and trailing blanks (also comma or dot)
for aeration are allowed and ignored.

Single-word sectors must be typed as '1t', 't' alone is illegal.

The c for double-precision sectors gives the number of words,
thus 14D specifies 7 double-precision numbers; 7D is illegal.

\ROUTA{CALL MZIOCH (IOWDS*,NWIO,chFORM)}
\begin{verbatim}
with       IOWDS*  the I/O words to receive the result,
                   a vector dimensioned to NWIO

             NWIO  the maximum size of IOWDS, < 17

           chFORM  the format as a CHARACTER string

\end{verbatim} 
\ROUTA{CALL MZIOBK (NAME*,NWMM,chFORM)}
\begin{verbatim}
with        NAME*  the bank description vector for MZLIFT,
                   the resulting characteristic will be stored
                   into the I/O words starting at NAME(5),
                   the IDH contained in NAME(1) will be used
                   if diagnostics are necessary,
                   a vector dimensioned to NWMM

             NWMM  the maximum size of NAME, < 21

           chFORM  the format as a CHARACTER string

\end{verbatim} 
\ROUTA{CALL MZFORM (chIDH,chFORM,IXIO*)}
\begin{verbatim}
with        chIDH  the Hollerith IDH of the bank, type CHARACTER

           chFORM  the format as a CHARACTER string

            IXIO*  returns the index to the characteristic stored
                   in a system data-structure,
                   this can be passed to MZBOOK/MZLIFT,
                   in which case it must not be modified
\end{verbatim} 

\ul{Examples}
\begin{verbatim}
      DIMENSION    IOHEAD(4), MMLIFT(8)

      CALL MZIOCH (IOHEAD,4, '8I -F')             for an FZIN user header
      CALL MZIOBK (MMLIFT,8, '2I / 2I 8F')        for MZLIFT
      CALL MZFORM ('RCBC', '2I/2I 8F'), IORCBC)   for reference by index
\end{verbatim} 

People creating data outside Zebra,
but destined to be read by FZ of Zebra,
will have to know the representation of the I/O characteristic
stored into any bank:

The physically first word of any bank contains:

\begin{verbatim}
   right half:  NOFF = NIO + NL + 12
   (bits 1-16)   where  NIO: the number of extra
                              I/O descriptor words for the bank
                        NL: the number of links in the bank


   left half:   the I/O control byte, which controls the
   (bits 17-32)  interpretation of the I/O characteristic
\end{verbatim} 

In the simplest cases the I/O control byte alone specifies
the nature of the data in the bank, without needing extra
descriptor words (in which case NIO is zero).
We give here the translation of some of these cases:

\begin{verbatim}
       -B:  0001             *B:  0009
       -I:  0002             *I:  000A
       -F:  0003             *F:  000B
       -D:  0004             *D:  000C
       -S:  0007
\end{verbatim} 

For example: suppose one were to prepare a bank with
two links and 4000 data words which are all un-signed
32-bit integer (type B), a bank which is to travel
in link-less mode such that all standard links are zero:

\begin{verbatim}
   word 1   0001 000E     -B | NOFF = 14
        2   zero          link 2
        3   zero               1
        4   zero          link next
        5   zero               up
        6   zero               origin
        7   IDN           numeric ID
        8   IDH           Hollerith ID
        9   2             number of links
       10   1, say        number of structural links
       11   4000          number of data words
       12   zero          status word
                          bits 19-22 give NIO, here zero
       13   data word 1
            ...
     4012   data word 4000
\end{verbatim} 

Note that the status word contains NIO on bits 19--22
to allow Zebra to reach the start-of-bank.

It is impraticable to tabulate the translation of more
complicated formats. There is a little program DIOCHAR
to interactively take a format chFORM, translate it
and display the result in hexadecimal.
This is not yet properly installed on the CERN machines,
but on the Apollo people at CERN can run it by giving
the command {\em  /user/zoll/uty/diochar}

\smark{MZIOTC}
The subroutine MZIOTC is provided to convert an encoded
IO characteristic back into printable form.
One may hand to this routine the address of a bank and
receive its IO characteristic in a {\tt CHARACTER} variable.
Alternatively one may pass to it an integer array as delivered
by MZIOCH for back-conversion to {\tt CHARACTER},
for example the IO characteristic of a user-header vector
read with FZIN.

\ROUTBS{CALL MZIOTC (IXST, !L, NCHTR*, chIOTR*)}
       {CALL MZIOTC (IOWDS, 0, NCHTR*, chIOTR*)}
\begin{verbatim}
with         IXST  the index of the store holding the bank,
                          or of any of its divisions
               !L  the address of the bank

            IOWDS  the integer array with the encoded characteristic
                   (L must be zero in this case)

           NCHTR*  number of useful characters stored into chIOTR
                   = 0 if trouble

          chIOTR*  the CHARACTER variable to receive the characteristic
\end{verbatim} 
The routine returns zero in NCHTR if L is non-zero and not
a valid bank address,
or if chIOTR is not long enough.

\Filename{H1-MZ-Data structure utilities}
\chapter{Data structure utilities}
\label{sec:H1-Data-structure-utilities}

\Filename{H2-MZFLAG-logical-walk-through-data-structure}
\section{MZFLAG {\it et al.} - logical walk through a data-structure}

By following the structural links,
MZFLAG sets the selected status-bit into the status words
of all the banks of the data-structure supported
by the down-links of the specified start bank.
Optionally it can include into the marking
also the banks of the linear structure supported
by link 0 of the start bank and all their dependents.
The start bank itself may or may not be marked.

The request is

\ROUTA{CALL MZFLAG (IXSTOR,!L,IBIT,chOPT)}
\begin{verbatim}
with
      IXSTOR  index of the store or of any division in this store,
              zero for the primary store

          !L  address of the start bank supporting
              the partial data-structure; no action if L=0.

        IBIT  the bit-number of the status-bit to be set

       chOPT  character string of options:

         default  mark the bank at L (and its down dependents),
                  the 'next' link of this bank is not followed,
                  status-bit ITBIT is set to one

               L  mark the linear structure pointed to by L
                  ie. the 'next' link of the bank at L is followed

               V  mark only the partial data-structure
                  dependent vertically downwards from the bank at L,
                  but not the bank itself

               Z  set to zero bit IBIT in each bank to be marked
\end{verbatim} 

MZFLAG will store into the two words of the common /ZLIMIT/
the addresses of the lowest and of the highest bank marked
during the scan, ready for use by the table-building routines
of FZOUT for example.

MZFLAG is not a routine commonly called directly by the users;
its main current use is as a service routine to MZDROP.

Similarly, the routine MZMARK described below
is not normally needed by the users except for a special problem
mentioned there.
MZMARK is also used as a service routine by FZOUT.

The function MZVOLM walks through a data-structure to calculate
the space occupied, returning the number of words as the function value.

\ROUTA{NWORDS = MZVOLM (IXSTOR,!L,chOPT)}
\begin{verbatim}
with
      IXSTOR  index of the store or of any division in this store,
              zero for the primary store

          !L  address of the start bank supporting
              the partial data-structure; no action if L=0.

       chOPT  character string of options:

         default  the 'next' link of the bank at L is not followed

               L  the 'next' link of the bank at L is followed


\end{verbatim} 

\ul{Example} for MZFLAG:

\begin{verbatim}
      CALL MZFLAG (0,LQMAIN,IQDROP,'L')
\end{verbatim} 

this will scan the banks of the data-structure supported by
the bank at LQMAIN and its sisters (option L),
setting the system bit IQDROP to be 'on' in each bank found.
This is equivalent to CALL MZDROP (0,LQMAIN,'L'),
except that it does not set the contents of the word LQMAIN to zero.

\vspace*{5mm}
\ul{Example} for MZMARK:

\begin{verbatim}
      PARAMETER  (NID=3)
      DIMENSION  IDLIST(NID)
      DATA       IDLIST  /  4HBGO , 4HTEC  , 4HMUC  /

      CALL MZMARK (0,LQMAIN,'L-',NID,IDLIST)
\end{verbatim} 
this will scan the banks of the data-structure supported by
the bank at LQMAIN and its sisters (option L),
but exclude (option --) from the scan any lower level linear structure
starting with a bank whose IDH is any of BGO, TEC, MUC
(and its dependents),
setting in each bank found system status bit IQMARK to be 'on'.

The primary purpose of MZMARK is to give the user a possibility
to select parts of a data-structure for output with FZOUT.
The selection works on IDH, the Hollerith ID, of the first bank
of each linear sub-structure of the full data-structure.
For convenience,
one may give to MZMARK either the list of the IDH's to be included
into the scan, or the list of the IDH's to be excluded from the scan;
hopefully one gets away with a short list by selecting the right
mode.

MZMARK is a modified version of MZFLAG, it is simpler in that
the bit-number and the bit value are not parameterized:
the bit is IQMARK and the value is 1, as needed by FZOUT;
it is more complex in that linear structures can be selected
or anti-selected.

The request is

\ROUTA{CALL MZMARK (IXSTOR,!L,chOPT,NID,IDLIST)}
\begin{verbatim}
with
      IXSTOR  index of the store or of any division in this store,
              zero for the primary store

          !L  address of the start bank supporting
              the data-structure; no action if L=0.

       chOPT  character string of options:

         default  mark the bank at L (and its down dependents),
                  the 'next' link of this bank is not followed,
                  lower level linear structures are accepted only
                     if they start with a bank whose IDH appears in
                     the list IDLIST (or if NID=0)

               L  mark the linear structure pointed to by L
                  ie. the 'next' link of the bank at L is followed

               V  mark only the partial data-structure
                  dependent vertically downwards from the bank at L,
                  but not the bank itself

               -  accept a lower level linear structure only if
                  it starts with a bank whose IDH does
                  n o t  appear in IDLIST

      NID       number of elements in the list IDLIST,
                if =zero all banks are accepted ('-' option ignored)

      IDLIST    list of the Hollerith ID for selection
\end{verbatim} 

On return IQUEST(2) contains the total number of words
occupied by all the banks marked (unless L is zero on entry).

As for MZFLAG, the addresses of the lowest and the highest
bank are stored into /ZLIMIT/, ready for FZOUT.

\Filename{H2-LZHEAD-find-first-bank-of-linear-structure}
\section{LZHEAD - find the first bank of a linear structure}

This routine will try to find the first bank of the linear
structure of which the bank at LGO is a member.
It does this by following the path indicated by the "origin"
link of the bank at LGO, and using its "up" link.


\ROUTA{!LF = LZHEAD (IXSTOR,!LGO)}


It returns the address of the first bank of the linear structure
as the function value; or zero if there is trouble.

If the linear structure is not a top-level structure,
ie. if the up-link LUP is non-zero, the path of origin-links should
end in the link region of the bank at LUP, at a word whose
off-set JBIAS can then be calculated. This is returned:
\begin{verbatim}
      IQUEST(1) negative:  = JBIAS
\end{verbatim}
ie. LQ(LUP+JBIAS) contains the address of the first bank of the
linear structure.

If LUP is zero, the origin-path should end at a word outside
the bank space of the store IXSTOR, which word
should contain the address of the first bank of the linear structure.
In this case LZHEAD returns:
\begin{verbatim}
      IQUEST(1) = 1: top-level structure
      IQUEST(2) = LS, relative adr of the supporting link-area link,
                      ie. LQ(LS) contains LF
\end{verbatim}

If LUP is zero, and if the origin-link in the last bank in the path
is zero, this is a stand-alone structure, in which case LZHEAD returns:
\begin{verbatim}
      IQUEST(1) = 2: stand-alone structure
\end{verbatim}

If there is trouble, LZHEAD will return the function value zero,
and set:
\begin{verbatim}
      IQUEST(3) = 1   if LGO is zero

                  2   if LUP non-zero and the last origin-link
                      points outside bank-space

                  3   if LUP non-zero and LQ(LUP+JBIAS) does not
                      point to the last bank in the origin-path

                  4   if LUP zero, and LQ(LS) does not point to
                      the last bank in the origin-path.
\end{verbatim}

\Filename{H2-ZSHUNT-change-structural-relation}
\section{ZSHUNT - change structural relation}

Unlike in HYDRA, and because of the reverse pointers,
the operation of moving a bank by re-linking from one data-structure
to another one is a non-trivial operation.
The routine ZSHUNT is provided to execute such an operation.

ZSHUNT may be used to extract either a single bank (IFLAG=0)
or a whole linear structure (IFLAG=1) from the old context,
for insertion into the new context as described by the parameters
LSUP and JB, which have the same significance as in MZLIFT.

\ROUTA{CALL ZSHUNT (IXSTOR,!LSH, !LSUP,JB,IFLAG)}
\begin{verbatim}
with
      IXSTOR  index of the store, zero for the primary store;
              IXDIV, the index of the division containing
              the bank to be shunted, may be given instead

        !LSH  address of the bank or of the linear structure

       !LSUP  if JB < 1:  address of the new supporting bank
              if JB = 1:  the new supporting link*

          JB  if JB < 1:  link bias in the new supporting bank
              if JB = 1:  LSUP is the new supporting link,
                           the origin-link in the bank at LSH
                           will be made to point to it
              if JB = 2:  detach without insertion

       IFLAG  if IFLAG = 0:  shunt the one single bank at LSH
              if IFLAG = 1:  shunt the whole linear structure
                              pointed to by LSH
\end{verbatim} 
If the bank or the structure to be re-linked is in fact inserted
or added into an existing linear structure,
both must be contained in the same division.

\ul{Examples}

Suppose we have the following data-structures to start with:

\begin{verbatim}
       ______
      |      |                         up
      |  UA  | <---.-------------.-------------.
      |______|     |             |             |
         |         |             |             |
      -3 |         |             |             |
         |       ______        ______        ______
         |      |      |  <-- |      |  <-- |      |
         `----> |  A1  | ---> |  A2  | ---> |  A3  |
                |______|      |______|      |______|

and
       ______
      |      |                              up
      |  UN  | <---.-------------.-------------.
      |______|     |             |             |
         |         |             |             |
      -7 |         |             |             |
         |       ______        ______        ______
         |      |      |  <-- |      |  <-- |      |
         `----> |  N1  | ---> |  N2  | ---> |  N3  |
                |______|      |______|      |______|


and
                    ______        ______        ______
              <--- |      |  <-- |      |  <-- |      |
      LQMAIN  ---> |  X1  | ---> |  X2  | ---> |  X3  |
                   |______|      |______|      |______|
\end{verbatim} 
Any bank may support further dependent partial data-structures,
the corresponding structural down-links are not changed
by ZSHUNT.

In what follows the notation  Lxx  is used to designate
a link pointing to bank xx.

\ul{Example 1} \hspace{12mm} CALL ZSHUNT (0,LA2,LUN,-7,0)   \hspace{5mm} gives:
\begin{verbatim}
       ______
      |      |
      |  UA  | <---.-------------.
      |______|     |             |
         |         |             |
      -3 |         |             |
         |       ______        ______
         |      |      |  <-- |      |
         `----> |  A1  | ---> |  A3  |
                |______|      |______|
and
       ______
      |      |
      |  UN  | <---.-------------.-------------.-------------.
      |______|     |             |             |             |
         |         |             |             |             |
      -7 |         |             |             |             |
         |       ______        ______        ______        ______
         |      |      |  <-- |      |  <-- |      |  <-- |      |
         `----> |  A2  | ---> |  N1  | ---> |  N2  | ---> |  N3  |
                |______|      |______|      |______|      |______|
\end{verbatim} 

This moves a single bank (with is dependents, if any) out of
a linear structure, and inserts it at the head of the linear
structure supported by link -7 of the bank UN.

\vspace{4mm}
\ul{Example 2} \hspace{12mm} CALL ZSHUNT (0,LA2,LUN,-7,1)   \hspace{5mm} gives:
\begin{verbatim}
       ______
      |      |
      |  UA  |
      |______|
         |
      -3 |
         |       ______
         |      |      |
         `----> |  A1  |
                |______|
and
  ______
 |      |
 |  UN  |  <--.-------------.-------------.------------------.
 |______|     |             |             |                  |
    |         |             |             |                  |
 -7 |         |             |             |                  |
    |       ______        ______        ______             ______
    |      |      |  <-- |      |  <-- |      |  <- ... - |      |
    `----> |  A2  | ---> |  A3  | ---> |  N1  | -- ... -> |  N3  |
           |______|      |______|      |______|           |______|
\end{verbatim} 
This is the same as example 1, except that the (partial) linear
structure starting with bank A2 is re-linked.


\ul{Example 3} \hspace{12mm} CALL ZSHUNT (0,LA2,LN2,0,0)    \hspace{5mm} gives:
\begin{verbatim}
       ______
      |      |
      |  UN  | <---.-------------.-------------.-------------.
      |______|     |             |             |             |
         |         |             |             |             |
      -7 |         |             |             |             |
         |       ______        ______        ______        ______
         |      |      |  <-- |      |  <-- |      |  <-- |      |
         `----> |  N1  | ---> |  N2  | ---> |  A2  | ---> |  N3  |
                |______|      |______|      |______|      |______|
\end{verbatim} 
This is again like example 1, but the bank is inserted inside
the linear structure, rather than ahead of it.

\vspace{4mm}
\ul{Example 4} \hspace{12mm} CALL ZSHUNT (0,LA2,LQMAIN,1,0) \hspace{5mm} gives:
\begin{verbatim}
                  0             0             0             0
                  ^             ^             ^             ^
                  |             |             |             |
                 ______        ______        ______        ______
         <----  |      |  <-- |      |  <-- |      |  <-- |      |
  LQMAIN -----> |  A2  | ---> |  X1  | ---> |  X2  | ---> |  X3  |
                |______|      |______|      |______|      |______|
\end{verbatim} 

This relinks bank A2 to be the first in the top-level linear
structure supported by LQMAIN.

\vspace{4mm}
\ul{Example 5}
\begin{verbatim}
          L = LQMAIN
          CALL ZSHUNT (0,LA2,L,1,0)
\end{verbatim} 
has exactly the same effect as Example 4 above because,
LQMAIN not being zero initially,
the origin-link of the bank pointed to by L
(and the up-link, but this is zero)
is used for the connection.


\ul{Example 6} \hspace{12mm} CALL ZSHUNT (0,LA1,LHEAD,1,1)  \hspace{5mm} gives:
\begin{verbatim}
       ______
      |      |
      |  UA  |
      |______|
         |
      -3 |
         |
          ----> zero

and               0             0             0
                  ^             ^             ^
                  |             |             |
                 ______        ______        ______
         <----  |      |  <-- |      |  <-- |      |
  LHEAD  -----> |  A1  | ---> |  A2  | ---> |  A3  |
                |______|      |______|      |______|
\end{verbatim} 

supposing LHEAD=0 initially; this connects the linear structure
to the (structural) link LHEAD, ie. the origin-link of the header bank A1
points back to the location of LHEAD.

\vspace{4mm}
\ul{Example 7} \hspace{12mm} CALL ZSHUNT (0,LA1,LDUMMY,2,1) \hspace{5mm} gives:
\begin{verbatim}
       ______
      |      |
      |  UA  |
      |______|
         |
      -3 |
         |
         `----> zero

and               0             0             0
                  ^             ^             ^
                  |             |             |
                 ______        ______        ______
         0 <--  |      |  <-- |      |  <-- |      |
  LA1    -----> |  A1  | ---> |  A2  | ---> |  A3  |
                |______|      |______|      |______|
\end{verbatim} 
This detaches the linear structure from its old context
without inserting it into a new one.
This should only be temporary, one should insert the floating
structure into a new context by a second call to ZSHUNT
not too much later.

\Filename{H2-ZSORT-utility-to-sort-banks-of-linear-structure}
\section{ZSORT  - utility to sort the banks of a linear structure}

These routines re-arrange the horizontal linking
within a given linear structure such that the key-words contained in
each bank increase monotonically when moving through the linear
structure with L=LQ(L).
For equal key-words the original order is preserved.

Key-words may be either floating, integer or Hollerith.
For Hollerith sorting a collating sequence
inherent in the representation is used,
thus the results will depend on the machine.

Sorting may be done either for a single key-word in every bank
or for a key vector in every bank:

\begin{verbatim}
 ________________________________________________________________________
|                                                                        |
|   CALL ZSORT  (IXSTOR,*!LLS*,JKEY)       single floating  key-words    |
|   CALL ZSORTI (IXSTOR,*!LLS*,JKEY)       single integer   key-words    |
|   CALL ZSORTH (IXSTOR,*!LLS*,JKEY)       single Hollerith key-words    |
|   CALL ZSORV  (IXSTOR,*!LLS*,JKEY,NKEYS)        floating  key vectors  |
|   CALL ZSORVI (IXSTOR,*!LLS*,JKEY,NKEYS)        integer   key vectors  |
|   CALL ZSORVH (IXSTOR,*!LLS*,JKEY,NKEYS)        Hollerith key vectors  |
|________________________________________________________________________|

    with the parameters

      IXSTOR  index of the store or of any division in this store,
              zero for the primary store;

      *!LLS*  the address of the first bank of the linear structure,
              reset on return to point to the new first bank;

        JKEY  in each bank at L, Q(L+JKEY) is the key word,
                                 or the first word of the key vector;

       NKEYS  the number of words in the key vector.
\end{verbatim}

The execution time taken by these routines is a function
of the re-ordering which needs to be done.
For perfect order the operation is a simple verification pass
through the structure.
The maximum time is taken if the banks are initially arranged with
decreasing key words.

Sorting re-links the banks such that the key-words are in
increasing order.
If one needs them in decreasing order on may use
CALL ZTOPSY (IXSTOR,LLS)
which reverses the order of the banks in the linear structure
pointed to be LLS.

\Filename{H2-ZTOPSY-utilities-to-operate-on-linear-structures}
\section{ZTOPSY {\it et al.} - utilities to operate on linear structures}

These routines perform service operations
on linear structures.
The parameter LLS is the address of the first bank
of the linear structure.


\ROUTA{CALL ZTOPSY (IXSTOR,*!LLS*)}

reverses the order of the banks in the linear structure,
ie. the first bank becomes the last, and the last the first,
for walking through the structure with L=LQ(L).
Starting with Zebra version 3.67, LLS is updated to point to
the first bank of the inverted structure on return.

\ROUTA{CALL ZPRESS (IXSTOR,!LLS)}

removes by bridging dead banks still present
in the linear structure pointed to by LLS.

\Filename{H2-LZFIND-utilities-to-interrogate-linear-structures}
\section{LZFIND {\it et al.} - utilities to interrogate linear structures}

These routines perform service functions for linear structures.
The parameter LLS is the address of the first bank
of the linear structure.

\ROUTA{!LF = LZLAST (IXSTOR,!LLS)}

searches the linear structure pointed to by LLS for its end.
It returns in LF the address of the last bank in the structure.
LF = 0 is returned if the structure is empty.

\ROUTA{!LF = LZFIND (IXSTOR,!LLS,IT,JW)}

searches the linear structure pointed to by LLS
for the first bank containing IT in word JW;
it returns its address in LF.
If none:  LF=0.

\ROUTA{!LF = LZLONG (IXSTOR,!LLS,NW,ITV,JW)}

has the same function as LZFIND,
but ITV is a vector of NW words expected
in words JW to JW+N--1 of the bank.

\ROUTA{!LF = LZBYT  (IXSTOR,!LLS,IT,JBIT,NBITS)}

has the same function as LZFIND,
but it looks for a bank having IT in byte (JBIT,NBITS)
of the status word.

\ROUTA{!LF = LZFVAL (IXSTOR,!LLS,VAL,TOL,JW)}

has the same function as LZFIND,
but it looks for a bank having in word JW a floating point number
which is equal to VAL within the tolerance TOL.

\ROUTA{N  = NZBANK (IXSTOR,!LLS)}

counts the number of banks in the linear
structure pointed to by LLS.

\ROUTA{N  = NZFIND (IXSTOR,!LLS,IT,JW)}

searches like LZFIND, but for all banks.
It returns the number of such banks in N
and stores the addresses of the first 100 such banks into IQUEST,
starting at IQUEST(1).

\ROUTA{N  = NZLONG (IXSTOR,!LLS,NW,ITV,JW)}

searches like LZLONG, but for all banks.
It returns the number of such banks in N
and stores the addresses of the first 100 such banks into IQUEST,
starting at IQUEST(1).

\Filename{H2-LZFID-utilities-to-find-bank-by-sequential-scan}
\section{LZFID {\it et al.} - utilities to find a bank by sequential scan}

Unlike the routines of the previous paragraphs which access
banks by following the links of the structure,
the routines of this paragraph perform a scan over the memory,
looking at each bank in turn in the order in which they happen
to be in the dynamic store,
to find the bank wanted.
For large memories with many banks this is likely to be an expensive
operation and should not be used unless there is no other way.


\ROUTA{!LF = LZFID (IXDIV, IDH,IDN, !LGO)}

searches the division indicated by IXDIV, either starting
at its beginning if LGO=0 or with the first bank after the bank
at LGO, for the first bank with has the identifiers IDH and IDN.


\ROUTA{!LF = LZFIDH (IXDIV, IDH, !LGO)}

searches the division indicated by IXDIV, either starting
at its beginning if LGO=0 or with the first bank after the bank
at LGO, for the first bank with has the Hollerith identifier IDH.


\ROUTA{!LF = LZSCAN (IXDIV, !LGO)}

searches the division indicated by IXDIV, either starting
at its beginning if LGO=0 or with the first bank after the bank
at LGO, for the first bank.

LZSCAN returns IQUEST(1) containing zero if the bank at LF is live,
or one if the bank is dead.
\chapter{Global operational aids}

\Filename{H2-ZPHASE-set-program-phase}
\section{ZPHASE - set program phase}

Primarily to avoid recovery to 'next event' at the wrong moment,
ZEBRA needs to know in which phase the user program is at any
given moment.
We distinguish three phases:
\begin{verbatim}
    a) during the initialization phase the user prepares the context for
    b) the operation phase of his program;
    c) during the termination phase accumulated results are output,
       files are closed, etc.
\end{verbatim} 
The user may subdivide the operation phase for his own purpose.
With the change to 'termination' one may request ZPHASE to take
action of tidying up the primary store,
to make room for end-processing routines (like output of histograms)
which may need a large amount of working memory.

\ROUTA{CALL ZPHASE (JPH)}

signals to Zebra a change of phase, preset by MZEBRA to 'initialization'.

\begin{verbatim}
   Normal operation phase:  JPH .GE. 0

   Termination phase:       JPH .LT. 0

      JPH = -1:  reset the working space to be of zero length,
                  but leave the store as it is;

            -2:  reset the working space,
                  wipe all user short-range divisions;

            -3:  reset, wipe, and collapse upwards all short-range user
                  divisions to be of zero length, giving their space to
                  division 1 for use by termination routines; perform
                  clean-up garbage collection in all other divisions.
\end{verbatim} 

The program phase is recorded on the variable NQPHAS of /ZSTATE/
and has the following significance~:
\begin{verbatim}
   NQPHAS =   0  initialisation phase
             >0  = MAX(JPH,1):  normal operation phase
             <0  = JPH: termination phase
\end{verbatim} 
If the user whishes to subdivide the operation phase,
he can pick up the current state from this variable.

ZPHASE prints a log message at level --1 for major phase changes,
and at level 2 for minor changes.

\Filename{H2-ZEND-normal-program-end}
\section{ZEND - normal program end}

\ROUTA{user routine ZEND}
is defined to be the entry-point for normal run termination.

This routine, without parameters, is normally provided by the user
to close files and print accumulated results.
It is important that all closing down operations are
done through this routine,
if the user wants them to happen even in abnormal
run termination.
It would normally look like this:
\begin{verbatim}
       SUBROUTINE ZEND
 +CDE, ZSTATE.

       CALL ZPHASE (-3)           start termination
       . . .                      any user termination code

       CALL MZEND
       IF (NQERR.NE.0)  CALL ABEND
       STOP
       END
\end{verbatim} 

MZEND is a little routine which prints statistics about
the usage of all divisions.

NQERR in /ZSTATE/ is zero for normal run termination,
it is non-zero if ZEND has been reached via ZFATAL.
On some machines the recovery system of the machine
expects the user to re-enter it.
The KernLib routine ABEND (Z 035) looks after this.

The Zebra system contains a little default routine
ZEND which is the above without any user termination.

A user routine similar to ZEND is defined for taking over control
of fatal error termination. It is called ZABEND and it is
described in the next paragraph.
This should perform in there any extra operations needed
for fatal termination and then it should transfer
to ZEND for closing down.

\Filename{H2-ZABEND-abnormal-program-end-through-ZFATAL}
\section{ZABEND - abnormal program end through ZFATAL}

\ROUTA{CALL ZFATAL}
is defined to be the entry point for fatal run termination.

\ROUTA{CALL ZFATAM (chMESSAGE)}
is identical to ZFATAL,
except that it prints a message,
given in the parameter as a character
string of not more than 30 characters,
terminated by the character '.'.

These routines are supplied by the system;
they are protected against recovery loops,
and they must not be supplied by the user.
They should only be called when the run cannot usefully continue.
If the application program discovers such a fatal condition
it too should call ZFATAL or ZFATAM,
preceded with some diagnostic printing or
with loading to IQUEST some clue to the trouble.

\vspace{4mm}
\ROUTA{SUBROUTINE ZABEND}
receives control from ZFATAL to handle fatal run termination.
This routine may be supplied by the user.

The Zebra system contains the standard routine as follows:

\begin{verbatim}
       SUBROUTINE ZABEND
 +CDE, ZSTATE.

       CALL ZPOSTM ('TCWM.')
       IF (NQPHAS.LE.0)  CALL ABEND
       NQPHAS = -2
       CALL ZEND
       RETURN
       END
\end{verbatim} 

This is not just a dummy,
it causes a post-mortem dump,
including a subroutine trace-back (if possible),
followed by any normal user output programmed in ZEND.
Transfer to ZEND takes place only if the break-down happened
during normal operation,
but not if the program is still in the initialisation phase
or if it is already under ZEND control.

The parameter to ZPOSTM is passed from there to DZSNAP
to select the options for dumping the dynamic store,
see   ...   for details.

\Filename{H2-ZTELL-recover-trouble-through-ZTELL-ZTELUS}
\section{ZTELL  - recover trouble through ZTELL-ZTELUS}

During normal operation any request from the user for space
with MZWORK, MZLIFT {\it et al.} is satisfied,
after garbage collection if necessary and possible.
If however the request cannot be satisfied,
the normal course of the program must be broken.
To deliver the user from the burden of checking for success
after each space request,
the garbage collector sends control to the user at the
entry-point QNEXT (via ZTELL and the KERNLIB routine QNEXTE),
where he can program the recovery of the problem.
Normally this will be to skip the current event and to continue
with processing the next one.

Other Zebra packages, apart from MZ, and maybe the user himself,
have similar problems.
Therefore a general trouble control routine ZTELL has been
included into Zebra.
This is a switching routine with several modes of continuation,
one of which is to send control to QNEXT.
ZTELL can also be called by the user program, thus:

\ROUTA{CALL ZTELL (ID,IFLAG)}

\begin{verbatim}
    ID     is an integer between 101 and 999,
           ID's below 100 are reserved for system usage,
           ID=99 for 'no memory left' from MZGARB.

    IFLAG  is a flag indicating whether the calling code can
           accept a RETURN from ZTELL:
           = 0 ZTELL may return;
           = 1 the calling code is not capable to accept a RETURN;
           = 2 fatal error, the run must stop.
\end{verbatim} 

ZTELL prints a message,
sets up a reasonable exit mode as a function of ID and IFLAG
into the little labelled common /ZTELLC/
and calls ZTELUS to give the user a chance to modify this mode.
On the
{\large\bf obligatory}
return from ZTELUS it takes the selected exit as follows:

\begin{verbatim}
    COMMON /ZTELLC/ ID, MODE

    ID     is a copy of the first parameter to ZTELL;
    MODE   is the selected exit mode:
           = 0 RETURN to let the calling routine continue;
           = 1 CALL QNEXTE to enter QNEXT
           = 2 CALL ZFATAL
           = 3 CALL ZEND
\end{verbatim} 

The exit mode to QNEXT is enabled only if NQPHAS in /ZSTATE/
is larger than zero, indicating that the program is
in the normal operation phase.
During the initialisation or the termination phase of the program
transfer is to ZFATAL instead,
to avoid a program crash to be 'recovered' into normal operation.

The pre-loading of MODE is  MODE=IFLAG  for user calls {\tt (ID>100)};
and for system calls {\tt (ID<100)} it is as shown
in book DIA, entry ZTELL.

Here is an example of a ZTELUS
which is happy with the default modes,
except that it wants to go to ZEND for ID=8:

\begin{verbatim}
      SUBROUTINE ZTELUS
      COMMON /ZTELLC/ ID,MODE

      IF (ID.EQ.8)  MODE=3
      RETURN
      END
\end{verbatim} 
The default subroutine ZTELUS on the library is a do-nothing dummy.
The default subroutine QNEXT goes straight to ZFATAL.

On most machines repeated recovery directly to QNEXT causes
trouble with the FORTRAN trace-back and the subroutine stack.
For this reason ZEBRA relies on the KERNLIB routine QNEXTE,
which implies an organization for event processing as follows:

\begin{verbatim}
 _________________________
|                         |          Program flow with QNEXT recovery
|    MAIN program         |
|                         |
|    CALL MZEBRA (0)      |
|    CALL MZSTOR (...)    |
|                         |
|    program              |                                _________
|    initialization       |                               |         |
|                         |                               |         |
|    CALL ZPHASE (0)      | first entry           recover |         |
|    CALL QNEXTE          | ------       _________       _|         |
|_________________________|      `----> |         |     |           |
                                        | routine |     |  routine  |
          <---------------------------< | QNEXTE  | <-- |  ZTELL    |
         |                              |_________|     |_          |
 ________|________________                                |         |
|                         |                               |         |
|    subr QNEXT           |                               |         |
|                         |                               |         |
| 11 CALL MZWIPE (0)      |       requests                |         |
|    read event           | ----------------------------> |  Zebra  |
|    IF (end) CALL ZEND   |                               |  system |
|    process event        | <---------------------------- |         |
|    output event         |               normal RETURN's |         |
|    GO TO 11             |                               |         |
|_________________________|                               |         |
         |                                                |         |
 ________|________________        __________              |         |
|                         |      |          |             |         |
|    subr ZEND            | <--- |  subr    | <---        |         |
|                         |      |  ZABEND  |     |      _|         |
|    CALL ZPHASE (-3)     |      |__________|     |     |           |
|                         |                       |     |  routine  |
|    program              |                       `---< |  ZFATAL   |
|    termination          |                             |_          |
|_________________________|                               |_________|

\end{verbatim} 
The initialisation part prepares the program to be ready
for execution and then calls itself QNEXTE for entry to QNEXT
to process the 'next event',
being the first event in this case.
QNEXT loops internally to process all events.

CALL's from the processing program to the Zebra system
are normally satisfied,
and control comes back to the user with normal RETURN.
Abnormal returns are either via ZFATAL to ZEND,
or straight to ZEND,
or to QNEXT via QNEXTE.


The Fortran version of QNEXTE is a simple CALL QNEXT
followed by RETURN.
If necessary on a given computer,
QNEXTE is a machine-language or a C routine to unwind to itself
the FORTRAN trace-back and the subroutine stack.
So, if the user wishes at some point to abandon himself
the curent event and to go to the next one,
he should CALL QNEXTE and not QNEXT.

QNEXT is a user routine to the KERNLIB routine QNEXTE and
has thus the usual problem of user routines called
from a library routine in that it must be loaded explicitely:
\begin{verbatim}
   either:  compile it together with the other material
       or:  if it resides on a user library it must be
             INCLUDEd explicitly, for example on the VAX with
             $ LINK  ...   MYLIB/INC=QNEXT/LIB  ...
\end{verbatim} 

This flow-diagram is only an example for the most common case
of actual usage of Zebra.
If one's program is not of the event-processing type
one has to look at QNEXTE/QNEXT from a different angle:
program flow from MAIN has to go to QNEXTE to initialize
for re-entry.
Entry and all re-entries are then to QNEXT,
which has to control the further program flow according
to some flags,
conveniently ID in /ZTELLC/ and NQPHAS in /ZSTATE/.
ID is not initialized by MZEBRA,
it is only changed by ZTELL which copies its first parameter to ID.
This number is an integer in the range 1 to 99 for calls
from the Zebra system.
NQPHAS is initialized to zero by MZEBRA,
it is then changed only by the user (or the default ZABEND)
either directly or with ZPHASE.

\Filename{H2-ZVERIF-check-logical-integrity-of-Zebra-stores}
\section{ZVERIF - check logical integrity of Zebra stores}

\ROUTA{CALL ZVERIF (IXDIV,IFLRTN,chTEXT)}

\begin{verbatim}
with   IXDIV  1) the index of a single division to be verified;
              2) the index of the store to be verified;
                 = -1: all stores
                        (other negative values are reserved)

      IFLRTN  return flag, see below;

      chTEXT  identifying text to be printed if trouble.
\end{verbatim} 

This routine has two somewhat different modes of operation:

Normally ZVERIF checks all the data in and relevant to a complete
store, or even several stores. It goes to ZFATAL if it finds
trouble.

But if verification is for one single division only, only the banks
of this division are checked and control is given back to the caller
if the flag IFLRTN is non-zero, in which case the number of
normally fatal conditions is delivered in IQUEST(1).
This is intended to be used if one receives a data-structure
in memory (with FZIN, say) which may be bad. Rather than letting
the program run to crash one can realize the situation and
wipe the division. This must be done with MZWIPE because this
is the only way to get rid of the data without Zebra looking
at them.

This routine inspects all link areas and all banks in the stores
selected, performing the following checks:

\begin{verbatim}
    (0)  process and store control tables intact;
    (1)  bank chaining intact;
    (2)  a non-zero structural link must point to a true bank address;
    (3)  consistency of "up" and "origin" links;
    (4)  a link in a bank in division A pointing to division B
         is checked for consistency with the cross-reference matrix;
    (5)  a non-zero reference link in a bank must not point
         into a reserve area;
\end{verbatim} 

ZVERIF is used by the automatic verification procedure
ZVAUTO, see next paragraph.

\Filename{H2-ZVAUTO-running-with-automatic-verification}
\section{ZVAUTO - running with automatic verification}

The routines of this complex are provided as the {\it ultima ratio regis}
for finding program errors which destroy data in a Zebra store
by faulty handling of links; they consume a lot of CPU time.

With automatic verification switched on calls to the Zebra system
transfer to ZVAUTY,
which handles the verification process:

The "verify identifier" (VID) is constructed by counting the number
of entries to ZVAUTY. The VID is a two-word object, because 32 bits
may not be sufficient, it identifies every call to Zebra, it gives
us a handle on where we are in the job, and it is available to the
user via the common /ZVFAUT/ to be printed together with the event
number, for example. ZVAUTY will remember separately the VID for the
last garbage collection, for the last MZWIPE, and for the last FZIN,
for printing on fatal error.

In "active mode" ZVAUTY will call ZVERIF for the stores selected
by the user initially, causing a transfer to ZFATAL in case of trouble,
which in turn will transfer control to the user by calling ZABEND,
where he may code any display wanted.

In "dummy mode" the time-consuming calls to ZVERIF will not take place,
but the VID is checked to see whether the "action threshold" defined
by the user is reached. If so, the mode is switched to "active", and
the log-level is set for all stores to be verified.
From this point on one may get rather a lot of output, tracing the
execution of the program step-by-step.

Suppose now one has a particular run, "run A" say, of a lengthy
production job which collapses with the memory destroyed.
The reason for this will be somewhere in the past.
One could re-submit the job switching on true automatic verification 
right away, but this could be excessively time consuming and produce
much too much output.
If so, it is better to re-submit the job first with "dummy verification"
switched on by inserting a call to ZVAUTO at the point where one wants
to start verification, after the initialization phase for example,
and with the connection routine ZVAUTX supplied,
"run B" say.
Dummy verification goes through the
same motion as active verification, but it does not call ZVERIF
and hence costs only little more time than run A. This run should also
end up in ZFATAL, which will now print the VID's remembered for the last
garbage collection, for the last wipe, for the last FZIN,
and maybe other VID's entered by the user with ZVAUTU.

One can now re-submit the job, "run C" say, having changed the call to
ZVAUTO by giving the action threshold, thereby defining the moment
when active verification will start.

Automatic verification is initiated with

\ROUTA{CALL ZVAUTO (MSTORE,IDVTH1,IDVTH2,LOGLEV)}

\begin{verbatim}
with  MSTORE  a bit-by-bit word indicating the stores to be verified,
              store i = 0,1,2,... selected if bit (i+1) is set to one;
              if MSTORE = 0: all stores

    IDVTH1/2  the two-word VID at which true verification should start,
              give both words zero if dummy verification only
              ("run B" in the discussion above);
              give 0,-1 if active verification is to start right away;

      LOGLEV  set the log-level of the existing stores to be verified
              to this value at the moment of changing to active mode,
              give 2 for maximum logging.
\end{verbatim} 


To avoid loading the non-negligeable code of the ZVAUTO complex with
normal production jobs, this trick is used: all relevant Zebra routines
contain a conditional call to ZVAUTX (which does not take place if
ZVAUTO has not been called), and the Zebra library contains a dummy
routine ZVAUTX. But the true process is controlled by ZVAUTY, which
is not normally loaded. The user makes the connection by supplying
this routine:

\begin{verbatim}
      SUBROUTINE ZVAUTX
      CALL ZVAUTY
      END
\end{verbatim} 
as part of the material to be compiled and linked.

In the list of "last VIDs remembered" there are 3 places for user
triggered storage, which is done with

\ROUTA{CALL ZVAUTU}
                 
this will push the VID in position 2 to position 3, and the one in
position 1 to position 2, entering the current VID into position 1.
Thus, if one were to call ZVAUTU at the start of each event,
one would see the start points of the last 3 events in a dump
from ZFATAL.

The user has access to the verification parameters via this common:

\begin{verbatim}
      COMMON /ZVFAUT/IQVID(2),IQVSTA,IQVLOG,IQVTHR(2),IQVREM(2,6)

           IQVID  the current VID

          IQVSTA  the verification status:
                    zero  automatic verifying not running
                      -1  dummy verification
                      +1  active verification

          IQVLOG  the log level to be set at the activation threshold
          IQVTHR  the threshold VID

     IQVREM(2,J)  VID remembered in position J,
                    J = 1  last garbage collection
                        2  last call to MZWIPE
                        3  last call to FZIN
                        4  last call to ZVAUTU
                        5  last-but-one call to ZVAUTU
                        6  last-but-two call to ZVAUTU
\end{verbatim} 


\ul{Examples}
\begin{verbatim}
      Do not forget to supply the connection routine ZVAUTX !

      CALL ZVAUTO (7,0,0,0)

           to start a dummy verification run for store numbers 0, 1, 2;
           "run B" of the discussion above.

           Supposing that the ZFATAL output of this run tells us the
           VID of the last-but-one event, by changing to
 
      CALL ZVAUTO (7,0,123456,2)

           we may start "run C" to give detailed logging for the
           last two events.

      CALL ZVAUTO (1,0,-1,0)

           to start an active verification run for store number 0;
           one might do this on a limited data sample before going
           into production with a new program, just to verify that
           the program is alright.
\end{verbatim} 
\chapter{MZ routines for special applications}

\Filename{H2-MZCOPY-copy-data-structure}
\section{MZCOPY - copy a data-structure}

MZCOPY allows one to copy a data-structure from memory to memory.
\begin{verbatim}
The source memory may be

  either (case 1) one or several divisions in some store F;

      or (case 2) ordinary stand-alone memory,
                  called 'flat memory' below.

The target memory may be

  either (case 3) one particular division in a store T, which may
                  or may not be the same as store F;

      or (case 4) ordinary stand-alone memory.
\end{verbatim} 

For case 1, the data-structure may be identified in the usual way
as the collection of banks which depend structurally
from the header bank.
This (option D not selected) requires a logical walk through
the structure,
marking each bank which belongs to it by setting a system bit
in the status word.
A sequential scan over the memory between the first and the last bank
involved constructs then the table of the material to be copied.
(This way of identifying the d/s must not be used with
a read-only store.)

Alternatively, the d/s may be identified as the contents
of one or several divisions,
in which case the user has to be certain that the complete contents
of this division(s) belong to one single d/s.
This (option D selected) does not require the logical walk,
and is by that much faster.
Moreover, the copy may or may not be immediate (option I):
if yes (option DI) dead banks present in the division(s) are
also copied, if not (option D without I) a sequential scan
over the division(s) concerned constructs a more detailed
table of all the sectors of contiguous live banks in the division(s).
Thus, if the user knows a priori that the amount of dead material
is negligeable he should select option DI for speed,
saving the sequential scan.
However, if the amount of dead material is significant,
he should select option D only,
again for speed saving the time to copy the useless material.

Cases 2 and 4 (flat memory) are intendend for use in communication
between separate processes running on the same computer
through shareable Common blocks:
the sending process places a d/s into a flat memory area,
the receiving process copies it from there into its Zebra store.
Again, the data must all belong to one single d/s.

If the No-link option is selected the copied result
is a simple collection of banks,
with all links except link 0 reset to zero,
connected to be a linear structure;
the entry address returned is the address of
the physically first bank.


\ul{\large Copy from and to Zebra storage:}

\ROUTA{CALL MZCOPY (IXDVFR,!LENTRY, IXDVTO,*!LSUP*,JBIAS, chOPT)}
\begin{verbatim}
with
     IXDVFR:  index of division(s) to be copied from
               may be zero [or IXSTOR] if the D option is not selected
               may be a compound index if the D option is selected

    !LENTRY:  entry address of the d/s

     IXDVTO:  index of the division to be copied to;
               this must be the index of one particular division

            The d/s copied is linked into a pre-existing d/s
            as directed by LSUP and JBIAS, which have the same
            significance as for MZLIFT:

      !LSUP:  if JBIAS < 1: LSUP is the supporting bank,
      JBIAS:                 connection to link LQ(LSUP-JBIAS)

               if JBIAS = 1: LSUP is the supporting link,
                              connection to LSUP (top-level d/s)
                              LSUP* returns the entry adr to the d/s

               if JBIAS = 2: stand-alone d/s, no connection
                              LSUP* returns the entry adr to the d/s

      chOPT:  character string,
               individual characters select options:

            d/s:    by default the d/s supported by the bank at
                     LENTRY is copied (link 0 not followed)

                   L copy the d/s supported by the linear structure
                     at LENTRY (link 0 followed)

                   M  copy the banks pre-marked by the user,
                      see para. 2.01 for details
                      (also: book FZ, para. 1.21, User marking ...)

                   D  copy complete division(s)
                      default: dead banks are squeezed out
                               (slower but maybe more economic than DI)
                   DI immediate dump of division(s),
                      dead banks, if any, are also copied

                   S copy the single bank at LENTRY

                    see separate descriptions below for

                   F   copy from flat memory
                   T   copy   to flat memory
\end{verbatim} 

\begin{verbatim}

         others:  N no links, ie. linkless handling,
                     default: links are significant

                   P permit error returns
                     default: error exit is with CALL ZTELL (15,1)

                   Z zero all links pointing outside the d/s
                     (this is implied if store T is not store F)
\end{verbatim} 

\ul{\large Copy from flat memory:}

\begin{verbatim}
      IQUEST(1) = NWFR:    number of words
      IQUEST(2) = JOFFFR:  link relocation off-set, usually zero
\end{verbatim} 

\ROUTA{CALL MZCOPY (MEMFR,!LENTRY, IXDVTO,*!LSUP*,JBIAS, chOPT)}
\begin{verbatim}
with
      MEMFR:  the words MEMFR(1:NWFR) hold the d/s to be copied,
               links in the d/s, if any, assume addressing such that
               MEMFR(L) corresponds to LQ(L+JOFFFR)

      In this case options L, D, I, S are irrelevant.
\end{verbatim} 

\ul{\large Copy to flat memory:}

\begin{verbatim}
      IQUEST(3) = NWMXTO:  maximum # of words, ie. capacity of MEMTO
      IQUEST(4) = JOFFTO:  link relocation off-set, usually zero
\end{verbatim} 

\ROUTA{CALL MZCOPY (IXDVFR,!LENTRY, MEMTO*,LENTO*,NWTO*, chOPT)}

\begin{verbatim}
with
     MEMTO*:  the words MEMTO(1:NWTO) receive the d/s copied,
               NWTO is not allowed to exceed NWMXTO,
               links in the d/s, if any, will be relocated such that
               MEMTO(L) corresponds to LQ(L+JOFFTO)
     !LENTO*   is the entry address into the copied d/s.
\end{verbatim} 

\ul{Error return}

IQUEST(1/2) will signal on return the status of the copy:
\begin{verbatim}
        = 0:  all is well,  IQUEST(2) = number of words
   and if option P given:
        = 1:  LENTRY invalid
        = 2:  bank chaining clobbered found in the input d/s
        = 3:  not enough space to construct the relocation table
        = 4:  d/s larger than the target space
        = 5:  d/s to be copied is empty
        = 6:  bank chaining clobbered found in the copied d/s
\end{verbatim} 

\Filename{H2-MZNEED-calculate-free-space-in-division}
\section{MZNEED - calculate free space in division}

This routine measures the free space available in a particular
division, either the space available immediately
or the space after garbage collection.

\ROUTA{CALL MZNEED (IXDIV, NEED, chOPT)}

\begin{verbatim}
with
      IXDIV     index of the division

      NEED      number of words needed, trigger garbage collection
                if G option selected and less than NEED words available

      chOPT     character string of options:

                G  execute garbage collection
                   if less than NEED words available

On return /QUEST/IQUEST(100) contains the following information
about the division IXDIV:

   IQUEST(11)   NEX, the number of free words beyond NEED,
                negative if less than NEED words available,
                ie. the total available is NEED+IQUEST(11);

   IQUEST(12)   the space occupied by the banks, including dead banks
                if any, of the division;

   IQUEST(13)   the maximum space allowed for this division,
                as given to MZDIV.
\end{verbatim} 

\ul{Example}

Suppose IXDIV is a forward division, and the user wants to
take all its space into one single maximum size bank,
which will be filled by a variable amount of data, but less
than NEED words; when the filling is complete the bank
will be reduced to its true size:
\begin{verbatim}
      NEED = 8000
      CALL MZNEED (IXDIV, NEED, 'G')
      IF (IQUEST(11).LT.0)              GO TO no luck

      ND = NEED + IQUEST(11) - 19       (7 links, 12 system words)
      CALL MZBOOK (IXDIV,L,LSUP,JB, 'name',7,2,ND,2, -1)

      fill words IQ(L+1) TO IQ(L+NU)

      CALL MZPUSH (IXDIV,L,0, NU-ND, 'R')
\end{verbatim}

\Filename{H2-MZINQS-inquiry-to-get-parameters-of-store}
\section{MZINQS - inquiry to get the parameters of a store}

This routine returns the parameters of the selected store in IQUEST:

\ROUTA{CALL MZINQS (IXSTOR)}

\begin{verbatim}
with     IXSTOR     index of the store

On return /QUEST/IQUEST(100) contains the following information:

    IQUEST(1):  opmode   operation mode of the store, = zero normally
    IQUEST(2):  NQSTRU   number of structural permanent links
    IQUEST(3):  NQREF    number of permanent links
    IQUEST(4):  NQLINK   number of permanent + working space links
    IQUEST(5):  LSTA1    first word after the working space
    IQUEST(6):  LSTA21   first word after the store
    IQUEST(7):  NQMINR   minimum size of the reserve area after div. 1
    IQUEST(8):  LQ2END   minimum value of LQEND(2)
    IQUEST(9):  JQDVLL   last low division
   IQUEST(10):  JQDVSY   system division, also: first high division
   IQUEST(11):  NQLOGL   current log level for this store
   IQUEST(12):  NQSNAM   chars 1:4 of the printing name of the store
   IQUEST(13):           chars 5:8
   IQUEST(14):  lowORG   lowest  possible origin-adr for the store
   IQUEST(15):  highORG  highest possible origin-adr for the store
   IQUEST(16):  initalc  initial allocation to last division created
   IQUEST(17):  toffset  offset of the system tables
\end{verbatim} 
The permanent links occupy  LQ(1) to LQ(NQREF).
\begin{verbatim}
The working space links go from LQ(NQREF+1)  to LQ(NQLINK),
the working space data  go from LQ(NQLINK+1) to LQ(LSTA1-1).
\end{verbatim} 
Divsion 1 starts at LSTA1.

Bank space goes from LQ(LSTA1) to LQ(LSTA21-1).

There must always be a gap of at least NQMINR words between divisions 1 and 2.

LQ2END is the lowest possible address for the end of division 2.

The existing short range divisions have numbers 1 to JQDVLL.

The system division has the number JQDVSY at the moment,
this will change if MZDIV is called to create a new long-range division.
If JQDVSY is less than 20 the user's long range divisions have numbers JQDVSY+1 TO 20.

No 'origin link' may have a value less than lowORG or larger than highORG,
both numbers could be negative; they depend on where the link-areas are
physically in absolute memory, relative to the store.

For a secondary store the system tables are at the end of
the memory declared to MZSTOR, they start in LQ(toffset).
For the primary store toffset=0 is returned.

At log-level 2 the information returned is also printed by MZINQS.

\Filename{H2-MZINQD-inquiry-to-get-parameters-of-division}
\section{MZINQD - inquiry to get the parameters of a division}

This routine returns the parameters of the selected division in IQUEST:

\ROUTA{CALL MZINQD (IXDIV)}

\begin{verbatim}
with     IXDIV     index of the division

On return /QUEST/IQUEST(100) contains the following information
about the division IXDIV:

    IQUEST(1):  mode = 0 forward, 1 reverse division
    IQUEST(2):  kind = 1 short range, 2 long, 4 package, 8 system
    IQUEST(3):  start adr LSTA
    IQUEST(4):  end   adr LEND
    IQUEST(5):  LEXTR highest or lowest possible value of LEND or LSTA
    IQUEST(6):  initial size declared to MZDIV
    IQUEST(7):  maximum size declared to MZDIV
    IQUEST(8):  char 1:4 of the printing name of the division
    IQUEST(9):  char 5:8
   IQUEST(10):  JDVSH, if non-zero: sharing neighbour division

This means:

The memory actually occupied by the division goes
from LQ(LSTA) to LQ(LEND-1), ie. LEND-LSTA words.

The reserve space immediately available to the division would go
   from  LQ(LEND)  to LQ(LEXTR-1) for a forward division,
or from  LQ(LEXTR) to LQ(LSTA-1)  for a reverse division (mode=1).
\end{verbatim} 

If JDVSH is non-zero this division shares the reserve area with
the division designated by IXDIV.

At log-level 2 the information returned is also printed by MZINQD.

\Filename{H2-MZDRED-reduce-space-reserved-to-division}
\section{MZDRED - reduce the space reserved to a division}

The situation can arise that one needs a lot of space in some
long-range division during the initialization phase,
which is then abandoned.
Since divisions can only grow, but never shrink, this space
could be blocked for the whole run.

To reduce this space of a particular division
to the initial allocation for for this division
(or to its present occupation, whichever is larger),
one should first CALL MZGARB (IXDIV), to collect dead banks,
followed by

\ROUTA{CALL MZDRED (IXDIV)}

\begin{verbatim}
with
      IXDIV     index of the division

\end{verbatim} 

This will shift all divisions below division IXDIV upwards
by the amount of space liberated.
\chapter{Internal technical details for ZEBRA MZ}
\smark{Chapter 5}

\vspace{20mm}
  \begin{itemize}
     \in{30mm}
     \item[5.1] Master tables
     \item[5.2] Relocation tables
     \item[5.3] MZFORM data structure
  \end{itemize}

\Filename{H2-MZ-Master-Tables}
\section{Master Tables}

\begin{flushright} \ul{\large in /MZCA/:  Master parameters of the process}
\end{flushright}
\begin{verbatim}
NQSTOR       number of the latest store initialized
             (NQSTOR = 0, 1, ..., 15)    set to -1 by MZEBRA

NQOFFT(J+1)  table off-set KQT for store J, J=0,1,...,NQSTOR

NQOFFS(J+1)  store off-set KQS for store J,  = LQSTOR - LQASTO

NQALLO(J+1)  allocation status of store J:
               = 0:  normal, this store belongs to the running
                      process for exclusive use
                -1:  store disconnected
                -2:  read-only store attached by MZATTM
                -3:  store attached by MZATTM

NQIAM        process identifier, currently zero and unused

LQATAB       = LOCF(IQTABV(0))
             absolute adress-1 of the store control table
             for the primary store

LQASTO       = LOCF(LQ(0))
             absolute adress-1 of the general store base

LQBTIS       = LQATAB - LQASTO
             off-set for the primary table to the store base

LQWKTB       adr of the Zebra working area IQWKTB rel. to /ZEBQ/,
                    ie. LQ(LQWKTB) is word IQWKTB(1)
NQWKTB       size of IQWKTB available to MQ, NORMALLY = 2560
                    may be reduced temporarily by FQ
NQWKTT       total size of IQWKTB, = 2560
LQWKFZ       adr of the FZ work area in IQWKTB rel. to /ZEBQ/

MQKEYS(3)    check on MZEBRA initialization already done
NQINIT       initialization status

NQTSYS       size of each system table
             = LOCF(IQDN2(21)) - LOCF(IQTABV(1))

NQM99        unused
NQPERM       = zero: normal, error exit to ZFATAL
                = 1: permit error return
                     (used at the moment only in NZNEED for MZGAR1)

NQFATA       number of words to be printed from IQUEST on fatal
             error termination; zero during normal operation

NQCASE       fatal error case; zero during normal operation

NQTRAC       number of active trace words
MQTRAC(44)   ZEBRA internal trace words
\end{verbatim} 

\begin{flushright} \ul{\large in /MZCB/:  Current store / division}
\end{flushright}

When a particular store is selected by the user, and if it changes,
these numbers are collected into here for easy access.
Also, the parameters relevant for the division
whose free reserve space has been calculated last,
are kept in here.

\begin{verbatim}
JQSTOR       number of the current store
KQT          table off-set for the current store
KQS          store off-set for the current store

JQDIVI       number of the current division

JQDIVR       division whose free space has last been calculated
JQKIND       = JBYT (IQKIND(JQDIVR),21,4)
JQMODE       = JBIT (IQMODE(JQDIVR),1)
JQDIVN       neighbour division of division JQDIVR at the moving end
JQSHAR       = JQDIVN if the two divisions share the same reserve area
             = zero otherwise
JQSHR1       = forward division sharing the reserve area with JQDIVI
JQSHR2       = reverse division sharing the reserve area with JQDIVI
NQRESV       free space available to division JQDIVR

IQCUR(1->12) this vector receives a copy of IQTABV(1->12),
             see just below
\end{verbatim} 
\begin{flushright} \ul{\large IQTABV / IQCUR  vector}
\end{flushright}
\begin{verbatim}
     primary   current  store
     IQTABV    IQCUR

 (1) LQPSTO    LQSTOR    = LOCF(LQ(@0)),  absolute adr-1 of store
 (2) NQPFEN    NQFEND    number of fence words preceding the store

 (3) NQPSTR    NQSTRU    number of structural permanent links
 (4) NQPREF    NQREF     number of permanent links
 (5) NQPLK     NQLINK    number of permanent + working space links

 (6) NQPMIN    NQMINR    minimum size of the reserve area
 (7) LQP2E     LQ2END    minimum value of LQEND(2)

 (8) JQPDVL    JQDVLL    last low division
 (9) JQPDVS    JQDVSY    system division, also: first high division

(10) NQPLOG    NQLOGL    log level for this store

(11) NQPNAM(2) NQSNAM(2) printing name of the store

(13) IQTABV(13)  lowest  possible origin-adr for the store
(14) IQTABV(14)  highest possible origin-adr for the store
(15) IQTABV(15)  initial allocation to last division created
(16) IQTABV(16)  table-offset J of the table, if contained in store
                 ie. LQ(@J+1) is the first word of the table, unless
                 J = 0: the table is not contained in the store
\end{verbatim} 

\begin{flushright} \ul{\large Store control table}
\end{flushright}

This table exists for each store in the 'system table'
at the end of the space
provided for the store by the user in the call to MZSTOR,
except for the primary store when it is held in /MZCC/.

For a particular store JQSTOR, acces to a particular number,
say IQTABV(J), is via
\begin{verbatim}
            KQT  = NQOFFT(JQSTOR+1)
            IVAL = IQTABV(KQT+J)
\end{verbatim} 
ie. by off-setting to the store control table for the primary store.
In the following we give simply the structure of this primary
store control table; exactly the same structure is repeated in every
store control table.

\begin{verbatim}
IQTABV(16)   store parameters, see the separate IQTABV vector
             description on the previous page

LQSYSS(  )   store system links, see the separate LQSYS vector
LQSYSR(  )   descriptions on the next page

LQSTA(21)    start adr of division J, J=1,2,...20
             LQSTA(21) is end-of-store+1
LQEND(20)    end+1 adr of division J
NQDMAX(20)   maximum size of division J

IQMODE(20)   mode of the division J
                bit 1 = 0 / 1  division forward / reverse

IQKIND(20)   kind of the division J
             JBYT(IQKIND(J),21,4)= 1  user short-range division
                                   2  user long-range division
                                   4  package division
                                   8  system division
             JBIT(IQKIND,J) = 1, other bits 1 to 20 are zero

IQRCU(20)    division bits: 'references to' are active, globals unfolded
IQRTO(20)    division bits: 'references to' are active
IQRNO(20)    division bits: 'references to' are permitted (mask)
NQDINI(20)   initial size of division J
NQDWIP(20)   number of times 'wipe' executed
NQDGAU(20)   number of times user garbage collection
NQDGAF(20)   number of times forced garbage collection for no space
NQDPSH(20)   number of times 'push' executed with relocation pass
NQDRED(20)   number of times 'reduced'
NQDSIZ(20)   maximum size of the division seen
IQDN1(20)    chars  1:4
IQND2(20)           5:8 of the printing name of the division

Primary store only:
KQFT         such that LQFSTA(1) is LQSTA(KQFT+1)  (=342 at present)
LQFSTA(21)   used in MZCOPY
\end{verbatim} 

\begin{flushright} \ul{\large in /MZCC/:  LQSYS system link vectors}
\end{flushright}

The vectors LQSYSS and LQSYSR are the system link areas
attached to each store.
Their words are reserved as follows:

\vspace{4mm}
\ul{structural}: LQSYSS

\begin{verbatim}
   1     :  system bank holding the link area descriptions
   2 LQMST:  working link for MQ
   3 LQT :  TZ package support link
   4 LQFS:  FZ package support link
   5 LQFORM:  MZFORM format d/s support link
   7 LQRS:  RZ package support link
  10 LQZEP:  ZEP support link
\end{verbatim} 

\ul{reference}: LQSYSR

\begin{verbatim}
   1 + 2 :  2 working links for MZ
   4 LQFF:  FZ, last control-bank retrieved
   5 LQFI:  FZ, control-bank for current input unit
   6 LQFX:  FZ, control-bank for current output unit
   7 LQRRF:  RZ package working link
\end{verbatim} 

\begin{flushright} \ul{\large Table of Link Areas}
\end{flushright}

This table, one for each store,
is held in a bank in the system division of the store;
its address is  L = LQSYSS(KQT+1)

The table contains one entry of 5 words for each link area
as follows:

\begin{verbatim}
   word 1:  LLA   for  LQ(@LLA)    being the first word,
        2:  LLAE  for  LQ(@LLAE-1) being the  last word of the area

        3:  bits 1-15:  number of structural links
             bit  31      = 0  if permanent link area
                          = 1  if temporary link area

        4:  characters 1:4
        5:             5:8  of the printing name of the area
\end{verbatim} 

The first entry of the table describes the permanent links
plus the working space links of the store;
this area may be of length zero.
The second entry describes the system links contained in
the store control table for this store.
Further entries appear in the order in which they have been
created by either MZLINK or MZLINT.

The table as a whole in the bank at L is as follows:

\begin{verbatim}
         IQ(L+1)   NWU  active size of the table of n entries,
                        NWU = 5*n + 1
   entry 1
             +2)   LLA = 1
             +3)   LLE = 1 + NQLINK
             +4)   number of permanent structural links
             +5)   4HQWSP
             +6)   blank
   entry 2
             +7)   LLA = LOCF(LQSYSS(KQT+1)) - LQSTOR
             +8)   LLE = LLA + number of system links
             +9)   number of structural system links
            +10)   4HQLAS
            +11)   4HYST
   entry 3
            +12)   LLA = LOCF(first word) - LQSTOR
             ...   ...
   entry i

      IQ(L+5i-3)   LLA                    for entry i
             ...   ...
\end{verbatim} 

\Filename{H2-MZ-relocation-tables}
\section{Relocation Tables}

\begin{flushright} \ul{\large Memory Occupation Table}
\end{flushright}

The table has one entry of 8 words for each memory region
taking part in global memory operations.
Its bounds are given by LQMTA and LQMTE:

\begin{verbatim}
      LQMTA  start adr of the first entry
      LQMTE  end+1 adr of the last  entry
\end{verbatim} 

A particular entry has this format:

\begin{verbatim}
   LQ(L+0)   division number
             if zero: pseudo division

   LQ(L+1)   activity flag:

                -1  division is empty
                 0  inactive
                 1  relocation only
                 2  active, eg. global shift, or MZPUSH, or MZREPL
                 3  garbage collection
                 4  wipe

   LQ(L+2)   NWSH, global shift of this division by NWSH words

               -ve  shift to low
                 0  no shift
               +ve  shift to high

   LQ(L+3)   start adr of the first bank
   LQ(L+4)   end+1 adr of the last  bank

   LQ(L+5)   rel. adr of the first associated relocation entry
             ie. LF = LQRTA + LQ(L+5), LQ(LF) is the first entry
             in the link relocation table for this memory region

   LQ(L+6)   rel. adr of the last+1 entry, (only if LQ(L+1)=3)
             = -3 if division with garbage collection reset because
                  of 'table full'

   LQ(L+7)   NFREE, number of words collected or wiped
\end{verbatim} 

\begin{flushright} \ul{\large Link Relocation Table}
\end{flushright}

This table contains the prescription of how any link is to be
updated.

The table covers the 'total relocation interval'
(LFIXLO,LFIXHI),
links pointing outside this interval are not changed.

The area covered by the relocation interval is considered as a
series of alternating 'live' and 'dead' regions,
described by the link relocation table.
One entry in this table specifies a particular live region
\ul{and}
the dead region just behind it.
Each region represents an integral number of live or dead banks,
or alternatively a region of non-occupied store (reserve area).
For the live region the table entry specifies the relocation
constant, for the dead region it specifies whether a structural link
pointing into this region should be bridged or not.

The table is constructed in the largest gap of the ZEBRA stores,
with the following structure:

\begin{verbatim}
  it starts at     LQ(LQTA-1)  holding LFIXLO
  continues        LQ(LQTA)    first entry
                   LQ(LQTA+4)  second entry
           . . .
                   LQ(LQTE-4)  last entry
  and ends at      LQ(LQTE)    holding LFIXHI.

\end{verbatim} 
The format for entry i at  L = LQTA + 4*(i-1) is:

\begin{verbatim}
  LQ(L+0)    LAi:    start adr of the live area
      +1)    LDi:    start adr of the adjacent dead area
      +2)    NRELi:  relocation constant for the live area
      +3)    IFLi:   bridging flag for the dead area
                      zero: no, 1: yes, -1: link to remain unchanged
\end{verbatim} 

The store from LQ(@LAi) to LQ(@LDi-1) is occupied
by live banks.
The store from LQ(@LDi) to LQ(@LAj)-1 with j=i+1
is a dead region,
containing dead banks to be bridged only if  IFLi=1.

The store from LQ(@LFIXLO) to LQ(@LA1)
is a non-bridging dead region.

It is in the nature of this table that the link relocation table
represents at the same time the detailed instructions for
the memory move, if any.

\begin{flushright} \ul{\large relocation parameters in /MZCT/}
\end{flushright}

The parameters in /MZCT/ control the relocation process.

\begin{verbatim}
   IQFLIO      flag I/O, init to zero by MZTABM
               flags special treatment for IO relocation
               non-zero: relocation called from FZIN (also MZCOPY)
               triggers actions in MZRELB:
               1) links pointing outside <LQ(LQTA),LQ(LQTE> are
                  simply reset to zero
               2) do not go to ZFATAL for bank-chaining clobbered
                  but return IQFLIO = -7 to signal bad input structure

   IQGAP(5,4)  returns the parameters of the largest gaps found by MZFGAP
               Entry 'J' in the table contains:
                 IQGAP(1,J) = number of words available
                 IQGAP(2,J) = abs. zebra adr of first word in the gap
                 IQGAP(3,J) = division number
                 IQGAP(4,J) = store number
                 IQGAP(5,J)   user flag
               Gaps J=1 and 2 are outside the moving region,
                      3 and 4 could be inside the moving region

   IQPART      partial garbage collection, init to zero by MZTABM
               handle not enough space for the relocation tables
               if = zero: no problem
               set to =7  by MZTABH if not enough table space
               set to =-7 by MZTABH if table moved into the forbidden gap
               set to =1  by MZTABC if not enough space
               set to =1  by MZTABR if not enough space

   IQTBIT      status-bit for table building, init to IQDROP by MZTABM
   IQTVAL      value of status-bit,           init to   zero by MZTABM
               construct table for banks having IQTVAL in bit IQTBIT

   IQTNMV      init to zero by MZTABM if JQSTMV < 0 on entry
                set to zero by MZGAR1, MZDIV
                set to   -7 by MZCOPY
               used by MZTABH on first entry:
               = 0: alright to move Mem.occ.table to forbidden gap
                     return IQPART = 7 if no gap at all
               < 0: alright to move Mem.occ.table to forbidden gap
                     return IQPART = -7 if so done
               > 0: Mem.occ.table may not go into a forbidden gap
                     return IQPART = 7 if no allowed gap

   JQGAPM      gap for Memory occupation table, init to zero by MZTABM
               if non-zero: the Memory occupation table has been moved
               to gap JQGAPM described by IQGAP(1/5,JQGAPM)

   JQGAPR      gap for the Relocation table, init to zero by MZTABM
               if non-zero: the Link Relocation table has been moved
               to gap JQGAPR

\end{verbatim}

\begin{verbatim}
   JQSTMV      moving store, -1 if none, controlling MZFGAP
   JQDVM1      first moving division
   JQDVM2      last  moving division
   NQDVMV      move divisions JQDVM1/2 by that many words, -ve: left
               JQDVM1, JQDVM2, NQDVMV init to zero
               by MZTABM if JQSTMV < 0 on entry

      Original meaning of JQSTMV,JQDVM1,JQDVM2,NQDVMV (cf. MZTABS):
      unless JQSTMV<0, the divisions JQDVM1 to JQDVM2 inclusive
      of the store JQSTMV may have to be shifted by NQDVMV words,
      to the left if -ve, to the right if +ve.
      NQDVMV=0 means to the left by a yet unknown amount.
      It follows that the gaps after divisions JDIV are not available
      for the relocation table because they may be over-written
      by the memory move, with:
         NQDVMV  0, -ve:  JDIV = <JQDVM1-1,JQDVM2-1>
                    +ve:  JDIV = <JQDVM1,  JQDVM2  >

      MZFGAP is also used for I/O to find a gap where to put the
      relocation table on input. In this case there is just one
      gap unavailable, namely the one which will receive the data.
      This is the gap before a reverse division (JQMODE=1) or after
      a forward division (JQMODE=0). To block this gap one can give
      JQDVM1=JQDVM2=JDIV and NQDVMV=1-2*JQMODE, setting JQSTMV also.

   LQMTE       end+1 adr of the Memory occupation table,
                     init to LQWKTB+NQWKTB-1 by MZTABM
   LQMTA       start adr of the Memory occupation table,
                     init to LQMTE-160 by MZTABM
   LQMTB       curr. adr of the Memory occupation table,
                     init to LQMTA by MZTABM

   LQMTLU      adr in the Mem.occ.table of the last division used
               init by MZTABM to point to division 20
               reset by MZTABX

   LQMTBR      init to zero by MZTABM
               set by MZTABR to point to the current entry into
               the Memory occupation table when it runs out of
               table space.
               The code handling this is a remenant from an
               earlier approach to handle "table space full"
               and has been left in MZTABR for safety.
               In fact, now MZGAR1 recalls MZTABM after the first
               partial garbage collection.
               (The code handling LQ(LQMTB+6) = -3 is also part
               of this old approach)

   LQRTA       start adr of the memory available to the Relocation table
               init to LQWKTB by MZTABM
   LQRTE       end+1 adr of the memory available to the Relocation table
               init to LQMTA-10 by MZTABM
               both are updated by MZTABH if the tables are moved

   LQTA        start adr of the Link Relocation table,
               init to LQRTA+1 by MZTABM
   LQTE        end+1 adr of the Link relocation table,
               init to LQRTE    by MZTABM

\end{verbatim}

\begin{verbatim}
               parameters driving MZTABC, set by MZTABR
   LQMTC1      start adr of first bank
   LQMTC2      end+1 adr of  last bank
   LQTC1       first table word available, init to LQRTA+1 by MZTABM
   LQTC2       last  table word available, init to LQRTE   by MZTABM

   MQDVGA      flag word indicating the divisions with garbage collection
               bit j, j=1,2,...,20 for garbage collection in division j

   MQDVWI      flag word indicating the divisions to be wiped
               bit j, j=1,2,...,20 for wiping division j

   MQDVAC      flag word indicating the active divisions
               constructed from MQDVGA + MQDVWI by MZTABM
               and updated by MZTABS and MZTABR
               seems to be used only for printing

   NQDVMV      move divisions JQDVM1/2 by that many words, -ve: left
               init to zero by MZTABM if JQSTMV < 0 on entry

   NQFREE      number of words to become free, init to zero by MZTABM
   NQFRTC      number of words found free by MZTABC (per call)

   NQGAPN      number of normal gaps available in IQGAP
   NQGAP       number of all gaps, including forbidden, in IQGAP

   NQLIVE      number of live banks found by MZTABC (per call)

   NQNOOP      no-operation flag constructed by MZTABF
               = zero normally
               = -7 really no operation
               = +7 only a memory shift with no garbage collection
                    and no links pointing into the shifted region
                    for example: left-shift empty division 2
               usage in MZMOVE: if non-zero update the division tables
                                 without execution of MOVE
\end{verbatim} 

\Filename{H2-MZFORM-data-structure}
\section{MZFORM data structure}

\begin{verbatim}
                                 LID

                                  |
                                 / \
                ______________         ______________
               |              |       |              |
LQFORM ---->   |     QID      |       |     QID      |  list of
               | for +ve ID's |       | for -ve ID's |  identifiers
            -- |______________|       |______________|
           |
           |      |                      |
           | -2   | -1                   | -1
           |    ______________         ______________
           |   |              |       |              |
           |   |     QIOX     |       |     QIOX     |  adr in QIOD
           |   |     +ve      |       |     -ve      |  corresponding
           |   |______________|       |______________|
           |
           |                     \ /
           |                      |
           |
           |                     LIX
          __________
         |          |
LIOD ->  |   QIOD   |   I/O characteristics
         |__________|   stored
\end{verbatim} 
\begin{verbatim}

  IQ(LID+1)  n = number of ID's stored
         2   ji  of the last characteristic retrieved
         3   IDH of the last characteristic retrieved
         4   IDH  #1
             ...
       n+3   IDH  #n


  IQ(LIX+1)  ji in QIOD for IDH #1
         2   ji         for IDH #2
             ...


 IQ(LIOD+1)  n = number of words occupied;          j1 = 1
      j1+1   start of 1st characteristic, n1 words, j2 = j1 + n1
      j2+1            2nd                 n2 words, j3 = j2 + n2
             ...

    characteristic i is stored at ji in QIOD as NW+1 words:

         word 1:   16 bits: control-byte |
                    | 5 bits: NW | 5 bits: NW+1 | 6 bits: 1
         word 2:   first extra I/O word (if any)
                    ...
      word NW+1:   last  extra I/O word
\end{verbatim} 

\Filename{H1-Summary-of-KERNLIB-routines}
\chapter{Summary of KERNLIB routines}

% N103  JX = IUWEED (AV,N)              locate floating-point exception in AV
\spsmall

\begin{verbatim}
        The first letter indicates the mode of the variable according
        to the Fortran convention, thus I -> N is INTEGER, else REAL,
        and we add: small c  to designate CHARACTER
                    small l  to designate LOGICAL
     *  marks output variables
     V  as last or last-but-one letter indicates a vector variable
     W  as last or last-but-one letter indicates a matrix variable
    sW  as last letters indicate a packed symmetric  matrix variable
    tW  as last letters indicate a packed triangular matrix variable
     M  as last letter means that the mode of the variable is immaterial
    FL  stands for FORTRAN Library

      COMMON /SLATE/ISL(40)  returns extra information

. General
---------

Z035  CALL ABEND                      abnormal job-step termination
Z007  CALL DATIME (IDATE*,ITIME*)     integer date / time, IDATE= yymmdd, ITIME= hhmm
                                      ISL(1/6) = 19yy mm dd hh mm ss
Z007  CALL DATIMH (IDTV*,ITMV*)       Hollerith date and time
                                      IDT= 8Hyy/mm/dd, ITM= 8Hhh.mm.ss
M220  CALL IE3FOD (DV,TV*,NDP,JBAD)   convert IEEE <- Double
M220  CALL IE3FOS (SV,TV*,N,  JBAD)   convert IEEE <- Single
M220  CALL IE3TOD (TV,DV*,NDP,JBAD)   convert IEEE -> Double
M220  CALL IE3TOS (TV,SV*,N,  JBAD)   convert IEEE -> Single
Z044  lo = INTRAC ()                  .true. if interactive running
Y201  IX = IUBIN  (A,PAR,SPILL)       histogram bin,  PAR=NA,DA,ALOW
Y201  IX = IUCHAN (A,ALOW,DA,NA)      histogram bin IX
V304  JX = IUCOMP (IT,IAV,N)          find first IT in IA, JX=O if not found
V304  JX = IUCOLA (IT,IAV,N)          find  last IT in IA, JX=O if not found
V304  JX = IUFIND (IT,IAV,JL,JR)      find first IT in IA(JL/JR), JX=JR+1 if not found
V304  JX = IUFILA (IT,IAV,JL,JR)      find  last IT in IA(JL/JR), JX=JR+1 if not found
Y201  IX = IUHIST (A,PAR,SPILL)       histogram bin,  PAR=NA,1./DA,ALOW
V304  JX = IUHUNT (IT,IAV,N,NA)       find IT in IA(1/N),every NA, JX=0 if not found
V304  JX = IULAST (IT,IAV,N)          find last word in IA(1/N) not having IT
                                                   JX=0 if all are IT
M501  NX = IUSAME (IAV,JL,JR,MIN,JS*) search IA(JL/JR) for string of at
                                      least MIN same elements,
                                      if found: NX same elements, first is IA(JS)
                                      else:     NX=0, JS=JR+1
Z100  CALL JOBNAM (IDV*)              get name of job into 8 char. Hollerith
Z042  IAD= JUMPAD (external)          get the target transfer adr
Z042  CALL JUMPST (IAD)               set the target transfer adr
Z042  CALL JUMPX0                     transfer with no parameters as set
Z042  CALL JUMPX1 (p1)                transfer with  1 parameter
Z042  CALL JUMPX2 (p1,p2)             transfer with  2 parameters
Z043  CALL JUMPT0 (IAD)               transfer with no parameters to IAD
Z043  CALL JUMPT1 (IAD,p1)            transfer with  1 parameter
Z043  CALL JUMPT2 (IAD,p1,p2)         transfer with  2 parameters
Z001  CALL KERNGT (LUN)               print current version of KERNLIB
M432  NX = LNBLNK (cTEXT)             find last non-blank character in cTEXT
N100  JX = LOCF   (A)                 absolute word adr of variable A
N100  JX = LOCB   (A)                 absolute byte adr of variable A
M215  FA = PSCALE (NX,NMAX,A,NDIG)    find power of ten to scale for printing
Z041  CALL QNEXTE                     enter or re-enter into user routine QNEXT
V104  X  = RNDM   ()                  simple random number
V104  CALL RDMIN  (ISEED)             set the seed for RNDM
V104  CALL RDMOUT (ISEED*)            get the seed for RNDM
M107  CALL SORTI  (IXW*,NCOL,NROW,JEL)  sort rows of integer matrix on element JEL
M107  CALL SORTR   (XW*,NCOL,NROW,JEL)  sort rows of    real matrix on element JEL
M107  CALL SORTD  (DXW*,NCOL,NROW,JEL)  sort rows of  double matrix on element JEL
N203  CALL TCDUMP (TITL,AVM,N,MODE)   MODE=3HFIH  floating,integer,hollerith
\end{verbatim}

\begin{verbatim}
Z007  CALL TIMED  (T*)                T= seconds since last call to TIMED
Z007  CALL TIMEL  (T*)                T= seconds left until time-limit
Z007  CALL TIMET  (T*)                T= seconds of job ex. time used so far
Z007  CALL TIMEST (TLIM)              init. for TIMEL on IBM
N105  CALL TRACEQ (LUN,N)             print subroutine trace-back N levels deep
V300  CALL UBLANK (IXV*,JL,JR)        IX(J)= 'blanks'           for J=JL,JR
M409  CALL UBLOW  (IAm,  IA1*,  NCH)  copy Hollerith  Am to A1, NCH characters
M409  CALL UBUNCH (IA1,  IAm*,  NCH)  copy Hollerith  A1 to Am
M409  CALL UCTOH  (cHO,  IAn*,N,NCH)  copy Char/Holl  Am to An
M409  CALL UCTOH1 (cHO,  IA1*,  NCH)  copy Char/Holl  Am to A1
M409  CALL UHTOC  (IAn,N,cCH*,  NCH)  copy Hollerith  An to Character
M409  CALL UH1TOC (IA1,  cCH*,  NCH)  copy Hollerith  A1 to Character
V301  CALL UCOPY  (AVM,XVM*,N)        copy X(J) = A(J)          for J=1,N
V301  CALL UCOPY2 (AMV,XVM*,N)        copy A to X, any overlap
V301  CALL UCOPYN (IAV,IXV*,N)        copy -ve integer:   IX(J) = -IA(J)
V301  CALL UCOPIV (AVM,XVM*,N)        copy inverted, eg. X(1)=A(N)
V302  CALL UCOCOP (AVM,XVM*,IDO,N,NA,NX) copy IDO times N words, every NA,NX
V302  CALL UDICOP (AVM,XVM*,IDO,N,NA,NX) copy IDO times N words, every NA,NX
V300  CALL UFILL  (XVM*,JL,JR,AM)     X(J)= A                   for J=JL,JR
M502  CALL UOPT   (IACT,IPOSS,IXV*,N) select options from poss., Hollerith
M502  CALL UOPTC  (cACT,cPOSS,IXV*)   select options from possibilities
V301  CALL USWOP  (XVM*,YVM*,N)       swop  X(J)=Y(J), Y(J)=X(J)
V300  CALL UZERO  (IXVM*,JL,JR)       IX(J)= O                  for J=JL,JR
F121  CALL VBLANK (IXV*,N)            IX(J)= hollerith BLANK    for J=1,N
F121  CALL VFILL  (XVM*,N,AM)         X(J)= A                   for J=1,N
F121  CALL VFIX   (AV,IXV*,N)         IX(J) =  A(J)             for J=1,N
F121  CALL VFLOAT (IAV,XV*,N)          X(J) = IA(J)             for J=1,N
F121  CALL VZERO  (IXVM*,N)           IX(J)= O                  for J=1,N
J200  CALL VIZPRI (LUN,cTEXT)         print 1 line of large characters
Z203  CALL XINB   (LUN,XV*,*NX*)      var.  length:   READ (LUN) NX,XV
Z203  CALL XINBF  (LUN,XV*,*NX*)      fixed length:   READ (LUN) XV
Z203  CALL XINBS  (LUN,XAV*,NA,XV*,*NX*)     split:   READ (LUN) NX,XAV,XV
Z203  CALL XOUTB  (LUN,V,N)           var.  length:  WRITE (LUN) N,V
Z203  CALL XOUTBF (LUN,V,N)           fixed length:  WRITE (LUN) V
Z203  CALL XOUTBS (LUN,AV,NA,V,N)     split mode:    WRITE (LUN) N,AV,V
\end{verbatim}

\begin{verbatim}
. Bitwise logical operations on full words
-------------------------------------------------

M441  IX = IAND   (IWD1,IWD2)    logical AND
M441  IX = IEOR   (IWD1,IWD2)    logical exclusive OR
M441  IX = IOR    (IWD1,IWD2)    logical OR
M441  IW = NOT    (IWD)          logical NOT
M441  IX = ISHFT  (IWD,NSH)     +ve:  logical  left shift by  NSH places
                                -ve:  logical right shift by -NSH places
M441  IX = ISHFTC (IWD,NSH,NBITS) left-circular shift by (+-)NSH places
                                  of the right-most NBITS bits

. Bit / byte handling, least significant bit is 1
-------------------------------------------------

  symbolic:  'byt' ==  IWD,J,NBITS  is byte at J of NBITS bits in IWD

M421  IX = JBIT   (IWD,J)        get bit J
M421  IX = JBYT   ( byt)         get byte at J of NBITS bits in IWD
M421  IX = JBYTET (IA, byt)      get logical AND of IA and "byt"
M421  IX = JBYTOR (IA, byt)      get logical  OR of IA and "byt"
M421  IX = JRSBYT (IA, byt*)     get "byt" and reset it to IA

M421  CALL  SBIT  (I,IWD*,J)     set in IWD   bit J to I
M421  IX = MSBIT  (I,IWD, J)     get IWD with bit J set to I
M421  CALL  SBIT0   (IWD*,J)     set in IWD   bit J to zero
M421  IX = MSBIT0   (IWD, J)     get IWD with bit J set to zero
M421  CALL  SBIT1   (IWD*,J)     set in IWD   bit J to 1
M421  IX = MSBIT1   (IWD, J)     get IWD with bit J set to 1
M421  CALL  SBYT  (I, byt*)      set in IWD   "byt" to I
M421  IX = MSBYT  (I, byt)       get IWD with "byt" set to I
M421  CALL SBYTOR (I, byt*)      set in IWD   "byt"     to  OR of "byt" and I
M421  IX = MBYTOR (I, byt)       get IWD with "byt" set to  OR of "byt" and I
M421  IX = MBYTET (I, byt)       get IWD with "byt" set to AND of "byt" and I
M421  CALL  CBYT  (IWS,JS, byt*) copy byte at JS in IWS to "byt"
M421  IX = MCBYT  (IWS,JS, byt)  get IWD with "byt" set to byte at JS of IWS

M503  CALL UBITS  (IAV,NBITS,IXV*,NX*)  make list of bit-nos of non-zero bits
M428  JX = LOCBYT (I,IAV,N,NEV,J,NBITS)  is IUHUNT for byte-content I

. Handling packed byte vectors
------------------------------

  MPAK=NBITS,INWORD: bytes of NBITS bits packed r-to-l, INWORD of them per word
       if NBITS = 0: assume NBITS=1 and INWORD = maximum

M423  IX = INCBYT (INC,IPV*,J,MPAK)   increment packed byte
M422  IX = JBYTPK (IPV,J,MPAK)        get packed byte
M422  CALL SBYTPK (I,IPV*,J,MPAK)     set packed byte
M422  CALL PKBYT  (IAV,IPV*,J,N,MPAK) pack byte-vector right-to-left
M422  CALL UPKBYT (IPV,J,IAV*,N,MPAK) unpack byte-vector right-to-left

  MPAR=NBITS,NCHAR,NZONE,IGNOR,NFILL   packing control, l-to-r

M427  CALL PKCHAR (IAV,IPV*,N,MPAR)   pack integers left-to-right
M427  CALL UPKCH  (IPV,IAV*,N,MPAR)   unpack byte-vector left-to-right

. Bit / byte handling, least significant bit is 0
-------------------------------------------------

M441  IX = IBITS  (IWD,J,NBITS)  get byte at J of NBITS bits
M441  lo = BTEST  (IWD,J)        true if bit J is 1
M441  IX = IBSET  (IWD,J)        IX= IWD with bit J set to 1
M441  IX = IBCLR  (IWD,J)        IX= IWD with bit J set to 0
M441  CALL MVBITS (IA,JA,NBITS,IWD*,J)  store byte at JA of IA into byte at J in IWD
\end{verbatim}

\begin{verbatim}
. Unix functions
----------------

      some use COMMON /SLATE/ND,DUMMY(39), and status IST=0 if good

Z265  IST= CHDIRF (cNAME)             set current working directory
Z265  CALL CTIMEF (ICLOCK,cTIME*)     decode time from STATF to TIME*24
Z265  CALL GETENVF(cNAME,cVAL*)       get value of environment variable
                                      ND = LNBLNK(cVAL)  =0 if not found
Z265  CALL GETPIDF(IPID*)             get ID of the current process
Z265  CALL GETWDF (cNAME*)            get current working directory
                                      ND = LNBLNK(cNAME)
Z265  CALL GMTIMEF(ICLOCK,ITM*)       decode time from STATF TO ITM(1-9)
      CALL JMPSET (AREA*,external)    do "setjmp" and go to "external"
      CALL JMPLONG(AREA,NUM)          do "longjmp"
Z265  IST= KILLF  (IPID,ISGNAL)       send signal to process IPID
Z265  CALL PERRORF(cTEXT)             print last Unix error tagged with cTEXT
Z265  IST= RENAMEF(cFROM,cTO)         rename file cFROM --> cTO
      IPR= SIGNALF(NUMSIG,ext,IFLAG)  establish signal handler
Z265  CALL SLEEPF (NSECS)             suspend process for NSECS seconds
Z265  IST= STATF  (cNAME,INF*)        get info about file cNAME to INF(1-12)
Z265  IST= SYSTEMF(cCOMMAND)          submit shell command
      CALL TMINIT (INIT*)             initialize TMPRO / TMREAD
      CALL TMPRO  (cPROMPT)           display prompt on terminal
      CALL TMREAD (MAXCH,cLINE*,NCH*,IST*)  read line from terminal
                                      NCH characters read into LINE
                                      IST -ve: EoF signal

. Z310  C interface to read/write fixed-lenth records, CFIO
-----------------------------------------------------------

  symbolic  "lmr" == LUNDES,MEDIUM,NWREC

            LUNDES: file-descriptor of C, output parameter of CFOPEN
            MEDIUM: 0 disk, 1 tape, 2 user disk, 3 user tape
            NWREC:   number of Fortran words per record
            status return ISTAT is zero for success

  CALL CFOPEN (LUNDES*,MEDIUM,NWREC,cMODE,0,cNAME,ISTAT*)   open the file cNAME
                                          cMODE= r r+ w w+ a a+
  CALL CFGET  ("lmr",*NB*,MBUF*,ISTAT*)   read next record into MXBUF
                                             in: NB words to be tranfered
                                            out: NB words transfered
  CALL CFPUT  ("lmr",MBUF, ISTAT*)        write next record
  CALL CFSEEK ("lmr",NREC, ISTAT*)        set current file position
  CALL CFTELL ("lmr",NREC*,ISTAT*)        get current file position
  CALL CFSIZE ("lmr",NREC*,ISTAT*)        seek file to end and get its size
                                     NREC: so many records before the next
  CALL CFREW  (LUNDES,MEDIUM)             rewind the file
  CALL CFCLOS (LUNDES,MEDIUM)             close  the file
\end{verbatim}

\begin{verbatim}
. Maths General
---------------

FL     X = ACOS   (A)                 arcus cosinus, 0 -> PI
FL     X = ASIN   (A)                 arcus sinus, -PI/2 -> PI/2
FL     X = ATAN   (A)                 arcus tangens, -PI/2 -> PI/2
FL     X = ATAN2  (RSIN,RCOS)         arcus tangens, -PI -> PI
B101   X = ATG    (RSIN,RCOS)         arcus tangens, 0 -> 2*PI
F117  CALL CROSS  (AV,BV,XV*)         A CROSS B  into  X
F116   X = DOTI   (AV,BV)             X = A(1)B(1) +...+ A(3)B(3) - A(4)B(4)
C300   X = ERF    (A)                 error function, integral 0 -> A
C300   X = ERFC   (A)                 compl. error function, A to infinity
E104   X = FINT   (...)               interpolation routine
C300   X = FREQ   (A)                 normal frequence function, -INF to A
U101  CALL LOREN4 (AV,BV,XV*)         Lorentz transformation
U102  CALL LORENB (EN,REFV,STV,XV*)   Lorentz transformation, backward
U102  CALL LORENF (EN,REFV,STV,XV*)   Lorentz transformation, forward
G100   X = PROB   (CHI2,N)            convert CHI-square to probability
B102   X = PROXIM (ALPHA,REF)         X = ALPHA + 2N*PI  nearest to REF
F118  CALL ROT    (AV,TH,XV*)         rotate around Z-axis
FL     X = TAN    (A)                 tangens

. Maths  F 121  Vector handling package, VECMAN
-----------------------------------------------

E103   X = AMAXMU (AV,IDO,NWD,NA)     largest ABS element in scattered vector
F121  LX = LVMAX  (AV,N)              loc of biggest  A(J)      for J=1,N
F121  LX = LVMAXA (AV,N)              loc of biggest  ABS(A(J))
F121  LX = LVSMX  (AV,N,INC)          loc of biggest  A(J) every INC
F121  LX = LVSDMX (DV,N,INC)          loc of biggest  D(J) every INC, double
F121  LX = LVSIMX (IV,N,INC)          loc of biggest  I(J) every INC
F121  LX = LVMIN  (AV,N)              loc of smallest A(J)
F121  LX = LVMINA (AV,N)              loc of smallest ABS(A(J))
F121  LX = LVSMI  (AV,N,INC)          loc of smallest A(J) every INC
F121  LX = LVSDMI (DV,N,INC)          loc of smallest D(J) every INC, double
F121  LX = LVSIMI (IV,N,INC)          loc of smallest I(J) every INC
F121  CALL VADD   (AV,BV,XV*,N)       X(J) = A(J) + B(J)        for J=1,N
F121   X = VASUM  (AV,N)              X = sum ABS(A(J))         for J=1,N
F121  CALL VBIAS  (AV,C,XV*,N)        X(J) = A(J) + C           for J=1,N
F121  CALL VCOPYN (AV,  XV*,N)        copy -ve:   X(J) = -A(J)  for J=1,N
F121   X = VDIST  (AV,BV,N)           X = SQRT (VDIST2(A,B,N))
F121   X = VDIST2 (AV,BV,N)           X = (A-B)*(A-B)
F121   X = VDOT   (AV,BV,N)           X = A * B
F121   X = VDOTN  (AV,BV,N)           X =  A*B / SQRT(A*A * B*B)
F121   X = VDOTN2 (AV,BV,N)           X = (A*B)**2 / (A*A * B*B)
F121  CALL VEXCUM (AV,XV*,N)          X=Minimum,Maximum,Sum -- cumulative
F121  CALL VFILL  (XVM*,N,AM)         X(J) = A                  for J=1,N
F121  CALL VFIX   (AV,IXV*,N)         IX(J)=  A(J)              for J=1,N
F121  CALL VFLOAT (IAV,XV*,N)         X(J) = IA(J)              for J=1,N
F121  CALL VLINCO (AV,S,BV,T,XV*,N)   X(J) = A(J)*S + B(J)*T    for J=1,N
F121  CALL VMATL  (GW,CV,XV*,NI,NJ)   X = G * C
F121  CALL VMATR  (AV,GW,YV*,NI,NJ)   Y = A * G
F121   X = VMAX   (AV,N)              biggest  A(J)             for J=1,N
F121   X = VMAXA  (AV,N)              biggest  ABS(A(J))        for J=1,N
F121   X = VMIN   (AV,N)              smallest A(J)             for J=1,N
F121   X = VMINA  (AV,N)              smallest ABS(A(J))        for J=1,N
F121   X = VMOD   (AV,N)              X = SQRT ( VDOT(A,A,N) )
F121  CALL VMUL   (AV,BV,XV*,N)       X(J) = A(J) * B(J)        for J=1,N
F121  CALL VSCALE (AV,C, XV*,N)       X(J) = A(J) * C           for J=1,N
F121  CALL VSUB   (AV,BV,XV*,N)       X(J) = A(J) - B(J)        for J=1,N
F121   X = VSUM   (AV,N)              X = sum A(J)              for J=1,N
F121  CALL VUNIT  (AV,XV*,N)          X(J) = A(J) / VMOD(A,N)   for J=1,N
F121  CALL VZERO  (IXVM*,N)           IX(J)= O                  for J=1,N
\end{verbatim}

\begin{verbatim}
. Maths  Householder fitting and triangular matrices
----------------------------------------------------

E230  CALL TLERR  (AW,XW*,AUX,IPIV)   error matrix after fit
E230  CALL TLRES  (AW,XV*,AUX)        residuals after fit
E230  CALL TLS    (AW,BW,AUX,IPIV,EPS,XW*)      unconstrained L.S.FIT
E230  CALL TLSC   (AW,BW,AUX,IPIV,EPS,XW*)      constrained L.S.FIT

F112  CALL TRAAT  (AW,XsW*,M,N)       rectang * rectang(T) X = A * AT
F112  CALL TRAL   (AW,BtW,XW*,M,N)    rectang * triang     X = A * B
F112  CALL TRALT  (AW,BtW,XW*,M,N)    rectang * triang     X = A * BT
F112  CALL TRAS   (AW,BsW,XW*,M,N)    rectang * symm       X = A * B
F112  CALL TRASAT (AW,BsW,XsW*,M,N)   transform symm       X = A * B * AT
F112  CALL TRATA  (AW,XsW*,M,N)       rectang(T) * rectang X = AT* A
F112  CALL TRATS  (AW,BsW,XW*,M,N)    rectang * symm       X = AT* B
F112  CALL TRATSA (AW,BsW,XsW*,M,N)   transform symm       X = AT* B * A
F112  CALL TRCHLU (AsW,XtW*,N)        Choleski decomposition A = X*XT
F112  CALL TRCHUL (AsW,XtW*,N)        Choleski decomposition A = XT*X
F112  CALL TRINV  (AtW,XtW*,N)        inversion of triangular matrix
F112  CALL TRLA   (AtW,BW,XW*,M,N)    triang. * rectang.   X = A * B
F112  CALL TRLTA  (AtW,BW,XW*,M,N)    triang. * rectang.   X = AT* B
F112  CALL TRPCK  (AW,XsW*,N)         pack A into symmetric form
F112  CALL TRQSQ  (AsW,BsW,XsW*,N)    transform symm       X = A * B * A
F112  CALL TRSA   (AsW,BW,XW*,M,N)    symm * rectang       X = A * B
F112  CALL TRSAT  (AsW,BW,XW*,M,N)    symm * rectang       X = A * BT
F112  CALL TRSINV (AsW,XsW*,N)        inversion of symmetric matrix
F112  CALL TRSMLU (AtW,XsW*,N)        product of triang matrices  X = A * AT
F112  CALL TRSMUL (AtW,XsW*,N)        product of triang matrices  X = AT* A
F112  CALL TRUPCK (AsW,XW*,N)         unpack symm. A into full  form
\end{verbatim}

\begin{verbatim}
. M432  Utilities for Character string analysis, CHPACK
--------------------------------------------------------

  using  COMMON /SLATE/ND,NE,NF,NG,NUM(2),DUMMY(34) to return information
                standard meaning:  ND: number of digits or characters seen
                                   NE: COL(NE) is the terminating character

         CHARACTER    LINE*(512), COL(512)*1
         EQUIVALENCE (LINE,COL)

  most routines have the 3 parameters  LINE,JL,JR  to designate the
  field LINE(JL:JR) to be used, abbreviated to 'llr' if short of space.

CALL CFILL  (cIT,LINE*,JL,JR) fill llr with as many copies of cIT*(*) as poss.
CALL CKRACK (LINE,JL,JR)      krack numeric field; ND digits seen, NE term.
                              NF= -ve bad, 0 blank, 1 B, 2 I, 3 F, 4 D seen
                              NG= 0 good termination, NUM returns the number
CALL CLEFT  (LINE*,JL,JR)     left-justify squeezing blanks,
                              ND non-blanks, COL(NE) first blank
CALL CRIGHT (LINE*,JL,JR)     right-justify squeezing blanks,
                              ND non-blanks, COL(NE) last blank
CALL CLTOU  (LINE(JL:JR)*)    convert low to up
CALL CUTOL  (LINE(JL:JR)*)    convert  up to low
CALL CSETDI (INT,LINE*,JL,JR) set decimal integer right-justified, ND digits
                              COL(NE+1)  most significant digit set
                              COL(NF+1)  most significant character set
                              NG=0 good, else field too small
CALL CSETHI (INT,LINE*,JL,JR) set hex integer right-justified,
                              ND,NE,NF,NG as for CSETDI
CALL CSQMBL (LINE*,JL,JR)     left-justify squeezing multiple blanks,
                              ND retained, COL(NE) first after
                              NE=JR+1 if no multiple blanks
CALL CSQMCH (cSG,LINE*,JL,JR) left squeeze multiple occurrences of cSG*1
                              ND,NE as for CSQMBL
CALL CTRANS (cOLD,cNEW, llr*) replace each occurrence of cOLD*1 by cNEW*1
IX = ICDECI (LINE,JL,JR)      read decimal integer, ND digits, COL(NE) term.
                              NG=0 if terminated by blank or end-of-field
JX = ICFIND (cSG,LINE,JL,JR)  find COL(JX) first occ. of cSG*1 or JX=JR+1
                              NG=0 not found, else =JX
JX = ICFILA (cSG,LINE,JL,JR)  find COL(JX) last occ. of cSG*1 or JX=JR+1
                              NG=0 not found, else =JX
JX = ICFMUL (cIT,LINE,JL,JR)  find COL(JX) first occ. of any cIT(j:j) or JX=JR+1
                              ND=j, NG=0 not found, else =JX
JX = ICFNBL (LINE,JL,JR)      find COL(JX) first non-blank, or JX=JR+1
                              NG=0 all blank, else =JX
IX = ICHEXI (LINE,JL,JR)      read hex integer, ND,NE,NG as for ICDECI
JX = ICLOC  (cIT,NI, llr)     locate cIT(1:NI) as is in LINE(JL:JR)
                              COL(JX) start, JX=0 if not found
JX = ICLOCL (cIT,NI, llr)     as ICLOC case insensitive, cIT given as lower
JX = ICLOCU (cIT,NI, llr)     as ICLOC case insensitive, cIT given as upper
JX = ICLUNS (LINE,JL,JR)      COL(JX) first 'unseen', else JX=0
JX = ICNEXT (LINE,JL,JR)      LINE(JX:NE-1) is next 'word' in llr
                              ND chars in word; no next: JX=NE=JR+1 ND=0
JX = ICNTH  (cACT,cPOSS,NPO)  cACT as is matches cPOSS(JX), else JX=0
JX = ICNTHL (cACT,cPOSS,NPO)  as ICNTH case insensitive, cPOSS given as lower
JX = ICNTHU (cACT,cPOSS,NPO)  as ICNTH case insensitive, cPOSS given as upper
JX = ICNUM  (LINE,JL,JR)      find COL(JX) first non-numeric, non-blank
                              JX=JR+1 NG=0 if none, ND digits before JX
JX = ICNUMA (LINE,JL,JR)      COL(JX) first non-alphameric, non-blank
                              JX=JR+1 if none, ND alphamerics chars.
                              NG=0 all alphanumeric, else =JX
                              COL(NE) first numeric, else NE=0
                              COL(NF) first alpha,   else NF=0
IX = ICTYPE (cSG)             cSG*1 of type IX = 0 unseen, 1 others,
                                    2 numeric, 3 lower, 4 upper case
NX = LNBLNK (LINE(JL:JR))     find last non-blank character
\end{verbatim}
