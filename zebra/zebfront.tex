%%%%%%%%%%%%%%%%%%%%%%%%%%%%%%%%%%%%%%%%%%%%%%%%%%%%%%%%%%%%%%%%%%%
%                                                                 %
%   ZEBRA DZ - Reference Manual -- LaTeX Source                   %
%                                                                 %
%   Front Material: Title page,                                   %
%                   Copyright Notice                              %
%                   Preliminary Remarks                           %
%                   Table of Contents                             %
%   EPS files     : cernlogo.eps, cnastit.eps                     %
%                                                                 %
%   Editor: Michel Goossens / CN-AS                               %
%   Last Mod.: 27 Jan 1995  9:00 mg                               %
%                                                                 %
%%%%%%%%%%%%%%%%%%%%%%%%%%%%%%%%%%%%%%%%%%%%%%%%%%%%%%%%%%%%%%%%%%%

%%%%%%%%%%%%%%%%%%%%%%%%%%%%%%%%%%%%%%%%%%%%%%%%%%%%%%%%%%%%%%%%%%%%
%    Tile page                                                     %
%%%%%%%%%%%%%%%%%%%%%%%%%%%%%%%%%%%%%%%%%%%%%%%%%%%%%%%%%%%%%%%%%%%%
\def\Ptitle#1{\special{ps: /Printstring (#1) def}
                       \epsfbox{cnastit.eps}}
\begin{titlepage}
\vspace*{-23mm}
\includegraphics[height=30mm]{cern15.eps}%
\hfill
\raisebox{8mm}{\Large\bf CERN Program Library Long Writeups Q100/Q101}
\hfill\mbox{}
\begin{center}
\mbox{}\\[6mm]
\mbox{\Ptitle{ZEBRA}}\\[2cm]
{\LARGE Overview of the ZEBRA System}\\[4mm]
{\LARGE MZ -- Memory Management}\\[4mm]
{\LARGE FZ -- Sequential Input/Output}\\[4mm]
{\LARGE RZ -- Random-access Input/Output}\\[4mm]
{\LARGE DZ -- Debugging Tools}\\[4mm]
{\LARGE DZDOC -- Bank documentation tools}\\[4mm]
{\LARGE TZ -- Title Handling}\\[4mm]
{\LARGE JZ91 -- Processor Support}\\[4mm]
{\LARGE Error Diagnostics}\\[25mm]
\end{center}
\vfill
\begin{center}\Large CERN Geneva, Switzerland\end{center}
\end{titlepage}

%%%%%%%%%%%%%%%%%%%%%%%%%%%%%%%%%%%%%%%%%%%%%%%%%%%%%%%%%%%%%%%%%%%%
%    Copyright  page                                               %
%%%%%%%%%%%%%%%%%%%%%%%%%%%%%%%%%%%%%%%%%%%%%%%%%%%%%%%%%%%%%%%%%%%%
\thispagestyle{empty}
\framebox[\textwidth][t]{\hfill\begin{minipage}{0.96\textwidth}%
\vspace*{3mm}
\begin{center}Copyright Notice\end{center}
\setlength{\parskip}{.6\baselineskip}

CERN Program Library entries \textbf{Q100} and \textbf{Q101}
 
\textbf{The ZEBRA system}

\copyright{} Copyright CERN, Geneva 1995
 
Copyright and any other appropriate legal protection of these
computer programs and associated documentation reserved in all
countries of the world.
 
These programs or documentation may not be reproduced by any
method without prior written consent of the Director-General
of CERN or his delegate.
 
Permission for the usage of any programs described herein is
granted apriori to those scientific institutes associated with
the CERN experimental program or with whom CERN has concluded
a scientific collaboration agreement.
 
Requests for information should be addressed to:
\vspace*{-.5\baselineskip}
\begin{center}\ttfamily
\begin{tabular}{l}
CERN Program Library Office              \\
CERN-CN Division                         \\
CH-1211 Geneva 23                        \\
Switzerland                              \\
Tel.      +41 22 767 4951                \\
Fax.      +41 22 767 8630                \\
Internet: cernlib@cern.ch
\end{tabular}
\end{center}
\vspace*{2mm}
\end{minipage}\hfill}%end of minipage in framebox
\vspace{6mm}
 
{\bf Trademark notice: All trademarks appearing in this guide are acknowledged as such.}
\vfill

\begin{tabular}{l@{\quad}l@{\quad}>{\small\tt}l}
{\em Contact Persons\/}: general        & Jamie Shiers /CN      & (shiers\atsign cern.ch)  \\
\phantom{\em Contact Persons\/}: FZ, MZ, TZ, JZ91 & Julius Zoll /ECP & (zoll\atsign cern.ch)  \\
\phantom{\em Contact Persons\/}: DZDOC  & Otto Schaile/PPE-Opal & (o.schaile\atsign cern.ch)\\[1mm]
\textem{Technical Realization\/}:       & Michel Goossens /CN   & (goossens\atsign cern.ch)\\[1cm]
\textem{Edition -- February 1995}
\end{tabular}
\newpage

%%%%%%%%%%%%%%%%%%%%%%%%%%%%%%%%%%%%%%%%%%%%%%%%%%%%%%%%%%%%%%%%%%%%
%    Introductory material                                         %
%%%%%%%%%%%%%%%%%%%%%%%%%%%%%%%%%%%%%%%%%%%%%%%%%%%%%%%%%%%%%%%%%%%%

\pagenumbering{roman}
\setcounter{page}{1}

\section*{Preliminary remarks}

This manual consists of several parts:

\begin{Itemize}
\item An overview of the ZEBRA system.
\item A reference section with a description of the DZ, MZ, FZ, 
      JZ91, RZ and TZ packages.
\item An example program showing how to use the MZ and DZ routines of ZEBRA.
\item A description of the DZDOC documentation system.
\item A list of the diagnostics messages generated by the FZ and MZ
      parts of the ZEBRA system.
\end{Itemize}

\subsection*{Conventions}

In this manual
examples are in \texttt{monotype face} and strings to be input by the user 
are {\Ucom{underlined}}.
In the index the page where a routine is defined is in {\bf bold},
page numbers where a routine is referenced are in normal type.

This manual flags output parameters in subroutine calls,
i.e. parameters which return values to the caller,
by an asterisk \Lit{"*"} following the argument's name.
If the input value of such a parameter is also significant
this is marked by prefixing a second asterisk.
A parameter which is a link is marked by an exclamation mark \Lit{"!"}.

The types of variables follow from the Fortran default typing
convention, except that variables beginning with the letters
"{\tt ch}" are of type {\tt CHARACTER}. 

The Fortran labelled \Lit{COMMON /\QUEST/IQUEST(100)} serves
for communication between the Zebra system and the user,
and also as scratch area in Zebra.

This document has been produced using \LaTeX~\cite{bib-LATEX}
with the \Lit{cernman} style option, developed at CERN. 
A gzipped compressed PostScript file \Lit{zebra.ps.gz}, 
containing a complete printable version
of this manual, can be obtained by anonymous ftp as follows
(commands to be typed by the user are underlined)%
\footnote{If you do not have the gnu {\tt gunzip} utility on your system
          you can get the uncompressed PostScript version by typing the
          command \Ucom{get zebra.ps}, without the {\tt gz} suffix. 
          In order to save Internet bandwidth, you are, however, strongly 
          urged to try and install the {\tt gunzip} utility since gzipped files 
          are about three times smaller than their unzipped equivalents.}:

\vspace*{3mm} 
\begin{XMP}
    \Ucom{ftp asisftp.cern.ch}
    Trying 128.141.201.136...
    Connected to asis01.cern.ch.
    220 asis01 FTP server (Version 6.10 ...) ready.
    Name (asis01:username): \Ucom{anonymous}
    Password: \Ucom{your\_{}mailaddress}
    230 Guest login ok, access restrictions apply.
    ftp> \Ucom{cd cernlib/doc/ps.dir}
    ftp> \Ucom{binary}
    ftp> \Ucom{get zebra.ps.gz}
    ftp> \Ucom{quit}
\end{XMP}
\vspace*{3mm} 

\newpage

%%%%%%%%%%%%%%%%%%%%%%%%%%%%%%%%%%%%%%%%%%%%%%%%%%%%%%%%%%%%%%%%%%%%
%    Tables of contents ...                                        %
%%%%%%%%%%%%%%%%%%%%%%%%%%%%%%%%%%%%%%%%%%%%%%%%%%%%%%%%%%%%%%%%%%%%
\tableofcontents
\listoffigures
\cleardoublepage
%\listoftables

% Local Variables: 
% mode: latex
% TeX-master: "zebramain"
% End: 
