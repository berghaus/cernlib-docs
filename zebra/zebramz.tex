%%%%%%%%%%%%%%%%%%%%%%%%%%%%%%%%%%%%%%%%%%%%%%%%%%%%%%%%%%%%%%%%%%%
%                                                                 %
%   ZEBRA User Guide -- LaTeX Source                              %
%                                                                 %
%   Chapter MZ (Memory Manager)                                   %
%                                                                 %
%   The following external EPS files are referenced:              %
%   fmzstor, fmzwork, fmzbok1, fmzbok2, fmzbok3, fmzbok4          %
%                                                                 %
%   Editor: Michel Goossens / AS-MI                               %
%   Last Mod.:  7 Dec. 1990   mg                                  %
%                                                                 %
%%%%%%%%%%%%%%%%%%%%%%%%%%%%%%%%%%%%%%%%%%%%%%%%%%%%%%%%%%%%%%%%%%%
\chapter{MZ: The memory manager}
\section{System initialization and end-of-run statistics}
\par Routine MZEBRA must be called to initialize the ZEBRA system.
In particular the ZEBRA standard input and output unit numbers are
preset to the values they should have on any given machine.
\par MZEBRA prints at log level -1 or above (see MZLOGL below)
an initialization message
showing, amongst other things, the version number of the
current ZEBRA system.
\par The parameter in the call to MZEBRA may select initialization
options either with an integer constant or with a list as detailed below.
\Subr{CALL MZEBRA (I)      or    CALL MZEBRA(LIST)}
\index{MZ!MZEBRA}
\index{initialization}
\Idesc
\begin{DL}{MM}
\item[I]Integer variable specifying the output options
\begin{DL}{MM}
\item[ 0]Standard defaults
\item[-1]Direct the ZEBRA system output to the terminal
\item[-2]Print only error messages
\item[-3]Combination of -1 and -2.
\end{DL}
\end{DL}
See the ZEBRA reference guide for a description of the meaning
of the parameters in the array LIST.
\subsection{Example}
\par The following call to initialise the MZEBRA system:
\begin{verbatim}
      CALL MZEBRA(-1)
\end{verbatim}
will print a message like the following on the terminal:
\begin{verbatim}
ZEBRA SYSTEM  3.6100      Date/Time 890801/1200
ZEMQ     3.61  890626 17.10
\end{verbatim}
MZEBRA only initializes the general ZEBRA system commons,
it does not initialize the dynamic store.
Before any request to the ZEBRA system involving the store,
the user must initialize it by calling MZSTOR (see below)
\Subr{CALL MZEND}
\index{MZ!MZEND}
\par At the end of execution the user should issue a call to routine
MZEND to obtain
the statistics of usage of the dynamic store.
\section{Initialization of a dynamic store}
\par
ZEBRA can handle up to 16 different dynamic stores. Each such store
must reside in a COMMON block.
Before any operation using a store
it must be initialized by calling MZSTOR.
All transactions with ZEBRA involving stores require the specification
of the
{\bf\it store index IXSTOR}
(or IXDIV which identifies both the store and the division).
The first store initialized is called the primary store and
its store index IXSTOR will be zero, i.e.
when one uses only one store then zero can be specified
as store index in all calls to the ZEBRA routines.
Further secondary stores, when initialized, will have
non-zero IXSTOR values.
\index{store!index}
\index{store!initialization}
\index{store!primary}
\index{store!secondary}
\par Each call to MZSTOR automatically creates two user short term
divisions (called division 1 and 2) and a system division (see MZDIV
for more details).
\Subr{CALL MZSTOR}
(IXSTOR*,CHNAM,CHOPT,FENCE,LQ(1),LQ(LR),LQ(LW),LQ(LIM2),LQ(LAST))
\par Initialises a given dynamic store.
\index{MZ!MZSTOR}
\index{store!definition}
\Idesc
\begin{DL}{MMMM}
\item[CHNAM]Character variable specifying the name of the store for
printing purposes (8 characters maximum).
\item[CHOPT]Character variable specifying the options selected
\begin{DL}{MM}
\item['Q']Quite, set the log level to -2 for this store
\end{DL}
\item[FENCE]Safety area immediately in front of the store to
protect against illegal use of LQ(0), LQ(-1), etc.
The fence extends from FENCE(1) to LQ(0).
\item[LQ(1)]First word of the dynamic store
\item[LQ(LR)]First permanent reference link, if any
\item[LQ(LW)]First word in the store after the permanent links,
this is the first word available to the working
space, or to division 1. The following words are
allocated as permanent user links, if any:\\
LQ(1)  to LQ(LR-1)  structural links, if any\\
LQ(LR) to LQ(LW-1)  reference links, if any.
\item[LQ(LIM2)]Lowest position of the upper limit of division 2,
to protect divisions 1 and 2 from being squeezed out
of existence by other divisions created later.
\item[LQ(LAST)]Last word of the dynamic store.
\end{DL}
\Odesc
\begin{DL}{MMMM}
\item[IXSTOR*]The index of the store,
to be used when specifying this store in subsequent calls
to the ZEBRA system. IXSTOR is set to zero for the primary store.
\end{DL}
\par The division indices for divisions 1 and 2 created by MZSTOR are:\\
For division 1:  IXDIV = IXSTOR + 1\\
For division 2:  IXDIV = IXSTOR + 2
\subsection{Notes:}
\par
\begin{UL}
\item The fence must have one word at least, but at most 1000 words.
\item The data region of the store
(i.e. the total store minus the permanent links)
must be at least 2000 words.
\item The sizes of the various parts of the store must be such
that there are always at least 164 words of reserve area
between divisions 1 and 2 available as working area for the system
[40 words for a secondary store].
\end{UL}
\subsection{Layout of memory after a call to MZSTOR}
\index{IQNIL}
\par
\begin{figure}[h]
\epsffile{fmzstor.ps}
\caption{Layout of memory after a call to MZSTOR}
\label{FMZSTOR}
\end{figure}
The fence region is preset to contain IQNIL
\footnote{IQNIL is a machine dependent bitstring which
is as incompatible as possible with any acceptable data type on the
given machine.}
in every word, this must never be changed;
the debug aids will check for over-writing.
The permanent links are cleared to zero; the rest of the store is not
touched.
\subsection{Examples}
\par For the primary store:
\begin{verbatim}
      PARAMETER   (NAQ = 120000)
 
      COMMON //    FENDA(16),              the fence area
     +             LMAIN, ...              the structural links
     +             L1, ...                 the reference  links
     +             DIV12(70000) , OTHER
 
      DIMENSION    LA(NAQ) , IA(NAQ) , A(NAQ)
      EQUIVALENCE (LMAIN   , LA(1))  , (A(1) , IA(1) , LA(9))
 
      CALL MZSTOR (0,'//',' ',FENDA,LA,L1,DIV12,OTHER,A(NAQ))
\end{verbatim}
\par For a secondary store without permanent links:
\begin{verbatim}
      COMMON /BDYN/ IXBST     , IXBDV1 , IXBDV2 , IXBHIT ,
     +              FENDB(16) , LB     , LASTB
 
      DIMENSION     LB(40000) , IB(40000) , B(40000)
      EQUIVALENCE  (B(1) , IB(1) , LB(9))
 
      CALL MZSTOR (IXBST,'/BDYN/','Q',FENDB,LB,LB,LB,LB(30000),LASTB)
      IXBDV1 = IXBST + 1
      IXBDV2 = IXBST + 2
\end{verbatim}
\section{Set the log level for a store}
\par Various parts of the ZEBRA system write log messages
to the standard system output,
and occasionally also to the on-line terminal, if any.
\par Examples of messages provided for are:
\begin{UL}
\item Messages for recoverable errors: read errors, data errors
\item Initialization messages for: stores, divisions, link areas, files
\item Termination messages giving statistics of usage of
various facilities like memory, files
\item Operation messages: change in program phase, end-of-file
\item Watch messages for hopefully rare, expensive events:
garbage collection, MZPUSH with relocation
\item Monitor messages to help the user debug his program
\end{UL}
\par To control the amount of information thus provided to the user,
a log level is defined and can be set and reset by the user
at execution time.
The default log level zero enables the messages which one would
usually like to see for record in a production run.
The user may reduce the log level to cut out most or all messages;
he may increase the level to watch the running of his program,
or even to debug his data or his input files.
\par Separate ZEBRA entities, such as dynamic stores or files,
each have their own attached log level,
which may be changed by the user at any time.
By default they inherit the global system-wide default log-level
set by MZEBRA, whose own default is zero.
\par The log level attached to a particular dynamic store is initialized
by MZSTOR, normally to the global default log level.
The user may change it at any time with:
\Subr{CALL MZLOGL (IXSTOR,LOGLEV)}
\index{MZ!MZLOGL}
\index{logging level}
\Idesc
\begin{DL}{MMMM}
\item[IXSTOR]Index of the store
for which the logging level has to be set
\item[LOGLEV]Logging level
\begin{DL}{MM}
\item[-3]Suppress all log messages
\item[-2]Error messages: \Rind{ZFATAL}, \Rind{ZPHASE}
\item[-1]Terse logging: \Rind{MZEBRA}, \Rind{MZSTOR}, \Rind{ZPHASE}
\item[ 0]Normal logging: \Rind{MZDIV}, \Rind{MZLINK}
\item[ 1]Log to watch: \Rind{MZLINT}, \Rind{MZGARB}, \Rind{MZPUSH}, 
\Rind{MZFORM}, \Rind{MZIOBK}, \Rind{MZIOCH}
\item[ 2]Log to monitor ZEBRA calls: \Rind{MZLINT},
\Rind{MZWORK}, \Rind{MZBOOK}, \Rind{MZLIFT}, \Rind{MZDROP},
\Rind{MZPUSH}, \Rind{MZGARB}, \Rind{MZLOGL}
\end{DL}
\end{DL}
\index{MZ!MZSTOR}
\section{Creation of a division}
\par A dynamic store can be physically subdivided into {\bf divisions}.
Up to 20 divisions are allowed, which permits 17 divisions
to be created by the user with \Rind{MZDIV}, beyond the 3 divisions
created by \Rind{MZSTOR}.
\index{division!creation}
\index{division!mode}
\begin{DL}{MMM}
\item[Mode of a division:]\ Depending on whether a division grows
at its higher or its lower end,
a division is said to be of mode ``forward' or ``reverse'.
By default a division is created with mode ``forward'.
\item[Kind of a division:]\ Depending on its usage,
we distinguish three kinds of divisions (apart from the
system division, which is used by the ZEBRA system itself):
\index{division!kind}
\begin{DL}{MMM}
\item[User short term]\ A division which will be wiped clean after every
``event'(default);
\item[User long term]\ A division for data associated to several events or
needed throughout the program execution up to the termination phase;
\item[Package]\ A division used by some service
packages, and whose contents are not normally of any concern to the user.
\end{DL}
\item[Division cross-reference matrix:]\ As a default, links in a user division
may point to any other user division in the same store,
except that a linear structure must be contained within one division.
More restricted inter-division cross-reference maps can be defined,
either by using the
'C' (contained) option of \Rind{MZDIV}, which indicates that the division
in question does not reference any other division, or, for more general
cases, by using routine \Rind{MZXREF}
as described on page~\pageref{SR_MZXREF}.
\index{MZ!MZXREF}
\end{DL}
\Subr{CALL MZDIV (IXSTOR,IXDIV*,CHNAM,NW,NWMAX,CHOPT)}
\index{MZ!MZDIV}
\index{division!initialization}
\Idesc
\begin{DL}{MMMM}
\item[IXSTOR]Store index (zero for the primary store)
\item[CHNAM]Character variable specifying the
name of the division for printing purposes
(8 characters maximum).
\item[NW]Number of words (minimum 100) to be allocated to the division
initially. Later the division
may grow, but not beyond NWMAX.
\item[NWMAX]Maximum size of the division (at least NW).
\item[CHOPT]Character string specifying the options desired.
\begin{DL}{MM}
\item['R']Reverse division (default is Forward)
\item['L']Long term division (default is user short term)
\item['P']Package division (default is user short term)
(P implies C, overrides L)
\item['C']Division is Contained (i.e. no links point outside)\\
Default: all user divisions point to all other user divisions.
\end{DL}
\end{DL}
\Odesc
\begin{DL}{MMMM}
\item[IXDIV*]Index of the created division. This index
must be used when specifying this division in subsequent calls
to ZEBRA.
\end{DL}
\subsection{Examples}
\par Create a user short term division called {\tt HITS} in
forward mode in the primary store.
\begin{verbatim}
      CALL MZDIV (0 , IXHITS , 'HITS' , 10000 , 20000 , ' ')
\end{verbatim}
\par Create a user long term division called {\tt CALIB} in
reverse mode in the secondary store with index {\tt IXCST}.
\begin{verbatim}
      CALL MZDIV (IXCST , IXCALI , 'CALIB' , 8000 , 8000 , 'RL')
\end{verbatim}
\section{The initialization of a link area}
\par A link area is a vector of links outside any dynamic store, with
all its links pointing to one particular store. A link area
consists of NS structural links followed by NR reference links.
Either NS or NR may be zero.
\index{link area!initialization}
\par We distinguish two kinds of link areas:
\begin{OL}
\item A {\bf permanent}
link area is initialized once at the steering level
and is valid for the whole program;
\index{link area!permanent}
\item A {\bf temporary}
link area is requested and deactivated
by lower-level code any number of times.
\index{link area!temporary}
\end{OL}
A permanent link area consists of just the vector of links;
a temporary link area has two words prefixed to the link-vector
for efficiency:
\begin{UL}
\item {\bf Word 1}
is a flag indicating whether the area is active or not;
if this word is zero, the contents of the area will
not be updated in case of a memory move.
The user may reset this word to zero to de-active the link area.
\item {\bf Word 2} is a key allowing the system to find easily the
whereabouts of this area in the system tables without searching.
This word must never be redefined by the user.
\end{UL}
A link area must be in COMMON storage;
if it is in local storage there is a danger that FORTRAN
optimization causes a register to preserve the old value of a link
across a relocation operation,
for garbage collection,
but also for simple updating with Rind{MZDROP}, \Rind{ZSHUNT}, etc.
\Subr{CALL MZLINK (IXSTOR,CHNAM,LAREA,LREF,LREFL)}
\par Declares a permanent link area. It should only be {\bf called
once} for any given area.
\index{link area!declaration}
\Idesc
\begin{DL}{MMMM}
\item[IXSTOR]Index of the store into which the links will point.
\item[CHNAM]Character variable specifying the
name of the FORTRAN COMMON in which the link area resides,
for printing purposes (8 characters maximum).
\item[LAREA]First word of the link area,
being also the first link of this area
\item[LREF]First reference link, if any;
last structural link, if no reference links
\item[LREFL]Last reference link, if any,
if none: give LAREA in this parameter position
\end{DL}
\par \Rind{MZLINK} will clear the links to zero.
\subsection{Examples [for the primary store]}
\subsubsection{Mixed link area}
\begin{verbatim}
      COMMON /LAMIX/ LS1,...,LSN, LR1,...,LRN
 
      CALL MZLINK (0,'/LAMIX/',LS1,LR1,LRN)
\end{verbatim}
\subsubsection{Structural link area}
\begin{verbatim}
      COMMON /LASTR/ LS1, ..., LSN
 
      CALL MZLINK (0,'/LASTR/',LS1,LSN,LS1)
\end{verbatim}
\subsubsection{Reference link area}
\begin{verbatim}
      COMMON /LAREF/ LR1, ..., LRN
 
      CALL MZLINK (0,'/LAREF/',LR1,LR1,LRN)
\end{verbatim}
\Subr{CALL MZLINT (IXSTOR,CHNAM,LAREA,LREF,LREFL)}\index{MZ!MZLINT}
\index{link area!declaration}
\par This call declares and initialises or re-initialises
a temporary link area.
\Idesc
\begin{DL}{MMMM}
\item[IXSTOR]Index of the store into which the links will point.
\item[CHNAM]Character variable specifying the
name of the FORTRAN COMMON in which the link area resides,
for printing purposes (8 characters maximum).
\item[LAREA]First word of the link area, with
\begin{DL}{MMMM}
\item[LAREA(1)]The user flag word
\item[LAREA(2)]A system word, which must not be redefined by the user
\item[LAREA(3)]The first link of the area
\end{DL}
\item[LREF]First reference link, if any;
last structural link, if no reference links
\item[LREFL]Last reference link, if any;
if none: give LAREA in this parameter position
\end{DL}
\par \Rind{MZLINT} will clear the links to zero,
set the flag-word {\tt LAREA(1)} to be non-zero,
and set the system-word {\tt LAREA(2)} on first use.
\par To deactivate a temporary link area the user sets {\tt LAREA(1)=0}.
From then on the links in this area are no longer relocated,
and hence will be meaningless.
\par To reactivate the area the user should call \Rind{MZLINT} again.
\subsection{Examples [primary store]}
\subsubsection{Mixed link area}
\begin{verbatim}
      COMMON /LAMIX/ LAMIX(2), LS1,...,LSN, LR1,...,LRN
 
      CALL MZLINT (0,'/LAMIX/',LAMIX,LR1,LRN)
\end{verbatim}
\subsubsection{Structural link area}
\begin{verbatim}
      COMMON /LASTR/ LASTR(2), LS1, ..., LSN
 
      CALL MZLINT (0,'/LASTR/',LASTR,LSN,LASTR)
\end{verbatim}
\subsubsection{Reference link area}
\begin{verbatim}
      COMMON /LAREF/ LAREF(2), LR1, ..., LRN
 
      CALL MZLINT (0,'/LAREF/',LAREF,LR1,LRN)
\end{verbatim}
\section{Working space allocation}
\par The region at the beginning of a dynamic store just after
the permanent links may be used as working space,
consisting of a set of reference links followed by
a set of data words.The use of this facility is as follows:
\Subr{CALL MZWORK (IXSTOR,DFIRST,DLAST,IFLAG)}
\index{working space!declaration}
\Idesc
\begin{DL}{MMMM}
\item[IXSTOR]Index of the store where the working space has to be provided
\item[DFIRST]First data word of the working space,
the preceding words are taken as reference links;
this parameter is ignored if IFLAG is 2 or -1
\item[DLAST]Last data word,
this parameter is ignored if IFLAG is -1
\item[IFLAG]Integer constant defining action desired
\begin{DL}{MM}
\item[ 0]Define a new working space.
The previous contents are not to be retained
\item[ 1]Vary the limits of the working space and keep
intact the links which are common to
both the old and the new working space.
\item[ 2]Vary only the DLAST limit of the working space and retain
all links and the data words
which are common to the old and the new working space.
\item[-1]Reset the working space to null,
i.e. zero links and zero data words
\end{DL}
\end{DL}
\par A call to MZWORK destroys division 1 of the store
{\bf without}
a relocation pass to reset links pointing into division 1.
This is for efficiency,
hence normally only very temporary data should be kept in division 1,
and only working space links should point to them.
To say it differently : division 1 is logically part of the
working space,
its time of existence is the same as that of the working space,
and it is good practice to maintain links into division 1
only in the working space.
If however it is necessary to hold such links elsewhere,
one should either reset them explicitly or wipe the division with
a call to \Rind{MZWIPE} before calling \Rind{MZWORK}.
\par The links of the working space (if {\tt IFLAG=1:} the new links only)
are initialized to  {\tt IQNIL+L},
where L is the location of the link in the store.
\index{IQNIL}
\subsection{Example}
\begin{verbatim}
   COMMON /STORE1/ IFENCE , LQPERM(100),    fence and permanent links
  +                LR1, ...                 working space reference links
  +                DFIRST, ...              working space data words
  +                DLAST
 
   CALL MZWORK (0 , DFIRST , DLAST , 0)
\end{verbatim}
\begin{figure}[h]
\epsffile{fmzwork.ps}
\caption{Layout of the store after a call to MZWORK}
\label{FMZWORK}
\end{figure}
\section{Bank creation}
\par A bank may be created by calling either \Rind{MZBOOK} or \Rind{MZLIFT}.
The functions of both routines are identical, and so is the
significance of the parameters. The routines differ only in the way the
input parameters are passed. They are both available since they
answer different needs.
\Subr{CALL MZBOOK (IXDIV,L*,*LSUP*,JBIAS,CHIDH,NL,NS,ND,NIO,NZERO)}
\index{bank!booking}
\par Routine
\Rind{MZBOOK} is provided for local creation of a bank, with all parameters
describing the bank present in the calling sequence itself.
\Idesc
\begin{DL}{MMMM}
\item[IXDIV]Index of the division into which the bank is
to be introduced, see below
\item[*LSUP*]If {\tt JBIAS < 0}:  address of the supporting up bank\\
If {\tt JBIAS = 0}:  address of the supporting previous bank\\
If {\tt JBIAS = 1}:  supporting link
\item[JBIAS]If {\tt JBIAS < 1}:  link bias in the supporting bank\\
If {\tt JBIAS = 1}:  create top-level bank
If {\tt JBIAS = 2}:  create stand-alone bank
\item[CHIDH]Character variable specifying the bank Hollerith identifier
\item[NL]Total number of links, {\tt NL < 64000}
\item[NS]Number of structural links, {\tt NS $leq$ NL}
(not counting the 3 links next, up, and origin)
\index{link!next}
\index{link!up}
\index{link!origin}
\item[ND]Number of data words
\item[NIO]IO characteristic,
describing the type of the data words given as:\\
- An I/O characteristic itself, as returned by MZIOCH\\
- The index of the I/O characteristic, as returned by MZFORM\\
- An immediate value, if the whole bank is of the same type:
\begin{DL}{MM}
\item[ 0]Undefined data type (the bank cannot be transported between
computers)
\item[\ 1]The whole bank is of type {\bf bit string}
\item[\ 2]The whole bank is of type {\bf integer}
\item[\ 3]The whole bank is of type {\bf floating point}
\item[\ 4]The whole bank is of type {\bf double precision}
\item[\ 5]The whole bank is of type {\bf Hollerith}
\item[\ 7]The whole bank is {\bf self describing}
\item[\ 9]Retrieve the I/O characteristic by scanning the system
bank created by \Rind{MZFORM} using the IDH identifier.
\item[11]Take the I/O characteristic of any bank in the target
linear structure, if this is empty act as for 9.
\end{DL}
\item[NZERO]Controls whether and how much of the data-part
of the bank is preset to zero:
\begin{DL}{MM}
\item[-1]No presetting
\item[ 0]The whole bank is cleared
\item[>0]The first N words are preset
\end{DL}
The links of the bank are always cleared to zero.
\end{DL}
\Odesc
\begin{DL}{MMMM}
\item[L*]Address of the created bank.
\item[*LSUP*]Address of the supporting bank.
\end{DL}
\Subr{CALL MZLIFT (IXDIV,L*,*LSUP*,JBIAS,MMBK,NZERO)}
\index{MZ!MZLIFT}
\index{bank!lifting}
\par Routine
MZLIFT is provided in the context of a centralized organization
of bank descriptors, held normally in a COMMON block.
\Idesc
\begin{DL}{MMMM}
\item[IXDIV]Index of the division into which the bank is
to be introduced, see below
\item[*LSUP*]If {\tt JBIAS < 0}:  address of the supporting up bank\\
If {\tt JBIAS = 0}:  address of the supporting previous bank\\
If {\tt JBIAS = 1}:  supporting link
\item[JBIAS]If {\tt JBIAS < 1}:  link bias in the supporting bank\\
If {\tt JBIAS = 1}:  create top-level bank
If {\tt JBIAS = 2}:  create stand-alone bank
\item[MMBK]The bank descriptor vector:
\begin{DL}{MMMMMM}
\item[MMBK(1) = IDH]Hollerith string
\footnote{Hollerith
constants are not explicitly part of the FORTRAN standard and
hence their use in DATA or assignment statements can yield problems
with some compilers. In that case routine \Rind{UCTOH}
(available as CERN program library routine {\tt M409}) can be used
to translate between character variables and Hollerith strings.}
of the form {\tt 4Hxxxx} specifying the bank name
\item[MMBK(2) = NL]Total number of links, {\tt NL < 64000}
\item[MMBK(3) = NS]Number of structural links, {\tt NS $\leq$ NL}
(not counting the 3 links next, up, and origin)
\index{link!next}
\index{link!up}
\index{link!origin}
\item[MMBK(4) = ND]Number of data words
\item[MMBK(5) = NIO]IO characteristic,
describing the type of the data words given as: \\
- An I/O characteristic itself, as returned by \Rind{MZIOCH} \\
- The index of the I/O characteristic, as returned by \Rind{MZFORM}\\
- An immediate value, if the whole bank is of the same type:
\begin{DL}{MM}
\item[\ 0]Undefined data type (the bank cannot be transported between
computers)
\item[\ 1]The whole bank is of type {\bf\it bit string}
\item[\ 2]The whole bank is of type {\bf\it integer}
\item[\ 3]The whole bank is of type {\bf\it floating point}
\item[\ 4]The whole bank is of type {\bf\it double precision}
\item[\ 5]The whole bank is of type {\bf\it Hollerith}
\item[\ 7]The whole bank is {\bf\it self describing}
\item[\ 9]Retrieve the I/O characteristic by scanning the system
bank created by \Rind{MZFORM} using the {\tt IDH} identifier.
\item[11]Take the I/O characteristic of any bank in the
target linear structure, if this is empty act as for 9.
\end{DL}
\end{DL}
\item[NZERO]Controls whether and how much of the data-part
of the bank is preset to zero:
\begin{DL}{MM}
\item[-1]No presetting
\item[ 0]The whole bank is cleared
\item[>0]The first N words are preset
\end{DL}
The links of the bank are always cleared to zero.
\end{DL}
\Odesc
\begin{DL}{MMMM}
\item[L*]Address of the created bank.
\item[*LSUP*]Address of the supporting bank.
\end{DL}
\subsection{Settings of bank parameters after bank creation}
\subsubsection{Division}
\par The division in which the bank has to be created can be specified by
the first calling sequence parameter IXDIV, which then corresponds
to the index of a specific division.
The user is, however, not obliged to do so, as he can ask
ZEBRA to create the bank in the same division as the other banks of the
surrounding data structure. This is done by giving as the first parameter
the index of the store only, which in the case of the primary store is
simply zero. Then, when no specific division is specified, the
rules for selecting the division in which the bank will be created
are as follows:
\begin{UL}
\item If the bank is to be part of a non-empty linear
structure, then it will be created
in the same division where the other banks reside.
\item
If this linear structure is empty, the bank will be created in the
same division as its supporting bank.
\item
If there is no supporting bank (JBIAS=1 or 2) division 2 of the given
store will be selected.
\end{UL}
\par {\bf Note:}
All banks in a linear structure must reside in the same division, i.e.
link zero of any bank is not allowed to point
outside the division of that bank.
\subsubsection{Input-output characteristic}
\par The I/O characteristic describes the nature of the data of a bank to
ZEBRA. This information is needed for sending the data across
different computers, and also for interpretation dumps. The I/O
characteristic can be specified directly with \Rind{MZBOOK}/\Rind{MZLIFT},
or it can be created with \Rind{MZFORM}. A detailed discussion
is given on page~\pageref{SR_MZFORM}.
\subsubsection{The numeric and Hollerith bank identifiers}
\par Each bank has a numeric and a Hollerith
identifier IDN and IDH, which are stored respectively
at IQ(L-5) and IQ(L-4). The IDH is specified explicitly
in the calling sequences to \Rind{MZBOOK} (CHIDH) or \Rind{MZLIFT} (first
element of the {\tt MMBK} vector) while the bank numberic identifier
is set at bank creation as follows:
\begin{UL}
\item If the bank after creation has a 'next' bank, its IDN is taken
and incremented by one.
\item If the bank is created at the end of a linear structure, the IDN
of its last bank is taken and incremented by one.
\item If no such bank exits, the link number
of the supporting link is taken.
\item If there is no supporting bank IDN is set to one.
\end{UL}
In general banks in a linear structure will have the same IDH
and different IDN. The user can change IDN,
if he so wishes, by simply putting a new value into {\tt IQ(L-5)}.
\subsection{Examples of MZBOOK}
\subsubsection{Top level bank}
\begin{verbatim}
      CALL MZFORM('MAST',CHFORM,IOMAST)       -- at initialization
 
      CALL MZBOOK(0,L,LMAST,1,'MAST',5,3,10,IOMAST,0)
\end{verbatim}
\par Create in division 2 of the primary store
a bank supported by the link {\tt LMAST}, which must be part of a
link area. The bank identifier is {\tt 'MAST'} and the bank
consists of 5 links, of
which 3 are structural and 10 data words. The format of
the latter is described by the I/O index {\tt IOMAST} (as calculated
by \Rind{MZFORM}) and all data words are to be preset to zero.
The value of IDN will be set equal to 1.
\par Supposing {\tt LMAST}
to be zero initially, this will give the situation in
Figure~\ref{MZBOOK1}.
\begin{figure}
\epsffile{fmzbok1.ps}
\caption{The creation of a top level bank}
\label{MZBOOK1}
\end{figure}
\subsubsection{First dependent bank}
\begin{verbatim}
      CALL MZBOOK(0,L,LMAST,-3,'DOWN',0,0,16,3,0)
\end{verbatim}
\par Create in the primary store and in the same division as 'MAST'
a bank called 'DOWN', which should be attached as the third
structural link to {\tt LMAST} (created in 1 above). It has zero links
and 16 data words all of type floating.
All data words are to be preset to zero.
The bank at {\tt LDOWN} is said to be a ``down' bank of MAST; 
its address could
later be reobtained as {\tt LDOWN = LQ(LMAST-3)}.
The IDN will be set equal to 3. Supposing link 3 of the bank at
LMAST to be zero initially, this will give the picture of
Figure~\ref{MZBOOK2}.
\index{link!down}
\begin{figure}
\epsffile{fmzbok2.ps}
\caption{The creation of a first dependent bank}
\label{MZBOOK2}
\end{figure}
\subsubsection{Further dependent banks}
\begin{verbatim}
      CALL MZBOOK(0,L,LMAST,-3,'DOWN',0,0,16,3,0)
\end{verbatim}
creates a new bank in the primary store and in the same division
as 'DOWN' and inserts it in the linear in front of the 'DOWN' bank
of example 2.
Its address could be obtained as {\tt LDOWN4 = LQ(LMAST-3)},
while the address of the
bank 'DOWN' created in example 2 could be obtained as {\tt LDOWN = LQ(LDOWN4)}.
The IDN will be set equal to 4 (see Figure~\ref{MZBOOK3}).
\begin{figure}
\epsffile{fmzbok3.ps}
\label{MZBOOK3}
\caption{The creation of further dependent banks}
\end{figure}
\subsubsection{Add to end of linear structure}
\begin{verbatim}
C--     Calculate address of last bank in linear chain
      L = LZLAST(0,LMAST-3)
 
      CALL MZBOOK(0,L,L,0,'DOWN',0,0,16,3,0)
\end{verbatim}
creates a next bank in the primary store and in the same division
as the other 'DOWN' banks.
This bank has the same characteristics as bank
'DOWN' created in 2 above. It is inserted in the linear chain
behind bank 'DOWN' of example 2.
The IDN will be set equal to 4 (see Figure~\ref{MZBOOK4}).
\begin{figure}
\epsffile{fmzbok4.ps}
\caption{The creation of banks at the end of a linear structure}
\label{MZBOOK4}
\end{figure}
\subsubsection{Stand alone bank}
\begin{verbatim}
      CALL MZBOOK(0,LSEUL,0,2,'SEUL',0,0,30,2,-1)
\end{verbatim}
books a bank in division 2 of the primary store,
which is addressable only via the address LSEUL as returned by
routine \Rind{MZBOOK}. It is not inserted in any data structure.
No presetting of the 30 integer data words is requested.
This can be useful for getting very short term working banks or for
building complex data structures starting from individual unrelated
banks, which are afterwards inserted into an
existing data structure using routine \Rind{ZSHUNT}.
The bank number will be set equal to zero.
\index{MZ!ZSHUNT}
\subsection{Examples of MZLIFT}
\par The same examples described above with \Rind{MZLIFT} would look like this:
\subsubsection{Top level bank}
\begin{verbatim}
C--     During initialization
      COMMON /BANK/  MMMAST(7),MMDOWN(5),MMSEUL(5)
 
      DATA  IMAST /4HMAST/
      DATA MMDOWN /4HDOWN,0,0,16,3/
      DATA MMSEUL /4HSEUL , 0 , 0 , 30 , 2/
 
      MMMAST(1) = IMAST
      MMMAST(2) = 5
      MMMAST(3) = 3
      MMMAST(4) = 10
      CALL MZIOBK (MMMAST,7,'format')
 
C--     During execution
      COMMON /LINKS/ L , LDOWN , LMAST , LSEUL
      COMMON /BANK/  MMMAST(7)
      CALL MZLIFT (0,L,LMAST,1,MMMAST,0)
\end{verbatim}
\subsubsection{First dependent (down) bank}
\par
\begin{verbatim}
      CALL MZLIFT (0,L,LMAST,-3,MMDOWN,0)
\end{verbatim}
\subsubsection{Further dependent banks}
\par
\begin{verbatim}
      CALL MZLIFT (0,L,LMAST,-3,MMDOWN,0)
\end{verbatim}
\subsubsection{Insert bank in linear structure}
\par
\begin{verbatim}
C--
C--     Calculate address of last bank in linear chain
      L = LZLAST(0,LMAST-3)
 
      CALL MZLIFT (0,L,L,0,MMDOWN,0)
\end{verbatim}
\subsubsection{Stand alone bank}
\begin{verbatim}
      CALL MZLIFT (0,LSEUL,0,2,MMSEUL,-1)
\end{verbatim}
\section{Alter the size of a bank}
\index{division!forward}
\index{division!reverse}
\par The size of a bank can be increased or decreased by using \Rind{MZPUSH}.
Decreasing the size of a bank is done in situ. Increasing the data
part of the last bank in a forward division, as well as increasing
the link part of the first bank in a reverse division, is also done
in situ.
In all other cases a new bank with increased dimensions is lifted
to receive a copy of the data from the original bank, which is then
dropped. The new link and data words acquired are preset to zero.
In general the operation of \Rind{MZPUSH} requires a relocation pass over all
the links to update any link pointing to the bank for its new position.
If the user knows that no such link exists, except for the immediate
system links around this bank, he can signal this to \Rind{MZPUSH} with the
'I' option, thus saving the relocation pass.
\Subr{CALL MZPUSH (IXSTOR,*L*,INCNL,INCND,CHOPT)}
\index{relocation}
\index{bank!size increase}
\index{bank!size decrease}
\index{bank!size change}
\Idesc
\begin{DL}{MMMM}
\item[IXSTOR]Index of the store with the bank whose size has to be changed
\item[*L*]Address of the bank whose size has to be changed
\item[INCNL]Number of additional links;
zero for no change, negative for decrease.
Additional links will be given type ``reference',
unless the original bank has only structural links.
\item[INCND]Number of additional data words;
zero for no change, negative for decrease.
\item[CHOPT]Character variable specifying the relocation strategy
\begin{DL}{MM}
\item[' ']Default - Any link may point to the bank
\item['R']No link points into the abandoned bank region
(in case of bank reduction)
\item['I']Isolated case (especially efficient).
No link other than the supporting
structural link, the link passed in the parameter L,
and the reverse links in the first level dependents,
point to this bank.
In this case the immediate links will be updated explicitly
and the relocation pass will be saved.
\end{DL}
\end{DL}
\Odesc
\begin{DL}{MMMM}
\item[*L*]Relocated address of the bank whose size has been changed
\end{DL}
\par The new link and data words acquired by the bank
are cleared to zero.
\section{Dropping complete divisions}
\Subr{CALL MZWIPE (IXWIPE)}
\index{division!wipe}
\par A call tp MZWIPE deletes the
contents of complete divisions.
\Idesc
\begin{DL}{MMMM}
\item[IXWIPE]Index of the divisions in a particular store which have
to be wiped out.
A value of zero means wipe all short term divisions in the primary
store.
\end{DL}
\par \Rind{IXWIPE} may be any of the three possible forms of a division index:
\begin{OL}
\item A specific division index, as returned by MZDIV
\item A generic division index, [IXSTOR+] n with
\begin{DL}{MMMM}
\item[n = 21]all user short term divisions
\item[n = 22]all user long  term divisions
\item[n = 23]all package divisions
\end{DL}
\item A compound division index, as created by \Rind{MZIXCO},
see page~\pageref{SR_MZIXCO} for details.
\end{OL}
Wiping divisions resets the divisions to be empty
(but without reducing the space reserved for them),
followed by a relocation pass to reset to zero all links
pointing into the wiped divisions.
Included in this pass are the links of all link areas,
of the working space and of all divisions which are declared
to point to the divisions in question
(all of this for one particular store only, of course).
\par Wiping several divisions should be done by a single call to \Rind{MZWIPE}
rather than by several calls in succession,
to save the time of multiple relocation passes,
each of which would take longer than the single pass.
\subsection{Examples}
\subsubsection{Division 1}
\begin{verbatim}
      CALL MZWIPE (1)
\end{verbatim}
\subsubsection{Last event}
\begin{verbatim}
      CALL MZWIPE (21)          or          CALL MZWIPE (0)
\end{verbatim}
\subsubsection{All user's divisions}
\begin{verbatim}
      IX = MZIXCO (21,22,0,0)
      CALL MZWIPE (IX)
\end{verbatim}
\subsubsection{Divisions IX1, IX2, IX3, IX4, and IX5}
\begin{verbatim}
      IX = MZIXCO (IX1,IX2,IX3,IX4)
      IX = MZIXCO (IX,IX5,0,0)
      CALL MZWIPE (IX)
\end{verbatim}
\section{User requested garbage collection}
\par Garbage collection is triggered by the system if there is not
enough space to satisfy an \Rind{MZWORK}, \Rind{MZBOOK}/\Rind{MZLIFT} or 
\Rind{MZPUSH} request (although this should not normally occur).
Thus the user does not have to worry about initiating
garbage collection to gain space in the dynamic store.
To remove the last event from the store,
the user calls \Rind{MZWIPE} which is much more efficient
than dropping the banks of the event followed by garbage collection.
\par He may, however, occasionally want to force a garbage collection
in order to tidy up his data,
particularly during the initialization phase of his program.
Again, as in \Rind{MZWIPE}, if there are several divisions to be tidied up,
this should be done by {\bf one} call to \Rind{MZGARB}. Also,
if one or several divisions are to be wiped out at the same time,
this should be included into the call to \Rind{MZGARB};
a single relocation pass can handle wiping
and garbage collection simultaneously.
\Subr{CALL MZGARB (IXGARB,IXWIPE)}
\index{division!garbage collection}
\Idesc
\begin{DL}{MMMM}
\item[IXGARB]Index of the divisions in a given store where a garbage collection
is requested (none if 0).
\item[IXWIPE]Index of the divisions in a particular store
to be wiped (none if 0).
\end{DL}
\par
Both IXGARB or IXWIPE may be any of the three possible forms
of a division index as described for MZWIPE on page~\pageref{SR_MZWIPE}.
\section{Compound division index creation}
\par A compound division index permits an indication of several divisions
of the same store in a single word,
as used with \Rind{MZWIPE} for example.
\Func{IXC = MZIXCO (IX1,IX2,IX3,IX4)}
\index{MZ!MZIXCO}
\index{division!compound index}
\par Function \Rind{MZIXCO} joins up to four division indices into a compound:
\Idesc
\begin{DL}{MMM}
\item[IXi]Index of the divisions in the same store which are to be
combined.
If less than 4 indices are to be joined trailing zeros
should be given.
If more than 4 indices are to be joined this is done by
repeated calls.
\end{DL}
\Odesc
\begin{DL}{MMM}
\item[IXC*]The combined division index (function value).
\end{DL}
\par IXi can be one of the following forms of a division index:
\begin{OL}
\item A specific division index, as returned by \Rind{MZDIV}
\item A generic division index, [IXSTOR+] n with
\begin{DL}{MMMM}
\item[n = 21]all user short term divisions
\item[n = 22]all user long  term divisions
\item[n = 23]all package divisions
\end{DL}
\item A compound division index, as created by previous call to \Rind{MZIXCO}.
\end{OL}
\subsection{Examples}
\subsubsection{Divisions 1 and 2:}
\begin{verbatim}
      IXCO = MZIXCO (1,2,0,0)
\end{verbatim}
\subsubsection{Divisions 1, 2, and \tt IXHITS:}
\begin{verbatim}
      IXCO = MZIXCO (1,2,IXHITS,0)
\end{verbatim}
\subsubsection{All short term divisions and \tt IXCUMU:}
\begin{verbatim}
      IXCO = MZIXCO (21,IXCUMU,0,0)
\end{verbatim}
\subsubsection{All user divisions of store \tt IXSTOR}
\begin{verbatim}
      IXCO = MZIXCO (IXSTOR+21,IXSTOR+22,0,0)
\end{verbatim}
\section{Set the cross-reference matrix between divisions}
\par To save time when wiping a given division (or divisions),
and also on garbage collection,
ZEBRA will relocate the links of only those divisions
which reference the division(s) being changed.
To know which division may have links pointing
to which other division,
ZEBRA keeps internally a cross-reference matrix;
the entry for a given division is intialized by \Rind{MZDIV}
and this may be modified by the user calling \Rind{MZXREF}.
\Subr{CALL MZXREF (IXFROM,IXTO,CHOPT)}
\index{division!cross-references}
\Idesc
\begin{DL}{MMMM}
\item[IXFROM]Index of the division which contains links
pointing to the divisions indicated by {\tt IXTO};
this must be the index of one particular division.
\item[IXTO]Index of the division(s) which are referenced
\item[CHOPT]Character variable specifying the options
\begin{DL}{MM}
\item[' ']Normal - Set reference(s), i.e. overwrite the previous
content of the matrix entry
\item['A']Add reference(s), i.e. add to the matrix entry,
keeping what was there before.
\item['C']Contained division, i.e. clear the matrix entry
(C over-rules A and R)
\item['R']Remove reference(s), i.e. take the references
out from the matrix entry, but keep the others (R over-rules A)
\end{DL}
\end{DL}
\par
{\tt IXTO} can be one of the following forms of a division index:
\begin{OL}
\item A specific division index, as returned by \Rind{MZDIV}
\item A generic division index, {\tt[IXSTOR+] n} with
\begin{DL}{MMMM}
\item[n = 21]all user short term divisions
\item[n = 22]all user long  term divisions
\item[n = 23]all package divisions
\item[n = 24]the system division
\end{DL}
\item A compound division index, as created by previous call to \Rind{MZIXCO}.
\end{OL}
When a division is created, the matrix row for this division
is initialized by \Rind{MZDIV}
as follows (unless the C option is given to \Rind{MZDIV}):
\begin{DL}{MMMMMMMM}
\item[User division]References all other user divisions.
\item[Package division]No references at all.
\end{DL}
\subsection{Examples}
\par
\begin{OL}
\item Suppose the user's division {\tt IXTHIS} may reference
all other user divisions:
\par
\phantom{\tt\      }nothing needs to be done, this is the default assumption 
\item Suppose division {\tt IXTHIS} references
only banks in division 2:
\begin{verbatim}
      CALL MZXREF (IXTHIS, 2, '.')
\end{verbatim}
\item Suppose division {\tt IXTHIS} references only,
but maybe all, the short term divisions:
\begin{verbatim}
      CALL MZXREF (IXTHIS, 21, '.')
\end{verbatim}
\item Suppose division {\tt IXTHIS} references all short term
divisions and also the long term division {\tt IXLONG}:
\begin{verbatim}
      CALL MZXREF (IXTHIS, 21, '.')
      CALL MZXREF (IXTHIS, IXLONG, 'A')
\end{verbatim}
\item Suppose division {\tt IXTHIS} references all short term
divisions except division {\tt IXSH}:
\begin{verbatim}
      CALL MZXREF (IXTHIS, 21, '.')
      CALL MZXREF (IXTHIS, IXSH, 'R')
\end{verbatim}
\end{OL}
\section{Handle the bank I/O characteristic}
\subsection{Bank format}
\par The nature of the contents of any bank which is to be transported
from one computer to another one has to be indicated to ZEBRA,
so that it can carry out the necessary transformations.
The bank format is also used for printing purposes by the DZ package.
In the simplest case that all the data words of a bank are
of the same type, this can be indicated by specifying a value between 0
and 5 for the parameter NIO of \Rind{MZBOOK}/\Rind{MZLIFT}.
For anything more complicated the user specifies the ``format''
of the bank by calling \Rind{MZIOBK} or \Rind{MZFORM} which encode the format
into a variable number of words to be included into each bank
in the system part as the ``I/O characteristic''.
Thus the content description is carried by each bank,
and complicated bank formats require a relative large number
of extra system words.
This could represent a substantial overhead on memory
or file space occupation,
which the user can avoid in the design of his bank formats.
Thus ZEBRA cannot handle arbitrary bank formats,
but by using the concept of the ``self-describing'' sector (see below)
the user can indeed store any combination in a bank
and transport it from one computer to another one.
\index{bank!I/O characteristic}
\index{bank!sector}
\par The basic element for setting up a bank descriptor is the
{\bf\it sector},
which is a number of consecutive words in the bank which are
all of the same type.
A sector is described in the ``format'' parameter to \Rind{MZFORM} et al.
as a combination of its word-count
{\tt 'c'} and its data type {\tt 't': 'ct'}.
For example
{\tt '24F'}
is a sector of 24 single precision floating-point
numbers, and {\tt '1I'}
is a sector of one integer word.
\index{bank!data type identifier}
\par The possible types for the data type identifier 't'
are listed in Table~\ref{MZFORMT}.
\begin{table}[h]
\caption{MZFORM data type identifier syntax}
\label{MZFORMT}
\begin{center}\begin{tabular}{|l|l|}
\hline
\rule{0mm}{5mm}B&bit string of 32 bits, right justified in a machine word\\
I&integer\\
F&floating-point\\
D&double precision\\
H&4-character Hollerith\\
\rule{-2mm}{6mm}S&self-describing sector (see below)\\ \hline
\end{tabular}\end{center}\end{table}
\index{bank!sector!static}
\par A {\bf static} sector is a sector of a fixed number of words,
such as the {\tt '24F'} in the example above.
\index{bank!sector!static}
\index{bank!sector!indefinite-length}
\par An {\bf indefinite-length}
sector is a sector whose end is defined
by the end of the bank.
This is written as {\tt -t}, for example
{\tt '-F'} signals that the rest
of the bank is all floating-point words.
\index{bank!sector!dynamic}
\par A {\bf dynamic} sector is
a sector which is preceded in the bank
by a single positive integer word indicating the sector length;
if this number is zero this means that the rest of
the bank is currently unused.
This is written as  {\tt *t}, for example
{\tt '*F'} indicates a dynamic sector of type floating.
\index{bank!sector!self-describing}
\par A {\bf self-describing} sector is a dynamic sector whose type
is also encoded into the word preceding the sector as follows:
{\tt 16*NW + IT}, where {\tt NW} is the length of the
sector and {\tt IT}, is a numeric descriptor of the type of
data, as defined in Table~\ref{MZFORMS}
(cf. \Rind{MZBOOK} and \Rind{MZLIFT}).
\index{MZ!MZBOOK}
\index{MZ!MZLIFT}
\begin{table}[h]
\caption{Numeric representation of data type in self-describing sector header word}
\label{MZFORMS}
\begin{center}\begin{tabular}{|c|l|c|l|}
\hline
{\bf 1}&bit string (as 'B')           &{\bf 2}&integer (as 'I')\\
{\bf 3}&floating-point (as 'F')       &{\bf 4}&double precision (as 'D')\\
{\bf 5}&4-character Hollerith (as 'H')&{\bf 6}&reserve\\
{\bf 7}&special                       &&\\ \hline
\end{tabular}\end{center}\end{table}
The word-count {\tt 'c'} for a sector,
which must {\bf always} be specified, can take the forms
given in Table~\ref{MZFORMC}.
It has to be stressed that this parameter {\tt c} is in units of
``ZEBRA words'', hence 4 double precision data items,
occupying 8 machine words, have to be specified as {\tt '8D'},
and a string of 17 Hollerith characters as {\tt '5H'}.
\index{bank!word-count}
\begin{table}[h]
\caption{MZFORM word count syntax}
\label{MZFORMC}
\begin{center}\begin{tabular}{|l|l|}\hline
\tt n&numeric, n words: static length sector\\
\tt -&all remaining words: indefinite length sector\\
\tt *&dynamic length sector (see text)\\
\tt /&signals the start of the trailing part of a bank format (see text)
\\ \hline
\end{tabular}\end{center}
\end{table}
Note that the form {\tt 'nS'} is meaningless;
that the form {\tt '*S'}
indicates one particular sector while
{\tt '-S'}is {\bf special} and
indicates that the rest of the bank is filled
with self-describing sectors,
as many as there may be.
\par Looking at the bank as a whole,
we divide it into a ``leading'' part and a ``trailing'' part,
either of which may be empty.
\index{bank!leading part}
\par The {\bf leading}
part consists of one region of maybe several sectors,
occurring once at the beginning of the bank.
This leading region may end with an indefinite-length sector,
in which case the trailing part is empty.
\index{bank!trailing part}
\par The {\bf trailing} part of the bank may be empty or it may consist
of an indefinite number of ``regions'' which
{\bf all have the same structure},
such that the same format description is valid for all of them.
\index{bank!region}
\par
The symbol {\tt '/'} marks the break between the leading region
and the trailing regions in the format parameter to \Rind{MZFORM} et al.
\subsection{Specifying the bank format}
\par Three routines are provided as an interface for the user to
specify the bank format in a readable form.
The I/O characteristic is included in any bank at creation time.
\Subr{CALL MZIOCH (IOWDS*,NWIO,CHFORM)}
\par Routine \Rind{MZIOCH} analyses the format {\tt CHFORM} to convert
and pack it into the output vector {\tt IOWDS}.
This is the basic routine,
but it is usually called by the user only to specify formats
of objects other than banks
(e.g. the user header vector in the FZ routines).
\Idesc
\begin{DL}{MMMM}
\item[NWIO]The maximum size of IOWDS
\item[CHFORM]Character variable specifying the format
\end{DL}
\Odesc
\begin{DL}{MMMM}
\item[IOWDS*]A vector dimensioned to NWIO which is to receive the I/O characteristic
\end{DL}
\Subr{CALL MZIOBK (MMBK*,NWMM,CHFORM)}
\par Routine \Rind{MZIOBK} is provided in the context of \Rind{MZLIFT};
like\Rind{MZIOCH} it analyses the format {\tt CHFORM},
but it stores the result as part of the bank-description vector
{\tt MMBK} for \Rind{MZLIFT}.
\Idesc
\begin{DL}{MMMM}
\item[NWMM]The maximum size of {\tt MMBKS}
\item[CHFORM]Character variable specifying the format
\end{DL}
\Odesc
\begin{DL}{MMMM}
\item[MMBK*]The bank description vector for \Rind{MZLIFT}, 
dimensioned to {\tt NWMM}.
The resulting I/O characteristic will be stored
in the vector starting at {\tt MMBK(5)}.
The IDH contained in {\tt MMBK(1)} will be used if diagnostics are necessary.
\end{DL}
\Subr{CALL MZFORM (CHIDH,CHFORM,IXIO*)}
\index{MZ!MZFORM}
\par Routine \Rind{MZFORM} analyses the format {\tt CHFORM},
but it does not return the result to the user.
Instead, it stores the I/O characteristic in a system data structure,
returning to the user only the index to the characteristic
in the system.
The user may then either pass this index to \Rind{MZBOOK} (or \Rind{MZLIFT})
at bank creation time,
or alternatively he may request \Rind{MZBOOK} (or \Rind{MZLIFT}) to search
the system data structure for the I/O characteristic
associated with the Hollerith identifier {\tt IDH} of the bank to be
created.
\index{bank!I/O characteristic}
\Idesc
\begin{DL}{MMMM}
\item[CHIDH]Character variable specifying the Hollerith identifier of the
bank whose data part is described by the given format
\item[CHFORM]Character variable specifying the format of the data
part of a bank with name identifier CHNAM.
\end{DL}
\Odesc
\begin{DL}{MMMM}
\item[IXIO*]The encoded index of the I/O characteristic stored in a system
data structure.
This index can be passed to MZBOOK/MZLIFT, in which case it must
not be modified
\end{DL}
\subsubsection{CHFORM examples}
{\bf Banks with an empty trailing part:}
\begin{DL}{MMMMMMM}
\item['-F']Whole bank is floating
\item['3I -F']First 3 words are integer, the rest is F
\item['*I -F']First word n=IQ(L+1) is a positive integer,
words 2 to n+1 are integers, the rest is F
\item['3B *I -F']First sector consists of 3 words bit-string,
the second sector is dynamic of type integer,
the rest of the bank is floating
\end{DL}
{\bf Banks with a leading as well as a trailing part:}
\begin{DL}{MMMMMMM}
\item['3B 7I / 2I 4F 16D']Leading region has 3 B and 7 I words,
each trailing region consists of
2 integer words, followed by 4 F words,
followed by 16 D words, i.e. 8 double-precision numbers
\item['4I / *H']Bank starts with 4 integer words,
the rest is filled with dynamic Hollerith sectors
\item['*I / 2I *F']Leading region is one dynamic I sector,
each trailing region consists of 2 integers
followed by a dynamic F sector
(i.e. 3 integers plus a number of floating-point words,
this number being indicated by the third integer)
\end{DL}
{\bf Banks with an empty leading part:}
\begin{DL}{MMMMMMM}
\item['/ *H']Bank is filled with dynamic Hollerith sectors
\item['/ 4I 29F']4 integers and 29 floating-point numbers alternate
\item['/ *S' or '-S']Data of the whole bank are self-describing
\end{DL}
 
