%%%%%%%%%%%%%%%%%%%%%%%%%%%%%%%%%%%%%%%%%%%%%%%%%%%%%%%%%%%%%%%%%%%
%                                                                 %
%   GKS User Guide -- LaTeX Source                                %
%                                                                 %
%   Appendices                                                    %
%                                                                 %
%   The following external EPS files are referenced:              %
%                                                                 %
%   Editor: Michel Goossens / CN-AS                               %
%   Last Mod.: 14 July 1992 12:30 mg                              %
%                                                                 %
%%%%%%%%%%%%%%%%%%%%%%%%%%%%%%%%%%%%%%%%%%%%%%%%%%%%%%%%%%%%%%%%%%%
 
\chapter{\protect\label{sec:exmpref}Example Programs}
\index{examples!polylines and fill areas}
 
The programs which follow have been written in VAX FORTRAN-77
as it is easier to read.
The VAX-specific features which would need to be changed
in order for the programs to become standard FORTRAN-77 are:
\begin{OL}
\item Upper case only
\item In-line comments moved to new line with 'C' in column 1
\item Include statements replaced by the expanded code.
\end{OL}
 
Machine-readable versions of these programs are available at CERN on all
the supported systems.
\section{\protect\label{sec:expfa}Polylines and Fill Areas}
 
This program demonstrates the use of polylines and fill area.
The workstation chosen here is the Tektronix 4107 (or a compatible).
\begin{XMP}
      PROGRAM   BOX
      INCLUDE  'GKS$GTSDEV'
      INCLUDE  'GKS$ENUM'
      INTEGER   errfil
      PARAMETER(errfil = 10)
      INTEGER   wkid
      PARAMETER(wkid   = 1)
      INTEGER   conid
      PARAMETER(conid  = 1)
      REAL      pxa(5), pya(5)
      INTEGER   asflst(1:13)
      DATA      pxa /3.0,3.0,7.0,7.0,3.0/
      DATA      pya /2.0,8.0,8.0,2.0,2.0/
      DATA      asflst /13 * gindiv/  ! set all ASFs
C
      CALL gopks(errfil, 0)           ! open gks (BUFA not used)
      CALL gopwk(wkid, conid, T4107)  ! open workstation
      CALL gsasf(asflst)              ! set attributes individually
      CALL gacwk(wkid)                ! activate workstation
      CALL gselnt(1)                  ! select normalization tran
      CALL gswn(1, 0.0,10.0,0.0,10.0) ! set window
C
      CALL gsln(gldash)               ! set line type
      CALL gsfais(ghatch)             ! set fill area style
      CALL gpl(5, pxa, pya)           ! draw polyline
      CALL gfa(5, pxa, pya)           ! draw fill area
C
      CALL gdawk(wkid)                ! deactivate workstation
      CALL gclwk(wkid)                ! close workstation
      CALL gclks                      ! close GKS
      END
\end{XMP}
\section{Viewports, Polymarkers, and Text}
\index{examples!viewports, polymarkers, and text}
 
This program draws two ducks in different viewports on the same display
surface (see \ref{fig:ducks}). As well as the polyline and fill area
primitives, it also uses polymarkers and text. The position and size of the
text are determined by setting the character up vector and the
character height. The workstation selected in this example is the
Hewlett Packard 7470A 2 pen plotter.
\begin{XMP}
      PROGRAM   DUCK
      INCLUDE  'GKS$GTSDEV'
      INCLUDE  'GKS$ENUM'
      INTEGER   errfil, wkid
      PARAMETER(errfil = 10, wkid = 1)
      REAL      pxa(44), pya(44)
      INTEGER   asflst(1:13)
      DATA      pxa/0.0,2.0,4.0,6.0,8.0,10.0,12.0,14.0,
     *              16.4,17.0,17.3,17.8,18.5,20.0,22.0,24.0,
     *              26.0,28.0,29.0,28.8,27.2,25.0,23.0,21.5,
     *              21.1,21.5,22.8,24.1,25.1,25.2,24.2,22.1,
     *              20.0,18.0,16.0,14.0,12.0,10.0,8.0,6.1,
     *              4.2,3.0,1.3,0.0/
      DATA      pya/8.8,7.6,7.1,7.4,8.0,8.9,9.6,9.9,
     *              9.4,9.7,12.0,14.0,16.1,17.0,17.0,16.0,
     *              13.9,13.1,13.2,12.3,11.5,11.5,11.5,11.2,
     *              10.5,9.0,8.0,7.0,5.1,3.6,1.9,1.1,
     *              0.9,0.7,0.8,1.0,1.0,1.2,1.8,2.1,
     *              2.9,4.1,6.0,8.8/
      DATA      asflst /13 * gindiv/  ! set all ASFs
C
      CALL gopks(errfil, 0)           ! open gks (BUFA not used)
      CALL gopwk(wkid, 6, H475L4)     ! open workstation
      CALL gsasf(asflst)              ! set attributes individually
      CALL gacwk(wkid)                ! activate workstation
C
      CALL gstxfp(1, gstrkp)          ! set font 1, stroke precision
      CALL gschh(1.5)                 ! set char. height
C
C...  Set Normalization Transformation
C
      CALL gswn(1, 0.0,30.0,0.0, 30.0)! set window
      CALL gsvp(1, 0.0, 0.5, 0.0, 0.4)! set viewport in lower left
      CALL gselnt(1)
C
      CALL gsln(glsoli)               ! set solid line type
      CALL gpl(44, pxa, pya)          ! polyline
      CALL gsfais(ghatch)             ! set hatch fill area
      CALL gfa(44, pxa, pya)          ! fill area
      CALL gschup(-1.0, 8.0)          ! set char. up vector
      CALL gtx(8.0, 20.0, '"Quack"')  ! text at position 8,20
C
C...  Re-Set Normalization Transformation
C
      CALL gsvp(1, 0.5, 1.0, 0.4, 0.8)! move vpt to upper right
      CALL gsmk(gast)                 ! set asterisk marker type
      CALL gpm(44, pxa, pya)          ! polymarker
      CALL gsfais(gsolid)             ! set solid fill area
      CALL gfa(44, pxa, pya)          ! fill area
      CALL gschup(0.5, 1.0)           ! reset char. up vector
      CALL gtx(7.0, 20.0, '"Quack"')  ! text at position 8,20
C
      CALL gdawk(wkid)                ! deactivate workstation
      CALL gclwk(wkid)                ! close workstation
      CALL gclks                      ! close GKS
      END
\end{XMP}
\begin{figure}[h]
\caption{Output produced by example 'Ducks'}
\label{fig:ducks}
\end{figure}
\section{Text Fonts}
\index{text!example}
\index{examples!text fonts}
 
This program demonstrates the setting of different
text Fonts, precisions and character up vectors.
String precision may ignore the character-up-vector.
In Font 1, char precision is treated like stroke.
Font -2 is implementation-dependent.
\begin{XMP}
      PROGRAM   FONTS
      INCLUDE  'GKS$GTSDEV'
      INCLUDE  'GKS$ENUM'
      INTEGER   errfil
      PARAMETER(errfil = 6)
      INTEGER   wkid
      PARAMETER(wkid = 1)
      INTEGER   asflst(1:13)
      DATA      asflst /13 * gindiv/  ! set all ASFs
C
      CALL gopks(errfil, 0)           ! open gks (BUFA not used)
      CALL gopwk(wkid, 6, T4107)      ! open workstation
      CALL gsasf(asflst)              ! set attributes individually
      CALL gacwk(wkid)                ! activate workstation
      CALL gselnt(1)                  ! select norm transf.
      CALL gswn(1, 0.0,10.0,0.0,10.0) ! set window
C
      CALL gschh(0.3)                 ! set character height
      CALL gstxfp(1, gstrp)           ! set font 1, string precision
      CALL gtx(1.0, 1.5, 'Font 1 string prec')
      CALL gschup(1.0, 3.0)           ! set character up vector
      CALL gstxfp(1, gcharp)          ! font 1, char precision
      CALL gtx(5.0, 8.0, 'Font 1 char prec')
      CALL gschup(-1.0, 0.5)
      CALL gstxfp(1, gstrkp)          ! font 1, stroke precision
      CALL gtx(2.0, 4.5, 'Font 1 stroke prec')
      CALL gstxfp(-2, gstrp)          ! font -2, string precision
      CALL gtx(2.0, 3.0, 'Font -2 string prec')
      CALL gschup(0.5, -1.0)
      CALL gstxfp(-2, gcharp)         ! font -2, char precision
      CALL gtx(7.0, 6.5, 'Font -2 char prec')
      CALL gschup(1.0, 0.0)
      CALL gstxfp(-2, gstrkp)         ! font -2, stroke precision
      CALL gtx(8.0, 5.5, 'Font -2 stroke prec')
      CALL gdawk(wkid)                ! deactivate workstation
      CALL gclwk(wkid)                ! close workstation
      CALL gclks                      ! close GKS
      END\end{XMP}
\section{Request Input}
\index{input!example}
\index{examples!request input}
 
This program demonstrates REQUEST input for the Choice, Locator,
Valuator and String input classes.
It prompts for locator positions corresponding to the
vertices of a polygon, and indicates them with markers.
Request Choice is then used to find out the interior style
for the fill area, and a title is requested for the picture.
The character height of the title is selected using valuator input
and is followed by another locator request so that the operator can
position the text on the screen.
\begin{XMP}
      PROGRAM INP
      INCLUDE  'GKS$GTSDEV'
      INCLUDE  'GKS$ENUM'
C
      INTEGER      errfil
      PARAMETER   (errfil=6)
      INTEGER      wktyp, wkid,  conid
      PARAMETER   (wkid = 1, conid = 1)
      INTEGER      chcdev, locdev, strdev, valdev  ! device numbers
      PARAMETER   (chcdev = 1, locdev = 1, strdev = 1, valdev = 1)
      INTEGER      tnr, i, status
      INTEGER      lstr, nsides, fill
      REAL         px(10), py(10)
      REAL         chrht, sides
      INTEGER      errind             ! error flag
      CHARACTER*80 str
      INTEGER      asflst(13)
      DATA         asflst/13 * gindiv/! set all ASFs
C
C     Open error log file, GKS and a Workstation
C
      OPEN (unit=errfil, file='errors', status='unknown')
      wktyp = T4107                   ! set workstation type
      CALL gopks(errfil, 0)           ! open gks (bufa not used)
      CALL gopwk(wkid, conid, wktyp)  ! open workstation
      CALL gacwk(wkid)                ! activate workstation
      CALL gsasf(asflst)              ! set attributes individually
C
C     request locator positions
C
      nsides = 6                      ! Why not a hexagon?
      CALL gmsg(wkid, 'Point to 6 vertices')
      DO 30 i = 1, nsides
         CALL grqlc(wkid, locdev, status, tnr, px(i), py(i))
         CALL gpm(1, px(i), py(i))    ! Plot positions
30    CONTINUE
C
C     Request choice for fill area interior style
C
50    CALL gmsg(wkid, 'Type 1-4: hollow,solid,pattern,hatch')
      CALL grqch(wkid, chcdev, status, fill)
      CALL gsfais(fill-1)             ! Set fill area style
      CALL gfa(nsides, px, py)        ! Draw fill area
C
C     Request string for title of picture
C
      CALL gmsg(wkid, 'Give the title of the picture')
      CALL grqst(wkid, strdev, status, lstr, str)
C
C     Request valuator for CHARACTER height
C
      CALL gmsg(wkid, 'Give CHARACTER height (0.01 to 0.1)')
      CALL grqvl(wkid, valdev, status, chrht)
      CALL gschh(chrht)
C
C     Request locator for text position
C
      CALL gmsg(wkid, 'Give text position')
      CALL grqlc(wkid, locdev, status, tnr, px(1), py(1))
      CALL gstxfp(1, gstrkp)          ! font 1, stroke precision
      CALL gtx(px(1), py(1), str(1:lstr))
      CALL gdawk(wkid)                ! deactivate workstation
      CALL gclwk(wkid)                ! close workstation
      CALL gclks                      ! close gks
      END
\end{XMP}
\section{\protect\label{sec:iinput}Input Initialization}
\index{output!example}
\index{examples!input initialization}
 
This program is an extension of the last one.
It starts by prompting for the number of sides for the polygon,
and then continues as before. However, rather than just accept the default
settings for the input devices it initializes them in a suitable way.
As an example, the choice device is set-up to
provide a menu, and the valuator has reasonable limiting values.
 
Note that initialization is implementation-dependent and may also be
device-dependent. This example works for GKSGRAL.
.cc 6
\begin{XMP}
      PROGRAM initst
      INCLUDE  'GKS$GTSDEV'
      INCLUDE  'GKS$ENUM'
C
      INTEGER      errfil
      PARAMETER   (errfil=6)
      INTEGER      wktyp, wkid,  conid
      PARAMETER   (wkid = 1, conid = 1)
      INTEGER      chcdev, locdev, strdev, valdev  ! device numbers
      PARAMETER   (chcdev = 1, locdev = 1, strdev = 1, valdev = 1)
      INTEGER      tnr, pet, i, status, dcunit, lx, ly
      INTEGER      lstr, nsides, fill, lfi(4)
      REAL         px(10), py(10)
      REAL         chrht, sides, rx, ry, inival
      PARAMETER   (inival = 3.0)
      INTEGER      errind             ! error flag
      REAL         dum(2)             ! dummy array
      CHARACTER*40 filename
      CHARACTER*80 str(4)             ! array used by GPREC
      CHARACTER*30 fi
      INTEGER      asflst(13)
      DATA         asflst/13 * gindiv/! set all ASFs
C
C     Open error log file, GKS and a Workstation
C
      OPEN (unit=errfil, file='errors', status='unknown')
      wktyp = T4107                   ! set workstation type
      CALL gopks(errfil, 0)           ! open gks (bufa not used)
      CALL gopwk(wkid, conid, wktyp)  ! open workstation
      CALL gacwk(wkid)                ! activate workstation
      CALL gsasf(asflst)              ! set attributes individually
      CALL gqdsp(wktyp, status, dcunit, rx, ry, lx, ly)
C
C     Request valuator for no of sides of polygon (3-10)
C     Initialize echo area and valuator upper and lower limits
C
      CALL gmsg(wkid, 'Enter the number of sides for the polygon')
      pet = 1                         ! define prompt/echo type
      lstr = 0                        ! data record not used
      CALL ginvl(wkid, valdev, inival, pet, 0.0, rx, 0.0, ry/5,
     *            3.0, 10.0, lstr, str)
      CALL grqvl(wkid, valdev, status, sides)
      nsides = ifix(sides)
C
C     request locator positions
C
      CALL gmsg(wkid, 'point to vertices')
      CALL grqlc(wkid, locdev, status, tnr, px(1), py(1))
      CALL gpm(1, px(1), py(1))
      DO 30 i = 2,nsides
C
C        Request locator for other positions
C        Initialize cursor to previous position
C        (for devices with this capability (eg not tek 4014))
C
         CALL ginlc (wkid,locdev,0,px(i-1),py(i-1),pet,
     *      0.,rx,0.,ry,lstr,str)
         CALL grqlc(wkid, locdev, status, tnr, px(i), py(i))
         CALL gpm(1, px(i), py(i))
30    CONTINUE
C
C     Request choice for fill area interior style (0-3)
C     Use gprec to set up the DATA record used to initialize
C     the logical choice device with 4 menu strings.
C
C     Note:
C     The call to GPREC does not conform to the final FORTRAN Binding.
C
      fi(1:6)   = 'hollow'
      lfi(1)    = 6
      fi(7:11)  = 'solid'
      lfi(2)    = 5
      fi(12:18) = 'pattern'
      lfi(3)    = 7
      fi(19:23) = 'hatch'
      lfi(4)    = 5
      CALL gprec(4, lfi, 0, dum, 23, fi, 4, status, lstr, str)
      CALL ginch(wkid, chcdev, 1, 2, 3, 0.8*rx, rx, 0.5*ry, ry,
     *           lstr, str)
      CALL grqch(wkid, chcdev, status, fill)
      CALL gsfais(fill-1)             ! Set fill area style
      CALL gfa(nsides, px, py)        ! Draw fill area
C
C     Request string for title of picture
C     Initialize default string and echo area
C
      CALL gmsg(wkid, 'Give the title of the picture')
      CALL ginst(wkid, strdev, 14, 'picture title', pet,
     *            0.0 ,rx ,0.0, 0.1*ry, 80, 1, 1, str)
      CALL grqst (wkid, strdev, status, lstr, str)
C
C     Request valuator for CHARACTER height
C     Initialize valuator echo area plus value limits 1/100 to 1/10.
C
      CALL gmsg(wkid, 'Give CHARACTER height (0.01 to 0.1)')
      CALL ginvl(wkid, valdev, 0.1, pet, 0.0, rx, 0.0, ry/5,
     *           0.01 ,0.1, 1, str)
      CALL grqvl(wkid, valdev, status, chrht)
      CALL gschh(chrht)
C
C     Request locator for text position
C
      CALL gmsg(wkid, 'Give text position')
      CALL grqlc(wkid, locdev, status, tnr, px(1), py(1))
      CALL gstxfp(1, gstrkp)          ! font 1, stroke precision
      CALL gtx(px(1), py(1), str(1)(1:lstr))
      CALL gdawk(wkid)                ! deactivate workstation
      CALL gclwk(wkid)                ! close workstation
      CALL gclks                      ! close gks
      END
\end{XMP}
\section{\protect\label{sec:ex3dv}GKS-3D Segments and Viewing}
\index{GKS3D!example}
\index{examples!3D, segments and viewing}
 
This example illustrates the use of segments, segment transformations,
and 3D viewing. The program first sets up the Normalization Transformation
to map the whole of the WC space onto the display surface, assuming that
this is either a square or a landscape-oriented rectangle. The viewing
parameters are set to look along the Z axis towards the origin.
The program then draws a tetrahedron in a segment with a Text 3 character
string along the front bottom edge, and in the plane of the front face.
Next, it re-draws the same tetrahedron in another segment,
which is positioned with a segment transformation.
 
Note that the second tetrahedron is first drawn
with visibility off, so that when the transformation is set the
un-drawing and re-drawing in the new orientation does not wipe out
the first tetrahedron, which would have been exactly underneath.
As an alternative, the segment transformation for the second segment
could have been set before calling the routine to draw the tetrahedron.
 
Finally, the program modifies the View Plane Normal, so that the two
tetrahedra may be seen from a different direction.
\begin{XMP}
      PROGRAM demo3d
C
C     Version 2.0, 18.03.88    - New FORTRAN binding
C
      INCLUDE  'GKS$GTSDEV'
      INCLUDE  'GKS$ENUM'
C
      INTEGER      errfil
      PARAMETER   (errfil=10)
      INTEGER      wktyp, wkid,  conid
      PARAMETER   (wkid = 1, conid = 1)
      INTEGER      chcdev, locdev, strdev, valdev  ! device numbers
      PARAMETER   (chcdev = 1, locdev = 1, strdev = 1, valdev = 1)
      INTEGER      errind
      REAL         vrpx,   vrpy,   vrpz
      REAL         vupx,   vupy,   vupz
      REAL         vpnx,   vpny,   vpnz
      REAL         prpu,   prpv,   prpn
      REAL         vp(6),  wn(6),  wkvp(6),wkwn(6),prvp(6)
      REAL         vpd,    bpd,    fpd
      REAL         umin,   umax,   vmin,   vmax
      INTEGER      iclw,   iclb,   iclf
      INTEGER      dcunit, lx,     ly,     lz
      REAL         rx,     ry,     rz
      REAL         sgmtx(3,4)
      REAL         vwmtx(4,4)
      REAL         prmtx(4,4)
      INTEGER      tnr,    vwi
      PARAMETER   (tnr=1,  vwi=1)
      CHARACTER*80 str
      INTEGER      lstr
      INTEGER      asflst(13)
      DATA         asflst/13 * gindiv/! set all ASFs
C
C     Set viewing parameters. Look along Z axis TOWARDS origin
C
      DATA         vrpx, vrpy, vrpz / 0.5, 0.5, 0.5 /
      DATA         vupx, vupy, vupz / 0.0, 1.0, 0.0 /
      DATA         vpnx, vpny, vpnz / 0.0, 0.0, 1.0 /
      DATA         prpu, prpv, prpn / 0.0, 0.0, 1.0 /
      DATA         prvp             / 0.0, 1.0, 0.0, 1.0, 0.0, 1.0 /
      DATA         umin, umax       /-0.5, 0.5 /
      DATA         vmin, vmax       /-0.5, 0.5 /
      DATA         bpd,  fpd, vpd   /-0.5, 0.5, 0.0 /
C
      DATA         vp         / 0.0, 1.0, 0.0, 1.0, 0.0, 1.0 /
      DATA         wn         /-15.0, 15.0, -15.0, 15.0, -15.0, 15.0/
      DATA         wkwn       / 0.0, 1.0, 0.0, 1.0, 0.0, 1.0 /
      DATA         iclw       / gnclip /
      DATA         iclb, iclf / gnclip, gnclip /
C
C     Open error log file, GKS and a Workstation
C
      OPEN (unit=errfil, file='errors', status='unknown')
C
C  request the workstation type on which the program is to be run
C
      wktyp = T4014                   ! set workstation type
      CALL gopks(errfil, 0)           ! open gks (bufa not used)
      CALL gopwk(wkid, conid, wktyp)  ! open workstation
      CALL gacwk(wkid)                ! activate workstation
      CALL gsasf(asflst)              ! set attributes individually
      CALL gsds(wkid, gasap, gperfo)  ! As Soon As Possible + regen
C
C     Set Workstation Window and Viewport to use whole display
C     and also the Normalization Transformation Window and Viewport.
C     If the window is not square the Aspect Ratio will be distorted.
C
      CALL gqdvol(wktyp ,errind, dcunit, rx,ry,rz, lx,ly,lz)
      IF (rx .ge. ry) THEN
        wkwn(4) = ry/rx
        vp(4) = ry/rx
      ELSE
        wkwn(2) = rx/ry
        vp(2) = rx/ry
      ENDIF
      CALL gswkw3(wkid, wkwn)
      CALL gsv3(tnr,  vp)
      wkvp(1) = 0.0
      wkvp(2) = rx
      wkvp(3) = 0.0
      wkvp(4) = ry
      wkvp(5) = 0.0
      wkvp(6) = rz
      CALL gswkv3(wkid,  wkvp)
      CALL gsw3 (tnr, wn)
C
      CALL gselnt(tnr)                ! Select Normalization Tfrm
      CALL gsvwi(vwi)                 ! Select Viewing Tfrm
      CALL gsclip(gnclip)             ! Set clipping off
C
C     Evaluate View Matrix & Projection Matrix (with parallel projection)
C     Set View Representation (use same projection viewport and clip limits).
C
      CALL gevvwm(vrpx, vrpy, vrpz, vupx, vupy, vupz,
     *            vpnx, vpny, vpnz, gndc, errind, vwmtx)
      IF (errind .ne. 0) THEN
         WRITE(6, *) 'Error in EValuate VieW Matrix ', errind
         GOTO 9999
      ENDIF
      CALL gevpjm(umin, umax, vmin, vmax, prvp, gparl,
     *            prpu, prpv, prpn, vpd,  bpd,  fpd, errind, prmtx)
      IF (errind .ne. 0) THEN
         WRITE(6, *) 'Error in EValuate ProJection Matrix ', errind
         GOTO 9999
      ENDIF
 
      CALL gsvwr(wkid, vwi,  vwmtx, prmtx, prvp, iclw, iclb, iclf)
 
      CALL guwk(wkid, gperfo)         ! Update Workstation
C
C     Create two 3D segments, one with a transformation
C
      CALL gcrsg(1)                   ! Create segment 1
      CALL tetra                      ! Draw a tetrahedron
      CALL gclsg                      ! close segment
      CALL gcrsg(2)                   ! Create segment 2
      CALL gsvis(2, ginvis)           ! Make it invisible
      CALL tetra                      ! Draw a tetrahedron
      CALL gclsg                      ! Close segment
C
      CALL gevtm3(2.0, 3.0, 4.0, -6.0, 3.0, 0.5,
     *            0.0, 0.3, 0.7,  0.5, 0.5, 0.7, gwc, sgmtx)
      CALL gssgt3(2, sgmtx)           ! Transform segment
      CALL gsvis (2, gvisi)           ! and make it visible
C
C     Give user a chance to see result, then transform view
C
      CALL gmsg (wkid, 'Hit <return> to continue')
      CALL grqst(wkid, strdev, errind, lstr, str)
C
      vpnx = 0.3                      ! Change direction of
      vpny = 0.6                      ! View Plane Normal
      vpnz = 1.0
      CALL gevvwm(vrpx, vrpy, vrpz, vupx, vupy, vupz,
     *            vpnx, vpny, vpnz, gndc, errind, vwmtx)
      IF (errind .ne. 0) THEN
         WRITE(6, *) 'Error in EValuate VieW Matrix ', errind
         GOTO 9999
      ENDIF
      CALL gsvwr(wkid, vwi,  vwmtx, prmtx, prvp, iclw, iclb, iclf)
      CALL guwk(wkid, gperfo)         ! Update Workstation
C
C     Give user a chance to see result, then exit
C
      CALL gmsg(wkid, 'Hit <return> to continue')
      CALL grqst(wkid, strdev, errind, lstr, str)
C
9999  CALL gdawk(wkid)                ! deactivate workstation
      CALL gclwk(wkid)                ! close workstation
      CALL gclks                      ! close gks
      END
      SUBROUTINE tetra
C
C     Draw a tetrahedron with a 3D text string along one edge
C
      INCLUDE  'GKS$ENUM'
C
      REAL    plax(6), play(6), plaz(6)
      REAL    plbx(2), plby(2), plbz(2)
      REAL    vx(2)  , vy(2)  , vz(2)
      DATA    plax /  0.0, -5.7,  5.7,  0.0,  0.0, -5.7 /
      DATA    play / -5.0, -5.0, -5.0, -5.0, 10.0, -5.0 /
      DATA    plaz /-10.0,  5.0,  5.0,-10.0,  0.0,  5.0 /
      DATA    plbx /  0.0,  5.7/
      DATA    plby / 10.0, -5.0 /
      DATA    plbz /  0.0,  5.0 /
C
      CALL gpl3(6, plax, play, plaz)
      CALL gpl3(2, plbx, plby, plbz)
      CALL gschh(1.5)                 ! Set character height
      vx(1) = plax(3) - plax(2)       ! Text Direction Vector 1
      vy(1) = play(3) - play(2)       ! Text in front plane of
      vz(1) = plaz(3) - plaz(2)       !       the tetrahedron
      vx(2) = plax(5) - plax(2)       ! Text Direction Vector 2
      vy(2) = play(5) - play(2)
      vz(2) = plaz(5) - plaz(2)
      CALL gschup(0.0, 1.0)           ! Set character up vector
      CALL gstxfp(1, gstrkp)          ! font 1, stroke precision
      CALL gtx3(-5.0, -5.0, 5.0,      ! Write along bottom line
     *          vx, vy, vz, 'Demo-3D')
      END
\end{XMP}
\chapter{GKS/GKS-3D Error Codes}
\index{error codes}
 
The error codes defined in this section are of two types:
\begin{OL}
\item Those which are defined by GKS/GKS-3D.
\item Those which are implementation-dependent.
These are between 900 and 2000.
\end{OL}
\section{STATES}
\begin{DLtt}{123456}
\item[1]
GKS not in proper state: GKS shall be in the state GKCL
\item[2]
GKS not in proper state: GKS shall be in the state GKOP
\item[3]
GKS not in proper state: GKS shall be in the state WSAC
\item[4]
GKS not in proper state: GKS shall be in the state SGOP
\item[5]
GKS not in proper state: GKS shall be either in the state WSAC or in
the state SGOP
\item[6]
GKS not in proper state: GKS shall be either in the state WSOP or in
the state WSAC
\item[7]
GKS not in proper state: GKS shall be in one of the states WSOP, WSAC,
or SGOP
\item[8]
GKS not in proper state: GKS shall be in one of the states GKOP, WSOP,
WSAC or SGOP
\end{DLtt}
\section{WORKSTATIONS}
\begin{DLtt}{123456}
\item[20]
Specified workstation identifier is invalid
\item[21]
Specified connection identifier is invalid
\item[22]
Specified workstation type is invalid
\item[23]
Specified workstation type does not exist
\item[24]
Specified workstation (wkid or [condi,wtype] pair) is open
\item[25]
Specified workstation is not open
\item[26]
Specified workstation cannot be opened
\item[27]
Workstation Independent Segment Storage is not open
\item[28]
Workstation Independent Segment Storage is already open
\item[29]
Specified workstation is active
\item[30]
Specified workstation is not active
\item[31]
Specified workstation is of category MO (Metafile Output)
\item[32]
Specified workstation is not of category MO (Metafile Output)
\item[33]
Specified workstation is of category MI (Metafile Input)
\item[34]
Specified workstation is not of category MI (Metafile Input)
\item[35]
Specified workstation is of category INPUT
\item[36]
Specified workstation is Workstation Independent Segment Storage
\item[37]
Specified workstation is not of category OUTIN
\item[38]
Specified workstation is neither of category INPUT nor of category OUTIN
\item[39]
Specified workstation is neither of category OUTPUT nor of category OUTIN
\item[40]
Specified workstation has no pixel store read-back capability
\item[41]
Specified workstation type is not able to generate the specified
drawing primitive
\item[42]
Maximum number of simultaneously open workstations would be exceeded.
\item[43]
Maximum number of simultaneously active workstations would be exceeded.
\end{DLtt}
\section{2D TRANSFORMATIONS}
\begin{DLtt}{123456}
\item[50]
Transformation number is invalid
\item[51]
Rectangle definition is invalid
\item[52]
Viewport is not within the Normalized Device Coordinate unit square
\item[53]
Workstation window is not within the Normalized Device Coordinate unit
square
\item[54]
Workstation viewport is not within the display space
\end{DLtt}
\section{2D OUTPUT ATTRIBUTES}
\begin{DLtt}{123456}
\item[60]
Polyline index is invalid
\item[61]
A representation for the specified polyline index has not been defined
on this workstation
\item[62]
Line type is less than or equal to zero
\item[63]
Specified line type is not supported on this workstation
\item[64]
Polymarker index is invalid
\item[65]
A representation for the specified polymarker index has not been defined
on this workstation
\item[66]
Marker type is less than or equal to zero
\item[67]
Specified marker type is not supported on this workstation
\item[68]
Text index is invalid
\item[69]
A representation for the specified text index has not been defined on
this workstation
\item[70]
Text font is less than or equal to zero
\item[71]
Requested text font is not supported for the specified precision on
this workstation
\item[72]
Character expansion factor is less than or equal to zero
\item[73]
Character height is less than or equal to zero
\item[74]
Length of character up vector is zero
\item[75]
Fill area index is invalid
\item[76]
A representation for the specified fill area index has not been defined
on this workstation
\item[77]
Specified fill area interior style is not supported on this workstation
\item[78]
Style (pattern or hatch) index is less than or equal to zero
\item[79]
Specified pattern index is invalid
\item[80]
Specified hatch style is not supported on this workstation
\item[81]
Pattern size value is not positive
\item[82]
A representation for the specified pattern index has not been defined
on this workstation
\item[83]
Interior style PATTERN is not supported on this workstation
\item[84]
Dimensions of colour array are invalid
\item[85]
Colour index is less than zero
\item[86]
Colour index is invalid
\item[87]
A representation for the specified colour index has not been defined
on this workstation
\item[88]
Colour is outside range [0,1]
\item[89]
Pick identifier is invalid
\item[90]
Interior style PATTERN not supported by this workstation.
\item[91]
Dimensions of colour array invalid
\item[92]
Colour index less than zero
\item[93]
Colour index invalid
\item[94]
A representation for the specified colour index has not been
specified on this workstation.
\item[95]
A representation for the specified colour index has not been
pre-defined on this workstation.
\item[96]
Colour is outside range [0,1].
\item[97]
Pick identifier is invalid
\end{DLtt}
\section{2D OUTPUT PRIMITIVES}
\begin{DLtt}{123456}
\item[100]
Number of points is invalid
\item[101]
Invalid code in string
\item[102]
Generalized drawing primitive identifier is invalid
\item[103]
Content of generalized drawing primitive data record is invalid
\item[104]
At least one active workstation is not able to generate the specified
generalized drawing primitive
\item[105]
At least one active workstation is not able to generate the specified
generalized drawing primitive under the current transformations and
clipping rectangle.
\end{DLtt}
\section{SEGMENTS}
\begin{DLtt}{123456}
\item[120]
Specified segment name is invalid
\item[121]
Specified segment name is already in use
\item[122]
Specified segment does not exist
\item[123]
Specified segment does not exist on specified workstation
\item[124]
Specified segment does not exist on workstation Independent Segment
Storage
\item[125]
Specified segment is open
\item[126]
Segment priority is outside the range [0,1]
\item[-127]
(GTS-GRAL only) Number of segments exceeded, create failed.
\item[-128]
(GTS-GRAL only) Transformation matrix invalid
\end{DLtt}
\section{INPUT}
\begin{DLtt}{123456}
\item[140]
Specified input device is not present on workstation
\item[141]
Input device is not in REQUEST mode
\item[142]
Input device is not in SAMPLE mode
\item[143]
EVENT and SAMPLE input are not available at this level of GKS
\item[-143]
(GTS-GRAL only) EVENT and SAMPLE input are not available for the specified
workstation.
\item[144]
Specified prompt and echo type is not supported on this workstation
\item[145]
Echo area is outside display space
\item[146]
Contents of input data record are invalid
\item[147]
Input queue has overflowed
\item[148]
Input queue has not overflowed since GKS was opened or the last invocation
of Inquire Input Queue Overflow
\item[149]
Input queue has overflowed, but associated workstation has been closed
\item[150]
No input value of the correct class is in the current event report
\item[151]
Timeout is invalid
\item[152]
Initial value is invalid
\item[153]
Length of initial string is greater than the implementation defined
maximum
\item[154]
Length of initial string is greater than the buffer size
\end{DLtt}
\section{METAFILES}
\begin{DLtt}{123456}
\item[160]
Item type is not allowed for user items
\item[161]
Item type is invalid
\item[162]
No item is left in GKS metafile input
\item[163]
Metafile item is invalid
\item[164]
Item type is not a valid GKS item
\item[165]
Content of item data record is invalid for the specified item type
\item[166]
Maximum item data record length is invalid
\item[167]
User item cannot be interpreted
\item[168]
Specified function not supported by this level of GKS-3D
\item[-170]
(GTS-GRAL only) Wrong CGM code during TEST OPEN
\item[-171]
(GTS-GRAL only) CGM not in state TEST OPEN during APPEND TEXT
\item[-172]
(GTS-GRAL only) Local text buffer too small
\end{DLtt}
\section{ESCAPE}
\begin{DLtt}{123456}
\item[180]
Specified function is not supported
\item[181]
Specified escape function identification is invalid
\item[182]
Contents of escape data record are invalid
\end{DLtt}
\section{MISCELLANEOUS}
\begin{DLtt}{123456}
\item[200]
Specified error file is invalid
\end{DLtt}
\section{SYSTEM}
\begin{DLtt}{123456}
\item[300]
Specified function is not supported in this level of GKS
\item[301]
Storage overflow has occurred in GKS
\item[302]
Input/Output error has occurred whilst reading (eg font files)
\item[303]
Input/Output error has occurred whilst writing
\item[304]
Input/Output error has occurred whilst sending data to a workstation
\item[305]
Input/Output error has occurred whilst receiving data from a workstation
\item[306]
Input/Output error has occurred during program library management
\item[307]
Input/Output error has occurred whilst reading workstation
description table
\item[308]
Arithmetic error has occurred
\end{DLtt}
\section{3D TRANSFORMATIONS}
\begin{DLtt}{123456}
\item[400]
View up vector and view plane normal are collinear
\item[401]
View plane normal is null a vector
\item[402]
View up vector is a null vector
\item[403]
Projection viewport limits are not within NPC range
\item[404]
Projection reference point is between front and back clipping planes
\item[405]
Projection reference point is on the view plane
\item[406]
Box definition is invalid
\item[407]
Viewport is not within NDC unit cube
\item[408]
Specified view index is invalid
\item[409]
A representation for the specified view index has not been defined
on this workstation.
\item[410]
A representation for the specified view index has not been pre-defined
on this workstation.
\item[411]
Workstation window limits are not within the NPC unit cube
\item[412]
Back clipping plane is in front of front clipping plane
\end{DLtt}
\section{3D OUTPUT ATTRIBUTES}
\begin{DLtt}{123456}
\item[420]
Edge index is invalid
\item[421]
A representation for the specified edge index index has not been
defined on this workstation.
\item[422]
A representation for the specified edge index index has not been
pre-defined on this workstation.
\item[423]
Edge type equal to zero
\item[424]
Specified edge type not supported by this workstation
\item[425]
Edge width scale factor less than zero
\item[426]
Pattern reference vectors are collinear
\item[427]
Specified HLHSR mode not supported on this workstation
\item[428]
Specified HLHSR identifier invalid
\item[429]
Specified HLHSR mode invalid
\end{DLtt}
\section{3D OUTPUT PRIMITIVES}
\begin{DLtt}{123456}
\item[430]
The text direction vectors are collinear
\item[431]
List of points is invalid
\item[432]
At least one active workstation is not able to generate the specified
generalized drawing primitive under the current transformations and
clipping volume.
\end{DLtt}
\section{IMPLEMENTATION DEPENDENT}
\subsection{Device independent errors}
\begin{DLtt}{123456}
\item[-600]
(GTS-GRAL only) Invalid values for cubic curve
\item[900]
Enumeration type out of range
\item[901]
Output parameter size insufficient
\item[902]
List member or set element not available (for inquiry routines)
\item[903]
Invalid data record - cannot be decoded
(Errors 900-903 may also appear as errors 500-503. Stay tuned.)
\item[-920]
(GTS-GRAL only) DSYS-error: keys outside range (MINKEY,MAXKEY)
\item[-921]
(GTS-GRAL only) DSYS-error: key (K1,K2) already present
\item[-922]
(GTS-GRAL only) DSYS-error: key (K1,K2) does not exist
\item[-924]
(GTS-GRAL only) DSYS-error: buffer size in GKDSGC too small,
records truncated
\item[-925]
(GTS-GRAL only) DSYS-error: incorrect record size used
(installation problem)
\item[-927]
(GTS-GRAL only) DSYS-error: whilst reading record from external file
\item[-928]
(GTS-GRAL only) DSYS-error: whilst writing record to external file
\item[-929]
(GTS-GRAL only) DSYS-error: whilst reading record from external file
(Garbage file)
\item[-930]
(GTS-GRAL only) DSYS-error: whilst writing record to external file
(Garbage file)
\item[-950]
(GTS-GRAL only) Missing font files
\item[-1000]
(GTS-GRAL only) Implementation Error
\end{DLtt}
\subsection{Device dependent errors}
\begin{DLtt}{123456}
\item[-7301]
(GTS-GRAL only) UIS driver error in file WINDOW.DAT
\item[-7302]
(GTS-GRAL only) UIS driver error: not enough coordinates to open window
\item[-7303]
(GTS-GRAL only) UIS driver error: no logical unit is free
\end{DLtt}
 
Typical 7xxx errors include the inability to open or read Font files,
Apollo \Lit{gks_characteristic} files, etc. These are usually due to
incorect installation. For example, files may be missing or have the
wrong format, windows may have illegal sizes, etc.
\subsection{Error codes specific to the FORTRAN language binding}
\begin{DLtt}{123456}
\item[2000]Enumeration type out of range; the integer passed as a
GKS enumerated type is not within the range of valid values.
\item[2001]Output parameter size insufficient; a FORTRAN array or
string being passed as an output parameter is too small to contain
the returned information.
\item[2002]List element or member not available; for a non-empty
list or set, a value less than zero or greater than the size of a
list or set was passed as the requested list element or set
member in an inquiry routine.
\item[2003]Invalid data record; the data record cannot be decoded,
or there was a problem encountered when the data record was
created, making the result invalid.
\end{DLtt}
\section{RESERVED ERRORS}
 
Unused error numbers less than 2000 are reserved for
future standardization.
 
Error numbers 2000-3999 are reserved for language bindings.
 
Error numbers greater than 4000 are reserved for future registration.
\chapter{\protect\label{sec:metitm}Metafile Items}
 
Note that whilst the structure of the 2D and 3D GKS Appendix E metafiles
is the same, their contents are not inter-changeable. For example, item 11
corresponds to a polyline in both cases. However, in the 3D metafile each
point in item 11 is represented by three coordinates, rather than by 2
coordinates in the 2D case.
\section{GKS}
\index{metafile!2D item numbers}
 
The following is a list of the GKS Appendix E Metafile item numbers (2D).

\subsection*{General}

\begin{DLtt}{123456}
\item[{\rm\bf Item}] {\bf Meaning}
\item[0]End item (last item of every metafile)
\item[1]Clear workstation
\item[2]Re-draw all segments on workstation
\item[3]Update workstation
\item[4]Set deferral state
\item[5]Message
\item[6]Escape
\end{DLtt}

\subsection*{Items for Output Attributes}

\begin{DLtt}{123456}
\item[{\rm\bf Item}] {\bf Meaning}
\item[11]Polyline
\item[12]Polymarker
\item[13]Text
\item[14]Fill Area
\item[15]Cell Array
\item[16]Generalized Drawing Primitive
\end{DLtt}

\subsection*{Items for Output Primitive Attributes}

\begin{DLtt}{123456}
\item[{\rm\bf Item}] {\bf Meaning}
\item[21]Polyline index
\item[22]Line type
\item[23]Line width scale factor
\item[24]Polyline colour index
\item[25]Polymarker index
\item[26]Marker type
\item[27]Marker size scale factor
\item[28]Polymarker colour index
\item[29]Text index
\item[30]Text font and precision
\item[31]Character expansion factor
\item[32]Character spacing
\item[33]Text colour index
\item[34]Character vectors
\item[35]Text path
\item[36]Text alignment
\item[37]Fill area index
\item[38]Fill area interior style
\item[39]Fill area style index
\item[40]Fill area colour index
\item[41]Pattern size
\item[42]Pattern reference point
\item[43]Aspect source flags
\item[44]Pick identifier
\end{DLtt}

\subsection*{Items for Workstation Attributes}

\begin{DLtt}{123456}
\item[{\rm\bf Item}] {\bf Meaning}
\item[51]Polyline representation
\item[52]Polymarker representation
\item[53]Text representation
\item[54]Fill area representation
\item[55]Pattern representation
\item[56]Colour representation
\end{DLtt}

\subsection*{Items for Transformation}

\begin{DLtt}{123456}
\item[{\rm\bf Item}] {\bf Meaning}
\item[61]Clipping area
\item[62]Clipping indicator
\item[71]Workstation window
\item[72]Workstation viewport
\end{DLtt}

\subsection*{Items for Segment Manipulation}

\begin{DLtt}{123456}
\item[{\rm\bf Item}] {\bf Meaning}
\item[81]Create segment
\item[82]Close segment
\item[83]Rename segment
\item[84]Delete segment
\end{DLtt}

\subsection*{Items for Segment Attributes}

\begin{DLtt}{123456}
\item[{\rm\bf Item}] {\bf Meaning}
\item[91]Set segment transformation
\item[92]Set visibility
\item[93]Set highlighting
\item[94]Set segment priority
\item[95]Set detectability
\end{DLtt}

\subsection*{User Items}

\begin{DLtt}{123456}
\item[{\rm\bf Item}] {\bf Meaning}
\item[>100]User item (comments etc)
\end{DLtt}

\section{GKS-3D}
\index{metafile!3D item numbers}
 
The following is a list of the GKS-3D Appendix E Metafile item numbers.

\subsection*{General}

\begin{DLtt}{123456}
\item[{\rm\bf Item}] {\bf Meaning}
\item[0]End item (last item of every metafile)
\item[1]Clear workstation
\item[2]Re-draw all segments on workstation
\item[3]Update workstation
\item[4]Set deferral state
\item[5]Message
\item[6]Escape
\end{DLtt}

\subsection*{Items for Output Attributes}

\begin{DLtt}{123456}
\item[{\rm\bf Item}] {\bf Meaning}
\item[11]Polyline 3
\item[12]Polymarker 3
\item[13]Text 3
\item[14]Fill Area 3
\item[15]Fill Area Set 3
\item[16]Cell Array 3
\item[17]Generalized Drawing Primitive 3
\end{DLtt}

\subsection*{Items for Output Primitive Attributes}

\begin{DLtt}{123456}
\item[{\rm\bf Item}] {\bf Meaning}
\item[21]Polyline index
\item[22]Line type
\item[23]Line width scale factor
\item[24]Polyline colour index
\item[25]Polymarker index
\item[26]Marker type
\item[27]Marker size scale factor
\item[28]Polymarker colour index
\item[29]Text index
\item[30]Text font and precision
\item[31]Character expansion factor
\item[32]Character spacing
\item[33]Text colour index
\item[34]Character vectors
\item[35]Text path
\item[36]Text alignment
\item[37]Fill area index
\item[38]Fill area interior style
\item[39]Fill area style index
\item[40]Fill area colour index
\item[41]Set edge index
\item[42]Set edge flag
\item[43]Set edge type
\item[44]Set edge width scale factor
\item[45]Set edge colour index
\item[46]Pattern size
\item[47]Pattern reference point and vectors
\item[48]Aspect source flags 3
\item[49]Aspect source flags
\item[50]Pick identifier
\end{DLtt}

\subsection*{Items for Workstation Attributes}

\begin{DLtt}{123456}
\item[{\rm\bf Item}] {\bf Meaning}
\item[51]Polyline representation
\item[52]Polymarker representation
\item[53]Text representation
\item[54]Fill area representation
\item[55]Edge representation
\item[56]Pattern representation
\item[57]Colour representation
\end{DLtt}

\subsection*{Items for Transformation}

\begin{DLtt}{123456}
\item[{\rm\bf Item}] {\bf Meaning}
\item[61]Clipping volume
\item[62]Clipping indicator
\item[63]View index
\item[64]View representation 3
\item[65]HLHSR identifier
\item[66]HLHSR mode
\item[71]Workstation window 3
\item[72]Workstation viewport 3
\end{DLtt}

\subsection*{Items for Segment Manipulation}

\begin{DLtt}{123456}
\item[{\rm\bf Item}] {\bf Meaning}
\item[81]Create segment
\item[82]Close segment
\item[83]Rename segment
\item[84]Delete segment
\end{DLtt}

\subsection*{Items for Segment Attributes}

\begin{DLtt}{123456}
\item[{\rm\bf Item}] {\bf Meaning}
\item[91]Set segment transformation 3
\item[92]Set visibility
\item[93]Set highlighting
\item[94]Set segment priority
\item[95]Set detectability
\end{DLtt}

\subsection*{User Items}

\begin{DLtt}{123456}
\item[{\rm\bf Item}] {\bf Meaning}
\item[>100]User item (comments etc)
\end{DLtt}

\chapter{\protect\label{sec:hdenum}GKS Include Files}
\index{include files}
\index{enumerated types}
 
At CERN there are several machine-readable include files which
should be used when specifying GKS enumerated types and workstations.
The use of symbolic names greatly aids programming and, even
more important, program debugging.
 
The include file GTSDEV.INC contains the symbolic names and codes
for the various Workstation Types available for GKSGRAL.
 
The following is a listing of the file ENUM.INC, containing
mnemonic FORTRAN names and their values for the GKS ENUMERATION types.
\begin{XMP}
C
C     ENUM.INC
C     GKS and GKS-3D Enumeration Types
C     ISO/DIS-8651-1 & ISO/IEC DIS 8806-1
C     (Last Update: 27-04-89)
C
C  aspect source:  bundled  individual
      INTEGER     GBUNDL,    GINDIV
      PARAMETER  (GBUNDL=0,  GINDIV=1)
C
C clear control flag:  conditionally, always
      INTEGER     GCONDI,    GALWAY
      PARAMETER  (GCONDI=0,  GALWAY=1)
C
C clipping disable and enable
      INTEGER     GNCLIP,    GCLIP
      PARAMETER  (GNCLIP=0,  GCLIP=1)
C
C colour available:  monochrome, colour
      INTEGER     GMONOC,    GCOLOR
      PARAMETER  (GMONOC=0,  GCOLOR=1)
C
C coordinate switch:  World Coordinates, Normalized Device Coordinates
      INTEGER     GWC,       GNDC
      PARAMETER  (GWC=0,     GNDC=1)
C
C deferral mode:   ASAP,      BNIG,       BNIL,      ASTI
      INTEGER     GASAP,     GBNIG,      GBNIL,     GASTI
      PARAMETER  (GASAP=0,   GBNIG=1,    GBNIL=2,   GASTI=3)
C
C detectability:   undetectable, detectable
      INTEGER     GUNDET,    GDETEC
      PARAMETER  (GUNDET=0,  GDETEC=1)
C
C device coordinate units: meters,  other
      INTEGER     GMETRE,    GOTHU
      PARAMETER  (GMETRE=0,  GOTHU=1)
C
C display surface: empty  not-empty,  empty
      INTEGER     GNEMPT,    GEMPTY
      PARAMETER  (GNEMPT=0,  GEMPTY=1)
C
C dynamic modification: IRG,      IMM
      INTEGER     GIRG,      GIMM
      PARAMETER  (GIRG=0,    GIMM=1)
C
C echo switch:   no-echo, echo
      INTEGER     GNECHO,    GECHO
      PARAMETER  (GNECHO=0,  GECHO=1)
C
C fill area interior style:  hollow, solid, pattern, hatch
      INTEGER     GHOLLO,    GSOLID,    GPATTR,    GHATCH
      PARAMETER  (GHOLLO=0,  GSOLID=1,  GPATTR=2,  GHATCH=3)
C
C highlighting:   normal,  highlighted
      INTEGER     GNORML,    GHILIT
      PARAMETER  (GNORML=0,  GHILIT=1)
C
C input device status:   none,  ok,  no-pick,  no-choice
      INTEGER     GNONE,     GOK,       GNPICK,    GNCHOI
      PARAMETER  (GNONE=0,   GOK=1,     GNPICK=2,  GNCHOI=2)
C
C input class: none, locator, stroke, valuator, choice, pick, string
      INTEGER     GNCLAS,    GLOCAT,    GSTROK,    GVALUA,
     *            GCHOIC,    GPICK,     GSTRIN
      PARAMETER  (GNCLAS=0,  GLOCAT=1,  GSTROK=2,  GVALUA=3,
     *            GCHOIC=4,  GPICK=5, GSTRIN=6)
C
C implicit regeneration: mode suppressed, allowed
      INTEGER      GSUPPD,    GALLOW
      PARAMETER   (GSUPPD=0,  GALLOW=1)
C
C level of GKS:   L0a,  L0b,  L0c,  L1a,  L1b,  L1c,  L2a,  L2b,  L2c
      INTEGER     GL0A,    GL0B,    GL0C,    GL1A,   GL1B,
     *            GL1C,    GL2A,    GL2B,    GL2C
      PARAMETER  (GL0A=0,  GL0B=1,  GL0C=2,  GL1A=3, GL1B=4,
     *            GL1C=5,  GL2A=6,  GL2B=7,  GL2C=8)
C
C new frame action necessary: no, yes
      INTEGER     GNO,       GYES
      PARAMETER  (GNO=0,     GYES=1)
C
C off/on switch for edge flag
      INTEGER     GOFF,      GON
      PARAMETER  (GOFF=0,    GON=1)
C
C operating mode:     request,    sample,    event
      INTEGER     GREQU,      GSAMPL,    GEVENT
      PARAMETER  (GREQU=0,    GSAMPL=1,  GEVENT=2)
C
C operating state value: GKS closed, GKS open, Workstation open,
C                 Workstation active, Segment open
      INTEGER     GGKCL,     GGKOP,     GWSOP,     GWSAC,
     *            GSGOP
      PARAMETER  (GGKCL=0,   GGKOP=1,   GWSOP=2,   GWSAC=3,
     *            GSGOP=4)
C
C presence of invalid values:  absent, present
      INTEGER     GABSNT,    GPRSNT
      PARAMETER  (GABSNT=0,  GPRSNT=1)
C
C projection type for 3D: Parallel or Perspective
      INTEGER     GPARL,     GPERS
      PARAMETER  (GPARL=0,   GPERS=1)
C
C regeneration flag:   postpone,   perform
      INTEGER     GPOSTP,    GPERFO
      PARAMETER  (GPOSTP=0,  GPERFO=1)
C
C relative input priority:  higher,  lower
      INTEGER     GHIGHR,    GLOWER
      PARAMETER  (GHIGHR=0,  GLOWER=1)
C
C simultaneous events flag: no-more, more
      INTEGER     GNMORE,   GMORE
      PARAMETER  (GNMORE=0, GMORE=1)
C
C text alignment: horizontal normal, left, center, right
      INTEGER     GAHNOR,    GALEFT,    GACENT,    GARITE
      PARAMETER  (GAHNOR=0,  GALEFT=1,  GACENT=2,  GARITE=3)
C
C text alignment: vertical  normal, top, cap, half, base, bottom
      INTEGER     GAVNOR,    GATOP,     GACAP,     GAHALF,
     *            GABASE,    GABOTT
      PARAMETER  (GAVNOR=0,  GATOP=1,   GACAP=2,   GAHALF=3,
     *            GABASE=4,  GABOTT=5)
C
C text path:    right,  left,   up,   down
      INTEGER     GRIGHT,    GLEFT,     GUP,       GDOWN
      PARAMETER  (GRIGHT=0,  GLEFT=1,   GUP=2,     GDOWN=3)
C
C text precision:     string,   character,   stroke
      INTEGER     GSTRP,     GCHARP,    GSTRKP
      PARAMETER  (GSTRP=0,   GCHARP=1,  GSTRKP=2)
C
C type of returned values:  set, realized
      INTEGER     GSET,      GREALI
      PARAMETER  (GSET=0,    GREALI=1)
C
C update state:     not-pending, pending
      INTEGER     GNPEND,    GPEND
      PARAMETER  (GNPEND=0,  GPEND=1)
C
C vector/raster/other: type vector, raster, other
      INTEGER     GVECTR,    GRASTR,    GOTHWK
      PARAMETER  (GVECTR=0,  GRASTR=1,  GOTHWK=2)
C
C visibility: invisible, visible
      INTEGER     GINVIS,    GVISI
      PARAMETER  (GINVIS=0,  GVISI=1)
C
C workstation category: Output, Input, Output+Input, Workstation
C Independent Segment Storage, Metafile Output, Metafile Input
      INTEGER     GOUTPT,     GINPUT,    GOUTIN,    GWISS,
     *            GMO,        GMI
      PARAMETER  (GOUTPT=0,   GINPUT=1,  GOUTIN=2,  GWISS=3,
     *            GMO=4,      GMI=5)
C
C workstation state:  inactive, active
      INTEGER     GINACT,    GACTIV
      PARAMETER  (GINACT=0,  GACTIV=1)
C
C list of GDP attributes:  polyline, polymarker, text, fill area
      INTEGER     GPLATT,    GPMATT,    GTXATT,    GFAATT,
     *            GEDATT
      PARAMETER  (GPLATT=0,  GPMATT=1,  GTXATT=2,  GFAATT=3,
     *            GEDATT=4)
C
C line type:    solid,   dash,   dot,   dash-dot
      INTEGER     GLSOLI,    GLDASH,    GLDOT,     GLDASD
      PARAMETER  (GLSOLI=1,  GLDASH=2,  GLDOT=3,   GLDASD=4)
C
C marker type:   '.',   '+',   '*',   'o',   'x'
      INTEGER     GPOINT,    GPLUS,     GAST,      GOMARK,
     *            GXMARK
      PARAMETER  (GPOINT=1,  GPLUS=2,   GAST=3,    GOMARK=4,
     *            GXMARK=5)
C
C For use in Inquiry Functions returning both Current & Requested Values
      INTEGER     GCURVL,    GRQSVL
      PARAMETER  (GCURVL=0,  GRQSVL=1)
C
\end{XMP}
\chapter{\protect\label{sec:gtstyp}GKSGRAL Workstation Types}
\index{GKSGRAL!workstation types}
\index{device drivers}
 
The include file \Lit{GKS$GTSDEV.INC} contains the symbolic names and codes
for the various Workstation Types available in GKSGRAL and GKSGRAL-3D.
It is reproduced below for reference. However, it should be noted that the
version currently available on a particular machine may include later
changes and corrections. To find how it should be accessed on a particular
system check the index entry for 'include files'.
\begin{XMP}
C     GKS$GTSDEV.INC
C
C     GTS-GRAL GKS and GKS-3D Workstation Types
C
C     (07.11.88, new PostScript orientation and colour)
C     (22.12.88, Add QMS Lasergrafix plotter)
C     (31.05.89, Add VT240/340, MX7000, correct HP550P4, etc.
C                Add special workstation for Pericom MX series)
C     (06.10.89, Correct VE27A3 and VE27A4)
C     (27.02.90, Add VT240M and FALCO)
C
C **  Workstation Independent Segment Storage  **
C
      INTEGER    WISS
      PARAMETER (WISS=3)
C
C **  Metafile Output/Input  **
C
C     GKS-2D: Metafile Appendix E 2D Output/Input
      INTEGER    MAE2DO
      PARAMETER (MAE2DO=4)
      INTEGER    MAE2DI
      PARAMETER (MAE2DI=5)
C
C     GKS-3D: Metafile Appendix E 3D Output/Input
C     (** May not yet be installed **)
      INTEGER    MAE3DO
      PARAMETER (MAE3DO=0)
      INTEGER    MAE3DI
      PARAMETER (MAE3DI=0)
C
C     GKS Appendix E 2D Output for GKS-3D
      INTEGER    MAEXDO
      PARAMETER (MAEXDO=10201)
C
C     Computer Graphics Metafile  !NOT YET SUPPORTED!
      INTEGER    CGMO
      PARAMETER (CGMO=0)
C
      INTEGER    CGMI
      PARAMETER (CGMI=0)
C
C **  APOLLO Workstations  **
C     N.B. The Window size may be altered by editing
C          the file: 'gks_characteristic' (GPR driver) or
C          the file: 'gks_workstations.config' (GSR driver)
C          ('GPXXX2' corresponds to the second line, etc.)
C
C     GPr interface, models DN300, DN3000, Monochrome
      INTEGER    GP30M2
      PARAMETER (GP30M2=10002)
C
C     GPr interface, model DN3000, Colour
      INTEGER    GP30C4
      PARAMETER (GP30C4=10004)
C
C     GPr interface, models DN550, DN660, Colour
      INTEGER    GP55C3
      PARAMETER (GP55C3=10003)
C
C     GSR interface, all models
C     (eight simultaneous workstations/windows are provided)
      INTEGER    GSR001
      PARAMETER (GSR001=9701)
      INTEGER    GSR002
      PARAMETER (GSR002=9702)
      INTEGER    GSR003
      PARAMETER (GSR003=9703)
      INTEGER    GSR004
      PARAMETER (GSR004=9704)
      INTEGER    GSR005
      PARAMETER (GSR005=9705)
      INTEGER    GSR006
      PARAMETER (GSR006=9706)
      INTEGER    GSR007
      PARAMETER (GSR007=9707)
      INTEGER    GSR008
      PARAMETER (GSR008=9708)
C
C  ** FALCO terminals  **
C
      INTEGER    FALCO
      PARAMETER (FALCO=7878)
C
C  ** PERICOM terminals  **
C
C     GRAFPAC    PG7800
      INTEGER    PG7800
      PARAMETER (PG7800=7878)
C
C     Monterey   MG600
      INTEGER    MG600
      PARAMETER (MG600=7800)
C
C     Monterey   MX2000
      INTEGER    MX2000
      PARAMETER (MX2000=221)
C     Monterey   MX7000
      INTEGER    MX7000
      PARAMETER (MX7000=221)
C
C     Monterey   MX8000
      INTEGER    MX8000
      PARAMETER (MX8000=227)
C
C  ** TEKTRONIX terminals  **
C
C     TEK 4010
      INTEGER    T4010
      PARAMETER (T4010=101)
C
C     TEK 4014
      INTEGER    T4014
      PARAMETER (T4014=101)
C
C     TEK 4014 with Enhanced Graphics Option
      INTEGER    T4015
      PARAMETER (T4015=103)
C
C     TEK 4107/4207
      INTEGER    T4107
      PARAMETER (T4107=121)
      INTEGER    T4207
      PARAMETER (T4207=121)
C
C     TEK 4109/4209
      INTEGER    T4109
      PARAMETER (T4109=122)
      INTEGER    T4209
      PARAMETER (T4209=122)
C
C     TEK 4111
      INTEGER    T4111
      PARAMETER (T4111=123)
C
C     TEK 4113
      INTEGER    T4113
      PARAMETER (T4113=125)
C
C     TEK 4115
      INTEGER    T4115
      PARAMETER (T4115=127)
C
C **  VAXstation 2000 and GPX  **
C     N.B. The Window size may be altered by editing
C          the file: 'window.dat'.
C          Six workstations are available.
C
C     via UIS interface
      INTEGER    VXUIS
      PARAMETER (VXUIS=8601)
      INTEGER    VXUIS1
      PARAMETER (VXUIS1=8601)
      INTEGER    VXUIS2
      PARAMETER (VXUIS2=8602)
      INTEGER    VXUIS3
      PARAMETER (VXUIS3=8603)
      INTEGER    VXUIS4
      PARAMETER (VXUIS4=8604)
      INTEGER    VXUIS5
      PARAMETER (VXUIS5=8605)
      INTEGER    VXUIS6
      PARAMETER (VXUIS6=8606)
C
C     via X-Windows interface     !NOT YET SUPPORTED!
      INTEGER    VXXW
      PARAMETER (VXXW=0)
C
C **  VT-240/340 (DEC)  **
C
      INTEGER    VT240
      PARAMETER (VT240=1020)
      INTEGER    VT240M
      PARAMETER (VT240M=1021)
      INTEGER    VT340
      PARAMETER (VT340=1030)
C
C  ** PostScript  **
C     4th character: Portrait or Landscape Orientation (P/C)
C     5th character: Colour or Monochrome (C/M)
C
      INTEGER    PSTPC
      PARAMETER (PSTPC=12201)
      INTEGER    PSTLC
      PARAMETER (PSTLC=12202)
      INTEGER    PSTPM
      PARAMETER (PSTPM=12203)
      INTEGER    PSTLM
      PARAMETER (PSTLM=12204)
C     The following two parameters are kept for compatability
      INTEGER    PSTSCR
      PARAMETER (PSTSCR=12201)
      INTEGER    PSTSCL
      PARAMETER (PSTSCL=12202)
C
C  ** QMS Lasergrafix  **
C     4th character: Portrait or Landscape Orientation (P/C)
C     5th character: Upper Box or Lower Box (U/L)
C
      INTEGER    QMSPU
      PARAMETER (QMSPU=13001)
      INTEGER    QMSLU
      PARAMETER (QMSLU=13002)
      INTEGER    QMSPL
      PARAMETER (QMSPL=13003)
      INTEGER    QMSLL
      PARAMETER (QMSLL=13004)
C
C  ** IBM PC (From VAX via Kermit)  ** !NOT SUPPORTED AT CERN!
C
      INTEGER    IBMPC
      PARAMETER (IBMPC=171)
C
C **  VERSATEC Plotters  **
C     (Available via Metafile Interpretation Service)
C
C     Versatec 11 inch roll (V80)
      INTEGER    VE80
      PARAMETER (VE80=16001)
C
C     Versatec 22 inch roll
      INTEGER    VE8222
      PARAMETER (VE8222=16002)
C
C     Versatec 36 inch roll
      INTEGER    VE8236
      PARAMETER (VE8236=16003)
C
C     Versatec A3 Colour cut sheet (VE2700)
      INTEGER    VE27A3
      PARAMETER (VE27A3=16602)
C
C     Versatec A4 Colour cut sheet (VE2700)
      INTEGER    VE27A4
      PARAMETER (VE27A4=16601)
C
C **  XEROX Laser Printers  **
C     (Available via Metafile Interpretation Service)
C
C     XEROX 87/9700 and 4050 at 300 Points Per Inch
      INTEGER    X9730
      PARAMETER (X9730=16010)
C
C     XEROX 87/9700 and 4050 at 150 Points Per Inch
      INTEGER    X9715
      PARAMETER (X9715=16011)
C
C     XEROX 2700 300 Points Per Inch
      INTEGER    X2730
      PARAMETER (X2730=16012)
C
C     XEROX 2700 150 Points Per Inch
      INTEGER    X2715
      PARAMETER (X2715=16013)
C
C     XEROX 2700  75 Points Per Inch
      INTEGER    X2775
      PARAMETER (X2775=16014)
C
C     XEROX 4045 300 Points Per Inch
      INTEGER    X4030
      PARAMETER (X4030=16015)
C
C     XEROX 4045 150 Points Per Inch
      INTEGER    X4015
      PARAMETER (X4015=16016)
C
C **  IBM APA6670 and 3812 Laser Printers  **
C     (Available via Metafile Interpretation Service)
C
      INTEGER    A6670
      PARAMETER (A6670=16017)
C
C **  HEWLETT-PACKARD Plotters  **
C     (Replace leading '7' by 'H' or 'R' for Roll Paper)
C     (Add 'L'/'P' for Landscape/Portrait)
C     (Add paper size index: A0, A1, A2, A3, A4)
C
C     Model      7475 cut paper,  Portrait,   A4
      INTEGER    H475P4
      PARAMETER (H475P4=4011)
C     Model      7475 cut paper,  Landscape,  A4
      INTEGER    H475L4
      PARAMETER (H475L4=4012)
C     Model      7475 cut paper,  Portrait,   A3
      INTEGER    H475P3
      PARAMETER (H475P3=4013)
C     Model      7475 cut paper,  Landscape,  A3
      INTEGER    H475L3
      PARAMETER (H475L3=4014)
C     Model      7550 cut paper,  Portrait,   A4
      INTEGER    H550P4
      PARAMETER (H550P4=4121)
C     Model      7550 cut paper,  Landscape,  A4
      INTEGER    H550L4
      PARAMETER (H550L4=4122)
C     Model      7550 cut paper,  Portrait,   A3
      INTEGER    H550P3
      PARAMETER (H550P3=4123)
C     Model      7550 cut paper,  Landscape,  A3
      INTEGER    H550L3
      PARAMETER (H550L3=4124)
C     Model      7585 cut paper,  Portrait,   A4
      INTEGER    H585P4
      PARAMETER (H585P4=4221)
C     Model      7585 cut paper,  Landscape,  A4
      INTEGER    H585L4
      PARAMETER (H585L4=4222)
C     Model      7585 cut paper,  Portrait,   A3
      INTEGER    H585P3
      PARAMETER (H585P3=4223)
C     Model      7585 cut paper,  Landscape,  A3
      INTEGER    H585L3
      PARAMETER (H585L3=4224)
C     Model      7585 cut paper,  Portrait,   A2
      INTEGER    H585P2
      PARAMETER (H585P2=4225)
C     Model      7585 cut paper,  Landscape,  A2
      INTEGER    H585L2
      PARAMETER (H585L2=4226)
C     Model      7585 cut paper,  Portrait,   A1
      INTEGER    H585P1
      PARAMETER (H585P1=4227)
C     Model      7585 cut paper,  Landscape,  A1
      INTEGER    H585L1
      PARAMETER (H585L1=4228)
C     Model      7585 cut paper,  Portrait,   A0
      INTEGER    H585P0
      PARAMETER (H585P0=4229)
C     Model      7585 cut paper,  Landscape,  A0
      INTEGER    H585L0
      PARAMETER (H585L0=4230)
C     Model      7586 cut paper,  Portrait,   A4
      INTEGER    H586P4
      PARAMETER (H586P4=14401)
C     Model      7586 cut paper,  Landscape,  A4
      INTEGER    H586L4
      PARAMETER (H586L4=14402)
C     Model      7586 cut paper,  Portrait,   A3
      INTEGER    H586P3
      PARAMETER (H586P3=14403)
C     Model      7586 cut paper,  Landscape,  A3
      INTEGER    H586L3
      PARAMETER (H586L3=14404)
C     Model      7586 cut paper,  Portrait,   A2
      INTEGER    H586P2
      PARAMETER (H586P2=14405)
C     Model      7586 cut paper,  Landscape,  A2
      INTEGER    H586L2
      PARAMETER (H586L2=14406)
C     Model      7586 cut paper,  Portrait,   A1
      INTEGER    H586P1
      PARAMETER (H586P1=14407)
C     Model      7586 cut paper,  Landscape,  A1
      INTEGER    H586L1
      PARAMETER (H586L1=14408)
C     Model      7586 cut paper,  Portrait,   A0
      INTEGER    H586P0
      PARAMETER (H586P0=14409)
C     Mode       7586 cut paper,  Landscape,  A0
      INTEGER    H586L0
      PARAMETER (H586L0=14410)
C     Model      7586 Roll paper, Portrait,   A4
      INTEGER    R586P4
      PARAMETER (R586P4=14411)
C     Model      7586 Roll paper, Landscape,  A4
      INTEGER    R586L4
      PARAMETER (R586L4=14412)
C     Model      7586 Roll paper, Portrait,   A3
      INTEGER    R586P3
      PARAMETER (R586P3=14413)
C     Model      7586 Roll paper, Landscape,  A3
      INTEGER    R586L3
      PARAMETER (R586L3=14414)
C     Model      7586 Roll paper, Portrait,   A2
      INTEGER    R586P2
      PARAMETER (R586P2=14415)
C     Model      7586 Roll paper, Landscape,  A2
      INTEGER    R586L2
      PARAMETER (R586L2=14416)
C     Model      7586 Roll paper, Portrait,   A1
      INTEGER    R586P1
      PARAMETER (R586P1=14417)
C     Model      7586 Roll paper, Landscape,  A1
      INTEGER    R586L1
      PARAMETER (R586L1=14418)
C     Model      7586 Roll paper, Portrait,   A0
      INTEGER    R586P0
      PARAMETER (R586P0=14419)
C     Model      7586 Roll paper, Landscape,  A0
      INTEGER    R586L0
      PARAMETER (R586L0=14420)
C
\end{XMP}
\chapter{\protect\label{sec:wdtref}GKSGRAL Workstation Description Tables}
\index{workstation description tables}
 
The following consists of abridged versions of the most commonly used
Workstation Description Tables for the GTS-GRAL workstation drivers.
A copy of the complete set is kept in the UCO, and up-to-date
machine-readable versions are available via the \Ucom{FIND GKS} command.
 
Note that the include file GTSDEV (see Appendix on Page~\pageref{sec:gtstyp})
defines mnemonic names for the workstation type codes, and users are
advised to use these rather than the integer codes themselves.
All workstations support the following software fonts,
examples of which are illustrated in \ref{fig:fonts}:
\index{text fonts}
\index{fonts}
\begin{DLtt}{123456}
\item[proportional spaced:]font -1 to -11, precision 2
\item[proportional spaced italics:]font -101 to -111, precision 2
\item[mono-spaced:]font -201 to -211, precision 2
\item[mono-spaced italics:]font -301 to -311, precision 2
\item[Greek:]font -13, -113, -213, -313
\item[solid filled font:]font -51, -151, -251, -351
\end{DLtt}
 
In addition, the CERN version of GKSGRAL has defined 24 hatch styles
which are identical on all workstations (indices -101 to -124);
see \ref{fig:hatch}.
\section{APOLLO}
\index{APOLLO}
\begin{DLtt}{123456}
\item[workstation type:]Both GPR and GSR interfaces
\item[-]GPR: 10002/10003/10004
\item[-]GSR: 9701-9708 (allows multiple workstations)
\item[-]See Appendix on Page~\pageref{sec:gtstyp}
\item[GKS Level]GPR: 2b and GSR: 2b (with sample; 2c for next release)
\item[max. display space:]0.275*0.263 metres
\item[device specific line types:]none
\item[user definable line types:]none
\item[device specific marker types:]none
\item[hardware characters:]font 1, precision 0/1, character height 7mm
\item[-]The GSR driver has access to all the hardware fonts
listed in the file \Lit{'gks_fonts.config'}. They are accessed via
font indices -1, -2...
\item[DIN 66003:]font 1, precision 2
\item[software characters:]see introduction to this Appendix
\item[CERN-defined hatch styles:]-101 to -124
\item[device specific hatch styles:]-1 to -8
\item[pattern:]none
\item[Colour Table (10001/2):]Monochrome
\item[      (10003):]256 colours, indices 240-255 fixed
\item[      (10004):]16 colours, indices 14,15 fixed
\item[Pre-defined indices (0-7):]background (White), foreground (Black), Red,
Green, Blue, Yellow, Magenta, Cyan
\item[Colour Table (97xx):]up to 256 depending on hardware configuration,
of which 16 are pre-defined.
\item[segment priority:]supported
\item[segment highlighting:]not supported
\item[locator device 1:]mouse
\item[locator Prompt/Echo 1:]tracking cross
\item[locator Prompt/Echo 2:]cross-hair
\item[locator Prompt/Echo 3:]tracking cross
\item[locator Prompt/Echo 4:]rubber band
\item[locator Prompt/Echo 5:]rectangle
\item[locator trigger:]left button
\item[locator break:]right button
\item[stroke device 1:]mouse
\item[stroke Prompt/Echo 1,3:]marker at each point
\item[stroke max. buffer size:]64
\item[stroke trigger:]middle button
\item[stroke break:]right button
\item[stroke enter new point:]left button
\item[stroke skip backward:]cursor left
\item[stroke skip forward:]->
\item[stroke delete last point:]\Lit{<BACKSPACE>}
\item[stroke delete current point:]\Lit{<DELETE>}
\item[stroke toggle insert:]\Lit{<INSERT MARK>}
\item[valuator Prompt/Echo 1,3:]digital representation
\item[valuator def. initial value:]0.0
\item[valuator def. limits:]0., 1.
\item[valuator trigger:]\Lit{<CR>}
\item[valuator break:]\Lit{<ESC>}
\item[valuator skip backward:]\Lit{<-}
\item[valuator skip forward:]->
\item[valuator delete previous character:]\Lit{<BACKSPACE>}
\item[valuator delete current character:]\Lit{<DELETE>}
\item[valuator toggle insert:]\Lit{<INSERT MARK>}
\item[choice Prompt/Echo 1,3:]menu
\item[choice trigger:]numerical character \Lit{<CR>} mouse
\item[choice break:]\Lit{<CR>}
\item[pick device:]mouse
\item[pick Prompt/Echo 1]
\item[pick trigger:]left button
\item[pick break:]right button
\item[string Prompt/Echo 1,2:]display current string
\item[string max. buffer size:]72
\item[string trigger:]\Lit{<CR>}
\item[string break:]\Lit{<ESC>}
\item[string enter new character:]alphanumeric character
\item[string skip backward:]\Lit{<-}
\item[string skip forward:]->
\item[string delete previous character:]\Lit{<BACKSPACE>}
\item[string delete current character:]\Lit{<CHAR DELETE>}
\item[string toggle insert:]\Lit{<INSERT MARK>}
\item[window size:]The size of the actual window is looked for in a descriptor
file \Lit{gks_characteristic} (for GPR) and \Lit{gks_workstations.config} (for GSR).
Private versions of these files may be installed in the user's naming
directory, otherwise default versions are taken.
 
The format of \Lit{gks_characteristic} is 5I5 and that of
\Lit{gks_workstations.config} is 7I5.
For \Lit{gks_characteristic} the first two numbers are the position
in raster units of the top left-hand corner of the window,
and the second two numbers are the width and height.
The file contains six lines corresponding to the workstation
types 10001...10006.
For \Lit{gks_workstations.config} the first number is the workstation
type, and the next four numbers are the same as above.
The seventh number is a switch (0/1) to control whether or not
the graphics windows have a border.
\end{DLtt}
\section{HP PLOTTERS}
\index{HP PLOTTERS}
\begin{DLtt}{123456}
\item[workstation type:]See Appendix on Page~\pageref{sec:gtstyp}
\item[GKS Level]2a
\item[max. display space:]see GTS-GRAL manual
\item[device specific line types:]-1 to -13
\item[user definable line types:]none
\item[line-width:]0.3 mm to ymax
\item[special marker types:]-1 to -96 except -27 (printable characters)
\item[user definable marker types:]none
\item[minimum marker size:]0.3 mm
\item[nominal marker size:]5 mm
\item[maximum marker size:]max. y dimension
\item[hardware characters:]font 1, precision 0/1, character height 1. to ymax
\item[HP fonts:]negative value of HP-font - 20
\item[DIN 66003:]font 1, precision 2
\item[software characters:]see introduction to Appendix
\item[CERN-defined hatch styles:]-101 to -124
\item[device dep. hatch styles:]-1 to -36
\item[pattern:]none
\end{DLtt}
\begin{DLtt}{123456}
\item[Connection id.]
 \begin{DLtt}{123456}
 \item[On-line:]1 - 99
 \item[Off-line:]101 - 199 (Generates a file with
the name: \Lit{PL<WKTYP>.PLT})
 \end{DLtt}
\end{DLtt}
\begin{DLtt}{123456}
\item[Plotter Settings]The device driver expects that
the plotter is switched to
EXPAND mode. In this mode only the pen can be moved as close as
possible to the physical edges of the paper. If you are using
RS232 serial interface, be sure that the HARDWARE/MONITOR switch
at the rear is set to 'HARDWARE'. If the handshake between host
and plotter does not work correctly, data can be lost.
\end{DLtt}
\section{MEGATEK 7xxx/33xx series}
\index{MEGATEK}
\begin{DLtt}{123456}
\item[workstation type:]63000; Changes have been made to the
driver so that the workstation type 63000 may be used for any Megatek
hardware configuration (except in the case where two displays are to
be run simultaneously on a single controller). This is performed by
reading information stored in files that are created by the WANDGEN
utility, and which are accessed via the logical names \Lit{WAND$CONFIG}
and \Lit{WAND$SYSTEM}. Information on these files will be found in the
WAND Installation Guide, Section 4.
\item[GKS Level]2c  (This is a 3D driver for VMS only)
\item[max. display space:]0.34*0.34 metres (4096*4096 pixels)
\item[device specific line types:]none
\item[Nominal/min/max line width:]0.5/0.5/0.5 mm
\item[GTS-GRAL specific marker types:]none
\item[Nominal/min/max marker size:]6.0/1.5/12.0 mm (8 sizes)
\item[hardware characters:]font 1, precision 0, 1
\item[hardware character heights:]8 (1.5 to 12 mm)
\item[hatch styles:]0=hollow, 1=solid; pattern and hatch not supported.
\item[pattern:]not supported
\item[number of colours:]16 simultaneous out of 4096
\item[-]Default: background, foreground, red, green, blue, yellow, magenta,
cyan, dark red, orange, brown, light green, olive, light violet, rose.
\item[segment priority:]not supported
\item[segment highlighting:]supported
\item[locator device 1, 2:]tablet puck
\item[locator device 3:]joystick
\item[-]prompt/echo 1, 2 by crosshair, 3 by cross, 4 by rubber band,
5 by rectangle.
\item[-]-11..-15 provide closed loop control of viewing transformations
(see below).
\item[locator trigger:]puck keys 1..4; joystick button
\item[locator break:]\Lit{<ESC>}
\item[stroke device 1..3:]tablet puck
\item[stroke max buffer size:]64
\item[stroke Prompt/Echo 1, 3:]marker at each point
\item[stroke trigger:]\Lit{<BLANK>}; lower puck key
\item[stroke break:]\Lit{<ESC>}
\item[stroke enter new point:]\Lit{<CR>}; upper puck key
\item[stroke skip backward:]\Lit{<CTRL>}L; left puck key
\item[stroke skip forward:]\Lit{<CTRL>}R; right puck key
\item[stroke delete last point:]\Lit{<CTRL>}H
\item[stroke delete current point:]\Lit{<DEL>}
\item[stroke toggle insert:]\Lit{<CTRL>}A
\item[valuator device 1..3:]numeric keys
\item[valuator Prompt/Echo 1,3]Digital representation
\item[valuator def. initial value:]0.5
\item[valuator def. limits:]0.0, 1.0
\item[valuator trigger:]\Lit{<CR>}
\item[valuator break:]\Lit{<ESC>}
\item[valuator skip backward:]\Lit{<CTRL>}L
\item[valuator skip forward:]\Lit{<CTRL>}R
\item[valuator delete last character:]\Lit{<CTRL>}H
\item[valuator delete current character:]\Lit{<DEL>}
\item[valuator toggle insert:]\Lit{<CTRL>}A
\item[choice device 1,2,3:]Function keys; Puck; Joystick
\item[choice Prompt/Echo 1, 3:]numbered menu (1); menu (3)
\item[choice trigger:]Function Keys; Puck keys; Joystick button
\item[choice break:]\Lit{<ESC>}
\item[pick device 1..3:]Tablet Puck
\item[pick Prompt/Echo 1:]crosshair
\item[pick trigger:]puck keys
\item[pick break:]\Lit{<ESC>}
\item[string device 1..3:]keyboard
\item[string Prompt/Echo 1:]display current string
\item[string max. buffer size:]80
\item[string trigger:]\Lit{<CR>}
\item[string break:]\Lit{<ESC>}
\item[string enter new character:]alphanumeric character
\item[string skip backward:]\Lit{<CTRL>}L
\item[string skip forward:]\Lit{<CTRL>}R
\item[string delete last character:]\Lit{<CTRL>}H
\item[string delete current character:]\Lit{<DEL>}
\item[string toggle insert:]\Lit{<CTRL>}A
\end{DLtt}
\begin{DLtt}{123456}
\item[Setup:]The hardware configuration is assumed to include a tablet and/or
joystick plus a HCRST3 3D hardware transformation unit.
Parameters may be set in the routine GKM9ZZ specifying the following
defaults: driver interface (Parallel=GX or DR11W=MG),
Raster or Stroke device, Monochrome or Colour, Display List size,
availability of 3D surface processor, screen size,
allocation of Display List memory if several screens are available.
Note that if the interface card is specified as being MG, then one
can use a GX interface simply by creating Logical Assignments
of MG0..8 to GX0..8 (or vice versa).
\end{DLtt}
\subsection{Control of Viewing Transformations}
 
It is possible to produce closed loop control by the user of the viewing
transformations without passing through the application program.
This is achieved via use of locator prompt/echo types and the use joystick
or puck as follows:
\begin{UL}
\item Locator Prompt/Echo types -11, -12, and -13 produce rotations matrices
corresponding to rotation about two axes which are applied in VRC before the
view mapping matrix. The axes are y+x (-11), x+z (-12), and y+z (-13).
\item Locator Prompt/Echo -14 enlarges or reduces ('scales' or 'zooms') the
view window in VRC symetrically by up to 10 or 1/10 times.
\item Locator Prompt/Echo -15 shifts (pans) the view window in VRC by up to
$\pm1.0$.
\end{UL}
 
To start closed loop interaction the application sets the mode of the
locator device via GSLCM and then issues a request locator (GRQLC3),
or sets the device into event mode.
Dynamic modification is then enabled by pressing the joystick button or
puck key and moving the joystick or puck, depending on which physical
device was selected. Interaction is terminated by releasing the key.
Only one viewing transformation can be modified at a time, and the
view index is specified by calling GINLC3.
The BREAK action \Lit{<ESC>} works normally, and resets the viewing transformation
to the value before closed loop control began.
As closed loop control is performed in the driver, the application program
must call GQVWR to inquire what is the current state of the
viewing transformation.
 
Closed loop control implies the following restrictions:
\begin{UL}
\item Locator devices used in this mode may only be enabled in Request and
Event mode. No entries are placed in the event queue.
\item Whilst a viewing transformation is being modified it should not be
set via GSVWR, nor should output be produced using it.
\item Whilst a viewing transformation is being modified the workstation
window and viewport should not be modified, nor should GUWK be called.
\item A viewing transformation cannot be modified by more than one
locator device at a time.
\end{UL}
\section{PERICOM MONTEREY MG600, GRAPH PAC, and FALCO}
\index{PERICOM MG series and Graph Pac}
\index{FALCO}
\index{Macintosh via Versaterm}
\index{Versaterm for Macintosh}
\begin{DLtt}{123456}
\item[workstation type:]See Appendix on Page~\pageref{sec:gtstyp}
\item[-]7800 for MG600, MG200
\item[-]FALCO for Falco terminals
\item[-]7878 for Pericom Graph Pac (old Pericom)
\item[-]For Macintosh (Versaterm) use the
workstation type for Pericom Graph Pac (old Pericom).
\item[GKS Level]2b
\item[max. display space:]0.28*0.20 metres
\item[device specific line types:]-1, to -5.
Note that -1 to -4 are software emulations of types 1 to 4.
They are slower but give better quality.
\item[user definable line types:]-1001 to -n
\item[device specific marker types:]none
\item[hardware characters:]font 1, precision 0/1, character height 4-8mm
\item[DIN 66003:]font 1, precision 2
\item[software characters:]see introduction to Appendix
\item[CERN-defined hatch styles:]-101 to -124
\item[pattern:]none
\item[number of colours:]2
\item[segment priority:]supported on MG600
\item[segment highlighting:]not supported
\item[locator device 1:]arrow keys or mouse
\item[locator Prompt/Echo 1,2:]cross-hair
\item[locator Prompt/Echo 4:]rubber band (MG600)
\item[locator trigger:]\Lit{<CR>} A-Z,a-z,0-9
\item[locator break:]\Lit{<LF>},\Lit{<CTRL>}E,\Lit{<CTRL>}e
\item[stroke device 1:]arrow keys or mouse
\item[stroke Prompt/Echo 1:]cross-hair, marker at each point
\item[stroke max. buffer size:]64
\item[stroke trigger:]\Lit{<BLANK>}
\item[stroke break:]\Lit{<LF>},\Lit{<CTRL>}E,\Lit{<CTRL>}e
\item[stroke enter new point:]\Lit{<CR>},A-Z,a-z,0-9
\item[stroke skip backward:]\Lit{<CTRL>}L,\Lit{<CTRL>}l
\item[stroke skip forward:]\Lit{<CTRL>}R,\Lit{<CTRL>}r
\item[stroke delete last point:]\Lit{<CTRL>}D,\Lit{<CTRL>}d
\item[stroke delete current point:]\Lit{<DELETE>}, \Lit{<BS>}
\item[stroke toggle insert:]\Lit{<CTRL>}I,\Lit{<CTRL>}i
\item[valuator device:]keyboard
\item[valuator Prompt/Echo 1]
\item[valuator def. initial value:]0.5
\item[valuator def. limits:]0., 1.
\item[valuator trigger:]\Lit{<CR>}
\item[valuator break:]\Lit{<LF>},\Lit{<CTRL>}E,\Lit{<CTRL>}e
\item[valuator skip backward:]\Lit{<CTRL>}L,\Lit{<CTRL>}l
\item[valuator skip forward:]\Lit{<CTRL>}R,\Lit{<CTRL>}r
\item[valuator delete last character:]\Lit{<CTRL>}D,\Lit{<CTRL>}d
\item[valuator delete current character:]\Lit{<DELETE>}, \Lit{<BS>}
\item[valuator toggle insert:]\Lit{<CTRL>}I,\Lit{<CTRL>}i
\item[choice device:]keyboard
\item[choice Prompt/Echo 1,3:]menu
\item[choice trigger:]numerical character \Lit{<CR>}
\item[choice break:]\Lit{<LF>},\Lit{<CTRL>}E,\Lit{<CTRL>}e
\item[pick device:]mouse or arrow keys
\item[pick Prompt/Echo 1: crosshair]
\item[pick trigger:]A-Z,a-z,0-9
\item[pick break:]\Lit{<LF>},\Lit{<CTRL>}E,\Lit{<CTRL>}e
\item[string device:]keyboard
\item[string Prompt/Echo 1:]display current string
\item[string max. buffer size:]72
\item[string trigger:]\Lit{<CR>}
\item[string break:]\Lit{<LF>},\Lit{<CTRL>}E,\Lit{<CTRL>}e
\item[string enter new character:]alphanumeric character
\item[string skip backward:]\Lit{<CTRL>}L,\Lit{<CTRL>}l
\item[string skip forward:]\Lit{<CTRL>}R,\Lit{<CTRL>}r
\item[string delete last character:]\Lit{<CTRL>}D,\Lit{<CTRL>}d
\item[string delete current character:]\Lit{<DELETE>}, \Lit{<BS>}
\item[string toggle insert:]\Lit{<CTRL>}I,\Lit{<CTRL>}i
\end{DLtt}
\begin{DLtt}{123456}
\item[SETUP for MGxxx]graphics directory: H) GS/CAN sets terminal+display
\item[-]graphics mode: D) graphics state III (set by GKS automatically)
\item[-]graphics general P) normal text drawing
\item[-]graphics general Q) normal block fill
\item[-]graphics general M) CR Status Term
\end{DLtt}
\begin{DLtt}{123456}
\item[SETUP for Falco]a guide is available from the UCO
\end{DLtt}
\section{POSTSCRIPT LASER PRINTERS}
\index{PostScript}
\begin{DLtt}{123456}
\item[workstation type:]See Appendix on Page~\pageref{sec:gtstyp}
\item[-]12201 Portrait -Colour
\item[-]12202 Landscape-Colour
\item[-]12203 Portrait -Monochrome
\item[-]12204 Landscape-Monochrome
\item[Connection id:]If \Lit{conid>200} then an output file will be written in
{\em Encapsulated PostScript}      format. See section on Page~\pageref{sec:epsref}.
\item[GKS Level]2a
\item[max. display space:]0.1894*0.2794 metres (portrait)
\item[max. display space:]0.2794*0.1894 metres (Landscape)
\item[device specific line types:]none
\item[user definable line types:]none
\item[line-width scale factor:]1.
\item[special marker types:]
\item[user definable marker types:]none
\item[minimum marker size:]1.0 mm
\item[nominal marker size:]2.0 mm
\item[maximum marker size:]max. y dimension
\item[hardware characters:]font 1 (Times-Roman), precision 0,1
\item[-]font -1 (Times-Roman)
\item[-]font -2 (Times-Italic)
\item[-]font -3 (Times-Bold)
\item[-]font -4 (Times-BoldItalic)
\item[-]font -5 (Helvetica)
\item[-]font -6 (Helvetica-Oblique)
\item[-]font -7 (Helvetica-Bold)
\item[-]font -8 (Helvetica-BoldOblique)
\item[-]font -9 (Courier)
\item[-]font -10 (Courier-Oblique)
\item[-]font -11 (Courier-Bold)
\item[-]font -12 (Courier-BoldOblique)
\item[-]font -13 (Greek and Symbols)
\item[DIN 66003:]font 1, precision 2
\item[software characters:]see introduction to Appendix
\item[Escape Character:]The back-slash ('\bs ') character within
a string is used as an 'escape' character with the following effect:
\begin{DLtt}{123456}
\item[\bs n]linefeed (newline)
\item[\bs r]carriage return
\item[\bs t]horizontal tab
\item[\bs b]backspace
\item[\bs f]form feed
\item[\bs \bs ]back-slash
\item[\bs (]left parenthesis
\item[\bs )]right parenthesis
\item[\bs ddd]octal character code {\it ddd}
\end{DLtt}
If the character following the back-slash is not one of the above
then it is ignored.
For more details see the PostScript Language Reference Manual.
\item[CERN-defined hatch styles:]-101 to -124
\item[device dep. hatch styles:]-1 to -13
\item[pattern:]none
\item[available colours or intensities:]16 (via shading)
\item[pre-defined colour indices:]8
\item[Connection id.]On-line:1 - 99
\item[-]Off-line:101 - 199 (Generates a file on FORTRAN logical unit
(conid - 100))
\item[VAX OPEN statement]use CARRIAGECONTROL='LIST'
\end{DLtt}
\section{QMS Lasergrapfix 800, 1200}
\index{QMS}
\begin{DLtt}{123456}
\item[workstation type:]See Appendix on Page~\pageref{sec:gtstyp}
\item[-]13001 Portrait -Upper Box
\item[-]13002 Landscape-Upper Box
\item[-]13003 Portrait -Lower Box
\item[-]13004 Landscape-Lower Box
\item[max. display space:]0.2000*0.2870 metres (13001)
\item[-]0.2870*0.2000 metres (13002)
\item[-]0.2700*0.4100 metres (13003)
\item[-]0.4100*0.2700 metres (13004)
\item[-]07874*11299 device units (13001)
\item[-]11299*07874 device units (13002)
\item[-]10629*16141 device units (13003)
\item[-]16141*10629 device units (13004)
\item[GKS Level]2a
\item[device specific line types:]-1 to -12
\item[user definable line types:]-1001 to -n
\item[line-width scale factor:]1. to 15.
\item[minimum line width factor:]0.08 mm
\item[nominal line width factor:]0.08 mm
\item[maximum line width factor:]2.62 mm
\item[device specific marker types:]not implemented
\item[GTS-GRAL specific marker types:]-101 to -114
\item[user definable marker types:]-1001 to -n
\item[minimum marker size:]1.00 mm
\item[nominal marker size:]5.00 mm
\item[maximum marker size:]287, 200, 410, 270 mm (13001,2,3,4)
\item[hardware characters:]font 1, precision 0/1 character, height 1. to 32767.
\item[DIN 66003:]font 1, precision 2
\item[software characters:]see introduction to Appendix
\item[CERN-defined hatch styles:]-101 to -124
\item[device dep. hatch styles:]none
\item[pattern:]none
\item[Connection id.]On-line:1 - 99
\item[-]Off-line:101 - 199 (Generates a file on FORTRAN logical unit
(conid - 100))
\item[Device SETUP:]see full WTD
\item[VAX OPEN statement]use CARRIAGECONTROL='NONE'
\end{DLtt}
\section{TEKTRONIX 4010, 4012, 4014 and Compatible Terminals.}
\index{TEKTRONIX 4010/12/14}
\begin{DLtt}{123456}
\item[workstation type:]See Appendix on Page~\pageref{sec:gtstyp}
\item[-]101 for Tektronix 4010, 4014
\item[-]103 for 4014 with enhanced graphics option
\item[GKS Level]2b
\item[max. display space:]0.36355*0.27359 metres
\item[device specific line types:]-5
\item[user definable line types:]-1001 to -n
\item[special marker types:]-101 to -114
\item[user definable marker types:]-1001 to -n
\item[hardware characters:]font 1, precision 0/1 character, height 4-8mm
\item[DIN 66003:]font 1, precision 2
\item[software characters:]see introduction to Appendix
\item[CERN-defined hatch styles:]-101 to -124
\item[pattern:]none
\item[locator device 1:]thumbwheels
\item[locator Prompt/Echo 1,2:]cross-hair
\item[locator trigger:]\Lit{<CR>},A-Z,a-z,0-9
\item[locator break:]\Lit{<LF>},\Lit{<CTRL>}E,\Lit{<CTRL>}e
\item[stroke device 1:]thumbwheels
\item[stroke Prompt/Echo 1:]cross-hair
\item[stroke max. buffer size:]64
\item[stroke trigger:]\Lit{<BLANK>}
\item[stroke break:]\Lit{<LF>},\Lit{<CTRL>}E,\Lit{<CTRL>}e
\item[stroke enter new point:]\Lit{<CR>},A-Z,a-z,0-9
\item[stroke skip backward:]\Lit{<CTRL>}L,\Lit{<CTRL>}l
\item[stroke skip forward:]\Lit{<CTRL>}R,\Lit{<CTRL>}r
\item[stroke delete last point:]\Lit{<CTRL>}D,\Lit{<CTRL>}d
\item[stroke delete current point:]\Lit{<DELETE>}, \Lit{<BS>}
\item[stroke toggle insert:]\Lit{<CTRL>}I,\Lit{<CTRL>}i
\item[valuator device:]keyboard
\item[valuator Prompt/Echo 1,3:]digital rep.
\item[valuator def. initial value:]0.5
\item[valuator def. limits:]0., 1.
\item[valuator trigger:]\Lit{<CR>}
\item[valuator break:]\Lit{<LF>},\Lit{<CTRL>}E,\Lit{<CTRL>}e
\item[valuator skip backward:]\Lit{<CTRL>}L,\Lit{<CTRL>}l
\item[valuator skip forward:]\Lit{<CTRL>}R,\Lit{<CTRL>}r
\item[valuator delete last character:]\Lit{<CTRL>}D,\Lit{<CTRL>}d
\item[valuator delete current character:]\Lit{<DELETE>}, \Lit{<BS>}
\item[valuator toggle insert:]\Lit{<CTRL>}I,\Lit{<CTRL>}i
\item[choice device 1:]keyboard
\item[choice Prompt/Echo 1,3]
\item[choice trigger:]numerical character, \Lit{<CR>}
\item[choice break:]\Lit{<LF>},\Lit{<CTRL>}E,\Lit{<CTRL>}e
\item[pick device 1:]thumbwheels
\item[pick Prompt/Echo 1,3]
\item[pick trigger:]A-Z,a-z,0-9
\item[pick break:]\Lit{<LF>},\Lit{<CTRL>}E,\Lit{<CTRL>}e
\item[string device:]keyboard
\item[string Prompt/Echo 1,2:]display current string
\item[string max. buffer size:]72
\item[string trigger:]\Lit{<CR>}
\item[string break:]\Lit{<LF>},\Lit{<CTRL>}E,\Lit{<CTRL>}e
\item[string enter new character:]alphanumeric character
\item[string skip backward:]\Lit{<CTRL>}L,\Lit{<CTRL>}l
\item[string skip forward:]\Lit{<CTRL>}R,\Lit{<CTRL>}r
\item[string delete last character:]\Lit{<CTRL>}D,\Lit{<CTRL>}d
\item[string delete current character:]\Lit{<DELETE>}, \Lit{<BS>}
\item[string toggle insert:]\Lit{<CTRL>}I,\Lit{<CTRL>}i
\item[Connection id.]1 - 99
\item[-]101 - 199 (Generates a file on FORTRAN logical unit
(conid - 100))
\item[VAX OPEN statement]use CARRIAGECONTROL='NONE'
\end{DLtt}
\section{TEKTRONIX 4100/4200 and PERICOM MX2000/7000/8000 series}
\index{TEKTRONIX 4100/4200 series}
\index{PERICOM MX2000/7000/8000 series}
\begin{DLtt}{123456}
\item[workstation type:]See Appendix on Page~\pageref{sec:gtstyp}
\item[-]121 for Tektronix 4107 or 4207
\item[-]122 for Tektronix 4109
\item[-]123 for Tektronix 4111
\item[-]125 for Tektronix 4113
\item[-]127 for Tektronix 4115
\item[-]221 for Pericom MX2000 or MX7000
\item[-]227 for Pericom MX8000
\item[GKS Level]2b on VM/CMS, 2c on VMS
\item[max. display space:]0.24*0.18 metres (4107, MX2000)
\item[-]0.35733*0.268 metres (4109)
\item[-]0.3854*0.2878 metres (4111)
\item[-]0.3560*0.2670 metres (4113)
\item[-]0.3430*0.2700 metres (4115)
\item[device specific line types:]-4 to -7
\item[user definable line types:]-1001 to -n
\item[device specific marker types:]-1, -6 to -10
\item[hardware characters:]font 1, precision 0/1 character, height 4-8mm
\item[DIN 66003:]font 1, precision 2
\item[software characters:]see introduction to Appendix
\item[CERN-defined hatch styles:]-101 to -124
\item[pattern:]textured -1 to -16 ]dithered -50 to -174
\item[Colours:]16 colours out of 256, 8 pre-defined
\item[Pre-defined indices (0-7):]background (Black), foreground (White), Red,
Green, Blue, Yellow, Magenta, Cyan.
\item[segment priority:]supported
\item[segment highlighting:]supported
\item[locator device 1:]joystick or arrow keys
\item[-](mouse if SETUP "mousemap yes")
\item[locator device 2, 3:]VAX only: mouse or tablet with SETUP "mousemap no"
\item[locator Prompt/Echo 1,2:]cross-hair
\item[locator trigger:]\Lit{<CR>},A-Z,a-z,0-9
\item[locator break:]\Lit{<LF>},\Lit{<CTRL>}E,\Lit{<CTRL>}e
\item[stroke device 1:]joystick or arrow keys
\item[-](mouse if SETUP "mousemap yes")
\item[stroke device 2, 3:]VAX only: mouse or tablet with SETUP "mousemap no"
\item[stroke Prompt/Echo 1:]cross-hair, marker at each point
\item[stroke max. buffer size:]64
\item[stroke trigger:]\Lit{<BLANK>}
\item[stroke break:]\Lit{<LF>},\Lit{<CTRL>}E,\Lit{<CTRL>}e
\item[stroke enter new point:]\Lit{<CR>},A-Z,a-z,0-9
\item[stroke skip backward:]\Lit{<CTRL>}L,\Lit{<CTRL>}l
\item[stroke skip forward:]\Lit{<CTRL>}R,\Lit{<CTRL>}r
\item[stroke delete last point:]\Lit{<CTRL>}D,\Lit{<CTRL>}d
\item[stroke delete current point:]\Lit{<DELETE>}, \Lit{<BS>}
\item[stroke toggle insert:]\Lit{<CTRL>}I,\Lit{<CTRL>}i
\item[valuator device:]keyboard
\item[valuator Prompt/Echo 1]
\item[valuator def. initial value:]0.5
\item[valuator def. limits:]0., 1.
\item[valuator trigger:]\Lit{<CR>}
\item[valuator break:]\Lit{<LF>},\Lit{<CTRL>}E,\Lit{<CTRL>}e
\item[valuator skip backward:]\Lit{<CTRL>}L,\Lit{<CTRL>}l
\item[valuator skip forward:]\Lit{<CTRL>}R,\Lit{<CTRL>}r
\item[valuator delete last character:]\Lit{<CTRL>}D,\Lit{<CTRL>}d
\item[valuator delete current character:]\Lit{<DELETE>}, \Lit{<BS>}
\item[valuator toggle insert:]\Lit{<CTRL>}I,\Lit{<CTRL>}i
\item[choice device 1:]keyboard
\item[choice Prompt/Echo 1,3:]menu
\item[choice trigger:]numerical character \Lit{<CR>}
\item[choice break:]\Lit{<LF>},\Lit{<CTRL>}E,\Lit{<CTRL>}e
\item[pick device 1:]joystick or arrow keys
\item[-](mouse if SETUP "mousemap yes")
\item[pick device 2, 3:]VAX only: mouse or tablet with SETUP "mousemap no"
\item[pick Prompt/Echo: 1]
\item[pick trigger:]A-Z,a-z,0-9
\item[pick break:]\Lit{<LF>},\Lit{<CTRL>}E,\Lit{<CTRL>}e
\item[string device:]keyboard
\item[string Prompt/Echo 1:]display current string
\item[string max. buffer size:]72
\item[string trigger:]\Lit{<CR>}
\item[string break:]\Lit{<LF>},\Lit{<CTRL>}E,\Lit{<CTRL>}e
\item[string enter new character:]alphanumeric character
\item[string skip backward:]\Lit{<CTRL>}L,\Lit{<CTRL>}l
\item[string skip forward:]\Lit{<CTRL>}R,\Lit{<CTRL>}r
\item[string delete last character:]\Lit{<CTRL>}D,\Lit{<CTRL>}d
\item[string delete current character:]\Lit{<DELETE>}, \Lit{<BS>}
\item[string toggle insert:]\Lit{<CTRL>}I,\Lit{<CTRL>}i
\end{DLtt}
\begin{DLtt}{123456}
\item[SETUP]factory
\item[-]baud 4800 4800
\item[-]code ansi
\item[-]tabtype mxmouse (if you have a mouse instead of a tablet)
\item[-]mousemap yes
\item[-]nvsave
\end{DLtt}
\section{VAXstation GPX}
\index{VAXstation}
\begin{DLtt}{123456}
\item[workstation type:]See Appendix on Page~\pageref{sec:gtstyp}
\item[-](6 open workstations are allowed: 8601-6)
\item[GKS Level]2c
\item[max. display space:]0.3358*0.2834 metres
\item[device specific line types:]none
\item[user definable line types:]none
\item[device specific line widths:]15
\item[device specific marker types:]none
\item[device specific marker sizes:]50
\item[hardware characters:]font 1, precision 0, character, height 4.5 to 20mm
\item[-]font -1 to -23, precision 0 correspond to UIS
fonts 1 to 26 (1,8,14 excluded)
\item[DIN 66003:]font 1, precision 2
\item[software characters:]see introduction to Appendix
\item[CERN-defined hatch styles:]-101 to -124
\item[UIS hatch styles:]-1 to -58
\item[pattern:]none
\item[Colour Table:]8 index entries
\item[Pre-defined indices (0-7):]background (Black), foreground (White), Red,
Green, Blue, Yellow, Magenta, Cyan
\item[segment priority:]supported
\item[segment highlighting:]supported (but not on monochrome screens)
\item[locator device 1:]mouse
\item[locator Prompt/Echo 1,3:]tracking cross
\item[locator trigger:]left mouse button
\item[locator break:]right mouse button
\item[stroke device 1:]mouse
\item[stroke Prompt/Echo 1,3:]cursor, marker at each point
\item[stroke max. buffer size:]64
\item[stroke trigger:]middle button
\item[stroke break:]right button
\item[stroke enter new point:]left button
\item[stroke skip backward:]cursor left \Lit{<-}
\item[stroke skip forward:]cursor right ->
\item[stroke delete last point:]\Lit{<BACKSPACE>}
\item[stroke delete current point:]. (full stop)
\item[stroke toggle insert:]\Lit{<INSERT>}
\item[valuator device:]keyboard
\item[valuator Prompt/Echo 1,3:]digital representation
\item[valuator def. initial value:]0.0
\item[valuator def. limits:]0., 1.
\item[valuator trigger:]\Lit{<CR>}
\item[valuator break:]\Lit{<ESC>}
\item[valuator skip backward:]\Lit{<-}
\item[valuator skip forward:]->
\item[valuator delete previous character:]\Lit{<BACKSPACE>}
\item[valuator delete current character:].
\item[valuator toggle insert:]\Lit{<INSERT>}
\item[choice device:]keyboard or mouse
\item[choice Prompt/Echo 1:]numbered menu
\item[choice Prompt/Echo 3:]menu
\item[choice trigger:]numerical character \Lit{<CR>}
\item[choice break:]\Lit{<ESC>}
\item[pick device:]mouse
\item[pick Prompt/Echo 1]
\item[pick trigger:]left button
\item[pick break:]right button
\item[string device:]keyboard
\item[string Prompt/Echo 1,2:]display current string
\item[string max. buffer size:]72
\item[string trigger:]\Lit{<CR>}
\item[string break:]\Lit{<ESC>}
\item[string enter new character:]alphanumeric character
\item[string skip backward:]\Lit{<-}
\item[string skip forward:]->
\item[string delete previous character:]\Lit{<BACKSPACE>}
\item[string delete current character:]\Lit{<DELETE>}
\item[string toggle insert:]\Lit{<INSERT>}
\item[Cursor visibility:]The cursor foreground {\bf must} be set to
non-black (i.e. not background) in the workstation set-up on colour
stations in order to ensure visibility inside the GKS window.
\item[window size:]The size of the actual window is looked for in a descriptor
file WINDOW.DAT under the current working directory or, if the file is
not found, in the directory defined by the logical name \Lit{GKS_WINDOW}.
 
The format of the file is 4F5.3, A20 as follows:
\begin{OL}
\item window size X in metres
\item window size Y in metres
\item window position X on screen in metres
\item window position Y on screen in metres
\item window title
\end{OL}
For example: \Lit{0.15 0.15 0.20 0.15 GKS_WINDOW_1}.
The first line in the file corresponds to workstation type 8601,
the second to 8602 etc.
 
See also the utility routines GCCDWN and GCSDWN for information on
how to set the window size and title from within an application
program (on Page~\pageref{sec:vstnref}).
\end{DLtt}
\section{VERSATEC}
\index{VERSATEC}
\begin{DLtt}{123456}
\item[workstation type:]See Appendix on Page~\pageref{sec:gtstyp}
\item[GKS Level]2a
\item[max. display space:]5.0*0.265 metres (V80)
\item[-]5.0*0.535 metres (8222)
\item[-]5.0*0.580 metres (8224) !Not available at CERN!
\item[-]5.0*0.893 metres (8236)
\item[-]5.0*1.030 metres (8242) !Not available at CERN!
\item[-]0.24*0.20 metres (Versacolor)
\item[device specific line types:]none
\item[user definable line types:]none
\item[line-width scale factor:]1. to 5.
\item[special marker types:]-1 to -9
\item[user definable marker types:]none
\item[minimum marker size:]0.127 mm
\item[nominal marker size:]2 mm
\item[maximum marker size:]max. y dimension
\item[hardware characters:]font 1, precision 0/1 character, height 1. to ymax
\item[-]No lower case hardware characters
\item[DIN 66003:]font 1, precision 2
\item[software characters:]see introduction to Appendix
\item[CERN-defined hatch styles:]-101 to -124
\item[device dep. hatch styles:]-1 to -13
\item[pattern:]none
\item[colours (Versacolor):]8 line colours
\item[-]256 solid fill area colours (dithering)
\end{DLtt}
\section{VT240}
\index{VT240/241}
\begin{DLtt}{123456}
\item[workstation type:]1020, (1021 for Monochrome)
see Appendix on Page~\pageref{sec:gtstyp}
\item[GKS Level]2b (VMS only)
\item[max. display space:]0.218*0.138 metres (800*240 pix)
\item[device specific line types:]-4, -5, -7, -8
\item[user definable line types:]-1001 to -n
\item[Nominal/min/max line width:]0.3/0.3/0.3 mm
\item[GTS-GRAL specific marker types:]-101 to -114
\item[Nominal/min/max marker size:]5.0/0.3/138.0 mm (240 sizes)
\item[hardware characters:]font 1, precision 0/1
\item[hardware character heights:]3 (3 to 9 mm)
\item[DIN 66003:]font 1, precision 2
\item[software characters:]see introduction to Appendix
\item[CERN-defined hatch styles:]-101 to -124
\item[pattern:]not supported
\item[number of colours:]4 simultaneous out of 64
\item[-]Default: background (Black), foreground (White), Red, Green
\item[segment priority:]supported
\item[segment highlighting:]not supported
\item[locator device 1:]arrow keys
\item[-]prompt/echo 1 and 2 by crosshair
\item[locator device 2 and 3:]arrow keys
\item[-]prompt/echo 4/5 by rubber band/rectangle
\item[locator trigger:]\Lit{<CR>}, \Lit{<LF>}, \Lit{<ENTER>}
\item[locator break:]\Lit{<ESC>}, x, X
\item[stroke device 1, 2, 3:]arrow keys
\item[stroke max buffer size:]64
\item[stroke Prompt/Echo 1:]marker at each point
\item[stroke trigger:]\Lit{<BLANK>}
\item[stroke break:]\Lit{<ESC>}, x, X
\item[stroke enter new point:]\Lit{<CR>}, O, \Lit{<LF>}
\item[stroke skip backward:]\Lit{<} or g
\item[stroke skip forward:]> or .
\item[stroke delete last point:]\Lit{<BS>}, \Lit{<DEL>}, b, B
\item[stroke delete current point:]d or D
\item[stroke toggle insert:]i, I
\item[valuator device:]keyboard (0-9, +, -, ., E, e)
\item[valuator Prompt/Echo 1,2,3]Digital representation
\item[valuator def. initial value:]0.0
\item[valuator def. limits:]0.0, 1.0
\item[valuator trigger:]\Lit{<CR>}, \Lit{<LF>}, \Lit{<ENTER>}
\item[valuator break:]\Lit{<ESC>}, keypad -
\item[valuator skip backward:]cursor keys, keypad 4
\item[valuator skip forward:]cursor keys, keypad 6
\item[valuator delete last character:]\Lit{<BS>}, \Lit{<DEL>}
\item[valuator delete current character:]keypad .
\item[valuator toggle insert:]\Lit{<PF4>}, keypad 0
\item[choice device 1:]keyboard
\item[choice Prompt/Echo 1,3:]menu
\item[choice trigger:]numerical character \Lit{<CR>}
\item[choice break:]\Lit{<ESC>}, keypad -
\item[pick device 1, 2, 3:]arrow keys
\item[pick Prompt/Echo 1:]crosshair
\item[pick trigger:]\Lit{<CR>}, \Lit{<LF>}, O
\item[pick break:]\Lit{<ESC>}, x, X
\item[string device 1:]keyboard
\item[string Prompt/Echo 1:]display current string
\item[string max. buffer size:]72
\item[string trigger:]\Lit{<CR>}, \Lit{<LF>}, \Lit{<ENTER>}
\item[string break:]\Lit{<ESC>}, keypad -
\item[string enter new character:]alphanumeric character
\item[string skip backward:]cursor key, keypad 4
\item[string skip forward:]cursor key, keypad 6
\item[string delete last character:]\Lit{<BS>}, \Lit{<DEL>}
\item[string delete current character:]keypad .
\item[string toggle insert:]PF4, keypad 0
\end{DLtt}
\begin{DLtt}{123456}
\item[SETUP]The routine GK10AZ may be used to define the default
configuration.
\item[Current parameters are set to:]
\item[Local echo:]off
\item[Keypad mode:]application keypad active on close workstation
\item[Break:]100 milli-seconds
\item[Echo text for Message and Choice:]ANSI text type (not Graphic)
\end{DLtt}
\section{VT340}
\index{VT340}
\begin{DLtt}{123456}
\item[workstation type:]1030; see Appendix on Page~\pageref{sec:gtstyp}
\item[GKS Level]2b (VMS only)
\item[max. display space:]0.240*0.145 metres (800*480 pix)
\item[device specific line types:]-1 to -5
\item[user definable line types:]-1001 to -n
\item[Nominal/min/max line width:]0.3/0.3/0.3 mm
\item[GTS-GRAL specific marker types:]-101 to -114
\item[Nominal/min/max marker size:]5.0/0.3/145.0 mm (480 sizes)
\item[hardware characters:]font 1, precision 0/1
\item[hardware character heights:]17 (3 to 72.5 mm)
\item[DIN 66003:]font 1, precision 2
\item[software characters:]see introduction to Appendix
\item[CERN-defined hatch styles:]-101 to -124
\item[pattern:]not supported
\item[number of colours:]16 simultaneous out of 4096
\item[-]Default: background (black), foreground (white), red, green, blue,
yellow, magenta, cyan, green, blue, yellow, magenta, cyan, foreground, red.
\item[segment priority:]supported
\item[segment highlighting:]not supported
\item[locator device 1:]mouse, keys A1..A4
\item[locator device 2, 3:]keys A1..A4
\item[-]prompt/echo 1, 2 by crosshair, 3 by cross, 4 by rubber band,
5 by rectangle
\item[locator trigger:]\Lit{<CR>}, \Lit{<LF>}, \Lit{<ENTER>}; Mouse B1
\item[locator break:]\Lit{<ESC>}, x, X; keypad -;Mouse B3
\item[stroke device 1:]keys A1..A4; mouse
\item[stroke device 2, 3:]keys A1..A4
\item[stroke max buffer size:]64
\item[stroke Prompt/Echo 1, 3:]marker at each point
\item[stroke trigger:]\Lit{<BLANK>}; mouse B2
\item[stroke break:]\Lit{<ESC>}, x, X; mouse B3 (2,3 also keypad -)
\item[stroke enter new point:]\Lit{<CR>}, O, \Lit{<LF>}; mouse B1
\item[stroke skip backward:]\Lit{<} or ,
\item[stroke skip forward:]> or .
\item[stroke delete last point:]\Lit{<BS>}, \Lit{<DEL>}, (1 also b, B)
\item[stroke delete current point:]d or D (2,3 also keypad .)
\item[stroke toggle insert:]i, I (2,3 also keypad 0)
\item[valuator device 1, 2, 3:]numeric keys
\item[valuator Prompt/Echo 1,2,3]Digital representation
\item[valuator def. initial value:]0.5
\item[valuator def. limits:]0.0, 1.0
\item[valuator trigger:]\Lit{<CR>}, \Lit{<LF>}, \Lit{<ENTER>}
\item[valuator break:]\Lit{<ESC>}, keypad -
\item[valuator skip backward:]A3, keypad 4
\item[valuator skip forward:]A4, keypad 6
\item[valuator delete last character:]\Lit{<BS>}, \Lit{<DEL>}
\item[valuator delete current character:]keypad .
\item[valuator toggle insert:]\Lit{<PF4>}, keypad 0
\item[choice device 1, 2:]keyboard
\item[choice device 3:]mouse
\item[choice Prompt/Echo 1, 3:]numbered menu (1); menu (3)
\item[choice trigger:]numerical character; mouse B1
\item[choice break:]\Lit{<ESC>}, keypad -; mouse B3
\item[pick device 1:]keys A1..A4; mouse
\item[pick device 2, 3:]keys A1..A4
\item[pick Prompt/Echo 1:]crosshair
\item[pick trigger:]\Lit{<CR>}, \Lit{<LF>}, O; mouse B1 (2, 3 \Lit{<ENTER>})
\item[pick break:]\Lit{<ESC>}, x, X; mouse B3 (2, 3 keypad -)
\item[string device 1:]keyboard
\item[string Prompt/Echo 1:]display current string
\item[string max. buffer size:]72
\item[string trigger:]\Lit{<CR>}, \Lit{<LF>}, \Lit{<ENTER>}
\item[string break:]\Lit{<ESC>}, keypad -
\item[string enter new character:]alphanumeric character
\item[string skip backward:]cursor key, keypad 4
\item[string skip forward:]cursor key, keypad 6
\item[string delete last character:]\Lit{<BS>}, \Lit{<DEL>}
\item[string delete current character:]keypad .
\item[string toggle insert:]PF4, keypad 0
\end{DLtt}
\begin{DLtt}{123456}
\item[SETUP]The routine GK06AZ may be used to define the default
configuration.
\item[Current parameters are set to:]
\item[Local echo]off
\item[Keypad mode:]application keypad active on close workstation
\item[Break:]100 milli-seconds
\item[Echo text for Message and Choice:]ANSI text type (not Graphic)
\end{DLtt}
If arrow keypad is not available then keys A1..A4 are defined as:
\begin{DLtt}{123456}
\item[A1:]Keypad 8
\item[A2:]Keypad 2
\item[A3:]Keypad 4
\item[A4:]Keypad 6
\end{DLtt}
\section{XEROX, APA 6670 and 3812 LASER PRINTERS}
\index{XEROX}
\index{APA 6670}
\index{IBM!3812}
\begin{DLtt}{123456}
\item[workstation type:]only accessible via metafile interpreters
\item[GKS Level]2a
\item[max. display space:]
\item[-]0.2073*0.2946 metres (XEROX 8700 300 ppi)
\item[-]0.2059*0.2926 metres (XEROX 8700 150 ppi)
\item[-]0.0840*0.1192 metres (XEROX 2700 300 ppi)
\item[-]0.1504*0.2127 metres (XEROX 2700 150 ppi)
\item[-]0.2032*0.2899 metres (XEROX 2700  75 ppi)
\item[-]0.1295*0.1822 metres (XEROX 4045 300 ppi)
\item[-]0.2059*0.2926 metres (XEROX 4045 150 ppi)
\item[-]0.1465*0.2142 metres (APA 6670 and IBM 3812)
\item[-]ppi = pixels per inch
\item[device specific line types:]none
\item[user definable line types:]none
\item[line-width scale factor:]1. to 5.
\item[special marker types:]none
\item[user definable marker types:]none
\item[minimum marker size:]0.127 mm
\item[nominal marker size:]5 mm
\item[maximum marker size:]max. y dimension
\item[hardware characters:]none
\item[DIN 66003:]font 1, precision 2
\item[software characters:]see introduction to Appendix
\item[CERN-defined hatch styles:]-101 to -124
\item[device dep. hatch styles:]none
\item[pattern:]none
\end{DLtt}
\chapter{\protect\label{sec:instref}GKSGRAL Software Installation}
\Lit{$==>$} Uptodate ???
\index{GKSGRAL!installation}
\index{installation}
 
This appendix contains information on the installation of GKSGRAL on
machines not supported by the CERN computer centre.
Further details may be found in the installation guide for each
system available in the full GKSGRAL manual \cite{bib-gtsref}.
\begin{note}
GKSGRAL and GKSGRAL-3D are proprietary software products for which CERN
holds a site licence. Thus, they may be installed and used at CERN
on any machines running one of the following systems:
VM/CMS, UNIX, UNICOS, VMS, AEGIS, SINTRAN-3.
Information on how to obtain a licence for use outside CERN may be
found in Appendix on Page~\pageref{sec:gtsdist}.
 
{\bf It is illegal to take copies of the software off the CERN site.}
\end{note}
\section{Library Configuration}
 
Although most users will simply link to one or other of the GTS-GRAL
libraries, it is perhaps as well to mention that there is a fairly
straight-forward procedure to re-configure these libraries to add
or omit drivers. In both GKSGRAL and GKSGRAL-3D all driver code
is accessed via the routine GKDDLK. Thus, in order to change the configuration,
it is only necessary to edit and re-compile this routine. Of course,
the corresponding driver routines must be added to/deleted from the library.
Where access to GKDDLK is not available, it is possible to use the
non-shared library version and to link to dummy versions of the
un-used drivers.
 
Some of the drivers may make use of low-level proprietary software,
such as VERSAPLOT,
which cannot be included in the library as a separate licence is required.
Installations which configure drivers of this type, but which do not
have the low level software, may wish to provide a dummy library in order to
avoid a long list of unresolved external references.
\section{VAX Installation}
\index{VAX!GKSGRAL distribution}
VAX implementations can be distributed within CERN without going through the
Program Library if the target machine is connected to DECNET, and the following
automated procedures have been written to carry out the installation:
\begin{XMP}
VXCERN::CERN:[GKS.PRO.MGR]INSTALL_GKS.COM
VXCERN::CERN:[GKS.PRO.MGR]INSTALL_GKS_GPX.COM
VXCERN::CERN:[GKS.PRO.MGR]INSTALL_GKS_MEGATEK.COM
VXCERN::CERN:[GKS.PRO.MGR]INSTALL_GKS_SRC.COM
VXCERN::CERN:[GKS.PRO.MGR]INSTALL_GR.COM
\end{XMP}
The first one should be used for VAXes which support time-sharing graphics
terminals, and the second for VAXstations booted with the UIS
graphics interface.
\Lit{INSTALL_GKS_MEGATEK.COM}
 deals with the special case of MEGATEK support,
\Lit{INSTALL_GKS_SRC.COM}
may be used to copy the source code for particular drivers,
and \Lit{INSTALL_GR.COM}
should be used to install the utilities GRVIEW, GRCONV and GRPLOT.
 
In all cases it is necessary to answer various questions during the
installation. For example, these include the name of the root directory
on the target node in which GKS is to be stored and the version (OLD, PRO,
or NEW) which is to be copied. To make GKS-related commands and Logical
Names known at the system level system privileges are required to run
the procedures. Further command files will be made available in the future
to install various graphics utilities individually.
\section{IBM Installation}
\index{IBM!GKSGRAL distribution}
Distribution for IBM VM/CMS is done on magnetic tape in CMS TAPE DUMP
format and can be read using the REXX exec:
\begin{XMP}
 /* REXX */
 /*BATCH TAPES 1 */
 /*BATCH PUN 400 */
 /*BATCH TIME 4  */
 TRACE C
 'TAPE LOAD * * A ( EOT  '
\end{XMP}
The first file on the tape, \Lit{'README INFO'}, contains information on
the tape's contents, as well as installation execs and a guide to modifying
drivers. New CERN products and example programs are continually being added
as they are developed.
\section{\protect\label{sec:unixdis}UNIX Installation}
\index{Unix!GKSGRAL installation}
\subsection{\protect\label{sec:unixdna}Sites with hardware for which object code is not available.}
 
The following describes the first version of the installation procedure
for sites starting from the GKS source code.
This should be the case only for
those institutes which are the first to order a licence to run
the GTS-GRAL software under UNIX on a machine type which is not
available at CERN, and for which no other institute has yet installed
the software.
 
For CERN-supported systems other than UNIX, affiliated institutes
are able legally to receive only the GTS-GRAL kernel and driver libraries,
and the driver source, but NOT the kernel source.
This requirement has been relaxed for UNIX in order to support machines
not available at CERN, but in this case the institute must sign a source
licence with GTS-GRAL, for which there will be no charge.
(If compiled libraries are already available,
then the distribution is as described in section on Page~\pageref{sec:objcode}.)
 
Once the software has been delivered and compiled, the institute
is requested to act as a subsidiary distribution centre to CERN,
so that the libraries may be made available to institutes
which later request support for the same machine type.
However, they should only distribute libraries if authorized to do so.
CERN can make available example scripts
to be used in order to build a tar distribution file.
 
\index{Unix!tar file}
\index{tar file}
The source code is made available in a tree stored in a tar file
called \Lit{unix_src.tar}, which can be distributed either on tape or via
a network (eg via FTP).
Later, there may be a distinction made between the OLD, PROduction,
and NEW versions.
 
The first thing to do is to create a root directory,
for example, \Lit{/user/gts-root}, and set this as current.
The tar file should then be copied to this root from
tape or via a network, and then be unpacked:

\begin{XMP} 
mkdir /user/gts-root
(cp-or-ftp-or-mv  unix_src.tar /user/gts-root/unix_src.tar)
cd    /user/gts-root
tar   -xf unix_src.tar .
\end{XMP}
 
So far the development has been done at CERN on an Apollo DN3000
under SR~10.2 using SYSTYPE 'sys5.3', and the tar file has been moved
to CRAY X/MP and Y/MP machines, SEL Gould, and a DECstation RISC under Ultrix.
In all cases both GKS and GKS-3D have been compiled successfully.
The drivers tested have been the ones for PostScript, the 2D metafile driver,
the Apollo GPR and GSR drivers, and the Tektronix 4107 (on the DECstation).
 
Any problems and/or modifications made to code or scripts should be
carefully commented and sent to \Lit{GRAPHICS@CERNVM.CERN.CH} in order that
they may be incorporated in later versions.
\subsubsection{\protect\label{sec:tarref}Tar file contents.}
\index{Unix!tar file}
\index{tar file}
 
The root contains the directories bin, dmo, doc, gks, gks3d, and utl,
plus the files:
\begin{DLtt}{123456}
\item[.login]C shell login script
\item[.cshrc]C shell new process script
\end{DLtt}
bin contains command files needed to compile the libraries, etc.
All scripts starting with \Lit{'c_'} were written at CERN,
and most of the others have been modified:
\begin{DLtt}{123456}
\item[startgks]Script to define symbolic names (Bourne or C shell)
\item[startgks.apollo]Apollo development version of startgks
\item[startgks.cray]CRAY version of startgks
\item[startgks.ux]Ultrix version of startgks
\item[c\_system\_rebuild]Re-build GKS, GKS-3D, and driver libraries
\item[c\_gkslib\_create]Create GKS library
\item[c\_gksdriv\_create]Create Driver library
\item[c\_gkslib3d\_create]Create GKS-3D library
\item[compcc]Compile all C routines (\Lit{*.c}) in a directory
and insert them into a library.
(Uses the symbol \Lit{'$ccomp'})
\item[compf]Compile all Fortran routines (\Lit{*.f}) in a directory
and insert them into a library.
(Uses the symbol \Lit{'$fortran'})
\item[c\_fort]Apollo only; this uses the Aegis Fortran compiler
\item[c\_fix\_include]Utility to convert format of INCLUDE statement
\item[c\_gks\_comp\_single]Compile routine and replace in a library
\item[c\_gkz\_compile]Utility to test compile gkz routines only
\item[lkgks2d]Example script to link a GKS application
\item[lkgks3d]Example script to link a GKS-3D application
\item[rencmn]Utility to convert .cmn file names to upper case
\item[reninc]Utility to convert .inc file names to upper case
\item[renfor]Utility to convert .for file names .f file names
\item[renftn]Utility to convert .ftn file names .f file names
\end{DLtt}
doc contains the files:
\begin{DLtt}{123456}
\item[help.unix]UNIX help file
\item[help.changes]The list of updates made to date
\item[help.driver]Some notes on how to write a driver
\item[help.errors]List of GKS Error codes
\end{DLtt}
gks contains:
\begin{DLtt}{123456}
\item[demo]GTS-GRAL demo programs
\item[drivers]driver source
\item[kernel]kernel source with CERN mods
\item[kernel.original]original of routines modified by CERN
\item[fonts]data files for software fonts
\item[libs](empty) directory for libraries
\end{DLtt}
gks3d contains:
\begin{DLtt}{123456}
\item[demo]GTS-GRAL demo programs
\item[kernel]kernel source with CERN mods
\item[kernel.original]original of routines modified by CERN
\item[exe]data files for demos
\item[libs](empty) directory for libraries
\end{DLtt}
gks/drivers contains:
\begin{DLtt}{123456}
\item[driver.list]file defining which drivers to compile
\item[gkx]gkx utility routines with CERN mods
\item[gkx.original]original of routines modified by CERN
\item[gkz]copy of relevant gkz.xxx directory
\item[gkz.standard]gkz machine-dependent routines with CERN mods
\item[gkz.original]original of routines modified by CERN
\item[gkz.apollo]gkz routines for Apollo
\item[gkz.cray]gkz routines for CRAY
\item[metafile]2D metafile driver for GKS-3D
\item[tek4014]Tektronix 4014 driver
\item[tek4107]Tektronix 4107 driver
\item[vev9c]Versatec driver
\item[xxpscrip]PostScript driver
\end{DLtt}
utl contains the files:
\begin{DLtt}{123456}
\item[gks\_enum]Include file for GKS enumerated types
\item[gks\_gtsdev]Include file for GTS-GRAL workstation type codes
\item[gks\_implem.gts]data file used by various GKSPACK routines
\item[gkspack.build]script to unpack gkspack.car
\item[gkspack.car]GKSPACK 'CAR' file
\item[gkspack.f]GKSPACK Fortran source for UNIX systems
\item[gkspack\_test.f]Simple GKSPACK test program
\item[grconv]Script used for metafile conversion
\item[gredit.f]Source of editor part of GRVIEW program
\item[grview]Script used for metafile viewing and editing
\item[grview.comp]Script used to build GRVIEW
\item[grview.f]Source of GRVIEW program
\end{DLtt}
\subsubsection{\protect\label{sec:proc}Procedure}
\begin{OL}
\item
Edit the .login and .cshrc to set the local value for \Lit{$gkshome}, etc.
if the C shell is being used, and make sure they are executed.
(If using the Bourne or Korn shells, then modify the files
as necessary.)
\item
Rename the most appropriate file to \Lit{$gkshome/bin/startgks} and edit
it to set up local conditions. Then execute it,
for example by typing \Lit{'source startgks'} under the C shell.
The points to watch for include:
\begin{UL}
\item Use of Bourne Shell or C Shell
\item Use of UNIX System V or BSD 4.2,
\item Local home directory definition (if not set in .login file)
\item Local Fortran and C command syntax
\item Local librarian. ie bld not ar, or special ar syntax.
Note that if ar must be used under BSD without ranlib, then the use of
ranlib must be uncommented in the files \Lit{c_gkslib_create}, 
\Lit{c_gksdriv_create}, and \Lit{c_gkslib3d_create}.
\item Ensure definitions of \Lit{gkschXX} are correct.
\end{UL}
\item If necessary (eg under Ultrix, SEL-Gould, or perhaps other BSD systems)
edit the files \Lit{c_gks*create} and \Lit{c_gks_comp_single} in order to remove the
comment character in front of the ranlib command.
\item Process all Fortran files to change the format of the INCLUDE
statement to whatever works locally. (They are delivered using
VMS format.)
This may be done with the script \Lit{$gkstools/c_fix_include} after setting
the target working directory as current. Modify \Lit{c_fix_include}
as necessary.
\item Replace the files in \Lit{$gkshome/gks/drivers/gkz} with the set most
appropriate to the local system. It may be necessary to edit
all the .c files in order to change the subroutine names to/from
upper/lower case with/without a trailing underscore, etc.
There exist already directories \Lit{gkz.cray}, 
\Lit{gkz.apollo} and \Lit{gkz.standard}
all under \Lit{$gkshome/gks/drivers} and, as delivered, gkz contains the
files from \Lit{gkz.standard}. Make sure that you start with the most
appropriate set of gkz files. Copy the tree using (eg) the command:
\begin{XMP}
$gkstools/c_cpt $gkshome/gks/drivers/gkz.cray $gkshome/gks/drivers/gkz
\end{XMP}
Note that C routines for different machines have the following
characteristics:
\begin{DLtt}{123456}
\item[STANDARD]global symbolic names are lower case with trailing underscore.
\item[CRAY]global symbolic names are upper case with no underscore.
\item[APOLLO]global symbolic names are lower case with no underscore
if Aegis Fortran compiler is used. (This is the CERN recommendation,
and is backwards compatible from SR~10 to SR~9.7)
\end{DLtt}
\item
Edit the file \Lit{$gksdrivlist} to specify the drivers you wish to include.
The file has one line per driver, with the format \Lit{'dn name'}.
The first character is a literal 'd' and is followed by an integer
which numbers the lines. The second argument is the name of the
sub-directory below \Lit{$gkshome/gks/drivers} containing the driver which
is to be added.
\item
Edit the routines \Lit{$gkshome/gks/kernel/gkddlk.f} and
\Lit{$gkshome/gks3d/kernel/gkddlk.f} (they are not identical) to ensure
that the drivers included in \Lit{$gksdrivlist} are known to GKS and GKS-3D.
\item
Execute \Lit{$gkstools/c_system_rebuild} (and cross your fingers).
\item
Check that the environment variable \Lit{GKS_FONTS} is defined correctly
and is available to all users.
\end{OL}
\subsubsection{Problem areas}
 
There may be problems in the following areas:
\begin{OL}
\item Although most compilers accept the INCLUDE statement, this is not
defined in the Fortran~77
standard, and so its syntax may be vary on different machines.
\item The format of External Global Symbol Names is not defined anywhere
to be a standard, even though most UNIX systems use lower case symbol names
with a trailing underscore. This is crucial for the linking of routines
written in different languages. (GKSGRAL uses the languages Fortran *and* C.)
\item Machine dependencies in some of the gkz routines. GTS-GRAL have
tried to keep all system dependencies within these routines.
If UNIX were really a standard then there should be
nothing to change. Nevertheless, there are variations,
(eg gkzof for opening files, etc).
\item Special features of local compilers.
\item Some BSD implementations seem to use ar without ranlib.
In this case use of ranlib must be uncommented in the files \Lit{c_gkslib_create},
\Lit{c_gksdriv_create} and \Lit{c_gkslib3d_create}.
\end{OL}
\subsection{\protect\label{sec:objcode}Sites with hardware for which object code is available.}
 
If compiled libraries are already available,
then the distribution is made via a tar file called \Lit{unix_lib.tar}.
As this file contains binary libraries
it must be compatible with the target machine.
It may also be distributed by an institute affiliated to CERN,
rather than by the CERN Program Library, if the target machine
is not available on the CERN site.
 
The first thing to do is to create a root directory,
for example, /user/gts-root, and set this as current.
The tar file should then be copied to this root from
tape or via a network, and then be unpacked:

\begin{XMP} 
mkdir /user/gts-root
(cp-or-ftp-or-mv  unix_lib.tar /user/gts-root/unix_lib.tar)
cd    /user/gts-root
tar   -xf unix_lib.tar .
\end{XMP}
 
If the driver library contains the required drivers, then the only
thing that remains to be done is to define the environment variables
gkshome, gkslib, gksdriv, gks3dlib, \Lit{GKS_FONTS}, etc.
This is best done by modifying the file startgks, as described above.
If there is a requirement to use a driver which is not included
in the the driver library, then the mechanism to add a driver
is described below.
\subsubsection{Tar file contents.}
\index{Unix!tar file}
\index{tar file}
 
The tar file contains the same directory structure described
in section on Page~\pageref{sec:tarref}, but the contents are modified slightly:
\begin{UL}
\item
bin may not contain those scripts used to compile the kernel.
\item
gks/kernel contains only gkddlk.f
\item
gks/libs contains the kernel and driver libraries
\item
gks3d/kernel contains only gkddlk.f
\item
gks3d/libs contains the 3D kernel library
\end{UL}
\subsubsection{Procedure to add drivers.}
 
To add drivers to the driver library one follows a procedure
similar to that given in section on Page~\pageref{sec:proc}
which describes how to build the libraries from a source distribution.
\begin{OL}
\item
Edit the file \Lit{$gksdrivlist} to specify the drivers you wish to include.
The file has one line per driver, with the format \Lit{'dn name'}.
The first character is a literal 'd' and is followed by an integer
which numbers the lines. The second argument is the name of the
sub-directory below \Lit{$gkshome/gks/drivers} containing the driver which
is to be added.
\item
Execute \Lit{$gkstools/c_gksdriv_create}, which will compile the whole
driver library. Alternatively, one can simply add the driver
to the existing library via the commands:
\begin{XMP}
cd        $gkshome/gks/drivers/new_driver
$compile  $gksdriv
$ccompile $gksdriv
\end{XMP}
\item
Edit the routines \Lit{$gkshome/gks/kernel/gkddlk.f} and
\Lit{$gkshome/gks3d/kernel/gkddlk.f} (they are not identical) to ensure
that the drivers compiled above are known to GKS and GKS-3D.
The routines must then be compiled and replaced in their
respective libraries using the tool \Lit{$gkstools/c_gks_comp_single}:
\begin{XMP}
c_gks_comp_single $gkshome/gks/kernel/gkddlk.f   $gkslib
c_gks_comp_single $gkshome/gks3d/kernel/gkddlk.f $gks3dlib
\end{XMP}
\end{OL}
 
Those people who wish to write new drivers are basically on their own.
They are advised to start from the source of an existing driver
with features similar to the one they wish to write. However,
a few points of interest are to be found in the file 'help.drivers'.
\section{\protect\label{sec:apodis}Apollo Installation}
\index{Apollo!GKSGRAL installation}
 
No installation procedure is necessary for GKSGRAL or GKSGRAL-3D
on APOLLO nodes connected to the standard DOMAIN network at CERN.
For nodes on other DOMAINs the procedure is to obtain
a cassette from the Program Library Office which contains
the following trees written with wbak:

\begin{XMP}
/cern/gks/pro
/cern/gks/new
/cern/gks/distribution
\end{XMP}

They can be restored with the command:

\begin{XMP}
rbak -dev ct -f 1 -int -ld -all -as /cern/gks/pro
rbak -dev ct -f 2 -int -ld -all -as /cern/gks/new
rbak -dev ct -f 3 -int -ld -all -as /user/gts-root
\end{XMP}
 
For those institutes that wish to use the GKS libraries without change,
it should be sufficient simply to restore either /cern/gks/pro or
/cern/gks/new/. For those institutes which want to add drivers, or get
access to the machine readable documentation, etc., it will be necessary to
restore /cern/gks/distribution. This is equivalent to the root
/user/gts-root referred to in the section on Page~\pageref{sec:unixdna},
and so should preferably be restored with \Lit{'-as&nbsp/user/gts-root'}.
\chapter{\protect\label{sec:gtsdist}Distribution of GKSGRAL to Affiliated Institutes}
\Lit{$==>$} Uptodate ???
\index{distribution}
\index{GKSGRAL!distribution}
 
As part of the agreement signed between CERN and GTS-GRAL,
GTS-GRAL contracted to allow CERN to distribute the GKSGRAL and GKSGRAL-3D
packages to affiliated institutes (in Europe and worldwide)
on payment by the institute of a nominal embedded-licence fee.
This licence allows the software to be used
{\bf only in experiments conducted in collaboration with CERN}.
This means that all members of a CERN collaboration should be able
to use a common implementation of GKS/GKS-3D.
 
Since signing the original contract, GTS-GRAL have agreed to
extend the scope of the embedded licence to include the use of the
software for {\it any} HEP experiments, not only those performed
at CERN, on payment of an additional sum.
However, the very low fee agreed with GTS-GRAL {\bf does not entitle
the packages to be made available for general applications outside of
HEP}, and institutes are advised that they have both legal and moral
obligations to take a standard licence (less 25% educational discount)
if they wish to do this.
 
In addition, a further extension to the original contract now provides CERN
with the right to print and distribute GTS-GRAL User Manuals to affiliated
institutes holding GTS-GRAL licences. This means that both
software distribution and manuals may be obtained from CERN.
CERN will charge the institute for printing plus a small
handling charge.
\section{Summary of Offer for an Embedded Licence}
 
With the conditions of use specified above, an embedded licence
entitles the holder to receive the following software:
\begin{OL}
\item The kernel of GKSGRAL and GKSGRAL-3D at the current revision level
in object-code format for one of the following operating systems:
MVS, VM/CMS, VMS, AEGIS, UNIX%
\footnote{%
The director responsible for computing, Dr. J. Thresher, has decided
that it will be necessary to ask the initial users of the software on
UNIX to contribute 3000.-- CHF to the cost of extending the CERN
licence. This is independent of the licencing arrangements made
between the institute and GTS-GRAL.}
or SINTRAN-3.
\item
\index{driver list}
The drivers available at CERN%
in both object-code and source-code format.
 
As of January 1989, CERN holds licences to distribute
the following drivers
for use on the systems listed in parentheses:
\begin{UL}
\item TEKTRONIX 4014 (MVS, VM/CMS, VMS, AEGIS, UNIX, SINTRAN-3)
\item TEKTRONIX 4107/4109 (MVS, VM/CMS, VMS, AEGIS, UNIX, SINTRAN-3)
\item TEKTRONIX 4235/6 (VMS) =3D driver= (in preparation)
\item Hewlett-Packard Plotters (MVS, VM/CMS, VMS, AEGIS, UNIX, SINTRAN-3)
\item VERSATEC Plotters via VERSAPLOT (MVS, VM/CMS, VMS, AEGIS, UNIX, SINTRAN-3)
\item PERICOM MG600 (VM/CMS, UNIX, VMS)
\item PERICOM MX2000/7000/8000 (VM/CMS, UNIX, VMS)
\item VAXstation (VMS-UIS)
\item APOLLO GSR and GSR (AEGIS/UNIX)
\item MEGATEK 33xx (VMS) =3D driver=
\item PostScript (VM/CMS, VMS, AEGIS, UNIX)
\item QMS LG(VMS)
\item VT240 and VT340 (VMS)
\item XEROX printers via EPIC (VM/CMS)
\item IBM APA 6670 and 3812
\end{UL}
\item The skeleton drivers in source-code format.
\item The source-code of the link routines necessary to re-configure
the system with extra drivers.
\end{OL}
 
CERN is entitled to distribute updates and additional drivers%
\footnote{Institutes wishing to purchase extra drivers are advised to
take a licence including access to the source in case modification
are required, for example to enable the use of GCGTOA and GCATOG.}
to institutes holding such a licence.
However, additional software not available at CERN, may be
purchased directly from GTS-GRAL with the following conditions:
\begin{OL}
\item A 35% discount on GKSGRAL for operating systems
other than those already licenced at CERN.
\item A 50% discount on source-code price (40% discount on object-code)
for drivers not already licenced at CERN.
\end{OL}
 
A current list of products available from GTS-GRAL may be obtained
by writing to the address given below. Over 50 computers are
supported and a similar number of drivers exist.
\section{Pricing}
The {\it base} prices for embedded licences are:
\begin{flushleft}
 A: on IBM DM 28.000,--                           \\
 B: on VAX DM 28.000,--                           \\
 C: on APOLLO or Microvax DM 14.000,--            \\
 D: on UNIX for multi-user system DM 28.000,--    \\
 E: on UNIX for single-user system DM 14.000,--   \\
\end{flushleft}

of which the following percentages are payable {\it per
institute and operating system}:
 
\begin{center}
\begin{tabular}{lll}
1 -  49  &licences bought world-wide  &10.0\%  of A, B...E\\
50 - 100 & licences bought world-wide & 5.0\% of A, B...E\\
more than 100 & licences bought world-wide & 2.5\% of A, B...E\\
\end{tabular}
\end{center}
 
This implies that for the first 49 licences%
\footnote{Since end 1990 the total number of licences has passed 100.}
to be signed {\it world-wide},
the licencee will pay DM 2.800,-- for a multi-user computer,
and DM 1'400,-- for single-user machines.
In the case that the affiliated institute wishes to use the software
to work on HEP experiments not performed at CERN, the licence fee
is surcharged by 50%. Thus, for example, to use a VAX or IBM system
to analyse data from a LEP experiment {\it and} an SLD or Gran Sasso
experiment would cost DM 2.100,-- instead of DM 1.400,-- once the limit
of 49 licences has been passed.
 
The prices for a full licence which would permit the software to be used
for applications outside the domain of HEP are:
 
\begin{center}
\begin{tabular}{ll}
GKSGRAL-3D for IBM & DM 28.000,--\\
GKSGRAL-3D for DEC/VMS & DM 28.000,--\\
GKSGRAL-3D for APOLLO & DM 14.500,--\\
GKSGRAL-2D for IBM & DM 19.250,--\\
GKSGRAL-2D for VAX & DM 19.250,--\\
GKSGRAL-2D for APOLLO & DM 9.950,--\\
\end{tabular}
\end{center}
 
Institutes are entitled to receive a 25% research discount on
these figures. (The pricing for MicroVAX is the same as for APOLLO.)
If the institute wishes to have the software installed by GTS-GRAL,
then the installation costs would be:
\begin{OL}
\item IBM (MVS,VM/CMS) DM 5.000,--
\item DEC/VAX (VMS) DM 5.000,--
\item APOLLO DM 2.500,--
\end{OL}
\section{Procedure}
 
An institute wishing to obtain an embedded licence for the use
of GTS-GRAL software must place an order with the company directly,
who will deliver the manual and give CERN permission for distribution
when the licence has been signed and on payment of the fee.
The address to contact is:
\begin{XMP}
GTS-GRAL GmbH
Friedberger Strasse 25
6100 Darmstadt
West Germany.
 
Tel. +49.6151.73090
Fax. +49.6151.77718
\end{XMP}
It is possible that in some countries it may be necessary to deal with
a local GTS-GRAL distributor, although one may still contact GTS-GRAL in the
first instance.
 
Once the licence has been signed and payed for, {\it and CERN has received
confirmation of this from GTS-GRAL}, then it will be possible to
obtain the software via the CERN Program Library Office.
To order additional copies of GKSGRAL or GKSGRAL-3D manuals
contact the Program Library.
 
The Electronic Mail address for the program library is:
\Lit{CERNLIB@CERNVM.CERN.CH}
 
An attempt is being made to maintain a list of embedded-licence
holders. Thus, the following information should be provided:
\begin{OL}
\item The address of the institute.
\item The date the contract was signed.
\item The name of the contact person(s).
(Not the person who signed the contract, but the person
responsible for installation and maintenance.)
\item The Electronic Mail address and telephone number
by which they can be contacted.
\item The operating system for which the software is licenced.
\item The distribution medium required.
\end{OL}
Note that for VM/CMS sites it will be necessary to discuss the
installation procedure in more detail in order to resolve possible problems
concerning different FORTRAN compilers, ASCII-to-EBCDIC
conversion tables, and so forth. For UNIX sites installation may
also be a problem if the target machine is not available at CERN
for compiled libraries to be built.
 
Affiliated institutes should be aware that there is little spare effort
available at CERN to provide software support,
although a member of the Graphics Section has been designated to
coordinate the interface between CERN and external graphics users,
and may be contacted via
\Lit{GRAPHICS@CERNVM.CERN.CH}
The graphics contact people at affiliated institutes will also
receive a Newsletter which is distributed via E-Mail.
In addition, some first-line support is available from 'volunteers'
in various CERN member states.
The up-to-date list of names of these people will be sent to those
institutes holding licences.

