%%%%%%%%%%%%%%%%%%%%%%%%%%%%%%%%%%%%%%%%%%%%%%%%%%%%%%%%%%%%%%%%%%%
%                                                                 %
%   GKS - Reference Manual -- LaTeX Source                        %
%                                                                 %
%   Front Material: Title page,                                   %
%                   Copyright Notice                              %
%                   Preliminary Remarks                           %
%                   Table of Contents                             %
%   EPS file      : cern15.eps, cnastit.eps                       %
%                                                                 %
%   Editor: Michel Goossens / CN-AS                               %
%   Last Mod.: 14 July 1992 10:40 mg                              %
%                                                                 %
%%%%%%%%%%%%%%%%%%%%%%%%%%%%%%%%%%%%%%%%%%%%%%%%%%%%%%%%%%%%%%%%%%%

%%%%%%%%%%%%%%%%%%%%%%%%%%%%%%%%%%%%%%%%%%%%%%%%%%%%%%%%%%%%%%%%%%%%
%    Tile page                                                     %
%%%%%%%%%%%%%%%%%%%%%%%%%%%%%%%%%%%%%%%%%%%%%%%%%%%%%%%%%%%%%%%%%%%%
\def\Ptitle#1{\special{ps: /Printstring (#1) def}
\epsfbox{/user/goossens/cnasall/cnastit.eps}}
 
\begin{titlepage}
\vspace*{-23mm}
\mbox{\epsfig{file=/usr/local/lib/tex/ps/cern15.eps,height=30mm}}
\hfill
\raise8mm\hbox{\Large\bf CERN Program Library Documentation}
\hfill\mbox{}
\begin{center}
\mbox{}\\[10mm]
\mbox{\Ptitle{GKS}}\\[2cm]
{\LARGE GKS/GKS-3D at CERN}\\[3cm]
{\LARGE Graphics Section}\\[1cm]
{\Large Application Software Group}\\[1cm]
{\Large Computing and Networks Division}\\[2cm]
\end{center}
\vfill
\begin{center}\Large CERN Geneva, Switzerland\end{center}
\end{titlepage}

%%%%%%%%%%%%%%%%%%%%%%%%%%%%%%%%%%%%%%%%%%%%%%%%%%%%%%%%%%%%%%%%%%%%
%    Copyright  page                                               %
%%%%%%%%%%%%%%%%%%%%%%%%%%%%%%%%%%%%%%%%%%%%%%%%%%%%%%%%%%%%%%%%%%%%
\thispagestyle{empty}
\framebox[\textwidth][t]{\hfill\begin{minipage}{0.96\textwidth}%
\vspace*{3mm}\begin{center}Copyright Notice\end{center}
\parskip\baselineskip
{\bf GKS/GKS-3D at CERN}
 
CERN Program Library documentation
 
\copyright{} Copyright CERN, Geneva 1992

Copyright and any other appropriate legal protection of these
computer programs and associated documentation reserved in all
countries of the world.
 
These programs or documentation may not be reproduced by any
method without prior written consent of the Director-General
of CERN or his delegate.
 
Permission for the usage of any programs described herein is
granted apriori to those scientific institutes associated with
the CERN experimental program or with whom CERN has concluded
a scientific collaboration agreement.
 
CERN welcomes comments concerning this program
but undertakes no obligation for its maintenance,
nor responsibility for its correctness, and accepts no liability
whatsoever resulting from the use of this program.
 
Requests for information should be addressed to:
\vspace*{-.5\baselineskip}
\begin{center}
\renewcommand{\arraystretch}{.9}
\tt\begin{tabular}{l}
CERN Program Library Office              \\
CERN-CN Division                         \\
CH-1211 Geneva 23                        \\
Switzerland                              \\
Tel.      +41 22 767 4951                \\
Fax.      +41 22 767 7155                \\
Bitnet:   CERNLIB@CERNVM                 \\
DECnet:   VXCERN::CERNLIB (node 22.190)  \\
Internet: CERNLIB@CERNVM.CERN.CH
\end{tabular}
\end{center}
\vspace*{1mm}
\end{minipage}\hfill}%end of minipage in framebox
\vspace{6mm}
 
{\bf Trademark notice: All trademarks appearing in this guide are acknowledged as such.}
\vfill
\begin{tabular}{l@{\qquad}l@{\quad}>{\tt}l}
{\em Contact Person\/}:        & Carlo Vandoni/CN    & (VANDONI\atsign CERNVM.CERN.CH) \\
{\em Technical Realization\/}: & Michel Goossens /CN & (GOOSSENS\atsign CERNVM.CERN.CH)\\[1cm]
{\em Second edition - July 1992}
\end{tabular}
 
\newpage
 
%%%%%%%%%%%%%%%%%%%%%%%%%%%%%%%%%%%%%%%%%%%%%%%%%%%%%%%%%%%%%%%%%%%%
%    Introductory material                                         %
%%%%%%%%%%%%%%%%%%%%%%%%%%%%%%%%%%%%%%%%%%%%%%%%%%%%%%%%%%%%%%%%%%%%
\pagenumbering{roman}
\setcounter{page}{1}

\section*{Preliminary remarks}
 
This manual explains how to use GKS and GKS-3D at CERN.
 
In this manual
examples are in {\tt monotype face} and strings to be input by the user 
are {\tt\underline{underlined}}.
In the index the page where a routine is defined is in {\bf bold},
page numbers where a routine is referenced are in normal type.

In the description of the routines a \Lit{*} following
the name of a parameter indicates that this is an {\bf output} parameter.
If another \Lit{*} precedes a parameter in the calling sequence, the
parameter in question is both an {\bf input} and {\bf output} parameter.

Certain areas described in this manual are still subject to change.
Their description is included in the hope of soliciting feedback. 
They are flagged with a changebar in the margin.

This document has been produced using \LaTeX~\cite{bib-LATEX}
with the \Lit{cernman} style option, developed at CERN. 
A PostScript file \Lit{hbook.ps}, containing a complete printable version
of this manual, can be obtained from any CERN machine
by anonymous ftp as follows
(commands to be typed by the user are underlined):

\vspace*{3mm} 
\begin{tabular}{@{\hspace{12mm}}>{\tt}l}
\underline{ftp asis01.cern.ch}\\
Trying 128.141.8.104...\\
Connected to asis01.cern.ch.\\
220 asis01 FTP server (SunOS 4.1) ready.\\
Name (asis01:username): \underline{anonymous}\\
Password: \underline{your\_{}mailaddress}\\
ftp> \underline{cd doc/cernlib}\\
ftp> \underline{get gks.ps}\\
ftp> \underline{quit}\\
\end{tabular}

%%%%%%%%%%%%%%%%%%%%%%%%%%%%%%%%%%%%%%%%%%%%%%%%%%%%%%%%%%%%%%%%%%%%
%    Tables of contents ...                                        %
%%%%%%%%%%%%%%%%%%%%%%%%%%%%%%%%%%%%%%%%%%%%%%%%%%%%%%%%%%%%%%%%%%%%
\newpage
\tableofcontents
\newpage
\listoffigures
\listoftables

