%  21 oct 94  ksk
\Version{RTRINT}                                \Routid{D105}
\Keywords{ADAPTIVE NUMERICAL INTEGRATION TRIANGLE QUADRATURE}
\Author{K.S. K\"olbig}                          \Library{MATHLIB}
\Submitter{}                                     \Submitted{02.05.1966}
\Language{Fortran}                         \Revised{01.12.1994}
\Cernhead{Integration over a Triangle}
Function subprograms {\tt RTRINT} and {\tt DTRINT}
compute an approximate value of the integral
$$ I \ = \ \int \! \int f(x,y)\,dx dy, $$
evaluated over the interior of an arbitrary triangle $\Delta$ in the
$xy$-plane. An attempted accuracy may, optionally, be specified.
\par
On computers other than CDC or Cray, only
the double-precision version {\tt DTRINT} is available.
On CDC and  Cray computers, only the single-precision version
{\tt RTRINT} is available.
\Structure
{\tt FUNCTION} subprograms\\
User Entry Names: \Rdef{RTRINT}, \Rdef{DTRINT}\\
Obsolete User Entry Names: \Rdef{TRIINT} $\equiv$ {\tt RTRINT} \\
Files Referenced: {\tt Unit 6} \\
External References: \Rind{MTLMTR}{N002}, \Rind{ABEND}{Z035}
\Usage
For $\mathtt{t=R}$ (type {\tt REAL}), $\mathtt{t=D}$ (type
{\tt DOUBLE PRECISION}),
\begin{verbatim}
    tTRINT(F,NSD,NPT,EPS,X1,Y1,X2,Y2,X3,Y3)
\end{verbatim}
has, in any arithmetic expression, the approximate value of the
integral $I$.
\begin{DLtt}{123456}
\item [F] (type according to {\tt t})
Name of a user-supplied {\tt FUNCTION} subprogram, declared
{\tt EXTERNAL} in the calling program. This subprogram must set
$\mathtt{F(X,Y)} = f(\mathtt{X,Y})$.
\item [NSD] ({\tt INTEGER}) \\
${\tt= 0:}$  No subdivision of the given triangle.\\
${\tt= 1:}$  Subdivision of the given triangle (see Method).
\item[NPT] ({\tt INTEGER}) \\
${\tt= \ 7:}$ A 7-point integration formula is used.\\
${\tt= 25:}$  A 25-point integration formula is used.\\
${\tt= 64:}$  A 64-point integration formula is used.
\item[EPS] (type according to {\tt t}) Accuracy parameter
(see {\bf Accuracy}).
\item[X1,Y1] (type according to {\tt t}) The coordinates of the
vertices of $\Delta$.
\item[X2,Y2]
\item[X3,Y3]
\end{DLtt}
\newpage
\Method
$\mathtt{NSD = 0:}$ \\
An approximation $I_0 $ to $I$ is found by computing the {\tt NPT}-point
formula for the triangle $\Delta$. The value of {\tt EPS} has no
influence on the result.\\
$\mathtt{NSD = 1:}$ \\
After computing $ I_0 $, the triangle $\Delta$ is subdivided into two
subtriangles $\Delta'$ and $\Delta''$, the corresponding
approximations $I'$ and $I''$ are computed, and a test is made to
see whether
$$ \frac{|I_0-(I'+I'')|}{1+|I'+I''|} < \mathtt{EPS}  $$
If this test is satisfied, the routine terminates by setting the
function value to $I_0$. If it fails, the process of subdivision and
testing continues according to a tree structure. The routine
terminates either because the test is passed successfully by all the
subtriangles at some level, or because a maximum number of subdivisions
is reached (see {\bf Error Handling}).
\Accuracy
Unless  there is severe cancellation  of positive and negative
values of $f(x,y)$ over $\Delta$,the argument {\tt EPS} may,
if $\mathtt{NCD = 1}$, be considered as specifying a bound on the
relative error of $I$ in the case $ |I|>1$,
and a bound on the absolute error in the case $ |I|<1$.
\Restrict
"Mild" singularities are permitted if they coincide with the
vertices of $\Delta$. Any other singularity lying inside $\Delta$ or on
its boundaries will most likely lead to too many subdivisions (see
{\bf Error Handling}), or cause a wrong result.
\Errorh
Error {\tt D105.1}: $\mathtt{NPT \ne 7,\,25,\,64}$. \\
Error {\tt D105.2}: The number of subdivisions has reached 35 without
success. \\
In both cases,
the function value is set equal to zero, and a message is written on
{\tt Unit 6}, unless subroutine {\tt MTLSET} (N002) has been called.
\Refer
\begin{enumerate}
\item K.S. K\"olbig, A Fortran program and some numerical
test results for the integration over a triangle, CERN 64--32 (1964).
\end{enumerate}
$\bullet$
