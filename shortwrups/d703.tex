\Version{RFFT}                                 \Routid{D703}
\Keywords{REAL FAST FOURIER  TRANSFORM}
\Author{H.-H. Umst\"atter}                     \Library{KERNLIB}
\Submitter{}                                    \Submitted{01.12.1981}
\Language{Fortran}                        %\Revised{}
\Cernhead{Real Fast Fourier Transform}
Subroutine {\tt RFFT} computes either:
\begin{DLtt}{12}
\item[$\bullet$]  the finite Fourier transform of a real periodic
sequence, or
\item[$\bullet$]  the corresponding inverse transform.
\end{DLtt}
The period $n$ must be a power of 2.
\par
Given a sequence of real numbers $y_0,y_1,\ldots,y_{n-1}$,
the forward transform computes the terms
$$ \begin{array}{rcl}
c_k & = & \displaystyle \frac{1}{n}
\sum_{m=0}^{n-1} y_m \exp \left( \frac{-i 2\pi mk}{n} \right), \\[6mm]
& = & \displaystyle \frac{1}{n}
\sum_{m=0}^{n-1} y_m \cos \left( \frac{2\pi mk}{n} \right) -
\displaystyle \frac{i}{n}
\sum_{m=0}^{n-1} y_m \sin \left( \frac{2\pi mk}{n} \right),
\qquad (k=0,1,\ldots,n-1),
\end{array} $$
which have period $n$ in the subscript $k$ and passess the property
$c_{-k}=\bar{c}_k=c_{n-k}$, where $\bar{c}_k$ is the complex conjugate
of $c_k$. Therefore only those $c_k$ for which $0 \leq k \leq n/2$
are computed by {\tt RFFT}.
\par
Conversely, given a sequence of complex numbers $c_0,c_1,\ldots,c_{n-1}$
possessing the property $\bar{c}_{-k}=c_k$, the inverse transform
computes the terms
$$ \begin{array}{rcl}
y_m & = & \displaystyle
\sum_{k=0}^{n-1} c_k \exp \left( \frac{i 2\pi km}{n} \right),
\qquad (m=0,1,\ldots,n-1),
\end{array} $$
which have period $n$ in the subscript $m$ and are real. Only those
$c_k$ for which $0 \leq k \leq n/2$ need be supplied as input to
{\tt RFFT} for the calculation of (2).
\par
To ensure optimum use of storage, the same array is used for input and
output, and all intermediate calculations are carried out in this array.
\Structure
{\tt SUBROUTINE} subprogram \\
User Entry Names: \Rdef{RFFT} \\
External References: \Rind{CFFT} (D704)
\Usage
\begin{verbatim}
    COMPLEX C(ncdim)
    REAL Y(nydim)
    EQUIVALENCE (C,Y)
    ...
    CALL RFFT (C,M)
    ...
\end{verbatim}
\newpage
\begin{DLtt}{1234}
\item[C] ({\tt COMPLEX}) Array of length {\tt ncdim} not less than
$(n/2+1)$.
\item[Y] ({\tt REAL}) Array of length {\tt nydim} not less than $n$.
\item[M] ({\tt INTEGER}) On entry, the absolute value of {\tt M}
determines the value of $n$ through the relation $n=2^{\mathtt{|M|}}$.
If {\tt M} is negative, expression (1) is evaluated. If {\tt M} is
non-negative, expression (2) is evaluated. Unchanged on exit. \\
{\tt M < 0:} \\
On entry, $\mathtt{Y}(m+1)=y_m,\,(m=0,1,\ldots,n-1)$.\\
On exit, $\mathtt{C}(k+1)=c_k,\,(k=0,1,\ldots,n/2)$, as defined by
(1). \\
$\mathtt{M \geq 0:}$ \\
On entry, $\mathtt{C}(k+1)=c_k,\,(k=0,1,\ldots,n/2)$. \\
On exit, $\mathtt{Y}(m+1)=y_m,\,(m=0,1,\ldots,n/2)$, as
defined by (2) with $c_k=\bar{c}_k$.
\end{DLtt}
\Method
{\tt RFFT} calls the complex fast Fourier transform subroutine
{\tt CFFT} (D704) with sequences of half-length as explained in Ref. 1.
\Refer
\begin{enumerate}
\item  E.O. Brigham, The fast Fourier transformation,
(Prentice-Hall, Englewood Cliffs 1974) Chap. 10, Sect. 10, Fig. 10-10.
\end{enumerate}
$\bullet$
