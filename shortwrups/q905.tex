% ksk 08 may 96
\Version {WYLBUR}                    \Routid{Q905}
\Keywords{WYLBUR PHOENIX LINE EDITOR ASCII TEXT FILES}
\Author{J. Zoll}   \Library{None}
\Submitter{}             \Submitted{15.09.1994}
\Language{Fortran + C}                    %  \Revised{}
\Cernhead{Wylbur Phoenix -- a Line Editor for ASCII Text Files}
\begin{center}
\fbox{\parbox{120mm}{\OBSOLETE
Please note that this routine has been obsoleted. Users are
advised not to use it any longer. No maintenance for it will take
place and it will eventually disappear. }}
\end{center}
{\tt Wylbur Phoenix} is a portable command driven editor, capable of
embedding a full-screen editor of the user's choice as a sub-system.
It can operate with the simplest Telnet connection to some
remote machine. It is designed to give maximum power for the development
and maintenance of the source files of the large programs used in
particle physics, where it is neccessary to easily find in a large
volume what one is looking for. It has been written because no editor
is available which combines all the features considered essential:
\begin{tabular}[t]{rl}
a) & Ease of use for the casual user; \\
b) & 'undo' a series of mistaken edit operations; \\
c) & global changes displayed, and maybe confirmed individually; \\
d) & column sensitive editing; \\
e) & handling of program variable names, not only text strings,
but without language syntax analysis; \\
f) & direct handling of program units, ie. Fortran or C routines
or Patchy decks. \\
g) & 'master range' automatically limiting edit operations to an
arbitrary fraction of the whole file; \\
h) & usage of windows as monitors and for full-screen editing; \\
i) & immediate, context-free, display of critical lines. \\
j) & permanent line numbers, not hindering normal access
to the files by programs other than the editor; \\
k) & portability.
\end{tabular} \\[2mm]
Although {\tt Wylbur Phoenix} does have some aspects of 'full screen'
and interactive operations, these are distinct features which
can selectively be switched off in 'batch mode' or in
'nowindow mode'. Thus {\tt Wylbur} can be used in shell scripts
and across non-specialized computer links; indeed for some
applications {\tt Wylbur} in batch mode is very convenient.
\Structure
Complete program
\Usage
Shell command "use fn" calls the normal version of Wylbur into
operation to act on file "fn". This version is typically capable
of handling 60000 lines. For bigger files one may use "useb" on
some machines, which allows for 120000 lines.
\par
On the Unix machines "use" and "useb" are links in {\tt /cern/pro/bin}
pointing to the executable modules. On the Vax "use" should be
a symbol like
\begin{verbatim}
    $ USE :== $CERN_ROOT:[EXE]WYLBUR
\end{verbatim}
{\tt Wylbur} has not been made to work on IBM with VM/CMS.
\par
To print the file used for delivering on-line help proceed as
follows: \\
\begin{tabular}[t]{ll}
type "use"             & to call {\tt Wylbur} into operation, \\
type "help -p temp 84" & to create file "temp" for printing, \\
type "help h"          & for instructions on how to print file "temp".\\
\end{tabular} \\[2mm]
$\bullet$
