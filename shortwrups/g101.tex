\Version{CHISIN}                       \Routid{G101}
\Keywords{CHI-SQUARE DISTRIBUTION INVERSE}
\Author{K.S. K\"olbig}                \Library{MATHLIB}
\Submitter{}                      \Submitted{15.10.1976}
\Language{Fortran}            \Revised{15.03.1993}
\Cernhead{Inverse of Chi-Square Distribution}
Function subprogram {\tt CHISIN} calculates $\chi^2(P,N)$ for a given
probability $P(\chi^2)$ and a given degree of freedom $N$, where
$$P(\chi^2|N) \ = \
\frac{1}{\sqrt{2^N}\Gamma(\textstyle \frac{1}{2}N)} \
\int_0^{\chi^2(P,N)}\,e^{-\frac{1}{2}t}t^{\frac{1}{2}N-1}\, dt $$
and $N \geq 1$ and $0 \leq P(\chi^2) < 1$.
\Structure
{\tt FUNCTION} subprogram \\
User Entry Name: \Rdef{CHISIN} \\
Files Referenced: {\tt Unit 6} \\
External References: \Rind{GAUSIN}{G105},
\Rind{MTLMTR}{N002}, \Rind{ABEND}{Z035}
\Usage
In any arithmetic expression,
\begin{center}
{\tt CHISIN(P,N)} \quad has the value \quad $\chi^2(\mathtt{P,N})$,
\end{center}
where {\tt CHISIN} and {\tt P} are of type
{\tt REAL}, and {\tt N} is of type {\tt INTEGER}.
\Method
The method is described in Ref. 1. Note that there the complementary
integral is taken.
\Accuracy
Approximately three to six digits are correct. The case $N = 3$
is the least accurate.
\Errorh
Error {\tt G101.1}: $\mathtt{P<0}$ or $\mathtt{P \ge 1}$. \\
Error {\tt G101.2}: $\mathtt{N<1}$. \\
In both cases,
the function value is set equal to zero, and a message is written on
{\tt Unit 6}, unless subroutine {\tt MTLSET} (N002) has been called.
\Source
This subprogram is based on an Algol60 procedure published in Ref. 1.
\Refer
\begin{enumerate}
\item R.B. Goldstein, Algorithm 451, Chi-Square Quantiles, Collected
Algorithms from CACM (1972)
\end{enumerate}
$\bullet$
