\Version{IOSPAK2}                 \Routid{Z308}
\Keywords{SUBROUTINE PACKAGE ROUTINE}
\Keywords{IOSWRT IOSRD IOSDIM}
\Keywords{IBM FULLSCREEN FULL-SCREEN SCREEN IO I/O}
\Keywords{7171 3270 X3270 VT100 COMPATIBLE SERIES/1}
\Keywords{GENERAL IBM INTERFACE}
\Keywords{PACK VM/CMS}
\Author{E. Perotto}                \Library{KERNLIB, IBM VM/CMS only}
\Submitter{}                       \Submitted{22.02.1988}
\Language{IBM Assembler}          %  %\Revised{}
\Cernhead {3270 Full Screen I/O Routines}
These routines write a character string anywhere on a full
screen 3270 terminal, or any VT100 compatible terminal attached to
a 7171 or Series/1.
The screen may also be formatted into fields preceded by attributes,
to intensify (highlight), protect, or hide the field
(like in the password entry screen of CP).
The extended attributes (colors on the 3278) are not supported,
in this case the four default colors discriminate
protected/unprotected and normal/highlighted fields.
\Structure
{\tt FUNCTION} subprograms \\
User Entry Names: \Rdef{IOSWRT}, \Rdef{IOSRD}, \Rdef{IOSDIM}
\Usage
The variable {\tt IRC} and the functions {\tt IOSWRT}, {\tt IOSRD}
and {\tt IOSFLD} are of type {\tt INTEGER}.
\begin{verbatim}
    IRC = IOSWRT()
    IRC = IOSWRT('CLEAR')
\end{verbatim}
Clears the screen of the 3270 terminal at the time of next write to
screen, no actual I/O takes place immediately.
The reason for this is that CP leaves control of the terminal upon
an {\tt ERASE/WRITE} command (executed {\tt CCW}) by the user program,
so that the screen should be cleared and written to with the same
command.
\begin{verbatim}
    IRC = IOSWRT(LINE,LENGTH,LINENO,NCOL)
    IRC = IOSWRT(LINE,LENGTH,LINENO,NCOL,IATTR)
    IRC = IOSWRT(LINE,LENGTH,LINENO,NCOL,IATTR,0)
    IRC = IOSWRT(LINE,LENGTH,LINENO,NCOL,IATTR,IBUF)
\end{verbatim}
Puts a line of text in a buffer for subsequent writing to screen,
in this way a panel can be built in (conditional) steps.
(The call is sensitive to the number of parameters passed.)
\begin{DLtt}{12345678}
\item[IRC] Indicates whether the operation was successful or not. \\
$\mathtt{= \ 0:}$  Operation successful; \\
$\mathtt{= -1:}$ The screen is under CP control, must clear and rewrite
it;\\
$\mathtt{= -2:}$  Invalid line number; \\
$\mathtt{= -3:}$  Invalid column number; \\
$\mathtt{= -4:}$  Buffer full; \\
$\mathtt{= -5:}$  No space for attribute; \\
$\mathtt{= -6:}$  String length not specified; \\
$\mathtt{= -7,...,-12:}$  Non-fatal I/0 errors, retry.
\item [LINE] A character string to be displayed. This string may consist
of a string in quotes, a character variable or array, or any array
containing Hollerith constants.
\item [LENGTH] ({\tt INTEGER}) Contains the length of the character
string (or the number of bytes in the hollerith array).
It has a different meaning according to its value: \\
$\mathtt{> 0:}$  \parbox[t]{138mm}
{Maximum limit is the end of the buffer.
A string can occupy the whole screen, wrapping from the end to the
beginning of the screen.  The last 8 positions at the bottom right
corner are not visible, being overwritten by CP with
{\tt 'CERNVM\ \ '}.} \\
$\mathtt{= 0:}$ \parbox[t]{138mm}
{A field is started empty if the attribute is present.  An empty
(unprotected) field may be used for input.} \\
$\mathtt{< 0:}$ \parbox[t]{138mm}
{The cursor is positioned on the screen as indicated by the following
parameters.}
\item [LINENO] ({\tt INTEGER}) Defines the line in which the start
of the string should be positioned; the top line is line {\tt 1},
the bottom line is line {\tt 24} on 3270 Model 2,
line {\tt 35}, {\tt 43} or {\tt 27} on 3270 Models 3, 4, or 5.
\item [NCOL] ({\tt INTEGER}) Defines the column in which the start
of the string should be positioned; the leftmost column is column
{\tt 1}, the rightmost is {\tt 80} in Models 2-4, and 132 in Model 5.
\item [IATTR] ({\tt INTEGER}) Contains the attributes of the field.
If this parameter is present, the string is also the beginning of a
field, which inherits the properties of the attribute, and extends
across the screen to the next field.  The attribute is located
in the position at the left of the first character of the string,
it is hidden (dark) and protected from overwriting by the keyboard.
The rightmost 6 bits of {\tt IATTR} define the attribute as follows: \\
$\mathtt{= 32:}$ ({\tt PROTECT}) The field will not be overwritten
by the keyboard; \\
$\mathtt{= 16:}$ ({\tt NUMERIC}) The field will accept only numeric
input; \\
$\mathtt{= 12:}$ \parbox[t]{136mm}
{{\tt (HIDE)} The field will not be displayed,
like the password input field in the CP {\tt LOGON} screen;} \\
$\mathtt{= \ 8:}$ ({\tt HIGHLIGHT}) The field will be intensified and
selector pen detectable;  \\
$\mathtt{= \ 4:}$ ({\tt SELPEN}) The field will be selector pen
detectable only; \\
$\mathtt{= \ 2:}$ ({\tt Reserved}) Should be set to zero;  \\
$\mathtt{= \ 1:}$ \parbox[t]{136mm}
{({\tt MDT}) Modified Data Tag, when set the field will be included in
the next {\tt READ MODIFIED} command executed. This bit
is set when something is put in the field by the keyboard, but can be
set under program control to include default responses which will be
displayed on the screen, and read back even if the content has not been
modified by the keyboard.} \\
The attribute is the logical {\tt OR} of all the values.
\item [IBUF] ({\tt INTEGER}) Array used to prepare a mask for the screen
which can be used several times.  The first value must be initialized
by the user to the length in bytes of the {\tt N-2} remaining words
of the array, while the second value must be initialized to zero.
Every time this routine is called with all parameters this value
is updated, so that the user knows the amount used, and can clear
the buffer by resetting to zero the second element of the array.
\par
If {\tt IBUF} (or the first element of the array) is zero,
an internal array is used (around 3000 bytes, the rest of the 4K
memory page).  If this parameter is missing altogether, an immediate
write to screen is performed, useful to fill up the voids of the
already displayed mask with variable information, defaults
in input fields, or old inputs which may be modified later;
in this case the internal buffer is used, destroying its contents;
so the internal buffer must be used before an immediate write
is performed.
\end{DLtt}
\begin{verbatim}
    IRC = IOSWRT(IBUF)
    IRC = IOSWRT(0)
    IRC = IOSWRT('WRITE')
\end{verbatim}
This sequence writes the assembled mask to screen, from the user-supplied
buffer in the first case, from the internal buffer in the latter two
cases. The buffer remains intact and can be used repeatedly.
If $\mathtt{IRC = -1}$, the screen content is lost, so it can be rewritten.
Successive masks may be overlapped on the screen until it is cleared.
\newpage
\begin{verbatim}
    IRC = IOSRD(ECB)
\end{verbatim}
Sets the Attention Interrupt trap, which tells the program when the
{\tt ENTER} key or any {\tt PF}-key has been hit, without the need to
wait.
\begin{DLtt}{123456}
\item [ECB] ({\tt INTEGER}) Receives a non-zero value upon Attention
Interrupt (actually the lower part of CSW, with status bits).
This variable should be set to zero before the call, so that the program
may periodically inspect, and continue processing while it is zero,
or go to retrieve full-screen input if not.
\end{DLtt}
\begin{verbatim}
    IRC = IOSRD(BUFF,LENGTH,LINENO,NCOL)
\end{verbatim}
Starts a synchronous full-screen read and collects all the data from the
modified fields in {\tt BUFF}, or in an internal buffer.
\begin{DLtt}{12345678}
\item[IRC] Indicates whether the operation was successful or not. \\
$\mathtt{> \ 0:}$ Operation successfull, {\tt IRC} = Number of
the {\tt PF}-key pressed; \\
$\mathtt{= \ 0:}$ Operation successfull, {\tt ENTER} pressed; \\
$\mathtt{= -1:}$ Screen under CP control ({\tt PA1} key pressed before); \\
$\mathtt{= -2:}$ {\tt PA2} key pressed (Clear Output in CMS); \\
$\mathtt{= -3:}$ {\tt CLEAR} key pressed, the screen is now empty; \\
$\mathtt{= -4:}$ No input available; \\
$\mathtt{= -5:}$ Incorrect parameters number; \\
$\mathtt{= -6,...,-12:}$ I/O errors, retry.
\item [BUFF] Array of convenient length to receive the modified fields.
This buffer is useful if the input has to be saved, otherwise it
is not required, the internal buffer is used instead.
\item [LENGTH] ({\tt INTEGER}) This variable contains, on entry,
the length of {\tt BUFF}. On exit, it contains the actual length used of
this array.
The length of the buffer should be the sum of the field lengths,
three bytes per field, plus four bytes for the cursor coordinates.
If this parameter is zero, then the internal buffer is used.
\item [LINENO] ({\tt INTEGER}) Gives the number of the line where
the cursor is located at the moment of the full-screen read.
\item [NCOL] ({\tt INTEGER}) Gives the number of the column where
the cursor is located at the moment of the full-screen read.
\end{DLtt}
\begin{verbatim}
    CALL IOSFLD(LINE,LENGTH,LINENO,NCOL)
\end{verbatim}
Gives the modified fields taken by the last {\tt IOSRD} call, one by one
with their individual lengths and positions on the screen.
\begin{DLtt}{12345678}
\item [LINE] Array of convenient length to receive the modified field.
\item [LENGTH] ({\tt INTEGER}) This variable contains, on entry,
the length of {\tt LINE}. On exit, it contains the actual length used of
the field. When it becomes negative then there are no more fields left
in the buffer.
\item [LINENO] ({\tt INTEGER}) Gives the number of the line where
the first character of the field is located.
\item [NCOL] ({\tt INTEGER}) Gives the number of the column
where the first character of the field is located.
\end{DLtt}
$\bullet$
