% 13 mar 1996  ksk
\Version {FUNRAN}                    \Routid{V151}
\Keywords{DISTRIBUTION NUMBER RANDOM ARBITRARY}
\Author{F. James}                    \Library{MATHLIB}
\Submitter{}                         \Submitted{27.11.1984}
\Language{Fortran}                     %\Revised{}
\Cernhead {Random Numbers According to Any Function}
\begin{center}
\fbox{\parbox{120mm}{\OBSOLETE
Please note that this routine has been obsoleted in CNL 219. Users are
advised not to use it any longer and to replace it in older programs.
No maintenance for it will take place and it will eventually disappear.
\\[3mm]
Suggested replacement: {\tt FUNLUX} (V152)}}
\end{center}
{\tt FUNRAN} generates random numbers distributed according to any
(one-dimensional) distribution $f(x)$. The distribution is supplied by
the user in the form of a {\tt FUNCTION} subprogram. If the distribution
is known as a histogram only, {\tt HISRAN} (V150) should be used instead.
\Structure
{\tt SUBROUTINE} subprograms \\
User Entry Names: \Rdef{FUNRAN}, \Rdef{FUNPRE}\\
Internal Entry Names: {\tt FUNZER}\\
Files Referenced: Printer\\
External References: \Rind{GAUSS}{D103}, \Rind{RNDM}{V104},
user-supplied {\tt FUNCTION} subprogram \\
{\tt COMMON} Block Names and Lengths: {\tt /FUNINT/ 1}
\Usage
\begin{tabular}{@{\hspace*{8mm}}l@{\qquad}l}
{\tt CALL FUNPRE(F,FSPACE,XLOW,XHIGH)}  & (once for each function) \\
{\tt CALL FUNRAN(FSPACE,XRAN)}          & (for each random number)
\end{tabular}
\begin{DLtt}{12345678}
\item[F] ({\tt REAL}) A name of a {\tt FUNCTION} subprogram
declared {\tt EXTERNAL} in the calling program. This subprogram must
calculate the (non-negative) density function $f(\mathtt{X})$, for all
{\tt X} in the interval $\mathtt{XLOW \leq X \leq XHIGH}$.
\item[FSPACE] ({\tt REAL}) One-dimensional array of length {\tt 100}.
\item[XLOW] ({\tt REAL}) Lower limit of the requested interval.
\item[XHIGH] ({\tt REAL}) Upper limit of the requested interval.
\item[XRAN] ({\tt REAL}) A random number returned by {\tt FUNRAN}.
\end{DLtt}
A call to {\tt FUNPRE} calculates the percentiles of {\tt F} between
{\tt XLOW} and stores them into the array {\tt FSPACE}.
\Method
In {\tt FUNPRE}, the percentiles are calculated using a combination of
trapezoidal and Gaussian integration to a rather high accuracy,
which is printed out by {\tt FUNPRE}. If the desired accuracy is not
obtained, an warning is printed in addition.
\par
Subroutine {\tt FUNRAN} finds the desired random number by calling
{\tt RNDM} (V104) and doing a 4-point interpolation on {\tt FSPACE}
to transform the uniform random number to the distribution specified.
This method produces quite accurately distributed numbers even when
the function {\tt F} is badly skew or spiked as long as the width
of a spike is not less than 1/1000 of the total range.
\newpage
\Errorh
An error message is printed
\begin{DLtt}{1234}
\item[] -- if the integral of the user-supplied function {\tt F} is zero
or negative,
\item[] -- if $\mathtt{XLOW \geq XHIGH}$,
\item[] -- if $\mathtt{F(X) < 0}$ somewhere between {\tt XLOW} and
{\tt XHIGH}.
\end{DLtt}
\Notes
Some additional information which may be of use is contained in
\begin{verbatim}
    COMMON / FUNINT/ FINT
\end{verbatim}
After a call to {\tt FUNPRE}, {\tt FINT} contains the integral of
{\tt F} from {\tt XLOW} to {\tt XHIGH}.
\par
After a call to {\tt FUNRAN}, {\tt FINT} contains the integral of
{\tt F} from {\tt XLOW} to {\tt XRAN}, divided by the total integral
to {\tt XHIGH} (i.e., it will be a number uniformly distributed
between zero and  one).
\\ $\bullet$
