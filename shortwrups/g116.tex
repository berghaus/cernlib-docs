%  04 jan 95    ksk
\Version{VVILOV}               \Routid{G116}
\Keywords{DENSITY DISTRIBUTION LANDAU VAVILOV}
\Author{B. Schorr, K.S. K\"olbig}             \Library{MATHLIB}
\Submitter{K.S. K\"olbig}              \Submitted{10.12.1993}
\Language{Fortran}          % \Revised{}
\Cernhead{Vavilov Density and Distribution Functions}
The {\tt VVILOV} package contains subprograms for the calculation of
the Vavilov density and distribution functions. Though generally more
accurate, these routines are considerably slower than those in
{\tt VAVLOV} (G115).
\par
For $\kappa>0$ and $0 \le \beta^2 \le 1$,
the Vavilov density function is {\it mathematically} defined by
$$ \phi_V(\lambda;\kappa,\beta^2) \ = \ \displaystyle \frac{1}{2\pi i}\,
\int_{c-i\infty}^{c+i\infty} e^{\lambda s}\,f(s;\kappa,\beta^2)\,ds,$$
where $c$ is an arbitrary real constant and
$$ f(s;\kappa,\beta^2) \ = \ \displaystyle
C(\kappa,\beta^2)\,\exp \left\{s \ln \kappa + (s+\kappa \beta^2)\,
\left[ \ln \left(\frac{s}{\kappa}\right)
+E_1\left(\frac{s}{\kappa}\right) \right]
-\kappa\,\exp \left(-\frac{s}{\kappa}\right) \right\}. $$
$ E_1(x)=\int_0^x t^{-1}\,(1-e^{-t})\,dt$ is the exponential integral,
$C(\kappa,\beta^2)=\exp\{\kappa(1+\beta^2 \gamma)\}$,
and $\gamma=0.57721\dots\,$ is Euler's constant.
\par
The Vavilov distribution function is defined by
$$ \Phi_V(\lambda;\kappa,\beta^2) \ = \ \displaystyle
\int_{-\infty}^\lambda \phi_V(\lambda;\kappa,\beta^2)\,d\lambda.$$
\Structure
{\tt SUBROUTINE} and {\tt FUNCTION} subprograms \\
User Entry Names: \Rdef{VVISET}, \Rdef{VVIDEN}, \Rdef{VVIDIS} \\
Internal Entry Names:  {\tt G116F1}, {\tt G116F2} \\
External References: \Rind{RZERO}{C205},
                     \Rind{RSININ}{C336}, \Rind{RCOSIN}{C336},
                     \Rind{REXPIN}{C337} \\
{\tt COMMON} Block Names and Lenghts: {\tt /G116C1/ 322}
\Usage
\begin{verbatim}
    CALL VVISET(RKAPPA,BETA2,MODE,XL,XU)
\end{verbatim}
sets auxiliary quantities used in {\tt VVIDEN} and {\tt VVIDIS};
this call has to precede a reference to either of these entries.
\begin{DLtt}{12345678}
\item[RKAPPA] The variable $\kappa$ (the straggling parameter);
($0.01 \le \kappa \le 12$).
\item[BETA2] The variable $\beta^2$ (the square of velocity in unit $c$);
($0 \le \beta^2 \le 1$).
\item[MODE]
$\mathtt{= 0}$ if {\tt VVIDEN} is referenced after the call to
{\tt VVISET}; \\
$\mathtt{= 1}$ if {\tt VVIDIS} is referenced after the call to
{\tt VVISET}.
\item[XL,XU] On exit, {\tt XL} and {\tt XU} contain a lower and
upper limit as defined below.
\end{DLtt}
In any arithmetic expression,
\begin{center}
\begin{tabular}{r@{\qquad has the value \qquad}l}
{\tt VVIDEN(X)} & $\phi_V(\mathtt{X;RKAPPA,BETA2})$, \\
{\tt VVIDIS(X)} & $\Phi_V(\mathtt{X;RKAPPA,BETA2})$,
\end{tabular} \end{center}
\newpage
By definition
\begin{center}
\begin{tabular}{r@{\quad if \quad}l@{\qquad}r@{\quad if \quad}l}
$\mathtt{VVIDEN(X)=0}$ & $\mathtt{X<XL}$; &
$\mathtt{VVIDIS(X)=0}$ & $\mathtt{X<XL}$; \\
$\mathtt{VVIDEN(X)=0}$ & $\mathtt{X>XU}$; &
$\mathtt{VVIDIS(X)=1}$ & $\mathtt{X>XU}$.
\end{tabular} \end{center}
\par
{\tt RKAPPA}, {\tt BETA2}, {\tt XL} and {\tt XU} are defined by the last
call to {\tt VVISET} (with $\mathtt{MODE=0}$) prior to a reference
to {\tt VVIDEN}, or (with $\mathtt{MODE=1}$) prior to a reference
to {\tt VVIDIS}.
\par
{\tt VVIDEN}, {\tt VVIDIS}
and {\tt X}, {\tt RKAPPA}, {\tt BETA2}, {\tt XL}, {\tt XU} are of type
{\tt REAL}, and {\tt MODE} is of type {\tt INTEGER}.
\Method
The method, based on Fourier expansions, is described in Ref. 1.
\Accuracy
About five significant digits are usually accurate.
\Errorh
Error {\tt G116.1}: $\mathtt{\kappa<0.01}$ or $\mathtt{\kappa>10}$. \\
Error {\tt G116.2}: $\mathtt{\beta^2>1}$. \\
These errors can occur when calling {\tt VVISET}. In both cases,
a message is written on {\tt Unit 6}, unless subroutine {\tt MTLSET}
(N002) has been called.
\Notes
\begin{enumerate}
\item Representing the Vavilov functions by approximations which are
both fast and accurate is a difficult task. These
routines, though rather accurate, are slow. If speed
is of higher importance than accuracy, and for calculating Vavilov
random numbers, use {\tt VAVLOV} (G115).
\item For $\kappa \le 0.01$, the Vavilov distribution can be
replaced by the Landau distribution ({\tt LANDAU} (G110)),
taking into account that $\lambda_V=(\lambda_L-\ln \kappa)/\kappa$.
\item For $\kappa \ge 10$, the Vavilov distribution can be
replaced by the Gaussian distribution with mean \\
$\mu=\gamma -1-\beta^2-\ln \kappa$ and variance
$\sigma^2=(2-\beta^2)/(2\kappa)$.
\end{enumerate}
\Refer
\begin{enumerate}
\item B. Schorr, Programs for the Landau and the Vavilov distributions
and the corresponding random numbers,
Computer Phys. Comm. {\bf 7} (1974) 215--224.
\end{enumerate}
$\bullet$
