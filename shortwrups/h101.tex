%  18 oct 94  ksk
\Version {RSMPLX}                           \Routid{H101}
\Keywords{LINEAR OPTIMIZATION SIMPLEX CONSTRAINTS INEQUALITY}
\Author{M. Gyr}                \Library{MATHLIB}
\Submitter{K.S. K\"olbig}                     \Submitted{15.02.1994}
\Language{Fortran}                 %       \Revised{}
\Cernhead{Linear Optimization Using the Simplex Algorithm}
Subroutine subprograms {\tt RSMPLX} and {\tt DSMPLX}
calculate the quantities
$x_1,x_2,\ldots,x_m$ for which the linear form, or objective function,
$$z = z_0 - \sum_{i=1}^m b_i x_i$$
assumes a {\it maximum} value subject to the $n_1$ inequality
constraints
$$ \sum ^m_{i=1} a_{ik}x_i \le c_k \qquad  (k = 1,2,\ldots,n_1)$$
and the $n-n_1$ equality constraints
$$ \sum^m_{i=1}a_{ik} x_i = c_k \qquad (k = n_1+1,n_1+2,\ldots,n).$$
A number $m_1 \le m$ of the variables $x_i,\,(i=1,2,\ldots,m_1)$
can be restricted to non-negative values ($x_i \ge 0$). The remaining
$m-m_1$ variables $x_i,\,(i=m_1+1,\ldots,m)$ are then unrestricted
($-\infty <x_i<\infty$). In the case $m_1=0$, all variables $x_i$
are unrestricted. These subprograms can also be used for the so-called
degenerate case.
\par
On computers other than CDC or Cray, only the double-precision
version {\tt DSMPLX} is available. On CDC and Cray computers,
only the single precision version {\tt RSMPLX} is available.
\Structure
{\tt SUBROUTINE} subprograms \\
User Entry Names: \Rdef{RSMPLX}, \Rdef{DSMPLX} \\
Internal Entry Names: {\tt H101S1}, {\tt H101S2} \\
Files  Referenced:  {\tt Unit 6} \\
External References: \Rind{MTLMTR}{N002}, \Rind{ABEND}{Z035}
\Usage
For $\mathtt{t=R}$ (type {\tt REAL}), $\mathtt{t=D}$ (type
{\tt DOUBLE PRECISION}),
\begin{verbatim}
    CALL tSMPLX(A,B,C,Z0,IDA,M,M1,N,N1,LW,IDW,W,X,Z,ITYPE)
\end{verbatim}
\begin{DLtt}{12345}
\item[A] (type according to {\tt t}) Two-dimensional array of dimension
$\mathtt{(IDA,\ge N)}$. Contains, on entry, the coefficients
$a_{i,k}, (i = 1,2,\ldots,m;\,k =1,2,\ldots,n)$. Destroyed during
execution.
\item[B] (type according to {\tt t}) One-dimensional array of dimension
$\mathtt{\ge M}$. Contains, on entry, the coefficients
$b_i, (i = 1,2,\ldots,m)$. Destroyed during execution.
\item[C] (type according to {\tt t}) One-dimensional array of dimension
$\mathtt{\ge N}$. Contains, on entry, the coefficients
$c_k, (k = 1,2,\ldots,n)$. Destroyed during execution.
\item[Z0] (type according to {\tt t}) Contains, on entry,
the initial value of the objective function.
\item[IDA] ({\tt INTEGER}) Declared first dimension of {\tt A} in the
calling program ($\mathtt{IDA \ge M}$).
\item[M] ({\tt INTEGER}) The total number $m$ of variables $x_i$
($\mathtt{M \ge 0}$).
\item[M1] ({\tt INTEGER}) The number $m_1$ of restricted variables
$x_i \ge 0$ ($\mathtt{0 \le M1 \le M})$.
\item[N] ({\tt INTEGER}) The total number $n$ of constraints
($\mathtt{N \ge 0}$).
\item[N1] ({\tt INTEGER}) The number $n_1$ of inequality constraints
($\mathtt{0 \le N1 \le N})$.
\item[LW] ({\tt INTEGER}) Two-dimensional array of dimension
$\mathtt{(IDW, \ge 5)}$. Used as working space.
\item[IDW] ({\tt INTEGER}) Declared first dimension of {\tt LW} in the
calling program ($\mathtt{IDW \ge M+2*N}$).
\item[W] (type according to {\tt t}) One-dimensional array of dimension
$\mathtt{\ge M+N}$. Used as working space.
\item[X] (type according to {\tt t}) One-dimensional array of dimension
$\mathtt{\ge M+N}$. If $\mathtt{ITYPE=1}$ or $\mathtt{ITYPE=2}$,
its first $m$ elements {\tt X(1),\ldots,X(M)} contain, on exit,
the or a solution $x_1,\ldots,x_m$, respectively. The next
$n$ elements {\tt X(M+1),\ldots,X(M+N)} contain the
values of the so-called slack variables $x_{m+1},\ldots,x_{m+n}$.
If $\mathtt{ITYPE=3}$ or $\mathtt{ITYPE=4}$, the elements
$\mathtt{X(1),\ldots,X(M+N)}$ are undefined.
\item[Z] (type according to {\tt t})
If $\mathtt{ITYPE=1}$ or $\mathtt{ITYPE=2}$,
{\tt Z} contains, on exit, the result $z$ of the objective function.
Undefined for $\mathtt{ITYPE=3}$ and $\mathtt{ITYPE=4}$.
\item[ITYPE] ({\tt INTEGER}) Defines, on exit, the type of the result:
\item[] $\mathtt{=1:}$ There is exactly one finite solution.
\item[] $\mathtt{=2:}$ There is more than one solution.
\item[] $\mathtt{=3:}$ There is no finite solution.
\item[] $\mathtt{=4:}$ There is no feasable initial solution.
\end{DLtt}
\Method
The method is described in Ref. 1 and Ref. 2.
\Errorh
Error {\tt H101.1}: $\mathtt{M \le 0}$ or $\mathtt{N \le 0}$. \\
Error {\tt H101.2}: $\mathtt{M1 < 0}$ or $\mathtt{M1 > M}$
or $\mathtt{N1 < 0}$ or $\mathtt{N1 > N}$. \\
In both cases, a message is written on {\tt Unit 6}, unless subroutine
{\tt MTLSET} (N002) has been called.
\Refer
\begin{enumerate}
\item H.P. K\"unzi, H.G. Tzschach and C.A. Zehnder, Numerical
methods of mathematical optimization, (Academic Press, New York 1968)
\item E. Stiefel, Einf\"uhrung in die Numerische Mathematik,
(B.G. Teubner, Stuttgart 1965)
\end{enumerate}
$\bullet$
