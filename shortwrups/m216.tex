\Version{GETWI}                    \Routid{M216}
\Keywords{CDC NOS READ CONVERT TRANSLATE SCOPE}
\Author{E. McIntosh, E. Perotto}      \Library{KERNLIB, IBM only}
\Submitter{}                \Submitted{08.06.1989}
\Language{Fortran and Assembler}              %\Revised{}
\Cernhead {Read and Convert a CDC NOS/BE or Scope 2 W/I File}
{\tt GETWI} reads a CDC tape file in W/I format (or an IBM disk image of
such a file) and returns one CDC logical record on each call.
Each 60-bit data item is converted to a 32-bit IBM integer or
floating-point number and is stored in the ({\tt REAL}) array
{\tt RESULT}.
Note that if the input data contains mixed integers and floating-point
numbers, so will {\tt RESULT}.
\par
The original CDC data, 60 bits right-justified in a 64 bit ({\tt DOUBLE
PRECISION}) field, is also returned to the ({\tt DOUBLE PRECISION}) array
{\tt ARRAY} in case the user wishes to make his own conversion (e.g.
with {\tt CVT76A} (M216)).
\par
{\tt READWI} reads a CDC tape file in W/I format (or an IBM disk image
of such a file) and returns one CDC logical record on each call. The
record is stored in the ({\tt DOUBLE PRECISION}) array {\tt ARRAY},
one CDC 60-bit word right-justified in each element.
\par
{\tt CONCDC} expects 60-bit CDC data stored right-justified in 64-bits
(as for example data read by {\tt READWI}). Each 60-bit data item
is converted to a 32-bit IBM integer or floating-point number
and is stored in the ({\tt REAL}) array {\tt RESULT}. Note that if the
input data contains mixed integers and floating-point numbers, so will
{\tt RESULT}.
\par
{\tt CVT76A} expects 60-bit CDC data stored right-justified in 64 bits.
Each 60-bit data item is converted either to one IBM double-precision
floating-point number or a 32-bit IBM integer stored in the lower half
of an IBM double-precision floating-point number. The input array is
replaced by the converted values of its elements.
\par
The conversions will be done correctly (with loss of precision)
providing all CDC integers are in the range $-2^{32}$ to $2^{32}-1$
and there are no underflows (CDC floating-point numbers with a zero
exponent). Suitable diagnostic messages are issued provided the error
logging option is selected. If the modulus of the CDC floating-point
numbers is less than about {\tt 0.6E-78} they are converted to {\tt 0.0}
(as are underflows and indefinites) and if greater than {\tt 0.7E76}
to approximately this value (as are infinities).
\par
The IBM file should be declared {\tt RECFM U}, {\tt BLKSIZE 3840}
corresponding to the {\tt 512}-word block structure of CDC W/I files.
\Structure
{\tt SUBROUTINE} subprograms \\
User Entry Names: \Rdef{GETWI}, \Rdef{CONCDC}, \Rdef{CVT76A},
\Rdef{READWI}, \Rdef{CDCA4} \\
Internal Entry Names: \Lit{ERRLOG}, \Lit{S\dollar{}XTO4} \\
External References: \Rind{IOGET} (Z300) \\
Files referenced:
\parbox[t]{135mm}
{{\tt IOFILEnn} where $\mathtt{LUN=nn}$ and, standard output stream
(if $\mathtt{LERR=0}$) or {\tt FTmmF001} (if $\mathtt{LERR=mm}$) or none
(if $\mathtt{ LERR < 0}$).}
\newpage
\Usage
{\bf Subroutine GETWI:}
\begin{verbatim}
    CALL GETWI(LUN,ARRAY,MAXN,RESULT,IERR,N,LERR)
\end{verbatim}
\begin{DLtt}{12345678}
\item[LUN] ({\tt INTEGER}) Stream number of input file ({\tt IOFILEnn}),
$\mathtt{1 \leq LUN \leq 99}$.
\item[ARRAY] ({\tt DOUBLE PRECISION}) Array of length $\mathtt{\geq MAXN}$
for the input record (see above).
\item[MAXN] ({\tt INTEGER}) Maximum logical record length.
\item[RESULT] ({\tt REAL}) Array for the IBM 32-bit results.
\item[IERR] ({\tt INTEGER}) Set non-zero if any errors occurred;
it will be set to the {\tt READWI} or the {\tt CONCDC} error code
(see {\bf Error Handling}).
\item[N] ({\tt INTEGER}) Set to the actual logical record length in
words; i.e. the number of (64/32 bit) results returned to
{\tt ARRAY/RESULT}.
\item[LERR] ({\tt INTEGER}) List flag.  \\
$\mathtt{< 0:}$ No errors are listed. \\
$\mathtt{= 0:}$ Errors are listed on standard output. \\
$\mathtt{= mm > 0:}$ Errors are listed on {\tt FTmmF001}.
\end{DLtt}
\Method
{\tt GETWI} first calls {\tt READWI} to read in a CDC Logical Record in
unpacked format (60 bits right-justified in 64) to {\tt ARRAY}.
{\tt CONCDC} is then called to convert and store the 32-bit results in
{\tt RESULT}. See the write-ups of {\tt READWI} and {\tt CONCDC} below
for more information.
\Restrict
Apart from the limitations on precision and range of the
numbers to be converted as outlined above, only one input file
({\tt LUN}) may be used by the calling program. Attempts to read from
more than one input file will produce undefined results and spurious
error messages.
\Accuracy
CDC floating point-numbers will be represented correctly to about 6
decimal digits, but note that they can be converted to IBM double-
precision format by {\tt CVT76A}.
\Timing
Approximately 1 CPU second (3090/7880) per Megabyte, i.e. 5
normalised 168 units/Megabyte if there are few errors.
\Errorh
First note that the errors reported are detected by either {\tt READWI}
or {\tt CONCDC}. The value returned is either the {\tt READWI} error
code ($\mathtt{1-9}$) or the {\tt CONCDC} error code (${\tt 11-14}$).
{\tt N} is always zero when any {\tt READWI} error is detected.
The subroutine will attempt to continue successfully reading on calls
subsequent to any error but data may be lost. No recovery is possible
from errors like End-of-Data or bad stream number. Some {\tt READWI}
errors correspond to {\tt IOPACK} errors and {\tt IOPACK} issues its
own message.
\par
The errors reported by {\tt GETWI} from {\tt READWI} are: \\[2mm]
\begin{tabular}{@{\hspace*{10mm}}l@{\qquad}l}
{\tt IERR} & Meaning \\[2mm]
1 & End of Data Set (End-of-File, {\tt IOPACK IOERR 2}) \\
2 & End of CDC Partition \\
3 & End of CDC Section  \\
4 & End of Information ({\tt IOPACK IOERR 4}) \\
5 & I/O error (e.g. parity, {\tt IOPACK IOERR 3}) \\
6 & CDC W Control Word Error \\
7 & I/O error (any other {\tt IOPACK} error) \\
8 & (Block Sequence Error, listed but never returned to the caller) \\
9 & CDC I Control Word Error
\end{tabular} \\[2mm]
No data ($\mathtt{N=0}$) is ever returned on these errors, and they are
listed with additional diagnostic information if $\mathtt{LERR \geq 0}$.
\par
Data is always returned when a {\tt CONCDC} error is reported, and
indeed the entire record is always converted, and only the last error
returned in {\tt IERR}. All the errors are listed if $\mathtt{LERR \geq 0}$.
\par
The {\tt CONCDC} errors are: \\[2mm]
\begin{tabular}{@{\hspace*{10mm}}l@{\qquad}l}
{\tt IERR} & Meaning \\[2mm]
11 & CDC overflow (zero-exponent) or integer too large \\
12 & CDC infinite value (exponent of {\tt 3777} octal) \\
13 & CDC indefinite value (exponent of {\tt 1777} octal) \\
14 & CDC number outside IBM range ({\tt 0.6E-78 to 0.7E76}) \\
\end{tabular} \\[2mm]
\Examples
The following VM job would read a CDC labelled W/I staged tape and
create a (large) VM file ({\tt V} format) of 32-bit data items, if there
were sufficient disk space available and line limit was not exceeded.
\begin{verbatim}
/*BATCH TIME 1 TAPES 1 */
/*BATCH SEND 'TESTWI Ftn *' */
TRACE C
'VFORT TESTWI'
'CERNLIB'
'EXEC STAGE PRE IOFILE10 crnvsn SL DSN mydsn FSEQ 1 RECFM U BLK 3840'
'FILEDEF FT11F001 DISK IBM DATA (RECFM V LRECL 16000'
'LOAD TESTWI (START'
\end{verbatim}
where the file {\tt TESTWI FORTRAN} is the following program:
\begin{verbatim}
      PROGRAM TESTWI
      DOUBLE PRECISION ARRAY(2000)
      DIMENSION OUT(2000),IOUT(2000)
      EQUIVALENCE (OUT(1),IOUT(1))
      DATA N /100/
C     number of records for testing
      DO 1 I = 1,N
      CALL GETWI(10,ARRAY,2000,OUT,IERR,NWORDS,0)
C     read a logical record from IOFILE10, maximum 2000 words/record,
C     store the record in ARRAY 60 bits in 64, rigth-justified;
C     store the converted data in OUT, the record length in NWORDS,
C     and list any errors on standard output.
      IF(IERR .NE. 0 .OR. I .LE. 50) THEN
C     Messages for any error or for the first 50 records anyway
        WRITE (*,100) I,IERR,NWORDS,(ARRAY(J),J=1,10)
C     Now the first 50 converted records anyway
        WRITE(*,101) (IOUT(J),OUT(J),J=1,NWORDS)
      ENDIF
  100 FORMAT(/,1X,'RECNO',I10,'   ERRCODE',I5,'   LENGTH',I10//
     1       1X,5Z20/1X,5Z20/)
  101 FORMAT(4(I15,E17.7))
      WRITE(11) (OUT(J),J=1,NWORDS)
      IF(IERR .EQ. 1 .OR .IERR .EQ. 4) STOP
    1 CONTINUE
      END
\end{verbatim}
{\bf Subroutine READWI:}
\begin{verbatim}
    CALL READWI(LUN,ARRAY,MAXN,IERR,N,LERR)
\end{verbatim}
\begin{DLtt}{12345678}
\item[LUN] ({\tt INTEGER}) Stream number of input file ({\tt IOFILEnn}),
$\mathtt{1 \leq LUN \leq 99}$.
\item[ARRAY] ({\tt DOUBLE PRECISION}) Array of length $\mathtt{\geq MAXN}$
for the input record.
\item[MAXN] ({\tt INTEGER}) Maximum logical record length.
\item[IERR] ({\tt INTEGER}) Set non-zero if any errors occurred.
(See {\bf Error Handling} for {\tt GETWI} above.)
\item[N] ({\tt INTEGER}) Set to the actual logical record length in
words; i.e. the number of 64 bit results returned to {\tt ARRAY}.
\item[LERR] ({\tt INTEGER}) List flag. \\
$\mathtt{< 0:}$ No errors are listed. \\
$\mathtt{= 0:}$ Errors are listed on standard output. \\
$\mathtt{= mm > 0:}$ Errors are listed on {\tt FTmmF001}.
\end{DLtt}
\Method
The input data is read one block at a time from the input file. The
I block control word is decoded and verified, irrespective of
whether it conforms to NOS/BE or SCOPE 2 convention. The W
record/segment control words are then used to deliver a complete
logical record to the caller.
\par
Various consistency checks are performed, and are reported to the caller
via the flag {\tt IERR}, and are listed by the internal routine
{\tt ERRLOG} if this option is selected.
\Restrict
Only one input file {\tt (LUN)} may be used by the calling program.
Attempts to read from more than one input file will produce undefined
results and spurious error messages.
\Timing
Approximately 0.25 CPU second (3090/7880) per Megabyte, i.e.
1.25 normalised 168 units/Mega- \\
byte if there are few errors.
\Errorh
{\tt N} is always zero when any {\tt READWI} error is detected. The
subroutine will attempt to continue successfully reading on calls
subsequent to any error but data may be lost. No recovery is
possible from errors like End-of-Data or bad stream number. Some
{\tt READWI} errors correspond to {\tt IOPACK} errors and {\tt IOPACK}
issues its own message.
\Examples
The following VM job would read a CDC labelled W/I staged tape:
\begin{verbatim}
/*BATCH TIME 1 TAPES 1 */
/*BATCH SEND 'TESTWI Ftn *' */
TRACE C
'VFORT TESTWI'
'CERNLIB'
'EXEC STAGE PRE IOFILE10 crnvsn SL DSN mydsn FSEQ 1 RECFM U BLK 3840'
'LOAD TESTWI (START'
\end{verbatim}
where the file {\tt TESTWI FORTRAN} is the following program:
\begin{verbatim}
      PROGRAM TESTWI
      DOUBLE PRECISION ARRAY(2000)
      DATA N /100/
C     Number of records for testing
      IEOF=0
      DO 1 I = 1,N
      CALL READWI(10,ARRAY,2000,IERR,NWORDS,0)
C     Read a logical record from IOFILE10, maximum 2000 words/record
C     store the record in ARRAY 60 bits in 64 right-justified, the
C     record length in NWORDS, and list any errors on standard output.
      IF(IERR .NE. 0 .OR. I .LE. 50) THEN
C     Messages for any error or for the first 50 records anyway
        WRITE(*,100) I,IERR,NWORDS,(ARRAY(J),J=1,10)
      ENDIF
  100 FORMAT(/1X,'RECNO',I10,'   ERRCODE',I5,'   LENGTH',I10//
     1        1X,5Z20/1X,5Z20/)
      IF(IERR .EQ. 1 .AND. IEOF .EQ. 1 .OR. IERR .EQ. 4) STOP
      IEOF=IERR
    1 CONTINUE
      END
\end{verbatim}
{\bf Subroutine CONCDC:}
\begin{verbatim}
    CALL CONCDC(ARRAY,N,RESULT,IERR,LERR)
\end{verbatim}
\begin{DLtt}{12345678}
\item[ARRAY] ({\tt DOUBLE PRECISION}) Array of length $\mathtt{\geq N}$
for the input data.
\item[N] ({\tt INTEGER)} Number of items to convert.
\item[RESULT] ({\tt REAL}) Array for the IBM 32-bit results.
\item[IERR] ({\tt INTEGER}) Set non-zero if any errors occurred.
\item[LERR] ({\tt INTEGER}) List flag. \\
$\mathtt{< 0:}$ No errors are listed. \\
$\mathtt{= 0:}$ Errors are listed on standard output. \\
$\mathtt{= mm > 0:}$ Errors are listed on {\tt FTmmF001}.
\end{DLtt}
\Method
{\tt CONCDC} first checks the CDC sign bit and internally
complements the data item if the sign bit is set. If the data is a
CDC integer in the range $-2^{32}$ to $2^{32}-1$ an integer result
is returned. Otherwise the data item is assumed to be floating-point.
\par
After special treatment of the CDC cases (infinite, indefinite, and
underflow), {\tt CVT76A} is called to convert other data items. A single
precision result is then returned.
\Restrict
The limitations on precision and range of the numbers to be converted as
outlined above.
\Accuracy
CDC floating-point numbers will be represented correctly to about 6
decimal digits, but note that they can be converted to IBM
double-precision by {\tt CVT76A}.
\Timing
Approximately 6 micro-seconds (3090/7880) per item i.e. 30
normalised 168 units for 1 million items if there are few errors.
\Errorh
Data is always returned when a {\tt CONCDC} error is reported, and
indeed the entire record is always converted, and only the last error
returned in {\tt IERR}. All the errors are listed if $\mathtt{LERR \geq 0}$.
\par
The {\tt CONCDC} errors are as listed above. The error message shows the
CDC word interpreted as an integer, as a floating-point and as 10 CDC
display code characters (display code can be converted to EBCDIC with
{\tt CDC\$A4}).
\Examples
The following VM job would read a CDC labelled W/I staged tape and
create a (large) VM file ({\tt V} format) of 32-bit data items, if there
were sufficient disk space available and line limit was not exceeded.
\begin{verbatim}
/*BATCH TIME 1 TAPES 1 */
/*BATCH SEND 'TESTWI Ftn *' */
TRACE C
'VFORT TESTWI'
'CERNLIB'
'EXEC STAGE PRE IOFILE10 crnvsn SL DSN mydsn FSEQ 1 RECFM U BLK 3840'
'FILEDEF FT11F001 DISK IBM DATA (RECFM V LRECL 16000'
'LOAD TESTWI (START'
\end{verbatim}
where the file {\tt TESTWI FORTRAN} is the following program:
\begin{verbatim}
       PROGRAM TESTWI
       DOUBLE PRECISION ARRAY(2000)
       DIMENSION OUT(2000),IOUT(2000)
       EQUIVALENCE (OUT(1),IOUT(1))
       DATA N /100/
 C     Number of records for testing
       IEOF=0
       DO 1 I = 1,N
       CALL READWI(10,ARRAY,2000,IERR,NWORDS,0)
 C     Read a logical record from IOFILE10, maximum 2000 words/record,
 C     store the record in ARRAY, 60 bits in 64 right-justified, the
 C     record length in NWORDS, and list any errors on standard output.
       IF(IERR .EQ. 0) CALL CONCDC (ARRAY,NWORDS,OUT,IERR,0)
       IF(IERR .NE. 0 .OR. I .LE. 50) THEN
 C     Messages for any error of for the first 50 records anyway
         WRITE(*,100) I,IERR,NWORDS,(ARRAY(J),J=1,10)
 C     Now list the first 50 converted records anyway
         WRITE(*,101) (IOUT(J),OUT(J),J=1,NWORDS)
       ENDIF
  100  FORMAT(/1X,'RECNO',I10,'   ERRCODE',I5,'   LENGTH',I10//
      1        1X,5Z20/1X,5Z20/)
  101  FORMAT(4(I15,E17.7))
       WRITE(11) (OUT(J),J=1,NWORDS)
       IF(IERR .EQ. 1 .AND. IEOF .EQ. 1 .OR. IERR .EQ. 4) STOP
       IEOF=IERR
     1 CONTINUE
       END
\end{verbatim}
{\bf Subroutine CVT76A:}
\begin{verbatim}
    CALL CVT76A(ARRAY,N)
\end{verbatim}
\begin{DLtt}{12345678}
\item[ARRAY] ({\tt DOUBLE PRECISION}) Array of length $\mathtt{\geq N}$
for the input data whose entries are replaced by the converted values.
\item[N] ({\tt INTEGER}) Number of items to convert.
\end{DLtt}
\Method
{\tt CVT76A} first checks on the CDC sign bit and internally complements
the data item if its sign bit is set. If the data is a CDC integer in
the range $-2^{48}$ to $2^{48}-1$ an integer result is assumed; otherwise
the data item is assumed to be floating-point.
\par
Floating-point numbers are returned as IBM {\tt DOUBLE PRECISION}
in the same elements of {\tt ARRAY}. Integers are put back as full word
integers in the lower half of the {\tt DOUBLE PRECISION} element of
{\tt ARRAY} with binary zero in the upper half. Integers outside the
IBM range of $-2^{32}$ to $2^{32}-1$ are returned as the maximum IBM
value
with the correct sign bit set. \\[2mm]
{\bf Subroutine CDC\$A4:}
\begin{verbatim}
    CALL CDC$A4(CDCBUF,IBMBUF,LEN,BYTWD,FILL)
\end{verbatim}
\begin{DLtt}{12345678}
\item[CDCBUF] Array of 64-bit words containig a 60-bit CDC word
right-justified (output of {\tt READWI}).
\item[IBMBUF] Any memory space of at least {\tt LEN*BYTWD} bytes.
\item[LEN] Number of CDC words to decode.
\item[BYTWD] Number of characters taken from each CDC word (Optional,
defaults to {\tt 4}).
\item[FILL] Optional value to be stored in {\tt CDCBUF} (padded to 64
bits on the left) to avoid {\tt Error 14} from {\tt CONCDC}.
\end{DLtt}
\Method
This routine expects a 60-bit CDC data word stored right-justified
in a 64-bit IBM double-word (as data read by {\tt READWI}).
This CDC word contains 10 6-bit characters; they are translated
from CDC 6-bit Display code to EBCDIC 8-bit code, and assembled
contiguously into {\tt IBMBUF} (whose word boundaries are ignored). Only
if $\mathtt{BYTWD=4}$  the CDC and IBM word counts are equal, with the loss
of the 6 rightmost CDC characters (normally not used if the program
was originally written for 32-bit machines). If $\mathtt{BYTWD > 10}$ blanks
are inserted to the right of the 10 converted characters of each word.
\Timing
Approximately 0.51 seconds (3090) per million CDC words (records of
512), i.e. 2.6 normalised 168 units.
\\ $\bullet$
