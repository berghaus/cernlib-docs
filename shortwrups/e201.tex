%  14.10.94  ksk
\Version {RLSQPM}            \Routid{E201}
\Keywords{FIT LEAST SQUARES POLYNOMIAL}
\Author{K.S. K\"olbig}              \Library{MATHLIB}
\Submitter{}                \Submitted{01.12.1994}
\Language{Fortran}        %    \Revised{}
\Cernhead {Least Squares Polynomial Fit}
Subroutine subprograms {\tt RLSQPM} and {\tt DLSQPM} fit a polynomial
$$p_m(x) \ = \ \sum_{j=0}^m a_jx^j$$
of degree $m$ to $n$ equally-weighted data points ($x_i,y_i$).
The calculated coefficients $a_j$ are such that
$$S_m^2 \ = \ \sum_{i=1}^n \big(y_i-p_m(x_i)\big)^2 \ = \ \min.$$
\par
Subroutine subprograms {\tt RLSQP1} and {\tt DLSQP1} fit a straight
line $p_1(x) \ = \ a_0+a_1x$ to $n$ such points.
\par
Subroutine subprograms {\tt RLSQP2} and {\tt DLSQP2} fit a parabola
$p_2(x) \ = \ a_0+a_1x+a_2x^2$ to $n$ such points.
\par
An estimate $s \ = \ \sqrt{S_m^2/(n-m-1)}$
of the standard deviation $\sigma$ is calculated.
\par
On CDC and Cray computers, the double-precision
versions {\tt DLSQPM}, {\tt DLSQP1} and {\tt DLSQP2} are not available.
\Structure
{\tt SUBROUTINE} subprograms \\
User Entry Names: \Rdef{RLSQPM}, \Rdef{RLSQP1}, \Rdef{RLSQP2},
                  \Rdef{DLSQPM}, \Rdef{DLSQP1}, \Rdef{DLSQP2} \\
External References:
\Rind{RVSET}{F002}, \Rind{DVSET}{F002}, \Rind{DVSUM}{F002}, \Rind{DVMPY}{F002},
\Rind{DSEQN}{F012}
\Usage
For $\mathtt{t=R}$ (type {\tt REAL}), $\mathtt{t=D}$ (type
{\tt DOUBLE PRECISION}),
\begin{verbatim}
    CALL tLSQPM(N,X,Y,M,A,SD,IFAIL)
    CALL tLSQP1(N,X,Y,A0,A1,SD,IFAIL)
    CALL tLSQP2(N,X,Y,A0,A1,A2,SD,IFAIL)
\end{verbatim}
\begin{DLtt}{12345678}
\item[N] ({\tt INTEGER}) Number $n$ of data points.
\item[X] (type according to {\tt t})
One-dimensional array of length $\mathtt{\ge N}$.
On entry, {\tt X(i)} contains the abscissas $x_i,\,(i=1,2,\ldots,n)$.
\item[Y] (type according to {\tt t})
One-dimensional array of length $\mathtt{\ge N}$.
On entry, {\tt Y(i)} contains the ordinates $y_i,\,(i=1,2,\ldots,n)$.
\item[M] ({\tt INTEGER}) Degree $m$ of the polynomial to be fitted.
\item[A] (type according to {\tt t}) One-dimensional array of
dimension {\tt (0:d)}, where $\mathtt{d \ge M}$. Contains, on exit,
in {\tt A(j)} the coefficients $a_j,\,(j = 0,1,\ldots,m)$.
\item[A0,A1,A2] (type according to {\tt t})
Contain, on exit, the coefficients
$a_0$, $a_1$ for $p_1(x)=a_0+a_1x$ or $a_0,a_1,a_2$ for
$p_2(x)=a_0+a_1x+a_2x^2$, respectively.
\item[SD] (type according to {\tt t}) Contains, on exit,
the estimate $s$.
\item[IFAIL] ({\tt INTEGER}) Error flag. \\
$\mathtt{= 0:}$ Normal case, \\
$\mathtt{= 1:}$ $\mathtt{N \le 1}$ or $\mathtt{M<0}$ or
$\mathtt{M \ge N}$ or $\mathtt{M>20}$, \\
$\mathtt{= -1:}$ The matrix of normal equations is numerically
singular.
\end{DLtt}
In the case $\mathtt{IFAIL \ne 0}$: $\mathtt{M=0}$,
$\mathtt{A(j)=0}$ and $\mathtt{A0=A1=A2=0}$ on exit. \\[3mm]
\newpage
\Method
The normal equations are solved. On computers other than CDC or Cray,
double-precision mode arithmetic is used internally for {\tt RLSQPM},
{\tt RLSQP1} and {\tt RLSQP2}.
\Notes
Meaningful results can usually be obtained only for small values
of $m$ (typically $\mathtt{<10}$).
\Refer
\begin{enumerate}
\item D.H. Menzel, Fundamental formulas of physics, v. 1,
(Dover, New York 1960) 116--122.
\end{enumerate}
$\bullet$
