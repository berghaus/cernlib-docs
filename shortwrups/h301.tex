%24 jan 1994  ksk
\Version{ASSNDX}                           \Routid{H301}
\Keywords{ASSIGNMENT MATRIX PERMUTATION MINIMUM}
\Author{F. Bourgeois}                       \Library{MATHLIB}
\Submitter{K.S. K\"olbig}                 \Submitted{15.02.1994}
\Language{Fortran}                          \Revised{}
\Cernhead {Assignment Problem}
Subroutine subprogram {\tt ASSNDX} solves the so-called
{\it Assignment problem}:
Given an $n \times m$ matrix $A$ of real numbers $a(i,j)$, find either
\begin{enumerate}
\item a set
$\{k(1),k(2),\ldots,k(n)\} \in \{1,2,\ldots,m,0,\ldots,0\}$,
where $0,\ldots,0$ indicates $\max(n-m,0)$ zeros, and where for
non-zero elements $k(p) \ne k(q)$ for $p \ne q$, which minimizes
$$ S \ = \ \sum_{i=1}^n a(i,k(i))$$
assuming that $a(i,0)=0$, or
\item a set
$\{k(1),k(2),\ldots,k(m)\} \in \{1,2,\ldots,n,0,\ldots,0\}$,
where $0,\ldots,0$ indicates $\max(m-n,0)$ zeros, and where for
non-zero elements $k(p) \ne k(q)$ for $p \ne q$, which minimizes
$$ S \ = \ \sum_{j=1}^m a(k(j),j)$$
assuming that $a(0,j)=0$.
\end{enumerate}
\Structure
{\tt SUBROUTINE} subprogram \\
User Entry Names: \Rdef{ASSNDX} \\
Files Referenced: {\tt Unit 6}
\Usage
\begin{verbatim}
    CALL ASSNDX(MODE,A,N,M,IDA,K,SMIN,IW,IDW)
\end{verbatim}
\begin{DLtt}{123456}
\item [MODE] ({\tt INTEGER}) Must be set either {\tt 1} (for case (1)),
or {\tt 2} (for case (2)).
\item[A] ({\tt REAL}) Two-dimensional array of dimension
($\mathtt{IDA,\ge M}$). Must contain, on entry,
the matrix $A$. Destroyed during execution.
\item[N]({\tt INTEGER}) Number $n$ of rows of $A$.
\item[M]({\tt INTEGER}) Number $m$ of columns of $A$.
\item [IDA]({\tt INTEGER}) Declared first dimension of {\tt A}
in the calling program ($\mathtt{IDA \ge N})$.
\item[K] ({\tt INTEGER}) One-dimensional array of length
$\mathtt{\ge \max(N,M)}$. Contains, on exit, the assigned set of
integers $\{k(1),\ldots,k(n)\}$ or $\{k(1),\ldots,k(m)\}$,
respectively.
\item [SMIN] ({\tt REAL}) The calculated minimum value of $S$.
\item[IW] ({\tt INTEGER}) Two-dimensional array of dimension
($\mathtt{IDW,\ge 6}$). Used as working space.
\item [IDW]({\tt INTEGER}) Declared first dimension of {\tt IW}
in the calling program ($\mathtt{IDW \ge \max(N,M)}$).
\end{DLtt}
\Method
The subprogram is based on the Algol procedure given in Ref. 3.
\newpage
\Errorh
Error {\tt H301.1}: $\mathtt{N<1}$ or $\mathtt{M<1}$. \\
A message is written on {\tt Unit 6}, unless subroutine {\tt MTLSET}
(N002) has been called.
\Examples
The following example illustrates a possible use of the subprogram.
A workshop has to carry out $N$ jobs, each of
which can be performed on any of $M (>N)$ available machines.
The cost of performing job $I$ on machine $J$ is $A(I,J)$.
It is required to assign jobs to machines in such a way
as to minimize the total cost.
The solution is obtained by calling the subprogram
with $\mathtt{MODE=1}$ and then assigning job $I$ to machine
$\mathtt{K(I),\,(I=1,2,\ldots,N)}$.
\Refer
\begin{enumerate}
\item J. Munkres, Algorithms for the assignment and
transportation problems, J. SIAM 5 (1957) 32--38.
\item R. Silver, An algorithm for the assignment problem,
Comm. ACM {\bf 3} (1960) 605--606.
\item R. Silver, Algorithm 27 ASSIGNMENT, Collected Algorithms from CACM
(1960).
\end{enumerate}
$\bullet$
