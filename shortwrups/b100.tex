%  07 nov 1995  ksk
\Version{RBINOM}                     \Routid{B100}
\Keywords{BINOMIAL COEFFICIENT}
\Author{K.S. K\"olbig}              \Library{MATHLIB}
\Submitter{}                        \Submitted{15.02.1989}
\Language{Fortran}                  \Revised{15.11.1995}
\Cernhead{Binomial Coefficient}
Function subprograms {\tt RBINOM} and {\tt DBINOM} calculate
the binomial coefficient
$$ \binom{x}{k} \ = \ \left\{ \begin{array}{ll}
 x(x-1)\ldots(x-k+1)/k! & (k>0) \\
 1                      & (k=0) \\
 0                      & (k<0) \
\end{array} \right.$$
for real $x$ and integer $k$.
Function subprogram {\tt KBINOM} calculates the
binomial coefficient only for integer $x=n$.
\par
On CDC and Cray computers, the double-precision version
{\tt DBINOM} is not available.
\Structure
{\tt FUNCTION} subprograms \\
User Entry  Names: \Rdef{RBINOM}, \Rdef{DBINOM}, \Rdef{KBINOM} \\
Obsolete User Entry Names: \Rdef{BINOM} $\equiv$ \Rdef{RBINOM} \\
Files Referenced: {\tt Unit 6}
\Usage
In any arithmetic expression,
\begin{center}
{\tt RBINOM(X,K)}, \quad {\tt DBINOM(X,K)} \quad or \quad
{\tt KBINOM(N,K)}
\end{center}
has the value of the binomial coefficient. {\tt RBINOM} is of type
{\tt REAL}, {\tt DBINOM} is of type {\tt DOUBLE PRECISION}
and {\tt X} has the
same type as the function name. {\tt KBINOM, N} and {\tt K} are of type
{\tt INTEGER}.
\Restrict
Function subprogram {\tt KBINOM} can compute only binomial
coefficients which lie in the integer range of the machine.
\Accuracy
Full machine accuracy.
\Errorh
If the result of {\tt KBINOM} would lie outside the integer range of
the machine, {\tt KBINOM} is set equal to zero and an error
message is printed.
\\ $\bullet$
