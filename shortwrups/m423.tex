\Version{INCBYT}                         \Routid{M423}
\Keywords{INCREMENT BYTE HANDLING PACK VECTOR}
\Author{J. Zoll, P. Rastl}               \Library{KERNLIB}
\Submitter{}                             \Submitted{28.01.1971}
\Language{Fortran or Assembler}    \Revised{16.09.1991}
\Cernhead {Increment a Byte of a Packed Vector}
{\tt INCBYT} allows incrementing a specified byte of a packed byte
vector (cf. {\tt PACBYT} (M422)).
\Structure
{\tt FUNCTION} subprogram \\
User Entry Names: \Rdef{INCBYT}
\Usage
\begin{verbatim}
    LOST = INCBYT(INC,MX,JX,MPACK)
\end{verbatim}
The 3-word vector {\tt MPACK} specifies the packing parameters
(much like for {\tt PACBYT} (M422),
but $\mathtt{NBITS=0}$ is not allowed):
\begin{DLtt}{12345678}
\item [MPACK(1)] $\mathtt{\equiv NBITS}$, number of bits per byte.
\item [MPACK(2)] $\mathtt{\equiv INWORD}$, number of bytes per word.
\item [MPACK(3)] $\mathtt{\equiv MAXCAP}$, the maximum capacity of any byte,
{\tt $\leq$ 2**NBITS$-1$}.
\end{DLtt}
{\tt INCBYT} adds the increment {\tt INC} into the {\tt JX}'th byte of
the packed byte-vector {\tt MX} and returns any byte overflow,
i.e. the part of {\tt INC} which cannot be added into the byte,
because it now contains {\tt MPACK(3)}.
\\ $\bullet$
