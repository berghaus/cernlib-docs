%  10 dec 94  ksk
\Version {TMREAD}                           \Routid{Z313}
\Keywords{TERMINAL DIALOG}
\Author{J. Zoll}                     \Library{KERNLIB}
\Submitter{}                            \Submitted{01.11.1994}
\Language{Fortran}                       \Revised{}
\Cernhead {Terminal Dialogue Routines}
These routines prompt the user on-line to the executing program
for input from the terminal, and read it.
The prompt is written to standard output by calling {\tt TMPRO},
the input is read from standard input with {\tt TMREAD}.
Whether or not standard input is in fact a terminal can be
detected with {\tt INTRAC} (Z044);
if it is not the call to {\tt TMPRO} should be by-passed.
\Structure
{\tt SUBROUTINE} subprograms \\
User Entry Names: \Rdef{TMINIT}, \Rdef{TMPRO}, \Rdef{TMREAD} \\
Files references: standard input, standard output
\Usage
{\bf Initialize the dialogue} \\[3mm]
On some machines it is necessary to switch off buffered mode
on standard output, this is done by calling once, and before
the first call to {\tt TMPRO}:
\begin{verbatim}
    CALL TMINIT (IFINIT)
 
        IFINIT*  is reset to non-zero by TMINIT
\end{verbatim}
{\bf Put the prompt to standard output}
\begin{verbatim}
    CALL TMPRO (TEXT)
 
          TEXT   is the character string to be written
\end{verbatim}
{\bf Read next line from standard input}
\begin{verbatim}
    CALL TMREAD (MAXCH, CHLINE, NCH, ISTAT)
 
         MAXCH   maximum number of char. to be stored into LINE
          LINE*  text read, of type CHARACTER
           NCH*  number of characters read into LINE
         ISTAT*  status returned:
                 = 0  success
                 < 0  end-of-file seen
                 > 0  read error
\end{verbatim}
$\bullet$
