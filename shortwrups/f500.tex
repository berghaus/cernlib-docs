%  18 oct 94  ksk
\Version {RLHOIN}                     \Routid{F500}
\Keywords{LINEAR HOMOGENEOUS INEQUALITY INEQUALITIES}
\Author{K.S. K\"olbig, F. Schwarz}    \Library{MATHLIB}
\Submitter{}                          \Submitted{01.07.1979}
\Language{Fortran}                     \Revised{01.12.1994}
\Cernhead {Linear Homogeneous Inequalities}
Subroutine subprograms {\tt RLHOIN} and {\tt DLHOIN} find the basis
$\mathbf{v}_j,(j=1,2,\ldots,J)$, of the convex polyhedral cone
defining the solution of a system of homogeneous linear inequalities
$\mathbf{Ax} \ge 0$. ${\bf A}=a_{mn}$ is a given $M \times N$ matrix,
$ M\ge N$, and rank$(\mathbf{A})=N$.
$\mathbf{x}=(x_1,x_2,\ldots,x_n)$ is a column vector. Any solution
{\bf x} of $\mathbf{Ax} \ge 0$ can be expressed as
$$\mathbf{x}=\sum_{j=1}^J \lambda_j \mathbf{v}_j.$$
where all $\lambda_j\ge 0$. The number $J$ of vectors $\mathbf{v}_j$
depends on the matrix {\bf A} in an unknown way, except when $M=N$,
where $J=N$.
\par
On CDC and Cray computers, the double-precision version {\tt DLHOIN}
is not available.
\Structure
{\tt SUBROUTINE} subprogram \\
User Entry Names: \Rdef{RLHOIN}, \Rdef{DLHOIN}\\
Obsolete User Entry Names: \Rdef{LIHOIN} $\equiv$ {\tt RLHOIN} \\
Files Referenced: {\tt Unit 6}\\
External References:
  \begin{tabular}[t]{@{}llll}
     \Rind{RVCPY}{F002},&\Rind{RVMPY}{F002},&\Rind{RVSCL}{F002}, \\
     \Rind{DVCPY}{F002},&\Rind{DVMPY}{F002},&\Rind{DVSCL}{F002}, \\
     \Rind{RMCPY}{F003},&\Rind{RMSET}{F003},&\Rind{DMCPY}{F003}, &
     \Rind{DMSET}{F003},                                         \\
     \Rind{RINV}{F010}, &\Rind{DINV}{F010},  &
     \Rind{MTLMTR}{N002},&\Rind{ABEND}{Z035}
  \end{tabular}
\Usage
For $\mathtt{t=R}$ (type {\tt REAL}), $\mathtt{t=D}$ (type
{\tt DOUBLE PRECISION}),
\begin{verbatim}
    CALL tLHOIN(A,MA,M,N,MAXV,V,NV,JVEC,EPS,IOUT,W,IW)
\end{verbatim}
\begin{DLtt}{123456}
\item [A] (type according to {\tt t}) Two-dimensional array, dimensioned
$\mathtt{ (MA,\ge N)}$, whose rows contain the  coefficients of the
inequalities, arranged in such a way that the upper left
$\mathtt{N \times N}$ corner has a non-vanishing determinant.
Usually it is advisable to normalise the rows of {\tt A} to unity
before calling this subprogram.
\item [MA] ({\tt INTEGER}) First dimension parameter of {\tt A}.
\item [M] ({\tt INTEGER}) Number $M$ of inequalities.
\item [N] ({\tt INTEGER}) Number $N$ of variables.
\item [MAXV] ({\tt INTEGER}) Maximum number of basis vectors
which may occur at any intermediate step, to be chosen
sufficiently large and in any case $\mathtt{\ge N}$.
\item [V] (type according to {\tt t}) Two-dimensional array, dimensioned
$\mathtt{(NV,\ge MAXV)}$,  whose columns contain, on
return, the basis vectors $\mathbf{v}_j$ of the solution cone.
\item[NV] ({\tt INTEGER}) First dimension parameter of
$\mathtt{V (\ge N)}$.
\item [JVEC] ({\tt INTEGER}) Number $J$ of basis vectors of the final
cone.
\item [EPS] (type according to {\tt t}) A small parameter which
discriminates small quantities against zero, chosen to take into
account the accuracy of the machine used.
\item [IOUT] ({\tt INTEGER}) \\
$\mathtt{= 0:}$ Gives no intermediate printout, \\
$\mathtt{= 1:}$ Gives, for each iteration, the basis vectors of the
respective cone, the matrix of scalar products and the index of the
inequality taken into account in the next step.
\item [W] (type according to {\tt t}) Two-dimensional array, dimensioned
$(\mathtt{MAXV,\ge M+1})$, used as working space.
\item [IW]({\tt INTEGER)} Two-dimensional array, dimensioned
$\mathtt{(MA,5)}$ whose columns serve as book-keepers
for certain properties of the system during the iteration procedure.
\end{DLtt}
\Method
The Motzkin-Burger procedure is used to obtain the solution iteratively.
Ref. 1 should be consulted before using this subprogram.
\Restrict
The routine may fail if the matrix {\bf A} is "ill-conditioned" in a
certain sense.
\Notes
A given system of linear homogenous inequalities may have no solution.
\Errorh
Error {\tt F500.1}: $\mathtt{MAXV}$ too small. \\
Error {\tt F500.2}: Upper left $\mathtt{N \times N}$ corner of {\tt A}
is singular. \\
Error {\tt F500.3}: Inequality {\tt k} is inconsistent. \\
In all cases, a message is written on
{\tt Unit 6}, unless subroutine {\tt MTLSET} (N002) has been called.
\Refer
\begin{enumerate}
\item K.S. K\"olbig and F. Schwarz, A program for solving systems of
homogeneous linear inequalities. Computer Phys. Comm. {\bf 17} (1979)
375--382.
\end{enumerate}
$\bullet$
