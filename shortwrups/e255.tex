%  14.10.94  ksk
\Version {PARLSQ}                          \Routid{E255}
\Keywords{FIT FITTING LEAST-SQUARE PARABOLA}
\Author{H. Grote}                           \Library{MATHLIB}
\Submitter{M. Metcalf}                      \Submitted{01.05.77}
\Language{Fortran}                          %\Revised{}
\Cernhead {Least-Squares Fit to Parabola}
\begin{center}
\fbox{\parbox{120mm}{\OBSOLETE
Please note that this routine has been obsoleted in CNL 218. Users are
advised not to use it any longer and to replace it in older programs.
No maintenance for it will take place and it will eventually disappear.
\\[3mm]
Suggested replacement: {\tt RLSQP2} (E201)}}
\end{center}
Given a vector of values $Y$ measured at the points $X$, {\tt PARLSQ}
finds the best least-squares fit to the parabola $Y=c_1+c_2 x+c_3 x^2$.
\Structure
{\tt SUBROUTINE} subprogram \\
User Entry Names: \Rdef{PARLSQ}
\Usage
\begin{verbatim}
    CALL PARLSQ(X,Y,L,C,VAR)
\end{verbatim}
\begin{DLtt}{123456}
\item[X] ({\tt REAL}) Vector of abscissae.
\item[Y] ({\tt REAL}) Vector of values corresponding to points {\tt X}.
\item[L] ({\tt INTEGER)} Length of vectors {\tt X} and {\tt Y}.
\item[C] ({\tt REAL}) Array of dimension {\tt 3} in the calling program.
On exit, it contains the coefficients $c_1,c_2,c_3$.
\item[VAR] ({\tt REAL)} Residual sum of squares divided by
$\mathtt{L-3}$.
\end{DLtt}
\Notes
If $\mathtt{L < 3}$, {\tt C} and {\tt VAR} are set to zero.
\Refer
\begin{enumerate}
\item D.H. Menzel, Fundamental Formulas of Physics, Dover Publ., New York
(1960) 122
\end{enumerate}
$\bullet$
