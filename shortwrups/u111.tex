%  30.09.94  ksk
\Version {RWIG3J}                    \Routid{U111}
\Keywords{CLEBSCH-GORDAN WIGNER J-SYMBOLS RACAH COEFFICIENTS JAHN}
\Author{K.S. K\"olbig}                  \Library{MATHLIB}
\Submitter{}                            \Submitted{15.10.1994}
\Language{Fortran}                    %  \Revised{}
\Cernhead{Wigner 3-j, 6-j, 9-j Symbols; Clebsch-Gordan,
Racah W-, Jahn U-Coefficients}
Function subprograms {\tt RWIG3J}, {\tt DWIG3J};
{\tt RWIG6J}, {\tt DWIG6J}; {\tt RWIG9J}, {\tt DWIG9J};
{\tt RCLEBG}, {\tt DCLEBG}; {\tt RRACAW}, {\tt DRACAW}
and {\tt RJAHNU}, {\tt DJAHNU} calculate the
Wigner 3-$j$, 6-$j$ and 9-$j$ symbols, the Clebsch-Gordan coefficients,
the Racah $W$-coefficients and the Jahn $U$-coefficients, respectively.
\par
On CDC and Cray computers, the double-precision versions {\tt DWIG3J}
etc. are not available.
\Structure
{\tt FUNCTION} subprograms \\
User Entry Names:
\begin{htmlonly}
\begin{tabular}{llllll}
\end{htmlonly}
\begin{latexonly}
\begin{tabular}[t]{l*{5}{@{\hspace{4pt}}l}}
\end{latexonly}
\Rdef{RWIG3J}, & \Rdef{RWIG6J}, & \Rdef{RWIG9J}, & \Rdef{RCLEBG}, &
\Rdef{RRACAW}, & \Rdef{RJAHNU} \\
\Rdef{DWIG3J}, & \Rdef{DWIG6J}, & \Rdef{DWIG9J}, & \Rdef{DCLEBG}, &
\Rdef{DRACAW}, & \Rdef{DJAHNU}
\end{tabular}
\Usage
In any arithmetic expression,
for $\mathtt{t = R}$ (type {\tt REAL}), or
$\mathtt{t = D}$ (type {\tt DOUBLE PRECISION}), \\[2mm]
\begin{tabular}
{l@{\qquad \mbox{has the value of} \qquad}l}
{\tt tWIG3J(A,B,C,X,Y,Z)} &
$\displaystyle \left(\begin{array}{ccc}
a & b & c \\ x & y & z \end{array}\right)$; \\[5mm]
{\tt tWIG6J(A,B,C,X,Y,Z)} &
$\displaystyle \left\{\begin{array}{ccc}
a & b & c \\ x & y & z \end{array}\right\}$; \\[4mm]
{\tt tWIG9J(A,B,C,P,Q,R,X,Y,Z)} &
$\displaystyle \left\{\begin{array}{ccc}
a & b & c \\ p & q & r \\
x & y & z \end{array}\right\}$; \\[5mm]
{\tt tCLEBG(A,B,C,X,Y,Z)} & $(a\,b\,x\,y\,|\,a\,b\,c\,z)$; \\[2mm]
{\tt tRACAW(A,B,C,D,E,F)} & $W(a\,b\,c\,d\,;\,e\,f)$; \\[2mm]
{\tt tJAHNU(A,B,C,D,E,F)} & $U(a\,b\,c\,d\,;\,e\,f)$.
\end{tabular} \\[2mm]
All the arguments must have integral or half-integral values (see
{\bf Notes}). They have the same type as the function name.
For definitions and notations see {\bf References}. \\
The following relations hold (see Refs. 1 and 3): \\
Clebsch-Gordan coefficient (in terms of the Wigner 3-$j$ symbol):
\begin{eqnarray*}
(a\,b\,x\,y\,|\,a\,b\,c\,z) & = &
\displaystyle (-1)^{a-b-z}\,\sqrt{2c+1}
\left(\begin{array}{ccc} a & b & c \\ x & y & -z \end{array}\right);
\end{eqnarray*}
Racah $W$-coefficient (in terms of the Wigner 6-$j$ symbol):
\begin{eqnarray*}
W(a\,b\,c\,d\,;\,e\,f) & = &
\displaystyle (-1)^{a+b+c+d}
\left\{\begin{array}{ccc} a & b & e \\ d & c & f \end{array}\right\}. \\
\end{eqnarray*}
Jahn $U$-coefficient (in terms of the Wigner 6-$j$ symbol and the
Racah $W$-coefficient):
\begin{eqnarray*}
U(a\,b\,c\,d\,;\,e\,f) & = &
\displaystyle (-1)^{a+b+c+d}\,\sqrt{(2e+1)(2f+1)}\,
\left\{\begin{array}{ccc} a & b & e \\ d & c & f \end{array}\right\} \\
& = & \displaystyle \sqrt{(2e+1)(2f+1)}\,W(a\,b\,c\,d\,;\,e\,f).
\end{eqnarray*}
\Method
The Wigner 3-$j$ symbol and the Clebsch-Gordan coefficient are
calculated from formulas (5.1) and (5.10) of Ref. 1, respectively.
The Wigner 6-$j$ symbol, the Racah $W$- and the Jahn $U$-coefficient
are calculated from formulas (5.23) and (5.24) of Ref. 1.
In both cases, the factorials are replaced by their logarithms during
the calculation. The Wigner 9-$j$ symbol is calculated from formula
(5.37) of Ref. 1 in terms of Wigner 6-$j$ symbols.
\Notes
A Wigner-3$j$ symbol $\displaystyle \left(\begin{array}{ccc}
j_1 & j_2 & j_3 \\ m_1 & m_2 & m_3 \end{array}\right)$
is considered to be zero unless simultaneously \\
\begin{tabular}[t]{ll}
(i) & $j_i$ and $m_i$ have both either integral or half-integral
values (each $i$), \\
(ii) & $j_i \ge |m_i| \ge 0$ \ (each $i$), \\
(iii) & $m_1+m_2+m_3=0$, \\
(iv) & $j_1-j_2-m_3$ \ is an integer, \\
(v) & $j_1+j_2+j_3$ \ is an integer and \
$j_1+j_2 \ge j_3, \quad j_2+j_3 \ge j_1, \quad j_3+j_1 \ge j_2$.
\end{tabular}
\par
The conditions (v) are often denoted by $\delta(j_1\,j_2\,j_3)$ and
are called the {\it triangle relations}.
\par
For a Clebsch-Gordan coefficient
$(j_1\,j_2\,m_1\,m_2\,|\,j_1\,j_2\,j_3\,m_3)$,
condition (iii) reads $m_1+m_2=m_3$ and condition (iv) disappears.
\par
A Wigner-6$j$ symbol $\displaystyle \left\{\begin{array}{ccc}
j_1 & j_2 & j_3 \\ l_1 & l_2 & l_3 \end{array}\right\}$
is considered to be zero unless simultaneously \\
\begin{tabular}[t]{ll}
(i) & all $j_i$ and $l_i$ have non-negative integral or
half-integral values, \\
(ii) & the four {\it triangle relations} \quad
$\delta(j_1\,j_2\,j_3), \quad \delta(j_1\,l_2\,l_3), \quad
 \delta(l_1\,j_2\,l_3), \quad \delta(l_1\,l_2\,j_3)$ \quad hold.
\end{tabular}
\par
A Wigner-9$j$ symbol $\displaystyle \left\{\begin{array}{ccc}
j_{11} & j_{12} & j_{13} \\ j_{21} & j_{22} & j_{23} \\
j_{31} & j_{32} & j_{33} \\
\end{array}\right\}$
is considered to be zero unless simultaneously \\
\begin{tabular}[t]{ll}
(i) & all $j_{ik}$ have non-negative integral or half-integral
values,\\
(ii) & the arguments in each row and in each column satisfy the
{\it triangle relations}.
\end{tabular}
\Restrict
The sum of arguments in any {\it triangle relation} must not exceed 100.
No test is made.
\Refer
\begin{enumerate}
\item R.D. Cowan, The theory of atomic structure and spectra,
(Univ. of California Press, Berkeley CA 1981).
\item A.F. Nikiforov, V.B. Uvarov and Yu.L. Levitan, Tables of Racah
coefficients (Pergamon Press, Oxford 1965).
\item M. Rotenberg, R. Bivins, N. Metropolis and J.K. Wooten, Jr.,
The 3-$j$ and 6-$j$ symbols (Crosby Lockwood, London 1959).
\item D.A. Varshalovich, A.N. Moskalev and V.K. Khersonskii,
Quantum theory of angular momentum (World Scientific, Singapore 1988).
\end{enumerate}
$\bullet$
