%  07 nov 1995 ksk
\Version{LOCATR}                             \Routid{E106}
\Keywords{ARRAY BINARY ELEMENT ORDER SEARCH}
\Author{F. James, K.S. K\"olbig}               \Library{KERNLIB}
\Submitter{}                                 \Submitted{18.10.1974}
\Language{Fortran}                       \Revised{15.11.1995}
\Cernhead{Binary Search for Element in Ordered Array}
Integer function subprograms {\tt LOCATI} and {\tt LOCATR} perform
a binary search in an array of non-decreasing integer or real numbers
$a_1\leq a_2\leq \ldots \leq a_n$ to locate a specified value $t$.
\Structure
{\tt FUNCTION} subprograms \\
User Entry Names: \Rdef{LOCATI}, \Rdef{LOCATR} \\
Obsolete User Entry Names: \Rdef{LOCATF} $\equiv$ \Rdef{LOCATR}
\par
On CDC or Cray computers, the double-precision version {\tt LOCATD}
is not available.
\Usage
In any arithmetic expression, for $\mathtt{t=I}$ (type {\tt INTEGER}),
$\mathtt{t=R}$ (type {\tt REAL}), $\mathtt{t=D}$
(type {\tt DOUBLE PRECISION}),
\begin{center}
{\tt LOCATt(tA,N,tT)}
\end{center}
has the {\tt INTEGER} value according to the description below.
\begin{DLtt}{123456}
\item [tA] (type according to {\tt t}) One-dimensional array.
The numbers {\tt tA(j)} must form a non-decreasing
sequence for $j=1,2,\ldots,\mathtt{N}$.
\item [N] ({\tt INTEGER}) Number $n$ of elements in array {\tt tA}.
\item[tT] (type according to {\tt t}) Search value $t$.
\end{DLtt}
Depending on four possible outcomes of the search, each subprogram
returns the following value {\tt L} $(a = \mathtt{tA}$,
$t = \mathtt{tT}$):
\begin{center}
\begin{tabular}{ll}
$a_j = t$ for some $j$ with $1 \leq j \leq \mathtt{N}$ &
$\mathtt{L}=j$ \\
$t < a_1$ & $\mathtt{L = 0}$ \\
$a_k < t < a_{k+1}$ for some $k$ with $1 \leq k \leq \mathtt{N-1}$ &
$\mathtt{L}=-k$ \\
$a_n < t$ & $\mathtt{L=-N}$
\end{tabular}
\end{center}
If the value $t$ occurs more than once in the array $a$,
the result {\tt L} may correspond to any of the occurrences.
\Method
Repeated bisection of the subscript range.
\Notes
\begin{enumerate}
\item The number of comparisons performed is approximately proportional
to $\ln {\tt N}$. Therefore, for large {\tt N} the binary search is
considerably faster than a sequential search using a {\tt DO} loop.
However, for {\tt N} less than about 40 a {\tt DO} loop is faster.
\item
The obsolete older entry {\tt LOCATF} is kept for a transitional
period. It will eventually disappear.
\end{enumerate}
$\bullet$
