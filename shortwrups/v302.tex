% 27 May 1992 mg
\Version {UCOCOP}                            \Routid{V302}
\Keywords{ARRAY COPY SCATTER VECTOR}
\Author{F. Bruyant}                          \Library{KERNLIB}
\Submitter{C. Letertre}                      \Submitted{21.08.1971}
\Language{Fortran or Assembler}        \Revised{16.09.1991}
\Cernhead {Copy a Scattered Vector}
{\tt UCOCOP} and {\tt UDICOP} copy the contents of a scattered vector
into a new scattered vector.
\Structure
{\tt SUBROUTINE} subprograms \\
User Entry Names: \Rdef{UCOCOP}, \Rdef{UDICOP}
\Usage
\begin{verbatim}
    CALL UCOCOP(A,X,IDO,IW,NA,NX)
    CALL UDICOP(A,X,IDO,IW,NA,NX)
\end{verbatim}
extract $\mathtt{IDO}$ times $\mathtt{IW}$
consecutive words from {\tt A},
every {\tt NA} words, and place them into {\tt X}, every {\tt NX} words.
Both routines have the same effect if the vectors {\tt A} and {\tt X}
do not overlap. {\tt UCOCOP} allows concentration,
{\tt UDICOP} allows dilation of a vector {\it in situ}.
\par
For $\mathtt{IDO=0}$ or $\mathtt{IW=0}$, the routines act as
'do-nothing'.
\Examples
\begin{verbatim}
    DIMENSION IA(14),IX(12)
    DATA IA /1,2,3,4, 5,6,7,8, 9,10,11,12, 13,14/
 
    CALL UCOCOP(IA,IX,4,2,4,3)
    CALL UCOCOP(0,IX(3),4,1,0,3)
\end{verbatim}
gives
\begin{verbatim}
    IX = 1,2,0, 5,6,0, 9,10,0, 13,14,0
\end{verbatim}
$\bullet$
