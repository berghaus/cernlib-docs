\Version{CDIGAM}                      \Routid{C307}
\Keywords{COMPLEX DIGAMMA FUNCTION PSI}
\Author{K.S. K\"olbig}                 \Library{MATHLIB}
\Submitter{  }                        \Submitted{02.05.1966}
\Language{Fortran}                 \Revised{15.03.1993}
\Cernhead{Digamma or Psi Function for Complex Argument}
Function subprograms {\tt CDIGAM} and {\tt WDIGAM} calculate
the logarithmic derivative of the gamma function
(the digamma, or psi, function)
$\psi(z)$ defined by
$$ \psi (z) \ = \  \frac{d \ln \Gamma(z)}{dz} $$
for complex arguments $z \neq -n,(n = 0,1,2,\cdots)$.
\par
The double-precision version {\tt WDIGAM} is available only on computers
which support a {\tt COMPLEX*16} Fortran data type.
\Structure
{\tt FUNCTION} subprograms \\
User Entry Names: \Rdef{CDIGAM}, \Rdef{WDIGAM}  \\
Files Referenced: {\tt Unit 6} \\
External References: \Rind{MTLMTR} (N002), \Rind{ABEND} (Z035)
\Usage
In any arithmetic expression,
\begin{center}
{\tt CDIGAM(Z)} \quad or \quad {\tt WDIGAM(Z)} \quad has the value \quad
$\psi(\mathtt{Z})$,
\end{center}
where {\tt CDIGAM} is of type {\tt COMPLEX}, {\tt WDIGAM} is of type
{\tt COMPLEX*16}, and {\tt Z} has the same type as the function name.
\Method
The method is described in Ref. 1.
\Accuracy
{\tt CDIGAM} (except on CDC and Cray computers)
has full single-precision accuracy.
For most values of the argument {\tt Z}, {\tt WDIGAM}
(and {\tt CDIGAM} on CDC and Cray computers) has an accuracy of
approximately one significant digit less than the machine precision.
\Errorh
Error {\tt C307.1}: $\mathtt{Z} = -n,(n = 0,1,2,\cdots).$
The function value is set equal to zero, and a message is written on
{\tt Unit 6}, unless subroutine {\tt MTLSET} (N002) has been called.
\Refer
\begin{enumerate}
\item K.S. K\"olbig, Programs for computing the
logarithm of the gamma function, and the digamma function, for
complex argument, Computer Phys. Comm. {\bf 4} (1972) 221--226.
\end{enumerate}
$\bullet$
