%   20 feb 1995   ksk
\Version {HBOOK}                           \Routid{Y250}
\Keywords{ANALYSIS STATISTIC HISTOGRAM}
\Author{R. Brun, I. Ivanchenko, P. Palazzi}  \Library{PACKLIB}
\Submitter{}                              \Submitted{ }
\Language{Fortran}                      %\Revised{}
\Cernhead {Statistical Analysis and Histogramming}
{\tt HBOOK} offers as basic options the booking, filling and printing
of a histogram, scatter plot or table. Other available facilities
are:
\begin{DLtt}{12}
\item[$\bullet$]  Projections and slices of scatter plots and tables.
\item[$\bullet$]  Wide choice of editing options (what to print and how).
\item[$\bullet$]  Easy access to the information.
\item[$\bullet$]  Operations on histograms (arithmetic, smoothing,
filling, fitting).
\item[$\bullet$]  Packing of several channels in 1 computer word/or
extension of the memory on disk file, to allow simultaneous handling of
a very large number of plots.
\end{DLtt}
\Structure
{\tt SUBROUTINE} and {\tt FUNCTION} subprograms
\Usage
See {\bf Long Write-up}.
\\ $\bullet$
