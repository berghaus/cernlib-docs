%  07.10.1994
\Version{RTRGSM}                              \Routid{E409}
\Keywords{SERIES SUM TRIGONOMETRIC}
\Author{T. H{\aa}vie, K.S. K\"olbig}             \Library{MATHLIB}
\Submitter{}                                 \Submitted{01.12.1994}
\Language{Fortran}                       %\Revised{}
\Cernhead {Summation of Trigonometric Series}
Function subprograms {\tt RTRGSM} and {\tt DTRGSM}
compute the sum of the trigonometric series
$$f(x) \ = \ a_0+\sum^n_{k=1} a_k\cos kx+\sum^m_{k=1} b_k\sin kx$$
for a given argument $x$ in the range $-\pi \le x \le \pi$ and given
coefficients $a_k,b_k$.
\par
On CDC and Cray computers, the double-precision version {\tt DTRGSM}
is not available.
\Structure
{\tt FUNCTION} subprogram \\
User Entry Names: \Rdef{RTRGSM}, \Rdef{DTRGSM}
\Usage
In any arithmetic expression, for $\mathtt{t=R}$ (type {\tt REAL}),
$\mathtt{t=D}$ (type {\tt DOUBLE PRECISION}),
\begin{verbatim}
    tTRGSM(X,A,N,B,M,IOP)
\end{verbatim}
has the value $f(x)$.
\begin{DLtt}{12345}
\item[X] (Type according to {\tt t}) Argument $x$.
\item[A] (Type according to {\tt t}) One-dimensional array of
dimension {\tt (0:d)} where $\mathtt{d \ge N}$, containing the constant
coefficient $a_0$ in {\tt A(0)} and the cosine coefficients
$a_k\,(k=1,\ldots,n)$ in {\tt A(k)}.
\item[N] ({\tt INTEGER}) The number $n$ of cosine coefficients.
\item[B] (Type according to {\tt t}) One-dimensional array of length
$\mathtt{\ge M}$, containing the sine coefficients
$b_k\,(k=1,\ldots,n)$ in {\tt B(k)}.
\item[M] ({\tt INTEGER}) The number $m$ of sine coefficients.
\item[IOP] ({\tt INTEGER}) An option number: \\
$\mathtt{= 1:}$ the general case, \\
$\mathtt{= 2:}$ all $b_k$ are zero, i.e. $f(x)=f(-x)$,\\
$\mathtt{= 3:}$ all $a_k$ are zero, i.e. $f(x)=-f(-x)$.
\end{DLtt}
\Method
Standard recurrence relations are used for calculating the sum
(see Ref. 1).
\Notes
For a function $f(z)$ given in the range $a \le z \le b$,
use the transformation
\begin{eqnarray*}
x & = & \displaystyle \frac{2\pi}{b-a}\left(z-\frac{b+a}{2} \right)
\quad \mathrm{for} \ \mathtt{IOP=1}, \\[4mm]
x & = & \displaystyle \pi\,\frac{z-a}{b-a}
\quad \mathrm{for} \ \mathtt{IOP=2} \ \mathrm{or} \ \mathtt{IOP=3}.
\end{eqnarray*}
\Refer
\begin{enumerate}
\item W. Clenshaw, A note on the summation of Chebyshev series,
MTAC (later renamed Math. Comp.) {\bf 9} (1955) 118--120.
\end{enumerate}
$\bullet$
