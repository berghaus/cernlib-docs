% 8 jun 1995 ksk
\Version{RELI1C}                       \Routid{C347}
\Keywords{COMPLETE ELLIPTIC INTEGRAL FIRST SECOND THIRD}
\Keywords{JACOBIAN THETA FUNCTION HEUMAN LAMBDA}
\Author{K.S. K\"olbig}                  \Library{MATHLIB}
\Submitter{ }                           \Submitted{07.06.1992}
\Language{Fortran}                      \Revised{}
\Cernhead{Complete Elliptic Integrals of First, Second, and Third
Kind}
Function subprograms {\tt RELI1C}, {\tt RELI2C}, {\tt RELI3C}
and {\tt DELI1C}, {\tt DELI2C}, {\tt DELI3C} calculate
the complete elliptic integrals of the first, second and
third kind, respectively.
\par
Function subprograms {\tt RELIGC} and {\tt DELIGC} calculate a general
complete elliptic integral.
\par
Function subprograms {\tt RELIKC}, {\tt RELIEC} and {\tt DELIKC},
{\tt DELIEC} calculate the complete elliptic integrals K$(k)$ and
E$(k)$.
\par
On CDC and Cray computers, the double-precision versions {\tt DELI1C}
etc. are not available.
\par
Mainly for reasons of numerical stability, the algorithms are based on
the following definitions:
\par
{\bf First kind:}
\begin{eqnarray*}
\mathbf{F}_1^*(k') \ = \  \displaystyle
\int_0^\infty \frac{d\xi}{\sqrt{(1+\xi^2)(1+{k'}^2\xi^2)}}
\qquad ({k'}^2 > 0).
\end{eqnarray*}
{\bf Second kind:}
\begin{eqnarray*}
\mathbf{F}_2^*(k',a,b) \ = \ \displaystyle
\int_0^\infty \frac{a+b\xi^2}{(1+\xi^2)\sqrt{(1+\xi^2)(1+{k'}^2\xi^2)}}
\, d\xi \qquad ({k'}^2 > 0).
\end{eqnarray*}
{\bf Third kind:}
\begin{eqnarray*}
\mathbf{F}_3^*(k',p) \ = \ \displaystyle
\int_0^\infty \frac{1+\xi^2}{(1+p\xi^2)\sqrt{(1+\xi^2)(1+{k'}^2\xi^2)}}
\, d\xi \qquad ({k'}^2 > 0, \ p \ne 0).
\end{eqnarray*}
\par
Note that $\mathbf{F}_1^*(k') = \mathbf{F}_2^*(k',1,1) =
\mathbf{F}_3^*(k',1)$. For $p < 0$, the integral $\mathbf{F}_3^*$ is
defined by its principal value.
\par
{\bf The general integral:}
\begin{eqnarray*}
\mathbf{G}(k',p,a,b) & = & \displaystyle
\int_0^\infty \frac{a+b\xi^2}{(1+p\xi^2)\sqrt{(1+\xi^2)(1+{k'}^2\xi^2)}}
\, d\xi  \\[3mm]
& = & \displaystyle \int_0^{\pi/2} \frac{a \cos^2 \phi + b \sin^2 \phi}
{\cos^2 \phi + p \sin^2 \phi} \frac{d\phi}
{\sqrt{\cos^2 \phi + {k'}^2 \sin^2 \phi}} \qquad ({k'}^2 > 0).
\end{eqnarray*}
For $p < 0$, this integral is defined by its principal value.
See {\bf Notes} for special cases.
\par
{\bf The functions K(k) and E(k):}
\begin{eqnarray*}
\mathrm{K}(k) & = & \displaystyle \int_0^{\pi/2}
\frac{d\psi}{\sqrt{1-k^2\sin^2 \psi}} \qquad (|k| < 1), \\
\mathrm{E}(k) & = & \displaystyle \int_0^{\pi/2}
\sqrt{1-k^2\sin^2 \psi} \, d\psi \qquad (|k| \le 1).
\end{eqnarray*}
\newpage
Other common definitions of the complete elliptic integrals and their
relations to $\mathbf{F}_1^*$, $\mathbf{F}_2^*$, $\mathbf{F}_3^*$ are
listed here for convenience ($k^2+{k'}^2 = 1$): \\[2mm]
{\bf First kind:}
$$ \begin{array}{rcl}
F(k,\pi/2) & = & \mathrm{K}(k) \quad = \quad \mathbf{F}_1^*(k') \qquad
(|k| < 1), \\[6mm]
\widehat{F}(1,k) & = & \displaystyle \int_0^1
\frac{d\eta}{\sqrt{(1-\eta^2)(1-k^2\eta^2)}} \quad = \quad
\mathbf{F}_1^*(k') \qquad (|k| < 1).
\end{array} $$
{\bf Second kind:}
$$ \begin{array}{rcl}
E(k,\pi/2) & = & \mathrm{E}(k) \quad = \quad \mathbf{F}_2^*(k',1,{k'}^2)
\qquad (|k| \le 1), \\[6mm]
\widehat{E}(1,k) & = & \displaystyle \int_0^1
\sqrt{\frac{1-k^2 \eta^2}{1-\eta^2}} \, d\eta \quad = \quad
\mathbf{F}_2^*(k',1,{k'}^2) \qquad (|k| \le 1).
\end{array} $$
{\bf Third kind:}
$$ \begin{array}{rclcl}
\Pi(\pi/2,h,k) & = & \displaystyle \int_0^{\pi/2}
\frac{d\psi}{(1+h\sin^2 \psi)\sqrt{1-k^2\sin^2 \psi}} & = &
\mathbf{F}_3^*(k',h+1) \qquad (|k| < 1), \\[6mm]
\widehat{\Pi}(1,h,k) & = & \displaystyle \int_0^1
\frac{d\eta}{(1+h\eta^2)\sqrt{(1-\eta^2)(1-k^2\eta^2)}} & = &
\mathbf{F}_3^*(k',h+1) \qquad (|k| < 1).
\end{array} $$
\Structure
{\tt FUNCTION} subprograms \\
User Entry Names:
\begin{htmlonly}
\begin{tabular}{llllll}
\end{htmlonly}
\begin{latexonly}
\begin{tabular}[t]{l*{5}{@{\hspace{4pt}}l}}
\end{latexonly}
\Rdef{RELI1C}, & \Rdef{RELI2C}, & \Rdef{RELI3C}, & \Rdef{RELIGC}, &
\Rdef{RELIKC}, & \Rdef{RELIEC} \\
\Rdef{DELI1C}, & \Rdef{DELI2C}, & \Rdef{DELI3C}, & \Rdef{DELIGC}, &
\Rdef{DELIKC}, & \Rdef{DELIEC}
\end{tabular} \\
Obsolete User Entry Names:
\begin{tabular}[t]{l{@{\hspace{4pt}}l}}
\Rdef{ELLICK} $\equiv$ {\tt RELIKC}, &
\Rdef{ELLICE} $\equiv$ {\tt RELIEC}, \\
\Rdef{DELLIK} $\equiv$ {\tt DELIKC}, &
\Rdef{DELLIE} $\equiv$ {\tt DELIEC}
\end{tabular} \\
Files Referenced: {\tt Unit 6} \\
External References: \Rind{MTLMTR}{N002}, \Rind{ABEND}{Z035}
\Usage
In any arithmetic expression, with $\mathtt{AK}=k$ and $\mathtt{AKP}=k'$,
\begin{center}
\begin{tabular}{l@{\quad or \quad}l@{\quad has the value \quad}l}
{\tt RELI1C(AKP)} & {\tt DELI1C(AKP)} & $\mathbf{F}_1^*(k')$,\\
{\tt RELI2C(AKP,A,B)} & {\tt DELI2C(AKP,A,B)} &
$\mathbf{F}_2^*(k',\mathtt{A,B})$,\\
{\tt RELI3C(AKP,AK2,P)} & {\tt DELI3C(AKP,AK2,P)} &
$\mathbf{F}_3^*(k',\mathtt{P})$, \\
{\tt RELIGC(AKP,P,A,B)} & {\tt DELIGC(AKP,P,A,B)} &
$\mathbf{G}(k',\mathtt{P,A,B})$, \\
{\tt RELIKC(AK)} & {\tt DELIKC(AK)} & K$(k)$, \\
{\tt RELIEC(AK)} & {\tt DELIEC(AK)} & E$(k)$,
\end{tabular}
\end{center}
where {\tt RELI1C} etc are of type {\tt REAL}, {\tt DELI1C} etc are of
type {\tt DOUBLE PRECISION}, and {\tt AKP}, {\tt AK}, {\tt AK2}, {\tt A},
{\tt B} and {\tt P} have the same type as the function name.
\par
The redundant parameter {\tt AK2} in {\tt RELI3C} and {\tt DELI3C}
permits improved accuracy when $k^2$ is small, i.e. $k' \approx 1$. In
this case, $\mathtt{AK2} = k^2$ should be calculated using
higher-precision arithmetic and then truncated before calling the
subprogram.
\newpage
\Method
The evaluation of $\mathbf{F}_1^*$, $\mathbf{F}_2^*$, $\mathbf{F}_3^*$
is based on the Landen transformation, that of $\mathbf{G}$ on the Bartky
transformation. For details, see Ref. 1--3. For K$(k)$ and E$(k)$
Chebyshev approximations are used (see Ref. 4).
\Accuracy
The {\tt REAL} functions (except on CDC and Cray computers) have full
single-precision accuracy. The {\tt REAL} functions on CDC and Cray, and
the {\tt DOUBLE PRECISION} functions on all computers have an accuracy
approximately two significant digits less than the machine precision.
\Restrict
1. \quad {\tt RELI1C} and {\tt DELI1C:} \ $\mathtt{AKP \ne 0}$. \\
2. \quad {\tt RELI2C} and {\tt DELI2C:} \ $\mathtt{AKP \ne 0}$
or $\mathtt{AKP = 0}$ and $\mathtt{B = 0}$. \\
3. \quad {\tt RELI3C} and {\tt DELI3C:} \ {\tt AKP*P $\ne$ 0}. \\
4. \quad {\tt RELIGC} and {\tt DELIGC:} \ $\mathtt{AKP \ne 0}$. \\
5. \quad {\tt RELIKC} and {\tt DELIKC:} \ $\mathtt{|AK| \le 1}$, \qquad
{\tt RELIEC} and {\tt DELIEC:} \ $\mathtt{|AK| < 1}$. \\
\Errorh
Error {\tt C347.1:} Restriction 1 is not satisfied. \\
Error {\tt C347.2:} Restriction 2 is not satisfied. \\
Error {\tt C347.3:} Restriction 3 is not satisfied. \\
Error {\tt C347.4:} Restriction 4 is not satisfied. \\
Error {\tt C347.5:} Restriction 5 is not satisfied. \\
In all cases, the function value
is set equal to zero, and a message is written on {\tt Unit 6},
unless subroutine {\tt MTLSET} (N002) has been called.
\Notes
Every linear combination of the three special complete elliptic
integrals K$(k)$, E$(k)$, $\Pi(h,k)$ may be expressed in terms of
$\mathbf{G}(k',p,a,b)$:
$$\begin{array}{rcl}
\lambda \mathrm{K}(k) + \mu \mathrm{E}(k) & = &
\mathbf{G}(k',1,\lambda+\mu,\lambda+\mu {k'}^2) \\
\lambda \mathrm{K}(k) + \mu \Pi(h,k) & = &
\mathbf{G}(k',h+1,\lambda+\mu,\lambda (h+1)+\mu)
\end{array} $$
Special examples are
$$\begin{array}{rcl}
\mathrm{K}(k)                           & = & \mathbf{G}(k',1,1,1), \\
\mathrm{E}(k)                           & = & \mathbf{G}(k',1,1,{k'}^2)\\
(\mathrm{K}(k)-\mathrm{E}(k))/k^2       & = & \mathbf{G}(k',1,0,1), \\
(\mathrm{K}(k)-{k'}^2\mathrm{E}(k))/k^2 & = & \mathbf{G}(k',1,1,0), \\
\Pi(h,k)                                & = & \mathbf{G}(k',h+1,1,1),\\
(\mathrm{K}(k)-\Pi(h,k))/h              & = & \mathbf{G}(k',h+1,0,1),\\
\end{array} $$
If $ab \ge 0$ then $\mathbf{G}$ will evaluate any linear
combination of K$(k)$, E$(k)$, $\Pi(h,k)$ without cancellation
(such as would occur, for example, if (K$(k)-$E$(k))/k^2$ were to be
computed from values of K$(k)$ and E$(k)$ which had been computed
separately.
\par
Other functions which can be represented by $\mathbf{G}$ are the Jacobian
Zeta function $\mathbf{Z}(\Phi,k)$ and the Heuman Lambda function
$\Lambda_0(\Phi,k)$ (see Ref. 5):
$$\begin{array}{rcl@{\qquad}l}
\mathbf{Z}(\Phi,k) & = & \displaystyle k^2 \,
\frac{\sin \Phi \cos \Phi}{\mathrm{K}(k)} \, \mathbf{G}(k',q,0,\sqrt{q})
& (q = \cos^2 \Phi + {k'}^2 \sin^2 \Phi) \\[4mm]
\Lambda_0(\Phi,k) & = & \displaystyle \frac{2}{\pi}
\sqrt{q} \sin \Phi \ \mathbf{G}(k',q,1,{k'}^2) &
(q = 1 + k^2 \tan^2 \Phi).
\end{array} $$
{\it (Quoted from Ref. 3, slightly modified)}.\\
\newpage
\Source
The subprograms for $\mathbf{F}_1^*$, $\mathbf{F}_2^*$ are based on the
Algol60 procedures {\it cel1, cel2} in Ref. 1, those for
$\mathbf{F}_3^*$ on {\it cel3} in Ref. 2, and those for $\mathbf{G}$
on {\it cel} in Ref. 3.
\Refer
\begin{enumerate}
\item R. Bulirsch, Numerical calculation of elliptic integrals and
elliptic functions, Numer. Math. {\bf 7} (1965) 78--90.
\item R. Bulirsch, Numerical calculation of elliptic integrals and
elliptic functions II, Numer. Math. {\bf 7} (1965) 353--354.
\item R. Bulirsch, Numerical calculation of elliptic integrals and
elliptic functions III, Numer. Math. {\bf 13} (1969) 305--315.
\item W.J. Cody, Chebyshev approximations for the complete elliptic
integrals $K$ and $E$, Math. Comp. {\bf 19} (1965) 105--112.
\item P.F. Byrd and M.D. Friedman, Handbook of elliptic integrals
for engineers and scientists, 2nd ed., Springer-Verlag Berlin (1971)
33--37.
\end{enumerate}
$\bullet$
