% 07 nov  1995  ksk
\Version{PRMFCT}                     \Routid{B002}
\Keywords{PRIME FACTOR DECOMPOSITION}
\Author{K.S. K\"olbig}              \Library{MATHLIB}
\Submitter{}                        \Submitted{15.11.1995}
\Language{Fortran}               %  \Revised{}
\Cernhead{Prime Numbers and Prime Factor Decomposition}
Subroutine subprogram {\tt PRMFCT}
\begin{itemize}
\item sets the first $n \le 1229$ prime numbers
$p_1=2,\,p_2=3,\,p_5=5,\,\ldots,\,p_{1229}=9973$ into an array;
\item performs the decomposition of a positive number $N<10007$ into
its prime factors:
$$N \ = \ 2^{\alpha_1} \cdot 3^{\alpha_2} \cdot 5^{\alpha_3}\,\cdots\,
9973^{\alpha_{1229}};$$
\item performs the decomposition of the factorial $N!$ of
a positive number $N<10007$ into its prime factors:
$$N! \ = \ 2^{\alpha_1} \cdot 3^{\alpha_2} \cdot 5^{\alpha_3}\,\cdots\,
9973^{\alpha_{1229}}.$$
\end{itemize}
Note that this allows in particular to handle quotients of factorials
of rather large numbers in an exact way.
\Structure
{\tt SUBROUTINE} subprogram \\
User Entry  Names: \Rdef{PRMFCT} \\
Files Referenced: {\tt Unit 6}
\Usage
\begin{verbatim}
    CALL PRMFCT(MODE,N,NPRIME,NPOWER,M)
\end{verbatim}
$\mathtt{MODE = 0:}$ Sets the first $n$ prime numbers into an array.
\begin{DLtt}{123456}
\item[N] ({\tt INTEGER}) The number $n$ of prime numbers requested.
\item[NPRIME] ({\tt INTEGER}) One-dimensional array of length
$\mathtt{\ge N}$. On exit, {\tt NPRIME(j)}, ($\mathtt{j=1,2,\ldots,N}$)
contains the $j$-th prime numbers $p_j$,
where $p_1=2,\,p_2=3,\,p_3=5,\,\ldots$
\item[NPOWER] ({\tt INTEGER}) One-dimensional array of length
$\mathtt{\ge N}$. On exit, {\tt NPOWER(j)}, ($\mathtt{j=1,2,\ldots,N}$)
contains the value {\tt 1}.
\item[M] ({\tt INTEGER}) Contains, on exit, the number $n$.
\end{DLtt}
$\mathtt{MODE = 1,2:}$ Performs the decomposition of $N$
($\mathtt{MODE=1}$) or $N!$ ($\mathtt{MODE=2}$) into its prime factors.
\begin{DLtt}{123456}
\item[N] ({\tt INTEGER}) The number $N$ itself ($\mathtt{MODE=1}$) or
its factorial ($\mathtt{MODE=2}$) to be decomposed into prime factors.
\item[NPRIME] ({\tt INTEGER}) One-dimensional array of length
$\mathtt{\ge N}$. On exit, {\tt NPRIME(j)}, ($\mathtt{j=1,2,\ldots,M}$)
contains the $j$-th prime numbers $p_j$,
where $p_1=2,\,p_2=3,\,p_3=5,\,\ldots$.
\item[NPOWER] ({\tt INTEGER}) One-dimensional array of length
$\mathtt{\ge N}$. On exit, {\tt NPOWER(j)}, ($\mathtt{j=1,2,\ldots,M}$)
contains the power $\alpha_j$ corresponding to the prime number $p_j$.
\item[M] ({\tt INTEGER}) Contains, on exit, the index $M \le N$ defined
by $\alpha_M>0$ and $\alpha_j=0$ for $j>M$.
\end{DLtt}
\Restrict
$\mathtt{MODE=0:}$ $\mathtt{1 \le N \le 1229}$. \\
$\mathtt{MODE=1}$ or $\mathtt{MODE=2:}$ $\mathtt{2 \le N \le 10007}$.
\Errorh
Error {\tt B002.1:} $\mathtt{MODE \ne 0}$ and $\mathtt{MODE \ne 1}$
and $\mathtt{MODE \ne 2}$. \\
Error {\tt B002.2:} $\mathtt{N}$ out of range. \\
In both cases, {\tt NPRIME(j)} and {\tt NPOWER(j)},
($\mathtt{j=1,2,\ldots,N}$) are set to zero and a message is written
on \linebreak[4]
{\tt Unit 6}, unless subroutine {\tt MTLSET} (N002) has been called.
\\ $\bullet$
