% 20 may 1992  mg
\Version{ABUSER}                          \Routid{Z036}
\Keywords{ABUSER FORTRAN IBM REENT}
\Author{R. Matthews, A. Cass}           \Library{KERNLIB, IBM only}
\Submitter{}                              \Submitted{01.02.1983 }
\Language{Assembler}                      \Revised{19.07.1988}
\Cernhead {Intercept a Fortran Abend on IBM}
{\tt ABUSER} enables a user-supplied subroutine to receive control when
the user's program abends. A call to {\tt ABUSER} identifies the
user-supplied subroutine which is to receive control. The
identified subroutine will be called if the user's program
abends and can perform pre-termination processing such as
printing summaries or plotting histograms.
\Structure
{\tt SUBROUTINE} subprogram \\
User Entry Names: \Rdef{ABUSER}
\Usage
\begin{verbatim}
    CALL ABUSER(NAME)
\end{verbatim}
\begin{DLtt}{12345678}
\item [NAME] Name of a user-supplied {\tt SUBROUTINE} subprogram
declared {\tt EXTERNAL} in the calling program.
\end{DLtt}
This subprogram
receives control via a call of the form
\begin{verbatim}
    CALL NAME(KODES,KODEU)
\end{verbatim}
\begin{DLtt}{123456}
\item[KODES] A 4-byte integer containing, if available, the system
completion code as hexadecimal number (use {\tt Z} format for printing).
\item[KODEU] A 4-byte integer containing, if available, the user
completion code as integer number (use {\tt I} format for printing).
\end{DLtt}
\Restrict
This subprogram is compiler and system dependent.
\par
MVS:
\par
The Fortran 4 version relies on modifications to the IBM
H-extended compiler  library and is therefore not portable. The
Fortran 77 version uses a standard interface into the {\tt FACOM}
compiler library.
\par
CMS:
\par
The subprogram is compiler independent but {\tt KODES} and {\tt KODEU}
are not available and so are set to zero. Note that the routine uses
storage in the CMS nucleus -- the {\tt NUSERFWD} field and also 8-bytes
at {\tt NCCOPYR} -- which must not be overwritten. (No other CERN Library
routine uses these locations.)
\Notes
{\tt ABUSER} can be called at any time during normal processing,
(i.e. before an abend occurs), to re-specify the name of the
user-supplied subroutine. Alternatively, the effect of previous calls
can be cancelled by {\tt CALL ABUSER(0)}. A call to {\tt ABUSER}
after an abend will have no effect.
\par
A secondary abend which occurs while the user is processing the primary
abend will cause program termination.
\newpage
Under MVS the user-supplied subroutine will not receive control for the
following completion codes:\\
\begin{DLtt}{123456}
\item[122] -- job cancelled with dump
\item[222] -- job cancelled
\item[322] -- cpu time exceeded
\item[522] -- wait time limit exceeded
\end{DLtt}
\Examples
In the following example, {\tt ABUSER} is called to identify a subroutine
called {\tt FATAL} as the subroutine which is to receive control when the
user's program abends. If an abend occurs, subroutine {\tt FATAL} will
be called and will print the completion codes and then call {\tt HISTDO}
to plot histograms.
\begin{verbatim}
    EXTERNAL FATAL
    ...
    CALL ABUSER(FATAL)
    ...
    END
    SUBROUTINE FATAL(KODES,KODEU)
    WRITE(6,'(1X,''PROGRAM ABENDING WITH CODES '',Z3,I5)') KODES,KODEU
    CALL HISTDO
    RETURN
    END
\end{verbatim}
$\bullet$
