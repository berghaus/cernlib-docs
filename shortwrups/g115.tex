\Version{VAVLOV}               \Routid{G115}
\Keywords{DENSITY DISTRIBUTION LANDAU VAVILOV RANDOM NUMBER INVERSE}
\Author{A. Rotondi, P. Montagna, K.S. K\"olbig}  \Library{MATHLIB}
\Submitter{K.S. K\"olbig}              \Submitted{10.12.1993}
\Language{Fortran}         %  \Revised{}
\Cernhead{Approximate Vavilov Distribution and its Inverse}
The {\tt VAVLOV} package contains subprograms for fast approximate
calculation of functions related to the Vavilov distribution.
\par
For $\kappa>0$ and $0 \le \beta^2 \le 1$,
the Vavilov density function is {\it mathematically} defined by
$$ \phi_V(\lambda;\kappa,\beta^2) \ = \ \displaystyle \frac{1}{2\pi i}\,
\int_{c-i\infty}^{c+i\infty} e^{\lambda s}\,f(s;\kappa,\beta^2)\,ds,$$
where $c$ is an arbitrary real constant and
$$ f(s;\kappa,\beta^2) \ = \ \displaystyle
C(\kappa,\beta^2)\,\exp \left\{s \ln \kappa + (s+\kappa \beta^2)\,
\left[ \ln \left(\frac{s}{\kappa}\right)
+E_1\left(\frac{s}{\kappa}\right) \right]
-\kappa\,\exp \left(-\frac{s}{\kappa}\right) \right\}. $$
$E_1(x)=\int_0^x t^{-1}\,(1-e^{-t})\,dt$ is the exponential integral,
$C(\kappa,\beta^2)=\exp\{\kappa(1+\beta^2 \gamma)\}$,
and $\gamma=0.57721\dots\,$ is Euler's constant.
\par
The Vavilov distribution function is defined by
$$ \Phi_V(\lambda;\kappa,\beta^2) \ = \ \displaystyle
\int_{-\infty}^\lambda \phi_V(\lambda;\kappa,\beta^2)\,d\lambda$$
and its inverse by
$\Psi_V(x;\kappa,\beta^2)=\Phi_V^{-1}(x;\kappa,\beta^2)$.
\par
The function $\Psi_V(x;\kappa,\beta^2)$ can be used to generate
Vavilov random numbers (see {\bf Usage}).
\Structure
{\tt SUBROUTINE} and {\tt FUNCTION} subprograms \\
User Entry Names: \Rdef{VAVSET}, \Rdef{VAVDEN}, \Rdef{VAVDIS},
                  \Rdef{VAVRND}, \Rdef{VAVRAN} \\
External References: \Rind{LOCATR}{E106},
                     \Rind{DENLAN}{G110}, \Rind{DISLAN}{G110} \\
{\tt COMMON} Block Names and Lenghts: {\tt /G115C1/ 226}
\Usage
\begin{verbatim}
    CALL VAVSET(RKAPPA,BETA2,MODE)
\end{verbatim}
sets auxiliary quantities used in {\tt VAVDEN}, {\tt VAVDIS} and
{\tt VAVRND}; this call has to precede a reference to any of these
entries.
\begin{DLtt}{12345678}
\item[RKAPPA] The variable $\kappa$ (the straggling parameter);
($0.01 \le \kappa \le 12$).
\item[BETA2] The variable $\beta^2$ (the square of velocity in unit $c$);
($0 \le \beta^2 \le 1$).
\item[MODE] $\mathtt{= 1;}$ \\
$\mathtt{= 0}$ in the particular case that {\tt VAVDEN} only is
referenced after the call to {\tt VAVSET}. \\
\end{DLtt}
In any arithmetic expression,
\begin{center}
\begin{tabular}{r@{\qquad has an approximate value of \qquad}l}
{\tt VAVDEN(X)} & $\phi_V(\mathtt{X;RKAPPA,BETA2})$, \\
{\tt VAVDIS(X)} & $\Phi_V(\mathtt{X;RKAPPA,BETA2})$, \\
{\tt VAVRND(X)} & $\Psi_V(\mathtt{X;RKAPPA,BETA2})$,
\end{tabular} \end{center}
\par
{\tt RKAPPA} and {\tt BETA2} are defined by the last call to
{\tt VAVSET} prior to a reference to {\tt VAVDEN}, {\tt VAVDIS}, or
{\tt VAVRND}.
\par
To generate a {\it set} of Vavilov random numbers with identical $\kappa$
and $\beta^2$, {\tt VAVSET} should be called once and then {\tt VAVRND}
be referenced repeatedly, using as argument {\tt X} a random number
from a uniform distribution over the interval (0,1).
\par
In any arithmetic expression,
\begin{center}
{\tt VAVRAN(RKAPPA,BETA2,X)} \qquad has an approximate value of \qquad
$\Psi_V(\mathtt{X;RKAPPA,BETA2})$.
\end{center}
To generate {\it one} Vavilov random number for given values of $\kappa$
and $\beta^2$, {\tt VAVRAN} should be used, using as argument
{\tt X} a random number from a uniform distribution over the interval
(0,1).
\par
{\tt VAVDEN}, {\tt VAVDIS}, {\tt VAVRND}, {\tt VAVRAN}
and {\tt X}, {\tt RKAPPA}, {\tt BETA2} are of type
{\tt REAL}, and {\tt MODE} is of type {\tt INTEGER}.
\Method
The method is discribed in Ref. 1.
\Accuracy
The accuracy depends on the parameters. Although often rather poor
from a mathematical point of view, it is usually sufficient for
the intended application in physics (see {\bf Notes}).
\Restrict
No test is made whether the parameters $\kappa$ and $\beta^2$
are in the specified ranges.
\Notes
\begin{enumerate}
\item Representing the Vavilov functions by approximations which are
both fast and accurate is a difficult task. In view
of the requirements in physics, speed is much more important than
accuracy. This is taken into account for the present routines.
\item For a more accurate, but much slower, calculation of the
Vavilov density and distribution functions, use {\tt VVILOV} (G116).
\item For $\kappa \le 0.01$, the Vavilov distribution can be
replaced by the Landau distribution ({\tt LANDAU} (G110)),
taking into account that $\lambda_V=(\lambda_L-\ln \kappa)/\kappa$.
\item For $\kappa \ge 10$, the Vavilov distribution can be
replaced by the Gaussian distribution with mean \\
$\mu=\gamma -1-\beta^2-\ln \kappa$ and variance
$\sigma^2=(2-\beta^2)/(2\kappa)$.
\end{enumerate}
\Refer
\begin{enumerate}
\item A. Rotondi and P. Montagna, Fast calculation of Vavilov
distribution, Nucl. Instr. and Meth. {\bf B47} (1990) 215--224.
\end{enumerate}
$\bullet$
