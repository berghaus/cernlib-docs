% 20 jan 95   ksk
\Version {RM48}                         \Routid{V116}
\Keywords{UNIFORM DISTRIBUTION NUMBER RANDOM GENERATOR DOUBLE PRECISION}
\Author{F. James}             \Library{MATHLIB}
\Submitter{}          \Submitted{15.03.1994}
\Language{Fortran}                   %\Revised{}
\Cernhead{Double Precision Uniform Random Numbers}
{\tt RM48} generates pseudorandom numbers using a double-precision
(64-bit) adaptation of {\tt RANMAR} (V113).  The floating-point numbers
in the interval (0,1), the end points excluded,
have 48 significant bits of mantissa (additional bits of
mantissa, if supported by the hardware, are zero).  Both the
code and the results are portable, provided the floating-point mode
is adapted to the computer being used (for example, single-precision
mode on 64-bit machines, double-precision mode on 32-bit machines).
\Structure
{\tt SUBROUTINE} Subprograms\\
User Entry Names: \Rdef{RM48}, \Rdef{RM48IN}, \Rdef{RM48UT}
\Usage
\begin{verbatim}
    CALL RM48(RVEC,LEN)
\end{verbatim}
returns a vector {\tt RVEC} of {\tt LEN}
64-bit random floating-point numbers in (0,1), the end points excluded.
{\tt RVEC} is an array of length {\tt LEN} at least. It is of type
{\tt DOUBLE PRECISION} on 32-bit machines, and of type {\tt REAL}
otherwise.
\begin{verbatim}
    CALL RM48IN(I1,N1,N2)
\end{verbatim}
initializes the generator from one
32-bit integer {\tt I1}, and number counts {\tt N1}, {\tt N2}
(for initializing, set $\mathtt{N1=N2=0}$, but to restart
a previously generated sequence, use values output by {\tt RM48UT}).
\begin{verbatim}
    CALL RM48UT(I1,N1,N2)
\end{verbatim}
outputs the value of the original seed and the two number counts,
to be used for restarting by initializing to {\tt I1} and
skipping $\mathtt{100000000*N2+N1}$ numbers.
\Method
The method is that of {\tt RANMAR} (V113). \\
$\bullet$
