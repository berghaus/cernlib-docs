%  20 feb 1995   ksk
\Version {ZEBRA}                      \Routid{Q100}
\Keywords{DATA DYNAMIC STRUCTURE MEMORY MANAGE}
\Author{R. Brun, M. Goossens, B. Holl, O. Schaile, J. Shiers, J. Zoll}
\Library{PACKLIB}
\Submitter{}                            \Submitted{18.04.1986}
\Language{Fortran}              %\Revised{}
\Cernhead {Dynamic Data Structure and Memory Manager}
{\tt ZEBRA} is a dynamic data structure and memory manager. It allows
the management of large amounts of data in a computer store by providing
the functions required to construct a logical graph of the data and
their interrelations.
\par
The data are stored in Fortran {\tt COMMON} blocks, called
"stores". Each store can be subdivided into up to 20 "divisions".
Relations between the basic units of data, or "banks", are expressed by
attaching a structural significance to part of a bank. A bank is
accessed by specifying its address in a given store. Such addresses
(called "links") are kept inside the banks or in "link areas" inside
a common block.
\begin{itemize}
\item  The memory management part of {\tt ZEBRA} is performed by
the {\tt MZ} package. Utilities are available for reorganizing, sorting
and deleting banks and data structures.
\item  Individual banks, data structures or complete divisions
can be output with the {\tt FZ} package.
\item Direct access files for data structures and the
management of the data by keywords are provided by the {\tt RZ} package.
\item  Dumps and verification of {\tt ZEBRA} structures and
documentation tools are available in the {\tt DZ} package.
\end{itemize}
\Structure
{\tt SUBROUTINE} subprograms \\
User Entry Names: \Rdef{ZEBRA}\\
External References: \Rind{KERNLIB}{Q100} routines
\Usage
See {\bf Long Write-up.}
\\ $\bullet$
