% 29 Jun 1992 mg
\Version{SIGMA}                          \Routid{Q122}
\Keywords{Q121}
\Keywords{INTERACTIVE GRAPHIC MATHEMATIC APPLICATION}
\Author{C. Vandoni}                      \Library{PAWLIB}
\Submitter{}                   \Submitted{14.11.1988}
\Language{Fortran}                %\Revised{}
\Cernhead {SIGMA - System for Interactive Graphical Mathematical
Applications}
{\tt SIGMA} can be considered a system for interactive on-line numerical
analysis problem-solving which has been designed essentially for
mathematicians and theoretical physicists.
The major characteristics of {\tt SIGMA} are:
\begin{itemize}
\item  The basic data units are scalars, one-dimensional arrays, and
multi-dimensional rectangular arrays; {\tt SIGMA} provides automatic
handling of these arrays.
\item  The calculational operators of {\tt SIGMA} closely resembles the
operations of numerical mathematics; procedural operators are often
analogous to those of Fortran.
\item  The system is designed to be used in interactive mode; it
provides convenient facilities for graphical display of arrays in form
of (sets of) curves.
\item  The user can construct his own programs within the system
and has also access to a program library; he can store and retrieve his
data and programs; he obtains on request hard copy of alphanumeric and
graphical type.
\end{itemize}
{\tt SIGMA} was operational for many years on the CYBER computers
at CERN. Most of its functionality has been converted to run on other
machines as part of the {\tt PAW} (Q121) package.
\Usage
See Chapter 6 of the {\tt PAW} Manual.
\\ $\bullet$
