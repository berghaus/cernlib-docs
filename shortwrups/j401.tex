% 20 may 1992  mg
\Version {BANNER}                            \Routid{J401}
\Keywords{BANNER PRINT LARGE CHARACTER PAGE OUTPUT}
\Author{R. Matthews}                         \Library{PGMLIB}
\Submitter{T. Lindel\"of}                       \Submitted{01.03.1978}
\Language{IBM: Fortran and Assembler}       \Revised{14.11.1986}
\Cernhead {Print Banner Page in Large Characters}
{\tt BANNER} outputs alphanumeric text specified in an MVS {\tt JCL}
(job control card) statement, a VM/CMS parameter string or from a data
file, in large characters, at 4 lines per output page.
\Structure
IBM VM/CMS: Main {\tt PROGRAM}, IBM/MVS: {\tt JCL} Procedure \\
User Entry Names: \Rdef{BANNER}\\
Files Referenced: Printer\\
External References:
\Rind{DATIMH} (Z007), \Rind{JOBNAM} (Z100),
\Usage
{\tt BANNER} is called through a {\tt JCL} statement
on IBM/MVS and as a command in VM/CMS.
\par
On MVS the procedure {\tt BANNER} invokes a program which prints
lines of text of a type similar to that used for the first page of job
output. Input to the program may be supplied by using the {\tt TEXT}
parameter on the {\tt EXEC} statement or by data cards or both. Input
supplied using the {\tt TEXT} parameter is processed first and is
printed as a double banner page.
\par
On VM/CMS {\tt BANNER} is installed as a module on the automatically
accessed {\tt Q}-disk. Input to the program can be
given either as a parameter string on the command line or in a data file
on Fortran stream {\tt 5} but not both (this is unlike the
MVS version which accepts input from both). The input given on the
command line is printed as a double banner page.
\par
Three types of block character are available: large, small and
italic. Each of these character sets contains upper-case alphabetic
characters, numeric characters and the special characters
{\tt ( ) + - * / = < > \# @ \% \_ | \  ! .} and {\tt \&}.
Some examples of each character set are shown in {\bf Examples}.
\par
The number of block characters which can be printed on a page depends on
the paper size and character set being used as shown in the Tables given
below \\[3mm]
{\tt TEXT} {\bf parameter}{\bf :} \\[2mm]
The {\tt TEXT} parameter in the {\tt MVS} procedure may be used to input
up to 98 characters. The {\tt /} character causes the text which
follows it to be printed on a new line. The character set which is used
to print the text may be specified by using the {\tt TYPE} parameter.
\begin{DLtt}{1234567890}
\item [TYPE $=$ I] Selects the italic character set.
\item [TYPE $=$ L] Selects the large character set.
\item [TYPE $=$ S] Selects the small character set.
\end{DLtt}
If the type is not specified or if an invalid type is
specified, the large character set will be used. For example:
\begin{verbatim}
// EXEC BANNER,TEXT='I B M/RULES//OK',TYPE=I
\end{verbatim}
\newpage
On VM/CMS the type is given as an option behind a {\tt (} after the
text. The above example on VM/CMS would be:
\begin{verbatim}
BANNER I B M/RULES//OK (ITALIC
\end{verbatim}
Type {\tt XFIND BANNER} on VM for more detail.
\par
Text can be centred on a line by using blanks, and blank lines can
be inserted by using consecutive slashes. \\[3mm]
{\bf Data Cards:} \\[2mm]
The program accepts data cards containing either text or
commands. Cards containing commands may be used to select the
character set or cause page ejects. Commands consist of a {\tt .} in
column 1 and the character {\tt I}, {\tt L}, {\tt S} or {\tt /} in
column 2:
\begin{DLtt}{123456}
\item[.I]   Selects the italic character set.
\item[.L]   Selects the large character set.
\item[.S]   Selects the small character set.
\item[./]   Causes a page eject.
\end{DLtt}
The selected character set remains in effect until changed by a new
command card. If a character set is not specified the large character
set will be used by default. For example, on MVS, \\[3mm]
\begin{tabular}{l@{\hspace*{25mm}}|@{\hspace*{25mm}}l}
\multicolumn{2}{l}{{\tt // EXEC BANNER}} \\
\multicolumn{2}{l}{{\tt //B.SYSIN DD *}} \\
{\tt .S} \\
{\tt THIS IS AN} \\
{\tt EXAMPLE OF} \\
{\tt \ \ SMALL}  \\
{\tt CHARACTERS}  &   data cards for {\tt BANNER} \\
{\tt ./} \\
{\tt .I} \\
{\tt NEW PAGE} \\
{\tt \ \ \ OF} \\
{\tt ITALIC} \\
\multicolumn{2}{l}{{\tt /*}}
\end{tabular} \\[3mm]
On VM/CMS if no parameter string is given on the command line
the program will read from Fortran stream {\tt 5} (which is by
default the terminal -- this will cause an {\it Abend} in batch
execution). To read the data from a file use the {\tt INPUT} option.
The page capacities depend on the printer and forms code chosen.
\newpage
\Examples
Here are three examples of character sets (printer dependent):
\begin{verbatim}
 
   SMALL
 
  AAAAA   BBBBBB      1      22222   %%%   %
 A     A  B     B    11     2     2  % %  %      +
 A     A  B     B     1          2   %%% %       +
 AAAAAAA  BBBBBB      1        22       %      +++++
 A     A  B     B     1       2        % %%%     +
 A     A  B     B     1      2        %  % %     +
 A     A  BBBBBB    11111   2222222  %   %%%
 
   LARGE
 
 AAAAAAAAAA   BBBBBBBBBBB        11        2222222222
AAAAAAAAAAAA  BBBBBBBBBBBB      111       222222222222
AA        AA  BB        BB     1111       22        22
AA        AA  BB        BB       11                 22
AA        AA  BB       BB        11                 22
AAAAAAAAAAAA  BBBBBBBBBB         11                22
AAAAAAAAAAAA  BBBBBBBBBB         11              22
AA        AA  BB       BB        11            22
AA        AA  BB        BB       11          22
AA        AA  BB        BB       11        22
AA        AA  BBBBBBBBBBBB   1111111111   222222222222
AA        AA  BBBBBBBBBBB    1111111111   222222222222
 
   ITALIC
 
           AAAAAAAAAA   BBBBBBBBBBB        11
          AAAAAAAAAAAA  BBBBBBBBBBBB      111
         AA        AA  BB        BB     1111
        AA        AA  BB        BB       11
       AA        AA  BB       BB        11
      AAAAAAAAAAAA  BBBBBBBBBB         11
     AAAAAAAAAAAA  BBBBBBBBBB         11
    AA        AA  BB       BB        11
   AA        AA  BB        BB       11
  AA        AA  BB        BB       11
 AA        AA  BBBBBBBBBBBB   1111111111
AA        AA  BBBBBBBBBBB    1111111111
\end{verbatim}
$\bullet$
