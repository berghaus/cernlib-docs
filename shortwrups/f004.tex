\Version{RMMLT}                          \Routid{F004}
\Keywords{MATRIX MULTIPLICATION OPERATION}
\Authors {H. Lipps}                      \Library{KERNLIB}
\Submitter{}                           \Submitted{18.12.1979}
\Language{Fortran or Assembler or COMPASS} \Revised{27.05.1987}
\Cernhead{Matrix Multiplication}
These subprograms calculate the matrix product
$$\mathbf{Z=XY} \ \mbox{\rm or} \ \mathbf{Z=X\overline{Y}},$$
where {\bf $\overline{Y}$} denotes the conjugate of the complex matrix
{\bf Y}, or one of the matrix expressions
$$\mathbf{Z=XY+Z, \quad Z=XY-Z, \quad Z=-XY+Z, \quad Z=-XY-Z}.$$
\Structure
{\tt SUBROUTINE} subprograms \\
User Entry Names:
\begin{htmlonly}
\begin{tabular}{llllll}
\end{htmlonly}
\begin{latexonly}
\begin{tabular}[t]{l*{5}{@{\hspace{4pt}}l}}
\end{latexonly}
\Rdef{RMMLA}, & \Rdef{RMMLS}, & \Rdef{RMMLT}, & \Rdef{RMNMA}, &
\Rdef{RMNMS}, \\
\Rdef{DMMLA}, & \Rdef{DMMLS}, & \Rdef{DMMLT}, & \Rdef{DMNMA}, &
\Rdef{DMNMS}, \\
\Rdef{CMMLA}, & \Rdef{CMMLS}, & \Rdef{CMMLT}, & \Rdef{CMNMA}, &
\Rdef{CMNMS}, & \Rdef{CMMLTC}
\end{tabular} \\
External References: \Rind{LOCF}{N100}
(some Fortran versions only).
\Usage
For $\mathtt{t=R}$ (type {\tt REAL}), $\mathtt{t=D}$ (type
{\tt DOUBLE PRECISION}), $\mathtt{t=C}$ (type {\tt COMPLEX}):
\begin{center}
\begin{tabular}{l@{\qquad}l}
{\tt CALL tMMLT (M,N,K,X11,X12,X21,Y11,Y12,Y21,Z11,Z12,Z21,W)} &
$\mathbf{Z=XY}$ \\
{\tt CALL tMMLA (M,N,K,X11,X12,X21,Y11,Y12,Y21,Z11,Z12,Z21)} &
$\mathbf{Z=XY+Z}$ \\
{\tt CALL tMMLS (M,N,K,X11,X12,X21,Y11,Y12,Y21,Z11,Z12,Z21)} &
$\mathbf{Z=XY-Z}$ \\
{\tt CALL tMNMA (M,N,K,X11,X12,X21,Y11,Y12,Y21,Z11,Z12,Z21)} &
$\mathbf{Z=-XY+Z}$ \\
{\tt CALL tMNMS (M,N,K,X11,X12,X21,Y11,Y12,Y21,Z11,Z12,Z21)} &
$\mathbf{Z=-XY-Z}$ \\
{\tt CALL CMMLTC(M,N,K,X11,X12,X21,Y11,Y12,Y21,Z11,Z12,Z21,W)} &
$\mathbf{Z=X\overline{Y}}$ \\
\end{tabular}
\end{center}
\begin{DLtt}{123456789012}
\item[M,N,K] ({\tt INTEGER}) The mathematical dimensions of the
matrices: {\bf X} has {\tt M} rows and {\tt N} columns, {\bf Y}
has {\tt N} rows and {\tt K} columns, {\bf Z} has {\tt M} rows and
{\tt K} columns.
\item[X11,X12,X21] (Type according to {\tt t}) Array elements. They must
contain the elements $x_{11},x_{12},x_{21}$ of the matrix {\bf X}.
\item[Y11,Y12,Y21] (Type according to {\tt t}) Array elements. They must
contain the elements $y_{11},y_{12},y_{21}$ of the matrix {\bf Y}.
\item[Z11,Z12,Z21] (Type according to {\tt t}) Array elements. On exit,
they will contain the elements $z_{11},z_{12},z_{21}$ of the matrix
{\bf Z}.
\item[W] (Type according to {\tt t}) Working space array as specified
below, required only if {\bf Z} overlaps {\bf X} or {\bf Y}.
Otherwise a dummy variable.
\end{DLtt}
For $\mathtt{M < 1}$ or $\mathtt{N < 1}$ or $\mathtt{K < 1}$,
all subroutines return control without action.
\par
The matrices {\bf X}, {\bf Y} and {\bf Z} need not to be stored
according to the
Fortran conventions: any equidistant spacing of their rows and
columns is permitted. In particular, matrices may be stored row-wise.
Each subroutine can work with the transpose of a matrix. To make this
possible, each matrix is specified in the calling sequence by three
arguments. For example, the called subroutine will operate on the
matrix $\mathbf{A}=(a_{ij})$ if the actual arguments which replace
{\tt X11}, {\tt X12}, {\tt X21} in the calling sequence are
$a_{11},a_{12},a_{21}$, and will operate on the transpose $\mathbf{A'}$
of {\bf A} if the actual arguments are $a_{11},a_{21},a_{12}$.
\par
The only cases in which the result matrix {\bf Z} is permitted to
overlap {\bf X} or {\bf Y} are the following:
\begin{center}
\begin{tabular}{ll@{\quad or \quad}ll}
{\tt tMMLT:} & $\mathbf{X=XY}$ & $\mathbf{Y=Y'Y}$,
& provided {\tt W} is an array of at least {\tt K} elements. \\
            & $\mathbf{Y=XY}$ & $\mathbf{X=XX'}$,
& provided {\tt W} is an array of at least {\tt M} elements. \\
{\tt CMMLTC:} & $\mathbf{X=X\overline{Y}}$ & $\mathbf{Y=Y'\overline{Y}}$,
& provided {\tt W} is an array of at least {\tt K} elements. \\
              & $\mathbf{Y=X\overline{Y}}$ & $\mathbf{X=X\overline{X}'}$,
& provided {\tt W} is an array of at least {\tt M} elements. \\
\end{tabular}
\end{center}
\Accuracy
On computers with IBM 370 architecture, all routines that
accumulate the inner product of type {\tt REAL} or {\tt COMPLEX} use
double-precision arithmetic internally;
the final result is then rounded to single precision.
\Notes
The product of a matrix and its transpose (or Hermitian conjugate)
is recognized by {\tt tMMLT} (or {\tt CMMLTC}) and the computation is
shortened accordingly.
\Examples
Assume that the two-dimensional arrays {\tt A}, {\tt B}, {\tt C},
{\tt D}, {\tt E}, the one-dimensional array {\tt W}, and the dummy
variable {\tt V} are declared by
\begin{verbatim}
    COMPLEX A(9,9),B(9,9),C(9,9),D(9,9),E(9,9),V,W(99)
\end{verbatim}
and that a $4\times5$ matrix {\bf A}, a $5\times7$ matrix {\bf B},
and a $7\times3$ matrix {\bf C} have been stored according to the
Fortran conventions in arrays of corresponding name.
\begin{enumerate}
\item To compute $\mathbf{D=AB}$:
\begin{verbatim}
    CALL CMMLT (4,5,7,A,A(1,2),A(2,1),B,B(1,2),B(2,1),D,D(1,2),D(2,1),V).
    \end{verbatim}
To pack the $4\times7$ product matrix \textbf{AB} row-wise into array
{\tt W}:
\begin{verbatim}
    CALL CMMLT (4,5,7,A,A(1,2),A(2,1),B,B(2,1),B(1,2),W,W(2),W(8),V).
\end{verbatim}
(Note that $z_{11}$ goes into {\tt W(1)}, $z_{12}$ into {\tt W(2)},
and $z_{21}$ into {\tt W(8)}).
\par
For the purpose of abbreviation we shall denote \\
{\tt A,A(1,2),A(2,1)} by {\tt a}, \quad
{\tt A,A(2,1),A(1,2)} by {\tt a'}, \\
and similarly for arrays {\tt B}, {\tt C}, {\tt D}, {\tt E}. The first
example above then becomes
\begin{verbatim}
    CALL CMMLT(4,5,7,a,b,d,V).
\end{verbatim}
\item To compute $\mathbf{D=B'A'=(AB)'}$:
\begin{verbatim}
    CALL CMMLT(7,5,4,b',a',d,V)  or  CMMLT(4,5,7,a,b,d',V).
\end{verbatim}
\item To compute $\mathbf{D=AA'}$ and $\mathbf{E=A'A}$:
\begin{verbatim}
    CALL CMMLT(4,5,4,a,a',d,V)
    CALL CMMLT(5,4,5,a',a,e,V).
\end{verbatim}
\item To replace {\bf A} by {\bf AB} or by ${\bf AA'}$:
\begin{verbatim}
    CALL CMMLT(4,5,7,a,b,a,W)  or  CALL CMMLT(4,5,4,a,a',a,W).
\end{verbatim}
These two calls require a working vector {\tt W} containing 7 or 4
complex elements, respectively.
\item To compute $\mathbf{D=A\overline{B}}$ and
$\mathbf{E=\overline{B}C=(C'\overline{B}')'}$:
\begin{verbatim}
    CALL CMMLTC(4,5,7,a,b,d,V)
    CALL CMMLTC(3,7,5,c',b',e',V).
\end{verbatim}
\end{enumerate}
$\bullet$
