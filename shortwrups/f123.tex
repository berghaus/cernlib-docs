% 18 Aug 1994 jds
\Version{BVSL}                    \Routid{F123}
\Keywords{OPERATION BIT VECTOR SPARSE ARRAY FACILITY}
\Author{F. Antonelli}             \Library{MATHLIB}
\Submitter{F. Carminati}          \Submitted{27.11.1989}
\Language{Fortran, IBM Assembler} \Revised{16.08.1994}
\Cernhead {Bit Vector Manipulation Package}
This package contains high performance procedures to
operate with sparse arrays using Bit Vectors instead of ordinary Index
Vectors to address the elements of an arrays. The routines are,
at present, available only on IBM 3090 VF machines.
\Structure
{\tt SUBROUTINE} and {\tt FUNCTION} subprograms \\
User Entry Names: \\[1mm]
\begin{htmlonly}
\begin{tabular}{llllllllll}
\end{htmlonly}
\begin{latexonly}
\begin{tabular}[t]{l*{9}{@{\hspace{4pt}}l}}
\end{latexonly}
\Rdef{YLOSB},  & \Rdef{IYLOSB}, & \Rdef{YLOXB},  & \Rdef{IYLOXB}, \\
\Rdef{GTHRB},  & \Rdef{SCTTB},  & \Rdef{ANDB},   & \Rdef{XORB},   &
\Rdef{NOTB},   & \Rdef{NANDB},  & \Rdef{NORB},   & \Rdef{ORB},    &
\Rdef{BINVEC}, & \Rdef{ZEROB}, \\ \Rdef{ONEB},   & \Rdef{CNTOB},  &
\Rdef{CNTZB},  & \Rdef{RANGB},  & \Rdef{INTGB},  & \Rdef{RJCTB},  &
\Rdef{SXPYB},  & \Rdef{VXPYB},  & \Rdef{SXYB},   & \Rdef{XPWZB},  \\
\Rdef{DOTB},   & \Rdef{SCALB},  & \Rdef{VSETB},  & \Rdef{COPYB}
\end{tabular}
\Usage
The arguments in the calling sequences below are defined as follows:
\begin{DLtt}{1234567890}
\item[NW] ({\tt INTEGER}) Number of elements to process. The index
{\tt i} below runs from {\tt 1} to {\tt NW}.
\item[Y,X,V,W] ({\tt REAL}) Arrays of length {\tt NW} at least.
\item[IX,IY] ({\tt INTEGER}) Arrays of length {\tt NW} at least.
\item[S,T] ({\tt REAL}) Variables or expressions.
\item[IS,IT] ({\tt INTEGER}) Variables or expressions.
\item[BV,BV1,BV2] Arrays of length $\mathtt{(NW-1)/32 +1}$
at least, used to contain the bit vectors.
\item[IFOUND] ({\tt INTEGER}) Number of elements which satisfy the
                 condition, or set-bit count, for {\tt BV}.
\end{DLtt}
The expression {\tt X(BV)} indicates all these elements of
the vector {\tt X} for which the corresponding bit is set in
the bit array {\tt BV}. {\tt BV(i)} indicates the {\tt i}-th bit of the
array {\tt BV}, counted across words boundaries. The expression
$\mathtt{BV(i)=1}$ means that the {\tt i}-th bit of the array {\tt BV}
is set. \\[3mm]
{\bf Vector to scalar comparison:} \\[3mm]
Two {\tt SUBROUTINE} subprograms are provided for
{\tt REAL} and {\tt INTEGER} comparison. The subprogram {\tt YLOSB}
is for vectors with {\tt REAL} elements and the subprogram
{\tt IYLOSB} for vectors with {\tt INTEGER} elements. \\[3mm]
\begin{tabular}{@{\hspace*{5mm}}l@{\qquad $\mathtt{BV(i)=1}$ \ if \ }l}
{\tt CALL YLOSB(NW,Y,S,BV,IFOUND,'EQ')} & $\mathtt{Y(i) = S}$ \\
{\tt CALL YLOSB(NW,Y,S,BV,IFOUND,'NE')} & $\mathtt{Y(i) \ne S}$ \\
{\tt CALL YLOSB(NW,Y,S,BV,IFOUND,'GT')} & $\mathtt{Y(i) > S}$ \\
{\tt CALL YLOSB(NW,Y,S,BV,IFOUND,'LT')} & $\mathtt{Y(i) < S}$ \\
{\tt CALL YLOSB(NW,Y,S,BV,IFOUND,'GE')} & $\mathtt{Y(i) \ge S}$ \\
{\tt CALL YLOSB(NW,Y,S,BV,IFOUND,'LE')} & $\mathtt{Y(i) \le S}$ \\
{\tt CALL IYLOSB(NW,Y,S,BV,IFOUND,'EQ')} & $\mathtt{IY(i) = IS}$ \\
{\tt CALL IYLOSB(NW,Y,S,BV,IFOUND,'NE')} & $\mathtt{IY(i) \ne IS}$ \\
{\tt CALL IYLOSB(NW,IY,IS,BV,IFOUND,'GT')} & $\mathtt{IY(i) > IS}$ \\
{\tt CALL IYLOSB(NW,IY,IS,BV,IFOUND,'LT')} & $\mathtt{IY(i) < IS}$ \\
{\tt CALL IYLOSB(NW,IY,IS,BV,IFOUND,'GE')} & $\mathtt{IY(i) \ge IS}$ \\
{\tt CALL IYLOSB(NW,IY,IS,BV,IFOUND,'LE')} & $\mathtt{IY(i) \le IS}$
\end{tabular} \\[3mm]
{\bf Vector to vector comparison:} \\[3mm]
Two {\tt SUBROUTINE} subprograms are provided for
{\tt REAL} and {\tt INTEGER} comparison. The subprogram {\tt YLOXB}
is for vectors with {\tt REAL} elements and the subprogram
{\tt IYLOXB} for vectors with {\tt INTEGER} elements. \\[3mm]
\begin{tabular}{@{\hspace*{5mm}}l@{\qquad $\mathtt{BV(i)=1}$ \ if \ }l}
{\tt CALL YLOXB(NW,Y,X,BV,IFOUND,'EQ')} & $\mathtt{Y(i) = X(i)}$ \\
{\tt CALL YLOXB(NW,Y,X,BV,IFOUND,'NE')} & $\mathtt{Y(i) \ne X(i)}$ \\
{\tt CALL YLOXB(NW,Y,X,BV,IFOUND,'GT')} & $\mathtt{Y(i) > X(i)}$ \\
{\tt CALL YLOXB(NW,Y,X,BV,IFOUND,'LT')} & $\mathtt{Y(i) < X(i)}$ \\
{\tt CALL YLOXB(NW,Y,X,BV,IFOUND,'GE')} & $\mathtt{Y(i) \ge X(i)}$ \\
{\tt CALL YLOXB(NW,Y,X,BV,IFOUND,'LE')} & $\mathtt{Y(i) \le X(i)}$ \\
{\tt CALL IYLOXB(NW,Y,X,BV,IFOUND,'EQ')} & $\mathtt{IY(i) = IX(i)}$ \\
{\tt CALL IYLOXB(NW,Y,X,BV,IFOUND,'NE')} & $\mathtt{IY(i) \ne IX(i)}$\\
{\tt CALL IYLOXB(NW,IY,IX,BV,IFOUND,'GT')} & $\mathtt{IY(i) > IX(i)}$ \\
{\tt CALL IYLOXB(NW,IY,IX,BV,IFOUND,'LT')} & $\mathtt{IY(i) < IX(i)}$ \\
{\tt CALL IYLOXB(NW,IY,IX,BV,IFOUND,'GE')} & $\mathtt{IY(i) \ge IX(i)}$\\
{\tt CALL IYLOXB(NW,IY,IX,BV,IFOUND,'LE')} & $\mathtt{IY(i) \le IX(i)}$
\end{tabular} \\[3mm]
{\bf Scatter/gather operations:}
\begin{verbatim}
    CALL GTHRB(NW,X,BV,Y)       Y=X(BV)
    CALL SCTTB(NW,Y,BV,X)       Y(BV)=X
\end{verbatim}
Elements are gathered or scattered from vector {\tt X} into vector
{\tt Y} according to the bit mask contained in {\tt BV}. Only words for
which the corresponding bit is set are moved. \\[3mm]
{\bf Logical operations:} \\[3mm]
\begin{tabular}{@{\hspace*{5mm}}l@{\qquad $\mathtt{BV(i)=1}$ \ if \ }l}
{\tt CALL ANDB(NW,BV1,BV2,BV,IFOUND)} &
$\mathtt{BV1(i)=1 \wedge BV2(i)=1}$ \\
{\tt CALL ORB(NW,BV1,BV2,BV,IFOUND)} &
$\mathtt{BV1(i)=1 \vee BV2(i)=1}$ \\
{\tt CALL XORB(NW,BV1,BV2,BV,IFOUND)} &
\begin{tabular}{l}
                 $(\mathtt{BV1(i)=1 \vee BV2(i)=1) \wedge}$ \\
                 $\mathtt{\neg(BV1(i)=1 \wedge BV2(i)=1})$
\end{tabular}     \\
{\tt CALL NANDB(NW,BV1,BV2,BV,IFOUND)} &
$\mathtt{BV1(i)=0 \vee BV2(i)=0}$ \\
{\tt CALL NORB(NW,BV1,BV2,BV,IFOUND)} &
\begin{tabular}{l}
      $(\mathtt{BV1(i)=1 \wedge BV2(i)=1) \vee}$ \\
      $(\mathtt{BV1(i)=0 \wedge BV2(i)=0})$
\end{tabular}     \\
{\tt CALL NOTB(NW,BV1,BV,IFOUND)} & $\mathtt{BV(i)=1-BV1(i)}$ \\
\end{tabular} \\[3mm]
{\bf Miscellaneous operations:}
\begin{verbatim}
    CALL BINVEC(NW,BV,IVEC)
\end{verbatim}
 is equivalent to
\begin{verbatim}
    DO J = 1,NW
       IF bit J of BV is set THEN
          IVEC(IFOUND)=J
       ENDIF
    ENDDO
\end{verbatim}
\newpage
\begin{tabular}{@{\hspace*{5mm}}l@{\qquad}l}
{\tt CALL ZEROB(NW,BV)} & $\mathtt{BV(i)=0}$ \\
{\tt CALL ONEB (NW,BV)} & $\mathtt{BV(i)=1}$ \\
{\tt CALL CNTOB(NW,BV,IFOUND)} & $\mathtt{IFOUND=}$
Number of set bits \\
{\tt CALL CNTZB(NW,BV,IFOUND)} & $\mathtt{IFOUND=}$
Number of clear bits \\
{\tt CALL RANGB(NW,Y,S,T,BV,IFOUND)} &
$\mathtt{BV(i)=1}$ \ if \ $\mathtt{S \le Y(i) \le T}$ \\
{\tt CALL INTGB(NW,Y,V,W,BV,IFOUND)} &
$\mathtt{BV(i)=1}$ \ if \ $\mathtt{V(i) \le Y(i) \le W(i)}$ \\
\end{tabular}
\begin{verbatim}
    CALL RJCTB(RAN,X,FREJ,Y,BV,NW,NWOUT,ISWTCH)
\end{verbatim}
\begin{DLtt}{123456}
\item[RAN] Array of random numbers uniformly distributed
between zero and the maximum of the rejection function.
\item[X] Array of points where the rejection function is computed.
\item[FREJ] Array of values of the rejection function.
\item[Y] Array of accepted values of {\tt X}.
\item[BV] Bit vectors of length $\mathtt{(NW-1)/32+1}$ at least.
\item[NW] Initial number of values to extract.
\item[NWOUT] Current number of values left to extract.
\item[ISWTCH] Switch to be set to {\tt 1} for the first call.
\end{DLtt}
{\bf Linear algebra operations:} \\[3mm]
Let {\tt H} be an $\mathtt{NW \times NC}$ matrix. The {\tt FUNCTION}
subrogram {\tt DOTB} is of type {\tt REAL}. \\[3mm]
\begin{tabular}{@{\hspace*{5mm}}l@{\qquad}l}
{\tt CALL SXPYB(NW,BV,Y,X,S)} &   $\mathtt{Y(BV)=Y(BV)+S*X(BV)}$ \\
{\tt CALL VXPYB(NW,BV,X,Y,V)} &   $\mathtt{Y(BV)=Y(BV)+V(BV)*X(BV)}$ \\
{\tt CALL SXYB(NW,BV,X,Y,S)}  &   $\mathtt{Y(BV)=Y(BV)*V(BV)*S}$ \\
{\tt CALL XYPWZB(NW,BV,S,X,Y,T,W,Z)} &
$\mathtt{Y(BV)=S*X(BV)*Y(BV)+T*W(BV)*Z(BV)}$ \\
{\tt RES = DOTB(NW,BV,X,Y)}  &   $\mathtt{DOTB=\sum X(BV)*Y(BV)}$ \\
{\tt CALL SCALB(NW,BV,Y,S)}  &   $\mathtt{Y(BV)=Y(BV)*S}$ \\
{\tt CALL VSETB(NW,BV,Y,S)}  &   $\mathtt{Y(BV)=S}$       \\
{\tt CALL COPYB(NW,BV,Y,X)}  &   $\mathtt{Y(BV)=X(BV)}$
\end{tabular}
\\ $\bullet$
