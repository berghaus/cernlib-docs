% 19 oct 94  ksk
\Version{RMULLZ}                  \Routid{C202}
\Keywords{ZERO REAL POLYNOMIAL SOLUTION ROOT}
\Author{H.-H. Umst\"atter}         \Library{MATHLIB}
\Submitter{K.S. K\"olbig}         \Submitted{07.06.1992}
\Language{Fortran}        %     \Revised{ }
\Cernhead{Zeros of a Real Polynomial}
Subroutine subprogram {\tt RMULLZ} and {\tt DMULLZ} compute the
zeros of the polynomial
$$ P(z) \ = \ a_0z^n + a_1z^{n-1} +\ldots +a_{n-1}z + a_n $$
of degree $n$ with real coefficients $a_k$ and $a_0 \ne 0$.
\par
On computers other than CDC or Cray, only
the double-precision version {\tt DMULLZ} is available.
On CDC and  Cray computers, only the single-precision version
{\tt RMULLZ} is available.
\Structure
SUBROUTINE subprograms \\
User Entry  Names : \Rdef{RMULLZ}, \Rdef{DMULLZ} \\
Files Referenced: {\tt Unit 6} \\
External References: \Rind{MTLMTR}{N002}, \Rind{ABEND}{Z035}
\Usage
For $\mathtt{t=R}$ (type {\tt REAL}), $\mathtt{t=D}$ (type
{\tt DOUBLE PRECISION}),
\begin{verbatim}
    CALL tMULLZ(A,N,MAXIT,Z)
\end{verbatim}
\begin{DLtt}{123456}
\item[A] (type according to {\tt t}) One-dimensional array of dimension
{\tt (0:d)}, where $\mathtt{d \ge N}$, containing the coefficients
$a_k, (k = 0,1,\ldots,n)$.
\item[N] ({\tt INTEGER}) The degree $n$.
\item[MAXIT] ({\tt INTEGER}) The maximum number of iterations permitted.
\item[Z] ({\tt COMPLEX} for $\mathtt{t=R}$, {\tt COMPLEX*16} for
$\mathtt{t=D}$) One-dimensional array of length
$\mathtt{\ge N}$. On exit, $\mathtt{Z(i)}$ contains an approximation
to the zero $z_i$, listed in roughly decreasing order of $|z_i|$.
\end{DLtt}
\Method
The method of Muller (see Ref. 1) is used. This is based on iterated
inverse quadratic interpolation followed by deflation to remove
each zero as found.
\Accuracy
For well-conditioned polynomials (i.e. polynomials whose zeros
are not unduly sensitive to small errors in the coefficients), the
relative error of a computed zero of multiplicity $m$ is of order
$10^{-d/m}$ where $d$ is the machine precision expressed in decimal
digits. For $m>1$, the $m$ approximations to the single multiple zero
are uniformly distributed on a small circle of radius of order
$10^{-d/m}$ around the exact zero. Therefore, if the polynomial is
well-conditioned, the true value of the multiple zero will be close to
the centre $(z_{k+1}+\ldots +z_{k+m})/m$ of this circle.
\Errorh
Error {\tt C202.1:} $a_0 = 0$. \\
Error {\tt C202.2:} The number of iterations exceeds {\tt MAXIT}. \\
In both cases, a message is written on {\tt Unit 6},
unless subroutine {\tt MTLSET} (N002) has been called.
If the number of iterations exceeds {\tt MAXIT}, those
zeros which have not been found are set to $10^{20}$.
\newpage
\Notes
For difficult cases which lead to too many iterations the following
transformations may be applied, singly or together, to obtain a
better-conditioned polynomial:
\begin{enumerate}
\item Reverse the order of the coefficients to obtain a polynomial
whose zeros are $z_i^{-1}$.
\item  If the zeros $z_i$ are clustered, or are too unsymmetrically
positioned with respect to the origin,
compute by synthetic division
(see Ref. 3) the coefficients of the polynomial whose argument is
$w=z-\widehat{z}$, where $\widehat{z} = -a_1/(n a_0)$ is the arithmetic
mean of the zeros. The mean of the zeros of this new polynomial is
situated at the origin, which is where the subprogram starts searching.
Then, provided $|w_i|<|\widehat{z}|$ for most $i$,
$z_i = w_i+\widehat{z}$ will be more accurate zeros.
\end{enumerate}
\Refer
\begin{enumerate}
\item  D.E. Muller, A method for solving algebraic equations using an
automatic computer, MTAC (later renamed Math. Comp.) {\bf 10} (1956)
208--215.
\item J.W. Daniel, Correcting approximations to multiple roots of
polynomials, Numer. Math. {\bf 9} (1966) 99--102.
\item F.B. Hildebrand, Introduction to numerical analysis,
McGraw-Hill, New York (1956), Section 10.9.
\end{enumerate}
$\bullet$
