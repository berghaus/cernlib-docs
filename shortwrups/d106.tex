%   18 oct 94  ksk
\Version{RGS56P}                                \Routid{D106}
\Keywords{NUMERICAL GAUSSIAN QUADRATURE}
\Author{F. James}                          \Library{MATHLIB}
\Submitter{}                               \Submitted{01.12.1994}
\Language{Fortran}                    %    \Revised{}
\Cernhead{Gaussian Quadrature with Five- and Six-Point Rules}
Subroutine subprograms {\tt RGS56P} and {\tt DGS56P} calculate an
approximation and uncertainty for the integral
$$I \ = \ \int_a^b f(x)\,dx$$
equal respectively to the mean value and the difference of the
results $I_5$ and $I_6$ obtained by the five- and six-point
Gaussian integration rules.
\par
On CDC and Cray computers, the double-precision version {\tt DGS56P}
is not available.
\Structure
{\tt SUBROUTINE} subprograms \\
User Entry Names: \Rdef{RGS56P}, \Rdef{DGS56P} \\
External References: User-supplied {\tt FUNCTION} subprogram.
\Usage
For $\mathtt{t=R}$ (type {\tt REAL}), $\mathtt{t=D}$ (type
{\tt DOUBLE PRECISION}),
\begin{verbatim}
    CALL tGS56P(F,A,B,RES,ERR)
\end{verbatim}
\begin{DLtt}{123456}
\item[F] (type according to {\tt t}) Name of a user-supplied
{\tt FUNCTION} subprogram, declared {\tt EXTERNAL} in the calling
program. This subprogram must set $\mathtt{F(X)} = f(\mathtt{X})$.
\item[A,B] (type according to {\tt t}) End-points of integration
interval. Note that {\tt B} may be less than {\tt A}.
\item[RES] (type according to {\tt t}) The calculated approximation
for $I$, i.e. $\frac{1}{2}(I_5+I_6)$,
\item[ERR] (type according to {\tt t}) An estimated uncertainty
on this approximation, i.e. $|I_5-I_6|$.
\end{DLtt}
$\bullet$
