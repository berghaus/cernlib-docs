\Version {RAN2VS}                          \Routid{G901}
\Keywords{RANDOM POINT CIRCLE SPHERE}
\Author{H. Lipps}                           \Library{KERNLIB}
\Submitter{ }                              \Submitted{01.09.1983}
\Language{Fortran}                         \Revised{24.06.1985}
\Cernhead {Random Points on a Circle or Sphere}
These subroutines generate random points uniformly distributed on
the circumference of a circle ({\tt RAN2VS} and {\tt VRAN2S})
or on the surface of a sphere ({\tt RAN3VS} and {\tt VRAN3S});
i.e., 2- or 3-dimensional random vectors of specified length.
\Structure
{\tt SUBROUTINE} subprograms \\
User Entry Names: \Rdef{RAN2VS}, \Rdef{RAN3VS}, \Rdef{VRAN2S},
\Rdef{VRAN3S}\\
External References: \Rind{RANF} (G900)
\Usage
\begin{verbatim}
    CALL RAN2VS(RADIUS,X,Y)
    CALL RAN3VS(RADIUS,X,Y,Z)
    CALL VRAN2S(RADIUS,N,X,Y,R)
    CALL VRAN3S(RADIUS,N,X,Y,Z,R)
\end{verbatim}
\begin{DLtt}{12345678}
\item [RADIUS] ({\tt REAL}) Radius of the circle (sphere), with
centre at the origin, on which {\tt RAN2VS} and {\tt VRAN2S}
({\tt RAN3VS} and {\tt VRAN3S}) will calculate one or more points.
\item [N] ({\tt INTEGER}) Number of random points required.
\item [X,Y,Z] ({\tt REAL}) On exit, these contain the Euclidean
coordinates of the random point(s). In the case of {\tt VRAN2S} and
{\tt VRAN3S}, {\tt X}, {\tt Y}, {\tt Z} must be arrays of at least
{\tt N} elements.
\item [R]({\tt REAL}) Array of at least {\tt N} elements, required
as working space.
\end{DLtt}
\Method
{\tt RAN2VS} initially computes a random point $(x_1,y_1)$ uniformly
distributed over the interior of the square $-1 < x_1,y_1 < +1$,
using two calls to {\tt RANF} (G900). If this point lies outside the unit
circle $x_1^2 + y_1^2 \leq 1$ it is discarded and the process
is repeated until a point $(x_1,y_1)$ lying inside the unit circle
is found. The output point {\tt (X,Y)} is then the projection of
($x_1,y_1$) from the origin onto the circumference of the
specified circle.
\par
{\tt RAN3VS} proceeds similary, using a cube instead of a square.
\par
{\tt VRAN2S} and {\tt VRAN3S} apply the same method to generate
{\tt N} points at each call.
\Source
These subroutines are based on a similar subroutine {\tt RAN3D} (V130)
written by F. James.
\\ $\bullet$
