% 20 may 1992  mg
\Version {MXDIPR}                            \Routid{F150}
\Keywords{DIRECT MATRIX PRODUCT TENSOR KRONECKER}
\Author{K.S. K\"olbig}                       \Library{MATHLIB}
\Submitter{}                              \Submitted{15.09.1978}
\Language{Fortran}                          %\Revised{}
\Cernhead {Direct or Tensor Matrix Product}
Subroutine subprogram {\tt MXDIPR} computes the direct (sometimes
called tensor, or Kronecker) product $\mathbf{C=A \times B}$ of
two matrices {\bf A} and {\bf B}. Let ${\bf A}=(a_{ik}),
(i=1,2,\ldots,I; k=1,2,\ldots K)$; ${\bf B}=(b_{jl}),(j=1,2,\ldots,J;
l=1,2,\ldots,L)$; then ${\bf C} = (c_{ij;kl})$ with
$c_{ij;kl}=a_{ik}b_{jl}$. {\bf C} has $I \times J$ rows and
$K \times L$ columns. If, in particular, {\bf A} and {\bf B} are square
matrices, {\bf C} is also square.
\Structure
{\tt SUBROUTINE} subprogram \\
User Entry Names: \Rdef{MXDIPR}
\Usage
\begin{verbatim}
    CALL MXDIPR(A,B,C,IAD,JBD,IJD,IA,KA,JB,LB)
\end{verbatim}
\begin{DLtt}{12345678}
\item [A,B] ({\tt REAL}) Matrices {\tt A} and {\tt B}.
\item [C] ({\tt REAL}) On exit, {\tt C} contains the direct product
$\mathbf{A \times B}$.
\item [IAD] ({\tt INTEGER}) First dimension  of {\tt A}.
\item [JBD] ({\tt INTEGER}) First dimension  of {\tt B}.
\item [IJD] ({\tt INTEGER}) First dimension  of {\tt C}.
\item [IA,KA] ({\tt INTEGER}) Number of rows, columns of {\bf A}.
\item [JB,LB] ({\tt INTEGER}) Number of rows, columns of {\bf B}.
\end{DLtt}
\Restrict
{\tt A}, {\tt B}, {\tt C} must not overlap.
\Errorh
If {\tt IA} or {\tt KA} or {\tt JB} or {\tt LB} are equal to zero,
the subprogram acts as do-nothing.
\Examples
\begin{verbatim}
    DIMENSION A(2,2),B(2,2),C(4,4)
    ...
    CALL MXDIPR(A,B,C,2,2,4,2,2,2,2)
\end{verbatim}
assuming
$$\mathbf{A}=\left(\begin{array}{ll}
a_{11} & a_{12} \\ a_{21}& a_{22} \end{array}\right)\quad
\mathbf{B}=\left(\begin{array}{ll}
b_{11}& b_{12}\\  b_{21} & b_{22}\end{array}\right),$$
would set
$$\mathbf{C}=\left( \begin{array}{llll}
a_{11}b_{11}& a_{11}b_{12}& a_{12}b_{11}& a_{12}b_{12}  \\
a_{11}b_{21}& a_{11}b_{22}& a_{12}b_{21}& a_{12}b_{22}  \\
a_{21}b_{11}& a_{21}b_{12}& a_{22}b_{11}& a_{22}b_{12}  \\
a_{21}b_{21}& a_{21}b_{22}& a_{22}b_{21}& a_{22}b_{22}
 \end{array}\right) =
\left( \begin{array}{llll}
c_{11;11}& c_{11;12}& c_{11;21}& c_{11;22}  \\
c_{12;11}& c_{12;12}& c_{12;21}& c_{12;22}  \\
c_{21;11}& c_{21;12}& c_{21;21}& c_{21;22}  \\
c_{22;11}& c_{22;12}& c_{22;21}& c_{22;22}
\end{array}  \right ).$$
\newpage
\Refer
\begin{enumerate}
\item E.P. Wigner, Group Theory, (Academic Press, New York 1959) 17
\item W.I. Smirnow, Lehrgang der h\"oheren Mathematik, Vol. III.1,
(Deutscher Verlag der Wissenschaften, Berlin 1954) 221
\end{enumerate}
$\bullet$
