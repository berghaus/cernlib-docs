%   24 oct 94   ksk
\Version{REXPIN}                    \Routid{C337}
\Keywords{EXPONENTIAL INTEGRAL}
\Author{K.S. K\"olbig}               \Library{MATHLIB}
\Submitter{}                          \Submitted{07.12.1970}
\Language{Fortran}                     \Revised{15.03.1993}
\Cernhead{Exponential Integral}
Function subprograms {\tt REXPIN} and {\tt DEXPIN}
calculate the exponential integral
$$ E_1(x) \ = \ -\mathrm{Ei}(-x) \ = \
\int^{\infty}_x\frac{e^{-t}}{t}dt $$
for real arguments $x$. For $x<0$, the real part of the principal
value of the integral is taken.
\par
On CDC and Cray computers, the double-precision versions
{\tt DEXPIN} and {\tt DEXPIE} are not available.
\Structure
{\tt FUNCTION} subprograms \\
User Entry Names: \Rdef{REXPIN}, \Rdef{REXPIE},
\Rdef{DEXPIN}, \Rdef{DEXPIE}\\
Obsolete User Entry Names: \Rdef{EXPINT} $\equiv$ {\tt REXPIN} \\
Files Referenced: {\tt Unit 6} \\
External References: \Rind{MTLMTR}{N002}, \Rind{ABEND}{Z035}
\Usage
In any arithmetic expression,
\begin{center}
\parbox{.7\textwidth}{
{\tt REXPIN(X)} \quad or \quad {\tt DEXPIN(X)} \quad has the value
\quad $E_1(\mathtt{X})$, \\
{\tt REXPIE(X)} \quad or \quad {\tt DEXPIE(X)} \quad has the value
\quad $e^{\mathtt{X}}\,E_1(\mathtt{X})$,}
\end{center}
where {\tt REXPIN} and {\tt REXPIE} are of type {\tt REAL},
{\tt DEXPIN} and {\tt DEXPIE} are of type
{\tt DOUBLE PRECISION}, and {\tt X} has the same type as the
function name.
\Method
Polynomial and rational approximations.
\Accuracy
{\tt REXPIN} and {\tt REXPIE} (except on CDC and Cray computers)
have full single-precision accuracy.
For most values of the argument {\tt X}, {\tt DEXPIN}, {\tt DEXPIE}
(and {\tt REXPIN}, {\tt REXPIE} on CDC and Cray computers) have an
accuracy of
approximately one significant digit less than the machine precision.
\Errorh
Error {\tt C337.1}: $\mathtt{X=0}$.
The function value is set equal to zero, and a message is written on
{\tt Unit 6}, unless subroutine {\tt MTLSET} (N002) has been called.
\Refer
\begin{enumerate}
\item  W.J. Cody and H.C. Thatcher,Jr., Rational
Chebyshev approximations for the exponential integral $E_1(x)$,
Math. Comp. {\bf 22} (1968) 641--649.
\item  W.J. Cody and H.C. Thatcher,Jr., Chebyshev
approximations for the exponential integral Ei(x), Math. Comp. {\bf 23}
(1969) 289--303.
\end{enumerate}
$\bullet$
