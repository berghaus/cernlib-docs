% 20 may 1992  mg
\Version {UFLINT}                     \Routid{M251}
\Keywords{FLOAT INTEGER REAL NUMBER}
\Author{J. Zoll}                      \Library{KERNLIB}
\Submitter{}                          \Submitted{18.10.1978}
\Language{Fortran}                    \Revised{27.11.1984}
\Cernhead {Assure Integer or Floating Representation of Numbers}
\begin{center}
\fbox{\parbox{120mm}{
\begin{center}
{\bf OBSOLETE}
\end{center}
Please note that this routine has been obsoleted in CNL 204. Users are
advised not to use it any longer and to replace it in older programs.
No maintenance for it will take place and it will eventually disappear.
}}
\end{center}
{\tt UFLINT} scans a vector of numbers, checking every
number whether it is already in the wanted representation. If it is not,
conversion takes place.
\par
Integers and floating point numbers are distinguished by looking for
zero value of the biassed exponent. For machine independent
programs, integers must be $< 2^{22} = 4,194,304$ in absolute value;
also, for the parameter {\tt NW} below, {\tt |NW|} must be $\mathtt{< 32}$
if and only if {\tt NW} is negative.
\par
The wanted representation may be the same for all numbers of
the vector, or it may vary. In the latter case, individual bits of the
parameter {\tt MODE} indicate the wanted representation for each number.
\Structure
{\tt SUBROUTINE} subprogram \\
User Entry Names: \Rdef{UFLINT} \\
External References: \Rind{JBIT}, \Rind{JBYT} (M421)
\Usage
\begin{verbatim}
    CALL UFLINT(VECT,NW,MODE)
\end{verbatim}
\begin{DLtt}{123456}
\item [VECT] Vector of {\tt |NW|} words to be verified and to be
converted if necessary.
\item [NW] $\mathtt{> 0:}$ All words are to have the same representation as
indicated by bit 1 of {\tt MODE}, \\
$\mathtt{< 0:}$ Individual bits of {\tt MODE} indicate the wanted
representation of individual words of {\tt VECT},  \\
$\mathtt{= 0:}$ No action.
\item [MODE] Bit value {\tt 0/1} indicates whether floating/integer
representation wanted. In case of $\mathtt{NW < 0}$, bit {\tt i} of
{\tt MODE} applies to $\mathtt{VECT(i), (i=1,2,\ldots)}$.
\end{DLtt}
\Notes
This cannot be implemented correctly on a VAX.
\par
To convert a whole vector from a representation to the other one,
it is better to use {\tt VFIX} or {\tt VFLOAT} (F121) rather than {\tt
UFLINT}.
\newpage
\Examples
\begin{enumerate}
\item  Suppose a routine {\tt QH} receives 4 parameters, and one wants to
leave the calling program the freedom to pass these parameters either as
integers or as floating point numbers; before usage these numbers have
to be transformed to a definite representation:
\begin{verbatim}
    SUBROUTINE QH (ANX,DX,XLOW,AMAXCT)
    DIMENSION  PAR(4)
    PAR(1) = ANX
    PAR(2) = DX
    PAR(3) = XLOW
    PAR(4) = AMAXCT
\end{verbatim}
to get all floating:
\begin{verbatim}
    CALL UFLINT (PAR,4,0)
\end{verbatim}
to get all integer:
\begin{verbatim}
    CALL UFLINT (PAR,4,1)
\end{verbatim}
to get {\tt PAR(1)} and  {\tt PAR(4)} integer,
{\tt PAR(2)} and  {\tt PAR(3)} floating:
\begin{verbatim}
    CALL UFLINT (PAR,-4,9)
\end{verbatim}
\item  Suppose a set of {\tt N} numbers have been read to a vector
{\tt A} in an arbitrary mixture of integer and floating point format, and
that the whole vector {\tt A} needs to be used as floating point
numbers, although most are in fact integers. To make sure the
representation is right, do:
\begin{verbatim}
    CALL UFLINT (A,N,0)
\end{verbatim}
\end{enumerate}
$\bullet$
