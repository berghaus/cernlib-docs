% 29 Jun 1992 mg
\Version {NORRAN}                               \Routid{V101}
\Keywords{DISTRIBUTION FAST NORMAL RANDOM NUMBER}
\Author{See Method}                     \Library{MATHLIB}
\Submitter{T. Lindel\"of}                 \Submitted{15.06.1976}
\Language{Assembler or Fortran (Cray)}    \Revised{13.07.1988}
\Cernhead {Fast Random Numbers in Normal Distribution}
{\tt NORRAN} generates a pseudo random number with normal (Gaussian)
distribution, zero mean and unit variance. {\tt NORRAN} is much faster
than {\tt RANNOR} (V100).
\Structure
{\tt SUBROUTINE} subprogram  \\
User Entry Names: \Rdef{NORRAN}, \Rdef{NORRIN}, \Rdef{NORRUT}\\
Internal Entry Names: {\tt UNI},{\tt VNI} (IBM only)
\Usage
\begin{verbatim}
    CALL NORRAN(RANDOM)
\end{verbatim}
returns {\tt RANDOM} (type {\tt REAL}) as a pseudo random number in
Gaussian distribution with zero mean and unit variance. The value
returned in any particular call to {\tt NORRAN} is statistically
independent of earlier values. The sequence of values returned in
subsequent calls is, however, functionally dependent on a 'seed'
(two 'seeds' on 32 bits machines) which the user can extract or set as
follows:
\begin{verbatim}
    CALL NORRUT(SEED1,SEED2)
\end{verbatim}
{\tt SEED1} and {\tt SEED2} (on CDC and Cray {\tt SEED1} only)
(type {\tt REAL}) will be replaced by the current values (value) of the
internal 'seeds' ('seed') in {\tt NORRAN}.
\begin{verbatim}
    CALL NORRIN(SEED1,SEED2)
\end{verbatim}
{\tt SEED1} and {\tt SEED2} (on CDC and Cray {\tt SEED1} only)
(type {\tt REAL}) replaces the current values (value) of the internal
'seeds' ('seed') in {\tt NORRAN}.
\Method
CDC: See Ref. 1. IBM: See Ref. 2.
\Notes
{\tt NORRUT} and {\tt NORRIN} are called with two arguments,
the second of which is unused (dummy) in the CDC and Cray versions of
{\tt NORRAN}. {\tt NORRIN} should only be called with arguments set by
a previous call to {\tt NORRUT}. While {\tt NORRAN} may function from
arbitrary seed(s) its random behaviour may be unpredictable unless this
restriction is observed.
\Refer
\begin{enumerate}
\item J.H. Ahrens,  Math. Comp. {\bf 27} (1973) 927
\item G. Marsaglia, K. Ananthanarayanan and N. Paul,
Comm. ACM {\bf 15} (1972) 873.
\end{enumerate}
$\bullet$
