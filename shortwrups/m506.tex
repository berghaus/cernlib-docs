\Version {IUMODE}                             \Routid{M506}
\Keywords{ARGUMENT MODE}
\Author{TC}                                    \Library{KERNLIB}
\Submitter{}                                    \Submitted{30.01.1980}
\Language{Fortran or Assembler}          \Revised{09.02.1989}
\Cernhead {Mode of Argument}
\begin{center}
\fbox{\parbox{120mm}{
\begin{center}
{\bf OBSOLETE}
\end{center}
Please note that this routine has been obsoleted in CNL 204. Users are
advised not to use it any longer and to replace it in older programs.
No maintenance for it will take place and it will eventually disappear.
}}
\end{center}
Tries to detect the mode ({\tt INTEGER} or {\tt REAL}) of the argument.
\Structure
{\tt FUNCTION} subprogram \\
User Entry Names: \Rdef{IUMODE}
\Usage
\begin{verbatim}
    J = IUMODE(WORD)
\end{verbatim}
The routine analyses the bit pattern of {\tt WORD} and makes the best
possible guess in deciding whether it is of type {\tt REAL}, which
results in $\mathtt{J \neq O}$, or of type {\tt INTEGER}, which results in
$\mathtt{J=0}$.
\par
There is no guarantee that the routine will return always the correct
answer and users are strongly discouraged to use it in their
applications. It should be considered a CERN Program Library documented
internal routine.
\\ $\bullet$
