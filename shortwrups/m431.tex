\Version {IFROMC}                       \Routid{M431}
\Keywords{ASCII CHARACTER CONVERT TRANSLATER FORMAT STRING PACKED}
\Author{M. Metcalf}                      \Library{KERNLIB}
\Submitter{}                              \Submitted{15.01.1986}
\Language{Fortran}                        \Revised{16.05.1986}
\Cernhead{Convert Between Character String and Packed ASCII Form}
{\tt IFROMC} and {\tt CFROMI} provide a simple, portable facility for
storing character strings of 1--4 characters packed into integers.
\Structure
{\tt FUNCTION} subprograms \\
User Entry Names: \Rdef{IFROMC}, \Rdef{FROMI}\\
External References:  \Rind{CHTOI}{M400}, \Rind{ITOCH}{M400}
\Usage
\begin{verbatim}
    I=IFROMC('string')
\end{verbatim}
stores in {\tt I} a packed {\tt ASCII} representation of the 4
leftmost characters of {\tt 'string'}. If there are fewer than 4
characters, blanks are stored in the empty positions.
\begin{verbatim}
    CHARACTER*4 STRING
    ...
    STRING=CFROMI(I)
\end{verbatim}
stores in {\tt STRING} the four characters stored packed in {\tt I}
in their {\tt ASCII} representation.
\Refer
\begin{enumerate}
\item CERN Computer Newsletter {\bf 179} (April--May 1985) 11--14.
\end{enumerate}
$\bullet$
