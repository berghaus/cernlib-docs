\Version {TCDUMP}                \Routid{N203}
\Keywords{DUMP MEMORY}
\Author{C. Letertre, J. Zoll}    \Library{KERNLIB}
\Submitter{C. Letertre}          \Submitted{31.01.1972}
\Language{Fortran}                \Revised{15.09.1978}
\Cernhead {Memory Dump}
{\tt TCDUMP} may be used for dumping sections of memory in
octal (CDC) or hexadecimal (IBM), optionally combined with any or all
of the other modes ({\tt INTEGER}, {\tt REAL}, or Hollerith).
\par
The dump shows 5 words per line. The address of the first word
of each line is given 3 times. The absolute address in memory (using
{\tt LOCF}), the relative address within the vector in decimal, and in
octal (CDC) or hexadecimal (IBM).
\par
Continous strings of identical content or strings of
{\it preset indefinites} produce a single line.
\Structure
{\tt SUBROUTINE} subprogram \\
User Entry Names: \Rdef{TCDUMP}\\
Files Referenced: Printer\\
External References: \Rind{UBLOW}{M409}, \Rind{IUCOMP}{V304},
\Rind{IUSAME}{M501}, \Rind{LOCF}{N100}
\Usage
\begin{verbatim}
    CALL TCDUMP(TEXT,VECTOR,N,MODE)
\end{verbatim}
\begin{DLtt}{12345678}
\item [TEXT] 1 word of text printed as heading.
\item [VECTOR] Variable address for start of dump.
\item [N] Number of words for dumping.
\item [MODE] {\tt 1H\ \ } dump in octal, \\
             {\tt 1HI\ }  dump in {\tt INTEGER} and octal, \\
             {\tt 1HF\ }  dump in floating and octal, \\
             {\tt 1HH\ } dump in Hollerith and octal, \\
             {\tt 2HIH} dump in {\tt INTEGER}, Hollerith and octal, \\
             etc...
\end{DLtt}
\Examples
\begin{verbatim}
    COMMON /TOC /A,B(12),D
    CALL TCDUMP(5H/TOC/,A,14,1HF)
\end{verbatim}
dumps the common block {\tt TOC} in octal and floating.
\\ $\bullet$
