% 20 jan 95  ksk
\Version {RANECU}                         \Routid{V114}
\Keywords{DISTRIBUTION NUMBER RANDOM GENERATOR}
\Author{P. l'Ecuyer}             \Library{MATHLIB}
\Submitter{F. Carminati}          \Submitted{27.02.1989}
\Language{Fortran}                   %\Revised{}
\Cernhead {Uniform Random Number Generator}
{\tt RANECU} generates a sequence of uniformly distributed random numbers
in the interval (0,1). The numbers are returned in a vector. Several
independent sequences can be initialized and used in the same run.
\Structure
{\tt SUBROUTINE} Subprograms\\
User Entry Names: \Rdef{RANECU}, \Rdef{RANECQ} \\
{\tt COMMON} Block Names and Lengths: {\tt /RANEC1/ 402}
\Usage
\begin{verbatim}
    CALL RANECU(VEC,LEN,ISEQ)
\end{verbatim}
\begin{DLtt}{123456}
\item [VEC] ({\tt REAL}) Array of length {\tt LEN} at least.
On exit, it will contain the in (0,1) uniformly distributed random
numbers.
\item [LEN] ({\tt INTEGER}) Number of random numbers wanted. Unchanged
on exit.
\item [ISEQ] ({\tt INTEGER}) Number of the independent sequence from
which the {\tt LEN} numbers should be extracted. If $\mathtt{ISEQ \leq 0}$
then the extraction will be made from the sequence used last.
Unchanged on exit.
\end{DLtt}
Several independent sequences can be defined and used. Each sequence
MUST be initialized by the user, otherwise the result is unpredictable.
By default the routine contains a space buffer to handle only one
sequence. If more sequences are needed, then a bigger buffer should be
allocated in the calling program defining the {\tt COMMON} block
{\tt /RANEC1/}  appropriately. {\it Two} words have to be allocated plus
{\it four} words for every sequence initialized.
\par
{\it Two} integer seeds are used to initialize a sequence.
Not all pairs of integers  define a good random sequence
or one which is independent from others. Sections of the same random
sequence can be defined as independent sequences. The period of the
generator is $2^{60} \approx 10^{18}$. A generation has been performed
in order to provide the seeds to start any of the generated sections.
There are 100 possible seed pairs and they are all $10^9$ numbers
apart. Thus a sequence started from one of the seed pairs,
after $ 10^9$ numbers will start generating the next one.
Each of these sequences is of the same order of magnitude as the
basic sequence offered by {\tt RNDM} (V104). Longer sequences
will be generated and the corresponding seeds made available
to users. Note that, while the numbers generated by the default
sequence will always be the same, the introduction of more sequences
may modify some of them. In order to handle the initialization
of the package, the following routine is provided:
\newpage
\begin{verbatim}
    CALL RANECQ(ISEED1,ISEED2,ISEQ,CHOPT)
\end{verbatim}
\begin{DLtt}{12345678}
\item [ISEED1] ({\tt INTEGER}) On entry, it contains the first integer
seed from which to start the sequence. Unchanged on exit.
\item [ISEED2] {\tt (INTEGER)} On entry, it contains the second seed
from which to start the sequence. Unchanged on exit.
\item [ISEQ] ({\tt INTEGER}) On entry, it contains the number of the
independent sequence of random numbers to be addressed by this call.
If $\mathtt{ISEQ \leq 0}$, then the last valid sequence
used will be addressed either for a save or a store. In case the
option {\tt 'R'} is specified, on output the variable will contain the
sequence actually used.
\item [CHOPT] ({\tt CHARACTER*1}) A character specifying the action
which {\tt RANECQ} should take. Possible options are:
\begin{DLtt}{1234}
\item ['\ '] If $\mathtt{1 \leq ISEQ \leq 100}$, the sequence
number {\tt ISEQ} will be initialized with the default seeds of the
pre-computed independent sequence number {\tt ISEQ}.
{\tt ISEED1} and {\tt ISEED2} are ignored.
\item[] If $\mathtt{ISEQ \leq 0}$ or $\mathtt{ISEQ > 100}$, then sequence
number {\tt 1} will be initialized with the default seeds.
{\tt ISEED1} and {\tt ISEED2} are ignored.
\item ['R'] Get the present status of the generator. The two integer
seeds {\tt ISEED1} and {\tt ISEED2} will be returned for sequence
{\tt ISEQ}.
\item ['S'] Set the status of the generator to a previously saved state.
The two integer seeds {\tt ISEED1} and {\tt ISEED2} will be used to
restart the generator for sequence {\tt ISEQ}.
\item ['Q'] Get the pre-generated seeds for {\tt ISEQ}
$(\mathtt{1 \leq ISEQ \leq 100})$. There are 100 pre-generated sequences
each one will generate $10^9$ numbers before reproducing the following
one.
\end{DLtt}
\end{DLtt}
\Timing
Time in $\mu$sec for extractions:
\begin{center}\begin{tabular}{|l|r|r|r|r|}
\cline{1-1}
Extractions \\  \cline{2-5}
per call         &1       &    4     & 16    & 128 \\
\hline
Apollo 10000     &  6.2   &    4.4   & 3.9   & 3.8 \\
Apollo 4000      &  52    &    37    & 34    & 33  \\
IBM 3090E        &  4.9   &    2.9   & 2.5   & 2.4 \\
IBM 3090EVF      &  3.4   &    2.3   & 2.0   & 1.8 \\
Cray X-MP/48     &  4.2   &    2.2   & 1.7   & 1.5 \\
VAX 8650         &19      &    13    & 12    & 11.6 \\
\hline
\end{tabular}\end{center}
\Refer
\begin{enumerate}
\item P.  l'Ecuyer, Efficient and Portable Random Number Generators,
Comm. ACM {\bf 31} (1988) 742.
\item F. James, A Review of Pseudorandom Number Generators,
Computer Phys. Comm. {\bf 60} (1990) 329--344.
\end{enumerate}
$\bullet$
 
