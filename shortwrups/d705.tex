%  16 apr 1996 ksk
\Version{RFSTFT}                                 \Routid{D705}
\Keywords{REAL FAST FOURIER TRANSFORM}
\Author{K.S. K\"olbig, H.-H. Umst\"atter}  \Library{MATHLIB}
\Submitter{}                                    \Submitted{22.04.1996}
\Language{Fortran}                       %\Revised{}
\Cernhead{Real Fast Fourier Transform}
Subroutine {\tt RFSTFT} calculates the finite Fourier transform of a
real periodic sequence $y_0,y_1,\ldots,y_{n-1}$, whose period $n$ must
be a power of two. Either the direct transform
\begin{equation}
C_j \ = \ \frac{1}{n}
\sum_{k=0}^{n-1}y_k \exp \left( \frac{-i 2\pi jk}{n} \right),
\qquad (j=0,1,\ldots,n/2),
\end{equation}
or the inverse transform
\begin{equation}
y_k \ = \ \sum_{j=0}^{n-1} C_j \exp \left( \frac{i 2\pi jk}{n} \right),
\qquad (k=0,1,\ldots,n-1),
\end{equation}
where $y_k$ are real and $C_j$ are complex numbers, may be calculated.
Note that $C_j=\overline{C_{n-j}},\,(j=n/2+1,\ldots,n-1)$, where
$\overline{\alpha}$ denotes the complex conjugate of $\alpha$.
Thus, only the numbers $C_j$ for which $0 \le j \le n/2$ are
calculated.
\Structure
{\tt SUBROUTINE} subprogram \\
User Entry Names: \Rdef{RFSTFT} \\
External References: \Rind{CFSTFT}{D706}
\Usage
\begin{verbatim}
    COMPLEX C(0:..)
    REAL Y(0:..)
    EQUIVALENCE (C,Y)
    ...
    CALL RFSTFT(M,C)
    ...
\end{verbatim}
\begin{DLtt}{1234}
\item[M] ({\tt INTEGER}) On entry, $n$ is determined by the absolute
value of {\tt M} via $n=2^{\mathtt{|M|}}$. \\
$\mathtt{M<0:}$ The direct transform (1) is calculated. \\
$\mathtt{M \ge 0:}$ The inverse transform (2) is calculated. \\
Unchanged on exit.
\item[C] ({\tt COMPLEX}) One dimensional array of dimension {\tt (0:d)},
where $\mathtt{d \ge n/2}$.
\item[Y] ({\tt REAL}) One dimensional array of dimension {\tt (0:d)},
where $\mathtt{d \ge n+1}$. \\
$\mathtt{M<0:}$ \\
On entry, $\mathtt{Y}(k)=y_k,\,(k=0,1,\ldots,n-1)$.\\
On exit, $\mathtt{C}(j)=C_j,\,(j=0,1,\ldots,n/2)$,
as defined by (1). \\
$\mathtt{M \ge 0:}$ \\
On entry, $\mathtt{C}(j)=C_j,\,(j=0,1,\ldots,n/2)$. \\
On exit, $\mathtt{Y}(k)=y_k,\,(k=0,1,\ldots,n-1)$,
as defined by (2).
\end{DLtt}
\newpage
\Method
The subroutine uses {\tt CFSTFT} (D705) with sequences reduced to
half of their length as explaind in Ref. 1.
\Refer
\begin{enumerate}
\item  E.O. Brigham, The fast Fourier transform,
(Prentice-Hall, Englewood Cliffs, 1974) Ch. 10, Sect. 10, Fig. 10-10.
\end{enumerate}
$\bullet$
