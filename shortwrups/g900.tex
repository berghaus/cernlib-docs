\Version {RANF}                      \Routid{G900}
\Keywords{GENERATOR NUMBER RANDOM}
\Author{CDC}             \Library{KERNLIB or Fortran intrinsic}
\Submitter{H. Lipps (not CDC or Cray)} \Submitted{02.06.1980}
\Language{Fortran or Assembler}       \Revised{24.06.1985}
\Cernhead{Random Number Generator}
\begin{center}
\fbox{\parbox{120mm}{\OBSOLETE
Please note that this routine has been obsoleted in CNL 215. Users are
advised not to use it any longer and to replace it in older programs.
No maintenance for it will take place and it will eventually disappear.
\\[3mm]
Suggested replacement: \\
{\tt RANMAR} (V113) or {\tt RANECU} (V114) or {\tt RANLUX} (V115) }}
\end{center}
Function subprograms {\tt RANF} and {\tt DRANF} return pseudo-random
values uniformly distributed in the interval (0,1), the end points
excluded. The multiplicative congruential method is used.
\par
Subroutine subprogram {\tt RANGET} makes the current seed value of
{\tt RANF} and {\tt DRANF} available to the user, and subroutine
{\tt RANSET} restores a seed  value for further use by {\tt RANF} and
{\tt DRANF}.
\par
On CDC computers, the subprograms other than {\tt DRANF}
are part of Control Data's Fortran execution-time library.
 
The non-CDC versions of {\tt RANF} and {\tt DRANF} use the same
multiplier ({\tt 2875 A2E7 B175}), the same initial seed value
({\tt 2BC6 8CFE 166D}), and the same modulus ({\tt 2**48}).
They thus generate,
within the limitations of machine accuracy, the same random
sequence as the CDC versions.
\par
{\tt DRANF} is identical to {\tt RANF} except that it returns a function
value of type {\tt DOUBLE PRECISION}.
\Structure
{\tt SUBROUTINE} and {\tt FUNCTION} subprograms \\
User Entry Names: \Rdef{RANF}, \Rdef{DRANF}, \Rdef{RANGET},
\Rdef{RANSET}
\Usage
In any arithmetic expression,
\begin{center}
{\tt RANF()} \qquad or \qquad {\tt DRANF()}
\end{center}
is set to a value greater than zero and less than one. {\tt RANF}
is of type {\tt REAL}, {\tt DRANF} is of type {\tt DOUBLE PRECISION}.
\begin{verbatim}
    CALL RANGET(SEED)
    CALL RANSET(SEED)
\end{verbatim}
\begin{DLtt}{123456}
\item[SEED] ({\tt REAL} for CDC, {\tt DOUBLE PRECISION} otherwise).
On exit from {\tt RANGET},{\tt SEED} will be set to a value that
determines the current position in the sequence of random numbers.
This value may be used later as an actual argument in a call to
{\tt RANSET} in order to restart the random sequence at this point.
\end{DLtt}
\Refer
\begin{enumerate}
\item Fortran Version 5 Reference Manual (Control Data
Corporation, 1981).
\end{enumerate}
$\bullet$
