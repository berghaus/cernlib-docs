\Version {UBITS}              \Routid{M503}
\Keywords{ARRAY WORD BIT LOCATE}
\Author{M. Metcalf, R. Matthews}           \Library{KERNLIB}
\Submitter{}                           \Submitted{01.02.1982}
\Language{Fortran or Assembler}      \Revised{20.06.1985}
\Cernhead {Locate the One-Bits of a Word or an Array }
{\tt UBITS} locates and counts the {\tt 1}-bits in the right-most
{\tt NBITS} bits in a word or full-word array, returning their
positions. Bit numbering is right to left, bit number 1 being the least
significant bit in the first full word, bit number {\tt NBPW+1} being
the least significant bit in the second full word, where {\tt NBPW}
is the number of bits per machine word.
\Structure
{\tt SUBROUTINE} subprogram\\
User Entry Names: \Rdef{UBITS} \\
External References:  \Rind{UPKBYT}{M422}
(Fortran version only) \\
\Usage
\begin{verbatim}
    CALL UBITS(IWORDS,NBITS,IXV,NX)
\end{verbatim}
\begin{DLtt}{12345678}
\item [IWORDS]  Word or full-word array to be analysed.
\item [NBITS] Bits 1 to {\tt NBITS} of array {\tt IWORDS} are inspected.
\item [IXV] Bit positions of the {\tt 1}-bits in {\tt IWORD} are placed
into {\tt IXV(1)} through {\tt IXV(NX)} in increasing order. {\tt IXV}
must be dimensioned to {\tt NBITS} at least.
\item [NX] Number of {\tt 1}-bits found.
\end{DLtt}
\Examples
\begin{verbatim}
     DIMENSION IXV(9)
     IWORD=1676
C    1676 in base 2 is 11010001100
     CALL UBITS(IWORD,9,IXV,NX)
\end{verbatim}
sets
\begin{verbatim}
   NX = 3, IXV(1) = 3, IXV(2) = 4, IXV(3) = 8.
\end{verbatim}
$\bullet$
