\Version {CLEBSG}                    \Routid{U110}
\Keywords{CLEBSCH GORDAN WIGNER J-SYMBOLS RACAH COEFFICIENTS JAHN}
\Author{J. Soper (Harwell)}                  \Library{MATHLIB}
\Submitter{T. Lindel\"of}                \Submitted{01.03.1976}
\Language{Fortran}                       \Revised{27.11.1984}
\Cernhead {Clebsch-Gordan Coefficients; Wigner 3-j, 6-j, 9-j Symbols;
Racah Coefficients; Jahn U-Function}
{\tt CLEBSG} computes Clebsch-Gordan coefficients, Wigner's
$3-j$, $6-j$ and $9-j$ symbols, Racah coefficients and Jahn's
$U$-function. All these quantities occur in the theory of angular
momentum in quantum mechanics.
\Structure
{\tt FUNCTION} subprogram \\
User Entry Names: \Rdef{CLEBSG}, \Rdef{WIGN3J}, \Rdef{WIGN6J},
\Rdef{WIGN9J}, \Rdef{RACAHC}, \Rdef{JAHNUF} \\
External References: \Rind{NOARG} (Z029) \\
{\tt COMMON} Block Names and Length: {\tt /FGERCM/ 2}
\Usage
In any arithmetic expression, \\[2mm]
\begin{tabular}
{@{\hspace*{10mm}}l@{\qquad \mbox{has the value of} \qquad}l}
{\tt CLEBSG(A,B,C,X,Y,Z)}     & {\tt (ABXY/ABCZ)}, \\
{\tt CLEBSG(A,B,C)}           & {\tt (AB00/ABC0)}, \\
{\tt WIGN3J(A,B,C,X,Y,Z)}     &
$\displaystyle \mathtt{{A B C \choose X Y Z}}$, \\[4mm]
{\tt WIGN3J(A,B,C)}           &
$\displaystyle \mathtt{{A B C \choose 0 0 0}}$, \\[4mm]
{\tt WIGN6J(U,V,W,X,Y,Z)}     &
$\displaystyle \mathtt{{U V W \choose X Y Z}}$, \\[4mm]
{\tt WIGN9J(A1,A2,A3,A4,A5,A6,A7,A8,A9)}  &
$\left( \begin{array}{ccc}
\mathtt{A1} & \mathtt{A2} & \mathtt{A3} \\
\mathtt{A4} & \mathtt{A5} & \mathtt{A6} \\
\mathtt{A7} & \mathtt{A8} & \mathtt{A9}
\end{array} \right),$  \\[4mm]
{\tt RACAHC(A,B,C,D,E,F)}     & {\tt W(ABCD;EF)},\\
{\tt JAHNUF(A,B,C,D,E,F)}     & {\tt U(ABCD;EF)}.
\end{tabular} \\[2mm]
All the above functions and arguments are of type {\tt REAL}.
The type of {\tt JAHNUF} must be declared in the calling program.
\Restrict
The sum of the three angular momenta appearing in any "triangular
condition" must not exceed 100. This limit can be raised, if necessary,
by recompiling with larger dimensions for the arrays {\tt H} and {\tt J}
and a correspondingly larger upper limit on the index of the first
{\tt DO} loop.
\par
The following "geometrical" conditions are tested by the programs
and a correct value of zero is returned in case of violation:
\newpage
\begin{DLtt}{12}
\item[$\bullet$] All triangular conditions satisfied;
\item[$\bullet$] All angular momenta non-negative;
\item[$\bullet$] For {\tt CLEBSG}, {\tt WIGN3J}: $\mathtt{(A+X)}$,
$\mathtt{(B+Y)}$ and $\mathtt{(C+Z)}$ all integral;
\item[$\bullet$] For {\tt CLEBSG}: $\mathtt{(X+Y-Z) = 0.}$
\item[$\bullet$] For {\tt WIGN3J}: $\mathtt{(X+Y+Z) = 0.}$
\item[$\bullet$] Three-argument version of {\tt CLEBSG} and {\tt WIGN3J}:
{\tt A,B,C} all integral, $\mathtt{A+B+C}$ even.
\end{DLtt}
Since a violation of these conditions may be the result of an error in
the calling program which sets up the arguments, an "error
check" is provided. A labelled {\tt COMMON} area is set up:
\begin{verbatim}
    COMMON /FGERCM/ IERR,IERCT
\end{verbatim}
\begin{DLtt}{123456}
\item[IERR] On exit, $\mathtt{IERR=1}$ if any condition was violated,
$\mathtt{=0}$ otherwise. This flag is reset at each call to any of
the above entries.
\item[IERCT] Set to zero at the start of the job and is
increased by 1 whenever a condition is violated in a call to any of the
above entries; thus a single check at the end of the entire job
can ensure that no violations occured during the job.
\end{DLtt}
\Notes
See also {\tt CLEBS} (U100), which calculates Clebsch-Gordan
coefficients in rational form.
\par
All arguments are given as floating-point numbers, whose values must be
whole numbers or half-odd-integers. The nearest exact value will be
assumed, but errors may occur if the routines are called with numbers
which are $\pm 0.05$ or more in error.
\Source
Based on subprograms FG01, FG02 and FG03 of the Harwell Subroutine
Library.
\\ $\bullet$
