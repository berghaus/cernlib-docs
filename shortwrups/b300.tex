%  18.10.94  ksk
\Version{RSRTNT}                          \Routid{B300}
\Keywords{INTEGRATION, RATIONAL, SQUARE ROOT}
\Authors {K.S. K\"olbig}                \Library{MATHLIB}
\Submitter{}                     \Submitted{15.03.1993}
\Language{Fortran}                     %\Revised{}
\Cernhead{An integral of type
\protect\(\mathbf{R(x,\protect\sqrt{a+bx+cx^2})}\protect\)}
Subroutine subprograms {\tt RSRTNT} and {\tt DSRTNT} calculate, based on
indefinite integration, the definite integral
\begin{eqnarray*}
I(k,n;a,b,c;u,v) & = & \displaystyle \int_u^v
\frac{x^k\,dx}{(\sqrt{a+bx+cx^2})^n},
\end{eqnarray*}
for $k=-3,-2,-1,0,1,2,3$ and $n=1,3$, provided that
$a+bx+cx^2 > 0$ for $u < x < v$ and the limits $u,v$ are such that
the integral converges. In particular, the Cauchy principal value
is taken if $k=-1$ and $uv < 0$.
 
On CDC and Cray computers, the double-precision version {\tt DSRTNT}
is not provided.
\Structure
{\tt SUBROUTINE} subprograms \\
User Entry   Names : \Rdef{RSRTNT}, \Rdef{DSRTNT}\\
Files  Referenced : {\tt Unit 6} \\
External References: \Rind{MTLMTR}{N002}, \Rind{ABEND}{Z035}
\Usage
For $\mathtt{t=R}$ (type {\tt REAL}), $\mathtt{t=D}$ (type
{\tt DOUBLE PRECISION}),
\begin{verbatim}
    CALL tSRTNT(K,N,A,B,C,U,V,RES,LRL)
\end{verbatim}
\begin{DLtt}{123456}
\item[K] ({\tt INTEGER}) Power $k$ of $x$.
\item[N] ({\tt INTEGER}) Power $n$ of $\sqrt{a+bx+cx^2}$.
\item[A,B,C] (type according to {\tt t}) Coefficients $a,b,c$.
\item[U,V] (type according to {\tt t}) Limits of integration $u,v$.
\item[RES] (type according to {\tt t}) Contains, on exit, the value
$I$ provided $\mathtt{LRL = .TRUE.}$, the value zero otherwise.
\item[LRL] ({\tt LOGICAL}) Contains, on exit, the value {\tt .TRUE.}
if the integral exists in the sense described above, the value
{\tt .FALSE.} otherwise.
\end{DLtt}
\Restrict
1. \quad $\mathtt{|A|+|B|+|C| \ne 0}.$  \qquad
2. \quad $\mathtt{|K| \le 3}$; \quad
$\mathtt{N = 1}$ or $\mathtt{N = 3}$.
\Errorh
Error {\tt B300.1:} Restriction 1 is not satisfied.
Error {\tt B300.2:} Restriction 2 is not satisfied. \\
In both cases, {\tt RES} is set equal to zero and {\tt LRL} is set
equal to {\tt .FALSE.}, and a message is written on {\tt Unit 6},
unless subroutine {\tt MTLSET} (N002) has been called.
\Refer
\begin{enumerate}
\item  I.S. Gradshteyn and I.M. Ryzhik, Table of integrals, series,
and products, (Academic Press, New York 1980) Sect. 2.26
\end{enumerate}
$\bullet$
