% 18 oct 94  ksk
\Version{RZEROX}                                \Routid{C200}
\Keywords{REAL VARIABLE ZERO}
\Author{K.S. K\"olbig}                         \Library{MATHLIB}
\Submitter{ }                                  \Submitted{01.05.1990}
\Language{Fortran}                             \Revised{01.12.1994}
\Cernhead{Zero of a Function of One Real Variable}
Function subprograms {\tt RZEROX} and {\tt DZEROX} compute, to an
attempted  specified accuracy, a zero $ x_0 $ of a real-valued
function $f(x)$ lying in a given interval $[a,b]$,
where $f(a)*f(b)\le 0 $.
\par
On computers other than  CDC or Cray, only the double precision
version {\tt DZEROX} is  available.
On CDC and Cray computers, only the single-precision version
{\tt RZEROX} is available.
\Structure
{\tt FUNCTION} subprograms \\
User Entry Names: \Rdef{RZEROX}, \Rdef{DZEROX} \\
Obsolete User Entry Names: \Rdef{ZEROX} $\equiv$ {\tt RZEROX} \\
Files Referenced:  {\tt Unit 6} \\
External References: User-supplied {\tt FUNCTION} subprogram
\Usage
For $\mathtt{t=R}$ (type {\tt REAL}), $\mathtt{t=D}$ (type
{\tt DOUBLE PRECISION}),
\begin{verbatim}
    tZEROX(A,B,EPS,MAXF,F,MODE)
\end{verbatim}
has, in any arithmetic expression, the value $x_0$.
\begin{DLtt}{123456}
\item[A,B] (type according to {\tt t}) On entry, {\tt A} and {\tt B}
must specify the end points of the search interval. Unchanged on exit.
\item[EPS] (type according to {\tt t})
On entry, {\tt EPS} must be equal  to the accuracy  parameter
(see {\bf Accuracy}). Unchanged   on  exit.
\item[MAXF] ({\tt INTEGER})
On entry, {\tt MAXF} must be equal to the maximum  permitted
number of references  to the function {\tt F} within the iteration loop.
Unchanged on exit.
\item[F] (type according to {\tt t}) Name of a user-supplied
{\tt FUNCTION} subprogram, declared {\tt EXTERNAL} in the calling
program. This subprogram  must set $\mathtt{F(X)} = f(\mathtt{X})$.
\item[MODE] ({\tt INTEGER})
On entry, $\mathtt{MODE=1}$ or $\mathtt{MODE=2}$ defines the
algorithm for finding $x_0$ (see {\bf Method} and {\bf Notes}).
\end{DLtt}
\Method
Two algorithims are   incorporated in this subprogram. These are
described in Ref. 1 as ``Algorithm  M'' ($\mathtt{MODE=1}$) and
``Algorithm R'' ($\mathtt{MODE=2}$). Both  are mixtures  of linear
interpolation, rational interpolation and bisection.
\Accuracy
These subprograms try to compute two numbers $ x_0 $ and $ x_1$
lying in  the interval $[a,b]$ such that
\begin{enumerate}
\item $ f(x_0)f(x_1)\le 0 $
\item $ |f(x_0)| \le |f(x_1)|$
\item $ |x_0- x_1|\le 2*\mathtt{EPS}*  (1+|x_0|) $
\end{enumerate}
If successful, the value of $x_0 $ is assigned to the function name.
\newpage
\Notes
\begin{enumerate}
\item $\mathtt{MODE=1}$ should  be used for fairly simple functions whose
evaluation is cheap  in comparison with the calculations performed in
one iteration step of {\tt RZEROX}  or {\tt DZEROX}.
\item $\mathtt{MODE=2}$  should be used for more expensive functions.
Convergence should be faster than for $\mathtt{MODE=1}$, but the
evaluation steps are more expensive.
\item For functions which have a pole near the exact zero,
$\mathtt{MODE=1}$ is recommended because of the special character of
the interpolation formula which is used.
\end{enumerate}
\Errorh
\begin{enumerate}
\item $\mathtt{F(A)* F(B)} > 0$.
The function value is set equal to zero.
\item {\tt MODE} has a value other than {\tt 1} or {\tt 2}.
The function value is set equal to zero.
\item The number of references to {\tt F} exceeds {\tt MAXF}.
The function value is set equal to the last computed  value of $x_0$
(see {\bf Accuracy})
\end{enumerate}
For each error a message is printed.
\Source
The subprogram is based on Algol programs described  in Ref. 1.
\Refer
\begin{enumerate}
\item J.C.P. Bus and T.J. Dekker, Two efficient algorithms with
garanteed convergence for finding a zero of a function,
ACM Trans.  Math. Software {\bf 1} (1975) 330--345.
\end{enumerate}
$\bullet$
