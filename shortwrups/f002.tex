\Version{RVADD}                          \Routid{F002}
\Keywords{ELEMENTARY VECTOR OPERATION}
\Authors {H. Lipps}                      \Library{KERNLIB}
\Submitter{}                           \Submitted{18.12.1979}
\Language{Fortran or Assembler or COMPASS} \Revised{27.05.1987}
\Cernhead{Elementary Vector Processing}
These subprograms perform elementary vector operations.
\Structure
{\tt SUBROUTINE} and {\tt FUNCTION} subprograms \\
User Entry Names:
\begin{htmlonly}
\begin{tabular}{llllllll}
\end{htmlonly}
\begin{latexonly}
\begin{tabular}[t]{l*{7}{@{\hspace{4pt}}l}}
\end{latexonly}
\Rdef{RVADD},  & \Rdef{RVCPY},  & \Rdef{RVDIV},  & \Rdef{RVMPA},  &
\Rdef{RVMPY},  & \Rdef{RVMUL},  & \Rdef{RVMULA}, & \Rdef{RVMUNA}, \\
\Rdef{RVRAN},  & \Rdef{RVSCA},  & \Rdef{RVSCL},  & \Rdef{RVSCS},  &
\Rdef{RVSET},  & \Rdef{RVSUB},  & \Rdef{RVSUM},  & \Rdef{RVXCH},  \\
\Rdef{DVADD},  & \Rdef{DVCPY},  & \Rdef{DVDIV},  & \Rdef{DVMPA},  &
\Rdef{DVMPY},  & \Rdef{DVMUL},  & \Rdef{DVMULA}, & \Rdef{DVMUNA}, \\
\Rdef{DVRAN},  & \Rdef{DVSCA},  & \Rdef{DVSCL},  & \Rdef{DVSCS},  &
\Rdef{DVSET},  & \Rdef{DVSUB},  & \Rdef{DVSUM},  & \Rdef{DVXCH},  \\
\Rdef{CVADD},  & \Rdef{CVCPY},  & \Rdef{CVDIV},  & \Rdef{CVMPA},  &
\Rdef{CVMPY},  & \Rdef{CVMUL},  & \Rdef{CVMULA}, & \Rdef{CVMUNA}, \\
\Rdef{CVRAN},  & \Rdef{CVSCA},  & \Rdef{CVSCL},  & \Rdef{CVSCS},  &
\Rdef{CVSET},  & \Rdef{CVSUB},  & \Rdef{CVSUM},  & \Rdef{CVXCH},  \\
\Rdef{CVMPYC}, & \Rdef{CVMPAC}
\end{tabular} \\
External References: \Rind{LOCF}{N100}, \Rind{RANF}{G900}, \Rind{DRANF}{G900}
(some Fortran versions only).
\Usage
For $\mathtt{t=R}$ (type {\tt REAL}), $\mathtt{t=D}$ (type
{\tt DOUBLE PRECISION}), $\mathtt{t=C}$ (type {\tt COMPLEX}):
\begin{center}
\begin{tabular}{l@{\qquad}l}
{\tt CALL tVSET (N,S,Z1,Z2)} & $z_j=s$ \\
{\tt CALL tVRAN (N,A,B,Z1,Z2)} & $z_j=$ random (see {\bf Note} 2)\\
{\tt CALL tVCPY (N,X1,X2,Z1,Z2)} & $z_j=x_j$ \\
{\tt CALL tVXCH (N,X1,X2,Y1,Y2)} & interchanges $x_j$ with $y_j$ \\
{\tt CALL tVADD (N,X1,X2,Y1,Y2,Z1,Z2)} & $z_j=x_j+y_j$ \\
{\tt CALL tVSUB (N,X1,X2,Y1,Y2,Z1,Z2)} & $z_j=x_j-y_j$ \\
{\tt CALL tVMUL (N,X1,X2,Y1,Y2,Z1,Z2)} & $z_j=x_jy_j$ \\
{\tt CALL tVMULA(N,X1,X2,Y1,Y2,Z1,Z2)} & $z_j=x_jy_j+z_j$ \\
{\tt CALL tVMUNA(N,X1,X2,Y1,Y2,Z1,Z2)} & $z_j=-x_jy_j+z_j$ \\
{\tt CALL tVDIV (N,X1,X2,Y1,Y2,Z1,Z2,IFAIL)} & $z_j=x_j/y_j$
(see {\bf Note} 3) \\
{\tt CALL tVSCL (N,S,X1,X2,Z1,Z2)} & $z_j=sx_j$ \\
{\tt CALL tVSCA (N,S,X1,X2,Y1,Y2,Z1,Z2)} & $z_j=sx_j+y_j$ \\
{\tt CALL tVSCS (N,S,X1,X2,Y1,Y2,Z1,Z2)} & $z_j=sx_j-y_j$ \\
{\tt F =  tVSUM (N,X1,X2)} & $f = x_1+ \cdots +x_n$ \\
{\tt F =  tVMPY (N,X1,X2,Y1,Y2)} & $f = x_1y_1+ \cdots +x_ny_n$ \\
{\tt F =  tVMPA (N,X1,X2,Y1,Y2,S)} & $f = x_1y_1+ \cdots +x_ny_n+s$ \\
{\tt F =  CVMPYC(N,X1,X2,Y1,Y2)} &
                     $f = x_1\bar{y}_1+ \cdots +x_n\bar{y}_n$ \\
{\tt F =  CVMPAC(N,X1,X2,Y1,Y2,S)} &
                     $f = x_1\bar{y}_1+ \cdots +x_n\bar{y}_n+s$
\end{tabular}
\end{center}
where $\bar{y}_j$ is the complex conjugate of $y_j$.
\newpage
\begin{DLtt}{12345678}
\item[N] ({\tt INTEGER}) The mathematical dimension of the vectors
$(j = 1,2,\ldots,\mathtt{N})$.
\item[S,A,B] (Type according to {\tt t}) The scalar values $s$, $a$,
and $b$, respectively.
\item[X1,X2] (Type according to {\tt t}) Array elements. They must
contain the elements $x_1,x_2$ of the vector $(x_j)$.
\item[Y1,Y2] (Type according to {\tt t}) Array elements. They must
contain the elements $y_1,y_2$ of the vector $(y_j)$.
\item[Z1,Z2] (Type according to {\tt t}) Array elements. On exit, they
will contain the elements $z_1,z_2$ of the result vector $(z_j)$.
\item[IFAIL] ({\tt INTEGER}) On exit, {\tt IFAIL} is set to zero if all
elements $y_j$ are non-zero. Otherwise {\tt IFAIL} is set to the smallest
index $k$ for which $y_k = 0$.
\end{DLtt}
For $\mathtt{N < 1}$ all subroutines return control without action;
functions {\tt tVSUM}, {\tt tVMPY} and {\tt CVMPYC} assume the value
zero, and {\tt tVMPA} and {\tt CVMPAC} assume the value {\tt S}.
\Restrict
If vector $(z_j)$ overlaps with vector $(x_j)$ or $(y_j)$, results will
be correct provided each element $z_j$ coincides with an element $x_k$
or $y_k$, where $k<j$.
\Accuracy
On computers with IBM 370 architecture, {\tt RVMPY}, {\tt RVMPA},
{\tt CVMPY} and {\tt CVMPA} accumulate the inner product using
double-precision arithmetic internally; the final result is then rounded
to single precision.
\Notes
\begin{enumerate}
\item The vectors $(x_j)$ etc. need not be packed: any equidistant
spacing of their elements is permitted. The subprograms determine the
location of the vector element $x_j$ from the actual arguments
{\tt X1} and {\tt X2}.
\item {\tt tVRAN} sets $z_j$ to a random value of type {\tt t} that is
uniformly distributed in the interval {\tt (A,B)}. For {\tt CVRAN}, the
real and imaginary parts of $z_j$ are distributed uniformly and
independently in {\tt (REAL(A),REAL(B))} and in
{\tt (AIMAG(A),AIMAG(B))}.
\item If $y_k = 0$ and $y_1,\ldots,y_{k-1}$ are non-zero, {\tt tVDIV}
computes only $z_1,\ldots,z_{k-1}$ and sets $\mathtt{IFAIL} = k$.
\item The use of an in-line {\tt DO} loop will be more efficient than
calling the equivalent vector processing subprogram when the vector
length is sufficiently small, due to the overhead of the subprogram call.
\end{enumerate}
$\bullet$
