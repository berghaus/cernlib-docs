\Version {NRAN}                               \Routid{V105}
\Keywords{DISTRIBUTION ARRAY NUMBER RANDOM UNIFORM}
\Author{T. Lindel\"of, F. James}                 \Library{MATHLIB}
\Submitter{}                                   \Submitted{15.06.1976}
\Language{CDC: Compass, IBM: Fortran}         %\Revised{}
\Cernhead {Arrays of Uniform Random Numbers}
\begin{center}
\fbox{\parbox{120mm}{\OBSOLETE
Please note that this routine has been obsoleted in CNL 215. Users are
advised not to use it any longer and to replace it in older programs.
No maintenance for it will take place and it will eventually disappear.
\\[3mm]
Suggested replacement: \\
{\tt RANMAR} (V113) or {\tt RANECU} (V114) or {\tt RANLUX} (V115) }}
\end{center}
{\tt NRAN} on CDC is about 4 times faster than {\tt RNDM} when 'many'
uniformly distributed random numbers are to be generated at once.
\par
{\tt NRAN} on IBM is not recommended. It is merely a Fortran
interface to {\tt RNDM}. Thus this description applies only to
the CDC version.
\Structure
{\tt SUBROUTINE} subprogram \\
User Entry Names: \Rdef{NRAN}, \Rdef{NRANIN}, \Rdef{NRANUT}
\Usage
\begin{verbatim}
    CALL NRAN(VEC,N)
\end{verbatim}
fills the array {\tt VEC} (of length {\tt N} at least) with {\tt N}
independent pseudo random numbers uniformly distributed in the
interval (0,1), the end-points excluded. The other two entries may
be used to retrieve and set the 'seed' as follows:
\begin{verbatim}
    CALL NRANUT(SEED)
\end{verbatim}
returns in {\tt SEED} the current value of a quantitity which is
changed after each call to {\tt NRAN} and upon which the future
random number sequence depends. Its initial default value is \\
$\mathtt{17170000000000000001_8}$.
\begin{verbatim}
    CALL NRANIN(SEED)
\end{verbatim}
presets the above-mentioned quantity to {\tt SEED}.
{\tt SEED} may be any number of the form \\
$\mathtt{ 1717xxxxxxxxxxxxxxxy_8}$ where {\tt y} must be {\tt 1} or {\tt 5}
and the {\tt x}'s any octal digits.
\Method
Multiplicative congruential method with the multiplier
$\mathtt{ 20001170673633457725_8}$. The sequence generated is independent
of that of {\tt RNDM} (V104) so that both may be used together.
\Refer
\begin{enumerate}
\item Computing {\bf 6}, (1970) 121.
\end{enumerate}
$\bullet$
