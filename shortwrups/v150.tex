%  13 mar 1996  ksk
\Version {HISRAN}                               \Routid{V150}
\Keywords{DISTRIBUTION HISTOGRAM NUMBER RANDOM}
\Author{F. James}                                \Library{MATHLIB}
\Submitter{}                                     \Submitted{15.09.1978}
\Language{Fortran}                      %\Revised{}
\Cernhead {Random Numbers According to Any Histogram}
\begin{center}
\fbox{\parbox{120mm}{\OBSOLETE
Please note that this routine has been obsoleted in CNL 223. Users are
advised not to use it any longer and to replace it in older programs.
No maintenance for it will take place and it will eventually disappear.
\\[3mm]
Suggested replacement: {\tt RNHRAN} (V149)}}
\end{center}
{\tt HISRAN} generates random numbers distributed according to
any empirical (one-dimensional) distribution. The distribution is
supplied in the form of a histogram. If the distribution is known
in functional form, {\tt FUNRAN} (V151) should be used instead.
\Structure
{\tt SUBROUTINE} subprograms \\
User Entry Names: \Rdef{HISRAN}, \Rdef{HISPRE}\\
Files Referenced: Printer\\
External References: \Rind{LOCATR}{E106}, \Rind{RNDM}{V104}
\Usage
\begin{tabular}{@{\hspace*{8mm}}l@{\qquad}l}
{\tt CALL HISPRE(Y,NBINS)}               & (once for each histogram) \\
{\tt CALL HISRAN(Y,NBINS,XLO,XWID,XRAN)} & (for each random number)
\end{tabular}
\begin{DLtt}{12345678}
\item [Y] Array of length {\tt NBINS} at least containing the desired
distribution as histogram bin contents on input to {\tt HISPRE}.
\item [NBINS] Number of bins.
\item [XLO] Lower edge of first bin.
\item [XWID] Bin width.
\item [XRAN] A random number returned by {\tt HISRAN}.
\end{DLtt}
\Method
A uniform random number is generated using {\tt RNDM} (V104).
(The user may therefore use {\tt RDMOUT} and {\tt RDMIN} (V104)
to restart a run.) The uniform number is then transformed to the
user's distribution using the cumulative probability distribution
constructed from his histogram. The cumulative distribution is
inverted using a binary search for the nearest bin boundary and a
linear interpolation within the bin. {\tt HISRAN} therefore generates a
constant density within each bin.
\Notes
{\tt HISPRE} changes the values {\tt Y} to form the cumulative
distribution which is needed by {\tt HISRAN}. If {\tt Y} already
contains the cumulative distribution rather than the probability
density, then {\tt HISPRE} should not be called, but in that case
{\tt Y(NBINS)} must be exactly equal to one. Numbers may be drawn from
several different distributions in the same run by calling {\tt HISRAN}
with different arrays {\tt Y1}, {\tt Y2}, etc. and (if desired)
different values of {\tt NBINS},
{\tt XLO}, {\tt XWID} (but always the same values for a given array
{\tt Y}). The routine {\tt  HISPRE} should be used to initialize each
array {\tt Yi}.
\par
The performance of the above method is nearly independent of the shape
of the function or number of bins.
\Errorh
If the the input data to {\tt HISPRE} are not valid (some values
negative or all values zero), an error message is printed, the input
values are printed, and zero is returned instead of a random number.
As many as five such messages may be printed to allow for possible
errors in more than one distribution.
\par
If {\tt HISPRE} is not called, and the input data are not already in
cumulative form, {\tt HISRAN} performs the initialization itself and
prints a warning message. {\tt HISRAN} recognizes that the data are not
in cumulative form if $\mathtt{Y(NBINS) \neq 1}$.
\\ $\bullet$
