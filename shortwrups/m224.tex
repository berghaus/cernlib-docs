\Version {SETFMT}                    \Routid{M224}
\Keywords{EDIT FLOAT FORMAT PRINT OUTPUT VECTOR}
\Author{J. Zoll}                  \Library{KERNLIB}
\Submitter{}                \Submitted{21.08.1971}
\Language{Fortran}              %\Revised{}
\Cernhead {Edit Format for Printing a Floating-Point Vector}
\begin{center}
\fbox{\parbox{120mm}{
\begin{center}
{\bf OBSOLETE}
\end{center}
Please note that this routine has been obsoleted in CNL 204. Users are
advised not to use it any longer and to replace it in older programs.
No maintenance for it will take place and it will eventually disappear.
}}
\end{center}
{\tt SETFMT} optimises a {\tt FORMAT(Fa.b)} for maximum significance and
legibility when using it to print a vector of {\tt REAL} numbers.
\Structure
{\tt SUBROUTINE} subprogram \\
User Entry Names: \Rdef{SETFMT}
\Usage
{\tt SETFMT} presupposes a variable {\tt FORMAT} to be used later for
printing, blown up into {\tt A1} representation, 1 character per word.
It edits 2 parts of this {\tt FORMAT}: The function letter, setting
{\tt F} or {\tt E}, and the number of digits to be printed after
the decimal point, as a function of the data to be printed and the
field width.
\begin{verbatim}
    CALL SETFMT(FMTLET,FMTNUM,NDIG,X,NX)
\end{verbatim}
\begin{DLtt}{12345678}
\item[FMTLET] Contains, on exit, the function letter {\tt F} or {\tt E}.
\item[FMTNUM] Contains, on exit, the number of digits after the
decimal point.
\item[NDIG] ({\tt INTEGER}) Number of {\it numerical} places available
within the field.
\item[X] ({\tt REAL}) Vector of data.
\item[NX] ({\tt INTEGER}) Length of the vector {\tt X}.
\end{DLtt}
\Examples
\begin{verbatim}
    DIMENSION X(10,20),FMT(2),WORK(12)
    DATA FMT/12H(I4,10F10.0)/
    ...
    CALL UBLOW(FMT,WORK,12)
    CALL SETFMT(WORK(7),WORK(11),7,X,200)
    CALL UBUNCH(WORK,FMT,12)
    K=10*(I-1)+1
    PRINT FMT, K, X
    ...
\end{verbatim}
In this example, seven numerical places are declared to be
available. If the biggest absolute element is bigger than {\tt 9999999}
or smaller than {\tt .00001}, the format {\tt E10.2} will be constructed,
except that {\tt F10.0} is constructed for all zeros.
\par
If the biggest element is within range of {\tt F}-printability,
say it is {\tt 124.0000}, the {\tt FORMAT} is provisionally constructed
to accomodate its integer part: {\tt F10.4}. This {\tt 4} may be further
reduced to {\tt 3, 2, 1} or {\tt 0}, if all the fractional parts of the
vector would give {\tt 1, 2, 3} or {\tt 4} trailing zeros.
\\ $\bullet$
