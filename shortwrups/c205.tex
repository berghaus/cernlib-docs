%  05 jan 95 ksk
\Version{RZERO}                    \Routid{C205}
\Keywords{FUNCTION REAL VARIABLE ZERO}
\Author{T. Pomentale}                  \Library{MATHLIB}
\Submitter{K.S. K\"olbig}               \Submitted{20.04.1970}
\Language{Fortran}                     \Revised{15.03.1993}
\Cernhead{Zero of a Function of One Real Variable}
Subroutine subprograms {\tt RZERO} and {\tt DZERO} compute,
to an attempted  specified accuracy, a zero of a real-valued function
$f(x)$ lying in a given interval $[a,b]$, where $f(a)*f(b)\le 0$.
\par
On CDC and Cray computers, the double-precision version {\tt DZERO}
is not available.
\Structure
{\tt SUBROUTINE} subprograms \\
User Entry  Names: \Rdef{RZERO}, \Rdef{DZERO} \\
Files Referenced: {\tt Unit 6} \\
External References: \Rind{MTLMTR}{N002}, \Rind{ABEND}{Z035},
user-supplied {\tt FUNCTION} subprogram
\Usage
For $\mathtt{t=R}$ (type {\tt REAL}), $\mathtt{t=D}$ (type
{\tt DOUBLE PRECISION}),
\begin{verbatim}
    CALL tZERO(A,B,X0,R,EPS,MAXF,F)
\end{verbatim}
\begin{DLtt}{123456}
\item[A,B] (type according to {\tt t}) On entry, {\tt A} and {\tt B}
must specify the end-points of the search interval. Unchanged on exit.
\item[X0] (type according to {\tt t}) On exit, {\tt X0} is the
computed approximation to a zero $x_0$ of the function $f(x)$.
\item[R] (type according to {\tt t}) On exit, the value of {\tt R} is
such that {\tt X0} $-x_0< $ {\tt R}, unless an error condition is
detected (see {\bf Error Handling}).
\item[EPS] (type according to {\tt t}) On entry, {\tt EPS} must be
equal to the accuracy parameter (see {\bf Accuracy}). Unchanged on exit.
\item[MAXF] ({\tt INTEGER}) On entry, {\tt MAXF} must be equal
to the maximum permitted number of references to the
function {\tt F} within the iteration loop. Unchanged on exit.
\item[F] (type according to {\tt t}) Name of a user-supplied
{\tt FUNCTION} subprogram, declared {\tt EXTERNAL} in the calling
program.
\end{DLtt}
The user-supplied function subprogram {\tt F} must be of the form
{\tt FUNCTION  F(X,I)} and must set
$\mathtt{F(X)} = f(\mathtt{X})$. The {\tt INTEGER} argument {\tt I}
is set by {\tt RZERO} before each reference to {\tt F} as follows:
\begin{DLtt}{1234}
\item[] $\mathtt{I=1}$ First reference.
\item[] $\mathtt{I=2}$ Subsequent references.
\item[] $\mathtt{I=3}$ Final reference, before a normal
($\mathtt{R>0}$)
exit.
\end{DLtt}
\Method
A mixed strategy is used, based on the Muller method of
parabolic interpolation supplemented by bisection.
\newpage
\Accuracy
The routine tries to compute a value {\tt X0} such that
\begin{center}
$|\mathtt{X0} - x_0| \le (1 + \mathtt{X0}) * \mathtt{EPS}.$
\end{center}
If this accuracy is obtained with fewer than {\tt MAXF} references
to the function {\tt F} within the iteration loop, the subroutine
exits with {\tt R} positive.
\Errorh
Error {\tt C205.1}: $\mathtt{F(A,1)*F(B,1) > 0}$.
{\tt X0} is set equal to zero and {\tt R}
is set equal to $\mathtt{-2|B-A|}$. \\
Error {\tt C205.2}: The number of calls to {\tt F} exceeds {\tt MAXF}.
{\tt X0} is set equal to zero and {\tt R} is set to
$\mathtt{-|B-A|/2}$. \\
A message is written on {\tt Unit 6}, unless subroutine
{\tt MTLSET} (N002) has been called.
\\ $\bullet$
