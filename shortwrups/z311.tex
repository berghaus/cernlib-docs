% 24 mar 1994  ksk
\Version {CIO}                              \Routid{Z311}
\Keywords{UNIX DISK FILE OPEN CLOSE PUT GET}
\Author{J. Zoll}         \Library{KERNLIB, VAX and UNIX systems only}
\Submitter{}                                 \Submitted{31.10.1991}
\Language{Fortran + C}                       \Revised{01.04.1994}
\Cernhead {Handle Unix Disk Files}
 
The routines of this package are an interface to the
C library functions open, read, write, lseek, close,
to permit a Fortran program to handle an unstructured
Unix file as a string of bytes.
Both sequential and direct-access READ / WRITE can be done.
 
New files are opened with the default permissions 644;
one may set different permissions by calling {\tt CIPERM}
just before calling {\tt CIOPEN}, which resets to the
default after every call.
 
One parameter is common to almost all routines :
{\tt LUNDES} is the file-descriptor of C to identify the file;
with CIOPEN this is an output parameter,
for all other routines it is an input parameter.
 
\Structure
{\tt SUBROUTINE} subprograms \\
User Entry Names:
\Rdef{CIOPEN}, \Rdef{CIGET}, \Rdef{CIGETW},
\Rdef{CIPUT}, \Rdef{CIPUTW},
\Rdef{CISIZE}, \Rdef{CITELL}, \Rdef{CISEEK}, \Rdef{CIREW},
\Rdef{CICLOS}, \Rdef{CIPERM} \\
Files Referenced: Parameter
 
\Usage
Note: the symbol {\tt *} designates output parameters. \\[2mm]
 
{\bf Open a file}
\begin{verbatim}
    CALL CIOPEN(LUNDES, CHMODE, CHNAME, ISTAT)
 
        LUNDES* file-descriptor returned
 
        CHMODE  CHARACTER string selecting the IO mode :
 
                = 'r'   open for reading
                  'r+'  open for read/write
                  'w'   create or truncate for writing
                  'w+'  open for write/read, create or truncate
                  'a'   append
                  'a+'  open for append/read
 
        CHNAME  name of the file, of type CHARACTER
        ISTAT*  status, =zero if success
\end{verbatim}
   For example, create a new file in the current directory :
\begin{verbatim}
        CALL CIOPEN(LUNDES, 'w', 'concert.car', ISTAT)
        IF(ISTAT .NE. 0)          GO TO trouble
\end{verbatim}
\newpage
{\bf Read next string of bytes}
\begin{verbatim}
    CALL CIGET (LUNDES, CHBUF, NBDO, NBDONE, ISTAT)
 
        CHBUF*  text vector to be read into
         NBDO   maximum number of bytes to be read
       NBDONE*  number of bytes actually read
        ISTAT*  status, = zero if success,
                = -1   if end-of-file
\end{verbatim}
 
{\bf Read next string of full words}
\begin{verbatim}
    CALL CIGETW(LUNDES, MBUF, NWDO, NWDONE, ISTAT)
 
         MBUF*  vector to be read into
         NWDO   maximum number of words to be read
       NWDONE*  number of words actually read
        ISTAT*  status, = zero if success,
                = -1   if end-of-file
\end{verbatim}
A full word is normally 4 bytes; on the CRAY it is 8 bytes.
 
   To simulate direct-access reading one has to call {\tt CISEEK} first.
 
   For example:
\begin{verbatim}
    To read the next 2048 bytes:
 
     << starting at byte 8193 :
        CALL CISEEK(LUNDES, 8192, ISTAT)
        IF(ISTAT .NE. 0)              GO TO trouble   >>
 
        CALL CIGET(LUNDES, CHBUF, 2048, NBDONE, ISTAT)
        IF(ISTAT .EQ. -1)         GO TO eof
        IF(ISTAT .NE. 0)          GO TO trouble
\end{verbatim}
 
{\bf Write next string of bytes}
\begin{verbatim}
    CALL CIPUT(LUNDES, CHBUF, NBDO, ISTAT)
 
        CHBUF   text vector to be written, NBDO bytes
        ISTAT*  status, =zero if success
\end{verbatim}
 
{\bf Write next string of full words}
\begin{verbatim}
    CALL CIPUTW(LUNDES, MBUF, NWDO, ISTAT)
 
         MBUF   vector to be written, NWDO words
        ISTAT*  status, =zero if success
\end{verbatim}
 
\newpage
{\bf Get the size of the file}
\begin{verbatim}
    CALL CISIZE(LUNDES, NBYTT, ISTAT)
 
        NBYTT*  number of bytes on the file
        ISTAT*  status, =zero if success
 
    Careful : this will position the file to the end.
\end{verbatim}
 
{\bf Get the current file position}
\begin{verbatim}
    CALL CITELL(LUNDES, NBYTC, ISTAT)
 
        NBYTC*  number of bytes before current
        ISTAT*  status, =zero if success
\end{verbatim}
 
{\bf Set the current file position}
\begin{verbatim}
    CALL CISEEK(LUNDES, NBYTC, ISTAT)
 
        NBYTC   number of bytes before current
        ISTAT*  status, =zero if success
\end{verbatim}
   For example :
\begin{verbatim}
    CALL CISEEK(LUNDES, 0, ISTAT)    position to start-of-file
    CALL CISEEK(LUNDES, 8, ISTAT)    position to 9th byte
    use CISIZE to position to end-of-file
\end{verbatim}
 
{\bf Rewind the file}
\begin{verbatim}
    CALL CIREW(LUNDES)
\end{verbatim}
 
{\bf Close the file}
\begin{verbatim}
    CALL CICLOS(LUNDES)
\end{verbatim}
 
{\bf Set the permissions for the next open}
\begin{verbatim}
    CALL CIPERM(IPERM)
 
        IPERM   the permissions as a decimal integer,
                as returned by STATF (Z265) for example
\end{verbatim}
   For example (using {\tt NCOCTI} of {\tt M432)} :
\begin{verbatim}
    CALL CIPERM(NCOCTI('664'))
 
          set read for everybody, and write for owner and group.
\end{verbatim}
{\bf Note}: formally the buffer for reading and writing should be of
type {\tt CHARACTER} for {\tt CIGET} and {\tt CIPUT},
and of type {\tt INTEGER} for {\tt CIGETW} and {\tt CIPUTW}.
On most machines there is no difference, but on the
VAX this must be observed, because the parameter passing mechanisme
differs crucially for the two cases. Also, on the CRAY there would
be problems if one were using {\tt CIGETW} to read
into a Character address other than a word boundary.
\\ $\bullet$
