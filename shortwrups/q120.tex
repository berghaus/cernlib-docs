%  08 nov 94
\Version {HIGZ}                         \Routid{Q120}
\Keywords{HIGH LEVEL INTERFACE GRAPHIC PAW ZEBRA}
\Author{O. Couet}                      \Library{GRAFLIB}
\Submitter{}                             \Submitted{10.02.1988}
\Language{Fortran and C}                \Revised{01.11.1994}
\Cernhead {High Level Interface to Graphics and Zebra}
The {\tt HIGZ} package is part of {\tt PAW} (Q121) (Physics Analysis
Workstation), but can be used independently.
{\tt HIGZ} contains entries which look and act like many of the entries
of {\tt GKS} (Graphics Kernel System) and, in addition, has
entries providing a higher level of functionality such as plotting
whole histograms. {\tt HIGZ} also contains an option to create a device
independent metafile stored in {\tt ZEBRA} (Q100) format which can
hence be ported, and re-interpreted, on other machines and operating
systems.
\par
The complete {\tt HIGZ} facilities are available in the {\tt PAW}
(Q121) system.
\Structure
{\tt SUBROUTINE} subprograms
\Usage
See {\bf Long Write-up}.
\\ $\bullet$
