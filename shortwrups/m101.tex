% 20 may 1992  mg
\Version {SORTZV}                    \Routid{M101}
\Keywords{ARRAY SORT}
\Author{H. von Eicken}                 \Library{KERNLIB}
\Submitter{}                           \Submitted{14.08.1985}
\Language{CDC: Compass, IBM: Fortran}  %\Revised{}
\Cernhead {Sort One-Dimensional Array}
{\tt SORTZV} will sort a one-dimensional array containing
Hollerith or numerical integer or real information. The
user may specify his own collating sequence for characters;
otherwise that of the display code will be used. The array to be
sorted is not changed. The output of {\tt SORTZV} is an integer array
containing the ordered indices indicating the order of the original
array (see {\bf Examples}).
\Structure
{\tt SUBROUTINE} subprogram \\
User Entry Names: \Rdef{SORTZV}
\Usage
CDC:
\begin{verbatim}
    CALL SORTZV(A,INDEX,N,MODE,NWAY,NSORT,M,CARSET)
\end{verbatim}
Others:
\begin{verbatim}
    CALL SORTZV(A,INDEX,N,MODE,NWAY,NSORT)
\end{verbatim}
\begin{DLtt}{12345678}
\item [A] One-dimensional array of elements to be sorted.
\item [INDEX] One-dimensional array of indices. After execution
it contains the indices denoting the desired order of {\tt A}. On input
it may contain (depending on {\tt NSORT}) indices denoting which
elements of {\tt A} are to be sorted (see {\bf Examples}).
\item [N] Number of words to be sorted.
\item [MODE] Type of sort required: \\
$\mathtt{< 0:}$ Integer,  \\
$\mathtt{= 0:}$ Hollerith, \\
$\mathtt{> 0:}$ Real.
\item [NWAY] Order of sort: \\
$\mathtt{= 0:}$ Ascending order, \\
$\mathtt{\neq 0:}$ Descending order.
\item [NSORT] Elements to be sorted: \\
$\mathtt{= 0:}$ Sort the first {\tt N} elements of {\tt A}, \\
$\mathtt{\neq 0:}$ Sort {\tt N} words of {\tt A} as indicated by array
{\tt INDEX}.
\item [M] Character set to be used: (CDC only) \\
$\mathtt{= 0:}$ Use display code (only applicable to Hollerith sort), \\
$\mathtt{= K:}$ Use collating sequence specified in {\tt CARSET}
$\mathtt{(K \leq 64)}$.
\item [CARSET] Defines the collating sequence for a Hollerith sort.
This array must be at least 64 elements in length. On entering
{\tt SORTZV} the {\tt K} characters for which the user
wishes to specify the order, must be in the first {\tt K} words of
{\tt CARSET} (one character/word, left-adjusted and blank-filled).
Any characters found during the sort which have not been defined in
{\tt CARSET} will be added to {\tt CARSET}.
\end{DLtt}
\Restrict
The input order of equal elements is not necessarily
retained. The parameters {\tt M} and {\tt CARSET} are only used in the
CDC version.
\newpage
\Examples
\begin{enumerate}
\item Assume the array {\tt I} contains {\tt 0,1,-1,4,-2,0,4,5,7,8}.
Then the statement
\begin{verbatim}
    CALL SORTZV(I,INDEX,5,-1,0,0)
\end{verbatim}
({\tt M} and {\tt CARSET} omitted) sets the array {\tt INDEX} to
{\tt 5,3,1,2,4}.
\item  With the same array {\tt I} and the array {\tt INDEX} containing
{\tt 1,3,5,6,7,8},
\begin{verbatim}
    CALL SORTZV(I,INDEX,6,-1,0,1)
\end{verbatim}
sets the array {\tt INDEX} to {\tt 5,3,1,6,7,8}.
\end{enumerate}
For more details, see {\bf Long Write-up}.
\Source
Based on an Algol procedure described in Ref. 1.
\Refer
\begin{enumerate}
\item R.S. Scowen, Algorithm 271 QUICKERSORT, Collected Algorithms
from CACM (1965).
\end{enumerate}
$\bullet$
