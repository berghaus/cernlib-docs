% 30 mai 96  ksk
\Version{RPSIPG}                       \Routid{C316}
\Keywords{PSI DIGAMMA POLYGAMMA GAMMA LOGARITHMIC DERIVATIVE FUNCTION}
\Author{K.S. K\"olbig}                 \Library{MATHLIB}
\Submitter{}                           \Submitted{07.06.1992}
\Language{Fortran}   %              \Revised{ }
\Cernhead{Psi (Digamma) and Polygamma Functions}
Function subprograms {\tt RPSIPG} and {\tt DPSIPG} calculate either
the logarithmic derivative of the gamma function (the psi, or
digamma, function)
$$ \psi(x) \ \equiv \ \psi^{(0)}(x) \ = \ \frac{d\ln \Gamma(x)} {dx}$$
or one of the other polygamma functions
$$ \psi^{(k)}(x) \ = \ \frac{d^k}{dx^k} \, \psi(x) \ = \
\frac{d^{k+1}}{dx^{k+1}} \, \ln \Gamma(x) $$
for real arguments $ x \neq -n,(n=0,1,2,\ldots)$ and
$k = 0,1,2,\ldots,6$.
\par
Note that the Euler constant
$\mathbf{C} = -\psi(1)= 0.57721\,\ldots$\, (also denoted by $\gamma$)
and the Catalan constant
$\mathbf{G}=\frac{1}{8}\big(\psi'(\frac{1}{4})-\pi^2\big) =
0.91596\,\ldots$\, can be calculated by using this subprogram.
\par
On CDC and Cray computers, the double-precision version
{\tt DPSIPG} is not available.
\Structure
{\tt FUNCTION} subprograms\\
User Entry Names: \Rdef{RPSIPG}, \Rdef{DPSIPG} \\
Files Referenced: {\tt Unit 6} \\
External References: \Rind{MTLMTR}{N002}, \Rind{ABEND}{Z035}
\Usage
In any arithmetic expression,
\begin{center}
{\tt RPSIPG(X,K)} \quad or \quad {\tt DPSIPG(X,K)} \quad has the value
\quad $\psi^{(\mathtt{K})}(\mathtt{X})$,
\end{center}
where {\tt RPSIPG} is of type {\tt REAL}, {\tt DPSIPG} is of type
{\tt DOUBLE PRECISION}, and where {\tt X} has the same type as the
function name. {\tt K} is of type {\tt INTEGER}.
\Method
Rational Chebyshev approximation ($k = 0$), approximation by truncated
Chebyshev series ($k > 0$), and functional relations.
\Accuracy
{\tt RPSIPG} (except on CDC and Cray computers)
has full single-precision accuracy.
For most values of the argument {\tt X}, {\tt DPSIPG}
(and {\tt RPSIPG} on CDC and Cray computers) has an accuracy of
approximately one significant digit less than the machine precision.
\Errorh
Error {\tt C316.1:} $\mathtt{K < 0}$ or $\mathtt{K > 6}$. \\
Error {\tt C316.2:} $\mathtt{X} = -n, (n=0,1,2,\ldots)$. \\
In both cases, the function value is set to zero, and a message is
written on {\tt Unit 6}, unless subroutine {\tt MTLSET} (N002) has
been called.
\Refer
\begin{enumerate}
\item W.J. Cody, A.J. Strecock and H.C. Thather, Jr., Chebyshev
approximations for the psi function, Math. Comp. {\bf 27} (1973)
123--127.
\item Y.L. Luke, Mathematical functions and their approximations
(Academic Press, New York, l975) 5--6.
\end{enumerate}
$\bullet$
