\Version {RANGAM}                               \Routid{V109}
\Keywords{CHISQUARE DISTRIBUTION NUMBER RANDOM}
\Author{J.F. Chamayou, F. James}                 \Library{MATHLIB}
\Submitter{}                                     \Submitted{01.07.1979}
\Language{Fortran}                          %\Revised{}
\Cernhead {Random Numbers in Gamma or Chisquare Distribution}
{\tt RANGAM} generates positive random numbers $x$ according to the
gamma distribution with positive parameter $p$, i.e., according to the
density
$$ P(t < x < t+dt) \ = \
\displaystyle \frac{1}{\Gamma (p)}\, t^{p-1}e^{-t}dt. $$
A special case of this distribution is the $\chi ^2$-distribution.
\Structure
{\tt FUNCTION} subprogram   \\
User Entry Names: \Rdef{RANGAM}\\
External References: \Rind {NORRAN} (V101), \Rind{RNDM} (V104),
\Rind{NRAN} (V105)
\Usage
\begin{verbatim}
    Z = RANGAM(P)
\end{verbatim}
sets {\tt Z} to a gamma-distributed random number for any real parameter
$\mathtt{P > 0}$. The value of {\tt P} may vary from call to call without
infuencing the efficiency.
\Method
For integral values of $p \leq 15$, the logarithm of the product of
$p$ uniform random numbers is used. For any value of $p > 15$,
the Wilson-Hilferty approximation (a transformed normal distribution)
is used. For all other $p$, Johnk's algorithm is used.
\Notes
The routine is fast for small integer values of $p$,  and for $p > 15$,
(one Gaussian random number and one square root, plus a few
multiplications). Non-integral values of $p < 15$ are rather slow.
\Examples
\begin{verbatim}
    CHI2 = 2*RANGAM(0.5*ND)
\end{verbatim}
sets {\tt CHI2} to a random number distributed as $\chi ^2$ with
{\tt ND} degrees of freedom.
\\ $\bullet$
