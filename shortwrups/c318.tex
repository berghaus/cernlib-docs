%  05 jan 95   ksk
\Version{RELFUN}                       \Routid{C318}
\Keywords{JACOBIAN ELLIPTIC FUNCTION SN CN DN INVERSE INTEGRAL}
\Author{K.S. K\"olbig, H.-H. Umst\"atter}           \Library{MATHLIB}
\Submitter{}                   \Submitted{30.01.1980}
\Language{Fortran}                      \Revised{01.12.1994}
\Cernhead{Jacobian Elliptic Functions sn, cn, dn}
Function subprograms {\tt RELFUN} and {\tt DELFUN} calculate, for real
argument $x$ and real modulus $k$, the Jacobian elliptic functions
$\mathrm{sn}(x,k)$, $\mathrm{cn}(x,k)$ and $\mathrm{dn}(x,k)$.
The function $\mathrm{sn}(x,k)$ is the inverse of the elliptic
integral of the first kind and is defined implicitly by
$$ x \ = \ \displaystyle \int_0^{\mbox{\small sn(\,{\it x,\,k}\,)}}
\frac{du}{\sqrt{(1-u^2)(1-k^2u^2)}} \qquad (k^2 \le 1). $$
The functions $\mathrm{cn}(x,k)$ and $\mathrm{dn}(x,k)$ are defined by
$$ \mathrm{sn}^2(x,k) + \mathrm{cn}^2(x,k) = 1, \quad
k^2 \mathrm{sn}^2(x,k) + \mathrm{dn}^2(x,k) = 1, \quad
\mathrm{cn}(0,k) = \mathrm{dn}(0,k) = 1. $$
This definition can be extended for $k^2 > 1$ (Ref. 2) by means of
$$ \mathrm{sn}(x,k) = k_1 \mathrm{sn}(kx,k_1), \quad
\mathrm{cn}(x,k) = \mathrm{dn}(kx,k_1), \quad
\mathrm{dn}(x,k) = \mathrm{cn}(kx,k_1), $$
where $k_1 = 1/k$.
For $k = 0$ and $k^2 = 1$ these functions are elementary:
$$ \mathrm{sn}(x,0) = \sin x, \quad
\mathrm{cn}(x,0) = \cos x, \quad \mathrm{dn}(x,0) = 1,$$
$$ \mathrm{sn}(x,\pm 1) = \tanh x, \quad
\mathrm{cn}(x,\pm 1) = \mathrm{dn}(x,\pm 1) = \mathrm{sech}\, x.$$
Note that for $k^2 \ne 1$ the Jacobian elliptic functions are periodic
(with respect to $x$) with period $4\mathrm{K}(k)$ if $k^2 < 1$ and
$ 4k_1\mathrm{K}(k_1)$ if $k^2 > 1$, where $\mathrm{K}(k)$
is the complete elliptic integral of the first kind.
\par
On CDC and Cray computers, the double-precision version {\tt DELFUN}
is not available.
\Structure
{\tt SUBROUTINE} subprograms \\
User Entry Names: \Rdef{RELFUN}, \Rdef{DELFUN} \\
Obsolete User Entry Names: \Rdef{ELFUN} $\equiv$ \Rdef{RELFUN} \\
\Usage
For $\mathtt{t=R}$ (type {\tt REAL}), $\mathtt{t=D}$ (type
{\tt DOUBLE PRECISION}),
\begin{verbatim}
    CALL tELFUN(X,AK2,SN,CN,DN)
\end{verbatim}
\begin{DLtt}{1234}
\item[X] (type according to {\tt t}) The argument $x$.
\item[AK2] (type according to {\tt t})
The value of $k^2$ (the square of the modulus).
\item[SN] (type according to {\tt t}) On exit,
$\mathtt{SN}=\mathrm{sn}(\mathtt{X},k)$.
\item[CN] (type according to {\tt t}) On exit,
$\mathtt{CN}=\mathrm{cn}(\mathtt{X},k)$.
\item[DN] (type according to {\tt t}) On exit,
$\mathtt{DN}=\mathrm{dn}(\mathtt{X},k)$.
\end{DLtt}
\Method
The sequence of the Gaussian arithmetic-geometric mean is used together
with the Gauss transformation and, where appropriate, the Jacobi
imaginary transformation. For values $\mathtt{AK2 > 1}$, the reciprocal
modulus transformation is performed. For details see {\bf References}.
\Accuracy
{\tt RELFUN} (except on CDC and Cray computers)
has full single-precision accuracy.
For most values of the arguments, {\tt DELFUN}
(and {\tt RELFUN} on CDC and Cray computers) has an accuracy of
approximately two significant digits less than the machine precision.
\Restrict
$|x| \le 3 \mathrm{K}(k)$ $(0 < k^2 < 1)$,
$|x| \le 3 k_1\mathrm{K}(k_1)$ $(k^2 > 1)$,
where $\mathrm{K}(x)$ is the complete elliptic integral of the first
kind. (See entries {\tt RELIKC} and {\tt DELIKC} in {\tt RELI1C} (C347)).
\Refer
\begin{enumerate}
\item M. Abramowitz and I.A. Stegun, ed., Handbook
of Mathematical Functions with Formulas, Graphs, and Mathematical
Tables, Sections 16.12 and 17.6, 9th printing with corrections,
(Dover, New York 1972).
\item H.E. Salzer, Quick calculation of Jacobian elliptic functions,
Comm. ACM {\bf 5} (1962) 399.
\item L.V. King, On the dirct numerical calculation of elliptic
functions and integrals, Cambridge Univ. Press (1924) 32.
\item D.J. Hofsommer and R.P. van de Riet, On the numerical calculation
of elliptic integrals of the first and second kind and the elliptic
functions of Jacobi, Numer. Math. {\bf 5} (1963) 291--302.
\item P.F. Byrd and M.D. Friedman, Handbook of elliptic integrals for
engineers and scientists, 2nd ed., Springer-Verlag Berlin (1971).
\end{enumerate}
$\bullet$
