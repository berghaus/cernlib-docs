%  18 Apr 1996  ksk
\Version {DIVON4}                           \Routid{D151}
\Keywords{NUMERICAL MULTIPLE INTEGRATION RANDOM NUMBER GENERATOR}
\Author{J.H. Friedman, M.H. Wright (Stanford)}  \Library{MATHLIB}
\Submitter{F. James}                     \Submitted{01.12.1981}
\Language{Fortran}                       \Revised{14.08.1985}
\Cernhead {Multidimensional Integration or Random Number Generation}
\begin{center}
\fbox{\parbox{120mm}{\OBSOLETE
Please note that this routine has been obsoleted in CNL 223. Users are
advised not to use it any longer and to replace it in older programs.
No maintenance for it will take place and it will eventually disappear.
\\[3mm]
Suggested replacement: (in part) \Rind{RADMUL}{D120} }}
\end{center}
{\tt DIVON4} is designed for integration of scalar functions of several
variables, especially functions not smooth enough to be
integrated reliably using Gaussian quadrature. It can also be
used effectively to generate random points in a multidimensional
space, with point density given by any bounded function. The
heart of the package is an algorithm for recursive
multi-dimensional partitioning of the space into subregions of
approximately constant function value (minimum range criterion).
\Structure
{\tt SUBROUTINE} package \\
User Entry Names:
\begin{htmlonly}
\begin{tabular}{llllllll}
\end{htmlonly}
\begin{latexonly}
\begin{tabular}[t]{l*{7}{@{\hspace{4pt}}l}}
\end{latexonly}
\Rdef{BUKDMP}, & \Rdef{DIVON},  & \Rdef{DVNOPT}, & \Rdef{GENPNT}, &
\Rdef{INTGRL}, & \Rdef{PARTN},  & \Rdef{RANGEN}, & \Rdef{TREDMP}, \\
\Rdef{USRINT}, & \Rdef{USRTRM}, & \Rdef{DVNBKD}, & \Rdef{EXMBUC}, &
\Rdef{SPLIT},  & \Rdef{QUASI},  & \Rdef{RECPAR}, & \Rdef{BOUNDS}, \\
\Rdef{TREAUD}, & \Rdef{NODAUD}, & \Rdef{BUCMVE}, & \Rdef{QUAD},   &
\Rdef{FEQN},   & \Rdef{NOCUT},  & \Rdef{TSTEXT}, & \Rdef{DELSLV}, \\
\Rdef{FUN},    & \Rdef{BUFOPT}, & \Rdef{BNDOPT}, & \Rdef{SETTOL}, &
\Rdef{BNDTST}, & \Rdef{DVCOPY}, & \Rdef{GRDCMP}, & \Rdef{DELETE}, \\
\Rdef{BFGS},   & \Rdef{MODCHL}, & \Rdef{NMDCHL}, & \Rdef{DVDOT},  &
\Rdef{LDLSOL}, & \Rdef{SHRNK},  & \Rdef{FEASMV}, & \Rdef{ADDBND}, \\
\Rdef{MULCHK}, & \Rdef{DELBND}, & \Rdef{LOCSCH}, & \Rdef{ORTHVC}, &
\Rdef{MXSTEP}, & \Rdef{NEWPTR}, & \Rdef{RLEN},   & \Rdef{RANUMS}
\end{tabular} \\
Files Referenced: Printer and optional user-defined external file\\
External References: \Rind {NRAN}{V105}, user-supplied
{\tt FUNCTION} subprogram {\tt DFUN}
\Usage
The function (integrand) is defined by a user-supplied
{\tt FUNCTION} subprogram which must have the name {\tt DFUN}
and must calculate the integrand in double-precision mode:
\begin{verbatim}
   FUNCTION DFUN(ND,X)
   DOUBLE PRECISION DFUN,X(ND)
   ...
   DFUN = ...
   ...
   RETURN
   END
\end{verbatim}
\begin{DLtt}{123456}
\item[ND] ({\tt INTEGER}) Number of integration variables.
\item[X] ({\tt DOUBLE PRECISION}) Array containing the coordinates of the
point in the integration volume at which {\tt DFUN} is to be evaluated.
\end{DLtt}
See {\bf Long Write-up} for details.
\Refer
\begin{enumerate}
\item J.H. Friedman and M.H. Wright, A Nested Partitioning Procedure
for Numerical Multiple Integration. ACM Trans. Math. Software {\bf 7}
(1981) 76--92.
\end{enumerate}
$\bullet$
