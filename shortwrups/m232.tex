% 20 may 1992  mg
\Version {CVTND}                  \Routid{M232}
\Keywords{CONVERT FLOAT FORMAT HOST NORD}
\Author{IBM: R. Matthews, CDC: H. Renshall}
\Library{KERNLIB, IBM and CDC only}
\Submitter{}                                  \Submitted{01.12. 1980}
\Language{Assembler}                         %\Revised{}
\Cernhead {Convert Floating-Point Numbers Between Host Machine and NORD
Formats}
{\tt CVTND} converts floating point numbers between host machine and
NORD formats. Both short (32-bit) and long (48-bit) NORD formats are
catered for. Conversion between long NORD format and long
(i.e., double-precision) host format is only available on the IBM
machines. Conversion between long NORD format and short
(i.e., single-precision) host format is only available on the CDC
machines. Conversion between short (32-bit) NORD format and short
(i.e., single-precision) host format is available on both IBM and CDC.
\Structure
{\tt SUBROUTINE} package \\
User   Entry  Names: \Rdef{LNDLX}, \Rdef{LXLND}, \Rdef{LNDSX},
\Rdef{SXLND}, \Rdef{SXSND}, \Rdef{SNDSX}
\Usage
\begin{verbatim}
    CALL LNDLX(DWORDS,NWORDS)
\end{verbatim}
\begin{DLtt}{12345678}
\item [DWORDS] ({\tt DOUBLE PRECISION}) Array dimensioned to at least
{\tt NWORDS} in the calling program and containing on input in the first
{\tt NWORDS} elements NORD long (48-bit) normalised floating point
numbers, one per element, right-adjusted with zero-fill. After the call
the first {\tt NWORDS} elements will contain converted normalised
{\tt DOUBLE PRECISION} floating point numbers in the
host machine format. This routine is only available on IBM.
\item [NWORDS] ({\tt INTEGER}) Number of NORD numbers to be converted.
Unchanged on output. A value $< 1$ causes a do-nothing return.
\end{DLtt}
\begin{verbatim}
    CALL LXLND(DWORDS,NWORDS)
\end{verbatim}
\begin{DLtt}{12345678}
\item [DWORDS] ({\tt DOUBLE PRECISION}) Array dimensioned to at least
{\tt NWORDS} in the calling program and containing on input in the first
{\tt NWORDS} elements {\tt DOUBLE PRECISION} floating point
numbers in the host machine format. After the call the first {\tt NWORDS}
elements will contain converted NORD long (48-bit) floating point
numbers, one per element of {\tt DWORDS}, right-adjusted with
zero-fill. This routine is only available on IBM.
\item [NWORDS] ({\tt INTEGER}) Number of host numbers to be converted.
Unchanged on output. A value $< 1$ causes a do-nothing return.
\end{DLtt}
\begin{verbatim}
    CALL LNDSX(WORDS,NWORDS)
\end{verbatim}
\begin{DLtt}{12345678}
\item [WORDS] ({\tt SINGLE PRECISION}) Array dimensioned to at least
{\tt NWORDS} in the calling program and containing on input in the first
{\tt NWORDS} elements NORD long (48-bit) normalised floating point
numbers, one per element, right-adjusted with zero-fill. This routine is
only available on CDC (60-bit words). After the call the first
{\tt NWORDS} elements will contain converted normalised single-precision
floating point numbers in the host machine format.
\item [NWORDS] ({\tt INTEGER}) Number of NORD numbers to be converted.
Unchanged on output. A value $< 1$ causes a do-nothing return.
\end{DLtt}
\newpage
\begin{verbatim}
    CALL SXLND(WORDS,NWORDS)
\end{verbatim}
\begin{DLtt}{12345678}
\item [WORDS] ({\tt SINGLE PRECISION}) Array dimensioned to at least
{\tt NWORDS} in the calling program and containing on input in the first
{\tt NWORDS} elements single-precision floating point numbers in the
host machine format. After the call the first {\tt NWORDS} elements will
contain converted NORD long (48-bit) floating point numbers, one
per element of {\tt WORDS}, right-adjusted with zero-fill. This routine
is only available on CDC (60-bit words).
\item [NWORDS]({\tt INTEGER}) Number of host numbers to be converted.
Unchanged on output. A value $< 1$ causes a do-nothing return.
\end{DLtt}
\begin{verbatim}
    CALL SNDSX(WORDS,NWORDS)
\end{verbatim}
\begin{DLtt}{12345678}
\item [WORDS] ({\tt SINGLE PRECISION}) Array dimensioned to at least
{\tt NWORDS} in the calling program and containing on input in the first
{\tt NWORDS} elements NORD short (32-bit) normalised floating point
numbers, one per element, right-adjusted with zero-fill (no fill
necessary on IBM with 32-bit words). After the call the first
{\tt NWORDS} elements will contain converted normalised single-precision
floating point numbers in the host machine format.
\item [NWORDS] ({\tt INTEGER}) Number of NORD numbers to be converted.
Unchanged on output. A value $< 1$ causes a do-nothing return.
\end{DLtt}
\begin{verbatim}
    CALL SXSND(WORDS,NWORDS)
\end{verbatim}
\begin{DLtt}{12345678}
\item [WORDS] ({\tt SINGLE PRECISION}) Array dimensioned to at least
{\tt NWORDS} in the calling program and containing on input in the first
{\tt NWORDS} elements single-precision floating point numbers in the
host machine format. After the call the first {\tt NWORDS} elements
will contain converted NORD short (32-bit) floating point numbers,
one per element of {\tt WORDS}, right-adjusted with zero-fill
(no fill necessary on IBM with 32-bit words).
\item [NWORDS] ({\tt INTEGER}) Number of host numbers to be converted.
Unchanged on output. A value $< 1$ causes a do-nothing return.
\end{DLtt}
\Restrict
{\tt LNDLX} and {\tt LXLND} are available on IBM only. \\
{\tt LNDSX} and {\tt SXLND} are available on CDC only.\\
{\tt SNDSX} and {\tt SXSND} are available on IBM and CDC but require, or
generate, the NORD numbers right-adjusted with zero-fill on CDC.
\Accuracy
Precision in the mantissa will be lost by truncation of the least
significant bits when converting from a source format to a target format
with fewer bits in the mantissa. Note that the mantissa lengths are 23
bits for {\tt NORD} short, 32 bits for NORD long, 24 bits for IBM short,
56 bits for IBM long and 48 bits for CDC short floating point numbers.
\par
Exponent ranges also differ among the machines. The rule
followed on conversion is that when a source machine value is out of
range for the target machine the value set is the limiting value for
the target machine, i.e. the largest or smallest possible floating
point number on that machine. Note that CDC overflow, underflow and
indefinite numbers are converted into the largest, smallest and largest
possible target NORD numbers respectively. The sign of the source
number is preserved during these out-of-range conversions. The
exponent ranges are $10^{-76}$ to $10^{76}$ for NORD short, $10^{-4920}$
to
$10^{4920}$ for NORD long, $10^{-78}$ to $10^{76}$ for IBM short and long
and $10^{-293}$ to $10^{322}$ for CDC short floating point numbers.
Hence all long NORD numbers greater than $10^{76}$ will be set to
$10^{76}$ when converted to IBM {\tt DOUBLE PRECISION} while any CDC
numbers smaller than $10^{-76}$ would be set to $10^{-76}$ when
converted to short NORD format.
\newpage
\Notes
 In the calling sequences above {\tt S} stands for short
representation i.e., 32, 32 and 60 bits on NORD, IBM and CDC
respectively, while {\tt L} stands for long representation, i.e,. 48, 64
and 120 bits on NORD, IBM and CDC respectively. The default lengths
on IBM and CDC are short. The long forms must be explicitly
requested by a {\tt DOUBLE PRECISION} statement. The character {\tt X}
stands for the host machine and the position of {\tt ND} and {\tt X}
implies the direction of processing. Hence {\tt LNDSX} implies convert
long NORD format (48-bits) to short host machine format.
\\ $\bullet$
