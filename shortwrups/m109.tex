% 07 nov 94  ksk
\Version {SORTRQ}                               \Routid{M109}
\Keywords{SORT ROW MATRIX QUICKERSORT}
\Author{T. Lindel\"of}                            \Library{MATHLIB}
\Submitter{F. Carminati}                         \Submitted{15.09.1978}
\Language{Fortran}                        \Revised{09.02.1989}
\Cernhead {Sort Rows of a Matrix}
{\tt SORTRQ} rearranges the row order of a matrix in such a way that the
elements of a selected column are either in increasing or decreasing
order, as desired. Row orders are not necessarily preserved in case
these elements are equal. Otherwise, {\tt SORTRQ} does the same job as
{\tt SORTR} (M107), but {\tt SORTRQ} is sometimes faster.
\Structure
{\tt SUBROUTINE} subprogram  \\
User Entry Names: \Rdef{SORTIQ}, \Rdef{SORTRQ}, \Rdef{SORTDQ}\\
External References: \Rind{USWOP}{V301} (not on all machines)
\Usage
For $\mathtt{t=I}$ (type {\tt INTEGER}),
$\mathtt{t=R}$ (type {\tt REAL}), $\mathtt{t=D}$ (type
{\tt DOUBLE PRECISION}),
\begin{verbatim}
    CALL SORTtQ(MX,NC,NR,NCS)
\end{verbatim}
performs an ordering operation on the matrix {\tt MX} of type {\tt t},
dimensioned
{\tt (NC,NR)}, using the {\tt NCS}-th elements of each row as ordering
criterion.
\par
The matrix {\tt MX} is stored by rows, the first element of a row
following immediatly after the last element of the preceding row.
\par
Obviously, $\mathtt{1 \leq |NCS| \leq NC}$ is a condition. If
this is not met, or if $\mathtt{NR \leq 1}$, {\tt SORTtQ} will do
nothing.
\par
If $\mathtt{NCS > 0}$, {\tt SORTRQ} reorders the rows of {\tt MX} in such a
way that the {\tt NCS}-th element of each row is $\geq$ the {\tt NCS}-th
element of the preceding row. If $\mathtt{NCS < 0}$, the rows of {\tt MX}
are reordered in the strict reverse order to that for $\mathtt{NCS > 0}$.
\Source
Based on an Algol procedure described in Ref. 1.
\Refer
\begin{enumerate}
\item R.S. Scowen, Algorithm 271 QUICKERSORT, Collected Algorithms
from CACM (1965).
\end{enumerate}
$\bullet$
