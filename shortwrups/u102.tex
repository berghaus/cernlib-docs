\Version {LORENF}                          \Routid{U102}
\Keywords{LORENTZ TRANSFORM}
\Author{V. Framery, L. Pape}               \Library{KERNLIB}
\Submitter{}                               \Submitted{01.03.1968}
\Language{Fortran}                   \Revised{16.09.1991}
\Cernhead {Lorentz Transformations}
{\tt LORENF} transforms the momentum 4-vector of a particle from the
Lorentz-frame $\Sigma$ to the frame $\Sigma '$ like {\tt LOREN4} (U101);
it is faster than {\tt LOREN4} because the rest-mass M of $\Sigma '$
is passed as an argument to save the square root.
\par
{\tt LORENB} executes the inverse transformation.
\Structure
{\tt SUBROUTINE} subprograms\\
User Entry Names: \Rdef{LORENF}, \Rdef{LORENB}
\Usage
\begin{tabular}{@{\hspace*{10mm}}lll}
{\tt CALL LORENF(SM,SP,PB,PF)} & forward  transformation &
{\tt PB -> PF} \\
{\tt CALL LORENB(SM,SP,PF,PB)} & backward transformation &
{\tt PF -> PB}
\end{tabular} \\[3mm]
with
\begin{DLtt}{123456}
\item [SM] Rest-mass $M$ of system $\Sigma '$ with $M^2=E^2-P^2$.
\item [SP] Momentum 4-vector $(P,E)$ of $\Sigma '$ in $\Sigma$.
\item [PB] Momentum 4-vector $(p,e)$ in $\Sigma$.
\item [PF] Momentum 4-vector $(p',e')$ in $\Sigma '$.
\end{DLtt}
\Method
For {\tt LORENF} (cf. {\tt LOREN4} (U101)):
\begin{eqnarray*}
e' & = & (eE - pP)/M  \\
p' & = & p - P (e+e')/(E+M)
\end{eqnarray*}
because $pP=eE-e'M$ and $pP-e(E+M)=-M(e+e')$.
 
For {\tt LORENB:}
\begin{eqnarray*}
e & = & (e'E + p'P)/M \\
p & = & p' + P (e+e')/(E+M)
\end{eqnarray*}
\\ $\bullet$
