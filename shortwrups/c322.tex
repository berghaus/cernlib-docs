% 31 oct 94  ksk
\Version{RFRSIN}                                \Routid{C322}
\Keywords{FRESNEL INTEGRAL}
\Author{K.S. K\"olbig}                         \Library{MATHLIB}
\Submitter{}                                   \Submitted{15.05.1987}
\Language{Fortran}                            \Revised{01.12.1994}
\Cernhead{Fresnel Integrals}
Function subprograms
{\tt RFRSIN}, {\tt RFRCOS} and {\tt DFRSIN}, {\tt DFRCOS}
calculate the Fresnel integrals
$$ \begin{array}{l@{\quad = \quad}l@{\qquad}l@{\qquad}l}
S(x) & \displaystyle \int^x_0\frac{\sin t}{\sqrt{t}}dt & (x \ge 0), &
S(-x) \ = \ -S(x), \\[4mm]
C(x) & \displaystyle \int ^x_0\frac{\cos t}{\sqrt{t}}dt & (x \ge 0), &
C(-x) \ = \ -C(x),
\end{array} $$
for real arguments $x$.
\par
On  CDC and  Cray computers, the double-precision versions
{\tt DFRSIN}, {\tt DFRCOS} are not available.
\Structure
{\tt FUNCTION} subprograms \\
User Entry Names: \Rdef{RFRSIN}, \Rdef{RFRCOS}, \Rdef{DFRSIN},
\Rdef{DFRCOS} \\
Obsolete User Entry Names: \Rdef{FRSIN} $\equiv$ \Rdef{RFRSIN},
                           \Rdef{FRCOS} $\equiv$ \Rdef{RFRCOS}
\Usage
In any arithmetic expression,
\begin{center}
{\tt RFRSIN(X)} \quad or \quad {\tt DFRSIN(X)} \quad has the value \quad
$S(\mathtt{X})$,\\
{\tt RFRCOS(X)} \quad or \quad {\tt DFRCOS(X)} \quad has the value \quad
$C(\mathtt{X})$,
\end{center}
where {\tt RFRSIN}, {\tt RFRCOS} are of type {\tt REAL},
{\tt DFRSIN}, {\tt DFRCOS} are of type {\tt DOUBLE PRECISION},
and {\tt X} has the same type as the function name.
\Method
Approximation by truncated Chebyshev series.
\Accuracy
{\tt RFRSIN} and {\tt RFRCOS} (except on CDC and Cray computers)
have full single-precision accuracy.
For most values of the argument {\tt X}, {\tt DFRSIN} and {\tt DFRCOS}
(and {\tt RFRSIN} and {\tt RFRCOS} on CDC and Cray computers) have an
accuracy of approximately one significant digit less than the machine
precision.
\Refer
\begin{enumerate}
\item Y.L. Luke, The special functions and their
approximations, v. II, (Academic Press New York, 1969) 328--329.
\end{enumerate}
$\bullet$
