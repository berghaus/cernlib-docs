\Version {TURTLE}                           \Routid{W151}
\Keywords{BEAM DECAY SIMULATION TRANSPORT}
\Author{D.C. Carey, C. Iselin}             \Library{PGMLIB}
\Submitter{C. Iselin}                      \Submitted{01.07.1974}
\Language{Fortran 4}                           \Revised{27.11.1984}
\Cernhead {Beam Transport Simulation, Including Decay}
{\tt TURTLE} is designed to simulate charged particle beam transport
systems. It allows evaluation of the effects of aberrations in beams
with a  {\it small phase space volume}. These include higher-order
chromatic aberrations, non-linearities in magnetic fields and
higher-order geometric aberrations due to the accumulation of
second-order effects. The beam at any point in the system may be
represented by one- or two-dimensional histograms. {\tt TURTLE} also
provides a simulation of decay of pions or kaons into muons and
neutrinos.
\par
{\tt TURTLE} uses the same input format as {\tt TRSPRT} (W150). An input
stream set up for {\tt TRSPRT} can thus be used for {\tt TURTLE} with
only a few additions.
\Structure
Complete {\tt PROGRAM}\\
User Entry Names: \Rdef{TURTLE}\\
Files Referenced: {\tt INPUT}, {\tt OUTPUT}\\
External References: \Rind{RANF}{G900}, \Rind{UBUNCH}{M409},
\Rind{TIMEL}{Z007}, \Rind{ABEND}{Z035}
\Usage
See {\bf Long Write-up}.
{\tt TURTLE} is accessed from {\tt PGMLIB} as described in
'Execution of Complete Programs, {\tt PGMLIB}' in Chapter 1 of the
Program Library Manual. Page 50 of the {\bf Long write-up} is obsolete.
\Source
FNAL. The parts concerning decay have been written at CERN.
\Refer
\begin{enumerate}
\item K.L. Brown and C. Iselin
DECAY TURTLE, a Computer Program for Simulating Charged Particle Beam
Transport Systems, including Decay Calculations, CERN 74-2 (1974).
\end{enumerate}
A copy of Ref. 1 is available as {\bf Long Write-up}.
\\ $\bullet$
