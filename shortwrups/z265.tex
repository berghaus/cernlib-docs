%02 apr 94 ksk
\Version{CINTF}                         \Routid{Z265}
\Keywords{INTERFACE C FORTRAN LIBRARY}
\Author{see below}                      \Library{KERNLIB}
\Submitter{}                            \Submitted{19.09.1991}
\Language{Fortran + C}                  \Revised{01.04.1994}
\Cernhead {Immediate Interface Routines to the C Library}
{\bf Authors:} F. Carminati, M. Marquina, A. Rademakers,
J. Shiers, J. Zoll. \\[3mm]
The routines of this package are Fortran callable routines
which in turn call their corresponding C Library routines,
after having taken care of the Fortran way of passing parameters.
\par
The names of the interface routines are exactly the names
of the C functions with the letter F added;
the parameters are in one-to-one correspondence with the C functions;
thus {\tt "man <name>"} gives the exact details also for the
interface routine.
\par
Most Fortran systems on Unix machines are clever,
they protect the Fortran user against name-clashes with the C library,
for example a {\tt "CALL RENAME (...)"} compiles as a reference to
\verb'"rename_"' (or to {\tt "RENAME"} on the Cray).
\par
If this is not strictly true, and/or if moreover the Fortran Run-time
library does itself contain an interface routine {\tt "rename"} then
there might be trouble because it is not obvious which {\tt "rename"}
will be linked to the interface routine {\tt RENAMEF}.
The IBM 6000 machine has succeeded in creating this problem,
it has both {\tt "rename"} and \verb'"rename_"' on the
Fortran Run-time library.
In this case one has to give an explicite {\tt -lc}
on the link statement to ensure that the C library is
searched before the Fortran library (but after the Kernlib library).
\Structure
{\tt SUBROUTINE} and {\tt FUNCTION} subprograms \\
User Entry Names:
\begin{htmlonly}
\begin{tabular}{llllllll}
\end{htmlonly}
\begin{latexonly}
\begin{tabular}[t]{l*{7}{@{\hspace{4pt}}l}}
\end{latexonly}
\Rdef{ACCESSF},& \Rdef{CHDIRF}, & \Rdef{CTIMEF}, & \Rdef{EXITF},  &
\Rdef{GETENVF},& \Rdef{GETGIDF},& \Rdef{GETPIDF},& \Rdef{GETUIDF},\\
\Rdef{GETWDF}, & \Rdef{GMTIMEF},& \Rdef{KILLF},  & \Rdef{LSTATF}, &
\Rdef{PERRORF},& \Rdef{READLNF},& \Rdef{RENAMEF},& \Rdef{SETENVF},\\
\Rdef{SLEEPF}, & \Rdef{STATF},  & \Rdef{SYSTEMF},& \Rdef{UNLINKF}
\end{tabular} \\
{\tt COMMON} Block Names and Lengths: {\tt /SLATE/ ISLATE(40)}
\Usage
The types of all variables and functions follow from the Fortran
default typing convention (unless typed explicitly),
except that variables starting with the letters {\tt CH} are
of type {\tt CHARACTER}.
 
The symbol {\tt *} designates an output parameter.
 
For convenience, routines which return a {\tt CHARACTER} string
also return the occupied useful length of this string
in {\tt ISLATE(1)} of {\tt /SLATE/}. \\[2mm]
\newpage
{\bf 'access' --- determine accessibility of file}
\begin{verbatim}
    LOGICAL ACCESSF
    truth = ACCESSF(CHNAME,MODE)
 
      CHNAME  the path-name of the file
        MODE  a bit pattern specifying the type of access:
 
              bit 1 (1): execution permission
                  2 (2): write permission
                  3 (4): read permission
               all zero: existence
\end{verbatim}
 
{\bf 'chdir' --- set current working directory}
\begin{verbatim}
    INTEGER CHDIRF
    ISTAT = CHDIRF(CHNAME)
 
      CHNAME  the path-name of the new working directory
       ISTAT  function value returns zero if successful.
\end{verbatim}
 
{\bf 'ctime' --- convert encoded time to ASCII}
\begin{verbatim}
    CHARACTER CHTIME*24
    CALL CTIMEF(ITIME, CHTIME)
 
       ITIME  encoded time (as returned by STATF)
     CHTIME*  decoded time string of length 24
\end{verbatim}
 
{\bf 'exit' --- terminate the process with a status code}
\begin{verbatim}
    CALL EXITF(IRC)
\end{verbatim}
stops setting return status {\tt IRC}. This should not be used
for normal run termination. On the IBM VM this had to be implemented
with a computed {\tt GOTO}, hence if $\mathtt{IRC>20}$
a {\tt STOP 255} is executed.
\par
On the Unix machines {\tt IRC} will appear in the shell variable
"status"
which is reset after execution of each command, thus for more complicated
logic the value of status has to be saved like (in the C shell):
\begin{verbatim}
      set rc = $status
      if   (rc != 0)  then
        if (rc == 1)  then
          echo '  not quite happy, but continue'
         else
          echo '  stop for trouble'
          exit
         endif
       endif
\end{verbatim}
 
\newpage
{\bf 'getenv' --- get the text of an environment variable}
\begin{verbatim}
    CHARACTER CHTEXT*(big enough)
    CALL GETENVF(CHNAME, CHTEXT)
 
      CHNAME  the name of the environment variable,
     CHTEXT*  returns its value, with blank-fill
              ISLATE(1) occupied length, =0 if not found
\end{verbatim}
 
{\bf 'getgid' --- get group identification}
\begin{verbatim}
    CALL GETGIDF(IDG)
 
      IDG  returns the real group ID of the current process.
\end{verbatim}
 
{\bf 'getpid' --- get process identification}
\begin{verbatim}
    CALL GETPIDF(IDP)
 
      IDP  returns the process ID of the current process.
\end{verbatim}
 
{\bf 'getuid' - get user identification}
\begin{verbatim}
    CALL GETUIDF(IDU)
 
      IDU  returns the real user ID of the current process.
\end{verbatim}
 
{\bf 'getwd' --- get the path-name of the working directory}
\begin{verbatim}
    CHARACTER CHTEXT*(big enough)
    CALL GETWDF(CHTEXT)
 
     CHTEXT*  returns the path-name, with blank-fill
              ISLATE(1) occupied length, =0 if not found
\end{verbatim}
 
{\bf 'gmtime' --- blow encoded time to time elements for
Greenwich Mean Time}
\begin{verbatim}
    INTEGER ITMELS(9)
    CALL GMTIMEF(ITIME, ITMELS)
 
       ITIME  encoded time (as returned by STATF)
     ITMELS*  decoded time elements:
              (1) sec, (2) min, (3) hour, (4) day, (5) month, (6) year,
              (7) weekday, (8) yearday, (9) isdst
\end{verbatim}
 
{\bf 'kill' --- send a signal to a process}
\begin{verbatim}
    ISTAT = KILLF(IPID,ISIG)
 
        IPID  process ID
        ISIG  signal number
       ISTAT  function value returns zero if successful.
\end{verbatim}
 
\newpage
{\bf 'perror' --- print message for the most recent C Library error}
\begin{verbatim}
    CALL PERRORF(CHTEXT)
 
      CHTEXT  the text to be printed before the error message
\end{verbatim}
 
{\bf 'readlink' --- read value of a symbolic link}
\begin{verbatim}
    INTEGER   READLNF
    CHARACTER VAL*(big enough)
 
    NCH = READLNF(CHNAME,VAL)
 
          CHNAME  path-name of the link
      VAL(1:NCH)  returns the value of the link
             NCH  useful length returned,
                  = -1 if trouble, PERRORF may be called.
\end{verbatim}
 
{\bf 'rename' --- rename a file}
\begin{verbatim}
    INTEGER RENAMEF
    ISTAT = RENAMEF(CHFROM,CHTO)
 
      CHFROM  old file name
        CHTO  new file name
       ISTAT  function value returns zero if successful.
\end{verbatim}
 
{\bf 'setenv' - set environment variable}
\begin{verbatim}
    INTEGER SETENVF
 
    ISTAT = SETENVF(CHNAME,CHVAL)
 
       CHNAME  name of the environment variable
        CHVAL  its value to be set
        ISTAT  function value returns zero if succesful.
\end{verbatim}
On machines where the {\tt setenv} function of system BSD is not
available, {\tt putenv} is used instead on a string constructed from
{\tt CHNAME} and {\tt CHVAL} in allocated memory, hence one should
avoid re-defining the same variable very many times. \\
 
{\bf 'sleep' --- suspend execution}
\begin{verbatim}
    CALL SLEEPF(NSECS)
 
       NSECS  number of seconds to wait
\end{verbatim}
 
\newpage
{\bf 'stat' --- get file status}
\begin{verbatim}
    INTEGER INFO(12)
    INTEGER STATF
 
    ISTAT = STATF(CHNAME, INFO)
 
      CHNAME  path-name of the file
       INFO*  information returned
       ISTAT  function value returns zero if successful.
\end{verbatim}
This routine returns the properties of a given file
in a 12-word integer vector:
\begin{verbatim}
   INFO(1) =  dev       device inode resides on
   INFO(2) =  ino       this inode's number
   INFO(3) =  mode      protection
   INFO(4) =  nlink     number or hard links to the file
   INFO(5) =  uid       user-id of owner
   INFO(6) =  gid       group-id of owner
   INFO(7) =  size      total size of file
   INFO(8) =  atime     file last access time
   INFO(9) =  mtime     file last modify time
   INFO(10) = ctime     file last status change time
   INFO(11) = blksize   optimal blocksize for file system i/o ops
   INFO(12) = blocks    actual number of blocks allocated
\end{verbatim}
On machines where {\tt 'blksize'} and {\tt 'blocks'} are not available
(like Silicon Graphics) the words \\
{\tt INFO(11/12)} will always be zero. \\
 
{\bf 'lstat' --- get file status}
 
{\tt LSTATF} is like {\tt STATF} except in the case where the named file
is a symbolic link, in which case {\tt LSTATF} returns information
about the link, while {\tt STATF} returns information about
the file the link references.
 
For convenience {\tt LSTATF} stores into {\tt /SLATE/} some
information about the nature of {\tt CHNAME}:
\begin{verbatim}
      ISLATE(1) = 0  if CHNAME is a regular file
      ISLATE(2) = 0  if CHNAME is a symbolic link
      ISLATE(3) = 0  if CHNAME is a directory
\end{verbatim}
 
{\bf 'system' --- issue a shell command}
\begin{verbatim}
    INTEGER SYSTEMF
    ISTAT = SYSTEMF(CHTEXT)
 
      CHTEXT  the command to be executed
       ISTAT  returns the exit status of the shell
\end{verbatim}
\newpage
{\bf 'unlink' --- remove directory entry}
\begin{verbatim}
    INTEGER UNLINKF
    ISTAT = UNLINKF(CHNAME)
 
      CHNAME  the path-name of the file to be unlinked
       ISTAT  function value returns zero if successful.
\end{verbatim}
Normally this deletes file {\tt CHNAME}. If {\tt CHNAME} is a soft link,
the link is deleted, but not the file pointed to.
\Notes
The routine {\tt SIGNALF}, which belongs to this family,
will be described separately in the next paper \\[1mm]
 
These routines have also been implemented on some machines
which are not running Unix.
The present state is as follows:
 
{\bf VAX} system {\tt VMS} has :
\begin{verbatim}
    CHDIRF, EXITF, GETENVF, GETWDF, RENAMEF, SLEEPF, SYSTEMF
\end{verbatim}
Presently {\tt GETENVF} looks in the symbol table,
except if the name of the environment variable is {\tt "HOME"}
for which it will return the value of the logical name
\verb'SYS$LOGIN'.
 
Some other routines are available through the C run-time library.
 
{\bf IBM 3090} system {\tt VM/CMS} has :
\begin{verbatim}
CHDIRF, CTIMEF, EXITF, GETENVF, GETPIDF, GETWDF, GMTIMEF,
KILLF, PERRORF, RENAMEF, SLEEPF, STATF, SYSTEMF
\end{verbatim}
$\bullet$
