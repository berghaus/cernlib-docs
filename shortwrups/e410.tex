% 20 may 1992  mg
\Version {CPSC}                       \Routid{E410}
\Keywords{COMPLEX POWER SERIES COEFFICIENT}
\Author{B. Fornberg}                 \Library{MATHLIB}
\Submitter{}                         \Submitted{01.07.1974}
\Language{Fortran}             %\Revised{}
\Cernhead{Complex Power Series Coefficients}
{\tt CPSC} calculates a number of leading coefficients $\alpha _\nu$
in the power series
$$ f(\zeta) = \sum_{\nu=0}^\infty a_\nu(\zeta - z)^\nu $$
where the function $f(\zeta)$ is analytic in the neighbourhood of $z$
in the complex $\zeta$-plane.
Since $\alpha_\nu= f^{(\nu)}(z)/\nu !$, the routine can be used to find
numerically derivatives of high order of analytic functions.
\Structure
{\tt SUBROUTINE} subprogram \\
User Entry Names: \Rdef{CPSC}
\Usage
\begin{verbatim}
    CALL CPSC(F,Z,N,RS,ER)
\end{verbatim}
\begin{DLtt}{1234}
\item [F]({\tt COMPLEX}) Name of a  user-supplied function subprogram
calculating $f(\zeta)$, declared {\tt EXTERNAL} in the calling
program.
\item [Z]({\tt COMPLEX}) Point $z$ around which $f(\zeta)$ shall be
expanded.
\item [N] ({\tt INTEGER}) Number of coefficients wanted.
$(\mathtt{1 \leq N \leq 51})$.
\item [RS] ({\tt COMPLEX}) One-dimensional array. On exit, {\tt RS(n)}
contains {\tt N} coefficients corresponding to powers {\tt 0} to
$\mathtt{N}-1$.
\item [ER] ({\tt REAL}) One-dimensional array. On exit, {\tt ER(n)}
contains estimates of the absolute errors for the coefficients in
{\tt RS}.
\end{DLtt}
\Method
The routine makes use of fast Fourier transforms and repeated
Richardson extrapolation. See {\bf Long Write-up} for details.
\Restrict
The routine may fail to
evaluate the coefficients of low degree polynomials of functions
with a large number of leading coefficients equal to zero. For
details, see {\bf Long Write-up}.
\par
In case the radius of convergence is limited by a branch point on
the Riemann surface for a multiple-valued function $f$, which remains
bounded around that point, the obtained accuracy may be low and the error
estimate incorrect. This situation may be avoided be expanding $f+g$
instead of $f$, and subtracting the coefficients of $g$, where the
function $g$ has a pole at the same distance from the centre of
expansion as $f$ has
a branch point. For example $f=\zeta/\log(1+\zeta)$, expanded
around zero, has radius of convergence 1. We can choose
$g=1/(1-\zeta)$ or $g=i/(1-\zeta)$ and subtract 1 from all coefficients,
or, in the second case, look at the real parts only.
\newpage
\Accuracy
The error estimates for the coefficients, given in {\tt ER}, intend
to tell the true size of the error instead of being extremely
safe. Normally, the maximal relative error (for non-zero
coefficients) is of the order $10^{-10}$. If the function $f(\zeta)$ has
no singularities in the finite complex plane, and more than 25
coefficients are asked for, the accuracy of the leading coefficients
may decrease. This is shown, however, in the error estimate vector
{\tt ER}. If, for some reason, the routine should fail, the vector
{\tt RS} is filled with {\tt COMPLEX} zeros and all elements in {\tt ER}
are put to {\tt 1E10}.
\Refer
\begin{enumerate}
\item B. Fornberg, CPSC: Complex power series coefficients,
Report CERN  DD/73/29 (1973).
\end{enumerate}
A copy of Ref. 1 is available as {\bf Long Write-up}.
\\ $\bullet$
