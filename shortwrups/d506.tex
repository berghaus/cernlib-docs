\Version {MINUIT}                       \Routid{D506}
\Keywords{FUNCTION MINIMIZATION ERROR ANALYSIS}
\Keywords{FIT FITTING LIKELIHOOD CHISQUARE}
\Author{F. James, M. Roos}             \Library{PACKLIB}
\Submitter{F. James}                   \Submitted{03.05.1967}
\Language{Fortran}                     \Revised{15.01.1994}
\Cernhead {Function Minimization and Error Analysis}
The {\tt MINUIT} package performs minimization and analysis of the shape
of a multi-parameter function. It is intended to be used on Chisquare
or likelihood functions for fitting data and finding parameter errors
and correlations. The more important options offered are:
\begin{DLtt}{12}
\item[$\bullet$] Variable metric (Fletcher) minimization
\item[$\bullet$] Monte Carlo minimization
\item[$\bullet$] Simplex (Nelder and Mead) minimization
\item[$\bullet$] Parabolic error analysis (error matrix)
\item[$\bullet$] MINOS (non-linear) error analysis
\item[$\bullet$] Contour plotting
\item[$\bullet$] Fixing and restoring parameters
\item[$\bullet$] Global minimization
\end{DLtt}
\Structure
{\tt SUBROUTINE} subprograms \\
User Entry Names:
\begin{htmlonly}
\begin{tabular}{llllllll}
\end{htmlonly}
\begin{latexonly}
\begin{tabular}[t]{l*{7}{@{\hspace{4pt}}l}}
\end{latexonly}
{\tt MINTIO}, & {\tt MINUIT}, & {\tt MNCOMD}, & {\tt MNCONT}, &
{\tt MNERRS}, & {\tt MNEXCM}, & {\tt MNINPU}, & {\tt MNINTR}, \\
{\tt MNPARS}, & {\tt MNREAD}
\end{tabular} \\
Internal Entry Names:
\begin{htmlonly}
\begin{tabular}{llllllll}
\end{htmlonly}
\begin{latexonly}
\begin{tabular}[t]{l*{7}{@{\hspace{4pt}}l}}
\end{latexonly}
{\tt MNAMIN}, & {\tt MNBINS}, & {\tt MNCALF}, & {\tt MNCLER}, &
{\tt MNCNTR}, & {\tt MNCRCK}, & {\tt MNCROS}, & {\tt MNCUVE}, \\
{\tt MNDERI}, & {\tt MNDXDI}, & {\tt MNEIG},  & {\tt MNEMAT}, &
{\tt MNEVAL}, & {\tt MNEXIN}, & {\tt MNFIXP}, & {\tt MNFREE}, \\
{\tt MNGRAD}, & {\tt MNHELP}, & {\tt MNHESS}, & {\tt MNHES1}, &
{\tt MNIMPR}, & {\tt MNINEX}, & {\tt MNINIT}, & {\tt MNLIMS}, \\
{\tt MNLINE}, & {\tt MNMATU}, & {\tt MNMIGR}, & {\tt MNMNOS}, &
{\tt MNMNOT}, & {\tt MNPARM}, & {\tt MNPFIT}, & {\tt MNPINT}, \\
{\tt MNPLOT}, & {\tt MNPOUT}, & {\tt MNPRIN}, & {\tt MNPSDF}, &
{\tt MNRAZZ}, & {\tt MNRN15}, & {\tt MNRSET}, & {\tt MNSAVE}, \\
{\tt MNSCAN}, & {\tt MNSEEK}, & {\tt MNSET},  & {\tt MNSETI}, &
{\tt MNSIMP}, & {\tt MNSTAT}, & {\tt MNSTIN}, & {\tt MNTINY}, \\
{\tt MNUNPT}, & {\tt MNWARN}, & {\tt MNWERR}, & {\tt MNVERT}, &
{\tt STAND}
\end{tabular}
\Usage
{\tt MINUIT} can be used either
\begin{DLtt}{12}
\item[] as a ``master'' batch program which reads and
executes commands appearing in the input data stream;
\end{DLtt}
or
\begin{DLtt}{12}
\item[] as a ``master'' interactive program which reads
and executes commands given from the terminal;
\end{DLtt}
or
\begin{DLtt}{12}
\item[] as a Fortran callable ``slave'' package, called
from the user program or from an intermediate package such as
{\tt PAW} or {\tt HBOOK};
\end{DLtt}
or
\begin{DLtt}{12}
\item[] any combination of the above.
\end{DLtt}
\par
See {\bf Long Write-up} for details.
\\ $\bullet$
