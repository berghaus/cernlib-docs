\Version{CLOGAM}                    \Routid{C333}
\Keywords{LOGARITHM COMPLEX GAMMA FUNCTION}
\Author{K.S. K\"olbig}               \Library{MATHLIB}
\Submitter{}                         \Submitted{10.04.1972}
\Language{Fortran}                    \Revised{15.03.1993}
\Cernhead{Logarithm of the Gamma Function for Complex Argument}
Function subprograms {\tt CLOGAM} and {\tt WLOGAM} calculate
the logarithm of the gamma function
$$ \displaystyle\ln \Gamma(z)$$
for complex $z \neq -n, (n=0,1,2,\ldots)$. The imaginary part
Im $\ln \Gamma(z)$ is calculated in such a way that it is continuous
for $|\arg z|<\pi$, i.e. it is not taken mod ($2\pi$).
\par
The double-precision version {\tt WLOGAM} is available only on computers
which support a {\tt COMPLEX*16} Fortran data type.
\Structure
{\tt FUNCTION} subprograms\\
User Entry Names: \Rdef{CLOGAM}, \Rdef{WLOGAM}\\
Files Referenced: {\tt Unit 6} \\
External References: \Rind{MTLMTR} (N002), \Rind{ABEND} (Z035)
 
\Usage
In any arithmetic expression,
\begin{center}
{\tt CLOGAM(Z)} \quad or \quad {\tt WLOGAM(Z)} \quad has the value \quad
$\ln \Gamma(\mathtt{Z})$,
\end{center}
where {\tt CLOGAM} is of type {\tt COMPLEX}, {\tt WLOGAM} is of type
{\tt COMPLEX*16}, and {\tt Z} has the same type as the function name.
\Method
The method is described in Ref. 1.
\Accuracy
{\tt CLOGAM} (except on CDC and Cray computers)
has full single-precision accuracy.
For most values of the argument {\tt X}, {\tt WLOGAM}
(and {\tt CLOGAM} on CDC and Cray computers) has an accuracy of
approximately two significant digits less than the machine precision.
\Errorh
Error {\tt C333.1}: $\mathtt{Z} = -n,(n = 0,1,2,\cdots).$
The function value is set equal to zero, and a message is written on
{\tt Unit 6}, unless subroutine {\tt MTLSET} (N002) has been called.
\Refer
\begin{enumerate}
\item K.S. K\"olbig, Programs for computing the
logarithm of the gamma function, and the digamma function, for
complex argument, Computer Phys. Comm. {\bf 4} (1972) 221--226.
\end{enumerate}
$\bullet$
