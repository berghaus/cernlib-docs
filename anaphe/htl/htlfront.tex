%%%%%%%%%%%%%%%%%%%%%%%%%%%%%%%%%%%%%%%%%%%%%%%%%%%%%%%%%%%%%%%%%%%
%                                                                 %
%   Histogram Template Library UG Front matter                    %
%                                                                 %
%   Front Material: Title page,                                   %
%                   Copyright Notice                              %
%                   Preliminary Remarks                           %
%                   Table of Contents                             %
%   EPS file      : cern15.eps, cnastit.eps                       %
%                                                                 %
%   Editor: Michel Goossens / IT-ASD                              %
%                                                                 %
%%%%%%%%%%%%%%%%%%%%%%%%%%%%%%%%%%%%%%%%%%%%%%%%%%%%%%%%%%%%%%%%%%%

%%%%%%%%%%%%%%%%%%%%%%%%%%%%%%%%%%%%%%%%%%%%%%%%%%%%%%%%%%%%%%%%%%%%
%    Tile page                                                     %
%%%%%%%%%%%%%%%%%%%%%%%%%%%%%%%%%%%%%%%%%%%%%%%%%%%%%%%%%%%%%%%%%%%%
\def\Ptitle#1{\special{ps: /Printstring (#1) def}
                       \epsfbox{cnastit.eps}}
\begin{titlepage}
\vspace*{-23mm}
\includegraphics[height=30mm]{cern15.eps}%
\hfill
\raisebox{10mm}{\Large\bf CERN Computer Program Library Documentation}
\hfill\mbox{}
\begin{center}
\mbox{}\\[10mm]
\mbox{\Ptitle{HTL}}\\[2cm]
{\Huge Histogram Template Library}\\[1cm]
{\LARGE User Guide}\\[3cm]
{\Large Application Software and Databases Group}\\[5mm]
{\Large Information Technologies Division}\\[2cm]
{\Large CERN Geneva, Switzerland}
\end{center}
\end{titlepage}

%%%%%%%%%%%%%%%%%%%%%%%%%%%%%%%%%%%%%%%%%%%%%%%%%%%%%%%%%%%%%%%%%%%%
%    Copyright  page                                               %
%%%%%%%%%%%%%%%%%%%%%%%%%%%%%%%%%%%%%%%%%%%%%%%%%%%%%%%%%%%%%%%%%%%%
\thispagestyle{empty}
\framebox[\textwidth][t]{\hfill\begin{minipage}{0.96\textwidth}%
\vspace*{3mm}\begin{center}Copyright Notice\end{center}
\parskip\baselineskip
\textbf{Histogram Template Library}
 
CERN Program Library Documentation
 
\copyright{} Copyright CERN, Geneva 1999
 
Copyright and any other appropriate legal protection of these
computer programs and associated documentation reserved in all
countries of the world by their respective copyright holders.

These programs or documentation may not be reproduced by any method
without prior written consent of the Director-General of CERN or his
delegate or from the original copyright holders for the commercial
components.
 
Requests for information should be addressed to:
\vspace*{-.5\baselineskip}
\begin{center}
\tt\begin{tabular}{l}
CERN Program Library Office              \\
CERN-IT Division                         \\
CH-1211 Geneva 23                        \\
Switzerland                              \\
Tel.   +41 22 767 4951                   \\
Fax.   +41 22 767 8630                   \\
Email: cernlib@cern.ch
\end{tabular}
\end{center}
\vspace*{2mm}
\end{minipage}\hfill}%end of minipage in framebox
\vspace{6mm} \textbf{Trademark notice: All trademarks appearing in
  this guide are acknowledged as such.}  
\bigskip 

{\small The source of this document is marked up in \textsc{xml}
  using a \textsc{dtd} copied on a subset of \LaTeX's functionality.
  It is translated with an \textsc{xsl} style sheet and James
  Clark's \texttt{xt} Java program into \LaTeX{} and \textsc{html}.
  The \LaTeX{} source is further typeset using the \Lit{cernman}
  class file developed at CERN and a printable PostScript file is
  generated.}

\section*{Acknowledgements}

HTL has benefited from the suggestions, advice and help of many
individuals.  In particular the major part of HTL was written during
1998 by Savrak Sar, who worked for sixteen months at CERN as a French
\emph{coop\'erant}.

A special mention should be given to Yemi Adesanya and Jacub Moscicki
for the existing HistOOgrams package and the templated prototype; Dirk
D\"ullman and Marcin Nowak for help with Objectivity; Vincenzo
Innocente for his version of templated histograms; Olivier Couet and
Michel Goossens for general support.

\vfill
\begin{flushleft}
\begin{tabular}{@{}l@{\quad}l@{\quad}l}
\emph{Contact Person}:& Dino Ferrero Merlino /IT    & 
                        \texttt{Bernardino.ferrero.merlino@cern.ch}\\
\emph{Documentation}: & Michel Goossens /IT         & 
                        \texttt{michel.goossens@cern.ch}
\end{tabular}\\[5mm]
\emph{Edition -- April 1999} \hfill \footnotesize Printed \today
\end{flushleft}
\newpage

%%%%%%%%%%%%%%%%%%%%%%%%%%%%%%%%%%%%%%%%%%%%%%%%%%%%%%%%%%%%%%%%%%%%
%    Introductory material                                         %
%%%%%%%%%%%%%%%%%%%%%%%%%%%%%%%%%%%%%%%%%%%%%%%%%%%%%%%%%%%%%%%%%%%%
\pagenumbering{roman}
\setcounter{page}{1}

%%%%%%%%%%%%%%%%%%%%%%%%%%%%%%%%%%%%%%%%%%%%%%%%%%%%%%%%%%%%%%%%%%%%
%    Tables of contents ...                                        %
%%%%%%%%%%%%%%%%%%%%%%%%%%%%%%%%%%%%%%%%%%%%%%%%%%%%%%%%%%%%%%%%%%%%
\tableofcontents
%\newpage
%\listoffigures
%\listoftables
\endinput


