 
                FUNCTION MINIMIZATION
 
 
 
                      F. James
 
              CERN, Geneva, Switzerland
 
 
        Reprinted from the Proceedings of the
   1972 CERN Computing and Data Processing School,
Pertisau, Austria, 10-24 September, 1972 (CERN 72-21)
 
 
                            CONTENTS
 
INTRODUCTION
 
1.1  The motivation
 
1.2  Minimization, maximization,
     and optimization
 
1.3  Definition of the problem
1.4  Definition of a minimum
1.5  The shape of the function --
     Taylor's series
 
1.6  Non-existence of optimum in general
1.7  The role of the computer
 
 
ONE-DIENSIONAL MINIMIZATION
 
2.1  Usefulness in n-dimensional problems
2.2  Gri search
2.3  Fiboacci and golden section searches
2.4  Qu.dratic interpolation and extrapolation
2.5  Ttle success-failure method
 
 
STEPPING METHODS IN MANY VARIABLES
 
3.1  Grid searches and random searches
3.2  Single-parameter variation
3.3  Rosenbrock's method
3.4  The simplex method
 
 
GRADIENT METHODS
 
4.1  Calculating derivatives
4.2  Steepest descent
4.3  Newton's method
4.4  Positive-definite quadratic forms
4.5  Conjugate directions
4.6  Conjugate gradients
4.7   Variable metric methods (VMM)
4.8   Davidon's rank-two formula
4.9   The rank-one formula
4.10  Fletcher's unified approach to VMM
 
 
SPECIALIZED TECHNIQUES
 
5.1   Chisquare minimization
5.2   Likelihood maximization
5.3   Models with separable computing
5.4   Sets of related problems
 
 
LOCAL AND GLOBAL MINIMA
 
6.1   The problem of multiple minima
6.2   The Gelfand algorithm
6.3   The Goldstein-Price method
 
 
CONCLUSION
 
7.1   Global strategy
7.2   Computer programs
 
REFERENCES
APPENDIX - SOME SAMPLE PROBLEMS FOR MINIMIZATION ROUTINES
 
FUNCTION MINIMIZATION
 
F. James
CERN, Geneva, Switzerland
 
 
 
 
INTRODUCTION
1.1  The motivation
     A large class of problems in many different fields of research can
be reduced to the problem of finding the smallest value taken on by a
function of one or more variable parameters.  Examples come from fields
as far apart as industrial processing (minimization of production costs
and general relativity (determination of geodesics by minimizing the
path length between two points in curved space-time).  But the classic
example which occurs so often in scientific research is the estimation of
unknown parameters in a theory by minimizing the difference (chi-square)
between theory and experimental data.  In all these examples, the function
to be minimized is of course determined by considerations proper to
the particular field being investigated, which do not concern us here.
Our aim is to study the mathematical problem of minimization.
 
 
1.2  Minimization, maximization, and optimization
 
     Although traditionally one speaks of function minimization, some
authors refer to maximization.  Of course the two are entirely equivalent
since one can be converted to the other by changing the sign of the function.
Thus the problems of minimizing chi-square, maximizing likelihood,
minimizing cost, or maximizing efficiency can all be considered as
minimization (or maximization).  To avoid committing oneself, it is
fashionable to speak of optimization, to cover both cases.  This 
unfortunately causes confusion with optimization in control theory where
the principal techniques are analytical (calculus of variations) and
hence bear little relationship to the numerical methods used in function
minimization as treated here.
 
      To add to the confusion there is the term "programming", which is
also used to mean minimization (usually specified as linear programming,
non-linear programming, or mathematical programming), a historical usage
dating from the time when programmers in the modern sense did not exist,
and computer users were not programming but coding.
 
      Other terms used for minimization are extremization and 
hill-climbing.  Since these can also be used to mean other things, the
general conclusion is that in this field you can not tell a book from
its title.  While waiting for general agreement as to what the subject
should be called, we will stick to function minimization.
 
 
1.3  Definition of the problem
 
      Given a function F(x), the general problem is to find the value of
the variable or variables x for which the function F(x) takes on its
smallest value.  [As pointed out above, this is entirely equivalent to
finding the x for which the function -F (x) takes on its largest value,
but for consistency we will always consider only minimization.]  The
rules of the game are the following:
 
  i) The function F(x) is assumed not to be known analytically, but is
     specified by giving its value at any point x.
 
 ii) The allowed values of the variable or variables x may be restricted
     to a certain range, or to satisfy certain conditions,
     in which case one speaks of constrained
     minimization.  In these lectures we limit ourselves to the un-
     constrained problem.
 
iii) In some cases additional information about the function F may be
     available, such as the numerical values of the 
     first or second derivatives at any point x.  
     Such knowledge cannot in general be assumed, but
     should be used when possible.
 
 iv) The function F(x) is repeatedly evaluated at different points x
     until its minimum value is attained.
 
The method which finds the minimum (within a given tolerance) after the
fewest function evaluations is the best.  Occasionally other considerations
may be important, such as the amount of storage required by the
method  or the amount of computation required to implement the method,
but normally the dominating factor will be the time spent in evaluating
the function.
 
 
1.4  Definition of a minimum
 
     The theorems of elementary calculus tell us that the function
must take on its smallest value at a point where either:
 
  i) all derivatives aF/ax = O (a stationary point), or
 ii) some derivative aF/ax does not exist (a cusp), or
iii) the point x is on the boundary of the allowed region (an edge)
 
     Although we will sometimes find it useful to consider points 
satisfying the above properties, this approach of considering essentially
the analytic properties of the function is clearly not well adapted
the rules of the game as outlined above.  Indeed, when one considers that
there may be any number of stationary points, cusps, and edge points,
all of which may be arbitrarily hard to find by simply sampling the
function value, the whole problem begins to appear hopeless unless some
simplifying assumptions are made.
 
     The usual simplification consists in abandoning the attempt to find
the global minimum and being satisfied with a local minimum.  A local
minimum may be defined as a point xo, where for all points x in some
neighbourhood around xo we have F(x) > F(xo).
 
     Now the situation looks much brighter since the very definition
a local minimum suggests a general strategy for finding one:  we vary x
by small steps in a direction which causes F to decrease, and continue
until F increases in all allowed directions from some point xo.  This
does not yet tell us how to vary x, but at least it suggests that a solution
can be found.
 
 
1.5  The shape of the function -- Taylor's series
 
     With a view to making an intelligent minimizing method, it is of
interest to consider what we might reasonably expect about the behaviour
of F.  If F represents a physically meaningful function, we would certainly
expect all the derivatives of F to exist everywhere in the region of
interest.  Under these conditions we can write down the Taylor series
expansion for F about some point xl, assuming for the moment that x re-
presents just one variable:
 
 
               F(x) = F(xl) + aXl  (x-xl) + 2 |  (x-xl)  +
                       Xl             Xl
 
 
Although we do not know anything a priori about the domain of convergence
of this series, we do know that as the distance (x -xl) becomes smaller,
the higher order terms become less important, so that we would expect
that predictions based on the low-order terms should not be very wrong,
at least for small steps.  Before considering these terms in more detail
let us generalize the variable x to a vector of n variables x.  Then we
have
 
          F(x) = F(Xl) + T(x-xl) + 2 (x -xl)  |(x - xl) + 
 
 
          where the matrix G is defined by Gij = a2F/axiaXi and
          the gradient vector g is gi = aF/axi, all derivatives
          being evaluated at xl.  The T denotes transposition
          which turns a column vector into a row vector.  Note
          the difference between xi (the ith variable) and x
          (the position vector at the point i).
 
Now the first term of the above series is constant, so it will not tell
us much about where to look for a minimum.  The second term is propor-
tional to the gradient  and tells us in which direction the function is
decreasing the fastest, but since it is linear in x, it does not predict
a minimum and therefore does not tell us what step size to take.  More-
over, as we approach the minimum g ) O (if it exists) so we will have to
go further and consider the next term.  The third, or quadratic term
describes a parabolic behaviour and is therefore the lowest term to
predict a minimum.  Unlike g we can expect G to be roughly constant over
small regions, since it would be exactly constant if higher-order terms
were zero.
 
     We mention, in passing, one class of problems in which the above
analysis would not hold at all.  This is in the field known as linear
programming, which limits itself to minimizing functions which are linear
in the parameters, subject to constraints which are also linear.  A
linear function can not have a minimum in the sense described above (a
stationary point) but must take on its minimum at a constraint boundary
(edge point).  For such problems the description of the constraints therefore
takes on greater importance than the analysis of the function
itself, and will not be considered in these lectures.
 
 
1.6  Non-existence of optimum in general
 
     Although we will be studying and comparing
different minimization algorithms (recipes), the reader should be warned
at the outset that in the strict sense of the rules of the game as stated
in Section 1.3 above, we will not be able to show any algorithm to be
superior to any other for all functions.  In principle at least, no
matter how bad one algorithm is, or how good another, we can always find
a function which will be minimized faster by the bad method than by the
good one.  We should keep such essentially theoretical considerations in
mind, but should not be overly discouraged by them.  In particular,
certain objective criteria will emerge for comparing methods even though
the principal criterion -- minimization speed -- depends on the function.
In the past there has in my opinion been an overemphasis on such objective
criteria in an attempt to find the ideal universal minimization
algorithm.  More recently, the tendency is to adapt the algorithm to
the function, even to the point of introducing a super-algorithm which would
choose a sub-algorithm appropriate to the function at hand.  Such questions
of global strategy will be considered later.
 
     The reader should also be warned that in presenting part|cular
algorithms I will often omit details which are unimportant to an
understanding of the algorithm although they may be crucial in actually
making it work.  The original references should therefore be consulted
before programming such alorithms.

 
1.7  The role of the computer
 
     While our subject is essentially a mathematical one, it has been so
profoundly influenced by the existence of high-speed electronic computers
that it would certainly be unfair  not to mention them here.  Indeed,
real progress in the solving of large-scale practical problems has come
only in the last fifteen years, although much of the basic theory dates
back to Newton's time or even earlier.  This is, of course, because of
the renewed interest in numerical minimization techniques for use on
computers.  As it is no longer even thinkable to use these techniques
for hand calculations, it is best to place ourselves immediately in the
computer context and to conceive of our function F(x) rather as a sub-
routine which returns a value of F (and perhaps some other information
such as numerical values of derivatives) for given input values of the
arguments x.
 
     One unpleasant consequence of the computer-oriented approach is th
we will sometimes have to worry about rounding-off errors in the functi
value due to the finite word length of digital computers.  This roundin
error is usually of the order of 10 8 of the function value for compute
with 36-bit words in single precision, but the cumulative effects of
rounding inside a complicated function subroutine may be much larger.
In addition there may be problems of overflow or underflow.
 
 
 
ONE-DIMENSIONAL MINIMIZATION
_
 
2.1  Usefulness in n-dimensional problems
     We will first consider functions of just one variable, since some
general problems can be seen more easily in this simplest case and also
because some n-variable algorithms contain steps which require one-
dimensional minimization.  The one-variable problem is therefore both
instructive and useful even though our prime consideration will be that
of more complex problems.
 
                 Seen Rejected  Percentage
 
Characters       1522        0  100.00
 
 
2.2  Grid search
 
     The most elementary search technique consists in choosing k equ
spaced points within the range of the parameter x, evaluating the fu
tion at each of the points, and retaining the lowest value found.  I
the spacing between points is x, one of the points is sure to be wi
x/2 of the true minimum, although in principle it may not be the pc
corresponding to the lowest value.  Still, if the function does not
too wildly over the distances of the order of x, one generally assu
that this method gives the minimum within a range of about x.
 
     Of course the grid search method has some difficulties.  It is
directly applicable to the usual case where the range of x is infini
But in this case a simple remedy is to choose a reasonable range in
middle of the allowed range, and later to shift the sampling range i
minimum comes out at an end point.
 
     The most serious objection to the grid method is its inefficie
Given the assumption that F does not vary too much over a distance
many of the function evaluations are certainly unnecessary, namely
that are in regions where the function value is known to be large.
other words, the algorithm takes no account of what it has learned ;
the function.  This inefficiency becomes more striking, in fact prol
tive, when extended to many variables.
 
     On the other hand, this method has the prized virtues of extre
simplicity and absolute stability.  It always converges within the
desired tolerance in a known number of steps and is quite insensiti
the detailed behaviour of the function.
 
     The efficiency of the grid method may be greatly improved by
proceeding in several stages, using a smaller range and smaller ste
size in each succeeding stage.  In this way each stage takes accoun
the least value found in the preceding stage, and the method can be
to converge in the usual sense of increasing accuracy due to decrea
step size.  In the next section we consider optimum ways to arrange
staging in order to obtain the fastest decrease in step size.
 
                 Seen Rejected  Percentage
 
Characters       1638        4   99.76
 
 
2 3  Fibonacci and golden section searches
 .
 
     In order to optimize the grid search, we want to minimize the
number of function evaluations per stage, compatible with maintaining
a constant reduction of a factor t in the step sizes at each stage.
This will yield the fastest reduction in step size.  One function evalu
tion tells us nothing about the possible location of a minimum, but as
long as we restrict ourselves to local minima in a given range of x, tw
points are sufficient as shown in the diagram below.  If F(xl) < F(x2),
                                      then there must be at least one
                                      local minimum somewhere in the
                                      range O < x < X2. Now in this
                                      new range, we already have one
O          Xl     X2          l       point (xl), so that a further
                                      reduction in range is possible
                                      with only one new function eval
tion, and the procedure can now be continued with only one new evaluati
per stage.  It remains to be shown that this can be continued indefinit
with a constant reduction in step size, and to calculate what that red
tion will be.  Clearly we would get the maximum reduction on the first
step if xl and x2 were very close together, but we must not forget that
Xl (or X2) will then be used for the next stage and should therefore
close to the middle of this new interval as well.  The situation is
illustrated in the diagram below, where the distances indicated are
imposed by the symmetry of the intervals and the condition that the
reduction in range must be a factor of t in each stage.  The new range
after evaluation of F(x3) will be X3 < x < x2 and its length must be t
 
 
 
 
 
 
 
 
 
 
 
O         Xl     X2        1        0     X3 Xl     X
                                 q                  ,l _|2 _ q
                                             2
 
                 Seen Rejected  Percentage
 
Characters       1258        4   99.68
 
 
This will be possible since there is a real root to the equation:
                                  t2 = 1- t
                                  t  = g2 1  0.616 .
 
 
Since this ratio t is known as the goZden section, the minimization
technique is called a golden section search.  If the number of stages
to be taken is known in advance, it is possible to improve very slig
on this technique by using a Fibonacci seorch, as described for examF
in Kowalik and Osbornel).  Although Fibonacci can be shown to be opti
(in a sense described below), the slight improvement is probably not
worth the added complication.  The golden section search is optimal
among algorithms where the stopping point is not decided in advance.
 
     The above techniques are optimal only in the minimox sense, that
they minimize the maximum number of function evaluations necessary tc
obtain a given accuracy.  It might be called the pessimists optimalit
since in game theory it is the best strategy against an intelligent
opponent who is trying to make you lose.  It should therefore be effe
in minimizing pathological functions, but in more normal cases we sha
expect other methods to be better.  Such methods are described in the
following sections.
 
 
2.4  Quadratic interpolation and extrapolation
 
     A more optimistic approach consists in studying the expected
behaviour of the function and then hoping that the deviations of the
real funct_on from this behaviour are not too great.  From the Taylor
series analysis of Section 1.5, it would be reasonable to proceed by
assuming that the function is nearly quadratic.
 
     Since a parabola is determined by three points, this method requ
the function to have been evaluated for three different values xl, x2
and X3.  It then predicts the minimum to be at the minimum of the
parabola passing through these points.  If the three function values
Fl, F2, and F3, the predicted minimum is at X4, given by
 
                 Seen Rejected  Percentage
 
Characters       1518        3   99.80
 
                            - 10 -
 
 
            (X2 + x3)Fl      +     (Xl + x3)F2      +      (Xl + X2)F3
  x4 =  (xl - x2)(xl - x)      (X2 - xl)(x2 - X3)      (X3 - Xl)(X3- X2)
 
              2 r Fl         +          F2          +          F3
        L(xl - X2)(Xl - X3)     (X2 - xl)(x2 - x3)     (x3 - xl)(x3 - X2)J
 
 
Considerable simplification results when the three points are equally
spaced, a distance d apart, in which case
 
 
                    x4   X2   2 (Fl + F3 - 2F2) -
 
 
The function is then evaluated at X4 this point replaces one of the
first three, and a new point is predicted, again by quadratic interpola-
tion using the new set of three points.  The method terminates when the
predicted function value at some new point agrees with the actual value
within a specified tolerance.
 
     This algorithm usually performs quite well when applied to easy
(nearly quadratic) functions, but suffers from a number of instabilitie
which can be quite serious, as follows:
 
  i) At any step the three points may determine a parabola with a maxim
     rather than a minimum, in which case the method diverges.
 
 ii) If the three points lie nearly in a straight line, the algorithm
     takes an enormous step which may cause numerical difficulties as
     well as diverging.
 
iii) After each step there is a choice of which two of the three previo
     points to retain for the next step.  It is usually more convenient
     and logical to retain the most recent points, but this may also le
     to instabilities by throwing away the best points.
 
 iv) Even without any of the above difficulties, the method may oscilla
     about the minimum instead of converging toward it.
 
     All the problems can be fixed by including checks and safeguards i
the algorithm, but the remedies always involve abandoning, at least tem
rarily, the quadratic interpolation step.  The best remedy is probably
 
                 Seen Rejected  Percentage
 
Characters       1371        2   99.85
 
 
to reserve the method for well-behaved functions and to abandon it
entirely as soon as trouble arises.  It is most often used as the l
step in algorithms which depend principally on other methods, since
physical functions are usually quite parabolic in the immediate vic
of the minimum.
 
     When derivatives of the function are available, variations of 
ratic interpolation are possible, using instead of three points to t
termine the parabola, either two function values and one first deri
or the function value and the first two derivatives at one point.
variations tend to be even more unstable than the basic method, sin
they use information from fewer points.
 
 
2.5  The success-failure method
 
     A good compromise between the stability of the grid search and
rapid convergence of quadratic interpolation is found with the succ
failure technique of Rosenbrock2).  A start point xO and initial st
d are required, and the function is evaluated at xO and xo + d.  Th
step is termed a success if F(xO + d) ' F(xO), otherwise it is a fa
If it is a failure, d is replaced by -d, where  is a contraction
less than one, and the test is repeated.  If it is a success, xo is
placed by xO + d, d is replaced by d, where  is an expansion fact
greater than one, and the test is repeated.  The process continues
this way until the function values change by less than a specified
The numerical values usually used for the expansion and contraction
meters are   3.0 and   0.4.
 
     An interesting feature of this method is that a local minimum
always bracketed whenever a success is followed by a failure.  When
happens, the middle one of the last three points is always lower th
outer two, so that one is in a favourable position for trying a qua
interpolation step.  The success-failure method, with one quadratic
polation step each time a success is followed by a failure, is prob
the most effective one-dimensional technique for use on general fun
although in special cases other methods may be superior.
 
                 Seen Rejected  Percentage
 
Characters       1639       17   98.96
 
                                     - 12 -
   STEPPING METHODS IN MANY VARIABLES
   3.1  Grid searches and random searches
        An excellent illustration of the enormous increase in complexity in
   going to spaces of high dimensionality is afforded by the grid search
   technique in many variables.  In order to localize a minimum to 1% of
   the range of one variable by this technique requires 100 function evalua-
   tions;  in ten variables the number of points required is 102|.  Clearly
   we can forget about this method when more than one or two parameters are
   involved.
 
        In fact it is a general rule in function minimization, as in func-
   tion integration, that one should not expect good one-dimensional tech-
   niques to be good when extended to higher dimensionality.  Experience
   with integration suggests that a Monte CarZo search is more efficient
   than a grid search in many dimensions.  The Monte Carlo technique con-
   sists in choosing points randomly according to some distribution
   (usually uniform or normal).
 
        But even when these methods are refined by using variable search
   ranges, they prove far too slow for general use and we must turn to more
   efficient techniques.
 
 
   3.2  Single-parameter variation
 
        Since the condition for a minimum which is a stationary point in
   n variables xi is the vanishing of all n first derivatives aF/xi, it is
   natural to try to make each derivative vanish separately, one after the
   other.  This is the old method of single
   parameter variation, where one seeks a
   minimum with respect to one variable at a
       /           time using one of the techniques described
2 / /   MIN earlier.  Of course when you have finished
                                  minimizing with respect to X2 you may no
               -  / /            longer be at a minimum with respect to xl,
                    / /           so you generally have to start all over
       /                        again, but the process usually does con-
   SAR                          verge, as illustrated for two variables in
                           X l
                                  this diagram. Here the curves represent
 
                 Seen Rejected  Percentage
 
Characters       1533       11   99.28
 
                                - 13 -
 
 
contours of equal function value, and the straight lines show the step
taken in minimizing F with respect to xl, then X2, then xl, etc.  In
this case the method converges nicely after only four single-parameter
minimizations.
 
     Consider now the function represented by the contours shown below
Here the method proceeds much more slowly because of the narrow valley
 
 
 
 
 
 
 
 
 
 
 
 
 
 
 
 
                     S T `. V                '
                                              Xl
 
 
 
Such behaviour in many dimensions causes this method to be generally
considered as unacceptably slow.
 
     Two of the more successful improvements aimed at avoiding such
behaviour are due to Hooke and Jeeves3) and Rosenbrock2).  We discuss
the latter below.
 
 
3.3  Rosenorock's method
 
     Rosenbrock's algorithm2) starts by performing single-parameter
minimizations as above.  Then when one full cycle of all parameters ha
been completed, a new set of orthogonal axes is defined with one axis
taken as the vector from the start point to end point of the cycle.
This vector points in the direction of previous over-all improvement a
is expected to be a good direction for future improvement  In the cas
of the narrow valley seen above, it should point more or less along th
valley and avoid the zig-zag behaviour.  The next cycle of single-
variable minimizations is performed using multiples of the newly defin
axes as variables.
 
                 Seen Rejected  Percentage
 
Characters       1090       10   99.08
 
                                - 14 -
 
 
 
     The Rosenbrock method generally performs well, being quite stabl
and capable of following narrow valleys, but as the number of variabl
increases, the efficiency drops, probably because the new axis define
by past improvement is based on points so far apart that it no longer
points along a "hyper-valley" at the start point of the next cycle.
Also, its terminal convergence is slow compared with the more ''quadra
methods described in Section 4.
 
     Another technique, that of Davies, Swann, and Campey4) (unpublisl
see Ref. 4) is similar to Rosenbrock's and will not be described here
 
 
3.4  The simplex method
 
     One of the most successful stepping methods in many variables is
that of Nelder and Meads), based on the simplex.  A simplex is an n-
dimensional figure specified by giving its n + 1 vertices.  It is a
triangle in two dimensions, a tetrahedron in three, etc.  The algorit
takes the name simplex because at each step the information it carrie
about the function consists of its values at n + 1 points.  One can
easily visualize how the method works by considering the two-dimensio
case as in the diagram below.  The three starting simplex points are
somehow  chosen (perhaps randomly) and the function is evaluated at e
point.  Let the point PH be that at which the function value is highe
(worst) and PL that at which it is lowest.  Let P be the centre-of-ma
of all points in the simplex except PH;  that is:
 
 
                                 -n+l
                               n '   i    H' -
 
 
 
                                                                P-
 
 
 
                   2  P 
 
 
 
 
 
                             P1 =PL
                   P3 Ph
                                             xl
 
                 Seen Rejected  Percentage
 
Characters       1198        9   99.25
 
                                 - 15 -
 
 
From the original simplex, a new simplex is formed by replacing PH by
better point if possible.  The first attempt to find a better point is
made by reflecting PH with respect to P, producing P* = P + (P - PH).
If F(P*) < F(PL), a new point is tried at P   = P + 2(P - PH).  If
F(P*) > F(PH), a new point is tried at P** = P - (P - PH).  The best
the new points then replaces PH in the simplex for the next step, unle
none of them is better than PH.  In the latter case, a whole new simpl
is formed around PL, with dimensions reduced by a factor of 0.5.
 
     Variations on the method are possible by using different contract
or expansion factors when searching along the line from PH through P
(dotted in diagram).  Another interesting possibility is to attempt a
quadratic interpolation step along the dotted line whenever three poir
have been determined (PH, P*, P**).  However, one must be careful not
accept a point too close to P, for then the simplex collapses into a ]
(or in general a hyperplane of n- 1 dimensions) from which it can nev
recover.
 
     The simpl|x algorithm, being designed always to take as big step
as possible, is rather insensitive to shallow local minima or fine
structure in the function caused by rounding errors, statistical erro
(lonte Carlo output), etc.  Another of its virtues is that of requiri
few function evaluations, usudlly one or two per iteration.  In addit
each search is in an "intellient" direction, pointing from the highe
value to the average of the lowest values.  Com?are this with Rosenbr
;lethod, where really only the principal axis is an "intelligent" dire,
tion, and all other searches are for exploring along orthogonal axes
determine a new principal axis.
 
     A convenient convergence criterion for the simplex method is bas
on the difference F(PH) - F(PL).  The iterations are stopped when thi
difference is less than a preset value.  As a final step, the functio
evaluated at P, which is often slightly better than F(PL).
 
     In view of the danger mentioned above -- of the simplex collapsi
into a hyperplane of dimension n       it has been suggested to use n
or more points rather than n+ 1 at each step.  I have tested this ide
which is equivalent to introducing a dummy parameter of which the fun
tion is independent, and have always found the efficiency of the algo
 
  ,           t l r : .  r  n rl i t i  n c 
 
                 Seen Rejected  Percentage
 
Characters       1908       13   99.32
 
                                 - 16 -
GRADIENT METHODS
_
 
4.1  Calculating derivatives
 
     I will call a gradient method any technique which uses informatia
from a very small range of the variables (i.e. essentially derivatives
to predict good trial points relatively far away.  This does not neces
ily mean that they follow the gradient, but only that the gradient, ar
perhaps higher derivatives, are used or estimated.
 
     It is of course possible in most cases to calculate analytically
numerical values of the derivatives of a function, just as it is possi
to calculate the value of the function itseif.  However, it is often i
convenient and dangerous if the algebra is complicated, so that very
often we are faced with minimizing a function for which no derivativec
are provided.  Since the most powerful algorithms discussed below req
derivatives, a general minimization program must be able to estimate t
derivatives of the function by finite differences.
 
     A first derivative may be estimated from
 
 
                      aF      F(xo+ d) - F(xO)
                        x              d
 
                       x o
 
 
 
 
 
 
 
 
 
 
 
 
 
 
 
 
 
 
 
 
 
 
 
 
 
 
 
 
 
 
 
 
 
 
 
 
 
 
where d is a "small" displacement.  The error will be, to lowest orde
in the Taylor's expansion,
 
 
                                d   a2F
 
 
 
It is therefore advantageous to make d as small as possible, but stil;
large enough so that the rounding error in the computation of F does I
become larger than the error introduced by o.  Since the second deriv-
atives may not be known, it may not be possible to find an optimum st
size d, so we may just have to close our eyes and guess.
 
     A much safer method would be to use points chosen symmetrically
either side of xo giving
 
 
                      aF|     F(xo + d) - F(xo - d)
 
                 Seen Rejected  Percentage
 
Characters       1274        4   99.69
 
 
 
for in this case the error  vanishes to second order and the lowest
order term is proportional to the third derivative.  A disadvantage
this method is that it requires 2n function calls to estimate the n
derivatives, whereas the asymmetric steps require only n +l [or only
F(xo) has to be evaluated anyway].  An advantage of the symmetric st
method, however, is that it gives the second derivatives as a by-pro
[assuming F(xo) known]:
 
                 a2F   F(xO - d) + F(xO+ d) - 2F(Xo)
                                      2
 
 
and from the relationship for the error  in the asymmetric method,
conservative upper limit of the uncertainty in the first derivative
results assuming at least that the symmetric formula gives a smaller
error than the asymmetric one.  A complete treatment of step sizes i
beyond the scope of these lectures but can be found in a paper by
Stewart6 ) .
 
     The numerical evaluation of second derivatives is facilitated b
the fact that they should be approximately constant over small regio
so that symmetrical steps are usually not necessary.  Unfortunately,
however, there are a lot of second derivatives to evaluate;  since t
form a symmetric n x n matrix, there are n(n + 1)/2 independent comp
nents, requiring at least n(n - 1)/2 points  in addition to those
                                    required for the symmetric deriv
                                    atives. For two parameters, a m
                                    mum point pattern is shown in th
                                    diagram at left. The odd point
   2                 c             the mixed second derivative) cou
                 O O dO           have been chosen in any corner.
                                    two-dimensional diagram is somew
                     |              misleading since for large n, th
                                    number of "odd points" is n time
                                    larger than the number of "symme
                                    points.
 
                 Seen Rejected  Percentage
 
Characters       1297        6   99.54
 
                                  - 18
 
 
4.2  Steepest descent
 
     As soon as the function's first derivatives are known, it is natura]
to follow the direction of the negative gradient vector in seeking a
minimum, since this is the direction in which the function is decreasing
the fastest.  Such a technique was used by Cauchy more than a century
ago, and is the basis of what is now known as the method of steepest
descent.
 
     This method consists of a series of one dimensional minimizations,
each one along the direction of local steepest descent (gradient) at the
point where each search begins.  Of course the direction of the gradient
is not constant along a line even for a general quadratic function, so
we expect many iterations to be necessary, but the method can be shown
to converge for a quadratic func-
tion.  Let us follow its progress
                    / / / /  .-MIN  for a typical function whose con-
                        / / /  / /  tours are shown in the diagram.
                        / / /  / /  We immediately see an unfortunate
                   STARI  / / / /  property of the successive search
                        `J-' / /  directions: if each linear mini-
                                  /  mization is exact, successive
                                      searches must be in orthogonal
                                      directions.  In two dimensions,
                                      this yields steps which look just
like the single parameter variation method with the axes rotated to
line up with the gradient at the start point.  In many dimensions the
situation is not quite so bad, but successive directions are still
orthogonal and the algorithm cannot be considered acceptable.
 
     It is in fact easy to draw contours for a reasonably well-behaved
                    "'|           hypothetical function (as at left)
 
 X2               / where the direction to the
                /  ) minimum is just perpendicular to
 
                                      the gradient.
 
                 Seen Rejected  Percentage
 
Characters       1333       11   99.17
 
                            - 19 -
 
 
4.3  Newton's method
 
     It is clear that since a general quadratic function is determil
by specifying its value, first derivatives, and second derivatives .
point, it can be minimized in one step if and only if all this info
tion (or its equivalent) is taken into account.  Let us write a qua
ratic function as
 
              F( )   F(  ) +  T | (    ) + l (    )T G(
 
 
 
where the gradient  is evaluated at xo and the second derivative m
G is a constant.  Then the minimum is given directly by
 
 
                     Xm = xO- G lg = xO_,
 
where the inverse of the second derivative matrix is the covorience
matri v
 
     This is then the many-dimensional equivalent of quadratic inteI
tion discussed earlier, and it is subject to the same sort of diffic
when applied as an iterative technique to general non-quadratic func
But let us first point out its good features:
 
  i) the step size is no longer arbitrary, but is prescribed precise
     the method;
 
 ii) the step directions are no longer necessarily along the gradier
     vector but take account of parameter correlations (narrow valle
     or ridges) through the mixed second derivative terms.
 
     In practice, however, the method is unstable, essentially for t
reasons given in Section 2.4.  In particular, it diverges whenever t
matrix G (or V) is not positive-definite (see next section).  In its
modified form  the method is used only when the minimum is known to
very close or when the function is known to be positive quadratic (f
linear least squares).  However, it is clearly a powerful technique
and is worth studying in some detail since all the most successful
algorithms are based on ewton-Zike steps, as discussed below.
 
                 Seen Rejected  Percentage
 
Characters       1327        7   99.47
 
                                 - 20 -
 
 
4.4  Positive-definite quadratic forms
     We pause here briefly to consider the properties of quadratic forms
useful for understanding the more powerful gradient methods.  In one
dimension the description is simple;  a general quadratic form can be
written
 
                         F(x) z a + gx + 2 Gx2 ,
 
 
where g = aF/ax at x = 0, and G = a2F/ax2 also at x = 0.  This function
has a minimum if and only if G > 0.  If G = 0, the minimum is at infinity
The minimum (if it exists) is at x = -g/G.  len using a quatratic ap-
proximation to minimize a general non-linear function, it makes sense to
take a step to x = -g/G only if G > O since otherwise we step to a pre-
dicted maximum or to infinity.  A possible remedy if G < O is to take a
step x = -g;  that is, to set G arbitrarily equal to unity so that the
step will at least be in the right direction although it will now have
arbitrary length.  Consideration of the sketch below shows that this is
the only thing we can do unless more information is available, since the
quadratic part of the function is not convex or positive-definite at
the point xo:
 
 
 
          F t Gc O   G|
 
 
                  l  I
                      -g/G   I   \
 
                         -9 \
 
 
 
                       xo                       x
 
 
 
     These arguments may now be extended to many dimensions where g
becomes the gradient vector , and G becomes the second terivative matrix
G.  Then the Newton step to x - -G lg makes sense only if | (or | l) is
a positive-definite matrix, since only then does the quadratic form
 
                 Seen Rejected  Percentage
 
Characters       1117       10   99.10
 
 
                                   T      1  T
                        F(x) = a + g | x + 2 x Gx
 
 
have a minimum.  If G is singular, the predicted minimum (or maximur
not unique.
 
     Unfortunately there is no simple way of telling, in general, ii
matrix is positive-definite by inspecting individual components, bu
can at least state some of the many useful properties of such matri
Two necessary (but not sufficient) conditions for a (square, symmet
matrix to be positive-definite are:
 
  i) the diagonal elements must be positive (this is in fact suffici
     for a 1 x 1 matrix);
 
 ii) the off-diagonal elements must obey Gij < GiiGjj.
 
Properties (i) and (ii) together are sufficient for a 2 x 2 matrix
While the above conditions are easy to check, they are not in gener.
sufficient.  Some necessary ond sufficient conditions are the follo
 
 iii) All the eigenvalues of the matrix are positive.  This is gener
      a rather difficult calculation and is usually approximate.
 
 iv) The determinants of all the upper left square submatrices (forr
                                     as indicated in the diagram at
                                     left) are positive.  This is p
                     ..              ably the easiest method.
 
 
  v) The scalar eT|e is positive for all vectors e.  This is usuall
     taken as the definition of a positive-definite matrix, and exp]
     why a positive-definite matrix yields a quadratic form with a r
     mum:  the function increases in all directions from e  0.
 
 vi) The inverse G 1 s V is positive-definite.
 
     Now suppose that G-l is calculated for a Newton  step and turn
to be non-positive-definite.  In analogy to the one dimen8ional cas
would simply take G - I, the unit matrix, and the Newton step would
become a steepest-descent step of arbitrary length, which is probab:
not so bad an idea and is in fact often don.  But we can do better
 
                 Seen Rejected  Percentage
 
Characters       1383       12   99.13
 
                                - 22 -
 
 
trying to make a positive-definite matrix which is as "close" as possib
o the unacceptable G   The ways in which this can be done depend on wh;
is "wrong" with G.  For example:
 
  i) If all the diagonal elements of G are positive, the off-diagonal
     elements can simply be set - 0.  The resulting matrix will be bett
     than the unit matrix, since it will at least produce a scale-
     invariant step and non-arbitrary step length.
 
 ii) If the only thing "wrong" with G is that one or more off-diagonal
     elements of G or G 1 do not satisfy condition (ii) above, just the
     off-diagonal elements can be set to zero.
 
iii) The matrix (G +I) 1 can be used instead of G 1, where  is greate
     than the largest negative eigenvalue of G   This requires a large
     amount of calculation and so is not very convenient, but it is
     quite appealing since it amounts to taking a step which is inter-
     mediate between a Newton step and a steepest-descent step (for
     large valueg of  the step becomes short and in the direction of
     the gradient).
 
 iv) If one or more of the diagonal second derivatives are negative, th
     non-positive-definiteness can be turned into an advantage, since i
     indicates a direction (or directions) in which the negative first
     derivative is increaing in magnitude rather than decreasing.  Thi
     suggests an especially fruitful direction for a single-parameter-
     variation step which should not only lead to a good decrease of th
     function value but should also lead more quickly to a region of
     positive-definiteness.
 
     Minimization methods based on variations of Newton's method as
8uggested by the above considerations are usually called quasi-Newton
methods.  Many such algorithms have been published and some are quite
8ucce8sful, but the field is still open for new ideas.
 
     The principal drawback of such techniques is the repeated evalua-
tion and inversion of the second-derivative matrix.  The calculation of
the second derivatives usually requires a rather long time, proportiona
to n2, and the matrix inversion, although usually faster, increases wit
n like n.
 
                 Seen Rejected  Percentage
 
Characters       1701        6   99.65
 
                                - 23 -
 
 
     One of the most interesting results concerning quadratic forms is
the basis of a collection of related techniques described in the next
sections, which do not require explicit repeated evaluations of G.
 
 
4.5  Conjugate directions
 
     The vectors di and dj are said to be conJugate with respect to a
positive-definite symmetric matrix A if
 
 
                        d Ad. = O   for   i  j .
                            J
 
 
If A is the unit matrix I, the conjugate vectors d would be orthogonal
so conjugacy can be thought of as a generalization of orthogonality.
set of n conjugate vectors span an n-dimensional space, and any point
in the space can therefore be expressed as a linear combination of n
conjugate vectors.
 
     Although the matrix A does not uniquely define a set of conjugate
vectors, such a set can always be constructed by a procedure similar t
the Gram-Schmidt orthogonalization method.  Let us start for example w
an arbitrary vector dl.  Then the vector
 
 
                                    d Ad
                      d2 = Adl -  T   1 d
 
                                    dlAdl
 
 
can be seen to be conjugate to dl since the product dlAd2 vanishes
identically.  The process can then be continued in the same way to
construct a d3 which will be conjugate to both dl and d2, and so forth
up to dn.
 
 
 
 
 
 
 
 
 
 
 
 
 
 
 
 
 
 
 
 
 
 
 
 
 
 
 
 
 
 
 
 
 
 
 
 
 
 
 
 
 
 
 
 
 
 
 
     Such vectors become interesting for minimization problems when th
are conjugate with respect to the hessian (second derivative) matrix G
In this case a theorem of Fletcher and Reeves7) states that a sequence
of linear minimizations in each of the n conjugate directions will
minimize a general quadratic function of n variables.  That this is tr
can be seen quite easily as follows.  Let the quadratic function be
 
 
                      F(x) s F(O) + gTx + l XTG
 
                 Seen Rejected  Percentage
 
Characters       1304        6   99.54
 
                                 - 24 -
and the n directions di be conjugate with respect to |:
diGdj - O, i  j .
Then the vectors x and  can be expressed as linear combinations
 
 
 
 
 
 
                              g =  C
so that the general quadratic becomes
      F(x) = F(O) + (  cidTJ ( yjdj) + 2 ( yidTi)G ( yjdi) .
Now  if the last term above is regrouped as a double sum, the terms wi
i  j drop out because of the conjugacy condition, so that the whole
expression can be simplified as
 
 
               F(x) = F(O) +     Cididjyj + 2  yjdj|d
                       i j               i
 
 
               = F(O) +   (bjyi + b;yj2)
 
 
where
                                     bj =  cidid
 
 
and
 
                               bj   dTGd
 
 
 
 
 
 
 
 
 
 
 
 
 
 
 
 
 
 
 
 
 
 
 
 
 
 
 
 
 
 
 
 
 
 
 
 
 
 
 
 
 
 
 
 
 
 
 
 
 
 
 
 
are constants.  By expressing the quadratic in terms of y instead of x
we have separated it into a sum of independent one-parameter quadratic
functions.  A minimization with respect to yi (a linear minimization
along the direction di) will therefore be independent of the minimizat
along the other conjugate directions, which demonstrates the validity
the theorem.
 
                 Seen Rejected  Percentage
 
Characters        712       14   98.03
 
                                - 25 -
 
 
     The above theorem tells us what is "wrong" with the single-paramel
variation method:  we should be using conjugate directions rather than
simply orthogonal axes.  However, since the construction of conjugate
vectors seems to require knowledge of the hessian G, this does not yet
help very much in practice, for if we knew  (and g) we could minimize
quadratic immediately by means of Newton's method, and would not need
use n linear minimizations.
 
     The usefulness of conjugate directions comes from the fact that
there are ways of determining such directions implicitly, without firs
evaluating the entire hessian matrix .  Of course, by the time all n
conjugate directions are determined, by whatever method, information
equivalent to the matrix G must have been determined.  However, by tha
time considerable minimization may already have been performed, as in
the method implied by the following theorem.
 
     If xo and xl are minimum points in two parallel subspaces, then t
                                    direction xl-xo is conjugate to a
      X2                          vector which lies in either sub-
                 /                 space. This can easily be seen i
              / / / /              two dimensions as illustrated
                 W /                in the figure at left. Since xo
                                    is a minimum along the direction
                 dl                 the gradient of F at xo must be
                 x O
 
         _                 | X     orthogonal to dl:
 
 
                          dlT(g + Gxo) = O ,
where g is the gradient at x = 0.  Similarly at xl:
 
 
                          dl (g + GXl) = | -
 
 
Subtracting the above equations, the first terms drop out and we have:
 
 
                          dlG(xl- xo) = O,
 
 
showing that (xl- xo) is conjugate to dl.
     Unfortunately, extending this algorithm to three dimensions requi
three additional minimizations in order that the third direction be
conjugate to both of the first two, so that convergence for a general
 
                 Seen Rejected  Percentage
 
Characters       1381        8   99.42
 
                                - 26 -
 
 
quadratic in n variables is obtained only after n iterations involving
ll n(n+ 1)/2 linear minimizations.  Since this is just the number of
independent elements in the second derivative matrix, we would be bett
off for quadratic functions to calculate this matrix directly and avoi
the linear searches.  On the other hand, for non-quadratic functions tl
conjugate directions method should be much more stable since it procee
by a series of linear searches in independent directions and still
guarantees convergence in a finite number of steps once a quadratic
region is entered.  In addition, this method has the advantage of
requiring neither first nor second derivatives of the function.
(Strictly speaking, then, it should have been discussed in Section 3
rather than in this section.)
 
      A disadvantage of the algorithm described above is that for each
iteration, n minimizations are performed in direction dl, whilst only
one is performed in direction dn.  This undesirable asymmetry is large
avoided in a variation due to Powell).
 
 
4.6  Conjugate gradients
 
      When the first derivatives of the function are calculated, a som
what more elegant method can be used, known as the method of conugat
gradients7). Suppose that the function and its gradient are evaluatec
at two points xo and xl, giving differences:
 
 
                                                         x = Xl - XO
 
                            g = gl - go .
 
 
Then if the function were quadratic with hessian  we would have
                                g = G x.
 
 
Any vector dl orthogonal to g would then be conjugate to x:
 
 
                         dl g = dTlG x = O ,
 
 
which immediately suggests a method for obtaining conjugate direction
without knowing G, based on the change in gradient along a previous
direction.
 
                 Seen Rejected  Percentage
 
Characters       1373       16   98.83
 
                                 - 27
 
 
     In the method of conjugate gradients, successive one-dimensional
minimizations are performed along conjugate directions with each direc-
tion being used only once per iteration.  The first direction is taken
as do = -go, the steepest descent vector at xo.  Let the minimum along
this direction be at xl where the gradient is gl.  Then the next  searc
direction dl, which we want to be conjugate to do must be a linear
combination of the only vectors we have at hand, namely:
 
                             dl = -gl + bdo .
 
 
The conjugacy condition is
 
 
                        dlGdo = dTlG(xl - xo) = O
or
                (-gl + bd)Gdo = (-gTl - bgTo)(gl - go) = O .
 
 
Since xl is a minimum along direction do = -go, the direction go is
orthogonal to the gradient at xl, so that glgO = 0.  We are then left
with
 
                                b =
                                    gogo
so that the new conjugate direction is
 
 
                      dl = -gl + () do .
                                _ o _ o
 
 
This process can be continued to generate n directions, each one conju-
gate to all the others.  It turns out that the same simple formula hold
for all the successive conjugate directions
 
 
 
 
                  -i+l -i+l + (8i+l) di .
 
                 Seen Rejected  Percentage
 
Characters        825        5   99.39
 
                                 - 28 -
 
 
4.7  Variahle metric methods (VMM)
 
     In analogy with the methods of differential geometry and general
relativity, it is convenient to consider the properties of the functio]
F(x) as being in fact properties of the space of the variables x.  We
have already made some rudimentary use of this idea when we generalize
from the usual orthogonal coordinate axes to a system defined by axes
pointing in conjugate directions.  We now wish to go further and be ab
to express the properties of the function F geometrically as the prop-
erties of the non-Euclidean space of its variables x.
 
     The fundamental invariant in a non-Euclidean space is the squared
distance element
 
                               d 2   d TAd
 
 
 
 
 
 
 
 
 
 
 
 
 
 
 
 
 
 
 
 
 
 
 
 
 
 
where dx is a differential coordinate displacement and A is the covari
metric tensor which determines all the properties of the space under
consideration.  When A is just the unit matrix I, the above formula fc
ds2 just expresses the Pythagorean theorem for an n-dimensional Euclid
space.  When off-diagonal elements of A are non-zero and when the
elements are allowed to vary as functions of x, a generalized non-
Euclidean space is generated.
 
     It is easily verified that the second derivative (hessian) matri
behaves under coordinate transformations like a covariant tensor and
will identify it with the metric tensor of our space.  The inverse
v = | 1 is a contravariant tensor and becomes the contravariant metri
tensor.  (For a discussion of covariant and contravariant tensors, se
for example chapter 10 of reference 9.)  This immediately enables us 1
construct two scalar (invariant under coordinate transformations)
quantities:
a)    ds  = _d_x _Gd_x_
 
is the square of the generalized distance between the point x and the
point x + dx.  When F is a chisquare function which is minimized to
determine some best parameters x, then the physical meaning of the
generalized distance ds is just the number of "standard deviations"
x + dx is away from x.  That is, the use of the metric tensor _G enabl
us to scale the distance dx so that it comes out as a physically (or
 
                 Seen Rejected  Percentage
 
Characters       1692        4   99.76
 
                               - 29 -
 
 
statistically) meaningful invariant quantity instead of being expre
in arbitrary units (or a mixture of arbitrary units').
 
And                       b)    p = g Vg
 
is twice the difference between the function value at the point whe
and the gradient g are calculated and the minimum of a quadratic foI
with hessian matrix  = V l.  That is, p/2 is the expected (vertica]
distance to the minimum if the function F were quadratic.  This pro
us with an important scale-free convergence criterion for any method
which provides approximations to V and g.
 
     When the function F is quadratic, G is constant everywhere and,
the sense outlined above, this is equivalent to working in a space 
a constant metric.  For real non-linear functions we expect higher-c
terms to be small but not negligible, so that we can think of workir
a space with a slowly-varying metric tensor.  Minimization methods
on this approach are known as variabZe metric methods.  They differ
the basic Newton-Raphson method in that the matrix G is not complete
re-evaluated at each iteration, but is assumed to be well approximat
taking the G of the previous iteration and applying a correction bas
new information from the current iteration.  This correction is know
the matrix updating formuZa, which in general differs from method to
method.
 
     Variable metric methods therefore proceed generally by the foll
steps:
 
  i) A starting point xo is given, the gradient go at that point is
     calculated, and some approximation to  1, say VO, is construct
     The starting Vo may be only the unit matrix, or it may actually
     the inverse of the full second derivative matrix.
 
 ii) A step is taken to xl = xo - Vogo, which would be the minimum i
     were quadratic and if Vo were the true covariance matrix.  Sinc
     is not the position of the minimum in the general case, it is u
     to perform a linear search along this direction, finding the a
     which minimizes F(xO - aVgO).  In either case let the new point
     called xl and let the gradient calculated at xl be gl.
 
                 Seen Rejected  Percentage
 
Characters       1629        8   99.51
 
                                 - 30 -
 
 
iii) The matrix V is corrected using an updating formula of the form
                       Vl = Vo + f(voxoxlgogl) -
     Then go is replaced by gl. xo by xl, and VO by Vl, and steps (ii)
     and (iii) are repeated until some convergence criteria are satisfi
 
     The different methods differ chiefly in the choice of updating
function f, as described in the following sections, and in the extent t
which linear minimizations are necessary.  Less important variations
involve the starting approximation VO and various safeguards against
"unreasonable" steps and non-positive-definiteness as for the Newton
techniques.
 
 
4.8  Davidon's rank-two formula
 
     Probably the first -- and perhaps still the best -- variable metri
method was developed in 1959 by Davidon and later published in simplifi
form in 1963 by Fletcher and Powelll|).  Davidon's updating formula for
the covariance matrix is the following:
 
 
                         V 1 = V O +  T     T
                             y y Voy
 
 
where the changes in position and gradient on the last step were
 
 
                               = Xl - Xo
and
y = gl - go.
 
and Vo was the previous estimate of the covariance matrix.  This is
called a rank-two formula since the correction Vl -Vo is a matrix of
rank two in the space of  and Voy as can be seen directly by inspec-
tion of the formula.
 
     One fundamental requirement of an updating formula is that the new
matrix satisfies the relationship
 
                           VlY  .
 
 
since y - G for a quadratic with hessian G.  It is easily seen that
Davidon's formula satisfies this requirement:
 
                 Seen Rejected  Percentage
 
Characters       1188       14   98.82
 
                                 - 30
 
 
iii) The matrix V is corrected using an updating formula of the fcrm
                       Vl = Vo + f(voxoxlgogl) -
     Then go is replaced by gl, xo by xl, and VO by Vl, and steps (ii
     and (iii) are repeated until some convergence criteria are sati
 
     The different methods differ chiefly in the choice of updating
function f, as described in the following sections, and in the exten
which linear minimizations are necessary.  Less important variations
involve the starting approximation VO and various safeguards against
"unreasonable" steps and non-positive-definiteness as for the Newton
techniques.
 
 
4.8  Davidon's rank-two formula
    _
 
     Probably the first -- and perhaps still the best -- variable me
method was developed in 1959 by Davidon and later published in simpl
form in 1963 by Fletcher and Powelll|).  Davidon's updating formula
the covariance matrix is the following:
 
 
                        Vl = V| +  T     T
                              y y Voy
 
 
where the changes in position and gradient on the last step were
 
                               = Xl - Xo
and
y = gl - go.
 
and Vo was the previous estimate of the covariance matrix.  This is
called a rank-two formula since the correction Vl -Vo is a matrix o
rank two in the space of  and Voy as can be seen directly by insp
tion of the formula.
 
     One fundamentat requirement of an updating formula is that the
matrix satisfies the relationship
 
                              VlY_ ,
 
since y  G for a quadratic with hessian G.  It is easily seen tha
Davidon's formula satisfies this requirement:
 
                 Seen Rejected  Percentage
 
Characters       1166       15   98.71
 
 
VlY = [Vo    T
          j    V
 
 
 
 
 
   = Voy + -- Y _ oW o_y
 
                                     y    _y V o _y
 
 
 
 
 
 
 
 
 
 
 
 
 
 
 
 
 
 
                       = voy +  - voy =  -
 
 
 
     An unfortunate feature of the Davidon algorithm is the need to
perform at each iteration a linear minimization along the direction
given by a Newton step, -Vg.  This linear search step is, however,
necessary in order to assure convergence for general functions.  Fle
and Powell showl|) that if the starting approximation to V is positi
definite, then V will remain positive-definite after all updatings,
they have to use the fact that each iteration is a linear minimizati
that is                         T
                                              glVogo = O .
 
 
     It can be shown that this method is quadratically convergent, a
most n iterations (n linear searches and n gradient calculations) be
required for an n-dimensional quadratic form.
 
 
4.9  The rank-one formula
 
     In an effort to avoid the linear minimizations required by Davi
algorithm, several workers have independently developed an interesti
updating formula of rank one.  In this case Davidon in 1968 was the
to publish an algorithmll) based on the formula, and Powelll2) has s
marized the properties of this formula and of algorithms based on it
 
     The rank-one updating is:
 
 
                  Vl  Vo + (- VoY      Voy)T
                                     y ( - VoY)
 
 
It can be shownl2) that this is the only formula of rank two (or les
for which not only Vly =  but:
                           Vl-y 
 
                 Seen Rejected  Percentage
 
Characters       1042       11   98.94
 
                                 - 32 -
 
 
where i and Yi are the step and gradient changes at any previous itera-
-ion.  This is known as the hereditary property, since Vl can be said to
inherit the fundamental property Vy =  with respect to all previous
iterations (up to n).
 
     The hereditary prcperty assures that after n iterations, Vl will be
the true covariance matrix if F is quadratic, no matter what steps were
taken (almost), so that if Newton steps are taken, convergence for a
quadratic function is assured after n iterations, without the need for
 
         .
1 lnear mlnlml zat lons .
 
     In addition, the rank-one formula is symmetric, in the sense that
the expression for Vll in terms of Vol is the same as that for Vl in
terms of Vo, provided  and y are interchanged.  The meaning of this
symmetry property will be discussed in the next section.
 
     But, as nothing is perfect, so the elegance and mathematical beauty
of the rank-one formula hide a number of numerical and practical dif-
ficulties which can make it highly unstable when applied to a general
function.  In particular, if the vector y happens to be orthogonal to th
vector (- VOy), the denominator goes to zero in the updating formula,
and an unbounded correction is possible.  Since these vectors may be
orthogonal, even for a quadratic function, the problem of numerical
instability is a serious one.
 
     Moreover, the matrices Vl do not really converge to the true co-
variance matrix in the usual meaning of the term convergence.  Although
it is true that Yl will be equal to the true covariance matrix at the
nth step for a quadratic function (barring numerical difficulties), the
intermediate matrices V may vary wildly from step to step, so that on
any particular iteration Vl may be a rather poor approximation.  This is
especially dangerous when the function is not quadratic, since the large
corrections necessary in later iterations will generally not compensate
properly the fluctuations in early steps.  Also, there is no guarantee
that intermediate matrices will remain positive-definite, and hence no
guarantee of a reduction in the value of F at each step, even for a
quadratic F.
 
     All these difficulties can, of course, be overcome by programming
enough safeguards into the algorithm, but this can only be done at the
 
                 Seen Rejected  Percentage
 
Characters       1865        6   99.68
 
                               - 33 -
 
 
expense of efficiency and sometimes only by abandoning temporarily t:
updating formula itself, which makes it lose some of its appeal.
Different approaches are possible depending on whether it is conside
important to maintain positive definiteness as in the Davidon
algorithmll), or important not to abandon the exact rank-one formula
in Powell's methodl2).
 
 
4.10 Fletcher's unified approach to VMM
 
     The existence of two different updating formulas with very diff
properties generated a lot of interest in variable metric methods (
during the years 1967-1971, since it showed VMM to be very promising
left many questions unanswered, such as:
 
  i) How can it be that the rank-one and rank-two formulas have such
     different properties?  What is the relationship between them?
 
 ii) Is there a way to combine the best properties of both formulas?
 
iii) Are there other good formulas?  Is it possible to define a clas
     of "admissible" formulas?
 
     A certain understanding of the above problems has recently been
made possible by the work of a number of people.  In particular, a r
paper by Fletcherl3) presents a unified approach to VMM, which will
given here.
 
     Recall that the rank-one equation is symmetrical (in a sense de
in Section 4.9), but as we shall now see, the rank-two formula is no
Indeed the asymmetry suggests a way to construct a possible third fo
by taking the "mirror image" of the rank-two formula.  The basic ide
that a new formula should satisfy the fundamental relationship
 
                           VlY =  .
 
 
and therefore its inverse should satisfy
 
                                 V 
 
 
We can indeed write down the updating formula for ll which correspo
to the rank-two formula for Vl:
 
 
 
                 V           Vnl rI |  + Y _ Y
 
                 Seen Rejected  Percentage
 
Characters       1378        8   99.42
 
                                 - 34 -
 
 
This matrix Vll can now be thought of as a mapping from  ) y since
y = Vll.  If we interchange y and  in the formula, it will then give
a mapping from y  , thereby producing a new updating formula where
Vly = .  The new duaZ formuZa will be just
 
 
                        '    yT`   '   yT   T
                     V          T  Y| I    T      T
                           Y   Y,  Y
 
 
If we try this trick with the rank-one formula, we just get the same
rank-one formula back again, since it is symmetric in this sense, or
dual to itself.  But with the rank-two formula, the process of inverting
and interchanging yields a new formula, also of rank-two, which is also
a valid updating forr__ula in the sense that it gives rise to a quadrat-
ically convergent VM1 algorithm.
 
     Now we go further and consider the class of formulas which includes
both rank-two and dual formulas as special cases.  Let us introduce the
notation
                   Vl = T(Vo)   for the rank-two formula ,
and
                   Vl = D(Vo)   for the dual formula ,
 
 
and consider the class of updating expressions as introduced by
 
Fletcherl 3):
 
                         V,        ) T +  (D)
 
 
where  is some parameter which determines the exact formula.  Broydenl4
using a somewhat different notation, has also considered the same class
of formulas.]
 
     It then turns out that the rank-one formula is also in this class,
 
with
                                          Ty
                       (rank-one) 
                               ( y y_ Voy-)
 
 
 
     Having now constructed a wide class of updating formulas, which in
fact includes all formulas known to the author, it will prove interestin
to consider their properties as a function of the generating parameter
Probably the most important property, and the only one we will consider
here, is that of monotonic convergence of V toward the true covariance
 
                 Seen Rejected  Percentage
 
Characters       1325       28   97.89
 
matrix for a quadratic function.  [This is called Property 1 in Flet
paperl3) which should be consulted for details of the definition an
theorems concerning it.]  The use of an updating formula with this p
erty will guarantee an improvement in the approximation y at each it
tion (for a quadratic function).
 
     Any formula V with  in the interval [0,1] possesses the monot
convergence property.  Such a formula is said to belong to the conve,
cZss of formuZas. For any V with  outside the range [0,1], there
exists some quadratic function for which V diverges from the true co
variance matrix.
 
     From what we have already seen about the rank-one formula, it i
not surprising to find that it does not belong to the convex class.
Since  y > O for any step which is an improvement, and since yTvoy
if VO is positive-definite, it can be seen immediately from inspecti
of the equation for (rank-one) that it must either be less than zer
or greater than one.
 
     Ihe aboe considerations lead Fletcher to propose a new algorit
which is probably the most elegant and powerful of any VMM algorithm
Basically, he uses the genera] updating formula V, with the value o
chosen accord.n to the followirg clle!re:  If ,(rank-one) < 0, set
corresponding to the usua rdn-two formula.  If (rank-one) > 1, se
 = 1, corresponding to the dual formula.  In this way, one always u
a formula in the convex class, and chooses that one which is "closes
to the rank-ne formula.  It seems that the linear searches can then
eliminated and replaced simply by Newton's steps, unless the functio
highly non-quadratic.  The latter condition can easily be detected b
comparing the actual improvement with the expected improvement at ea
iteration.
 
 
 
SPECIALIZED TECHNIQUES
 
          All the methods outlined so far in these lectures are of rather
     general applicability, the only assumption being -- for some methods
     a predominantly quadratic behaviour in the immediate vicinity of the
      mihimum.  In order to develop more powerful methods than those alrea
     presented, we will have to give up some of this generality and explo
 
                                  ,, m;n;m   l  Tn  thi c  crt;
 
                 Seen Rejected  Percentage
 
Characters       1746       23   98.68
 
                                 - 36 -
 
 
we discuss a few specialized techniques which are still of rather wide
applicability in the sense that most functions of physical interest
fall in one or more of these classes.
 
5.1  Chisquare minimization
 
     Probably the most common application of minimization in scientific
research is in least squares fitting, where the function to be minimized
is the sum of squares of deviations, between measured values and predic-
tions of a model containing variable parameters:
 
 
                          K           K   Yk - Tk(x) 2
                 F(_x) =   fk(_x) =
 
                     k=l         k=l `         '
 
 
where Yk and k are measured values and errors, and Tk(x) are the values
predicted by the model, depending on some parameters x.  Minimizing F
then yields best values (estimates) of the n parameters x, based on K
measurements Y with random errors , where K must be greater than or
equal to n, and is usually much greater than n.
 
     Let us now consider the second derivative matrix for F(x), expressec
in terms of the individual fk(x):
 
                 2F a a 2
              ax ax  = axi ax   fl;
 
 
 
                                           a        afk
                    = aX  2fk aX
 
                              k
 
 
 
 
 
 
 
 
 
 
 
 
 
 
 
 
 
 
 
 
 
 
 
 
 
 
 
 
 
 
 
 
 
 
 
 
 
 
 
 
 
 
 
 
 
 
                                            2
                    =   2 aXk aXk +  2fk ax ax
                           k      i       k        1  J
 
 
In the above r.h.s., it is usual to make the approximation that the
second sum, involving second derivatives, is small compared with the
first term involving products of first derivatives.  This is called
ZinearizQtion.  [Note that it is the mode1, T(x) that is being linearized,
not the function F(x).]  In the important special case of 1,inear Zeast
quare, the second sum is exactly zero, so that F (x) is quadratic, and
the whole minimization problem reduces to the inversion of the above
matrix a2F/axjaxi (i.e. the taking of one Newton step).
 
                 Seen Rejected  Percentage
 
Characters       1300       12   99.08
 
 
     In the more general case of non-Zinear Zeast squares, the linea
tion approximation consists in taking
 
                     a2F        afk afk
                     ax ax   ax ax -
 
 
This has the advantage of being easy to calculate and, moreover, it
always positive-definite (under rather weak conditions such as the
existence of the derivatives, and provided it is non-singular).  In
in many cases the use of the above approximation in computing Newton
steps is actually more effective than using the exact second derivat
matrix because of the positive defiteness.  Of course it must be
remembered that the covariance matrix obtained by inverting this apF
imate matrix does not in general converge to the true covariance mat
even though the minimization  based on it may converge to the true
minlmum.
 
 
5.2  Likelihood maximization
 
     An increasingly important alternative to the least squares met
in data fitting is the method of maximum likelihood.  In this case
function to be minimized is of the form
 
 
 
                          F(x) = -  ln fk(X) ,
                                       kSl
 
 
that is, a sum of logarithms.  Here again, an approximation for the
second derivative matrix can be found which involves only products
first derivatives:
 
 
                        S - axi ax   ln fk
 
 
 
                         a l k
                        axi  f k axj
 
 
 
                            1 k afk 1 a2fk
                            k aXi axj     fk aXiaxi  -
 
                 Seen Rejected  Percentage
 
Characters        991       11   98.89
 
                                - 38 -
 
 
As with least squares, we can neglect the second sum, involving second
derivatives.  In the case of the likelihood function, the second deriv-
atives of f are never exactly zero over any finite range (exactly linea
maximum likelihood does not exist, essentially because the likelihood
function must be normalized so that its integral over the space of
measurements is independent of the parameters x).  However, the approxi
mation
a2F        1 afk afk
                         ax.ax.  k2 ax. ax.
                                   k
 
 
has the same advantages as in the non-linear least squares case, namel
speed of calculation and assured positive-definiteness.
 
 
5.3  Models with separable computing
 
     It often happens that the computation of the function value F can
be arranged so that large parts of the calculation depend on only a fe
of the variable parameters x.  These parts will then remain unchanged
the corresponding parameters have not changed since the previous funct
evaluation.  An important special case of this is when the calculation
can be separated into n pieces, each depending on only one parameter.
 
     Whenever the computing is separable in the above sense, large por
tions of the computation may be avoided by testing which parameters ha
not varied since the previous function call, and using the previous
results of the corresponding sub-calculation when appropriate.  The ov
all saving in computer time will then depend, of course, on the minimi
tion method used, and some otherwise inferior methods, such as single-
parameter variation, may become relatively efficient because of the ti
saved in computing the function.
 
     In particular, the cost of computing derivatives by finite dif-
ferences will generally be much lower when the computing is separable,
and in the extreme case of complete separability, all n first derivati
may be computed in a time comparable with that of one full function
evaluation.
 
                 Seen Rejected  Percentage
 
Characters       1559        4   99.74
 
                                - 39 -
 
 
5.4  Sets of related problems
 
     Many applications involve a series of minimizations for which t
functions involved are closely related.  For example, in determining
confidence intervals for a parameter y in a statistical problem inva
additional parameters x also to be estimated, one determines the cuc
p(y) traced by the minima of the chisquare function F with respect t
for different values of y:
p(y) = min F(x,y) .
 
 
A series of points on this curve is then determined by fixing y at
several different values, and for each of these, minimizing F as a f
tion of x.  Clearly then, information from the first minimization ca
used as a starting point for the second, and then extrapolated to gc
starting point for the third, and so on.  In particular, one does nc
expect the covariance matrix to vary considerably from point to poir
so this is an especially valuable piece of information to carry ove
one problem to the next.
 
     Similarly, if an experiment is repeated with new, independent,
identically distributed data, the minimizations involved in the datc
analysis can be speeded up by supplying covariance matrices from th
analysis of the first experiment.
 
 
 
LOCAL AND GLOBAL MIN
6.1  The problem of multiple minima
     All the methods presented so far have been designed to find a
minimum, without any consideration of whether or not other local mi
exist, or whether the minimum found is actually the global minimum.
If the function has more than one local minimum, there is not even ,
guarantee that these methods will find the minimum closest to the
starting point, let alone the global minimum.  In fact, it is usual
assumed, when using these algorithms, that the function is unimodal
(has one minimum) in the region of interest likely to be explored d
the minimization.
 
                 Seen Rejected  Percentage
 
Characters       1470        3   99.80
 
                                 - 40 -
 
 
     Whenever the function may have more than one local minimum, new
problems arise in addition to the problem  of local minimization.  Firs
of all, the user must decide what he wants to know about the function.
The following four possibilities are the most common and will be
discussed here:
 
  i) it is sufficient to know the location of any one local minimum;
 ii) only the global minimum is of interest;
 iii) only one minimum is of interest (the "physical solution"), but it
 need not be the global minimum; or
 iv) all local minima, including the global one, must be found and
 catalogued.
 
     The first possibility, (i), is quite rare, but is easy to deal wit
since any local min;mization routine is sufficient.
 
     Possibility (ii) is much more common, particularly in system
optimization where the cost must be the smallest possible, not just
small compared with other near-by solutions.  Several methods exist for
finding global minima, of which two will be discussed in the next sec-
tions.  All such methods suffer from the absence of a stopping rule:
even if the global minimum is found there is no way of recognizing it
unless the function is known to be bounded and has reached its lower
bound.
 
     Possibility (iii) often arises in scientific research where the
approximate values of some parameters are known in advance and one seeks
a solution not too far from these values, corresponding to "the right
valley" where the function may have several faraway valleys which may
be deeper.  The usual technique for making sure of staying in the right
valley is first to fix the approximately known parameters at their
assumed values and minimize with respect to all other variables, then
starting from this point minimize in the entire variable space.
 
     Possibility (iv), of having to find and record all local minima,
is the most difficult of all.  It arises, for example, in energy-depende
phase-shift analyses where all "solutions" are recorded at each energy,
and a continuous set of solutions is sought, one at each energy, which
have a smooth energy dependence.  Although the techniques described belo
 
                 Seen Rejected  Percentage
 
Characters       1742        2   99.89
 
 
 
may help in this problem, no exhaustive method is known to the author
except for the prohibitive one of using many starting points equally
spaced on an n-dimensional grid.
 
 
6 . 2 The Ge 1 f and a 1 gor i thm
 
      Relatively few minimization methods are specifically designed for
non-local search in many parameters. Probably the most successful of
the ad hoc stepping methods is that of Gelfandl S) . It is non-local
because it provides a natural way to allow for function increases as
well as decreases in any one step, while tending generally to decrease
the f unc t ion val ue .
 
      The procedure is as follows. From the starting point xo, a local
minimization is begun (for example along the gradient) until the func-
tion differences between steps become small (at the point aO). Then,
 
                        X 1
 
 
 
 
 
 
 
 
 
 
 
 
 
 
 
 
 
 
 
 
 
 
 
 
 
 
 
 
 
 
 
 
 
 
                              X2
 
 
                           1      ?
 
                  aO              a   X
                                 _ 2     _3
 
 
 
going back to the starting point, a "long" random step is taken to the
point xl, and another rough local minimization is performed to reach t
point al (see figure above). Then the so-called "precipitous step" is
taken along a line from aO to al, some distance past al to x2 . Then f'
X2 another rough local minimization is performed, yielding a2, and anol
precipitous step is taken from al past a2 to X3  and the search contin
in this way.
 
      The choice of the "precipitous step" length is important in deter-
mining whether the method will "roll over small ridges, but skirt a hi
mountain", as its authors say it should. But no precise way is given,
except that "the choice of the length of the precipitous step is carri
out experimentally (by trials) and it constitutes an important charact
istic of the function".
 
                 Seen Rejected  Percentage
 
Characters       1332       10   99.25
 
                                - 42 -
 
 
     Moreover, there is no stopping rule, since the method is essenti
ly searching rather than converging.  In practice one usually stops a
a given length of computer time, but one would also stop if the progr
went around in circles repeating itself (which is very possible but n
so easy to detect) or if a predetermined "acceptably small" function
value was attained.  This problem of stopping seems to be common to a
non-local minimization methods.
 
 
6 3  The Goldstein-Price method
 .
 
     Goldstein and Pricel6) have proposed an elegant yet simple metho
for seeking other local minima after one local minimum has been found
It is based on a consideration of the analytic (Taylor series) prope
of the function.  Let us assume that the function can be represented
a Taylor series about a local minimum xl, where the first derivativec
vanish:
                F(x) = F(xl) + 2 (x - xl) G(x -xl) + h.t. .
 
 
Now the higher terms (h.t.), involving third and higher derivatives,
important since these are the terms that will give rise to other locc
minima.  In fact, we seek a way of transforming the function so that
the higher terms remain.  Such a transformed function is Fl such that
 
 
                           2(F(x) - F(Xl))
            Fl(xl,_x) =        T         = 1 + h.t. .
                          (X -Xl) G(x - xl)
 
 
By means of this transformation, we have "removed" the minimum at xl,
and the way is cleared to search for other minima generated by the hi
terms of the expansion about xl.  The method therefore consists of
seeking a local minimum of the function Fl.  (It is required to know
second derivative matrix G at the local minimum xl.)  Since the quad
form (x -xl) G(x - xl) is always positive for positive-definite G, th
function Fl will become negative as soon as an improvement on xl is
found.  Then starting from this improved point, the original functio
can be minimized locally to yield a new, improved local minimum of F
 
     If the minimum value found for Fl is positive, then it may cor-
respond to a new local minimum of F, but not an improvement over xl.
 
                 Seen Rejected  Percentage
 
Characters       1609        2   99.88
 
                               - 43 -
 
 
In this case the procedure may be continued from this new point, fo
a new function F2, related to Fl just as Fl was related to F.  As us
no stopping rule is given by the theory.
 
     The method seems to work in practice, although experience with
is limited and no conditions are known under which it is guaranteed
work.  It is appealing for reasons of its elegance and simplicity, c
could prove to be an important tool in global minimization.
 
 
 
CONCLUSION
7.1  Global strategy
     After having studied the properties of many minimization methoc
we are finally faced with the problem of choosing one of them.  As
have already stated, no one method can be optimum in the sense of b
best for all functions.  And even for one given function, it is unli
to find a method which works well in all regions, far from the minin
as well as near.
 
     All this suggests that we should try to tailor the programs to
function's needs.  If we expect to be doing a lot of minimization,
several programs should be prepared, based on different methods wit
properties suited to different kinds of functions.  Then a decision
to which method to use would depend on a consideration of the parti
function to be minimized.  A decisio tree, enabling the user to ch
a method for his function, has been given in the review of Fletcher
Certainly no two experts would agree on the details of such a logic
diagram, since everyone has his own personal preferences in this fi
but the general idea is a good one and indicates a way in which a
routine could be chosen.
 
     In large problems, the properties of the function may change
drastically from one region to another, so that a good method far f
the minimum, for example, may converge very slowly in the vicinity
the minimum.  In particular, the simplex and Rosenbrock methods are
quite insensitive to the exact shape of the function, and so should
work as well in non-quadratic regions as in quadratic regions.
However, once the region of the minimum is reached, the function sh
 
                 Seen Rejected  Percentage
 
Characters       1633        8   99.51
 
                               _ 44 _
 
 
be reasonably quadratic so that a method with quadratic convergence
should certainly be used.
 
     With the present state of the art, the decision as to which meth
to use, or when to change methods if more than one is used, must be
based on a priori knowledge of the function.  Ultimately one could
imagine designing a super-algorithm which could examine the function
follow the progress of the minimization and choose the best method on
the basis of what it finds out, rejecting those whose progress is too
slow in favour of more suitable techniques.  At present this is some-
times done, but only in a most rudimentary way, in the sense that som
algorithms can tell when they are going astray and can signal this to
main program which can then try a safer method.
 
 
7.2  Computer programs
 
     The purpose of this last section is not to suggest particular
programs which may be available for general use, but rather to indica
in very general terms several ways in which such programs may be org2
nized.  The implementation and use of the techniques described in thc
lectures implies their being programmed for high-speed computers, wit
the program structure depending again on the nature of the problem tc
solved.
 
     One traditional program structure is that of the Zarge autonomoz
progro which as far as possible takes care of all details of initia]
tion, input-output formats, error returns, and other organizational
The function to be minimized is submitted by the user in the form of
subroutine which returns a function value F on each call, depending
the values of the formal parameters x.  Another formal parameter is
flag informing the function subroutine when the first and last calls
being made, so that the subroutine may do any private initialization
printing of final results if necessary.  The starting values and ste
sizes of the function parameters are usually read in by the main pro
on data cards, and there may be more data cards specifying which of
several options are requested if the main program is capable of perf
a variety of different tasks.  Programs organized in this way are po
in laboratories where the principal task to be performed in a typica
 
                 Seen Rejected  Percentage
 
Characters       1792        2   99.89
 
 
job is a single large minimization problem, for then the main program
can easily contain logic permitting change-over from one algorithm to
another in case of failure to converge, and can contain many other
features which relieve the user of a large amount of trivial "house-
keeping" work.
 
     A different approach is that of the smaZZ minimizing subroutine,
designed to save memory space and to allow the user maximum flexibili
In this case the user must write the main program which calls the
minimizer, as well as the function subroutine called by the minimizer
In return for the extra work, he then has complete control over the
organizational details, and the minimizer is more likely to be machin
independent when available in a higher-level language such as FORTRAX
This approach is especially adapted to jobs consisting of a series of
many related minimizations, or when the minimization is only a small
intermediate step in a larger calculation.
 
     A more recent development is that of interactive minimization,
where the user can follow the progress of the search by means of a CF
screen or other output device, and modify the search procedure ac-
cordingly.  Although this appears a potentially powerful technique, i
is rather expensive in terms of real-time computer resources, and I c
skeptical about its ultimate efficiency for two reasons:
 
  i) Computer output devices and human geometric insight are both
     notoriously poor in spaces of high dimensionality, and cases of
     many variables are usually the only ones difficult enough to
     warrant the use of expensive computing techniques.
 
 ii) If a human is able to direct a many-dimensional search more
     efficiently than a computer, it is probably a sign of deficienc
     in the numerical methods used by the computer rather than brilli
     insight on the part of the human.
 
     However, interactive minimization is still relatively new, and
could prove very useful for some problems, including the development
new  algorithms for off-line minimization programs.
 
                 Seen Rejected  Percentage
 
Characters       1676        1   99.94
 
                           - 46 -
                              REFERENCES
(References 18-31 ce not referred to specificaZZy in the textJ but a
added as usefuZ generaZ references.)
 
 
 l)  J. Kowalik and M.R. Osborne, Methods for unconstrained optimizat
        problems (American Elsevier Publishing Co., Inc., New York,
        1968) .
 
 2) H.H. Rosenbrock, An automatic method for finding the greatest or
        least value of a function, Comput. J. 3, 175 (1960).
 
 3)  R. Hooke and T.A. Jeeves, Direct search solution of numerical an
        statistical problems, J. Assoc. Comput. Mach. 8, 212 (1961).
 
 4)  L.C.W. Dixon, Non-linear optimization  (English Universities Pre
        London, 19 7 2 ) .
 
 S)  J.A. Nelder and R. Mead, A simplex method for function minimizat
        Comput . J. 7, 308 (1965) .
 
 6) G.W. Stewart, A modification of Davidon's method to accept dif-
        ference approximations of derivatives, J. Assoc. Comput. Mach
        14, 72 (1967).
 
 7)  R. Fletcher and C.M. Reeves, Function minimization by conjugate
        gradients, Comput . J . 7, 149 (1964) .
 
 8) M.J.D. Powell, An efficient method for finding the minimum of a
        function of several variables without calculating derivatives
        Comput . J. 7, 155 (1964) .
 
 9) L.D. Landau and E.M. Lifshitz, The classical theory of fields
        (Addison-Wesley Publ. Co., Inc., Reading, Mass., 1951) .
 
10)  R. Fletcher and M.J.D. Powell, A rapidly converging descent meth
        for minimization, Comput. J. 6, 163 (1963) .
 
11)  W.C. Davidon, Variance algorithm for minimization, Comput. J. lC
        406 (1968).
 
12) M.J.D. Powell, Rank one methods for unconstrained optimization,
        appearing in Integer and Non-linear Programming, J. Adabie
        editor (North-Holland Publ. Co., Amsterdam, 1970) .
 
13) R. Fletcher, A new approach to variable metric algorithms,
        Comput. J. 13, 317 (1970).
 
14) C.G. Broyden, Quasi-Newton methods and their application to func
        tion minimization, Math. Comput. 21, 368 (1967) .
 
15)  I.M. Gelfand and f.L. Tsetlin, The principle of non-local searc
        in automatic optimization systems, Soviet Phys . Dokl . 6, 192
        (1961) .
 
                 Seen Rejected  Percentage
 
Characters       1634        3   99.82
 
                              - 47 -
 
 
16) A.A. Goldstein and J.F. Price, On descent from local minima, Ma
        Comput . 25, 569 (1971) .
 
17) R. Fletcher, Methods for the solution of optimization problems,
        Comput. Phys. Commun. 3, 159 (1972).
 
18)  M.J.D. Powell, Minimization of functions of several variables,
        appearing in Numerical Analysis, an Introduction, J. Walsh
        editor (Academic Press, Inc., New York, 1966) .
 
19)  M.J. Box, D. Davies, and W.H. Swann, Non-linear optimization
        techniques (Oliver and Boyd, Edinburgh, 1969).
 
20) D.J. Wilde and C.S. Beightler, Foundations of optimization
        (Prentice-Hall, Inc., Englewood Cliffs, N.J., 1967) .
 
21) M.J. Box, A comparison of several current optimization methods
        the use of transformations in constrained problems, Comput.
        9, 67 (1966) .
 
22) R. Fletcher, Function minimization without evaluating derivativ
        a review, Comput . J . 8, 33 (1965) .
 
23) R. Fletcher (editor), Optimization - Symposium of the Institute
        Mathematics and its Applications  (Academic Press, Inc. ?
        iew York, 1969).
 
24) H.A. Spang, A review of minimization techniques for non-linear
        functions, SIAM Rev. 4, 343 (1962).
 
25) M.J.D. Powell, A survey of numerical methods for unconstrained
        optimization, SIAM Rev. 12, 79 (1970) .
 
26) M.J.D. Powell, A method for minimizing a sum of squares of non-
        linear functions without calculating derivatives, Comput. J.
        303 (1965).
 
27) J. Greenstadt, On the relative efficiencies of gradient methods
        Math. Comput . 21, 360 (1967) .
 
28) R.W.H. Sargent and B.A. Murtaugh, Computational experience wit
        quadratically convergent minimization methods, Comput. J.
        185 (1970).
 
29) J.D. Pearson, Variable metric methods of minimization, Comput.
        12, 171 (1969).
 
30) R. Bass, A rank two algorithm for unconstrained minimization,
        Math . Comput . 26, 129 (1972) .
 
31) A.A. Goldstein and J.F. Price, An effective algorithm for minir
        tion, Nurr . Math. 10, 184 (1967) .
 
                 Seen Rejected  Percentage
 
Characters       1582        2   99.87
 
                                - 48 -
 
 
APPENDrX
 
 
           SOME SAMPLE PROBLEMS FOR MINIMIZATION ROUTINES
 
 
     We assemble here a collection of test problems which found to be
useful in verifying and comparing different minimization routines.
Many of these are standard functions upon which it has become
conventional to try all new methods, quoting the performance in the
publication of the algorithm.
 
 
Al   Rosenbrock's curved valley
  .
 
 
                  F(x,y)  lOO(y - x2)2 + (1 - x)2
 
start point:              F(-1.2,1.0) = 24.20
minimum:                  F(l.O,l.O)  = O .
     This narrow, parabolic valley is probably the best known of all 1
cases.  The floor of the valley follows approximately the parabola
y = x2 + 1/200, indicated by the dashed line in the diagram.  In the
cross-hatched area above the dashed line, the covariance matrix is no
positive-definite.  On the dashed line it is singular.  Stepping meth
tend to perform at least as well as gradient methods for this functio
 
[Reference:  Comput. J. 3, 175 (1960).]
 
 
 
 
 
   s 
 
                 Seen Rejected  Percentage
 
Characters        772        3   99.61
 
                                - 49 -
 
 
A2.  Wood's function of four parameters
 
 
  F(Wx,y,z) = lOO(x -w2)2 + (w _l)2 + 9o(    2)2
               + (1- y)2 + lO.l[(x -1)2 + (z -1)2] + 19.8(x -l)(z - 1
start point:            F(-3,    3,-1) = 19192
minimum:                F(l,l,l,l)     = O .
 
     This is a fourth-degree polynomial which is reasonably well-beh
near the minimum, but in order to get there one must cross a rather
four-dimensional "plateau" which often causes minimization algorithm
get "stuck" far from the minimum.  As such it is a particularly good
test of convergence criteria and simulates quite well a feature of m,
physical problems in many variables where no good starting approxima
is known.
 
[Reference:  Unpublished.  See IBM Technical Report No. 320-2949.]
 
 
 
A3.  Powell's quartic function
 
 
     F(w,x,y,z) = (w + lOx)2 + S(y- Z)2 + (X - 2y)4 + lO(w -z)4
start point:               F(3,-1,0,1) = 215
minimum:                   F(O,O,O,O)  = O .
 
     This function is difficult because its matrix of second derivati
becomes singular at the minimum.  Near the minimum the function is gi
by (w +lOx)2 + S(y _Z)2 which does not determine the minimum uniquel
 
[Reference:  Comput. J. 5, 147 (1962).]
 
 
 
A4.  Fletcher and Powell's helical valley
 
 
          F(x,y,z) = lOOz -10(X,y)2 + (x2 + y2 _ l)2 + z2
where               2(x,y) = arctan (y/x) for x > O
                       =  + arctan (y/x) for x< O
 
start point:               F(-l,O,O) = 2500
minimum:                     F(l,O,O) = O .
 
F is defined only for -0.25 <  < 0.75.
 
                 Seen Rejected  Percentage
 
Characters       1091       13   98.81
 
                                - so -
 
 
     This is a curved valley problem, similar to Rosenbrock's, but in
three dimensions.
 
[Reference:  Comput. J. 6, 163 (1963).]
 
 
 
AS. Goldstein and Price function with four minima
 
 
     F(x,y) = (1 + (x + y + 1) 2 * (19-14x + 8x2 _ 14y + 6xy + 3y2)
             * ( 30 + (2x _ 3y) 2 * (18-32x + 12x2 + 48y - 36xy + 27y2
local minima:          F (1. 2, O . 8) = 840
                       F(1.8,0.2)   = 84
                       F(-0.6,-0.4) = 30
global minimum:        F (O ,-1. 0) = 3
 
     This is an eighth-order polynomial in two variables which is well
behaved near each minimum, but has four local minima and is of course
non-positive-definite in many regions.  The saddle point between the t
lowest minima occurs at F(-0.4,-0.6) = 35, making this an interesting
start point.
 
[Reference:  Math. Comp. 25, 571 (1971).]
 
 
 
A6. Goldstein and Price function with many minima
 
 
    F(x,y) = exp 2 (x2 + y2 _ 25)2 + sin4 (4x  3y)  +  1  (2     1  )2
 
 
 
gl obal minimum:             F ( 3, 4 )
      This function has "many" local minima.
[Reference:  Math. Comp. 25, 571 (1971).]
 
 
 
A7. Quadratic function in four parameters
 
 
 
        F(x,y,z,w) = 7l (21x2 + 20y2 + 19z2 _ 14xz - 20yz) + w2
 
                 Seen Rejected  Percentage
 
Characters        839        3   99.64
 
                                 - 51 -
minimum:                    F(O,O,O,O) = O
                            4 1 2
                             1 5 3
covariance matrix:                   6
 
 
                                O O O
 
 
 
     Except for the reasonably strong parameter correlations, this f
tion poses no special problem to any minimization routine.  But the
author has found it useful in debugging programs based on quadratica
convergent methods, since these programs should minimize the functioT
exactly in one iteration.  It is also used to check the calculation c
the covariance matrix.
 
     A variation consists of adding |x| - 1 whenever |x| > 1, and
similarly with the other variables.  This introduces in a reasonably
smooth way terms which alter the quadratic behaviour far from the miT
while leaving it unchanged inside the unit cube, Chus providing a te
for those methods which are supposed to converge to the correct
covariance matrix by updating.
 
 
 
A8.  Chebyquad
 
 
 
               F(x) =    Ti(X ) dx          T (x )
 
                      i=l O              =l
 
 
where Ti(x) are shifted Chebyshev polynomials of degree i;
start point:                 Xj = j/(n + l) .
     This function is designed to have a variable and possibly large
number of parameters, and to resemble functions encountered in actua
practice rather than being contrived to be especially difficult.  Eac
term of F represents the squared difference between the true integra
of a polynomial of degree i and the integral estimated by Chebyshev
(equal-weight) quadrature on n points:
 
 
                      J P(x) dx | P(xj) .
                    O                                   =l
 
                 Seen Rejected  Percentage
 
Characters       1115       15   98.65
 
                                - ;2 -
 
 
The starting values correspond to equally spaced points Xj which is r
too far away from the solution.  Fletcher gives a complete Algol-codec
procedure for this function in the reference quoted below.
 
[Reference:  Comput. J. 8, 33 (1965).]
A9.  Trigonometric functions of Fletcher and Powell
                    n r n                                 -2
                                  (Aij sin Xj + Bij cos Xj)
                i=l  j=l
 
where
 
 
                  E. = '  (A.. sin x . + B.. cos x .) .
                               0 1     0
 
 
 
B   and A.  are random matrices composed of integers between -100 and
 ij      l
 
100;  for j = 1, ..., n:  Xj are any random numbers, - < Xoj < ;
 
start point:         xj = x j + O.l     n < j <
minimum:             F(x = xO) = O .
 
     This is a set of functions of anv number of variables n, where t
minimum is always known in advance, but where the problem can be chan
by choosing different (random) values of the constants Aij, Bij, and .
The difficulty can be varied by choosing larger starting deviations
In practice, most methods find the "right" minimum, corresponding to
x = xo, but there are usually many subsidiary minima.
 
Reference:  Comput. J. , i63 (1963).
 
                 Seen Rejected  Percentage
 
Characters        831       18   97.83
