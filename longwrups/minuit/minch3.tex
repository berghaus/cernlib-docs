%%%%%%%%%%%%%%%%%%%%%%%%%%%%%%%%%%%%%%%%%%%%%%%%%%%%%%%%%%%%%%%%%%%
%                                                                 %
%   MINUIT User Guide -- LaTeX Source                             %
%                                                                 %
%   Chapter 3                                                     %
%                                                                 %
%   The following external EPS files are referenced:              %
%                                                                 %
%   Editor: Michel Goossens / CN-AS                               %
%   Last Mod.:  1 June 1992 10:10 mg                              %
%                                                                 %
%%%%%%%%%%%%%%%%%%%%%%%%%%%%%%%%%%%%%%%%%%%%%%%%%%%%%%%%%%%%%%%%%%%
 
\chapter{How to Use Minuit}

\section{The Function FCN.}
The user must always supply a Fortran subroutine which calculates
the function value to be minimized or analyzed.
\medskip
\Shubr{FCN}{(NPAR,GRAD,FVAL,XVAL,IFLAG,FUTIL)}

\begin{DLtt}{1234567}
\item[{\rm\bf Input parameters}] \mbox{}
\item[NPAR]  number of currently variable parameters.
\item[XVAL]  vector of (constant and variable) parameters.
\item[IFLAG] Indicates what is to be calculated (see example below).
\item[FUTIL] Name of utilitary routine (if needed, it
             must be declared \Lit{EXTERNAL} and provided by the user).
\item[{\rm\bf Output parameters}] \mbox{}
\item[FVAL]  The calculated function value.
\item[GRAD]  The (optional) vector of first derivatives).
\end{DLtt}

Note that when Minuit is being used through an intermediate package such as
HBOOK or PAW, then the \Rind{FCN} may be supplied by the this package.

\begin{XMPt}{Example of \Lit{FCN} routine}
        SUBROUTINE FCN(NPAR,GRAD,FVAL,XVAL,IFLAG,FUTIL)
        IMPLICIT DOUBLE PRECISION (A-H,O-Z)  !  for 32-bit machines
        DIMENSION GRAD(*),XVAL(*)
        EXTERNAL FUTIL   !    (if needed and supplied by user)
C-
        IF (IFLAG .EQ. 1)  THEN
C           read input data,
C           calculate any necessary constants, etc.
        ENDIF
        IF (IFLAG .EQ. 2)  THEN
C           calculate GRAD, the first derivatives of FVAL
C           (this is optional)
        ENDIF
C             Always calculate the value of the function, FVAL,
C             which is usually a chisquare or log likelihood.
C                  Optionally, calculation of FVAL may involve
        FTHEO = FUTIL(....)
C                  It is responsability of user to pass
C                  any parameter values needed by FUTIL,
C                  either through arguments, or in a COMMON block
         IF (IFLAG .EQ. 3)  THEN
C            will come here only after the fit is finished.
C            Perform any final calculations, output fitted data, etc.
        ENDIF
        RETURN
        END
\end{XMPt}

The name of the subroutine may be chosen freely (in documentation we
give it the generic name \Rind{FCN}) and must be declared \Lit{EXTERNAL} in the
user's program which calls Minuit (in data-driven mode) or calls
Minuit subroutines (in Fortran-callable mode).
The meaning of the parameters \Lit{XVAL} is of course defined by
the user, who uses the values of those parameters to calculate his
function value.
The starting values must be specified by the user
(either by supplying parameter
definitions from a file, or typing them at the terminal,
in data-driven mode; or
by calling subroutine \Rind{MNPARM} in Fortran-callable mode),
and later values are determined by Minuit as it searches for the
minimum or performs whatever analysis is requested by the user.
\Rind{FUTIL} represents the name of a function or subroutine which may
be defined and supplied by the user and called from \Rind{FCN}.
If the user does not use the \Rind{FUTIL} feature, the last argument
may be given as zero, but if used, the name of \Rind{FUTIL} must
be declared \Lit{EXTERNAL} and a subprogram of that name must be
supplied at loading time.
 
It is possible, by giving them different names, to analyze several
different \Rind{FCN}s in one job.
However, one analysis must be completed before the next is started.
In order to avoid interference between the analyses of two different
\Rind{FCN}s, the user should call Minuit (in data-driven mode) or \Rind{MNINIT}
(in Fortran-callable mode) each time a new \Rind{FCN} is to be studied.

\section{Running Minuit in Data-driven Mode.}

Minuit can be run in two different modes:
{\bf Data-driven mode}
\index{data driven mode}
\index{mode!data driven}
means that the user drives Minuit with data, either typed
interactively from a terminal or from a data file in batch; and
\index{Fortran-callable mode}
\index{mode!Fortran-callable}
{\bf Fortran-callable mode}
means that Minuit is driven directly from Fortran subroutine
calls, without data.
To some extent, the two modes may also be mixed.
This section describes the first mode, and is valid for both
interactive and batch running.
The differences between interactive and batch are described in
a separate subsection below.
 
\index{data driven mode}
\index{mode!data driven}
In {\bf data-driven mode}, the user must supply,
in addition to the subroutine \Rind{FCN}, a
{\bf main program} which includes the following
statements (the statements in upper case are required, those
given in lower case are optional):

\begin{XMPt}{Example of main program when using Minuit in data driven mode}
      EXTERNAL FCN
      external futil
      call mintio(ird,iwr,isav)
      CALL MINUIT(FCN,futil)
\end{XMPt}

The name of \Rind{FCN} may be chosen freely, and is communicated
to Minuit as its first argument.
\Rind{FUTIL} is the generic name of a function or subroutine which the
user may optionally call from \Rind{FCN}, and if he does call such
a routine, he must declare it external and communicate its
name to Minuit as well.  If \Rind{FUTIL} is not used, then the second
argument may be put equal to \Lit{0}, 
and need not be declared \Lit{EXTERNAL}; if \Rind{FUTIL} is
declared \Lit{EXTERNAL}, it must be supplied in the loading process.

\newpage

\Shubr{MINTIO}{(IREAD,IWRITE,ISAVE)}

\medskip\Action
The purpose of \Rind{MINTIO} is to communicate to Minuit the I/O units.

\begin{DLtt}{1234567}
\item[{\rm\bf Input parameters}] \mbox{}
\item[IREAD]  Fortran unit number for reading (default 5).
\item[IWRITE] Fortran unit number for writing (default 6).
\item[Isave]  Fortran unit number for saving (default 7).
\end{DLtt}

\index{input/output units}

If the default values are acceptable, then it is not necessary to
call \Rind{MINTIO}.
It is the user's responsibility that the I/O units are properly
opened for the appropriate operations.
 
\subsection*{Note}
 
In data-driven mode, that is with \Lit{CALL}~\Rind{MINUIT}, you should
{\bf not call \Rind{MNINIT}}, since Minuit takes care of all
initialization. 
To change unit numbers, call \Rind{MINTIO} 
{\bf before calling \Rind{MINUIT}}.
 
In order that control returns to the user program after
\Lit{CALL MINUIT}, the last command in the corresponding Data Block
should be \Cind[RETurn]{RETURN}.  
If the last command is \Cind{EXIT} or \Cind{STOP},
then Minuit will execute a Fortran \Lit{STOP}, and if the last
command is \Cind{END}, Minuit will read a new Data Block from the current
input unit.
 
\Subsection{10cm}{Data to drive Minuit}
 
In data-driven mode, either interactively or in batch,
Minuit reads the following data provided by the user:

\begin{UL}
\item {\bf Title:} (a string of 50 characters or less)
      which can be chosen freely by the user, to help identify the job.
\item {\bf Parameter definitions:} for each parameter one record giving:
      \begin{OL}
      \item {\bf The parameter number.}
            This is the index in the array \Lit{XVAL} by which the
            user function \Rind{FCN} will access the value of the parameter.
      \item {\bf The parameter name.}
            A string of ten characters to help the user in
            reading the Minuit output.
      \item {\bf The starting value} of the parameter.
      \item {\bf The starting step size,}
             or expected uncertainty in this parameter,
             if it is to be a variable parameter. Otherwise blank or zero if the
             value is to be constant.
      \item [Optional] The {\bf lower bound}
            (limit) below which the parameter value must not vary.
      \item [Optional] The {\bf upper bound}
            (limit) above which the parameter value must not vary.
      \end{OL}
      Normally the user should {\bf not} specify limits on the parameters, that is
      both should be left blank. If one limit is specified, then BOTH must
      be specified. The properties of limits are explained elsewhere
      in this document.
 
      The format of the parameter definitions may be either
      fixed-field (each item in a field of width ten columns),
      or in free-field format.
      In the free-field format, items are separated by blanks or one comma,
      and the parameter name must be given between single quotes.
      The program assumes free-field format if it finds two single
      quotes in the line.
      Parameter names will be blank-padded or truncated to be
      ten characters long.
\item {\bf A blank record:} indicates the end of parameter definitions.
\item If the user \Rind{FCN} reads input data from the same input stream as the
       Minuit data (the default stream is \Lit{UNIT 5}),
       then the \Lit{FCN} data should appear here.
\item {\bf Minuit commands:} these specify actions which should be performed by Minuit.
       Commands must not contain leading or embedded blanks, but may be
       truncated to three characters, and may be given in upper or lower case.
       Some commands have numerical arguments, and these may be given in
       free-field format, separated by blank(s) or one comma\footnote{%
       In older versions of Minuit, there was a special format for the \Cind{MINOs}
       command, when specifying a list of parameters; the new Minuit reads
       the \Cind{MINOs} command with the same free-field format as the other
       commands, so if parameter numbers are specified, they must now
       be separated by a blank or comma.}.
       The list of recognized commands is given and explained below.
       The command \Cind{HELP} causes Minuit to write to the output stream a list
       of currently recognized commands.
       The command \Cind{HELP SHOw} lists the available \Cind{SET} 
       and \Cind{SHOw} commands.
\end{UL}

Any or all of the above data read by Minuit
can reside on one or more different files,
and Minuit can be instructed
to switch to reading a different file with the \Cind[SET INPut]{SET INPUT} command.
Optionally, the {\bf title} record may be preceeded by a record
beginning with the characters \Cind[SET TITle]{SET TITLE}, and the
{\bf parameter definitions} may be preceeded by a record
beginning with the characters \Cind[PARameters]{PARAMETERS}.
It is in fact recommended always to include these optional
records when preparing a data file, since the file can then be
read at any time (not just at the beginning of a Minuit run)
and will always be interpreted correctly by Minuit.
 
\begin{XMPt}{Example of a typical Minuit data set}
SET TITLE
Fit to time distribution of K decays, Expt NA94
PARAMETERS
1 'Real(X)'  0.  .1
2 'Imag(X)'  0.  .1
5 'Delta M'  .535   .01
10 'K Short LT'  .892
11 'K Long LT'   518.3
 
fix 5
migrad
set print 0
minos
restore
migrad
minos
fix 5
set param  5   0.535
contour 1 2
stop
\end{XMPt}

\subsection{Batch and interactive running.}

\index{batch run}
\index{interactive session}
In its initialization phase, Minuit attempts to determine whether
or not it is running interactively, by calling the logical function
\Rind{INTRAC}, a routine in the CERN Program Library which can
be provided for all commonly used computers.
For our purposes, we define ``running interactively'' as meaning that
input is coming from a terminal under the control of an intelligent
being, able to make decisions based on the output he receives at
the terminal. It is not always easy for \Rind{INTRAC} to know whether this
is the case, so, depending on your operating system, Minuit can be
fooled in certain cases. When this happens, the user can always override
the beliefs of \Rind{INTRAC} with the commands \Cind{SET BATch} and
\Cind{SET INTeractive}. 
The command \Cind{SHOw INTeractive} informs the user of the current mode.
 
According to whether or not it believes it is running interactively,
Minuit behaves differently in the following ways:

\begin{UL}
\item If interactive, the user is prompted before each data record is read.
\item If interactive, Minuit recovers from many error conditions
      and prompts the user to enter correct data or to specify
      additional required input.
      If the same error conditions occur in batch mode, the program either
      exits (if no corrective action seems possible) or ignores the incorrect
      data (for example, a command it cannot interpret) and continues.
\item The default page size for output is a typical terminal dimension
      (80 by 24) if interactive, and a typical printed page size (120 by 56)
      if batch, but these can be overridden with the commands 
      \Cind[SET WIDthpage]{SET WIDth} and \Cind[SET LINesperpage]{SET LINes}.
\end{UL}

When an interactive user requests Minuit to read
further input from an external file (the \Cind{SET INPut} command),
then further input is considered to be temporarily
in batch mode, until input reverts to the primary input stream.

\Section{5cm}{Running Minuit in Fortran-callable mode.}

The following Minuit subroutines are provided in order to allow
the user to communicate with Minuit and perform all Minuit
functions (define parameters, execute commands, etc.) directly
from Fortran through subroutine calls.
In the following list of subroutines, output arguments are indicated
by appending a star \Lit{*} to its name.
It should also be noted that for the Double Precision version of
Minuit (recommended for all 32-bit machines such as IBM, Vax,
Unix workstations, etc.), all the \Lit{REAL} arguments given below must be
declared \Lit{DOUBLE PRECISION}.

\Subsection{4cm}{Initialize Minuit}
 
\Shubr{MNINIT}{(IRD,IWR,ISAV)}
 
\begin{DLtt}{123456}
\item[{\rm\bf Input parameters:}]
\item[IRD]  Unit number for input to Minuit.
\item[IWR]  Unit number for output from Minuit.
\item[ISAV] Unit number for use of the SAVE command.
\end{DLtt}
 
\Subsection{4cm}{Specify a title for a problem}
 
\Shubr{MNSETI}{(CTITLE)}
 
\begin{DLtt}{123456}
\item[{\rm\bf Input parameter:}]
\item[CTITLE] Character string of up to 50 characters containing
              an identification text for the present job or fit.
\end{DLtt}
 
\Subsection{4cm}{Define a parameter}
 
\Shubr{MNPARM}{(NUM,CHNAM,STVAL,STEP,BND1,BND2,IERFLG*)}
 
\begin{DLtt}{123456}
\item[{\rm\bf Input parameters:}]
\item[NUM]    Parameter number as referenced by user in \Rind{FCN}.
\item[CHNAM]  Character string of up to 10 characters containing
              the name which the user assigned to the given parameter.
\item[STVAL]  Starting value
\item[STEP]   Starting step size or approximate parameter error.
\item[BND1]   Lower bound (limit) on parameter value, if any (see below).
\item[BND2]   Upper bound (limit) on parameter value, if any (see below).
\item[{\rm\bf Output parameter:}]
\item[IERFLG] Error return code: \Lit{0} if no error, \Lit{>0} if request failed.
\end{DLtt}
 
If \Lit{BND1=BND2=0.}, then the parameter is considered unbounded, which
is recommended unless limits are needed to make things behave well.
 
\Subsection{4cm}{Execute a Minuit command}
 
\Shubr{MNEXCM}{(FCN,CHCOM,ARGLIS,NARG,IERFLG,FUTIL)}
 
\begin{DLtt}{123456}
\item[{\rm\bf Input parameters:}]
\item[FCN]    Name fo the function being analyzed (to be declared \Lit{EXTERNAL})
\item[CHCOM]  Character string containing the name of the Minuit
              command to be executed (see below).
\item[ARGLIS] Array of dimension \Lit{MAXARG}, containing the numeric arguments 
              to the command (if any),
\item[NARG]   Number of arguments specified (\Lit{NARG}$\leq${MAXARG}),
\item[FUTIL]  Name fo a function called by \Rind{FCN} (or \Lit{=0} if not used).
              If used this function must be declared \Lit{EXTERNAL}.
\item[{\rm\bf Output parameter:}]
\item[IERFLG] Error return code: \Lit{0} if the command was executed normally, 
              \Lit{>0} otherwise.
\end{DLtt}
 
Executing a command by calling \Rind{MNEXCM} has exactly the same
effect as reading the same command in data-driven mode, except that
a few commands would make no sense and are not available in
Fortran-callable mode (e.g. \Cind[SET INPut]{SET INPUT}).
The other difference is that {\bf control always returns to the
calling routine from \Rind{MNEXCM}}, even after commands \Cind{END},
\Cind{EXIT}, and \Cind{STOP}.
 
\Subsection{4cm}{Get the current value of a parameter}

This routine is the inverse of \Rind{MNPARM} and
can for instance be used after a fit.

\Shubr{MNPOUT}{(NUM,CHNAM*,VAL*,ERROR*,BND1*,BND2*,IVARBL*)}
 
\begin{DLtt}{123456}
\item[{\rm\bf Input parameter:}]
\item[NUM]    Parameter number as referenced by user in \Rind{FCN} and
              about which information is required.
\item[{\rm\bf Output parameters:}]
\item[CHNAM]  Character string of up to 10 characters containing
              the name which the user assigned to the given parameter.
\item[VAL]    Current parameter value (fitted value if fit has converged),
\item[ERROR]  Current estimate of parameter uncertainty (or zero if constant)
\item[BND1]   Lower limit on parameter value, if any (otherwise zero).
\item[BND2]   Upper limit on parameter value, if any (otherwise zero).
\item[IVARBL] Internal parameter number if parameter is variable, or
              zero if parameter is constant, or negative if parameter is undefined.
\end{DLtt}
 
\Subsection{4cm}{Get the current status of minimization}

\Shubr{MNSTAT}{(FMIN*,FEDM*,ERRDEF*,NPARI*,NPARX*,ISTAT*)}
 
\begin{DLtt}{123456}
\item[{\rm\bf Output parameters:}]
\item[FMIN]   The best function value found so far
\item[FEDM]   The estimated vertical distance remaining to minimum
\item[ERRDEF] The value of \Lit{UP} defining parameter uncertainties
\item[NPARI]  The number of currently variable parameters
\item[NPARX]  The highest (external) parameter number defined by user
\item[ISTAT]  A status integer indicating how good is the covariance matrix:
              \begin{DLtt}{1}
                \item[0] Not calculated at all
                \item[1] Diagonal approximation only, not accurate
                \item[2] Full matrix, but forced positive-definite
                \item[3] Full accurate covariance matrix (After \Cind[MIGrad]{MIGRAD}, 
                         this is the indication of normal convergence.)
              \end{DLtt}
\end{DLtt}
 
\Subsection{4cm}{Get the current value of the covariance matrix}

\Shubr{MNEMAT}{(EMAT*,NDIM)}
 
\begin{DLtt}{123456}
\item[{\rm\bf Input parameter:}]
\item[NDIM]   Integer variable specifying the number of rows and columns
              the suer has reserved in \Lit{EMAT} to store the matrix elements.
              \Lit{NDIM} should be at least as large as the number of parameters 
              variable at the time of the call, otherwise the user will get
              only part of the full matrix.
\item[{\rm\bf Output parameter:}]
\item[EMAT]   Array declared as \Lit{DIMENSION EMAT(NDIM,NDIM)} which
              is to be filled with the (external) covariance matrix.
\end{DLtt}
 
\Subsection{4cm}{Access current parameter errors}

\Shubr{MNERRS}{(NUM,EPLUS*,EMINUS*,EPARAB*,GLOBCC*)}
 
\begin{DLtt}{123456}
\item[{\rm\bf Input parameter:}]
\item[NUM]    Parameter number. 
              If \Lit{NUM>0}, this is taken to be an external parameter number; 
              if \Lit{NUM<0}, it is the negative of an internal parameter number.
\item[{\rm\bf Output parameters:}]
\item[EPLUS]  The positive \Cind[MINOs]{MINOS} error of parameter \Lit{NUM}.
\item[EMINUS] The negative \Cind[MINOs]{MINOS} error (a negative number).
\item[EPARAB] The ``parabolic'' parameter error, from the error matrix.
\item[GLOBCC] The global correlation coefficient for parameter \Lit{NUM}.
              This is a number between zero and one which gives the correlation
              between parameter \Lit{NUM} and that linear combination of all other
              parameters which is most strongly correlated with \Lit{NUM}.
\end{DLtt}
 
Note that this call does not cause the errors to be
calculated, it merely returns the current existing values.
If any of the requested values has not been calculated, or has
been destroyed (for example, by a redefinition of parameter values)
\Rind{MNERRS} returns a value of zero for that argument.
Thus the call to \Rind{MNERRS} will normally follow the execution of
commands \Cind[MIGrad]{MIGRAD}, \Cind[HESse]{HESSE}, 
\Cind{MNContour}, and/or \Rind[MINOs]{MINOS}.
 
\Subsection{4cm}{Find a function contour with the MNContour method}

\Shubr{MNCONT}{(FCN,NUM1,NUM2,NPT,XPT*,YPT*,NFOUND*,FUTIL)}
 
\begin{DLtt}{123456}
\item[{\rm\bf Input parameters:}]
\item[FCN]    Name of the function being treated (to be declared \Lit{EXTERNAL})
\item[NUM1/2] Parameter numbers with respect to which the contour
              is to be determined (external).
\item[NPT]    The number of points required on the contour (\Lit{>4}).
\item[FUTIL]  Name of a function called by \Rind{FCN} (or =0 if not used).
              If used this function must be declared \Lit{EXTERNAL}.
\item[{\rm\bf Output parameters:}]
\item[XPT]    Array of x-coordinates of contour points with 
              values for parameter \Lit{NUM1}.
              It must be declared with a \Lit{DIMENSION XPT(NPT)}.
\item[YPT]    Array of y-coordinates of contour points with
              values for parameter \Lit{NUM2}.
              It must be declared with a \Lit{DIMENSION YPT(NPT)}.
\item[NFOUND] The number of points actually found on the contour.
              If all goes well, this will be equal to \Lit{NPT}, but it can be
              negative (if the input arguments are not valid), or zero if
              less than four points have been found, or less than \Lit{NPT} if the
              program could not find \Lit{NPT} points.
\end{DLtt}
 
Note that alternatively \Cind{MNContour} can be calculated
by calling \Rind{MNEXCM} to issue the \Cind{MNContour} command,
but then the user does not have Fortran access to the actual
point coordinates \Lit{XPT} and \Lit{YPT}.
 
\Subsection{4cm}{Switch to command-reading mode}

This facility can be useful when one wants to continue interactively.

\Shubr{MNINTR}{(FCN,FUTIL)}
 
\begin{DLtt}{123456}
\item[{\rm\bf Input parameters:}]
\item[FCN]    Name of the function being treated (to be declared \Lit{EXTERNAL})
\item[FUTIL]  Name of a function called by \Rind{FCN} (or \Lit{=0} if not used).
              If used this function must be declared \Lit{EXTERNAL}.
\end{DLtt}
 
The call to \Rind{MNINTR} will cause Minuit to read commands from
the unit \Lit{IRD} (originally specified by the user in his call to \Rind{MNINIT},
\Lit{IRD} is usually 5 by default,
which in turn is usually the terminal by default).
Minuit then reads and executes commands until it encounters
a command \Cind{END}, \Cind{EXIT}, \Cind{RETurn}, or \Cind{STOP}, 
or an end-of-file on input
(or an unrecoverable error condition while reading
or trying to execute a command), in which case control returns
to the program which called \Rind{MNINTR}.
