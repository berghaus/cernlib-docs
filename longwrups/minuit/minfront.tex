%%%%%%%%%%%%%%%%%%%%%%%%%%%%%%%%%%%%%%%%%%%%%%%%%%%%%%%%%%%%%%%%%%%
%                                                                 %
%   MINUIT - Reference Manual -- LaTeX Source                     %
%                                                                 %
%   Front Material: Title page,                                   %
%                   Copyright Notice                              %
%                   Preliminary Remarks                           %
%                   Table of Contents                             %
%   EPS file      : cern15.eps, cnastit.eps                       %
%                                                                 %
%   Editor: Michel Goossens / CN-AS                               %
%   Last Mod.:  1 Apr. 1992 17:00 mg                              %
%                                                                 %
%%%%%%%%%%%%%%%%%%%%%%%%%%%%%%%%%%%%%%%%%%%%%%%%%%%%%%%%%%%%%%%%%%%

%%%%%%%%%%%%%%%%%%%%%%%%%%%%%%%%%%%%%%%%%%%%%%%%%%%%%%%%%%%%%%%%%%%%
%    Tile page                                                     %
%%%%%%%%%%%%%%%%%%%%%%%%%%%%%%%%%%%%%%%%%%%%%%%%%%%%%%%%%%%%%%%%%%%%
\def\Ptitle#1{\special{ps: /Printstring (#1) def}
\epsfbox{/user/goossens/cnasall/cnastit.eps}}
 
\begin{titlepage}
\vspace*{-23mm}
\mbox{\epsfig{file=/usr/local/lib/tex/ps/cern15.eps,height=30mm}}
\hfill
\raise8mm\hbox{\Large\bf CERN Program Library Long Writeup D506}
\hfill\mbox{}
\begin{center}
\mbox{}\\[10mm]
\mbox{\Ptitle{MINUIT}}\\[1cm]
{\Large Function Minimization and Error Analysis}\\[1cm]
{\LARGE Reference Manual}\\[2cm]
{\LARGE Version 92.1 (March 1992)}\\[3cm]
{\Large Application Software Group}\\[1cm]
{\Large Computing and Networks Division}\\[2cm]
\end{center}
\vfill
\begin{center}\Large CERN Geneva, Switzerland\end{center}
\end{titlepage}

%%%%%%%%%%%%%%%%%%%%%%%%%%%%%%%%%%%%%%%%%%%%%%%%%%%%%%%%%%%%%%%%%%%%
%    Copyright  page                                               %
%%%%%%%%%%%%%%%%%%%%%%%%%%%%%%%%%%%%%%%%%%%%%%%%%%%%%%%%%%%%%%%%%%%%
\thispagestyle{empty}
\framebox[\textwidth][t]{\hfill\begin{minipage}{0.93\textwidth}%
\vspace*{3mm}\begin{center}\Large\bf Copyright Notice\end{center}
\parskip\baselineskip
{\bf MINUIT -- Function Minimization and Error Analysis}
 
CERN Program Library Entry {\bf D506}
 
\copyright{} Copyright CERN, Geneva 1992
 
Copyright and any other appropriate legal protection of these
computer programs and associated documentation reserved in all
countries of the world.
 
These programs or documentation may not be reproduced by any
method without prior written consent of the Director-General
of CERN or his delegate.
 
Permission for the usage of any programs described herein is
granted apriori to those scientific institutes associated with
the CERN experimental program or with whom CERN has concluded
a scientific collaboration agreement.
 
Requests for information should be addressed to:
\vspace*{-.5\baselineskip}
\begin{center}
\tt\begin{tabular}{l}
CERN Program Library Office              \\
CERN-CN Division                         \\
CH-1211 Geneva 23                        \\
Switzerland                              \\
Tel.      +41 22 767 4951                \\
Fax.      +41 22 767 7155                \\
Bitnet:   CERNLIB@CERNVM                 \\
DECnet:   VXCERN::CERNLIB (node 22.190)  \\
Internet: CERNLIB@CERNVM.CERN.CH
\end{tabular}
\end{center}
\vspace*{2mm}
\end{minipage}\hfill}%end of minipage in framebox
\vspace{6mm}
 
{\bf Trademark notice: All trademarks appearing in this guide are acknowledged as such.}
\vfill
\begin{tabular}{l@{\quad}l}
{\em Contact Person\/}:        & Fred James /CN      \\[1mm]
{\em Technical Realization\/}: & Michel Goossens /CN \\[5mm]
{\em First combined edition May 1992} 
\end{tabular}
\newpage
 
%%%%%%%%%%%%%%%%%%%%%%%%%%%%%%%%%%%%%%%%%%%%%%%%%%%%%%%%%%%%%%%%%%%%
%    Introductory material                                         %
%%%%%%%%%%%%%%%%%%%%%%%%%%%%%%%%%%%%%%%%%%%%%%%%%%%%%%%%%%%%%%%%%%%%
\pagenumbering{roman}
\setcounter{page}{1}

\chapter*{Foreword}

\section*{What Minuit is intended to do.}

Minuit is conceived as a tool to find the minimum value of a
multi-parameter function and analyze the shape of the function
around the minimum. The principal application is foreseen for
statistical analysis, working on chisquare or
log-likelihood functions,
to compute the best-fit parameter values and uncertainties,
including correlations between the parameters.
It is especially suited to handle difficult problems, including those
which may require guidance in order to find the correct solution.
 
\section*{What Minuit is not intended to do.}

Although Minuit will of course solve easy problems faster than complicated
ones, it is not intended for the repeated solution of identically parametrized
problems (such as track fitting in a detector) where a specialized
program will in general be much more efficient.
 
\section*{Further remarks.}
 
This manual consists of three parts:
\begin{OL}
\item A reference guide explaining the concepts and
      how to use Minuit for maximum benefit.
\item A tutorial about function minimization
\item A tutotial on the interpretation of the error of the
      parameters given by Minuit
\end{OL}

In this manual
examples are in {\tt monotype face} and strings to be input by the user 
are {\tt\underline{underlined}}.
In the index the page where a routine is defined is in {\bf bold},
page numbers where a routine is referenced are in normal type.
In the description of the routines a \Lit{*} following
the name of a parameter indicates that this is an {\bf output} parameter.
If another \Lit{*} precedes a parameter in the calling sequence, the
parameter in question is both an {\bf input} and {\bf output} parameter.

This document has been produced using \LaTeX\cite{bib-LATEX}
with the \Lit{cernman} style file, developed at CERN. 
A PostScript file \Lit{minuit.ps}, containing a complete printable version
of this manual, can be obtained at CERN by anonymous ftp as follows
(commands to be typed by the user are underlined):

\vspace*{3mm} 
\begin{tabular}{@{\hspace{12mm}}>{\tt}l}
\underline{ftp asis01.cern.ch}\\
Trying 128.141.8.104...\\
Connected to asis01.cern.ch.\\
220 asis01 FTP server (SunOS 4.1) ready.\\
Name (asis01:username): \underline{anonymous}\\
Password: \underline{your\_{}mailaddress}\\
ftp> \underline{cd doc/cernlib}\\
ftp> \underline{get minuit.ps}\\
ftp> \underline{quit}\\
\end{tabular}

%\section*{Acknowledgements}
%
%Many people have contributed to Minuit
%through discussions, comments and suggestions.
% 
%Mat Roos .....

%%%%%%%%%%%%%%%%%%%%%%%%%%%%%%%%%%%%%%%%%%%%%%%%%%%%%%%%%%%%%%%%%%%%
%    Tables of contents ...                                        %
%%%%%%%%%%%%%%%%%%%%%%%%%%%%%%%%%%%%%%%%%%%%%%%%%%%%%%%%%%%%%%%%%%%%
\newpage
\tableofcontents
\newpage
\listoffigures
\listoftables
