%	$Id$	
%%%%%%%%%%%%%%%%%%%%%%%%%%%%%%%%%%%%%%%%%%%%%%%%%%%%%%%%%%%%%%%%%%%
%                                                                 %
%   HBOOK User Guide -- LaTeX Source                              %
%                                                                 %
%   Chapter 2                                                     %
%                                                                 %
%   The following external EPS files are referenced:              %
%                                                                 %
%   Editor: Michel Goossens / CN-AS                               %
%   Last Mod.: 25 May 1993 15:20 mg                               %
%                                                                 %
%%%%%%%%%%%%%%%%%%%%%%%%%%%%%%%%%%%%%%%%%%%%%%%%%%%%%%%%%%%%%%%%%%%
 
\chapter{One and two dimensional histograms -- Basics}
\label{HFUNDAMS}
%  ==================================================================

\section{Booking}
\label{HBOOKING}
\subsection{One-dimensional case}

\Shubr{HBOOK1}{(ID,CHTITL,NX,XMI,XMA,VMX)}
\index{one-dimensional histogram}
\index{histogram!identifier}
\index{histogram!title}
\index{VMX@{\tt VMX}}
\index{packing}

\Action Books a one-dimensional histogram.

\Idesc
\begin{DLttc}{1234567}
\item[ID] histogram identifier, integer non zero
\item[CHTITL] histogram title (character variable or constant up to 80
              characters)
\item[NX] number of channels
\item[XMI] lower edge of first channel
\item[XMA] upper edge of last channel
\item[VMX] upper limit of single channel content (see below).\\
           \Lit{VMX=0.} means 1 word per channel (no packing).
\end{DLttc}                                 

\subsubsection*{Special values:}
\begin{UL}
\item If \Lit{XMA}$\leq$\Lit{XMI}, origin and binwidth are
      calculated automatically, and one word is reserved per channel.
\item Zero (0) is an illegal histogram identifier.
\item If the histogram \Lit{ID} already exists it will be deleted and
      recreated with the new specifications. A warning message is printed.
\item \Lit{VMX} is used to compute the number
      of bits to be allocated per histogram channel.
      If \Lit{VMX<1.} then one full word is reserved per channel.
      When filling a histogram with weights the latter are
      truncated to the nearest integer unless one full word is
      reserved per channel (i.e., \Lit{VMX = 0.}).
      Filling with negative weights will give meaningless results
      unless one word per channel has been allocated.\\
      Automatic calculation of limits (\Lit{XMA}$\leq$\Lit{XMI})
      forces one word per channel.
\end{UL}

\subsection{Two-dimensional case}

\Shubr{HBOOK2}{(ID,CHTITL,NX,XMI,XMA,NY,YMI,YMA,VMX)}

\index{two-dimensional histogram}
\Action Books a two-dimensional histogram.
\Idesc
\begin{DLttc}{1234567}
\item[ID] histogram identifier, integer
\item[CHTITL] histogram title  
              (character variable or constant up to 80 characters)
\item[NX] number of channels in X
\item[XMI] lower edge of first X channel
\item[XMA] upper edge of last X channel
\item[NY] number of channels in Y
\item[YMI] lower edge of first Y channel
\item[YMA] upper edge of last Y channel
\item[VMX] maximum population to store in 1 cell.
\end{DLttc}

\Remarks
\begin{UL}
\item Similar to HBOOK1, apart from automatic binning.
\item By default, a 2-dimensional histogram will be
      printed as a scatterplot.
\item If the option \Lit{TABL} is selected via
      \Lit{CALL HIDOPT(ID,'TABL')}\Iind{TABL}
      the 2-dimensional histogram will be printed as a table.
\item When editing the table, the number of columns \Lit{NCOL} used to
      write the content of one cell depends on the value of \Lit{VMX}
      as follows \Lit{NCOL = ALOG10(VMX) + 2}.
      When \Lit{VMX} is zero, the contents is printed in
      10 columns in floating point format (including sign). If
      necessary, all contents are multiplied by a power of 10,
      this number being reported at the bottom of the table.
\end{UL}

%  ==============================================

\section{Filling}
\label{HFILLSEC}

\Shubr{HFILL}{(ID,X,Y,WEIGHT)}
\index{weight}

\Action 
Fills a 1-dimensional or a 2-dimensional histogram.
The channel which contains the value X and for two-dimensionals the cell that
contains the point \Lit{(X,Y)}, gets its contents increased by
\Lit{WEIGHT}.
All booked projections, slices, bands, are filled as well.

\Idesc
\begin{DLttc}{1234567}
\item[ID] histogram identifier
\item[X] value of the abscissa
\item[Y] value of the ordinate
\item[WEIGHT] event weight (positive or negative)
\end{DLttc}
\Remarks
\begin{UL}
\item If one full word per channel is reserved at booking time,
      \Lit{WEIGHT} is taken with its floating point value.
      In case of packing (i.e. more than one channel
      per word), \Lit{WEIGHT} must be
      positive and will be truncated to the nearest integer
      (\Lit{WEIGHT<0} will give meaningless results)
\item See section \ref{HOTHFILL} on page \pageref{HOTHFILL}
      for alternative filling routines.
\end{UL}
 
\section{Editing}
\label{HEDITSEC}
 
\Shubr{HISTDO}{ }
\index{print}
 
\Action
Outputs all booked histograms to the line printer.
An index is printed at
the beginning specifying the characteristics and storage use of each
histogram.
 
\Remark
\begin{UL}
\item If a histogram is empty, a message declares this condition, and the
      histogram is not printed.
\end{UL}
 
\Shubr{HPRINT}{(ID)}
\index{line printer}
 
\Action Outputs a given histogram to the line printer.
 
\begin{DLttc}{1234567}
\item[ID]  Histogram identifier.
\end{DLttc}
 
\Remarks
\begin{UL}
\item \Lit{CALL\ \Rind{HPRINT}(0)} is equivalent to
      \Lit{CALL\ \Rind{HISTDO}} apart from not printing the index
\item When a histogram is empty a message declares this
      condition and the histogram is not printed.
\end{UL}
\medskip
Some available booking options are shortly listed below.
For a full description see chapter~\ref{HEDITING}.
 
\begin{UL}
\item Creation of histograms with non-equidistant bins
\item Creation of profile histograms
\item Rounded scale
\item Projections and slices of 2-dimensional histograms
\item More statistical information
\item Comprehensive booking and filling with
      user-supplied functions of one or two real variables.
\item Dynamic creation of ordinary Fortran arrays (\Rind{HARRAY})
\end{UL}
 
%\finalnewpage

\section{Copy, rename, reset and delete}
\label{HCOREDEL}
 
\Shubr{HCOPY}{(ID1,ID2,CHTITL)}
\index{histogram!copy}

\Action Generates histogram \Lit{ID2}
as a copy of \Lit{ID1}, apart from the title.

\Idesc
\begin{DLttc}{1234567}
\item[ID1] existing identifier
\item[ID2] non existing identifier
\item[CHTITL] new title. \Lit{CHTITL=' '} means that the old title is kept.
\end{DLttc}

\Shubr{HCOPYR}{(ID1,ID2,CHTITL,IBINX1,IBINX2,IBINY1,IBINY2,CHOPT)}
\index{histogram!copy_range}

\Action Generates histogram \Lit{ID2}
as a copy of a range of the channels of \Lit{ID1}, and optionally
copies the stored errors on the channels.
If \Lit{ID2} already exists, it is deleted and re-created.

The new histogram is created with the same bin width in \Lit{X} (and
\Lit{Y}, for 2-D histograms) as in the original histogram.  The bin
number range is allowed to exceed the range of the original histogram,
in which case the new histogram will contain bins with zero contents.
This possibility is to allow users to copy a histogram into one of a
larger scale.

\Idesc
\begin{DLttc}{1234567}
\item[ID1] Existing identifier.
\item[ID2] New identifier.
\item[CHTITL] New title. \Lit{CHTITL=' '} means that the old title is kept.
\item[IBINX1] Bin in \Lit{X} from which to start the channel copy.
\item[IBINX2] Bin in \Lit{X} on which to end the channel copy.
\item[IBINY1] Bin in \Lit{Y} from which to start the channel copy
              (2-D histograms only).
\item[IBINY2] Bin in Y on which to end the channel copy
              (2-D histograms only).
\item[CHOPT]  \Lit{CHOPT='E'} causes the stored errors on each bin to be
              be copied as well. 
\end{DLttc}

 
\Shubr{HRENID}{(IDOLD,IDNEW)}
\index{histogram!rename}
 
\Action Renames a histogram or Ntuple.

\Idesc
\begin{DLttc}{MMMMMM}
\item[IDOLD]  Old histogram identifier.
\item[IDNEW]  New histogram identifier.
\end{DLttc}

\Shubr{HRESET}{(ID,CHTITL)}
\index{histogram!reset}\index{Reset!histogram}
\index{Ntuple!reset}\index{Reset!Ntuple}
 
\Action Resets the contents of all channels of a histogram
(and its projections) or Ntuple to zero and changes optionally the title.

\Idesc
\begin{DLttc}{1234567}
\item[ID] identifier of a histogram.
          \Lit{ID=0} resets all existing histogram contents.
\item[CHTITL] new title.
          \Lit{CHTITL=' '} means that the old title is kept.
\end{DLttc}
 
\Shubr{HDELET}{(ID)}
\index{histogram!deletion}
 
\Action Deletes a histogram and releases the corresponding storage space.
 
\Idesc
\begin{DLttc}{1234567}
\item[ID] identifier of a histogram. \Lit{ID=0} deletes all existing histograms.
\end{DLttc}

 
See section \ref{HSIMPLEXA} in the introductory chapter for a simple
example of how to book, fill and print histograms.

\endinput
