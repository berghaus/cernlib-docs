%%%%%%%%%%%%%%%%%%%%%%%%%%%%%%%%%%%%%%%%%%%%%%%%%%%%%%%%%%%%%%%%%%%%%%%%%%%%%%%%
%                                                                              %
%   HIGZ  User Guide -- LaTeX Source                                           %
%                                                                              %
%   Chapter: HIGZ inside a Motif application                                   %
%                                                                              %
%   Editor: Olivier Couet / CN-AS                                              %
%   Last Mod.: 25 April 1994 oc                                                  %
%                                                                              %
%%%%%%%%%%%%%%%%%%%%%%%%%%%%%%%%%%%%%%%%%%%%%%%%%%%%%%%%%%%%%%%%%%%%%%%%%%%%%%%%
\chapter{How to access HIGZ or HPLOT high level graphics with Motif}

People who have a Motif based application (concerning the user interface)
can open one (or several) graphical window(s) (drawing area) managed by
HIGZ. They mainly have to follow some strict (but well defined) rules and
call the specific HIGZ/X11 routine ``ixmotif'', especially written for
that purpose.

To follow these rules, that we will describe in detail in this section,
gives the application-programmer the possibility to benefit from the high
level graphics output provided by the HIGZ or HPLOT packages with a
minimum effort. (Note that to achieve the same level of graphical output
directly with X11 and Motif would be very painful). In particular, these
rules are used by KUIP-Motif, the high level CERN-written user interface
package, whose Motif interface is used, for instance, by the well-known
CERN written application Paw++ and Geant++.

The following is addressed to people who have some notion of Motif and
HIGZ/HPLOT programming.


1-  Application Main Program
____________________________

As HBOOK, HIGZ and HPLOT are all FORTRAN packages it is easier to write
the application main program (which will give control to the Motif main
loop of events at the end) in FORTRAN. The structure (skeleton) of the
application main program is:

1) Initialize HBOOK,
2) initialize a Motif application (XtVaAppInitialize), build the user
   interface and initialize the communication with HIGZ and HPLOT
   for X11 outputs by creating a ``drawing area'' (call ``ixmotif''),
3) initialize HIGZ,
4) initialize HPLOT by using the HIGZ/Motif ``specific'' workstation type
   999 (this means that no HIGZ graphics window will be opened but the
   Motif drawing area which has just been created before will be used),
5) and finally, give control to the Motif main loop of events
   (XtAppMainLoop).

N.B. ``ixmotif'' has to be called before HIGZ initialization.



2- ixmotif
__________

This HIGZ/X11 user callable function allows a Motif-based application
to pass to HIGZ the address of three user written routines:

   1) routine which open a Motif drawing area which will be used by HIGZ,
   2) initialization routine (in particular one has to add a callback
      for resizing the drawing area),
   3) routine which close the Motif drawing area.

The ``open'' routine has to return the ``Window identifier'' of the Motif
drawing area, so that HIGZ is able to use it (instead of opening a new
window).

The calling sequence (C code) of this routine is:

    ixmotif( Display *dsp, Window (*open)(), void (*init)(), void (*close)() )

Display *dsp is the display resulting from ``XtVaAppInitialize''.



3- Example (application ``Hmtest'')
___________________________________


The following is the code of a very basic Motif application which opens:

- a bulletin board with 3 buttons: ``NULL'', ``NEXT'' and ``Close'',
- a graphics window for HIGZ/HPLOT graphics.

Pressing on the ``NULL'' button executes the HPLOT routine HPLNUL.
Pressing on ``NEW'' executes the HPLOT routine HPLNEW. Pressing on ``Close''
terminates the application and closes all windows.

application main program (in FORTRAN)
------------------------------------

**************  START  ***********************
*
      PROGRAM KMTEST_MAIN
*
      PARAMETER (NWPAW=2000000)
*
      COMMON /PAWC/ PAWCOM(NWPAW)
*
*
*           Initialize HBOOK
*
      CALL HLIMIT(NWPAW)
*
*           Initialize Motif drawing area
*
      CALL init_motif
*
*           Initialize HIGZ
*
      NWHIGZ=10000
      CALL IGINIT(NWHIGZ)
*
*           Initialize HPLOT
*
      IWK = 999
      CALL HPLINT(IWK)
*
*           Give control to Motif main loop
*
      CALL motif_loop
*
      STOP
      END

      subroutine qnext
*     dummy
      end
*
**************   END   ***********************


Motif application code (in C) for user interface
------------------------------------------------

/**************************  start  ***********************************/

#include <stdio.h>
#include <Xm/Xm.h>
#include <Xm/MwmUtil.h>
#include <X11/Shell.h>
#include <Xm/MenuShell.h>
#include <Xm/Protocols.h>

#include <Xm/PushB.h>
#include <Xm/BulletinB.h>
#include <Xm/MainW.h>
#include <Xm/Frame.h>
#include <Xm/Form.h>
#include <Xm/DrawingA.h>

#define RES_CONVERT( res_name, res_value) \
        XtVaTypedArg, (res_name), XmRString, (res_value), strlen(res_value) + 1


/* Local functions */

static Window graph_open ();
static void graph_close ();
static void graph_init ();
static void drawing_area_callback( Widget widget, XtPointer client_data,
                                   XmDrawingAreaCallbackStruct *call_data);


static void create_motif_interface ();
static void close_cb( Widget widget, XtPointer client_data,
                      XtPointer call_data);
static void null_cb(Widget widget, XtPointer client_data,
                    XtPointer call_data);
static void new_cb(Widget widget, XtPointer client_data,
                   XtPointer call_data);


/* Global datas */

XtAppContext AppContext;
Widget TopLevel, Graphics, graphicsDraw;


/***********************************************************************
 *                                                                     *
 *   Initialize a Motif application (XtVaAppInitialize) and set up     *
 *   the interface with HIGZ and HPLOT (ixmotif).                      *
 *                                                                     *
 *   create a bulletin board with 3 buttons:                           *
 *   - NULL (call HPLOT routine HPLNUL)                                *
 *   - NEW  (call HPLOT routine HPLNEW)                                *
 *   - Close (terminate application and close windows)                 *
 *                                                                     *
 ***********************************************************************/
void init_motif_()
{
   int argc;
   char    *argv[1];

   /* init toolkit */
   argc       = 0;
   argv[argc] = NULL;
   TopLevel = XtVaAppInitialize(&AppContext, "Hmtest", NULL, 0,
                                &argc, argv, NULL,
                                XmNallowShellResize, True,
                                XmNdeleteResponse, XmDO_NOTHING,
                                XmNtitle, "Hmtest",
                                NULL);

   /* build interface */
   create_motif_interface ();

   /*
    * Creates Motif drawing area (graph_open) and pass information to HIGZ.
    *
    * ixmotif( Display *dsp, Window (*open)(), void (*init)(), void (*close)() )
    * allows a Motif-based application to give to HIGZ the address of three
    * routines:
    *    1) open  : to open a Motif window
    *    2) init  : to initialize a Motif window (add the callbacks for
    *               resizing)
    *    3) close : to close a Motif window
    */

   ixmotif (XtDisplay(TopLevel), graph_open, graph_init, graph_close);

}

/***********************************************************************
 *                                                                     *
 *   Start Motif infinite loop                                         *
 *                                                                     *
 ***********************************************************************/
void motif_loop_ ()
{
   XtRealizeWidget(TopLevel);
   XtAppMainLoop(AppContext);
}

/***********************************************************************
 *                                                                     *
 *   Build application user interface                                  *
 *                                                                     *
 ***********************************************************************/
static void create_motif_interface ()
{
   Widget toplevel, button_null, button_new, button_close;

   toplevel = XtVaCreateManagedWidget( "toplevel",
                        xmBulletinBoardWidgetClass,
                        TopLevel,
                        XmNresizePolicy, XmRESIZE_NONE,
                        XmNwidth, 320,
                        XmNheight, 293,
                        XmNunitType, XmPIXELS,
                        NULL );

   button_null = XtVaCreateManagedWidget( "button_null",
                        xmPushButtonWidgetClass,
                        toplevel,
                        RES_CONVERT( XmNlabelString, " NULL "),
                        XmNx, 70,
                        XmNy, 20,
                        XmNwidth, 180,
                        XmNheight, 60,
                        NULL );

   button_new = XtVaCreateManagedWidget( "button_new",
                        xmPushButtonWidgetClass,
                        toplevel,
                        RES_CONVERT( XmNlabelString, " NEW "),
                        XmNx, 70,
                        XmNy, 90,
                        XmNwidth, 180,
                        XmNheight, 60,
                        NULL );


   button_close = XtVaCreateManagedWidget( "button",
                        xmPushButtonWidgetClass,
                        toplevel,
                        RES_CONVERT( XmNlabelString, " Close "),
                        XmNx, 70,
                        XmNy, 160,
                        XmNwidth, 180,
                        XmNheight, 60,
                        NULL );

   /* Add callback to button "NULL" (call HPLOT routine HPLNUL) */
   XtAddCallback(button_null, XmNactivateCallback,
        (XtCallbackProc)null_cb,  NULL);

   /* Add callback to button "NEW" (call HPLOT routine HPLNEW) */
   XtAddCallback(button_new, XmNactivateCallback,
        (XtCallbackProc)new_cb,  NULL);

   /* Add callback to button "Close" (terminate application) */
   XtAddCallback(button_close, XmNactivateCallback,
        (XtCallbackProc)close_cb,  NULL);

   XmAddWMProtocolCallback(TopLevel,
        XmInternAtom(XtDisplay(TopLevel), "WM_DELETE_WINDOW", True),
        (XtCallbackProc)close_cb, (caddr_t) toplevel);
}

/***********************************************************************
 *                                                                     *
 *   Creates a Motif drawing area (for HIGZ)  and returns window id.   *
 *                                                                     *
 ***********************************************************************/
static Window graph_open ()
{
   Widget        Parent, graphicsFrame, graphicsForm;

   Parent = XtVaCreatePopupShell( "Graphics_shell",
                        topLevelShellWidgetClass, TopLevel,
                        XmNx, 570,
                        XmNy, 10,
                        XmNwidth, 530,
                        XmNheight, 550,
                        XmNtitle, "Hmtest Motif Graphics",
                        NULL );

   Graphics = XtVaCreateManagedWidget( "Graphics",
                        xmMainWindowWidgetClass,
                        Parent,
                        XmNwidth, 530,
                        XmNheight, 550,
                        XmNmappedWhenManaged, TRUE,
                        XmNborderWidth, 0,
                        NULL );

   graphicsFrame = XtVaCreateManagedWidget( "graphicsFrame",
                        xmFrameWidgetClass,
                        Graphics,
                        NULL );

   graphicsForm = XtVaCreateManagedWidget( "graphicsForm",
                        xmFormWidgetClass,
                        graphicsFrame,
                        XmNautoUnmanage, FALSE,
                        XmNborderWidth, 0,
                        XmNmarginHeight, 0,
                        XmNmarginWidth, 0,
                        XmNhorizontalSpacing, 0,
                        XmNverticalSpacing, 0,
                        NULL );

   graphicsDraw = XtVaCreateManagedWidget( "graphicsDraw",
                        xmDrawingAreaWidgetClass,
                        graphicsForm,
                        XmNresizePolicy, XmRESIZE_NONE,
                        XmNx, 0,
                        XmNy, 10,
                        XmNwidth, 530,
                        XmNheight, 540,
                        XmNbottomAttachment, XmATTACH_FORM,
                        XmNrightAttachment, XmATTACH_FORM,
                        XmNrightOffset, 0,
                        XmNtopOffset, 0,
                        XmNbottomOffset, 0,
                        XmNleftAttachment, XmATTACH_FORM,
                        XmNleftOffset, 0,
                        XmNtopAttachment, XmATTACH_FORM,
                        NULL );

   XmMainWindowSetAreas( Graphics, (Widget) NULL, (Widget) NULL,
                        (Widget) NULL, (Widget) NULL, graphicsFrame );

   XtPopup (Parent, XtGrabNone);

   return (XtWindow(graphicsDraw));
}

/***********************************************************************
 *                                                                     *
 *   Install callbacks for the graphics window (created for HIGZ)      *
 *   (This has to be called after the creation of the HIGZ window)     *
 *                                                                     *
 ***********************************************************************/
static void graph_init ()
{
   /* Add drawing area expose and resize callbacks */
   XtAddCallback(graphicsDraw, XmNexposeCallback,
                 (XtCallbackProc)drawing_area_callback, NULL);
   XtAddCallback(graphicsDraw, XmNresizeCallback,
                 (XtCallbackProc)drawing_area_callback, NULL);
}

/***********************************************************************
 *                                                                     *
 *   Handle the graphics window callbacks (expose and resize events).  *
 *                                                                     *
 ***********************************************************************/
static void drawing_area_callback(widget, client_data, call_data)
       Widget widget;
       XtPointer client_data;
       XmDrawingAreaCallbackStruct *call_data;
{
   XEvent *ev = call_data->event;

   /* window identifier (in this example we open only one window) */
   int wid = 1;

   if (call_data->reason == XmCR_EXPOSE) {
      if (ev->xexpose.count != 0) return;
      /* Set new size (HIGZ routine IGRSIZ) */
      igrsiz_ (&wid);
   }

   if (call_data->reason == XmCR_RESIZE) {
      /* Set new size (HIGZ routine IGRSIZ) */
      igrsiz_ (&wid);
   }

}

/***********************************************************************
 *                                                                     *
 *   Destroy Motif drawing area                                        *
 *                                                                     *
 ***********************************************************************/
static void graph_close ()
{
   XtDestroyWidget (Graphics);
}

/***********************************************************************
 *                                                                     *
 *   "NULL" callback routine.                                          *
 *   Call HPLOT routine HPLNUL                                         *
 *                                                                     *
 ***********************************************************************/
static void null_cb(widget, client_data,  call_data)
     Widget widget;
     XtPointer client_data;
     XtPointer call_data;
{
   hplnul_();
}

/***********************************************************************
 *                                                                     *
 *   "NEW" callback routine.                                           *
 *   Call HPLOT routine HPLNEW                                         *
 *                                                                     *
 ***********************************************************************/
static void new_cb(widget, client_data,  call_data)
     Widget widget;
     XtPointer client_data;
     XtPointer call_data;
{
   hplnew_();
}



/***********************************************************************
 *                                                                     *
 *   Close callback routine.                                           *
 *                                                                     *
 ***********************************************************************/
static void close_cb(widget, client_data,  call_data)
     Widget widget;
     XtPointer client_data;
     XtPointer call_data;
{
   exit (0);
}

/**************************   end   ***********************************/


Makefile (on Hp)
----------------

SHELL=/bin/sh

LPATH = /cern/new/lib/

XLIBS= -L/usr/lib/Motif1.2 -L/usr/lib/X11R5 -lXm -lXt -lX11

LIBC=  -ldld -Wl,-E -lm -lc -lPW


FFLAGS=+ppu
CFLAGS=-Ae -I/usr/include/X11R5 -I/usr/include/Motif1.2

CERNLIB=\
        $(LPATH)libgraflib.a \
        $(LPATH)libpacklib.a \
        $(LPATH)libgrafX11.a \
        $(LPATH)libpacklib.a \
        $(LPATH)libkernlib.a

hmtest: hmtest_main.o \
        hmtest_motif.o
	fort77 -o hmtest -g +ppu hmtest_main.o hmtest_motif.o $(CERNLIB) $(XLIBS) $(LIB

