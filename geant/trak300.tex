%%%%%%%%%%%%%%%%%%%%%%%%%%%%%%%%%%%%%%%%%%%%%%%%%%%%%%%%%%%%%%%%%%%
%                                                                 %
%  GEANT manual in LaTeX form                              %
%                                                                 %
%  Michel Goossens (for translation into LaTeX)                   %
%  Version 1.00                                                   %
%  Last Mod. Jan 24 1991  1300   MG + IB                          %
%                                                                 %
%%%%%%%%%%%%%%%%%%%%%%%%%%%%%%%%%%%%%%%%%%%%%%%%%%%%%%%%%%%%%%%%%%%
\Origin{R.Brun}
\Submitted{01.06.83}       \Revised{14.12.93}
\Version{Geant 3.16}       \Routid{TRAK300}
\Makehead{Storing secondary tracks in the stack}

\Shubr{GSKING}{(IT)}
\begin{DLtt}{MMMMMMMM}
\item[IT] ({\tt INTEGER}) number of the track to be stored for subsequent
tracking, if {\tt IT=0} all secondary tracks generated in the current
step are stored for tracking.
\end{DLtt}

During transport, particles can generate new particles, called 
secondary. In {\tt GEANT} these particles are stored in the common
\FCind{/GCKING/} (see {\tt [BASE030]} for an explanation of the
variables in this common). The number of particles generated in
the current step is {\tt NGKINE} (common \FCind{/GCKING/}). If no
user action is taken, the variable {\tt NGKINE} is reset to 0 at the
beginning of the subsequent step and these particles are not transported.

To be transported by {\tt GEANT}, these particles need to be stored
in the data structure {\tt JSTAK} (see {\tt [TRACK399]}) which
normally is a
LIFO stack: the last particle generated is the first one to be
tracked when the current one stops. The order of tracking can be changed 
by the user, plase see {\tt [TRAK310]} for more information on this
subject. 

The routine \Rind{GSKING} moves particles from the \FCind{/GCKING/} common
to the {\tt JSTAK} structure, where {\tt GEANT} looks for particles to 
transport.

When a particle is fetched from the {\tt JSTAK} data structure, its place
is freed for a new one, so the information on the initial kinematics of
this track is lost. It may be useful in some occasion to record the initial
kinematic of a track and save it till the end of the event. This is accomplished
by storing the particle information both in the {\tt JSTAK} data structure
and in the {\tt JVERTX/JKINE} one, which is permanent throughout the life
of the event. As already said in {\tt [KINE100]}, this should not be done
by the user calling directly the \Rind{GSVERT}/\Rind{GSKINE} routines
during tracking, but rather controlling the action of \Rind{GSKING} via
the array {\tt IFLGK} in common \FCind{/GCKING/}. If the particle number
{\tt IG} generated in the current step ({\tt 1} $\leq$ {\tt IG} $\leq$ 
{\tt NGKINE}) is being moved to the {\tt JSTAK} {\it stack}, either
singularly or together with all the other particles generated in the
current step (setting the argument of \Rind{GSKING} to 0), the action
performed depends on {\tt IFLGK(IG)}:

\begin{DLtt}{MMMMMMMMMMMM}
\item[IFLGK(IG)$<$0] the particle is ignored;
\item[~~~~~~~~~$=$0] the particle is stored on the temporary stack {\tt JSTAK}
to be transported;
\item[~~~~~~~~~$=$1] the particle is stored on the temporary stack {\tt JSTAK}
to be transported {\it and} in the {\tt JKINE} data structure attaching it to
a vertex created in the current position (array {\tt VECT(I), I}=1,3 in the
\FCind{/GCTRAK/} common);
\item[~~~~~~~~~$>$1] the particle is stored on the temporary stack {\tt JSTAK}
to be transported {\it and} in the {\tt JKINE} data structure attaching it to
vertex number {\tt IFLGK(IG)}, which must exist.
\end{DLtt}

If {\tt IFLGK(IG)}$>$0, after the call to \Rind{GSKING}, {\tt IFLGK(IG)} is
set to the newly created track number in {\tt JKINE}. The number of the
vertex used is returned in {\tt IFLGK(NGKINE+1)}. This feature allows the
user to identify the created vertex and tracks via the routines 
\Rind{GSVERU} and \Rind{GSKINU} (see {\tt [KINE100]}).

\section*{Example}

In case of a hadronic interaction, discard the neutrinos, store the protons
in the permanent data structure {\tt JVERTX/JKINE}, and store all the
other particles produced in the temporary stack.

Add to the protons in the {\tt JKINE} bank the {\tt JVOLUM}
and copy number of the 
volume and the number of the tracking medium where they have been produced
by an interaction.

\begin{verbatim}
      SUBROUTINE GUSTEP
+SEQ,GCKING,GCVOLU,GCTMED.
      CHARACTER*4 CHCASE
      DIMENSION UBUF(3)
      .
      .
      .
      CALL UHTOC(KCASE,4,CHCASE,4)
      IF(CHCASE.EQ.'HADR') THEN
         DO 10 IG=1, NGKINE
            IPART = GKIN(5,IG)
            IF(IPART.EQ.4) THEN
*---                Discard neutrinos
               IFLGK(IG)=-1
            ELSEIF(IPART.EQ.14) THEN
*---                Save protons in JKINE and transport them
               IFLGK(IG)=1
            ELSE
*---                Simply transport the rest
               IFLGK(IG)=0
            ENDIF
  10     CONTINUE
      ENDIF
*---                Perform action on all the particles
      CALL GSKING(0)
*---               Add information to the JKINE bank
      DO 20 IG=1,NGKINE
         IPART = GKIN(5,IG)
         IF(IPART.EQ.14) THEN
*---               This is a proton
*---               Get the track number
            IT = IFLGK(IG)
            NUBUF = 3
            UBUF(1) = LVOLUM(NLEVEL)
            UBUF(2) = NUMBER(NLEVEL)
            UBUF(3) = NUMED
            IADR = 0
*---               Store the information
            CALL GSKINU(IT,NUBUF,UBUF,IADR)
         ENDIF
  20  CONTINUE
\end{verbatim}
