%%%%%%%%%%%%%%%%%%%%%%%%%%%%%%%%%%%%%%%%%%%%%%%%%%%%%%%%%%%%%%%%%%%
%                                                                 %
%  GEANT manual in LaTeX form                                     %
%                                                                 %
%  Michel Goossens (for translation into LaTeX)                   %
%  Version 1.00                                                   %
%  Last Mod. Jan 24 1991  1300   MG + IB                          %
%                                                                 %
%%%%%%%%%%%%%%%%%%%%%%%%%%%%%%%%%%%%%%%%%%%%%%%%%%%%%%%%%%%%%%%%%%%
\Origin{R.Brun,R.Hemingway}
\Submitted{01.11.83}               \Revised{15.12.93}
\Version{Geant 3.16}               \Routid{KINE200}

\Makehead{Interface to the Lund Monte Carlo}

{\tt GEANT} contains a simple interface to the {\tt JETSET} Monte
Carlo generator~\cite{bib-JETS}. This interface is composed by an
initialisation routine and an event generation routine.

\Shubr{GLUNDI}{}
This is the initialisation routine which should be called before the
call to \Rind{GFFGO}, as it defines the data record {\tt LUND}. This routine
performs the following actions:
\begin{enumerate}
\item sets {\tt ECLUND}$=\sqrt{s}=92.25$ GeV;
\item sets {\tt IFLUND}$=0$;
\item activates initial state radiative effects {\tt MSTJ(107)= 1};
\item declares some particles ($K^0_s$, $\Sigma^{\pm}$, $\Xi^{\pm}$,
$\Lambda^0$, $\Omega^-$) to be stable in {\tt JETSET} so that {\tt GEANT}
can handle their tracking and decays:
\begin{verbatim}
         MDCY(LUCOMP(310) ,1)=0
         MDCY(LUCOMP(3222),1)=0
         MDCY(LUCOMP(3112),1)=0
         MDCY(LUCOMP(3322),1)=0
         MDCY(LUCOMP(3312),1)=0
         MDCY(LUCOMP(3122),1)=0
         MDCY(LUCOMP(3334),1)=0
\end{verbatim}
\end{enumerate}

Variables {\tt ECLUND}, total energy in the CMS, and {\tt IFLUND}, quark
flavours produced, are used as arguments for \Rind{LUEEVT} called by
\Rind{GLUND} and they can be changed via the {\tt LUND} data record
(see {\tt [BASE040]}).

\Shubr{GLUND}{}
Generate a \Pem\Pep collision via the {\tt JETSET} MonteCarlo and stores
the resulting final state particles for transport by {\tt GEANT}. All
particles are attached to a vertex in (0.,0.,0.). The {\tt K} and {\tt P}
arrays containing the full {\tt JETSET} event are copied in the user
buffer of the generated vertex, with the {\tt K} variables converted into
{\tt REAL} numbers. The content of the user buffer is the following:
\[
\begin{array}{ll}
\begin{array}{r@{}lc}
\mbox{\tt UBUF(} & \mbox{\tt 1)~} & 527.0 \\
\mbox{\tt UBUF(} & \mbox{\tt 2)~} & \mbox{{\tt N} number of particles}
\end{array} \\
\left .
\begin{array}{r@{}l@{\hspace{1cm}}l}
\mbox{\tt UBUF(} & \mbox{\tt 3)} & \mbox{\tt K(1,1)} \\
\mbox{\tt UBUF(} & \mbox{\tt 4)} & \mbox{\tt K(1,2)} \\
\mbox{\tt UBUF(} & \mbox{\tt 5)} & \mbox{\tt K(1,3)} \\
\mbox{\tt UBUF(} & \mbox{\tt 6)} & \mbox{\tt K(1,4)} \\
\mbox{\tt UBUF(} & \mbox{\tt 7)} & \mbox{\tt K(1,5)} \\
\mbox{\tt UBUF(} & \mbox{\tt 8)} & \mbox{\tt P(1,1)} \\
\mbox{\tt UBUF(} & \mbox{\tt 9)} & \mbox{\tt P(1,2)} \\
\mbox{\tt UBUF(} & \mbox{\tt 10)} & \mbox{\tt P(1,3)} \\
\mbox{\tt UBUF(} & \mbox{\tt 11)} & \mbox{\tt P(1,4)} \\
\mbox{\tt UBUF(} & \mbox{\tt 12)} & \mbox{\tt P(1,5)}
\end{array} \right \} & \mbox{{\tt N} times}
\end{array}
\]
