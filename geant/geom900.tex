%%%%%%%%%%%%%%%%%%%%%%%%%%%%%%%%%%%%%%%%%%%%%%%%%%%%%%%%%%%%%%%%%%%
%                                                                 %
%  GEANT manual in LaTeX form                              %
%                                                                 %
%  Michel Goossens (for translation into LaTeX)                   %
%  Version 1.00                                                   %
%  Last Mod. Jan 24 1991  1300   MG + IB                          %
%                                                                 %
%%%%%%%%%%%%%%%%%%%%%%%%%%%%%%%%%%%%%%%%%%%%%%%%%%%%%%%%%%%%%%%%%%%
\Origin{A.C.McPherson}
\Revision{F.Bruyant, R.Brun}   
\Submitted{16.12.83}         \Revised{16.12.93}
\Version{Geant 3.16}         \Routid{GEOM900}

\Makehead{End of geometry definition}
\Shubr{GGCLOS}{}
This routine should be called after all volumes, positions,
hits and digits have been defined. Its action is confined to
clean up the volume banks and to take any action required by the
ordering techniques. 

\Rind{GGCLOS} performs all the requested geometry optimisation, included
the creation of pseudo-divisions in volumes flagged by \Rind{GSORD}.
It performs the local development of tha last leaves of the
geometrical tree, where applicable.

It also calls the routine \Rind{GGDETV} which prepares the prototype lists of
volume names and maximum multiplicities which permit to identify uniquely
any sensitive detector whose generic name has been declared through the
routine \Rind{GSDETV} (and not \Rind{GSDET}) {\tt [HITS001]}.

\Rind{GGCLOS} must be called also after reading a geometry data structure
from disk.
