%%%%%%%%%%%%%%%%%%%%%%%%%%%%%%%%%%%%%%%%%%%%%%%%%%%%%%%%%%%%%%%%%%%
%                                                                 %
%  GEANT manual in LaTeX form                                     %
%                                                                 %
%  Michel Goossens (for translation into LaTeX)                   %
%  Version 1.00                                                   %
%  Last Mod. Jan 24 1991  1300   MG + IB                          %
%                                                                 %
%%%%%%%%%%%%%%%%%%%%%%%%%%%%%%%%%%%%%%%%%%%%%%%%%%%%%%%%%%%%%%%%%%%
\Origin{L.Urb\'{a}n}
\Version{Geant 3.16}  \Routid{PHYS450}
\Submitted{24.03.86}  \Revised{16.12.93}
\Makehead{Total cross-section and energy loss for e-e+ pair
production by muons}
\section{Subroutines}
\Shubr{GPRELA}{}
\Rind{GPRELA} adds to the muon energy loss tables the contribution 
due to $\mu$-nucleus interactions and \Pep\Pem-pair production in 
which the energy of the pair is below threshold (see later).
For the contribution of pair production, it calls the function 
\Rind{GPRELM}. The energy binning is in the array {\tt ELOW} (common 
\FCind{/CGMULO/}). The following pointers are used:

\begin{DLtt}{MMMMMMMMMMMMMMMMMMMM}
\item[JMA  = LQ(JMATE-I)] pointer to the I$^{th}$ material;
\item[JEL2 = LQ(JMA-2)]  pointer to $dE/dx$ for $\mu^+\mu^-$.
\end{DLtt}

\Rind{GPRELA} is called during initialisation by \Rind{GPHYSI}.

\Sfunc{GPRELM}{VALUE = GPRELM(Z,T,CUT)}

\begin{DLtt}{MMMMMMMM}
\item[Z] ({\tt REAL}) atomic number of the material;
\item[T] ({\tt REAL}) kinetic energy of the muon;
\item[CUT] ({\tt REAL}) maximum energy for the pair considered by the routine.
\end{DLtt}
\Rind{GPRELM} calculates the energy loss due to the direct
production by muons of \Pem\Pep pairs with energy below {\tt CUT}
({\tt PPCUTM} in common \FCind{/GCCUTS/}).
Above this cut, the direct pairs are produced as discrete processes
(see {\tt [PHYS451]}). \Rind{GPRELM} is called by \Rind{GPRELA}.
\Shubr{GPRSGA}{}
{\tt GPRSGA} calculates the total cross-section for 
the pair production by photons and the direct pair production by muons.
It tabulates the mean free path
in cm as a function of medium and energy. 
For the direct pair production of muons, it calls 
the function \Rind{GPRSGM}. The energy
binning is in the array {\tt ELOW} (common \FCind{/CGMULO/}).
The following pointers are used:
\begin{DLtt}{MMMMMMMMMMMMMMMMMMMM}
\item[JMA = LQ(JMATE-I)]  pointer to the I$^{th}$ material;
\item[JPAIR = LQ(JMA-10)] pointer to photon pair production cross-sections;
\item[JPAIR+NEK1]      pointer to muon pair production.
\end{DLtt}
{\tt GPRSGA} is called during initialisation by {\tt GPHYSI}.

\Sfunc{GPRSGM}{VALUE = GPRSGM(Z,T,CUT)}

\begin{DLtt}{MMMMMMMM}
\item[Z] ({\tt REAL}) atomic number of the medium;
\item[T] ({\tt REAL}) kinetic energy of the muon;
\item[CUT] ({\tt REAL}) minimum energy for the \Pem\Pep pair.
\end{DLtt}
\Rind{GPRSGM} calculates the total cross-section for
muon direct pair production where the energy of the pair is greater
than {\tt CUT} ({\tt PPCUTM} in common \FCind{/GCCUTS/}). It is called 
by \Rind{GPRSGA} (see {\tt [PHYS451]}).

\section{Method}
When a muon of energy $E$ moves in the field of
an atom of charge $Z$, it can radiate a \Pep\Pem-pair
($4^{th}$ order QED process) with 
differential cross-section \cite{bib-LOHM}:
\begin{equation}
\frac{\partial^2 \sigma}{\partial \nu \; \partial \rho}=
\alpha^4 \frac {2}{3 \pi}   (Z \lambdar)^2 \frac{1-v}{v}
\left [ \phi_e + \left(\frac{m_{e}} {m_{\mu}}\right)^2 \phi_{\mu} \right ]
\end{equation}
where
\[
\nu = \frac{ E^+ + E^-}{E} \mbox{\quad and\quad}\rho =
\frac{E^+ - E^-}{E^+ + E^-}
\]
and
\begin{DLtt}{MMMMMMMM}
\item[$\alpha$]   1/137, fine structure constant;
\item[$\lambdar$] $3.8616 \times 10^{-11}$ cm, electron Compton wavelength;
\item[$v$]        $k/E$ fraction of energy transferred to the pair;
\item[$T$]        $E-M$ kinetic energy of the muon.
\item[$E^{\pm}$]  the energy of the $e^{\pm}$.
\end{DLtt}

The explicit form of the terms $\phi_e$ and $\phi_{\mu}$
can be found in \cite{bib-LOHM}. The kinematic ranges of $\nu$ and $\rho $ are:
\begin{equation}
\begin{array}{RCCCCCL}
\frac{4m_{e}}{E}=\nu_{min} & \leq & \nu & \leq  & \nu_{max} & = &
1 -0.75 \sqrt{e}\frac{m_{\mu}}{E} Z^{1/3}       \\ [0.7cm]
0=\rho_{min} & \leq & |\rho (\nu) | & \leq & \rho_{max} (\nu) &
= & \left [1-\frac{6m_{\mu}^2}{E^2 (1-\nu)}\right]
         \sqrt{1- \frac{4m_{e}}{\nu E}} 
\end{array}
\end{equation}
 
where   $e = 2.718\dots$.
 
$E_c$ ({\tt PPCUTM} in the program) is the energy cut-off; below this 
energy \Pep\Pem-pair are treated as continuous energy loss,
and above they are explicitly
generated and $v_c = E_c/E$.
The mean value of the energy lost by the incident muon due to
\Pep\Pem-pair with energy below $E_{c}$ is:
\begin{equation}
\label{eq:phys450-1}
 E_{loss}^{pair}(Z,T,E_c ) =
 2E \int_{\nu_{min}}^{\nu_c}{ d\nu\;\nu \int_{0}^{\rho_{max}(\nu)}{
 d\rho \;
\frac{\partial^2 \sigma}{\partial\nu \; \partial\rho}}}
\mbox{\hspace{0.5cm} GeV barn/atom}
\end{equation}
whereas the total cross-section for the emission of a hard \Pep\Pem-pair is:
\begin{equation}
\label{eq:phys450-2}
\sigma (Z,T,E_c) =
 2 \int_{\nu_c}^{\nu_{max}} d \nu\;
   \int_{0}^{\rho_{max}(\nu)}
 d \rho  \; \frac{\partial^2\sigma}{ \partial\nu \; \partial\rho}
\mbox{\hspace{0.5cm} barn/atom}
\end{equation}

\subsection{Parameterisation of energy loss and total cross-section}
Instead of using the explicit formula (\ref{eq:phys450-1}) and
(\ref{eq:phys450-2}), we have chosen to parameterise directly 
$E_{loss}^{pair}(Z,T, E_c $) and $\sigma(Z,T,E_c)$ as:
\begin{eqnarray}
\label{eq:phys450-3}
\sigma (Z,T,E_c) & = & Z [Z+ \xi_{\sigma} (1+ \gamma \ln Z)]
     \left [\ln \frac{E_{max}}{E}\right]^{\alpha}
      F_{\sigma} (Z,X,Y)                           \\
E_{loss}^{pair}(Z,T,E_c)      & = & Z [Z+ \xi_l (1+ \delta \ln Z)] E
\label{eq:phys450-4}
      \left [ \frac{E_c - E_c^{min}}{E}\right]^{\beta}
     F_l(Z,X,Y)
\end{eqnarray}
where  $\xi_{\sigma,l}$, $\alpha$, $\beta$, $\gamma$ and $\delta$
are parameters obtained by a fitting procedure and
$ E_c^{min} = E\:v_{min} = 4 m_{e}$ and
$ E_{max} = E\:v_{max}$
 
The functions $ F_i $ (Z,X,Y) ($ i=\sigma,\;l $) have the form:
\begin{equation}
 F_i(Z,X,Y) = F_{i0}(X,Y)+ZF_{i1}(X,Y)
\end{equation}
where $F_{ij}(X,Y)$ denotes a function constructed from two polynomials
\begin{equation}
 F_{ij}(X,Y) = \left\{ \begin{array}{LlL}
                   P_{ij}^{neg}(X,Y)  & \mbox{if} & Y \leq 0 \\ [0.3cm]
                   P_{ij}^{pos}(X,Y)  & \mbox{if} & Y > 0
                   \end{array} \right.
\end{equation}
and $P_{ij}^{neg}$, $P_{ij}^{pos}$ fulfil the conditions:
\[
P_{ij}^{neg}(X,Y)_{Y=0} = P_{ij}^{pos}(X,Y)_{Y=0} 
\hspace{2cm}
\left . \frac{\partial P_{ij}^{neg}}{\partial Y } \right |_{Y=0}  =
\left . \frac{\partial P_{ij}^{pos}}{\partial Y } \right |_{Y=0}
\]
The detailed form of the $P_{ij}$ polynomials is:
\begin{eqnarray}
P_{i0}^{neg}(X,Y)& = &
        (C_1+C_2X+\cdots+C_6X^5)+(C_7+C_8X+\cdots+C_{12}X^5)Y
\nonumber \\
  & + &(C_{13}+C_{14}X+\cdots+C_{18}X^5)Y^2+\cdots+(C_{31}+\cdots+C_{36}X^5)Y^5
\nonumber\\
P_{i0}^{pos}(X,Y)& = &
        (C_1+C_2X+\cdots+C_6X^5)+(C_7+C_8X+\cdots+C_{12}X^5)Y
\nonumber\\
\label{eq:phys450-5}
  & + &(C_{37}+C_{38}X+\cdots+C_{42}X^5)Y^2+\cdots+(C_{55}+\cdots+C_{60}X^5)Y^5
         \\
P_{i1}^{neg}(X,Y)& = &
  (C_{61}+C_{62}X+\cdots+C_{65}X^4)+(C_{66}+C_{67}X+\cdots+C_{70}X^4)Y
\nonumber \\
  & + &(C_{71}+C_{72}X+\cdots+C_{75}X^4)Y^2+\cdots+(C_{81}+\cdots+C_{85}X^4)Y^4
\nonumber \\
 P_{i1}^{pos}(X,Y)& = &
  (C_{61}+C_{62}X+\cdots+C_{65}X^4)+(C_{66}+C_{67}X+\cdots+C_{70}X^4)Y
\nonumber \\
  & + &(C_{86}+C_{87}X+\cdots+C_{90}X^4)Y^2+\cdots+(C_{96}+\cdots+C_{100}X^4)Y^4
\nonumber
\end{eqnarray}
with
\[
X = \ln \frac{E}{m_{\mu}} \hspace{2cm} Y = \ln \frac{E_{c}}{v_{j} E}
\hspace{2cm} j = \sigma, l
\]
where $v_{\sigma,l}$ are additional parameters of the fit.
By a numerical (twofold) integration of the formulae (\ref{eq:phys450-3})
and (\ref{eq:phys450-4})
above, we have calculated  1800 {\it data points} in the range
$Z = 1, 6, 13, 26, 50, 82, 92$ and the energy ranges 1GeV $ \leq T \leq $ 10
TeV and $ 4m \leq E_c \leq T$
and performed a least-squares fit to determine the parameters.
The fitted values of the parameters are in the {\tt DATA} statements in the
functions \Rind{GPRSGM} and \Rind{GPRELM}, which compute the formulae 
(\ref{eq:phys450-3}) and (\ref{eq:phys450-4}) respectively.
The accuracy of the fit is:
\[ \begin{array}{LL}
\frac{\Delta \sigma}{\sigma} & \leq  \left\{
\begin{array}{RcLl}
10\% & \mbox{if} & T\leq 5 & \mbox{GeV} \\
5\%  & \mbox{if} & T > 5 & \mbox{GeV}
\end{array} \right . \\ [0.5cm]
\frac{\Delta E_{loss}^{pair}}{E_{loss}^{pair}}  & \leq \left\{
\begin{array}{RcLl}
10\% & \mbox{if} & T\leq 5 & \mbox{GeV} \\
4\%  & \mbox{if} & T > 5 & \mbox{GeV}
\end{array} \right.
\end{array} \]
 
The function \Rind{GPRELM} contains a second formula to calculate the total
energy lost by the muon due to direct \Pep\Pem-production 
used when $E_c\geq E_{\mbox{max}} = E-0.75 \surd e m_{\mu}Z^{1/3}$.
This formula describes the total energy loss with an error less than 1\%:
\begin{eqnarray}
E_{loss}^{pair}(Z,T) & = &
E_{loss}^{pair}(Z,T,E_c=E_{max}) \\
& = & Z(Z+1) E_{max} [d_1+(d_2X+d_3Y)+(d_4X^2+d_5XY+d_6Y^2)+ \nonumber\\
&+ & \cdots+(d_{22}X^6+d_{23}X^5Y+\cdots+d_{28}Y^6)] \nonumber
\end{eqnarray}
where $ X=\ln(E/m_{\mu}) $ and $Y=Z^{1/3}$.
The fitted parameters $d_i$ can be found in the {\tt DATA} 
statement in the function \Rind{GPRELM}.

