%%%%%%%%%%%%%%%%%%%%%%%%%%%%%%%%%%%%%%%%%%%%%%%%%%%%%%%%%%%%%%%%%%%
%                                                                 %
%  GEANT manual in LaTeX form                                     %
%                                                                 %
%  Michel Goossens (for translation into LaTeX)                   %
%  Version 1.00                                                   %
%  Last Mod. Jan 24 1991  1300   MG + IB                          %
%                                                                 %
%%%%%%%%%%%%%%%%%%%%%%%%%%%%%%%%%%%%%%%%%%%%%%%%%%%%%%%%%%%%%%%%%%%
\Origin{F.Carminati, R.Jones}
\Documentation{F. Carminati}
\Submitted{03.02.93}\Revised{16.12.93}
\Version{Geant 3.16}\Routid{PHYS260}
\Makehead{\v{C}erenkov photons}
\v{C}erenkov photons are produced when a charged particle traverses
a dielectric material.

\section{Physics processes for optical photons}
A photon is called optical when its wavelength is much greater than the
typical atomic spacing, for instance when $\lambda \geq 10nm $
which corresponds to an energy $E \leq 100eV$.
Production of an optical photon in a HEP detector is primarily due to:
\begin{enumerate}
\item \v{C}erenkov effect;
\item Scintillation;
\item Fluorescence.
\end{enumerate}

Fluorescence is taken into account in {\tt GEANT} in the context of the
photoelectric effect ({\tt [PHYS230], [PHYS231]}), but only above the energy cut
{\tt CUTGAM}. Scintillation is not yet simulated by {\tt GEANT}.

Optical photons undergo three kinds of interactions:
\begin{enumerate}
\item Elastic (Rayleigh) scattering;
\item Absorption;
\item Medium boundary interactions.
\end{enumerate}

\subsection{Elastic scattering}
For optical photons elastic scattering is usually unimportant. For
$\lambda=.200\mu m$ we have $\sigma_{Rayleigh} \approx .2b$ for $N_{2}$ 
or $O_{2}$ which gives a mean free path of $\approx 1.7$ km in air
and $\approx 1$ m in quartz. 
An important exception to this is found in aerogel, which is used as a 
\v{C}erenkov radiator for some special applications. Because of the 
spectral properties of this material, Rayleigh scattering is extremely
strong and this limits its usefulness as a RICH radiator. At present,
elastic scattering of optical photons is not simulated in {\tt GEANT}.

\subsection{Absorption}
Absorption is important for optical photons because it determines the
lower $\lambda$ limit in the {\it window} of transparency
of the radiator. Absorption competes with
photo-ionisation in producing the signal in the detector, so it must be
treated properly in the tracking of optical photons.

\subsection{Medium boundary effects}
When a photon arrives at the boundary of a dielectric medium,
its behaviour depends on the nature
of the two materials which join at that boundary:
\begin{itemize}
\item Case dielectric $\rightarrow$ dielectric. \\
The photon can be transmitted (refracted ray) or reflected (reflected ray).
In case where
the photon can only be reflected, total internal reflection takes place.
\item Case dielectric $\rightarrow$ metal. \\
The photon can be absorbed by the metal or reflected back into the
dielectric. If the photon is absorbed it can be detected according to the
photoelectron efficiency of the metal.
\item Case dielectric $\rightarrow$ black material. \\
A {\it black} material is a tracking medium for which the user has not
defined any optical property. In this case the photon is immediately
absorbed undetected.
\end{itemize}

\section{Photon polarisation}
The photon polarisation is defined as a two component vector normal to the
direction of the photon:
\[
\left ( \begin{array}{c}
a_{1} e^{i \phi_{1}} \\
a_{2} e^{i \phi_{2}}
\end{array} \right ) =
e^{\phi_{o}} \left (
\begin{array}{l}
a_{1} e^{i \phi_{c}} \\
a_{2} e^{-i \phi_{c}}
\end{array} \right )
\]
where
$\phi_{c} = (\phi_{1}-\phi_{2})/2$ is called circularity and
$\phi_{o} = (\phi_{1}+\phi_{2})/2$ is called overall phase. Circularity
gives the left- or right-polarisation characteristic of the photon. RICH
materials usually do not distinguish between the two polarisations and
photons produced by the \v{C}erenkov effect are linearly polarised, that is
$\phi_{c}=0$. The circularity of the photon is ignored by {\tt GEANT}.

The overall phase is important in determining interference effects between
coherent waves. These are important only in layers of thickness comparable
with the wavelength, such as interference filters on mirrors. The effects
of such coatings can be accounted for by the empirical reflectivity factor
for the surface, and do not require a microscopic simulation.
{\tt GEANT} does not keep track of the overall phase.

Vector polarisation is described by the polarisation angle
$\tan \psi = a_{2}/a_{1}$.
Reflection/transmission probabilities are sensitive to the state of linear
polarisation, so this has to be taken into account. One parameter is
sufficient to describe vector polarisation, but to avoid too many
trigonometrical transformations, a unit vector perpendicular to the direction
of the photon is used in {\tt GEANT}.

The polarisation vectors are stored in a special track structure which
is lifted at the link {\tt LQ(JSTACK-1)} when the first \v{C}erenkov
photon is stored in the stack.

\section{Method}

\subsection{Generation of the photons}

For the formulas contained in this chapter, please see \cite{bib-JACK}.
Let $n$ be the refractive index of the dielectric material
acting as a radiator
($n=c/c'$ where $c'$ is the group velocity of light in
the material: it follows that $1 \leq n$). In a dispersive material $n$ is
an increasing function of the photon momentum $p_{\gamma},
dn/dp \geq 0$.  A particle travelling
with speed $\beta = v/c$ will emit photons at an angle  $\theta$
with respect to its direction, where $\theta$ is given by the
relation:

\[
\cos \theta = \frac{1}{\beta n}
\]

from which follows the limitation for the momentum of the
emitted photons:

\[
n(p_{\gamma}^{min}) = \frac{1}{\beta}
\]

Additionally, the photons must be within the window of transparency of
the radiator. All the photons will be contained in a cone of opening
$\cos \theta_{max}  = 1/(\beta n(p_{\gamma}^{max}))$.

The average number of photons produced is given by the relations
(Gaussian units):
\[
dN =
\frac{2 \pi z^{2}e^{2}}{\hbar c} \sin^{2} \theta \frac{d \nu}{c} dx =
\frac{2 \pi z^{2}e^{2}}{\hbar c} \left ( 1- \cos^{2} \theta
\right ) \frac{d \nu}{c} dx =
\frac{2 \pi z^{2}e^{2}}{\hbar c} \left ( 1- \frac{1}{n^{2} \beta^{2}}
\right ) \frac{d \nu}{c} dx =
\]

\[
= \frac{z^{2}e^{2}}{\hbar^{2} c^{2}} \left ( 1- \frac{1}{n^{2} \beta^{2}}
\right ) dp_{\gamma} \: dx \approx
370 z^{2} \frac{\rm photons}{\rm cm \; eV}
\left ( 1- \frac{1}{n^{2} \beta^{2}} \right ) dp_{\gamma} \: dx
\]

and

\[
\frac{dN}{dx} \approx 370 z^{2}
\int^{p_{\gamma}^{max}}_{p_{\gamma}^{min}}{dp_{\gamma}
\left ( 1- \frac{1}{n^{2} \beta^{2}} \right ) }
= 370 z^{2} \left ( p_{\gamma}^{max} -
p_{\gamma}^{min} - \frac{1}{\beta^{2}}
\int^{p_{\gamma}^{max}}_{p_{\gamma}^{min}}{dp_{\gamma}
\frac{1}{n(p_{\gamma})^{2}}} \right )
\]

The number of photons produced is calculated from a Poissonian distribution
with average value $\bar{n} = \mbox{\tt STEP} \: dN/dx$. 
The momentum distribution of
the photon is then sampled from the density function:

\[
f(p_{\gamma}) = \left ( 1- \frac{1}{n^{2}(p_{\gamma}) \beta^{2}} \right )
\]

\subsection{Tracking of the photons}

\v{C}erenkov photons are tracked in the routine \Rind{GTCKOV}. 
These particles are subject to {\it in flight} absorption (process
{\tt LABS}, number 101) and {\it boundary action} 
(process {\tt LREF}, number 102, see above). As explained
above, the status of the photon is defined by 2 vectors, the 
photon momentum ($\vec{p}=\hbar \vec{k}$) and photon polarisation 
($\vec{e}$). By convention the direction of the polarisation vector
is that of the electric field. Let also $\vec{u}$ be the normal to
the material boundary at the point of intersection, pointing 
out of the material which the photon is leaving and toward the one
which the photon is entering.
The behaviour of a photon at the surface
boundary is determined by three quantities:
\begin{enumerate}
\item refraction or reflection angle, this represents the kinematics
of the effect;
\item amplitude of the reflected and refracted waves, this is 
the dynamics of the effect;
\item probability of the photon to be refracted or reflected,
this is the quantum mechanical effect which we have to take
into account if we want to describe the photon as a particle and
not as a wave;
\end{enumerate}

As said above, we distinguish three kinds of boundary action, dielectric
$\rightarrow$ black material, dielectric $\rightarrow$ metal, dielectric
$\rightarrow$ dielectric. The first case is trivial,
in the sense that the photon is immediately absorbed and it goes
undetected. 

To determine the behaviour of the photon at the boundary,
we will at first treat it as an homogeneous monochromatic plane wave:

\begin{eqnarray*}
\vec{E} & = & \vec{E}_{0} e^{i \vec{k} \cdot \vec{x} - i \omega t} \\
\vec{B} & = & \sqrt{\mu \epsilon} \frac{\vec{k} \times \vec{E}}{k}
\end{eqnarray*}

\subsubsection{Case dielectric $\rightarrow$ dielectric}

In the classical description the incoming wave splits into a reflected
wave (quantities with a double prime) and a refracted wave (quantities
with a single prime). Our problem is solved if we find the following 
quantities:

\begin{eqnarray*}
\vec{E'} & = & \vec{E'}_{0} e^{i \vec{k'} \cdot \vec{x} - i \omega t} \\
\vec{E''} & = & \vec{E''}_{0} e^{i \vec{k''} \cdot \vec{x} - i \omega t}
\end{eqnarray*}

For the wave numbers the following relations hold:

\begin{eqnarray*}
|\vec{k}| & = & |\vec{k}''| = k = \frac{\omega}{c} \sqrt{\mu \epsilon} \\
|\vec{k}'| & = & k' = \frac{\omega}{c} \sqrt{\mu' \epsilon'}
\end{eqnarray*}

Where the speed of the wave in the medium is $v=c/\sqrt{\mu \epsilon}$
and the quantity $n=c/v=\sqrt{\mu \epsilon}$ is called {\it refractive
index} of the medium. The condition that the three waves, refracted, reflected
and incident have the same phase at the surface of the medium, gives us the
well known Fresnel law:

\begin{eqnarray*}
(\vec{k} \cdot \vec{x})_{surf} & = & (\vec{k}' \cdot \vec{x})_{surf} =
(\vec{k}'' \cdot \vec{x})_{surf} \\
k \sin{i} & = & k' \sin{r} = k'' \sin{r'}
\end{eqnarray*}

where $i, r, r'$ are, respectively, the angle of the incident, refracted and
reflected ray with the normal to the surface. From this formula
the well known condition emerges:

\begin{eqnarray*}
i & = & r'  \\
\frac{\sin{i}}{\sin{r}} & = & \sqrt{\frac{\mu' \epsilon'}{\mu \epsilon}} =
\frac{n'}{n}
\end{eqnarray*}


The dynamic properties of the wave at the boundary are derived from Maxwell's
equations which impose the continuity of the normal components of $\vec{D}$ 
and $\vec{B}$ and of the tangential components of $\vec{E}$ and $\vec{H}$ 
at the surface boundary. The resulting ratios between the amplitudes of the
the generated waves with respect to the incoming one are 
expressed in the two following cases:

\begin{enumerate}
\item a plane wave with the electric 
field (polarisation vector) perpendicular to the plane defined by the
photon direction and the normal to the boundary:

\begin{eqnarray*}
\frac{E_{0}'}{E_{0}} & = & \frac{2 n \cos{i}}{n \cos{i} + \frac{\mu}{\mu'}
n' \cos{r}} = 
\frac{2 n \cos{i}}{n \cos{i} + n' \cos{r}} \\
\frac{E_{0}''}{E_{0}} & = & \frac{n \cos{i} - \frac{\mu}{\mu'} n' \cos{r}}
{n \cos{i} + \frac{\mu}{\mu'} n' \cos{r}} = 
\frac{n \cos{i} - n' \cos{r}} {n \cos{i} + n' \cos{r}}
\end{eqnarray*}

where we suppose, as it is legitimate for visible or near-visible light, that
$\mu/\mu' \approx 1$;

\item a plane wave with the electric
field parallel to the above surface: 

\begin{eqnarray*}
\frac{E_{0}'}{E_{0}} & = & \frac{2 n \cos{i}}
{\frac{\mu}{\mu'} n' \cos{i} + n \cos{r}} =
\frac{2 n \cos{i}} {n' \cos{i} + n \cos{r}} \\
\frac{E_{0}''}{E_{0}} & = & \frac{\frac{\mu}{\mu'} n' \cos{i} 
- n \cos{r}}{\frac{\mu}{\mu'} n' \cos{i} + n \cos{r}} = 
\frac{n' \cos{i} - n \cos{r}}{n' \cos{i} + n \cos{r}}
\end{eqnarray*}

whith the same approximation as above.

\end{enumerate}

We note that in case of photon perpendicular to the surface, the following
relations hold:
\[
\begin{array}{LcL@{\hspace{4cm}}LcL}
\frac{E_{0}'}{E_{0}} & = & \frac{2n}{n'+n} &
\frac{E_{0}''}{E_{0}} & = & \frac{n'-n}{n'+n}
\end{array}
\]

where the sign convention for the {\it parallel} field has been adopted.
This means that if $n' > n$ there is a phase inversion for the reflected
wave.

Any incoming wave can be separated into one piece polarised parallel to the
plane and one polarised perpendicular, and the two components treated
accordingly.

To mantain the particle description of the photon, the
probability to have a
{\it refracted} (mechanism 107) or {\it reflected} 
(mechanism 106) photon must be calculated. 
The constraint is that the number of photons be conserved, and this
can be imposed via the conservation of the energy flux at the boundary,
as the number of photons is proportional to the energy. The
energy current is given by the expression:

\begin{eqnarray*}
\vec{S} & = & \frac{1}{2} \frac{c}{4 \pi} \sqrt{\mu \epsilon} \vec{E} \times
\vec{H}^{*}
 = \frac{c}{8 \pi} \sqrt{\frac{\epsilon}{\mu}} E_{0}^{2} \hat{k}
\end{eqnarray*}

and the energy balance on a unit area of the boundary requires that:

\begin{eqnarray*}
\vec{S} \cdot \vec{u} & = & \vec{S}' \cdot \vec{u} 
- \vec{S}'' \cdot \vec{u} \\
S \cos{i} & = & S' \cos{r} + S'' \cos{i} \\
\frac{c}{8 \pi} \frac{1}{\mu} n E_{0}^{2} \cos{i} & = & 
\frac{c}{8 \pi} \frac{1}{\mu'} n' E_{0}'^{2} \cos{r} +
\frac{c}{8 \pi} \frac{1}{\mu} n E_{0}''^{2} \cos{i} 
\end{eqnarray*}

If we set again $\mu/\mu' \approx 1$, then the transmission
probability for the photon will be:

\[
T = \left( \frac{E_{0}'}{E_{0}} \right ) ^{2} 
\frac{ n' \cos{r}}{n \cos{i}}
\]

and the corresponding probability to be reflected will be $R=1-T$.

In case of reflection the relation between the incoming photon ($\vec{k},
\vec{e}$), the refracted one ($\vec{k}', \vec{e}\:'$)
and the reflected one ($\vec{k}'', \vec{e}\:''$) is given by
the following relations:

\begin{eqnarray*}
\vec{q} & = & \vec{k} \times \vec{u} \\
e_{\parallel} & = & \frac{\vec{e} \cdot \vec{u}}{|q|} \\
e_{\perp} & = & \frac{\vec{e} \cdot \vec{q}}{|q|} \\
e_{\parallel}' & = & e_{\parallel} \frac{2 n \cos{i}} {n' \cos{i} + n \cos{r}}\\
e_{\perp}' & = & e_{\perp} \frac{2 n \cos{i}} {n \cos{i} + n' \cos{r}} \\
e_{\parallel}'' & = & \frac{n'}{n} e_{\parallel}'  - e_{\parallel} \\
e_{\perp}'' & = & e_{\perp}' - e_{\perp}
\end{eqnarray*}

After transmission or reflection of the photon, the polarisation vector
is renormalised to 1.
In the case where $\sin{r} = n \: \sin i  /n' > 1$ then there cannot
be a refracted wave, and in this case we have a total internal reflection 
according to the following formulas:

\begin{eqnarray*}
\vec{k}'' & = & \vec{k} - 2 (\vec{k} \cdot \vec{u}) \vec{u} \\
\vec{e}\:'' & = & -\vec{e} + 2 (\vec{e} \cdot \vec{u}) \vec{u}
\end{eqnarray*}

\subsubsection{Case dielectric $\rightarrow$ metal}

In this case the photon cannot be transmitted. So the probability for the
photon to be absorbed by the metal is estimated according to the table
provided by the user. If the photon is not absorbed, it is reflected.

\subsection{Surface effects}

In the case where
the surface between two bodies is perfectly polished, then the
normal provided by the program is the normal to the surface defined by
the body boundary. This is indicated by the the value {\tt POLISH}$=1$
as returned by the \Rind{GUPLSH} function. When the value returned is 
$< 1$, then a random point is generated in a sphere of radius 
$1-${\tt POLISH}, and the corresponding vector is added to the normal.
This new normal is accepted if the reflected wave is still inside the
original volume.

\section{Subroutines}

\Shubr{GSCKOV}{(ITMED, NPCKOV, PPCKOV, ABSCO, EFFIC, RINDEX)}

\begin{DLtt}{MMMMMMMMMM}
\item[ITMED] ({\tt INTEGER}) tracking medium for which the optical
properties are to be defined;
\item[NPCKOV] ({\tt INTEGER}) number of bins in the tables;
\item[PPCKOV] ({\tt REAL}) array containing {\tt NPCKOV} values
of the photon momentum in GeV;
\item[ABSCO] ({\tt REAL}) array containing {\tt NPCKOV} values
of the absorption length in centimeters in case of dielectric and of the
boundary layer absorption probabilities in case of a metal;
\item[EFFIC] ({\tt REAL}) array containing {\tt NPCKOV} values of the
detection efficiency;
\item[RINDEX] ({\tt REAL}) array containing {\tt NPCKOV} values of the
refractive index for a dielectric, if {\tt RINDEX(1) = 0} the material
is a metal;
\end{DLtt}

This routine declares a tracking medium either as a radiator or as a
metal and stores the tables provided by the user. In the case of a 
metal the {\tt RINDEX} array does not need to be of length {\tt NPCKOV},
as long as it is set
to 0. The user should call this routine if he wants to use \v{C}erenkov
photons. Please note that for the moment only 
{\tt BOX}es, {\tt TUBE}s, {\tt CONE}s, {\tt SPHE}res, {\tt PCON}s,
{\tt PGON}s, {\tt TRD2}s and {\tt TRAP}s can be assigned optical properties
due to the current limitations of the \Rind{GGPERP} routine described
below.

\Shubr{GLISUR}{(X0, X1, MEDI0, MEDI1, U, PDOTU, IERR)}

\begin{DLtt}{MMMMMMMMMM}
\item[X0] ({\tt REAL}) current position ({\tt X0(1)=$x$},
{\tt X0(2)=$y$}, {\tt X0(3)=$z$})
and direction ({\tt X0(4)=$p_{x}$}, {\tt X0(5)=$p_{y}$},
{\tt X0(6)=$p_{z}$})
of the photon at the boundary of a volume;
\item[X1] ({\tt REAL}) position ({\tt X1(1)=x, X1(2)=y, X1(3)=z})
beyond the boundary of the current volume,
just inside the new volume along the direction of the photon;
\item[MEDI0] ({\tt INTEGER}) index of the current tracking medium;
\item[MEDI1] ({\tt INTEGER}) index of the tracking medium into which the
photon is entering;
\item[U] ({\tt REAL}) array of three elements containing the normal to
the surface to which the photon is approaching;
\item[PDOTU] ({\tt REAL}) $-\cos \theta$ where $\theta$ is the angle between
the direction of the photon and the normal to the surface;
\item[IERR] ({\tt INTEGER}) error flag, \Rind{GGPERP} could not determine
the normal to the surface if {\tt IERR} $\neq$ {\tt 0};
\end{DLtt}

This routine simulates the surface profile between two media as seen by
an approaching particle with coordinate and direction given by {\tt X0}.
The surface is identified by the arguments {\tt MEDI0} and {\tt MEDI1}
which are the tracking medium indices of the region in which the track
is presently and the one which it approaches, respectively. The input
vector {\tt X1} contains the coordinates of a point on the other side of
the boundary from {\tt X0} and lying in within medium {\tt MEDI1}. The
result is a unit vector {\tt U} normal to the surface of reflection at
{\tt X0} and pointing into the medium from which the track is approaching.

The quality of the surface finish is given by the parameter returned by
the user function \Rind{GUPLSH} (see below).

\Sfunc{GUPLSH}{VALUE = GUPLSH(MEDI0, MEDI1)}

This function must be supplied by the user. It returns a value between 0
and 1 which decsribes the quality of the surface finish between {\tt MEDI0}
and {\tt MEDI1}. The value 0 means maximum roughness with effective plane of
reflection distributed as $\cos \alpha$ where $\alpha$ is the angle
between the unit normal to the {\it effective} plane of reflection and
the normal to the nominal medium boundary at {\tt X0}. The value 1 means
perfect smoothness. In between the surface is modelled as a bell-shaped
distribution in $\alpha$ with limits given by:
\[
\sin \alpha = \pm (1-\mbox{\tt GUPLSH})
\]

At the interface between two media the routine is called to evaluate the
surface. The default version in {\tt GEANT} returns $1$, i.e. a perfectly
polished surface is assumed. When {\tt GUPLSH = 0} the distribution of the
normal to the surface is $\approx \cos \theta$.
\begin{DLtt}{MMMMMMMMMM}
\item[MEDI0] ({\tt INTEGER}) index of the current tracking medium;
\item[MEDI1] ({\tt INTEGER}) index of the tracking medium into which the
photon is entering;
\end{DLtt}

\Shubr{GGCKOV}{}

This routine handles the generation of \v{C}erenkov photons and is called
from \Rind{GTHADR}, \Rind{GTMUON} and \Rind{GTELEC} in radiative
materials for which
the optical characteristics have been defined via the routine \Rind{GSCKOV}.

\Shubr{GSKPHO}{(IPHO)}

\begin{DLtt}{MMMMMMMMMM}
\item[IPHO] ({\tt INTEGER}) number of the \v{C}erenkov photon to store
on the stack to be tracked. If {\tt IPHO = 0} all the generated photons
will be put on the stack to be tracked.
\end{DLtt}

This routines takes the \v{C}erenkov photon {\tt IPHO} generated during
the current step and stores it in the stack for subsequent tracking.
This routine performs for \v{C}erenkov photons the same function that
the \Rind{GSKING} routine performs for all the other particles. The
generated photons are stored in the common \FCind{/GCKIN2/} ({\tt 
[BASE030]}).

\Shubr{GTCKOV}{}

This routine is called to track \v{C}erenkov photons. The user routine
\Rind{GUSTEP} is called at every step of tracking. When {\tt ISTOP} = 2
the photon has been absorbed. If {\tt DESTEP} $\neq 0$ then the photon
has been detected.

\Shubr{GGPERP}{(X, U, IERR)}

\begin{DLtt}{MMMMMMMMMM}
\item[X] ({\tt REAL}) array of dimension 3 containing the current position
of the track in the MARS;
\item[U] ({\tt REAL}) array of dimension 3 containing on exit the normal to
the surface of the current volume at the point {\tt X};
\item[IERR] ({\tt INTEGER}) error flag: if {\tt IERR} $\neq 0$ \Rind{GGPERP}
failed to find the normal to the surface of the current volume.
\end{DLtt}

This routine solves the general problem of calculating the unit vector
normal to the surface of the current volume at the point X. The result
is returned in the array {\tt U}. X is assumed to be outside the current
volume and near a boundary
(within {\tt EPSIL}). The current volume is indicated by the common
\FCind{/GCVOLU/}.  U points from inside to outside in that
neighbourhood. If {\tt X} is within {\tt EPSIL} of more than one boundary
(i.e. in a corner) an arbitrary choice is made.
If {\tt X} is inside the current volume
or if the current volume is not handled by the routine,
the routine returns with
{\tt IERR=1}, otherwise {\tt IERR=0}. At the moment the routine only
handles {\tt BOX}es, {\tt TUBE}s, {\tt SPHE}res and {\tt CONE}s.

\section{Processes involving \v{C}erenkov photons}

The process of generating a \v{C}erenkov photon is called {\tt CKOV}
and corresponds to the process value 105 (variable {\tt LMEC} in
common \FCind{/GCTRAK/}).
This process is activated only in a radiator defined via the routine
\Rind{GSCKOV}.

The process of photon absorption (name {\tt LABS}, code 101) is controlled
by the {\tt LABS} {\tt FFREAD} data record. By default the process is
activated for all the materials for which optical properties have been
defined.

The action taken at the boundary is identified by the process name {\tt LREF},
code 102.

At a boundary a photon can be either reflected (name {\tt REFL}, code 106)
or refracted (name {\tt REFR}, code 107).
