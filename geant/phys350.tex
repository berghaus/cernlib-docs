%%%%%%%%%%%%%%%%%%%%%%%%%%%%%%%%%%%%%%%%%%%%%%%%%%%%%%%%%%%%%%%%%%%
%                                                                 %
%  GEANT manual in LaTeX form                                     %
%                                                                 %
%  Version 1.00                                                   %
%                                                                 %
%  Last Mod. 14 June 1993 1300   MG                               %
%                                                                 %
%%%%%%%%%%%%%%%%%%%%%%%%%%%%%%%%%%%%%%%%%%%%%%%%%%%%%%%%%%%%%%%%%%%
\Origin{L.Urb\'{a}n}
\Revision{G.Azuelos}
\Version{Geant 3.16}\Routid{PHYS350}
\Submitted{26.10.84}  \Revised{16.12.93}
\Makehead{Total cross-section for e+e- annihilation}
\section{Subroutines}
\Shubr{GANNII}{}
{\tt GANNII} tabulates the mean free path at initialisation as a 
function of the medium and of the energy (see {\tt JMATE} data structure).
The energy binning is set within the array {\tt ELOW} (common block 
\FCind{/GCMULO/})
in the routine \Rind{GPHYSI}. \Rind{GANNII} is called from \Rind{GPHYSI}.

\section{Method}
For the annihilation into two photons the cross-section formula 
of Heitler is used. The total 
cross-section is given by~\cite{bib-EGS3,bib-HEIT}

\begin{eqnarray}
\sigma_{2\gamma} (Z,E) & = & \frac{Z \pi r^2_0}{\gamma +1}
      \left[\frac{\gamma^2 + 4 \gamma +1}{\gamma^2 -1}
      \ln \left(\gamma +\sqrt{\gamma^2 -1} \right)  -\frac
       {\gamma +3}{\sqrt{\gamma^2 -1}} \right]  \\
r_0 & = & \mbox{classical electron radius} \nonumber
\end{eqnarray}

For compounds, the cross-section is calculated using an effective
atomic number as explained in {\tt [PHYS010]}.

Positrons can annihilate in a single photon if the electron
with which it interacts is bound to a nucleus.
The total cross-section for such a process is

\begin{eqnarray}
\sigma_{1\gamma} (Z,E) & = & 4 \pi r_0^2 \alpha^4 \frac{Z^5}
        {\gamma \beta (\gamma + 1)^2}
        \left[ \gamma^2 + \frac{2}{3}\gamma + \frac{4}{3}
        + \frac{\gamma + 2}{\gamma \beta} \ln(\gamma + \gamma \beta) \right] \\
\alpha & = & \mbox{fine structure constant} \nonumber 
\end{eqnarray}

In the derivation of this formula, only the interactions with
the K-shell electrons are taken into account.
As the cross-section depends on $Z^5$, a special value of $Z_{eff}$
is computed in {\tt GPROBI}:
\begin{equation}
Z_{eff} = \sum_i \frac{p_i}{A_i} Z^5
\end{equation}
The notation of {\tt [PHYS010]} is used.

The total cross-section for the positron annihilation is
$\sigma = \sigma_{2\gamma} + \sigma_{1\gamma}$.
The value of $\sigma_{1\gamma}$ is at it largest for heavy
materials, for example, it is $\sim$ 20 \% of $\sigma$ for a positron
of 440 keV of kinetic energy in lead. For lower and higher
energies the probability is lower.
