%%%%%%%%%%%%%%%%%%%%%%%%%%%%%%%%%%%%%%%%%%%%%%%%%%%%%%%%%%%%%%%%%%%
%                                                                 %
%  GEANT manual in LaTeX form                              %
%                                                                 %
%  Michel Goossens (for translation into LaTeX)                   %
%  Version 1.00                                                   %
%  Last Mod. Jan 24 1991  1300   MG + IB                          %
%                                                                 %
%%%%%%%%%%%%%%%%%%%%%%%%%%%%%%%%%%%%%%%%%%%%%%%%%%%%%%%%%%%%%%%%%%%
\Origin{R.Brun,A.McPherson}
\Submitted{15.08.83}   \Revised{16.11.93}
\Version{Geant 3.16}   \Routid{GEOM100}
\Makehead{Creation of a volume}
 
\Shubr{GSVOLU}{(CHNAME,CHSHAP,NMED,PAR,NPAR,IVOLU*)}
Defines a volume with a given name, shape, tracking medium number
and shape parameters.

\begin{DLtt}{MMMMMMMM}
\item[CHNAME]  ({\tt CHARACTER*4}) volume name -- it must be unique;
\item[CHSHAP] ({\tt CHARACTER*4}) name of one of the {\tt GEANT} shapes;
\item[NMED]  ({\tt INTEGER}) tracking medium number for the volume --
if other volumes are positioned within this one, their tracking medium
replaces the one of the mother;
\item[PAR]   ({\tt REAL}) array containing the shape parameters;
\item[NPAR]  ({\tt INTEGER}) number of parameters -- if zero, then the
volume must be positioned via the routine \Rind{GSPOSP} indicating
the parameters for each copy (see {\tt [GEOM120]});
\item[IVOLU] ({\tt INTEGER}) internal volume number -- if $\leq 0$
an error condition has happened.
\end{DLtt}
 
If one of the parameters expressing a length is negative,
{\tt GEANT} will try to maximise its value in each copy, without
extending beyond the limits of the mother. This
facility can be used in conjunction with the
division of volumes using \Rind{GSDVN}, \Rind{GSDVT}, and \Rind{GSDVX}.
Restriction apply to this facility:
\begin{itemize}
\item the daughter should be positioned without a rotation matrix;
\item all shapes can be positioned within a similar shape with negative
parameters, but only in the centre of the mother;
\item boxes with negative parameters can be positioned in: box, {\tt TRD1},
{\tt TRD2} and {\tt TRAP} with $\theta = \phi = 0$;
\item {\tt HYPE}, {\tt ELTU} and {\tt CTUB} cannot have negative parameters;
\item {\tt PCON} and {\tt PGON} can only have 2 z planes.
\end{itemize}
{\bf Examples}
\begin{verbatim}
      DIMENSION PECAL(3)
      DATA PECAL/1.71,4.,0.2/
 
      CALL GSVOLU('ECAL','BOX ',1,PECAL,3,IVOLU)
\end{verbatim}
\Shubr{GPVOLU}{(IVOLU)}
Prints the volume parameters:
\begin{DLtt}{MMMM}
\item[IVOLU]  ({\tt INTEGER}) {\tt GEANT} number of the volume to print -- if
{\tt IVOLU} $\leq 0$, all volumes will be printed;
\end{DLtt}
