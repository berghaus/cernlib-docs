%%%%%%%%%%%%%%%%%%%%%%%%%%%%%%%%%%%%%%%%%%%%%%%%%%%%%%%%%%%%%%%%%%%
%                                                                 %
%  GEANT manual in LaTeX form                                     %
%                                                                 %
%  Michel Goossens (for translation into LaTeX)                   %
%  Version 1.00                                                   %
%  Last Mod. Jan 24 1991  1300   MG + IB                          %
%                                                                 %
%%%%%%%%%%%%%%%%%%%%%%%%%%%%%%%%%%%%%%%%%%%%%%%%%%%%%%%%%%%%%%%%%%%
\Origin{G.Lynch}
\Documentation{F.Carminati}   
\Submitted {23.08.93}   \Revised {16.12.93}
\Version{Geant 3.16}    \Routid{PHYS328}
\Makehead {Plural scattering}
\section{Subroutines}

\Shubr{GMCOUL}{(OMEGA,DIN*)}
\begin{DLtt}{MMMMMMMM}
\item[OMEGA] ({\tt REAL}) $\Omega_0$ of the Moli\`ere theory;
\item[DIN]  ({\tt REAL}) array of dimension 3 containing the
direction of the scattered particle with respect to the
incoming direction;
\end{DLtt}
Generates single scatters in small angle approximation returning
the new direction with respect to the old one.
It is called by \Rind{GMULTS} when $\Omega_0 \leq 20$.

\section{Method} 

In the Moli\`{e}re theory the average number of Coulomb scatters 
for a charged particle in a step is 
expressed by the parameter $\Omega_0$ (see {\tt [PHYS325]}).
When 
$\Omega_0 \leq 20$, the Moli\`{e}re theory is not applicable any
more, even if it has been noted \cite{bib-THEO} that it still gives reasonable
results down to its {\it mathematical limit} $\Omega_0 = e$.
The range $1<\Omega_0 \leq 20$ is called the {\it plural
scattering} regime.
An interesting study of this regime can be found in \cite{bib-KEIL}.

In {\tt GEANT}, when $\Omega_0 \leq 20$, a direct simulation method
is used for the scattering angle. 
The number of scatters is distributed according to a Poissonian law
with average $\Omega = k \Omega_0 = e^{2\gamma-1} \Omega_0 
\approx 1.167 \Omega_0$ with $\gamma = 0.57721\dots$ Euler's constant.
Using the customary notations for
the probability distribution function for small
angle ($\sin \theta \approx \theta$) single scattering, we
can write:

\begin{equation}
\label{eq:phys328-1}
f(\theta) \theta d\theta = \frac{d \sigma}{\theta d\theta} \frac{1}
{\sigma} \theta d\theta
\end{equation}

where $\sigma$ is the cross section for single elastic scattering.
We use as cross section the one reported by Moli\`{e}re \cite{bib-MOLI}
\cite{bib-MOL1}:

\[
\frac{d \sigma}{\theta d\theta} = 2 \pi \left ( \frac{2 Z Z_{inc}
e^2}{p v} \right ) ^2 \frac{1}{(\theta^2 + \chi_{\alpha}^2)^2}
\]

This is the classical Rutherford cross section corrected by the
screening angle $\chi_{\alpha}$. This angle is described by Moli\`{e}re
as a correction to the Born approximation used to derive the
quantum mechanical form of the Rutherford cross section. We have then:

\[
\sigma = \int_{0}^{\infty}{\frac{d \sigma}{\theta d\theta} \theta d\theta}
= 2 \pi \left ( \frac{2 Z Z_{inc}e^2}{p v} \right ) ^2
\int_{0}^{\infty}{\frac{\theta d \theta}{(\theta^2 + \chi_{\alpha}^2)^2}} =
\pi \left ( \frac{2 Z Z_{inc}e^2}{p v} \right ) ^2 
\frac{1}{\chi_{\alpha}^2}
\]

and so equation (\ref{eq:phys328-1}) becomes:

\[
f(\theta) \theta d\theta = \chi_{\alpha}^2 \frac{2 \theta d \theta}
{(\theta^2 + \chi_{\alpha}^2)^2} = \frac{ 2 \Theta d \Theta}{(1+\Theta^2)^2}
\]

where we have set $\Theta = \theta / \chi_{\alpha}$. To sample from this
distribution we calculate the inverse of the distribution function:

\[
\eta = \int_{0}^{\Theta}{\frac{2 t d t}{(1+t^2)^2}} 
 =  1 - \frac{1}{1+\Theta^2} \hspace{1cm} \Rightarrow \hspace{1cm}
\Theta = \sqrt{\frac{1}{1-\eta}-1}
\]

where $\eta$ is a number uniformly distributed between 0 and 1. If we
observe that also $1-\eta$ is uniformly distributed between 0 and 1 and
we remember the definition of $\Theta$, we obtain:

\[
\theta = \sqrt{\chi_{\alpha}^2 \left (\frac{1}{\eta}-1 \right)}
\]

To calculate $\chi_{\alpha}^2$ we observe that:

\begin{eqnarray*}
\Omega_0 & = & \frac{\chi_{c}^2}{k \chi_{\alpha}^2} \\
\chi_{\alpha}^2 & = & \frac{\chi_{c}^2}{k \Omega_0} =
\frac{\chi_{cc}^2 Z_{inc}^2 t/E^2\beta^4}{k b_{c} Z_{inc}^2 t /\beta^2} =
\frac{\chi_{cc}^2}{k b_{c} p^2 c^2}
\end{eqnarray*}

where we have used the notations of {\tt [PHYS325]} and 
$k = 1.167$.
