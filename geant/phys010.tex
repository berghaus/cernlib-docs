%%%%%%%%%%%%%%%%%%%%%%%%%%%%%%%%%%%%%%%%%%%%%%%%%%%%%%%%%%%%%%%%%%%
%                                                                 %
% GEANT manual in LaTeX form                                      %
%                                                                 %
% Michel Goossens (for translation into LaTeX)                    %
% Version 1.00                                                    %
% Last Mod. Jan 24 1991 1300  MG + IB                             %
%                                                                 %
%%%%%%%%%%%%%%%%%%%%%%%%%%%%%%%%%%%%%%%%%%%%%%%%%%%%%%%%%%%%%%%%%%%
\hyphenation{bre-ms-strah-lung}
\Documentation{M. Maire, F.Carminati}
\Submitted{20.12.84}\Revised{26.07.93}
\Version{Geant 3.16}\Routid{PHYS010}
\Makehead{Compute the occurrence of a process}
\section {Principle}
The simulation of the processes which accompany the propagation of a
particle through the material of the detector
(e.g. bremsstrahlung,
$\delta$-rays production, Compton scattering and so on)
is performed by {\tt GEANT} in the following steps:
\begin{enumerate}
\item Fetch a new particle to be tracked (often called {\it track} or
{\it history}) from the stack supported by the link {\tt JSTAK} (see 
{\tt [TRAK399]}). This is done once at the beginning of each new track. 
The number of {\it interaction lengths} that the particle is going to 
travel, before undergoing each one of the possible discrete processes, 
is sampled at this point. These operations are done in the routine 
\Rind{GLTRAC}.
\item Evaluate the distance to the interaction point.
This is done by the individual tracking routines
(\Rind{GTGAMA}, \Rind{GTNEUT}, \Rind{GTHADR}, \Rind{GTNINO},
\Rind{GTMUON}, \Rind{GTHION} and \Rind{GTCKOV}) which control the 
tracking of particular particles.
The number of interaction lengths remaining to travel
before each of the
possible processes (often called {\it tracking mechanisms} or simply
{\it mechanisms}) is multiplied
by the inverse of the macroscopic cross-section
for that process in the current material (i.e. the interaction
length). This gives the distances that the particle has to travel before
each of the processes occurs in the current medium. 
The minimum among these numbers is the
{\it step} over which the particle will be transported. In addition to
the physics mechanisms, four
{\it pseudo-interactions} are taken into account in the calculation of
the step:
\begin{enumerate}
\item boundary crossing. The crossing of a volume boundary is treated
like a discrete process. A particle never crosses a boundary
during a step but rather stops there ({\tt NEXT} mechanism);
\item maximum step limit. For each tracking medium a value for the
maximum step can be specified by the user. Process {\tt SMAX};
\item maximum fraction of continuum energy loss, maximum angular
deviation in magnetic
field or maximum step for which the Moli\`ere formula, to simulate
multiple scattering is valid. These are continuous processes,
which introduce a limitation on the tracking step expressed by a
single variable (see section {\tt [PHYS325]} on \Rind{GMULOF}).
\item energy and time cut. Charged particles in matter are stopped when their
energy falls below their energy threshold or when their time of flight
exceeds the time cut;
\end{enumerate}
More information is given in the
individual sections explaining the implementation of the physical
processes.
\item Transport the particle either along a
straight line (if no magnetic field or for a
neutral particle) or along a helicoidal
path (for charged particles in magnetic field).
\item Update the energy of the particle if continuous energy loss was
in effect (charged particles in matter).
\item If a physical discrete process has been selected,
generate the final state of the interaction.
\item  If
the incident particle {\it survives} the interaction
(Compton, $\delta$-rays production, bremsstrahlung, direct pair
production by $\mu$ and $\mu$-nucleus interaction, hadronic elastic
scattering), sample
again the number of interaction lengths to travel before the next
event of the same kind. This is generally done by specialised routines:
\Rind{GMUNU}, \Rind{GCOMP}, \Rind{GBREM}, etc.
\item Update the number of interaction
lengths for all the processes and go back to $(2)$ till the particle
either leaves the detector or falls below its energy threshold or
beyond its time cut or disappears in an interaction.
\end{enumerate}
\section{Distance evaluation}
\subsection{The interaction length}
Let $\sigma(E,Z,A)$ be the total microscopic
cross section for a given interaction.
The mean free path, $\lambda$, for a particle to interact is given by:
\begin{equation}
\lambda  = \frac{1}{\Sigma}
\end{equation}
 
where $\Sigma$ is the macroscopic cross-section in $cm^{-1}$. This quantity
is given for an element by:
 
\begin{equation}
\Sigma  = \frac{N_{Av} \rho\sigma (E,Z,A)}{A}
\end{equation}
 
and for a compound or a mixture by:
 
\begin{equation}
\Sigma  = \frac{N_{Av}\rho \sum_i
          n_i\sigma(E,Z_i,A_i)}{\sum_i
          n_iA_i}
        = N_{Av} \rho \sum_i \frac{p_i}{A_i}\: \sigma(E,Z_i,A_i)
\end{equation}
 
\begin{tabular}[t]{ll}
$N_{Av}$         & Avogadro's number ($6.02486 \times 10^{23}$) \\
$Z$         & atomic number  \\
$A$         & atomic weight  \\
$\rho$    & density \\
$\sigma$  & total cross-section for the reaction \\
$n_i$   & \parbox[t]{14cm}{proportion by number of the $i^{th}$  
element in the material} \\
$p_i$ & \parbox[t]{14cm}{$=n_{i} A_{i}/ \sum_j n_jA_j$, proportion by 
weight of the $i^{th}$ element in the material}
\end{tabular}
 
For electromagnetic processes which depend linearly on the atomic number 
$Z$ we can write:

\begin{eqnarray*}
\Sigma(E)  & =  & N_{Av} \rho \sum_i \frac{p_i}{A_i}\: \sigma(E,Z_i) =
N_{Av} \rho \sum_i \frac{p_i}{A_i}\: Z_i f(E) \\
& = & N_{Av} \rho f(E) \sum_i \frac{p_i}{A_i}\: Z_i = N_{Av} \rho f(E) Z_{eff} \\
Z_{eff} & = & \sum_i \frac{p_i}{A_i}\: Z_i 
\end{eqnarray*}

the value above of $Z_{eff}$ is calculated by \Rind{GPROBI}.
This mean free path is tabulated at initialisation time as a function
of the kinetic energy of the particle, or, for hadronic interactions,
it is calculated at tracking time.

Cross sections are tabulated in the energy range defined as: 
$\mbox{\tt ELOW(1)} \leq E \leq \mbox{\tt ELOW(NEK1)}$
in {\tt NEK1} bins. These values can be redefined by the data record
{\tt RANGE}. Default values are $\mbox{\tt ELOW(1)} = 10 keV$,
$\mbox{\tt ELOW(NEK1)} = 10 TeV$ and $\mbox{\tt NEKBIN} = 
\mbox{\tt NEK1}-1 = 90$. {\tt NEKBIN} cannot be bigger than 199.
The array {\tt ELOW} is in the common \FCind{/GCMULO/}.

Numerically, if we measure the microscopic
cross section in $b$ where $1b=10^{-24}
cm^{-2}$, we can express the macroscopic cross section as:
\begin{eqnarray}
\Sigma [cm^{-1}]  & = & \frac{6.02486 \times 10^{23} \rho
[g \: cm^{-3}] \sigma (E,Z,A) [b] \times 10^{-24}}{A} \\
& = & 0.602486 \: \frac{\rho [g \: cm^{-3}]}{A} \: \sigma (E,Z,A) [b]
\end{eqnarray}

which is the formula mostly used in {\tt GEANT}.
\subsection{Determination of the interaction point}
The mean free path of a particle for a given process,
$\lambda$, depends on the medium and cannot be used
directly to sample the probability of an interaction in a heterogeneous
detector. The number of mean free paths which a particle travels is:
\[
N_\lambda =\int \frac{dx}{\lambda(x)}
\]
and it is independent of the material traversed. If $N_R$ is
a random variable denoting the number of mean free paths from a given
point until the point of interaction, it can be shown that $N_R$ has the
distribution function
\[
P( N_R < N_\lambda ) = 1-e^{-N_\lambda}
\]
The total number of mean free paths the particle travels before
the interaction point, $N_\lambda$, is sampled at the beginning
of the trajectory as:
\begin{displaymath}
N_\lambda = -\log \left ( \eta \right )
\end{displaymath}
where $\eta$ is a random number uniformly distributed
in the range $(0,1)$.
$N_\lambda$ is updated after each step $\Delta x$ according the formula:
\begin{displaymath}
N'_\lambda=N_\lambda -\frac{\Delta x }{\lambda(x)}
\end{displaymath}
until the step originating from $s(x) = N_\lambda \lambda(x)$ is
the shortest and this triggers the specific process.
\section{Common {\tt /GCPHYS/}}
The variables described above are stored in the common \FCind{/GCPHYS/},
one process per line:
\begin{verbatim}
      COMMON/GCPHYS/IPAIR,SPAIR,SLPAIR,ZINTPA,STEPPA
     +             ,ICOMP,SCOMP,SLCOMP,ZINTCO,STEPCO
     +             ,IPHOT,SPHOT,SLPHOT,ZINTPH,STEPPH
     +             ,IPFIS,SPFIS,SLPFIS,ZINTPF,STEPPF
     +             ,IDRAY,SDRAY,SLDRAY,ZINTDR,STEPDR
     +             ,IANNI,SANNI,SLANNI,ZINTAN,STEPAN
     +             ,IBREM,SBREM,SLBREM,ZINTBR,STEPBR
     +             ,IHADR,SHADR,SLHADR,ZINTHA,STEPHA
     +             ,IMUNU,SMUNU,SLMUNU,ZINTMU,STEPMU
     +             ,IDCAY,SDCAY,SLIFE ,SUMLIF,DPHYS1
     +             ,ILOSS,SLOSS,SOLOSS,STLOSS,DPHYS2
     +             ,IMULS,SMULS,SOMULS,STMULS,DPHYS3
     +             ,IRAYL,SRAYL,SLRAYL,ZINTRA,STEPRA
      COMMON/GCPHLT/ILABS,SLABS,SLLABS,ZINTLA,STEPLA
     +             ,ISYNC
     +             ,ISTRA
*
\end{verbatim}
The first 9 processes (from {\tt PAIR} production up to {\tt MU}on
{\tt NU}clear interaction) and the {\tt RAYL}eigh scattering and
{\tt L}ight {\tt ABS}orbtion
have the same scheme. Let's take as an example
the first one (for a complete description of the common see
{\tt [BASE030]}):
\begin{DLtt}{MMMMM}
\item[IPAIR ]flag for secondaries:
\begin{DLtt}{MMM}
\item[0=] the process is turned off;
\item[1=] generation of secondaries enabled;
\item[2=] no generation of secondaries.
\end{DLtt}
\item [SPAIR ] $ N_\lambda \lambda (x) \: = \: $
remaining track-length before interaction, evaluated at the last point
where the mechanism was active, i.e. {\tt IPAIR}$ \neq 0$.
\item[SLPAIR] track length at the time when the interaction last
happened for the current particle. Only {\tt SLPAIR}
(direct pair production by $\mu$), {\tt SLDRAY} ($\delta$-ray production),
{\tt SLBREM} (bremsstrahlung for $\mu$), {\tt SLHADR} (hadronic
interactions), {\tt SLMUNU} (muon-nucleus interactions) and {\tt SLRAYL}
(Rayleigh effect) are used.  The variables
{\tt SOLOSS}, {\tt STLOSS}, {\tt SOMULS}, {\tt STMULS} are obsolete. 
They have been kept for backward compatibility, but their value
is undefined and should not be used.
\item[ZINTPA] $N_\lambda \: = \: $remaining number of interaction
lengths (mean free paths)
evaluated at the last point where the mechanism was active, i.e.
{\tt IPAIR}$ \neq 0$.
\item[STEPPA] $ \lambda (x) \: = \:$value of the interaction length at the
last point where the mechanism was active, i.e. {\tt IPAIR}$ \neq 0$;
\end{DLtt}
The evaluation and update of the quantities like {\tt STEPPA, SPAIR} and
{\tt ZINTPA} are turned off in the media where the mechanism is not
active ($\mbox{\tt IPAIR} \neq 0$). Turning off a mechanism in one 
tracking medium may give incorrect physics results because
not only will the mechanism not be active, but the interaction
probabilities will not be updated, as if that medium had not been
traversed at all. This
feature of the tracking routines is used mainly in the vacuum, (defined
as a medium with atomic number $\mbox{\tt Z} < 1$), where all the mechanisms
but {\tt DECA}y (and {\tt SYNC}hrotron radiation, if activated) are inactive.
 
The {\tt DECA}y in flight is simpler since the mean life time of
the particle, $\tau$, is not material dependent and can be
sampled directly.
\begin{DLtt}{MMMMM}
\item[SLIFE] not used.
\item[SUMLIF] proper time left before the decay. At the beginning of the
track $\mbox{\tt SUMLIF} \: = \: -c\tau \log \left  ( \eta \right )$.
\item[SDCAY] distance left to decay point evaluated at the last point
where the mechanism was active, i.e. $\mbox{\tt IDECA} \neq 0$.
\end{DLtt}
 
\section {Cross-section, energy loss and range tables}
Cross-sections, energy loss $dE/dx$ and range $R(E_{kin})$
are tabulated for all materials which enter in the definition of
a tracking medium by the routine \Rind{GPHYSI}.
The values of the energy for which the tabulated quantities are calculated
are stored in the common \FCind{/GCMULO/} (see {\tt [BASE030]}).
To evaluate one of the tabulated quantities for a particle of {\it kinetic}
energy $E_0$, a linear interpolation is used. Let $i$ be such that:
\[
E_i  < E_0 \leq E_{i+1}
\]
The integer variable
{\tt IEKBIN} in common \FCind{/GCTRACK/} is equal to $i$ during
tracking and its value is recomputed by the routine \Rind{GEKBIN} when
the energy of the particle changes. If the quantity $Y$ has been tabulated
so that $Y_i = Y(E_i)$ then the value $Y_0 = Y(E_0)$ is calculated as:

\begin{equation}
Y_0 = 
Y_i + \frac{E_0-E_i}{E_{i+1}-E_i} \left( Y_{i+1}-Y_i \right)
=  Y_i 
\left ( 1 - \frac{E_0-E_i}{E_{i+1}-E_i} \right ) +
Y_{i+1} \frac{E_0-E_i}{E_{i+1}-E_i}
\end{equation}

Inside the code the following quantities are used:

\[
\mbox{\tt GEKRAT} = 
\frac{E_0-E_i}{E_{i+1}-E_i} \hspace{4cm}
\mbox{\tt GEKRT1}  =  
\left ( 1 - \frac{E_0-E_i}{E_{i+1}-E_i} \right )
\]

where {\tt GEKRAT} is in common \FCind{/GCTRAK/} and {\tt GEKRT1} 
is a local variable recomputed when needed.

\section {The energy loss tables}
 
Energy loss and multiple
scattering are continuous processes that are applied at every step for
charged particles in matter ($\mbox{\tt Z} \geq 1$). 
 
As explained in {\tt [PHYS330]} and {\tt [PHYS430]},
energy loss tables are calculated at
initialisation time (\Rind{GPHYSI}) for all the materials which
enter in the definition of a tracking medium (see {\tt [CONS200]}).
These tables contain {\tt NEK1}
values of $dE/dx$ calculated for the corresponding values of energy
in {\tt ELOW}.
 
In case of mixtures/compounds, the rule~\cite{bib-PDGD} is to combine
the energy loss tables in GeV g$^{-1}$ cm$^{2}$ according to
the proportion by weight of the elements, that is:

\[
\frac{dE}{dx} = \rho \sum_{i}{\frac{p_{i}}{\rho_{i}} \left (
\frac{dE}{dx} \right )_{i}}
\]

\section{Limitations on the step size}
The routine \Rind{GMULOF} called by \Rind{GPHYSI} creates and fills
a table of {\tt NEK1} values corresponding to the {\tt ELOW} values 
containing the smaller of the upper limits for the step 
imposed by the three continuous processes: energy
loss, multiple scattering and bending of the track induced by
the magnetic field.
 
Continuous energy loss can introduce an upper limit on the step via the
variable {\tt DEEMAX}, an argument to the \Rind{GSTMED} routine. During 
tracking the value of {\tt DEEMAX} for the current medium is stored in 
the common \FCind{/GCMATE/}. {\tt DEEMAX} is the maximum fraction of 
kinetic energy which a particle can lose in a step due to continuous 
ionisation ($0<\mbox{\tt DEEMAX}<1$). 
The limitation on the step size coming from {\tt DEEMAX} is:
\[ 
step \leq \frac{\mbox{\tt DEEMAX}}{dE/dx}
\]
 
Multiple scattering as well can limit the step size, see
{\tt [PHYS325]}. The limitation is given as:
\[ 
step \leq \min \left( T_{Bethe}, 10 \, X_0 \right)
\]
where $X_0$ is the radiation length and $T_{Bethe}$ is the maximum step
for which the Moli\`ere approximation is valid (see {\tt [PHYS325]}).
 
Another upper limit on the step size comes from the
magnetic field. The bending of the particle trajectory
in the magnetic field may be
limited by the {\tt TMAXFD} argument to the \Rind{GSTMED} routine.
During tracking the
value of {\tt TMAXFD} for the current medium is
stored in the common \FCind{/GCMATE/}.
 
A lower limitation on the tracking step is not generally imposed. There
is, however, a protection against the step being reduced to a very small
value by continuous processes. In particular multiple scattering at low
energies ($< 1$ MeV) can impose a very small tracking step with serious
consequences on the tracking time. To avoid this, a lower limit on the
step imposed by continuous processes is introduced: {\tt STMIN}.
The meaning of {\tt STMIN} is the following:
below 1 MeV
the stopping range is usually small. If the
stopping range becomes smaller than {\tt STMIN}, the constraint imposed
by the multiple scattering is ignored and the minimum is taken between
the reduced stopping range (the distance the particle has to travel
to reach its threshold energy) and {\tt STMIN} itself. In this sense 
{\tt STMIN} is no more than a tracking accelerator for stopping 
particles.
 
Another limitation on the step size which is imposed by the tracking
routines during transport is the {\tt STEMAX} parameter of
\Rind{GSTMED} which sets an absolute upper limit to the size of a
step for each tracking medium.
 
\section{Automatic calculation of parameters}
The definition of a {\it tracking medium} requires the specification
by the user of a set of parameters (see {\tt [CONS200]}) which can
critically affect the tracking and hence the physics results of the
{\tt GEANT} MonteCarlo. To help the user to find the optimal set of
parameters, by default {\tt GEANT} overrides the values of
{\tt STMIN, DEEMAX, STEMAX} and {\tt TMAXFD}. This behaviour is
controlled by the {\tt AUTO} data record and interactive command. By
default {\tt AUTO=1} and automatic evaluation of the parameters
is enforced (see below for the partly
anomalous behaviour of {\tt TMAXFD}). When {\tt AUTO}$=0$ then only those
parameters which are $ < 0$ are recalculated by {\tt GEANT}, while
for the others the user input is accepted with minimal checking.
 
When the automatic calculation of parameters is active, the following
applies:

\begin{DLtt}{MMMMMM}
\item[DEEMAX] The formula used by {\tt GEANT} is the following: \\

\[ \mbox{\tt DEEMAX} \: = \: \left\{ \begin{array}{L@{\hspace{1cm}}cLcL}
  0.25 & \mbox{if} & \mbox{\tt ISVOL}=0 & \mbox{and} &
                       X_0<2 \: \mbox{cm} \\
  0.25-\frac{0.2}{\sqrt{X_0}} & \mbox{if} & \mbox{\tt ISVOL}=0 & \mbox{and} &
                       X_0 \geq 2 \: \mbox{cm} \\
  \frac{0.2}{\sqrt{X_0}} & \mbox{if} & \mbox{\tt ISVOL} \neq 0 & &  \\[4mm]
                   \end{array} \right .
\]

$\mbox{\tt ISVOL} > 0$ defines a {\it sensitive} detector, while 
$\mbox{\tt ISVOL} \leq 0$ is a non-{\it sensitive} detector (see {\tt [HITS]}).
 
\item[STMIN] The default value corresponds to a stopping 
range of 200 keV above {\tt CUTELE}.
 
\item[TMAXFD] The default value corresponds to $20^{\circ}$ if the
input value to \Rind{GSTMED} is $\leq 0$ or $> 20^{\circ}$, otherwise
the input value is taken.
 
\item[STEMAX] The default value is $10^{10}$ ({\tt BIG} variable
in common \FCind{/GCONST/}).
\end{DLtt}

These values have been tuned empirically on a variety of setups and
users are invited to start with automatic computation. The values of the
parameters can be checked with a call to {\tt GPRINT('TMED',0)} after
\Rind{GPHYSI} has been called. Use of {\tt AUTO} mode is strongly
recommended because it makes {\tt GEANT} a predictive tool. In this way
all parameters are automatically computed by the system as opposed to
tuning data and MonteCarlo via the tracking parameters.
 
\section{Energy loss}
\subsection{Energy loss tables}
In previous versions of {\tt GEANT} (up to version 3.13) the energy lost by
a charged particle travelling through matter was calculated by the
formula:
\[ \Delta E = \frac{dE}{dx} \: \times \: step
\]
The quantity $dE/dx$ was calculated and tabulated for $e^{\pm}$,
$\mu^{\pm}$ and protons and the value for a particle of a given energy
$E_0$ was calculated interpolating linearly the tables:
\begin{eqnarray}
\left. \frac{dE}{dx} \right|_{E=E_0} & = &
\left. \frac{dE}{dx} \right|_{E=E_i} + \frac{E_0-E_i}{E_{i+1}-E_i}
\left ( \left. \frac{dE}{dx} \right|_{E=E_{i+1}} -
\left. \frac{dE}{dx} \right|_{E=E_i} \right ) \vspace{.3cm} \\[5mm]
& = & \mbox{\tt GEKRT1} \times \left. \frac{dE}{dx} \right|_{E=E_i} +
\mbox{\tt GEKRAT} \times \left. \frac{dE}{dx} \right|_{E=E_{i+1}}
\hspace{2.5cm} (E_i < E_0 \leq E_{i+1})
\nonumber
\end{eqnarray}
These formula work of course on the assumption that the error is
negligible. But the error on the linear approximation is ${\cal O}
\left ( \partial^2E/\partial x \; \partial E \right )$ which is 
far from negligible at
low energy. In this way the energy loss at low energies was constantly
underestimated and the particles could travel too far before stopping.
 
\subsection{Stopping range tables}
To correct for this problem, a different approach was introduced in
{\tt GEANT} version 3.14. The {\it stopping range} of a particle is
defined as the distance that the particle will travel before stopping.
By definition the stopping range for a particle of energy $E_0$ is given
by:
\[ R = \int_{E_0}^{0}{\frac{dx}{dE} \: dE} =
\int_0^{E_0}{-\frac{dx}{dE} \: dE}
\]
Note that in the tables the positive quantity $-dE/dx$ is stored. The
method used was to build a table of stopping ranges based on {\tt ELOW}
by integrating the inverse of the $dE/dx$ tables in \Rind{GRANGI}.
At tracking time the algorithm was the following:
\begin{enumerate}
\item Evaluate the stopping range for the threshold energy ({\tt STOPC}).
This was done only once at the beginning of each new track.
\item From the energy of the particle derive the stopping range by
a linear interpolation of the range table:
\[ R_0 = \mbox{\tt GEKRT1} \times R_i + \mbox{\tt GEKRAT} \times R_{i+1} \]
where $E_i < E_0 \leq E_{i+1}$.
\item Evaluate the stopping range for the particle after the step:
$R'_0 = R_0 - step$. If this is less than the stopping range of a
particle with threshold energy, the particle is terminated as a stopping
particle below the energy cut. Otherwise the following quantities are
evaluated:
\[
R_j < R'_0 \leq R_{j+1} \hspace{2cm}
\mbox{\tt GEK} = \displaystyle{\frac{R'_0-R_j}{R_{j+1}-R_j}}  \hspace{2cm}
\mbox{\tt GEK1} = 1 - \mbox{\tt GEK}
\]
and the final energy is computed as:
\[ E'_0 = \mbox{\tt GEK1} \times \mbox{\tt ELOW(}j\mbox{\tt )} +
          \mbox{\tt GEK} \times \mbox{\tt ELOW(}j\mbox{\tt +1)}
\]
\item the energy loss is computed as:
\[
\Delta E = E_0-E'_0 \]
This value is then corrected to take into account the energy loss
fluctuations (see {\tt [PHYS332]}).
\end{enumerate}
This method has two main disadvantages. The first is due to the finite
precision of computers. As the energy loss in a step is calculated as
the difference of two numbers, it is subject to large relative errors.
The effect can be particularly serious in the case of light materials,
particles near the minimum ionisation point or with very short steps, 
where $\Delta E = \mbox{\tt DESTEP}$ can even result in a negative
quantity. As the relative precision of a 32-bit computer is around
$10^{-6}$, the error on the energy loss of a 100 GeV
track can be as large as 100 keV.
 
The second problem connected with this method can be easily shown
if we compute $dE/dx$ as:
\[
\begin{array}{RCL}
\frac{dE}{dx} & = & \frac{d}{dx} \left(E_0 -E'_0 \right ) \: =  \:
\frac{d}{dx} \left(E_0 - \mbox{\tt GEK1} \times 
\mbox{\tt ELOW(}j\mbox{\tt )}
-\mbox{\tt GEK} \times \mbox{\tt ELOW(}j\mbox{\tt +1)} \right ) \\[5mm]
& = & \frac{d}{dx} \left(E_0 - E_j - \frac{R'_0-R_j}{R_{j+1}-R_j}
                    \left(E_{j+1}-E_j \right ) \right ) \:  =  \:
-\frac{d}{dx}\frac{\Delta E_j}{\Delta R_j} \left(R'_0-R_j \right) \\[5mm]
& = & -\frac{\Delta E_j}{\Delta R_j}\frac{dR'_0}{dx} \: =   \:
\frac{\Delta E_j}{\Delta R_j}\frac{step}{dx} \: = \:
\frac{\Delta E_j}{\Delta R_j} 
\end{array}
\]
As we can see, the reconstructed $dE/dx$ curve due to continuous energy loss
is a step function and constant in each energy bin. Thus, although the 
results 
obtained with {\tt GEANT 3.14} were very satisfactory, this was felt to be 
an undesirable feature.
 
\subsection{Energy loss in {\tt GEANT}}
The two problems mentioned above have been solved in {\tt GEANT 3.15}. As
far as the precision is concerned, the solution was to revert to the
algorithm of {\tt GEANT 3.13} every time the relative energy loss in the
step ({\tt DESTEP/GEKIN}) is smaller than five times the machine precision.
This has given good results without loosing the substantial improvements
obtained in {\tt GEANT 3.14} with the introduction of the stopping range
algorithm. As a matter of fact, the above condition happens only the 
in the case of 
very small steps or when the $dE/dx$ curve is very flat, and in both cases
the algorithm based on the $dE/dx$ tables is a good approximation.
 
The second problem has been solved by changing the interpolation algorithm,
going from a linear to a quadratic interpolation. Instead of assuming
a linear relation between energy and stopping range in every energy bin,
we assume a quadratic relation of the sort:
\[ E \: = \: f(R) \: = \: {\rm a} R^2 + {\rm b} R + {\rm c} \]
The only problem now is the determination of the coefficients a, b and
c. To do this we recall that the general formula of the parabola which
passes through the points $(x_1,y_1), (x_2,y_2), (x_3,y_3)$ is the
following:
\begin{eqnarray*}
y & = & \displaystyle{
\frac{(x-x_2)(x-x_3)}{(x_1-x_2)(x_1-x_3)}y_1 +
\frac{(x-x_1)(x-x_3)}{(x_2-x_1)(x_2-x_3)}y_2 +
\frac{(x-x_1)(x-x_2)}{(x_3-x_1)(x_3-x_2)}y_3} \vspace{.3cm} \\
& = & \displaystyle{
\makebox[.5cm][r]{$x^2$} \left ( \frac{y_1}{(x_1-x_2)(x_1-x_3)}+
            \frac{y_2}{(x_2-x_1)(x_2-x_3)}+
            \frac{y_3}{(x_3-x_1)(x_3-x_2)} \right ) -} \vspace{.3cm} \\
&   & \displaystyle{
\makebox[.5cm][r]{$x$} \left ( \frac{y_1(x_2+x_3)}{(x_1-x_2)(x_1-x_3)}+
            \frac{y_2(x_1+x_3)}{(x_2-x_1)(x_2-x_3)}+
            \frac{y_3(x_1+x_2)}{(x_3-x_1)(x_3-x_2)} \right ) +}
                                                       \vspace{.3cm} \\
&   & \displaystyle{
\makebox[.5cm][r]{} \left ( \frac{y_1x_2x_3}{(x_1-x_2)(x_1-x_3)}+
            \frac{y_2x_1x_3}{(x_2-x_1)(x_2-x_3)}+
            \frac{y_3x_1x_2}{(x_3-x_1)(x_3-x_2)} \right )} \vspace{.3cm}
\end{eqnarray*}
This allows us to calculate and tabulate the coefficients a, b and c
just by replacing $y_i$ by {\tt ELOW(I)} and $x_i$ by the corresponding
stopping range {\tt ELOW(}$i${\tt )}.
In the routine \Rind{GRANGI} a table of length $3 \times (\mbox{\tt NEKBIN}
-1)$
is created: \\
 
\begin{tabular}{r@{}lcrcl}
{\tt Q(JINTRP+3*(I-1)} & {\tt +1)} & = &
${\rm a}({\tt R_I,R_{I+1},R_{I+2},E_I,E_{I+1},E_{I+2}})$
& = & ${\rm A_I}$  \vspace{.3cm} \\
& {\tt +2)} & = &
$\displaystyle{0.5 \,
\frac{{\rm b}({\tt R_I,R_{I+1},R_{I+2},E_I,E_{I+1},E_{I+2}})}
{{\rm a}({\tt R_I,R_{I+1},R_{I+2},E_I,E_{I+1},E_{I+2}})}}$
& = & ${\rm B_I}$ \vspace{.3cm} \\
& {\tt +3)} & = &
$\displaystyle{
\frac{{\rm c}({\tt R_I,R_{I+1},R_{I+2},E_I,E_{I+1},E_{I+2}})}
{{\rm a}({\tt R_I,R_{I+1},R_{I+2},E_I,E_{I+1},E_{I+2}})}}$
& = & ${\rm C_I}$ \vspace{.3cm} \\
\end{tabular}

where $E_I = \mbox{\tt ELOW(I)}$ and $\tt R_I = \mbox{\tt R(ELOW(I))}$.
The calculation of the energy loss is now done in the following steps:
\begin{enumerate}
\item Evaluate the stopping range for the threshold energy ({\tt STOPC}),
only once at the beginning of each new particle tracking.
\item From the energy of the particle derive the stopping range by
a quadratic interpolation of the range table:
\[ R_0 = -{\rm B_I} + \frac{{\rm A_I}}{\left | {\rm A_I} \right |}
\: \sqrt{{\rm B_I}^2-\left ( {\rm C_I}-\frac{E_0}{{\rm A_I}}\right)}\]
The value of I is chosen according to the following. Let $i$ be such that
$\mbox{\tt ELOW(}{\rm i}\mbox{\tt )} < E_0 \leq \mbox{\tt ELOW(}{\rm i+1}
\mbox{\tt )}$:
\[ \mbox{\tt I} \: = \: \left\{ \begin{array}{LcL}
                    \max(i-1,1) & \mbox{if} &
\frac{E_0-\mbox{\tt ELOW(}{\rm i}\mbox{\tt )}}
{\mbox{\tt ELOW(}{\rm i+1}\mbox{\tt )}-
\mbox{\tt ELOW(}{\rm i}{\tt )}} < 0.7 \vspace{.3cm} \\
                    \min(i,\mbox{\tt NEKBIN}-1) & \mbox{if} &
\frac{E_0-\mbox{\tt ELOW(}{\rm i}\mbox{\tt )}}
{\mbox{\tt ELOW(}{\rm i+1}\mbox{\tt )}-
\mbox{\tt ELOW(}{\rm i}\mbox{\tt )}} \geq 0.7 \vspace{.3cm}
                  \end{array} \right .
\]
The value of $0.7$ is an empirical number which minimises the 
discontinuities of the reconstructed $dE/dx$ curve.
\item Evaluate the stopping range for the particle after the step:
$R'_0 = R_0 - step$. If this is less than the stopping range of a
particle of threshold energy, the particle is terminated as a stopping
particle below the energy cut. Otherwise, the final energy is computed as:
\[ E'_0 = {\rm A_I}\left({\rm C_I}+R'_0\left(2{\rm B_I}+R'_0\right)\right)
\]
\item the energy loss is computed as:
\[
\Delta E = E_0-E'_0 \]
This value is then corrected to take into account the energy loss
fluctuations (see {\tt [PHYS332]}).
\end{enumerate}
This method results in a $dE/dx$ curve which is a set of connected
straight lines and not a step function. 
