%%%%%%%%%%%%%%%%%%%%%%%%%%%%%%%%%%%%%%%%%%%%%%%%%%%%%%%%%%%%%%%%%%
%                                                                 %
%  GEANT manual in LaTeX form                                     %
%                                                                 %
%  Michel Goossens (for translation into LaTeX)                   %
%  Version 1.00                                                   %
%  Last Mod. Jan 24 1991  1300   MG + IB                          %
%                                                                 %
%%%%%%%%%%%%%%%%%%%%%%%%%%%%%%%%%%%%%%%%%%%%%%%%%%%%%%%%%%%%%%%%%%%
\Documentation{P.Zanarini, S.Giani} 
\Version{Geant 3.16}\Routid{DRAW001}
\Submitted{07.03.92}  \Revised {11.12.92}
\Makehead{Introduction to  the Drawing  package}
\section {The drawing package }
The drawing package has been designed mainly to visualise:
\begin {itemize}
\item  detector components;
\item  the logical tree of detector components;
\item  the geometrical dimensions;
\item  particle trajectories
\item  the hits recorded in the sensitive elements of the detector.
\end{itemize}
\section{Graphics in {\tt GEANT}}

The set of routines in the drawing package
allows the visualisation of the volumes of the detector.
These routines can be called from the user program or interactively via
the command interface. The output can be directed onto the screen or to a 
file. The graphics of {\tt GEANT} is based on the 
{\tt HIGZ}~\cite{bib-HIGZ} package. 
\index{HIGZ}%
{\tt HIGZ} is a high-level interface between the user program (in this
case {\tt GEANT}), {\tt ZEBRA} and a basic graphics library.
At the moment of writing, {\tt HIGZ} can use various flavours
of the {\tt GKS}~\latexhtml{\cite{bib-gks2d,bib-gks3d,bib-GKS1}}%
                 {\cite{bib-gks2d}, \cite{bib-gks3d}, \cite{bib-GKS1}}
graphics system or the {\tt X11}\cite{bib-X11} system. A version based on
the {\tt PHIGS}\cite{bib-phigs} system is now being developed at CERN.

\section{Graphics functionality of {\tt GEANT}}
The graphics capabilities of {\tt GEANT} have been greatly enhanced in 
the last few revisions of the program. It is now possible to display 
volumes with
hidden lines removed from the drawing. This improves the understanding of
complicated objects. It is also possible to {\it cut} the displayed volumes
with various shapes (boxes, cones, tubes and spheres) in order
to visualise the internals of a detector element.
The surfaces of the volumes drawn can be filled with
colour, either solid or shaded according to a scan-line based
lighting algorithm. 
A more sophisticated surface rendering algorithm is now being developed
but it will be available only in the next version of the program.

Almost all drawing commands can be executed either by
calling a routine from the
user application or by
issuing a command when running {\tt GEANT} with the
standard interactive interface. 
Several interactive
tools ({\tt ZOOM}, {\tt LENS}) have been developed to analyse in detail 
the various parts of the detectors.
The interactive {\tt MOVE} facility allows the rotation, 
translation and zooming
in real time for simple components of the detectors.
Some of these commands are only available in the
interactive interface. These are described briefly in {\tt [DRAW500]} and
more extensively in the {\tt [XINT]} section.

A user interface based on X-Windows and Motif is also available
mainly on Unix workstations. Details of this interface are described
in {\tt [DRAW510]}.

Various graphical options can be set via the \Rind{GSATT} and \Rind{GDOPT} 
routines.

Every detector component can be visualised either in orthonormal projection
or in perspective (routines \Rind{GDRAW} and \Rind{GDRVOL}). Sections of the 
volumes and of their descendents can be visualised as well (routines 
\Rind{GDRAWX} and \Rind{GDRAWC}). 

The hierarchical relations of the various volumes which compose the
detector can be visualised via the \Rind{GDTREE} routine. In the 
interactive interface the user can select one of the volumes which
are nodes of the tree with the mouse,
and display the characteristics of the volume and walk up and down
the logical tree.

Tracks and hits can be drawn either alone or superimposed on the detector
(routines \Rind{GDCHIT}, \Rind{GDHITS}, \Rind{GDAHIT}, \Rind{GDXYZ}). It is also possible to 
draw the tracks while tracking is performed via the routine \Rind{GDCXYZ}. 

The drawing may be done with hidden-line removal
and with shading effects according to the value of the options {\tt HIDE}
and {\tt SHAD} (see next section).

Drawing complicated detectors can be a lengthy process even on a fast machine.
In particular,
when graphics options like hidden line removal or surface shading
are requested and many volumes are involved, the complete drawing can take
several minutes on the most powerful workstations. To alleviate this problem
drawings can be stored in memory in so called {\it view banks}. The routines
involved are \Rind{GDOPEN}, \Rind{GDCLOS}, \Rind{GDSHOW} and \Rind{GDELET}, 
and the corresponding interactive commands ({\tt DOPEN, DCLOSE, DSHOW, 
DELETE}). These banks can then be redisplayed quickly. They can also be 
written on disk and read back.

Attributes like colour (\Rind{GDCOL}) and line width (\Rind{GDLW}) can be
globally set for all 2D drawings (i.e. text, vectors, man, etc.);
such attributes do not affect the volume attributes that can be set by
the \Rind{GSATT} routine with {\tt COLO} or {\tt LWID} option.

The user can control some drawing options (\Rind{GDOPT}), by selecting,
for instance, to have either parallel or perspective projection,
either {\tt Y-Z} or {\tt R-Z} projection, etc.
 
The geometrical dimensions of an existing volume can be changed via the
routine \Rind{GEDITV}.
 
Various service routines allow the drawing of the 
profile of a human figure, the
axis, the scale of the dimension of the drawing and so on. These are 
described in {\tt [DRAW400]}.

\section{The shading algorithm}
In {\tt GEANT 3.16} a new shading algorithm has been introduced which
is 10 to 100 times faster than the algorithm which was implemented in
{\tt GEANT 3.15}. In the new algorithm the search for the visible portions
of all the faces of every volume is performed in the following steps:
\begin{enumerate}
\item the vertices of each face are transformed from the {\tt MARS} to
the screen reference system. In this way the dimensions of the face are
of the order of magnitude of centimeters.
\item in the screen reference system the face is {\it filled} with
horizontal lines {\it scan-lines} according to the desired resolution.
\item These lines are then transformed back in the {\tt MARS}, where the
hidden line removal technique is used to find their visible portions.
\item The colour luminosity is chosen as a function of the angle between
the normal to the surface and the line of view which is considered to be
also the line of illumination.
\item The visible portions of the lines are finally transformed into the
screen reference system and drawn with the calculated luminosity so that
they fill only the visible part of each face.
\end{enumerate}
The resulting picture can be saved into {\tt GEANT} view banks or {\tt HIGZ}
picture files.
 
\section{ Summary}
 
The best way to initialise the drawing package is to call the 
{\tt HPLOT}~\cite{bib-HPLOT}
initialisation routine \Rind{HPLINT}. This will work independently of the 
basic graphic package used. The reader should consult the {\tt HPLOT}
documentation for more information on this routine.
The drawing package is initialised by:
\begin{DLtt}{MMMMM}
\item[\Rind{HPLINT}] to initialise the basic graphics package and the 
{\tt HPLOT} package;
\item[\Rind{GDINIT}] to initialise the {\tt GEANT} drawing package.
\end{DLtt}
 
The main drawing routines are:
\begin{DLtt}{MMMMM}
\item[\Rind{GDRAW}]  to draw a projection view of the detector -- Case 1
\item[\Rind{GDRVOL}] to draw a projection view of the detector --  Case 2
\item[\Rind{GDRAWC}] to draw a cut view of the detector (along one axis)
\item[\Rind{GDRAWX}] to draw a cut view of the detector (from any point)
\item[\Rind{GDXYZ}]  to draw tracks at the end of the event
\item[\Rind{GDCXYZ}] to draw tracks at tracking time
\item[\Rind{GDPART}] to draw particle names and track numbers at end of tracks
\item[\Rind{GDAHIT}] to draw one single hit
\item[\Rind{GDHITS}] to draw hits for {\it trajectory} type detectors
\item[\Rind{GDCHIT}] to draw hits for {\it calorimeter} type detectors
\end{DLtt}
 
Routines that show how the detector has been modelled are:
\begin{DLtt}{MMMMM}
\item[\Rind{GDTREE}] to draw a picture with the geometrical tree
\item[\Rind{GDSPEC}] to draw a picture with volume specifications
\item[\Rind{GDFSPC}] to draw several GDSPEC pictures
\end{DLtt}
 
Routines that perform control operations on view banks are:
\begin{DLtt}{MMMMMM}
\item[\Rind{GDOPEN}] to open a given view bank, identified by a
                     view number; in this way we enter in bank mode
\item[\Rind{GDCLOS}] to close the current view bank, i.e. the last one
                     opened, and restore screen mode
\item[\Rind{GDSHOW}] to show all graphics information contained in a
                     given view bank
\item[\Rind{GDLENS}] to scan interactively the drawing contained in the view bank 
\item[\Rind{GDELET}] to delete a given view bank from memory
\end{DLtt}

Routines that allow the visualisation of the detector with hidden line removal are:
\begin{DLtt}{MMMMMM}
\item[\Rind{GDCGOB}] to create CG objects
\item[\Rind{GDCGSL}] to create clipping objects
\item[\Rind{GDCGCL}] to perform boolean operations
\item[\Rind{GDCGHI}] to insert CG objects in HIDE and WIRE structures of faces
\item[\Rind{GDSHIF}] to shift in the space a CG object
\item[\Rind{GDBOMB}] to make the detector explode
\end{DLtt}
 
Other routines are:
\begin{DLtt}{MMMMM}
\item[\Rind{GDOPT}]   to set drawing options
\item[\Rind{GDZOOM}]  to set the zoom parameters
\item[\Rind{GDAXIS}]  to draw the axes of the MARS,
                      oriented according to the current view parameters
\item[\Rind{GDSCAL}]  to draw the current scale
\item[\Rind{GDMAN}]   to draw a profile of a man within the current scale
\item[\Rind{GDRAWT}]  to draw text, with software characters
\item[\Rind{GDRAWV}]  to draw polylines in 2D user coordinates
\item[\Rind{GDHEAD}]  to draw a frame header
\item[\Rind{GDCOL}]   to set colour code
\item[\Rind{GDLW}]    to set line width
\item[\Rind{GDCURS}]  to have an input from the graphics cursor
\item[\Rind{GDFR3D}]  to convert from 3D coordinates (either in MARS or DRS)
                      to 2D user coordinates
\item[\Rind{GD3D3D}]  to convert from 3D MARS coordinates to
                      3D Projection Reference System coordinates.
\end{DLtt}
 
