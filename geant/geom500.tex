%%%%%%%%%%%%%%%%%%%%%%%%%%%%%%%%%%%%%%%%%%%%%%%%%%%%%%%%%%%%%%%%%%%
%                                                                 %
%  GEANT manual in LaTeX form                              %
%                                                                 %
%  Michel Goossens (for translation into LaTeX)                   %
%  Version 1.00                                                   %
%  Last Mod. Jan 24 1991  1300   MG + IB                          %
%                                                                 %
%%%%%%%%%%%%%%%%%%%%%%%%%%%%%%%%%%%%%%%%%%%%%%%%%%%%%%%%%%%%%%%%%%%
\Origin{R.Brun, P.Zanarini}
\Submitted{15.08.83}           \Revised{15.12.93}
\Version{Geant 3.16}           \Routid{GEOM500}

\Makehead{Volume attributes}

\Shubr{GSATT}{(CHNAME,CHIATT,IVAL)}
 
\begin{DLtt}{MMMMMMMM}
\item [CHNAME] ({\tt CHARACTER*4}) volume name;
\item[CHIATT] ({\tt CHARACTER*4}) attribute to be set;
\item[IVAL] ({\tt INTEGER}) value to which the attribute is to be set.
\end{DLtt}

Changes the attribute {\tt CHIATT} of the volume called {\tt CHNAME} to the 
value {\tt IVAL}. The names and meaning of the attributes and their allowed
values are:
\begin{center}
\begin{tabular}{|ccl|}
\hline
Number & Name & \multicolumn{1}{c|}{Description} \\ \hline
1& {\tt WORK}
& \begin{tabular}{rp{5cm}}
0 & inactive volume \\
1 & active volume
\end{tabular} \\[3mm]
 & ~ &\\
2& {\tt SEEN} 
& \begin{tabular}{rp{5cm}}
-2 & only the volume is visible, but none of its descendants \\
-1 & the volume is not visible together with all its descendants \\
0 & the volume is not visible \\
1 & the volume is visible
\end{tabular} \\
 & ~ &\\
3&{\tt LSTY} & line style parameter (see {\tt [XINT002]}) \\

4&{\tt LWID} & line width parameter (see {\tt [DRAW400]}) \\

5&{\tt COLO} & area filling colour (see {\tt [XINT002]}) \\

6&{\tt FILL} & area filling resolution (see {\tt [DRAW400]}) \\

7&{\tt SET} & set number associated to the volume\\
8&{\tt DET} & detector number associated to the volume\\
9&{\tt DTYP} & detector type associated to the volume (1,2)\\
10&{\tt NODE} & dummy \\ \hline

\end{tabular} 
\end{center}

\Shubr{GFATT}{(CHNAME,CHIATT,IVAL*)}
Returns in {\tt IVAL} the attribute {\tt CHIATT} of the volume {\tt CHNAME}.
The arguments have the same meaning than for the routine \Rind{GSATT}.

\Shubr{GFPARA}{(CHNAME,NR,INTEXT,NPAR*,NATT*,PAR*,ATT*)}
\begin{DLtt}{MMMMMMMM}
\item[CHNAME]  ({\tt CHARACTER*4}) volume name;
\item[NR]      ({\tt INTEGER}) copy number;
\item[INTEXT] ({\tt INTEGER}) type of volume parameters requested:
\begin{DLtt}{MMMM}
\item[0] user parameters;
\item[1] internal parameters;
\end{DLtt}
\item[NPAR] ({\tt INTEGER}) number of parameters returned;
\item[NATT] ({\tt INTEGER}) number of attributes returned;
\item[PAR]  ({\tt REAL}) array of parameters;
\item[ATT] ({\tt REAL}) array of attributes.
\end{DLtt}
Returns parameters {\tt PAR(1...NPAR)} and
attributes {\tt ATT(1...NATT)} for the volume {\tt CHNAME} with copy number
{\tt NR}.
