%%%%%%%%%%%%%%%%%%%%%%%%%%%%%%%%%%%%%%%%%%%%%%%%%%%%%%%%%%%%%%%%%%%
%                                                                 %
% GEANT manual in LaTeX form                                      %
%                                                                 %
% Michel Goossens (for translation into LaTeX)                    %
% Version 1.00                                                    %
% Last Mod. Jan 24 1991 1300  MG + IB                             %
%                                                                 %
%%%%%%%%%%%%%%%%%%%%%%%%%%%%%%%%%%%%%%%%%%%%%%%%%%%%%%%%%%%%%%%%%%%
\hyphenation{brems-strah-lung}
\Origin{G.N.Patrick, L.Urb\'{a}n}
\Submitted{26.10.84}\Revised{16.12.93}
\Version{Geant 3.16}\Routid{PHYS220}
\Makehead{Total cross-section for Compton scattering}

\section{Subroutines} 
\Shubr{GCOMPI}{}
 
{\tt GCOMPI} calculates the total cross-section for Compton
scattering. It tabulates the mean free path, $\lambda =
\frac{1}{\Sigma}$ (in cm), as a function
of the medium and of the energy (see {\tt JMATE} data structure).
The energy binning is set within the array {\tt ELOW} (common
\FCind{/GCMULO/}) in the routine \Rind{GPHYSI}.
{\tt GCOMPI} is called at initialisation time by {\tt GPHYSI}.
 
\section{Method}
 
The mean free path, $\lambda$, for a photon to interact via Compton
scattering is given by
\[
\lambda = \frac{1}{\Sigma} = \frac{A}{N_{Av}\rho}\frac{1}{\sigma(Z,E)}
\]
where:
\begin{eqnarray*}
N_{Av} &  & \mbox{Avogadro's number} \\
Z,A &  & \mbox{atomic and mass number of the medium} \\
\rho &  & \mbox{density of the medium} \\
\sigma &  & \mbox{total cross-section per atom for Compton scattering} \\
E &  & \mbox{energy of the photon.}
\end{eqnarray*}
 
For the total cross-section an empirical cross-section formula
is used which reproduces the cross-section data rather well down to 10 keV:
\[
\sigma(Z,E) = Z \left [ P_{1}(Z) \frac{\log(1+2X)}{X} +
\frac{P_{2}(Z)+P_{3}(Z) X + P_{4}(Z) X^{2}}{1+aX+bX^{2}+cX^{3}} \right ]
\mbox{barn atom}^{-1} 
\]
where
\begin{eqnarray*}
m & = & \mbox{electron mass}  \\
X & = & \frac{E}{m}  \\
P_{i}(Z) & = & D_{i} + E_{i}Z + F_{i}Z^{2}
\end{eqnarray*}
 
The values of the parameters are put in 
the {\tt DATA} statement within the routine \Rind{GCOMPI}.
The fit was made over 511 data points chosen between:
\[
1 \leq Z \leq 100 \makebox[2cm]{;} 10 \: \mbox{keV} \leq E \leq 100 \: 
\mbox{GeV}
\]
 
The accuracy of the fit is estimated to be:
 
\vspace{.3cm}
\[
\frac{\Delta\sigma}{\sigma} = \left\{ \begin{array}{lcl}
\approx 10\% & \makebox[2cm][r]{\rm for E} & \simeq 10\: \mbox{keV}-20\: 
\mbox{keV} \vspace{.3cm} \\
\leq 5-6\% & \makebox[2cm][r]{\rm for E} & > 20 \: \mbox{keV}
\end{array} \right.
\]
