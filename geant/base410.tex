%%%%%%%%%%%%%%%%%%%%%%%%%%%%%%%%%%%%%%%%%%%%%%%%%%%%%%%%%%%%%%%%%%%
%                                                                 %
%  GEANT manual in LaTeX form                                     %
%                                                                 %
%  Michel Goossens (for translation into LaTeX)                   %
%  Version 1.00                                                   %
%  Last Mod. Jan 24 1991  1300   MG + IB                          %
%                                                                 %
%%%%%%%%%%%%%%%%%%%%%%%%%%%%%%%%%%%%%%%%%%%%%%%%%%%%%%%%%%%%%%%%%%%
\Origin{R.Brun}
\Submitted{08.08.87}  \Revised{08.11.93}
\Version{Geant 3.16}\Routid{BASE410}
\Makehead{Utility Routines}
 
\Shubr{GLOOK}{(CHNAME,IVECT,N,ILOOK*)}
\begin{DLtt}{MMMMMMMM}
\item[CHNAME]({\tt CHARACTER*4}) variable containing the name to be
searched for in {\tt IVECT};
\item[IVECT] ({\tt INTEGER}) array containing the ASCII code of the 
names among which
{\tt NAME} is searched. The names are stored 4 characters per word;
\item[N]({\tt INTEGER}) number of items in {\tt IVECT};
\item[ILOOK]({\tt INTEGER}) position in {\tt IVECT} where {\tt NAME} has
been found, 0 if not found.
\end{DLtt}

This routine is very useful when searching for a string stored 
in a {\tt ZEBRA} bank. For instance to find the
position of the {\tt 'CRYS'} volume in the volume bank, the following
piece of code could be written:

\begin{verbatim}
+SEQ,GCBANK.
+SEQ,GCNUM.
      .
      .
      .
      CALL GLOOK('CRYS',IQ(JVOLUM+1),NVOLUM,IVO)
      IF(IVO.GT.0) THEN
         JVO = LQ(JVOLUM-IVO)
      ELSE
         JVO = 0
      ENDIF
\end{verbatim}

{\tt JVO}, if different from 0, is the pointer to the data
bank containing the information relative to the volume {\tt 'CRYS'}.

\Shubr{GEVKEV}{(EGEV,ENERU*,CHUNIT*)}
\begin{DLtt}{MMMMMMMM}
\item[EGEV] ({\tt REAL}) input, energy in GeV;
\item[ENERU] ({\tt REAL}) output, energy in the new unit;
\item[CHUNIT] ({\tt CHARACTER*4}) unit in which the energy has been converted.
\end{DLtt}

This subroutine converts the input energy in GeV to a unit in which 
$1 \leq E \leq 999$. {\tt CHUNIT} contains the new
unit. The following piece of code illustrates the use of \Rind{GEVKEV}:
\begin{verbatim}
+SEQ,GCTRAK.
      CHARACTER*4 CHUNIT
      .
      .
      .
      CALL GEVKEV(DESTEP, DE, CHUNIT)
      WRITE(6,10000) DE, CHUNIT
10000 FORMAT(' The energy loss in this step is ',F7.2,A)
\end{verbatim}

