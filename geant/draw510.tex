%%%%%%%%%%%%%%%%%%%%%%%%%%%%%%%%%%%%%%%%%%%%%%%%%%%%%%%%%%%%%%%%%%%
%                                                                 %
%  GEANT manual in LaTeX form                                     %
%                                                                 %
%  Michel Goossens (for translation into LaTeX)                   %
%  Version 1.00                                                   %
%  Last Mod. Jan 24 1991  1300   MG + IB                          %
%                                                                 %
%%%%%%%%%%%%%%%%%%%%%%%%%%%%%%%%%%%%%%%%%%%%%%%%%%%%%%%%%%%%%%%%%%%
\Authors{F.Carminati, S.Giani}   \Origin{S.Giani}
\Submitted{10.08.93}   \Revised{10.08.93}
\Version{Geant 3.15}\Routid{DRAW510}
\Makehead{The MOTIF user interface}

The interactive version 3.16 contains an object oriented Motif-based user
interface. It can be accessed specifying {\tt m} as workstation type.
The full functionality of the X11 version remains available, while new 
Motif-specific features have been added.

\section{Overview}

{\tt GEANT} data structures are considered as {\tt KUIP} browsable 
classes and their
contents as {\it objects} on which actions can be performed (the 
{\tt GEANT} commands).
According to the Motif conventions, an object can be selected clicking the
left button of the mouse when the cursor is on the icon representing that
object. 

\begin{figure}[hbt]
\centering
\hspace*{-1.5cm}
\epsfysize=10cm\epsffile{eps/draw510-1.eps}
\label{draw510-1}
\caption{Example of GEANT++ Motif windows}
\end{figure}

Clicking the right button of the mouse on a selected object, 
a menu of possible
commands will appear (a double click on the left button the will execute
by default the first action of the list). 
Moving the mouse on the commands in the list while the button is pressed
will {\it raise} or select them. When the button is released on a command,
this selected command will perform the relative actions
on the selected object. Actions (like drawing, for example) can be
executed either directly or via a Motif panel opened by the command.

\begin{figure}[hbt]
\centering
\hspace*{-1.5cm}
\epsfysize=10cm\epsffile{eps/draw510-2.eps}
\label{draw510-2}
\caption{Example of GEANT++ Motif windows}
\end{figure}

Examples can be seen in figs~\ref{draw510-1} and \ref{draw510-2}
Objects drawn in the graphics window can be {\tt picked} 
(i.e. selected with the mouse) as well (for example,
volumes, tracks, hits); clicking the right button when the cursor is on the
selected object, a menu of possible actions on that object is displayed.
Users can finally define Motif panels containing buttons corresponding to the
most frequently used commands. An on-line help is available for any specific
subject.

\section{The executive window}

It replaces the normal dialog window. It contains a {\it transcript pad}, 
where the
text output of the executed commands is displayed, and an {\it input pad}, 
where the
user can still type the desired commands in the old style. \\[1em]

\section{The file browser}

On the left side it displays a list of browsable classes:
{\tt GEANT} data structures, 
available commands, files, macros and Zebra divisions. Selecting one of
them, the full list of icons representing the objects of that class is shown
in the main area of the browser. Proceeding as described before, it is 
possible to perform actions on the classes (like create a new object) or on
the objects belonging to them. It is possible to create menus of commands
just clicking on the string `commands' at the top line of the browser.

\section{The graphics window}

Any object which can be drawn in the graphic window can be stored in the current
picture file (automatically opened after each NEXT command) via a call to 
\Rind{IGPID} (see Higz manual). It can be afterwards {\tt picked}
as described before.
In the case of commands executed via the use of Motif panels, some of the
input values
can be set with sliders in the specifed range for the relative
variable. Moving the slider (after having clicked on the right-hand 
{\it activating
box}) the relative action is performed in the graphics
window when the mouse button is released. When in {\tt drag mode}, the 
action is performed {\it while} the slider is moving (keeping the left button
pressed): especially when double buffering has been selected, this can be
useful for real time manipulations.

\section{An Example}

Start your GEANT316 executable module (linked with GXINT316 and Motif1.2);
\begin{enumerate}
\item type {\tt m} as workstation type;
\item click the left button of the mouse after positioning the cursor on the 
string {\tt VOLU} in the browser;
\item click the left button of the mouse after positioning the cursor on 
any icon in the main area of the browser;
\item click now the right button of the mouse and keep it pressed;
\item move the mouse to select the action {\tt Tree} and release the button;
\item the drawing of the logical tree will be displayed in the graphics
window;
\item position the cursor on the drawing of a box (containing a volume name)
in the graphics window, click the right button and keep it pressed;
\item release the button selecting the action {\tt Dspec};
\item the command {\tt DSPEC} for that volume will be executed in a separate
window;
\item repeat the exercise selecting this time the action {\tt Dspe3d};
\item the {\tt DSPEC} will be executed in the first window, the volume 
specifications will be printed in a separate window and a Motif panel will
appear;
\item click the left button of the mouse positioning the cursor in the Motif
panel on the {\tt Value changed} button, and select the {\tt DRAG} option;
\item click now the left button on the `activating box' on the right of
the {\tt Theta} slider;
\item click on the {\tt Theta} slider and, keeping pressed the 
left button of the mouse, move it right;
\item the drawing in the graphics window will rotate;
\item release the button and type {\tt igset 2buf 1} in the executive window;
\item restart moving the slider as before. 
\end{enumerate}
