%%%%%%%%%%%%%%%%%%%%%%%%%%%%%%%%%%%%%%%%%%%%%%%%%%%%%%%%%%%%%%%%%%%
%                                                                 %
%  GEANT manual in LaTeX form                                     %
%                                                                 %
%  Michel Goossens (for translation into LaTeX)                   %
%  Version 1.00                                                   %
%  Last Mod. Jan 24 1991  1300   MG + IB                          %
%                                                                 %
%%%%%%%%%%%%%%%%%%%%%%%%%%%%%%%%%%%%%%%%%%%%%%%%%%%%%%%%%%%%%%%%%%%
\Origin{F.Bruyant, A.McPherson}
\Submitted{16.12.83}                \Revised{15.12.93}
\Version{Geant 3.16}                \Routid{GEOM600}
\Makehead{User initialisation of the common block /GCVOLU/}

\Shubr{GLVOLU}{(NLEV,LNAM,LNUM,IER*)}
\begin{DLtt}{MMMMMMM}
\item[NLEV] ({\tt INTEGER}) number of levels to fill;
\item[LNAM] ({\tt INTEGER}) array of {\tt NLEV} volume names stored
via the \Rind{UCTOH} routine;
\item[LNUM] ({\tt INTEGER}) array of {\tt NLEV} volume copy (or 
division) numbers;
\item[IER] ({\tt INTEGER}) returns code different from 0 in case of 
error, a diagnostic is also printed;
\end{DLtt}

This routine fills the volume parameters in common \FCind{/GCVOLU/}
for a physical tree, specified by the lists {\tt LNAM} and {\tt LNUM} of
volumes names and numbers, and for all its ascendants up to
level 1.

The routine is optimised and does not re-compute the part of
history already available in \FCind{GCVOLU}. This means that if
it is used in user programs outside the usual framework of the tracking,
the user has to initialise to zero {\tt NLEVEL} in common \FCind{GCVOLU}.

An example of use of \Rind{GLVOLU} in this context could be to find the
position and direction of a particle in the local coordinate system of
a volume:
\begin{verbatim}
      .
      .
+SEQ,GCVOLU
      DIMENSION LNAM(15), LNUM(15), POS(3), DIR(3), POSL(3), DIRL(3)
      .
      .
      CALL UCTOH('MOTH',LNAM(1),4,4)
      LNUM(1) = 1
      CALL UCTOH('CAL1',LNAM(2),4,4)
      LNUM(2) = 2
      CALL UCTOH('MOD1',LNAM(3),4,4)
      LNUM(3) = 5
      CALL UCTOH('CHAM',LNAM(4),4,4)
      LNUM(4) = 18
*---
      NLEVEL  = 0
      CALL GLVOLU(4,LNAM,LNUM,IER)
*---
      CALL GMTOD(POS,POSL,1)
      CALL GMTOD(DIR,DIRL,2)
      .
      .
\end{verbatim}
