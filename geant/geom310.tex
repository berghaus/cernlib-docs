%%%%%%%%%%%%%%%%%%%%%%%%%%%%%%%%%%%%%%%%%%%%%%%%%%%%%%%%%%%%%%%%%%%
%                                                                 %
%  GEANT manual in LaTeX form                                     %
%                                                                 %
%  Michel Goossens (for translation into LaTeX)                   %
%  Version 1.00                                                   %
%  Last Mod. Jan 24 1991  1300   MG + IB                          %
%                                                                 %
%%%%%%%%%%%%%%%%%%%%%%%%%%%%%%%%%%%%%%%%%%%%%%%%%%%%%%%%%%%%%%%%%%%
\Origin{R.Brun, A.C.McPherson}
\Revision{S.Giani}
\Submitted{01.06.83}             \Revised{15.12.93}
\Version{Geant 3.21}             \Routid{GEOM310}

\Makehead{Finding distance to next boundary}

\Shubr{GNEXT}{(X,SNEXT*,SAFETY*)}
\Shubr{GTNEXT}{(X,SNEXT*,SAFETY*)}
Finds distance to the next boundary.
It takes explicit account of shape content and uniqueness.
\begin{DLtt}{MMMMMMMM}
\item[X] ({\tt REAL}) array of 6 of coordinates and direction cosines;
\item[SNEXT] ({\tt REAL}) distance to the nearest volume boundary 
along the particle direction;
\item[SAFETY] ({\tt REAL}) {\it safety} distance, that is smaller distance
to any boundary;
\end{DLtt}
 
This routine evaluates the two {\it distances} which are needed by the
{\tt GEANT} tracking routines. \Rind{GTNEXT} and \Rind{GNEXT} perform
the same function, but \Rind{GNEXT} is a static routine which can be
called by the user, while \Rind{GTNEXT} is the routines used internally
by {\tt GEANT} during tracking, and it should not be called by the user.

If {\tt INFROM} (common \FCind{/GCVOLU/}) is different from 0, \Rind{GTNEXT}
interprets it as the daughter out of which the particle just came, and
uses the list of daughters stored with that volume, possibly modified by
\Rind{GSNEXT}/\Rind{GSNEAR} to calculate the distance to the next boundary.

The first action of \Rind{GTNEXT} is to calculate the {\tt SAFETY} distance.
If this is larger than the current step candidate, no other calculation is
performed and the {\tt IGNEXT} flag (common \FCind{/GCTRAK/}) is set to 0,
indicating that no change of volume is occurring at the end of the current
step. If the step is smaller than safety, then {\tt SNEXT} is computed.
If the step is smaller than {\tt SNEXT}, again there will not be any
change of volume during the step and {\tt IGNEXT} is set to 0.
If on the contrary the candidate step is larger than {\tt SNEXT}, a
change of volume will occur at the end of the step, and {\tt IGNEXT}
is set to 1 and {\tt INGOTO} (common \FCind{/GCTRAK/}) is set to the
number of the daughter where the particle is entering, if any.

Charged particles in magnetic field are transported with a similar logic.
However, even if the candidate step is smaller than {\tt SNEXT}, the
particle can still cross into another volume due to its bent path. When
tracking in magnetic field, after every step greater than {\tt SAFETY}
it is checked whether the particle is still in the same volume. If this is
not the case, the step is divided by two and transport is tried again.
Conversely a charged particle in magnetic field may still be in the
current volume even after having travelled the distance to the nearest
boundary along a {\it straight} line. So boundary crossing is declared
only when {\tt IGNEXT} is different from 0 and 
the difference between the real trajectory and the bent one is smaller
than {\tt EPSIL} (common \FCind{/GCTMED/}).

