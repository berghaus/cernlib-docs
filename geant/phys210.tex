%%%%%%%%%%%%%%%%%%%%%%%%%%%%%%%%%%%%%%%%%%%%%%%%%%%%%%%%%%%%%%%%%%%
%                                                                 %
%  GEANT manual in LaTeX form                                     %
%                                                                 %
%  Version 1.00                                                   %
%                                                                 %
%  Last Mod.  8 June 1993 1900   MG                               %
%                                                                 %
%%%%%%%%%%%%%%%%%%%%%%%%%%%%%%%%%%%%%%%%%%%%%%%%%%%%%%%%%%%%%%%%%%%
\Origin{L. Urb\'{a}n}
\Submitted{10.11.84}
\Revised{16.12.93}
\Version{Geant 3.16}\Routid{PHYS210}
 
\Makehead{Total cross-section for e+e- pair production by photons}
\section{Subroutines} 
\Shubr{GPRSGA}{}
{\tt GPRSGA} calculates the total cross-section for
the pair production of photons and the direct pair production of muons
in all materials. It tabulates the mean free path, $\lambda =
1/\Sigma$ (in cm), as a function of the medium and the energy.
For the pair production by photons, it calls the function 
\Rind{GPRSGG}. The energy
bins are the one of the array {\tt ELOW} in common \FCind{/CGMULO/}
initialised in the
routine \Rind{GPHYSI}. The following pointers are used:
\begin{DLtt}{MMMMMMMMMMMMMMMMMMM}
\item[JMA = LQ(JMATE-I)]  pointer to the I$^{th}$ material
\item[JPAIR = LQ(JMA-10)] pointer to pair production cross-sections
\end{DLtt}
\Rind{GPRSGA} is called at initialisation time by \Rind{GPHYSI}.

\Sfunc{GPRSGG}{VALUE = GPRSGG(Z,E)}
\Rind{GPRSGG} calculates the total cross-section for
the pair production of a photon with energy {\tt E} in material
with atomic number {\tt Z}. It is called by \Rind{GPRSGA}.

\section{Method}
 
We have parameterised the total cross-section of \Pep\Pem-pair
production by photon as:
 
\begin{equation}
\sigma(Z,E_\gamma) = Z(Z+1) \:(F_1(X) + F_2(X) Z +
F_3(X)/Z)  \mbox{\quad (barn/atom)}
\label{eq:phys210-1}
\end{equation}

where

\begin{tabular}[t]{p{1.3cm}ll}
$X$           & = & $\ln(E_\gamma/m)$,\\
$m$           & = & electron mass,\\
$E_\gamma$    & = & photon energy \\
\end{tabular}

and

\begin{tabular}[t]{p{1.3cm}ll}
$F_i(X)$      & = & $\displaystyle \sum_{n=0}^5 c_n X^n$ \\
\end{tabular}
 
The parameters $c_n$ were taken from a least-square fit to the data 
\cite{bib-HUBB}
and can be found in the {\tt DATA} statement in the 
function \Rind{GPRSGG(Z,E)} which
computes the formula (\ref{eq:phys210-1}) (in barn/atom).
 
This parameterisation gives a good description of the data in the
range:
 
\begin{equation}
\left . \begin{array}{c} 1\leq Z\leq 100 \\
1.5 \; \mbox{MeV} \leq E_\gamma\leq 100 \; \mbox{GeV} 
\end{array} \right \} \hspace{0.5cm}
\frac{ \Delta\ \sigma}{\sigma}\leq 5\% 
        \mbox{\hspace{0.5cm} with a mean value of} \approx 2.2\% 
\end{equation}
 
This mean free path is tabulated at initialisation as a function
of the medium and of the energy by the routine \Rind{GPRSGA}.
 
