%%%%%%%%%%%%%%%%%%%%%%%%%%%%%%%%%%%%%%%%%%%%%%%%%%%%%%%%%%%%%%%%%%%
%                                                                 %
%  GEANT manual in LaTeX form                              %
%                                                                 %
%  Michel Goossens (for translation into LaTeX)                   %
%  Version 1.00                                                   %
%  Last Mod. Jan 24 1991  1300   MG + IB                          %
%                                                                 %
%%%%%%%%%%%%%%%%%%%%%%%%%%%%%%%%%%%%%%%%%%%%%%%%%%%%%%%%%%%%%%%%%%%
\Origin{R.Brun, F.Bruyant, M.Maire, A.McPherson}
\Submitted{15.08.83}               \Revised{18.11.93}
\Version{Geant 3.16}\Routid{GEOM120}
\Makehead{Positioning a volume inside its mother with parameters}

\Shubr{GSPOSP}{(CHNAME,NR,CHMOTH,X,Y,Z,IROT,CHONLY,PAR,NPAR)}

Position a copy of volume {\tt CHNAME} inside its mother {\tt CHMOTH}
assigning its parameters.

\begin{DLtt}{MMMMMMMMMM}
\item[CHNAME] ({\tt CHARACTER*4}) name of the volume being positioned;
\item[NR] ({\tt INTEGER}) copy number of the volume {\tt CHNAME} being
positioned;
\item[CHMOTH]({\tt CHARACTER*4}) name of the volume in which copy
{\tt NR} of {\tt CHNAME} is positioned;
\item[X] ({\tt REAL}) x position of the volume in the mother reference system;
\item[Y] ({\tt REAL}) y position of the volume in the mother reference system;
\item[Z] ({\tt REAL}) z position of the volume in the mother reference system;
\item[IROT] ({\tt INTEGER}) rotation matrix number
describing the orientation of the volume relative to
the coordinate system of the mother;
\item[CHONLY] ({\tt CHARACTER*4}) flag to indicate whether
a point found to be in this volume may also be in other volumes which
are not direct descendants of it -- possible values are {\tt ONLY} and
{\tt MANY};
\item[PAR] ({\tt REAL}) array of parameters for the volume being positioned;
\item[NPAR] ({\tt INTEGER}) number of parameters.
\end{DLtt}

It is often the case in a detector to have a family of similar objects,
differing only by their dimensions. A typical example are the crystals of
an electromagnetic calorimeter. In this case it is convenient and logically
consistent to give them all the same name, but assign to each copy a different
set of parameters.

This can be done defining the volume with 0 parameters through the routine
\Rind{GSVOLU} and then assigning the parameters to each copy via the routine
\Rind{GSPOSP}.

\Rind{GSPOSP} is similar in all other aspects to \Rind{GSPOS} and the user is
referred to the description of this routine for more information. An obvious
limitation is that a volume defined with 0 parameters can only be positioned
via \Rind{GSPOSP}, because otherwise its dimensions will be undefined.
