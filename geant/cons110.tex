%%%%%%%%%%%%%%%%%%%%%%%%%%%%%%%%%%%%%%%%%%%%%%%%%%%%%%%%%%%%%%%%%%%
%                                                                 %
%  GEANT manual in LaTeX form                                     %
%                                                                 %
%  Michel Goossens (for translation into LaTeX)                   %
%  Version 1.00                                                   %
%  Last Mod. Jan 24 1991  1300   MG + IB                          %
%                                                                 %
%%%%%%%%%%%%%%%%%%%%%%%%%%%%%%%%%%%%%%%%%%%%%%%%%%%%%%%%%%%%%%%%%%%
\Origin{R.Brun, M.Maire, J.Allison}
\Documentation{M.Maire}
\Version{Geant 3.16}\Routid{CONS110}
\Submitted{01.06.83}          \Revised{08.12.93}
\Makehead{Mixtures and Compounds}
\Shubr{GSMIXT}{(IMATE,NAMATE,A,Z,DENS,NLMAT,WMAT*)}
 
Defines mixture or compound {\tt IMATE} as composed by
{\tt NLMAT} materials defined via
the arrays {\tt A}, {\tt Z} and {\tt WMAT}.
Mixtures of compounds can also be defined.
 
\begin{DLtt}{MMMMMMMMMM}
\item[IMATE]({\tt INTEGER}) user material (mixture) number;
\item[NAMATE]({\tt CHARACTER*20}) mixture name;
\item[A]({\tt REAL}) array of atomic weights;
\item[Z]({\tt REAL}) array of atomic numbers;
\item[DENS]({\tt REAL}) density in g cm$^{-3}$;
\item[NLMAT]({\tt INTEGER}) number of elements in the mixture;
\begin{DLtt}{MMMMM}
\item[$> 0$]{\tt WMAT} contains the proportion
by weights of each material in the mixture;
\item[$< 0$]{\tt WMAT} contains the proportion by number of
atoms of each kind, the content of {\tt WMAT} in output is
changed to contain the relative weights;
\end{DLtt}
\end{DLtt}
 
\subsection*{Method}
For a compound ({\tt NLMAT} $< 0$), the {\it molecular} weight and charge are:
\[
A_{mol} = \sum_i n_i A_i\mbox{\hspace{3cm}} Z_{mol} = \sum_i n_i Z_i
\]
where $n_i$ {\tt = WMAT(I)} is the number of atoms of the i$^{th}$
component of the molecule. In this case the proportion by weight is:
\[
 p_i = \frac{n_i A_i}{A_{mol}}
\]
where $p_i$ {\tt = WMAT(I)} in output.
From the relative weights,
\Rind{GSMIXT} works out the effective atomic weight and atomic number
according to the following formulas:
\[
A_{eff} = \sum_i p_i A_i \mbox{\hspace{3cm}}   Z_{eff} = \sum_i p_i Z_i
\]
which are stored in the {\tt JMATE} data structure {\tt [CONS199]}
together with the radiation length.
 
The radiation length is computed according to the EGS
manual \cite{bib-EGS3}, formula 268 (37), for an element:
\[
\frac{1} {\rho \: X_0 }=4 \alpha r_{0}^2 N_{Av}  \frac{1}{A}
     Z(Z+ \xi (Z)) \left[ \ln \frac{183}{Z^{1/3}}
      -F_c (Z) \right]
\]
where
\begin{DLtt}{MMMMMMMM}
\item[$X_0$]   radiation length (in cm);
\item[$\rho$] density (in g cm$^{-3}$);
\item[$\alpha$] fine structure constant;
\item[$r$]classical electron radius (in cm);
\item[$N_{Av}$] Avogadro's number;
\item[$A$]atomic weight;
\item[$Z$]atomic number;
\item[$F_c(Z)$] Coulomb correction function.
\end{DLtt}
\[
F_c (Z) = (\alpha Z)^2 \left[ \frac{1}
     { 1+( \alpha Z)^2}  +0.20206-0.0369( \alpha Z)^2
      +0.0083(\alpha Z)^4 -0.0020(\alpha Z)^6 \right]
\]
\[
\xi (Z) = \frac{\ln \frac{1440}{Z^{2/3}}}
        {\ln \frac{183}{Z^{1/3}} - F_c (Z)}
\]
for a compound or mixture:
\[
 \frac{1}{\rho X_0}=\sum_i \frac{p_i}{\rho_i X_{0i}}
\]
where $p_i$ is the proportion by weight of the i$^{th}$ element.
For more information on the organisation of the data in memory
see {\tt [CONS199]}.

The effective absorption length $\lambda$ is defined as
the interaction length of a 5GeV pion in the material:
\[
\frac{1}{\lambda} = \sum_i \frac{p_i}{\lambda_i}
\hspace{2cm} \frac{1}{\lambda_i} = \frac{\sigma^{\pi}_{i}(5GeV)
 N_{Av}}{A_{i}} 
\hspace{0.5cm} \mbox{g$^{-1}$ cm$^{2}$}
\]
Once this value has been determined, an effective hadronic atomic weight 
($A_{h-eff}$) is calculated by the routine \Rind{GHMIX} \cite{bib-GHMI}
and stored in the data structure.
