%%%%%%%%%%%%%%%%%%%%%%%%%%%%%%%%%%%%%%%%%%%%%%%%%%%%%%%%%%%%%%%%%%%
%                                                                 %
%  GEANT manual in LaTeX form                                     %
%                                                                 %
%  Version 1.00                                                   %
%  Last Mod.  8 June 1993  1300   MG                              %
%                                                                 %
%%%%%%%%%%%%%%%%%%%%%%%%%%%%%%%%%%%%%%%%%%%%%%%%%%%%%%%%%%%%%%%%%%%
\Origin{G.N. Patrick, L.Urb\'{a}n}
\Version{Geant 3.16}\Routid{PHYS341}
\Submitted {26.10.84}  \Revised{16.12.93}
\Makehead{Simulation of discrete bremsstrahlung by electrons}

\section{Subroutines}

\Shubr{GBREME}{}

\Rind{GBREME} generates a bremsstrahlung photon from an electron 
as a discrete process. The photon energy is sampled from 
a parameterisation of the 
bremsstrahlung cross-section of Seltzer and Berger
\cite{bib-SEL1} for electron energies below $10$ GeV, and from 
the screened Bethe-Heitler cross-section 
above $10$ GeV.
Midgal corrections are applied in both cases. 
The angular distribution of the photon is calculated by the
function \Rind{GBTETH}.
 
\begin{tabular}{ll}
Input :  & via common block \FCind{/GCTRAK/} \\
Output:  & via common block \FCind{/GCKING/}.
\end{tabular}
 
\Rind{GBREME} is called from the tracking
routine \Rind{GTELEC} when
the parent electron reaches a radiation point during tracking.
 
\Sfunc{GBTETH}{THETA = GBTETH(ENER,PARTM,EFRAC)}

\begin{DLtt}{MMMMMMMM}
\item[THETA] ({\tt REAL}) angle of the radiated photon or $e^{\pm}$
pair;
\item[ENER] ({\tt REAL}) energy of the particle;
\item[PARTM] ({\tt REAL}) mass of the particle;
\item[EFRAC] ({\tt REAL}) ratio between the energy of the photon and the
energy of the particle.
\end{DLtt}
\Rind{GBTETH} calculates the angular distribution of the $e^+ e^-$-pair
in photon pair production and of the emitted photon in 
$\mu$ and $e^{\pm}$ bremsstrahlung.
\Rind{GBTETH} is called by \Rind{GBREME}.

\section {Method}

The photon energy is sampled according to the Seltzer and Berger
bremsstrahlung spectrum~\cite{bib-SEL1}. Seltzer and Berger have calculated the
spectra for materials with atomic numbers Z = 6,13,29,47,74,92
in the electron (kinetic) energy range  1 keV - 10 GeV. Their tabulated
results have been used as input in a parametrising-fitting procedure.
The functional form of the parameterisation for the quantity:

\[
S(x) = C k \frac{d \sigma}{d k}
\]
 
can be written as
\begin{equation}
\label{eq:phys341-1}
S(x) = \left \{
\begin{array}{L@{\hspace{1cm}}L}
(1-a_{h} \epsilon )F_{1}(\delta) + b_{h} \epsilon^{2} F_{2} (\delta)
& T \geq 1 MeV \\
1 + a_{l} x + b_{l} x^{2} & T < 1 MeV
\end{array} \right .
\end{equation}
where:
\[
\begin{array}{L@{}cL@{\hspace{2cm}}LcL}
C & & \mbox{normalisation constant} & 
k & & \mbox{photon energy} \\ [1mm]
T, E & & \mbox{kinetic and total energy of the primary electron} & 
x & = & \frac{k}{T} \\ [2mm]
\epsilon & = & \frac{k}{E} = x \frac{T}{E} 
\end{array}
\]

The $F_{i}(\delta)$ screening functions depend on the screening variable:
\[
\begin{array}{LcLL}
\delta & = & \frac{136 m_{e}}{Z^{1/3} E} \frac{\epsilon}{1-\epsilon} \\
F_{1}(\delta) & = & F_{0} (42.392 - 7.796 \delta +1.961 \delta^{2} - F)
& \delta \leq 1 \\
F_{2}(\delta) & = & F_{0} (41.734 - 6.484 \delta +1.250 \delta^{2} - F) 
& \delta \leq 1 \\
F_{1}(\delta) & = & F_{2}(\delta) =
F_{0} (42.24 - 8.368 \ln(\delta + 0.952) -F) & \delta > 1 \\
F_{0} & = & \frac{1}{42.392-F} \\
F & = & 4 \ln Z - 0.55 (\ln Z)^{2}
\end{array}
\]
 
$a_{h,l}$ and $b_{h,l}$ are parameters to be fitted.
 
The `high energy' (T $>$  1 MeV) formula comes from the
Coulomb-corrected, sceened Bethe-Heitler formula (see e.g. 
\cite{bib-WILL,bib-BUTC,bib-EGS3}). However, there are two things in eq. 
(\ref{eq:phys341-1}) which make a difference:
 
\begin{enumerate}
\item $a_{h}, b_{h}$ depend on T and on the atomic number Z ( in the case
of the Bethe-Heitler spectrum $a_{h} = 1$, $b_{h} =0.75$);
\item the function $F$ is not the same than that in the Bethe-Heitler
cross-section, this function gives a better behaviour in the
high frequency limit, i.e. when $k \rightarrow T$  ($x \rightarrow 1$).
\end{enumerate}
 
The T and Z dependence of the parameters are described by the equations:
 
\begin{eqnarray*}
a_{h} & = & 1 + \frac{a_{h1}}{u}+\frac{a_{h2}}{u^{2}}+\frac{a_{h3}}{u^{3}} \\
b_{h} & = & 0.75+\frac{b_{h1}}{u}+\frac{b_{h2}}{u^{2}}+\frac{b_{h3}}{u^{3}} \\
a_{l} & = & a_{l0} + a_{l1} u + a_{l2} u^{2} \\
b_{l} & = & b_{l0} + b_{l1} u + b_{l2} u^{2} \\
\mbox{with} \\
u & = & \ln \left ( \frac{T}{m_{e}} \right )
\end{eqnarray*}
 
the $a_{hi}, b_{hi}, a_{li}, b_{li}$ parameters are polynomials of second order
in the variable:

\[
v = [Z (Z+1)]^{1/3}
\]
 
It can be seen relatively easily that for the limiting case $T \rightarrow
\infty$, $a_{h} \rightarrow 1, b_{h} \rightarrow 0.75$,
so eq. (\ref{eq:phys341-1}) gives the Bethe-Heitler cross section.
 
There are altogether 36 linear parameter in the formulae , their
values are given in \Rind{GBREME}. The parameterisation reproduces
the Seltzer-Berger tables within a few \% (2-3 \% on average,
the maximum error being less than 10-12 \%), the tables, on the other hand,
agree well with the experimental data and theoretical (low- and high-energy)
results (less than 10 \% below 50 MeV, less than 5 \% above 50 MeV).
 
Apart from the normalisation the cross section differential in photon
energy can be written as:
\[
\frac{d \sigma}{d k} = \frac{1}{\ln \frac{1}{x_{c}}} \frac{1}{x}
g(x) = \frac{1}{\ln \frac{1}{x_{c}}} \frac{1}{x} \frac{S(x)}{S_{max}}
\]
 
where $x_{c} = k_{c}/T$, $k_{c}$ is the photon cut-off energy below
which the bremsstrahlung is treated as a continuous energy loss
(this cut is {\tt BCUTE} in the program). Using this decomposition of
the cross section and two random numbers $r_{1}$, $r_{2}$ uniformly
distributed in $]0,1[$, the sampling of $x$ is done as follows:
\begin{enumerate}
\item sample $x$ from
\[
\frac{1}{\ln \frac{1}{x_{c}}} \frac{1}{x} \mbox{\hspace{1cm}setting\hspace{1cm}}
x = e^{r_{1} \ln x_{c}}
\]
 
\item calculate the rejection function $g(x)$ and:
\begin{itemize}
\item if $r_{2} > g(x)$ reject $x$ and go back to 1;
\item if $r_{2} \leq g(x)$ accept $x$.
\end{itemize}
\end{enumerate}
 
To apply the Migdal correction \cite{bib-MIGD} all it has to be done is to
multiply the rejection function by the Migdal correction
factor:
 
\[
C_M (\epsilon)  =\frac{1 + C_0 / \epsilon_c^2}
               {1 + C_0 / \epsilon^2} 
\]
 where
\[
C_0 =\frac{nr_0 \lambdar^2 }{\pi}, \hspace{1cm} \epsilon_c = \frac{k_{c}}{E}
\]
\begin{DLtt}{MM}
\item[$n$]           electron density in the medium
\item[$r_0$]         classical electron radius
\item[$\lambdar$]    reduced Compton wavelength of the electron.
\end{DLtt}
This correction decreases the cross-section for low photon energy.
 
After the successful sampling of $\epsilon$, \Rind{GBREME} generates the
polar angles of the radiated photon with respect to the parent
electron's momentum. It is difficult to find in the literature
simple formulas for this angle. For example the double differential
cross section reported by Tsai~\cite{bib-TSAI,bib-TSAI-err} is the 
following:
\begin{eqnarray*}
\frac{d \sigma}{dkd \Omega} 
& = & \frac{2 \alpha^{2}e^{2}}{\pi k m^{4}}
  \left\{ \left[ \frac{2\epsilon-2}{(1+u^2)^2}+
\frac{12u^2(1-\epsilon)}{(1+u^2)^4}\right]
      Z(Z+1)  \right. \\
&   & \mbox{} + \left. \left[ \frac{2-2\epsilon-\epsilon^{2}}{(1+u^2)^2}- 
      \frac{4u^2(1-\epsilon)}{(1+u^2)^4}
      \right]
      \left[ X-2Z^{2}f_{c}((\alpha Z)^{2})\right]
      \right\} \\
u & = & \frac{E \theta}{m} \\
X & = & \int_{t_{min}}^{m^{2}(1+u^{2})^{2}}
{\left [ G_{Z}^{el}(t) + G_{Z}^{in}(t) \right ] \frac{t-t_{min}}
{t^{2}} dt} \\
G_{Z}^{el, in}(t) & & \mbox{atomic form factors} \\
t_{min} & = & \left [ \frac{k m^{2} (1+u^{2})}{2 E (E-k)} \right ] ^{2}
 = \left [ \frac{\epsilon m^{2} (1+u^{2})}{2 E (1-\epsilon)} \right ] ^{2}
\end{eqnarray*}

This distribution is complicated to sample, and it is anyway only an
approximation to within few percent, if nothing else, due to
the presence of the atomic form-factors. 
The angular dependence is contained in the
variable $u = E \theta m^{-1}$. For a given value
of $u$ the dependence of the shape of the function on $Z$, $E$,
$\epsilon = k/E$ is very weak.
Thus, the distribution can be approximated by a function
\begin{equation}
f(u) = C \left( u e^{-au} + d u e^{-3au} \right)
\end{equation}
where
\[
C = \frac{9a^{2}}{9 + d} \hspace{1cm} a = 0.625 \hspace{1cm} 
d = 0.13 \: \left ( 0.8+\frac{1.3}{Z} \right ) \left (100+\frac{1}{E} \right )
(1+\epsilon)
\]
where $E$ is in GeV. While this approximation is good at high energies,
it becomes less accurate around few MeV. However in that region the
ionisation losses dominate over the radiative losses.

The sampling of the function $f(u)$ can be done in the following way
($r_{i},\: i=1,2,3$ are uniformly distributed random numbers 
in [0,1]):
\begin{enumerate}
\item Choose between $u e^{-au}$ and $d u e^{-3au}$:
\[
b = \left \{ \begin{array}{Ll}
a & \mbox{if\hspace{0.5cm}}r_{1} < 9/(9+d) \\
3a & \mbox{if\hspace{0.5cm}}r_{1} \geq 9/(9+d) 
\end{array} \right .
\]
\item Sample $u e^{-bu}$:
\[
u=-\frac{\log ( r_{2} r_{3}) }{b}
\]
\item check that:
\[
u \leq u_{max} = \frac{E \pi}{m}
\]
otherwise go back to 1.
\end{enumerate}

The probability of failing in the last test is reported in 
table~\ref{tb:phys341-1}.

\begin{table}
\begin{centering}
\begin{tabular}{|l|l|}
\multicolumn{2}{c}{$\displaystyle
P = \int^{\infty}_{u_{max}}{f(u) \: du} \hfill $} \\ [0.5cm]
\hline
E (MeV) & P(\%) \\ \hline
0.511 & 3.4 \\
0.6 &  2.2 \\
0.8 & 1.2 \\
1.0 & 0.7 \\
2.0 & $<$ 0.1 \\ \hline
\end{tabular}
\caption{Angular sampling efficiency}
\label{tb:phys341-1}
\end{centering}
\end{table}

The function $f(u)$ can be used also to describe
the angular distribution of the photon in $\mu$ bremsstrahlung and to
describe the angular distribution in photon pair production.

The azimuthal angle, $\Phi$, is generated isotropically.
This information is used to calculate the momentum vector of the radiated
photon, to transform it to the {\tt GEANT} coordinate system and 
to store the result into common block \FCind{/GCKING/}. Also, the momentum of the parent electron 
is updated.

\subsection{Restrictions}

\begin{enumerate}
\item
Target materials composed of compounds or mixtures are treated identically
to elements using the effective atomic number $Z$ calculated in
\Rind{GSMIXT}
(this is not correct when computing the mean free path!).
\end{enumerate}
