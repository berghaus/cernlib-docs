%%%%%%%%%%%%%%%%%%%%%%%%%%%%%%%%%%%%%%%%%%%%%%%%%%%%%%%%%%%%%%%%%%%
%                                                                 %
%  GEANT manual in LaTeX form                              %
%                                                                 %
%  Michel Goossens (for translation into LaTeX)                   %
%  Version 1.00                                                   %
%  Last Mod. Jan 24 1991  1300   MG + IB                          %
%                                                                 %
%%%%%%%%%%%%%%%%%%%%%%%%%%%%%%%%%%%%%%%%%%%%%%%%%%%%%%%%%%%%%%%%%%%
\Origin{H.C.Fesefeldt, F.Carminati}
\Documentation{F. Carminati}
\Version{Geant 3.10}\Routid{PHYS460}
\Submitted{20.12. 85}     \Revised{16.12.93}
\Makehead{Muon-nucleus interactions}
\section{Subroutines}
\Shubr{GMUNUI}{}
\Rind{GMUNUI} computes and stores in the appropriate bank the value of
the muon-nucleus cross-section for a given material.
It is called at initialisation time by \Rind{GPHYSI}.
\Shubr{GMUNU}{}
\Rind{GMUNU} is called by \Rind{GTMUON} every time a muon-nucleus interaction
has to happen. It generates the final state particles as well as the outgoing
muon. A call to \Rind{GUHADR} is performed if {\tt IMUNU} (which is the
variable set by the {\tt  MUNU} data record) is equal to 1. 
An inelastic interaction is forced (which could also be a fission in case 
of heavy materials).
The secondaries from the $\pi$-nucleus interaction are always generated
if {\tt IMUNU}
is equal to 1, irrespectively of the value of {\tt IHADR}.

{\bf Note:} {\tt IMUNU} should be set to 1 only when using the {\tt GHEISHA}
hadronic interactions package. Setting it to 1 when using {\tt FLUKA} can
give unpredictable results.

\Sfunc{GMUSIG}{VALUE = GMUSIG(E,E1,COSTET)}
\begin{DLtt}{MMMMMMMM}
\item[E] ({\tt REAL}) initial $\mu$ energy;
\item[E1] ({\tt REAL}) final $\mu$ energy;
\item[COSTET] ({\tt REAL}) $\mu$ scattering angle.
\end{DLtt}
This function returns the value of the differential cross-section in
millibarns for a muon of energy {\tt E} to generate 
a nuclear interaction recoil with
energy {\tt E1} at an angle the cosine of which is
{\tt COSTET}.

\section{Method}
This set of routines generates the interactions of muons with the nuclei of
the tracking material. The code is a straight translation into the {\tt GEANT} 
style
of the corresponding code from the {\tt GHEISHA} Monte Carlo Program.
The {\tt GHEISHA} routines {\tt CASMU} and {\tt CALIM} have been used 
\cite{bib-GHEI}.
 
The information contained in this chapter is mainly taken from the {\tt GHEISHA}
manual to which the user is referred.
The muon-nucleus inelastic cross-section is taken as 0.0003 mb for
a nucleon when the energy of the incoming muon is $E<30$ GeV, and slowly
increases for $E>30$ GeV according to the law:
\begin{center}
\[
\sigma = 0.3 * (E/30)^{0.25}\quad [\mu \rm b]
\]
\end{center}
The energy and angle of final state muon is generated according to the ``free
quark parton model''. If $E$ is the energy of the incoming muon and
$\Omega$ and $W$ the
angle and the energy of the outgoing muon, the differential cross-section
can be written as:
 
\[
\frac{d \sigma}{d \Omega dW} =\gamma ( \sigma_T + \epsilon \sigma_S)
\]
where:
\begin{eqnarray*}
\Gamma   & = & \frac{k \alpha}{2 \pi}\frac{W}{E}\frac{1}{1- \epsilon } \\
\epsilon & = & \left [ 1+2 \frac{Q^2 + \nu^2}{Q^2}\tan^2 \theta^2 \right]
\end{eqnarray*}
 
$Q^2$ and $\nu$ are the normal scaling variables expressed by:
 
\[
Q^2 = -q^2 = 2(EW-|p||p'|\cos\theta-m^2_{-\mu})\mbox{\quad and\quad} \nu
= E-W 
\]
 here $\sigma_T$ and $\sigma_S$ are the photo-absorption
cross-sections for transverse and
longitudinal photons respectively for which the relation used is:
 
\[
\sigma_S = 0.3 \left ( 1- \frac{1}{1.868}{Q^2}{\nu}\right)  \sigma_T
\]
and $\sigma_T$ is assumed to be constant $\sigma_T = 0.12 $ mb.
For the incident flux $K$ of
the photons Gillman's convention is used:
\[
K = \nu + Q^2/2\nu
\]
 A three-dimensional importance sampling in the variables $E$, $W$ and
$\theta$ is performed each time an interaction has to occur.
 
This set of routines only works if \Rind{GUPHAD} calls \Rind{GPGHEI} and
\Rind{GUHADR} calls \Rind{GHEISH}.
The hadrons are generated in an approximate way. The virtual photon is
replaced by a real pion of random charge with the same kinetic energy. Then the
\Rind{GUHADR} routine is called to generate a $\pi$-nucleus inelastic
scattering. While the final state generated this way gives a good
approximation for calorimetric purposes, the kinematics of the final state
may be a rather poor approximation of reality.
The muon-nucleus interactions are activated by the {\tt  MUNU} data record of
{\tt GEANT}. After a muon-nucleus interaction the muon will still be the current
particle. If {\tt MUNU} 1 has been specified, secondaries coming from the
interaction of the virtual photon with the nucleus will be in the {\tt GEANT}
temporary stack. If {\tt MUNU} 2 has been specified, then the secondary
particles will not be generated and the energy lost by the muon will be
added to {\tt DESTEP}.
For each material a table of muon-nucleus cross-sections is stored at
initialisation time. See material bank structure for details.
