%%%%%%%%%%%%%%%%%%%%%%%%%%%%%%%%%%%%%%%%%%%%%%%%%%%%%%%%%%%%%%%%%%%
%                                                                 %
%  GEANT manual in LaTeX form                                     %
%                                                                 %
%  Michel Goossens (for translation into LaTeX)                   %
%  Version 1.00                                                   %
%  Last Mod. Jan 24 1991  1300   MG + IB                          %
%                                                                 %
%%%%%%%%%%%%%%%%%%%%%%%%%%%%%%%%%%%%%%%%%%%%%%%%%%%%%%%%%%%%%%%%%%%
\Origin{R.Brun, G.N.Patrick}
\Version{Geant 3.16}\Routid{CONS310}
\Submitted{23.01.84}     \Revised{19.10.94}
\Makehead {Branching Ratios and Particle Decay Modes}
\Shubr{GSDK}{(IPART,BRATIO,MODE)}
\begin{DLtt}{MMMMMMMM}
\item[IPART]       ({\tt INTEGER}) {\tt GEANT} particle number;
\item[BRATIO]   ({\tt REAL}) array of branching ratios in \%,
maximum 6;
\item[MODE]     ({\tt INTEGER}) array of partial decay modes.
\end{DLtt}
\Rind{GSDK} stores the branching ratios and partial decay modes for two
and three body particle decays into the data structure {\tt JPART}.
The decay modes should be coded into the array {\tt MODE} such that:
\begin{center}
{\tt MODE(I) =  10$^4$ N3 + 10$^2$ N2 + N1} \hspace{2cm}
for {\tt I=1,$\cdots$,6}
\end{center}
where:
\begin{DLtt}{MMMM}
\item [N1]     particle number for decay product 1;
\item [N2]     particle number for decay product 2;
\item [N3]     particle number for decay product 3 (if any).
\end{DLtt}
It is important to note the following:
\begin{itemize}
\item  prior to calling \Rind{GSDK}, all parent and secondary particles should
have been defined by \Rind{GSPART};
\item  if less than six decay modes are defined the remaining elements
of {\tt BRATIO} and {\tt MODE} must be set to zero.
\end{itemize}
For a given particle, {\tt IPART}, the decay parameters are stored into
the {\tt JPART} data structure as follows:

\begin{DLtt}{MMMMMMMMMM}
\item[JPA] {\tt = LQ(JPART-IPART)} pointer to the bank containing the
information on the particle {\tt IPART};
\item[JDK1] {\tt = LQ(JPA-1)} pointer to branching ratios bank;
 \item[JDK2] {\tt = LQ(JPA-2)} pointer to decay modes bank;
\item[BR(I)] {\tt = Q(JDK1+I)}     I$^{th}$ branching ratio;
 \item[MODE(I)] {\tt = IQ(JDK2+I)}   I$^{th}$ decay mode, where
{\tt I=1,$\cdots$,6}.
\end{DLtt}
When a particle decays during tracking,
the routine \Rind{GDECAY} is called. If branching ratios and modes
have been stored by \Rind{GSDK}, then \Rind{GDECAY}
generates the decay products in the 2- or 3-body phase space.
If no branching ratios have been defined by \Rind{GSDK},
the user routine
\Rind{GUDCAY} is called, where the user can code special decay modes and
branching ratios. All data taken from Aguilar Benitez \cite{bib-AGUI}.
\newpage

\[
\begin{array}{|l|l|}
\hline
\makebox[4cm][|l|]{Parent particle} &
\makebox[8cm][c|]{Decay \hfill Branching ratio (\%)} \\ \hline

\mu^{\pm}      &
\begin{array}{lr}
\makebox[5cm][l]{$e^{\pm}\nu\nu$}        &  100.00
\end{array} \\ \hline

\pi^{0}        &
\begin{array}{lr}
\makebox[5cm][l]{$\gamma\gamma$}         &  98.802   \\
\gamma e^+e^-          &  1.198
\end{array} \\ \hline

\pi^{\pm}      &
\begin{array}{lr}
\makebox[5cm][l]{$\mu^{\pm}\nu$}         &  100.00 \\
\end{array} \\ \hline

K^0_{l}        &
\begin{array}{lr}
\makebox[5cm][l]{$\pi^- e^+ \nu$}        &  19.35    \\
\pi^+ e^- \nu          &  19.35    \\
\pi^-\mu^+\nu          &  13.50    \\
\pi^+\mu^-\nu          &  15.50    \\
\pi^0\pi^0\pi^0        &  21.50    \\
\pi^+\pi^-\pi^0        &  12.38
\end{array} \\ \hline

K^{\pm}        &
\begin{array}{lr}
\makebox[5cm][l]{$\mu^{\pm}\nu$}         & 63.51     \\
\pi^{\pm}\pi^0         & 21.17     \\
\pi^{\pm}\pi^+\pi^-    &  5.59     \\
e^{\pm}\nu\pi^0        &  4.82     \\
\mu^{\pm}\nu\pi^0      &  3.18     \\
\pi^{\pm}\pi^0\pi^0    &  1.73
\end{array} \\ \hline

K^0_s          &
\begin{array}{lr}
\makebox[5cm][l]{$\pi^+\pi^-$}           &  68.61    \\
\pi^0\pi^0             &  31.39
\end{array} \\ \hline

\eta           &
\begin{array}{lr}
\makebox[5cm][l]{$\gamma\gamma$}         &  38.80    \\
\pi^0\pi^0\pi^0        &  31.90    \\
\pi^+\pi^-\pi^0        &  23.60    \\
\pi^+\pi^-\gamma       &  4.88     \\
e^+e^-\gamma           &  0.50     \\
\pi^0\gamma\gamma      &  0.071
\end{array} \\ \hline

\Lambda        &
\begin{array}{lr}
\makebox[5cm][l]{$p\pi^-$}               &  63.90    \\
n\pi^0                 &  35.80
\end{array} \\ \hline

\Sigma^{+}     &
\begin{array}{lr}
\makebox[5cm][l]{$p\pi^0$}               &  51.57    \\
n\pi^+                 &  48.30
\end{array} \\ \hline

\Sigma^{0}     &
\begin{array}{lr}
\makebox[5cm][l]{$\Lambda\gamma$}        &  100.00
\end{array} \\ \hline

\Sigma^{-}     &
\begin{array}{lr}
\makebox[5cm][l]{$n\pi^-$}               & 100.00
\end{array} \\ \hline

\Xi^0          &
\begin{array}{lr}
\makebox[5cm][l]{$\Lambda\pi^0$}         & 100.00
\end{array} \\ \hline

\Xi^-          &
\begin{array}{lr}
\makebox[5cm][l]{$\Lambda\pi^-$}         & 100.00
\end{array} \\ \hline

\Omega^-       &
\begin{array}{lr}
\makebox[5cm][l]{$\Lambda K^-$}          &  67.80    \\
\Xi^0\pi^-             &  23.60    \\
\Xi^-\pi^0             &  8.60
\end{array} \\ \hline

\bar{\Lambda}  &
\begin{array}{lr}
\makebox[5cm][l]{$\bar{p}\pi^+$}         &  63.90    \\
\bar{n}\pi^0           &  35.80
\end{array} \\ \hline

\bar{\Sigma}^- &
\begin{array}{lr}
\makebox[5cm][l]{$\bar{p}\pi^0$}         &  51.57    \\
\bar{n}\pi^-           &  48.30
\end{array} \\ \hline

\bar{\Sigma}^0 &
\begin{array}{lr}
\makebox[5cm][l]{$\Lambda\gamma$}        &  100.00
\end{array} \\ \hline

\bar{\Sigma}^+ &
\begin{array}{lr}
\makebox[5cm][l]{$\bar{n}\pi^+$}         &  100.00
\end{array} \\ \hline

\bar{\Xi}^0    &
\begin{array}{lr}
\makebox[5cm][l]{$\bar{\Lambda}\pi^0$}   &  100.00
\end{array} \\ \hline

\bar{\Xi}^+    &
\begin{array}{lr}
\makebox[5cm][l]{$\bar{\Lambda}\pi^+$}   &  100.00
\end{array} \\ \hline

\end{array}
\]

\[
\begin{array}{|l|l|}
\hline
\makebox[4cm][|l]{Parent particle} &
\makebox[8cm][|c|]{Decay \hfill Branching ratio (\%)} \\ \hline

\bar{\Omega}^+ &
\begin{array}{lr}
\makebox[5cm][l]{$\bar{\Lambda}K^+$}     &  67.80    \\
\bar{\Xi}^0\pi^+       &  23.60    \\
\bar{\Xi}^+\pi^0       &  8.60
\end{array} \\ \hline

\end{array}
\]
