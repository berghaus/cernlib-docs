%%%%%%%%%%%%%%%%%%%%%%%%%%%%%%%%%%%%%%%%%%%%%%%%%%%%%%%%%%%%%%%%%%%
%                                                                 %
%  GEANT manual in LaTeX form                                     %
%                                                                 %
%  Michel Goossens (for translation into LaTeX)                   %
%  Version 1.00                                                   %
%  Last Mod. Jan 24 1991  1300   MG + IB                          %
%                                                                 %
%%%%%%%%%%%%%%%%%%%%%%%%%%%%%%%%%%%%%%%%%%%%%%%%%%%%%%%%%%%%%%%%%%%
\Origin{G.Tromba, P.Bregant}
\Submitted{10.10.89}\Revised{16.12.93}
\Version{Geant 3.16}\Routid{PHYS250}
\Makehead{Total cross-section for Rayleigh scattering}
\section{Subroutines}
\Shubr{GRAYLI}{}
 
\Rind{GRAYLI} computes the mean free path, $\lambda$, for Rayleigh
scattering as a function of the medium and of the energy. It evaluates 
also an integral of atomic form factors which is used in the routine 
\Rind{GRAYL} (see {\tt [PHYS251]}) to sample the scattering angles. 
\Rind{GRAYLI} is called during initialization by \Rind{GPHYSI}.

\section{ Method }
 
\subsection{Total cross-section}
An empirical cross-section is used to produce the total cross-section
data~\latexhtml{\cite{bib-EGS3,bib-HUB1}}%
               {\cite{bib-EGS3}, \cite{bib-HUB1}}:
\begin{equation}
\sigma_c \left( Z,E \right) = aE^3 + b E^2 + c E + d 
\mbox{ \hspace{1cm} barn/atom }
\end{equation}
The values of the coefficients are stored in the 
{\tt DATA} statement within the routine
\Rind{GRAYLI}. For each element the fit was obtained over 27 experimental
values of the total coherent cross-section. The accuracy of the
fit is estimated to be about $\Delta \sigma/\sigma \approx 10\%$, but
it is better for most of the elements.
 
\subsection{The atomic form factors}
Different empirical formulae are used to produce the atomic form factor
F(Z,E) data \cite{bib-STOR}. For $Z = 1-3, 91-99$ the empirical expression is:
\begin{equation}
F \left( Z,E \right) = a_1 E^7 + b_1 E^6 + c_1 E^5 + d_1 E^4 + e_1 E^3
+ f_1 E^2 + g_1 E + h_1
\end{equation}
for Z=100 the formula used is:
\begin{equation}
F \left( Z,E \right) = a_2 E^5 + b_2 E^4 + c_2 E^3 + d_2 E^2 + e_2 E + f_2
\end{equation}
for the other elements $F$ is calculated as:
\begin{equation}
F \left( Z,E \right) = \left\{ \begin{array}{ll}
            a_3 E^3 + b_3 E^2 + c_3 E + d_3 & \mbox{ if } E \leq E_c   \\
            a_4 E^3 + b_4 E^2 + c_4 E + d_4 & \mbox{ if } E > E_c 
                              \end{array} \right .
\end{equation}
The value of the energy $ E_c $ depends on $Z$ and it is set in the
array {\tt ELIM}. The values of the coefficients are stored in the 
{\tt DATA} statement in the routine \Rind{GRAYLI}. For each element 
the fits were performed over 97 tabulated
values of the form factors. The accuracy of the
fit is estimated to about $\Delta \sigma/\sigma = 10\%$
for $E \leq 1$ MeV for most of the elements.
The integral over the atomic form factors 
$ A \left( E_i \right)$ is defined as follows:
\begin{equation}
A \left( E_i \right) = 2 \int_{0}^{E_i} {\left| F_T \left( E \right)
 \right|}^2 E dE = \int_{0}^{E_i}{\left| F_T \left( E \right) \right |}^2
dE^2
\end{equation}
where $ F_T \left( E \right) $ is the total atomic form factor which is
function of the energy E.
