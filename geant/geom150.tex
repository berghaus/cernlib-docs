%%%%%%%%%%%%%%%%%%%%%%%%%%%%%%%%%%%%%%%%%%%%%%%%%%%%%%%%%%%%%%%%%%%
%                                                                 %
%  GEANT manual in LaTeX form                              %
%                                                                 %
%  Michel Goossens (for translation into LaTeX)                   %
%  Version 1.00                                                   %
%  Last Mod. Jan 24 1991  1300   MG + IB                          %
%                                                                 %
%%%%%%%%%%%%%%%%%%%%%%%%%%%%%%%%%%%%%%%%%%%%%%%%%%%%%%%%%%%%%%%%%%%
\Origin{R.Brun, F.Bruyant, A.McPherson}
\Submitted{17.12.83}               \Revised{18.11.93}
\Version{Geant 3.16}\Routid{GEOM150}
\Makehead{Division of a volume - general case}

\Shubr{GSDVX}{(CHNAME,CHMOTH,NDIV,IAXIS,STEP,C0,NUMED,NDVMAX)}

Divide a volume in a given number of parts along a direction, with
a given step starting from an offset.

\begin{DLtt}{MMMMMMMM}
\item[CHNAME] ({\tt CHARACTER*4}) a unique name for the volume to be generated
by subdivision of the mother volume;
\item[CHMOTH] ({\tt CHARACTER*4}) volume that has to be subdivided;
\item[NDIV] ({\tt INTEGER}) number of divisions into which the mother volume
is to be divided;
\item[IAXIS] ({\tt INTEGER}) {\it axis} of the division.
\item[STEP] ({\tt REAL}) size of the divisions -- this value can be in
centimeters or degrees according to the value of {\tt IAXIS};
\item[C0] ({\tt REAL}) offset where division should start -- this value can be 
in centimeters or degrees according to the value of {\tt IAXIS};
\item[NUMED] ({\tt INTEGER}) medium number of the divisions -- this can be
different from the one of the mother, as the division cells may leave a
portion of the mother undivided (see below) --
if {\tt NUMED} $\leq 0$  the medium of the mother;
\item[NDVMX] ({\tt INTEGER}) expected (maximum) number of divisions -- if
$ \leq 0 $ or $ > 255 $, 255 is assumed.
\end{DLtt}

For more information on the division mechanism, see {\tt [GEOM130]} and
{\tt [GEOM140]}. For the moment either
{\tt NDIV} or {\tt STEP} must be set negative or 0, so that they
will be computed from the {\tt CHMOTH}'s size.
The case with both {\tt NDIV} and {\tt STEP}
positive is not coded yet. It would permit leaving different
gaps at both ends of the
{\tt CHMOTH}.

Provisionally the code consists of a call to either \Rind{GSDVN2} or
\Rind{GSDVT2}.

\Shubr{GSDVN2}{(CHNAME,CHMOTH,NDIV,IAXIS,C0,NUMED)}

Divide a volume in a given number of parts along a direction, 
starting from an offset.

\begin{DLtt}{MMMMMMMM}
\item[CHNAME] ({\tt CHARACTER*4}) a unique name for the volume to be generated
by subdivision of the mother volume;
\item[CHMOTH] ({\tt CHARACTER*4}) volume that has to be subdivided;
\item[NDIV] ({\tt INTEGER}) number of divisions into which the mother volume
is to be divided;
\item[IAXIS] ({\tt INTEGER}) {\it axis} of the division.
\item[C0] ({\tt REAL}) offset where division should start -- this value can be 
in centimeters or degrees according to the value of {\tt IAXIS};
\item[NUMED] ({\tt INTEGER}) medium number of the divisions -- this can be
different from the one of the mother, as the division cells may leave a
portion of the mother undivided (see below) --
if {\tt NUMED} $\leq 0$  the medium of the mother;
\end{DLtt}

The divisions start at the user specified coordinate value
and extend to the end of the volume. The range from this offset to
the upper coordinate limit of the mother volume will be divided
into the supplied number of cells. 
In the case of 
$\phi$ division of a complete tube or cone, the whole 360 degrees
will be divided into the user-supplied number of slices no matter
what the origin is. Specifying an origin for the division, in this
case, just moves the
division boundaries. This can be useful to avoid a rotation.
In all other cases the search routines will
assume that a point is in the mother if the coordinate value is
less than the value of the supplied offset.

\Shubr{GSDVT2}{(CHNAME,CHMOTH,STEP,IAXIS,C0,NUMED,NDVMX)}

Divide a volume along a direction with a given step starting from an offset.

\begin{DLtt}{MMMMMMMM}
\item[CHNAME] ({\tt CHARACTER*4}) a unique name for the volume to be generated
by subdivision of the mother volume;
\item[CHMOTH] ({\tt CHARACTER*4}) volume that has to be subdivided;
\item[STEP] ({\tt REAL}) size of the divisions -- this value can be in
centimeters or degrees according to the value of {\tt IAXIS};
\item[IAXIS] ({\tt INTEGER}) {\it axis} of the division;
\item[C0] ({\tt REAL}) offset where division should start -- this value can be
in centimeters or degrees according to the value of {\tt IAXIS};
\item[NUMED] ({\tt INTEGER}) medium number of the divisions -- this can be
different from the one of the mother, as the division cells may leave a
portion of the mother undivided (see below) --
if {\tt NUMED} $\leq 0$  the medium of the mother;
\item[NDVMX] ({\tt INTEGER}) expected (maximum) number of divisions -- if
$ \leq 0 $ or $ > 255 $, 255 is assumed.
\end{DLtt}

The division start at the user-specified coordinate value
and extend to the end of the volume. The range from origin to
upper coordinate limit of the mother volume is divided
in sections of the user supplied step. If the step is such that
the range of the mother cannot be filled with cells, the largest
possible number of cells is created.
The excess space up to the end
of the mother volume will be assumed to belong to the mother.

In the case of 
$\phi$ division of a complete tube or cone, the whole 360 degrees
will be filled with slices, no matter
what the origin is. Specifying an origin for the division, in this
case, just moves the
division boundaries. This can be useful to avoid a rotation.

In all other cases the search routines will
assume a point is just in the mother if the coordinate value is
less than the value of the user supplied origin.
