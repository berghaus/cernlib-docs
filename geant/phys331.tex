%%%%%%%%%%%%%%%%%%%%%%%%%%%%%%%%%%%%%%%%%%%%%%%%%%%%%%%%%%%%%%%%%%%
%                                                                 %
%  GEANT manual in LaTeX form                              %
%                                                                 %
%  Michel Goossens (for translation into LaTeX)                   %
%  Version 1.00                                                   %
%  Last Mod. Jan 24 1991  1300   MG + IB                          %
%                                                                 %
%%%%%%%%%%%%%%%%%%%%%%%%%%%%%%%%%%%%%%%%%%%%%%%%%%%%%%%%%%%%%%%%%%%
\Origin{D. Ward, L.Urb\'{a}n}
\Submitted{26.10.84} \Revised{16.12.93}
\Version{Geant 3.16}\Routid{PHYS331}
\Makehead{Simulation of the delta-ray production}
\section{Subroutines}
\Shubr{GDRAY}{}
\Rind{GDRAY} generates explicitly the delta-rays (see
{\tt [PHYS330]} for treatment of the ionization as continuous energy loss
and for the calculation of the total cross-section).
\begin{DLtt}{MMMMMMMMMM}
\item[input:]  common \FCind{/GCTRAK/}
\item[output:] common \FCind{/GCKING/}
\end{DLtt}
The routine is called from the tracking routines
\Rind{GTELEC}, \Rind{GTMUON}, \Rind{GTHION} and \Rind{GTHADR} 
when a charged particle reaches
its interaction point.
\section {Method }
\subsection{Differential cross-section}
The differential cross-section of the $\delta$-ray production can
be written as in equations (\ref{eq:phys331-1}, \ref{eq:phys331-2})
\cite{bib-MES1}, \cite{bib-EGS3}, \cite{bib-PDGB}. For the
electron/electron (M\"{o}ller) scattering we have:
\begin{equation}
\label{eq:phys331-1}
\frac{d\sigma }{d \epsilon }=\frac{2 \pi Z r^2 _0 m }{\beta^2 (E-m)}
\left[ \frac{(\gamma -1 )^2}  {\gamma^2 }+\frac{1}{\epsilon}
\left(\frac{1}{\epsilon}-\frac{2 \gamma -1 } {\gamma^2 } \right) +
\frac{1}{1- \epsilon}\left(\frac{1} {1- \epsilon} \frac{2 \gamma - 1}
{\gamma^2 }\right)  \right]
\end{equation}
and for the positron-electron (Bhabha) scattering:
\begin{equation}
\label{eq:phys331-2}
\frac{d \sigma}{d \epsilon}=\frac{2 \pi Z r^2_0 m }{(E-m)}\left[
\frac{1} {\beta^2 \epsilon^2}-\frac{B_1}{\epsilon}+B_2 - B_3 \epsilon
+ B_4 \epsilon^2\right]
 \end{equation}
 
where

\[
\begin{array}{LcL@{\hspace{1.6cm}}LcL}
Z        & = & \mbox{atomic number of the medium}   &
E        & = & \mbox{energy of the incident particle} \\ 
M        & = & \mbox{rest mass of the incident particle} &
\gamma   & = & \frac{E}{M}   \\ [.2cm]
\beta^2  & = & 1- \frac{1} {\gamma^2 } &
 y       & = & \frac{1} {\gamma + 1}               \\ [.2cm]
 B_1     & = & 2-y^2                   &
 B_2     & = & (1-2y)(3+y^2 )                       \\
 B_3     & = & (1-2y)^2+(1-2y)^3 &
 B_4     & = & (1-2y)^3  \\
\epsilon & = & \frac{T} {E-m}
\end{array}
\]

with $T$ the kinematic energy of the scattered electron (of the lower
energy in the case of $\Pem\Pep$ scattering).
 
The kinematical limits for the variable $\epsilon$ are:

\[
\epsilon_0 = \frac{\mbox{\tt TCUT}}{E-m} \leq \epsilon \leq \frac{1}{2}
\mbox{\hspace{.2cm} for \Pem\Pem} \hspace{2cm}
\epsilon_0 = \frac{\mbox{\tt TCUT}}{E-m} \leq \epsilon \leq 1
\mbox{\hspace{.2cm} for \Pep\Pem}
\]
For the other charged particles the differential cross-section
can be written:
\[
\begin{array}{LcLl}
\frac {d\sigma }{dT}&=& 2 \pi Z r^2_ 0 m\frac{1}{\beta^2 }\frac{1}{T^2}
\left[ 1- \beta^2 \frac{T} {\mbox{\tt TMAX }}\right] 
& \mbox{for spin 0 particle} \\ [.3cm]
\frac{d \sigma} {dT}&=& 2 \pi Z r^2_0 m \frac{1}{\beta^2 } \frac{1}{T^2}
\left[1- \beta^ 2 \frac{T}{ \mbox{\tt TMAX}}+ \frac{T^2} {2E^2}
 \right] &  \mbox{for spin 1/2 particle}
\end{array}
\]
where {\tt TMAX} is the maximum energy transferable to the free electron:
\[
\mbox{\tt TMAX} = \frac{2m(\gamma^2 -1)} {1+2\gamma (m/M) +
(m/M)^2}
\]
and {\tt TCUT} is the energy threshold for the $\delta$-ray emission:
$ \mbox{\tt TCUT} \leq T \leq \mbox{\tt TMAX} $

\subsection{Sampling} 
Apart from the normalisation, the cross-section can be written as
\[
\frac{d\sigma}{d\epsilon}=f(\epsilon) g(\epsilon),
\]
where, for $\Pem\Pem$ scattering,
\begin{eqnarray*}
f(\epsilon)&=&\frac{1}{\epsilon^2} \frac{\epsilon_0 }{1- 2\epsilon_0} \\
g(\epsilon)&=&\frac{4}{9\gamma^2 - 10 \gamma + 5}\left[(\gamma -1)^2
\epsilon^2 - (2 \gamma^2 +2\gamma -1) \frac{\epsilon} {1- \epsilon }+
\frac{\gamma^2}{(1- \epsilon )^2 }\right]
\end{eqnarray*}
and for $\Pep\Pem$ scattering
\begin{eqnarray*}
  f(\epsilon)&=&\frac{1}{\epsilon^2} \frac{\epsilon_0}{1- \epsilon_0 } \\
  g(\epsilon)&=&\frac{B_0 -B_1 \epsilon +B_2 \epsilon^2
     -B_3 \epsilon^3 +B_4 \epsilon ^4}{B_ 0-B_1\epsilon_0
+B_2\epsilon^2_0
    -B_3 \epsilon^3_0 +B_4 \epsilon^4_0 }
\end{eqnarray*}
Here $ B_0=\gamma^2/(\gamma^2-1)$ and
all the other quantities have been defined above.
For the other charged particles:
\begin{eqnarray*}
f(T) &=& \left(\frac{1}{\mbox{\tt TCUT}} -\frac{1}{\mbox{\tt TMAX}}\right)  
\frac{1}{T} \\
g(T) &=& 1 - \beta^2\frac{T}{\mbox{\tt TMAX}} +\frac{T^2}{2E^2}
\mbox{\quad(last term for spin-$\frac{1}{2}$ particle only)}
\end{eqnarray*}
\Rind{GDRAY} samples the variable $\epsilon$ by:
\begin{enumerate}
\item sample $\epsilon$ from $f(\epsilon)$
\item calculate the rejection function $g(\epsilon)$ and accept the
sampled $\epsilon$ with a probability of $g(\epsilon)$.
\end{enumerate}
 
After the successful sampling of $\epsilon$, \Rind{GDRAY} generates the polar
angles of the scattered electron with respect to the direction of the
incident particle. The azimuthal angle $\phi$ is generated isotropically;
the polar angle
$\theta$ is calculated from the energy momentum conservation.
This information
is used to calculate the energy and momentum of both scattered
particles and to transform them into the {\tt GEANT} coordinate system.
