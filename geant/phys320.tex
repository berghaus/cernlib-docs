%%%%%%%%%%%%%%%%%%%%%%%%%%%%%%%%%%%%%%%%%%%%%%%%%%%%%%%%%%%%%%%%%%%
%                                                                 %
%  GEANT manual in LaTeX form                                     %
%                                                                 %
%  Michel Goossens (for translation into LaTeX)                   %
%  Version 1.00                                                   %
%  Last Mod. Jan 24 1991  1300   MG + IB                          %
%                                                                 %
%%%%%%%%%%%%%%%%%%%%%%%%%%%%%%%%%%%%%%%%%%%%%%%%%%%%%%%%%%%%%%%%%%%
\Origin{R.Brun, M.Hansroul}
\Revision{F.Carminati, G.R.Lynch}
\Documentation{M.Hansroul, G.R.Lynch}
\Submitted {17.06.85}  \Revised {16.12.93}
\Version{Geant 3.16}\Routid{PHYS320}
\Makehead {Gaussian multiple scattering}
\section{Subroutines}

\Shubr{GMGAUS}{(BETA2,DIN*)}

\begin{DLtt}{MMMMMMMMMMMM}
\item[BETA2] ({\tt REAL}) $\beta^2$;
\item[DIN]  ({\tt REAL}) array of 3 containing the 
direction of the scattered particle with respect to the
incoming direction.
\end{DLtt}
This routine is called by \Rind{GTELEC}, \Rind{GTHADR},
\Rind{GTMUON} and \Rind{GTHION} via
\Rind{GMULTS} when the variable {\tt IMULS},
controlled by the data record {\tt MULS} is equal to 3.

\section{Method}

\subsection{Approximations in use}
 
A charged particle traversing a medium is deflected by many
scatters, mostly at small angle. These scatters are due to
the interaction with the Coulomb field of the nuclei, and
to a lesser degree, to the electron field, hence the name
of Coulomb scattering. For hadronic projectiles, however,
the strong interaction contributes to multiple scattering.
Multiple scattering is well described by Moli\`{e}re
theory~\cite{bib-BET1}.
Moli\`{e}re multiple scattering theory is used by default
in {\tt GEANT} (see {\tt [PHYS325]}). 
We define $\theta_0 = \theta^{rms}_{plane} = 
\theta^{rms}_{space}/\sqrt{2}$ as the r.m.s. of the angle between the
directions projected on a plane 
of a particle before and after traversing a thickness $t$
of absorber. In this case a simple form for the multiple Coulomb
scattering of singly charged particles which is widely used is:
\[
\theta_0 \approx \frac{15 MeV}{E \beta^2} \sqrt{\frac{t}{X_0}}
\]

where $X_0$ is the radiation length of the absorber. This form was
proposed by~\cite{bib-ROS1}, \cite{bib-ROS2}. It can
introduce large errors because it ignores significant dependences from
pathlength and $Z$. To compensate for this, a similar formula was
proposed~\cite{bib-HIG1}, \cite{bib-HIG2}:

\begin{equation}
\label{eq:phys320-1}
\theta_0 \approx \frac{14.1 MeV}{E \beta^2} \sqrt{\frac{t}{X_0}}
\left [ 1 + 0.038 \log \left( \frac{t}{X_0} \right ) \right ]
\end{equation}

A form taking into account the $\beta$ and $z$ dependence of
the particle has been proposed by~\cite{bib-LINC}:

\begin{equation}
\label{eq:phys320-2}
\theta_0 = \frac{S_{2}}{E \beta^2} \sqrt{\frac{t}{X_0}}
\left [ 1 + \epsilon \log_{10} \left( \frac{t Z_{inc}^2}{X_0 \beta^2} 
\right ) \right ]
\end{equation}

The problem with the formulae (\ref{eq:phys320-1}) and (\ref{eq:phys320-2})
is that if the distance $t$ in the absorber is travelled in two steps
$t = t_1 + t_2$, the r.m.s. angle $\theta_0(t) = \theta_0(t_1 +
t_2) \neq \sqrt{\theta_0^2(t_1) + \theta_0^2(t_2)}$, limiting their use
in a MonteCarlo like {\tt GEANT}.

Much of the difficulty in approximating multiple Coulomb scattering
in terms of the radiation length is that the number of radiation
lengths is a poor measure of the scattering. A better expression,
which is used in {\tt GEANT} has been proposed by \cite{bib-LINC}:

\begin{equation}
\theta_0^2 = \frac{\chi_c^2}{1+F^2} \left [
\frac{1+\nu}{\nu} \log( 1+\nu) -1 \right ]
\end{equation}

where

\begin{eqnarray*}
\nu & = & \frac{\Omega_0}{2 ( 1-F)} \\
F & = & \mbox{fraction of the tracks in the sample} \\
\Omega_0 & = & \frac{\chi_c^2}{1.167 \chi_{\alpha}^2} \mbox{\hspace{0.5cm}
is the mean number of scatters}
\end{eqnarray*}

$\chi_c^2$ and $\chi_{\alpha}^2$ are parameters of the Moli\`{e}re
theory, for which the reader is invited to see {\tt [PHYS325]},
and $1.167 \approx \exp(2\gamma -1)$ where $\gamma$ is the Euler's
constant $= 0.57721 \dots$. This form, which is the result of empirical
fits, is derived from the screened Rutherford cross section, which has
the form $2\chi_{\alpha}^2/(\chi_{\alpha}^2+\theta_0^2)$. For $F$ anywhere
from $0.9$ to $0.995$ this expression represents Moli\`{e}re scattering
to better than $2\%$ for $10 < \Omega_0 < 10^{8} $, which covers
singly charged relativistic particles ($\beta \approx 1$) in the
range $10^{-3} < t/X_0 < 100$.

In {\tt GEANT} we have adopted the values of $F=0.98$ and $\Omega_0 = 40,000$
scatters, obtaining the following formula:

\[
\theta_0 = 2.557 \chi_{cc} \frac{\sqrt{t}}{E \beta^2}
\]

where $\chi_{cc}$ is another parameter of the Moli\`{e}re theory
for which the reader is referred to {\tt [PHYS325]}. This algorithm
is implemented by the routine \Rind{GMGAUS}.
