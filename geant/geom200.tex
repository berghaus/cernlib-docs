%%%%%%%%%%%%%%%%%%%%%%%%%%%%%%%%%%%%%%%%%%%%%%%%%%%%%%%%%%%%%%%%%%%
%                                                                 %
%  GEANT manual in LaTeX form                              %
%                                                                 %
%  Michel Goossens (for translation into LaTeX)                   %
%  Version 1.00                                                   %
%  Last Mod. Jan 24 1991  1300   MG + IB                          %
%                                                                 %
%%%%%%%%%%%%%%%%%%%%%%%%%%%%%%%%%%%%%%%%%%%%%%%%%%%%%%%%%%%%%%%%%%%
\Origin{R.Brun, F.Carena}
\Submitted{01.06.83}             \Revised{14.12.93}
\Version{Geant 3.16}\Routid{GEOM200}
\Makehead{Rotation matrices}

The relative position of a volume inside its mother is expressed in
{\tt GEANT} by a translation vector and a rotation matrix which are 
arguments of the routines \Rind{GSPOS} and \Rind{GSPOSP}. The rotation
matrix expresses the transformation from the {\tt M}other {\tt R}eference
{\tt S}ystem to the {\tt D}aughter {\tt R}eference {\tt S}ystem.

A rotation matrix is described to {\tt GEANT} by giving the polar and
azimuthal angles of the axes of the {\tt DRS} ($x', y', z'$) in the
{\tt MRS} via the routine \Rind{GSROTM}.

\Shubr{GSROTM}{(IROT,THETA1,PHI1,THETA2,PHI2,THETA3,PHI3)}
\begin{DLtt}{MMMMMMMM}
\item[IROT] ({\tt INTEGER}) number of the rotation matrix;
\item[THETA1] ({\tt REAL}) polar angle for axis $x'$;
\item[PHI1] ({\tt REAL}) azimuthal angle for axis $x'$;
\item[THETA2] ({\tt REAL}) polar angle for axis $y'$;
\item[THI2] ({\tt REAL}) azimuthal angle for axis $y'$;
\item[THETA3] ({\tt REAL}) polar angle for axis $z'$;
\item[PHI3] ({\tt REAL}) azimuthal angle for axis $z'$.
\end{DLtt}
Stores rotation matrix {\tt IROT} in the data structure {\tt JROTM}. If the
matrix is not orthonormal, it will be corrected by setting $y' \perp x'$ and
then $z' = x' \times y'$. A warning message is printed in this case.

{\bf Note:}
the angles {\tt THETA} and {\tt PHI} must be given in degrees.
 
\section*{Examples of use}
The unit matrix is defined in the following way:

\[
\left . \begin{array}{lcl}
x' & \| & x \\
y' & \| & y \\
z' & \| & z
\end{array} \right \}
\Rightarrow
\left \{
\begin{array}{lcr@{\mbox{\hspace{3mm};\hspace{8mm}}}lcr}
\theta_1 & = & 90^{\circ} & \phi_1 & = & 0^{\circ} \\
\theta_2 & = & 90^{\circ} & \phi_2 & = & 90^{\circ} \\
\theta_3 & = & 0^{\circ} & \phi_3 & = & 0^{\circ} 
\end{array} \right .
\]

This is just an example. There is in fact no need to define a unit rotation
matrix. Giving the value 0 to the rotation matrix number in the call to
\Rind{GSPOS} and \Rind{GSPOSP} is equivalent to a positioning without 
rotation and it improves tracking performance.

The result of a $90^{\circ}$ counterclockwise rotation around $z$, followed
by a $90^{\circ}$ counterclockwise rotation around the new $x$ is a cyclic
shift of the axes: $x \rightarrow z', \; y \rightarrow x', \; z \rightarrow  
y'$. This is expressed by the following rotation matrix:

\[
\left . \begin{array}{lcl}
x' & \| & y \\
y' & \| & z \\
z' & \| & x
\end{array} \right \}
\Rightarrow
\left \{
\begin{array}{lcr@{\mbox{\hspace{3mm};\hspace{8mm}}}lcr}
\theta_1 & = & 90^{\circ} & \phi_1 & = & 90^{\circ} \\
\theta_2 & = & 0^{\circ} & \phi_2 & = & 0^{\circ} \\
\theta_3 & = & 90^{\circ} & \phi_3 & = & 0^{\circ} 
\end{array} \right .
\]

Sometimes the rotation matrix is known or it can be constructed. In this case
the arguments to the routine \Rind{GSROTM} can be calculated with the help
of the routine \Rind{GFANG} in the following way:

\begin{verbatim}
      DIMENSION ROTMAT(3,3), ROWMAT(3), PHI(3), THETA(3)
      LOGICAL ROTATE
      .
      .
      .
      DO 10 I=1,3
         ROWMAT(1) = ROTMAT(I,1)
         ROWMAT(2) = ROTMAT(I,2)
         ROWMAT(3) = ROTMAT(I,3)
         CALL GFANG(ROWMAT,COSTH,SINTH,COSPH,SINPH,ROTATE)
         THETA(I) = ATAN2(SINTH,COSTH)
         PHI(I)   = ATAN2(SINPH,COSPH)
  10  CONTINUE
      .
      . {\sl Transform to degrees}
      .
      CALL GSROTM(IROT,THETA(1),PHI(1),THETA(2),PHI(2),THETA(3),PHI(3))
\end{verbatim}
 
\Shubr{GPROTM}{(IROT)}
Prints the rotation matrix elements and angles.
\begin{DLtt}{MMMMMMMM}
\item[IROT]  ({\tt INTEGER}) rotation matrix number: if {\tt IROT}=0 all
rotation matrixes will be printed, if {\tt IROT}$<$0, matrix number
{\tt |IROT|} will be printed without header information.
\end{DLtt}
 
