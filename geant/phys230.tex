%%%%%%%%%%%%%%%%%%%%%%%%%%%%%%%%%%%%%%%%%%%%%%%%%%%%%%%%%%%%%%%%%%%
%                                                                 %
%  GEANT manual in LaTeX form                                     %
%                                                                 %
%  Michel Goossens (for translation into LaTeX)                   %
%  Version 1.00                                                   %
%  Last Mod. Jan 24 1991  1300   MG + IB                          %
%                                                                 %
%%%%%%%%%%%%%%%%%%%%%%%%%%%%%%%%%%%%%%%%%%%%%%%%%%%%%%%%%%%%%%%%%%%
\Origin{L.Urb\'{a}n}
\Revision{J. Chwastowski}
\Submitted{26.10.84}
\Revised{16.12.93}
\Version{GEANT 3.16}              \Routid {PHYS230}
\Makehead{Total cross-section for photoelectric effect}
\section{Subroutines}
\Shubr{GPHINI}{}
\Rind{GPHINI} steers initialisation of the constants. It calls \Rind{GHPRIN}
for initialisation of the atom decay rates, \Rind{GHSLIN} to initialise
atomic shell potentials and ionisation energies and \Rind{GPHXIN} for the 
initialisation of the cross-section constants. It is called by \Rind{GPHOTI}.
\Shubr{GPHRIN}{}
Initialisation of the atom decay rates. The routine is called by \Rind{GPHINI}.
\Shubr{GSHLIN}{}
Initialisation of the atomic shell potentials and ionisation energies.
\Rind{GSHLIN} is called by \Rind{GPHINI}.
\Shubr{GPHXIN}{}
Initialisation of the cross-section constants for elements with
${\tt Z} \leq  100$. It is called by \Rind{GPHINI}.
\Shubr{GPHXSI}{}
\Rind{GPHXSI} is called by \Rind{GPHYSI} for every tracking medium.
It calculates the cross-section coefficients.
The results of the calculations are stored in
the photoelectric effect constants bank created by this routine.
\Shubr{GFSHDC}{}
\Rind{GFSHDC} calls \Rind{GFRDT}, \Rind{GNRDT} and \Rind{GFSDPR} for the
probabilities of atomic state transitions and the transition energies
and stores them in the {\tt ZEBRA} bank. It is called by \Rind{GPHXSI}.
\Shubr{GFSHLS}{}
\Rind{GFSHLS} returns the shell potentials in eV for an atom.
It is called by \Rind{GFSHDC}.
\Shubr{GFRDT}{}
\Rind{GFRDT} returns probability distribution and energies of the atomic
radiative transitions. It is called by \Rind{GFSHDC}.
\Shubr{GNFRDT}{}
\Rind{GNFRDT} returns probability distribution and energies of the non-radiative
atomic transitions. It is called by \Rind{GFSHDC}.
\Shubr{GFSDPR}{}
\Rind{GFSDPR} returns probability of the shell radiative decay. It is called
by \Rind{GFSHDC}.
\Shubr{GPHOTI}{}
\Rind{GPHOTI} calculates the total cross-section for the photoelectric
effect. It tabulates the mean free path, $\lambda = 1/\Sigma$
(in cm), as a function of the medium and the energy.
If  ${\tt Z} \leq  100$ then it calls the function \Rind{GPHSG1} for the total
cross-section. Otherwise \Rind{GPHSIG} function is used.
\Rind{GPHOTI} is called at initialisation time by \Rind{GPHYSI}.
\Sfunc{GPHSG1}{VALUE = GPHSG1(EGAM)}
\Rind{GPHSG1} calculates the total cross-section for photoelectric effect.
It is called by {\tt GPHOTI} for ${\tt Z} \leq  100$.
\Sfunc{GPHSIG}{VALUE = GPHSIG(Z,EGAM)}
\Rind{GPHSIG} calculates the total cross-section for
the photoelectric effect  of a photon with energy {\tt EGAM} in material
with atomic number ${\tt Z} > 100$.  It is called by \Rind{GPHOTI}.
 
\section{Method}
 
\subsection{Materials with ${\tt Z} \leq 100$}
For media with $Z \leq 100$
we use SANDIA parametrisation \cite{bib-SANDIA}.
The cross-section for elements
%with {\it Z} between 1 and 100
was fitted with a linear combination of reciprocal powers of the photon
energy $E_\gamma$ ($E_\gamma$ in keV). The fits were performed in different
intervals {\it j} of the photon energy. In such an interval the cross-section
reads as:
\begin{equation}
 \mu_{ij}  =  \frac{C_{1,ij}}{E_\gamma}+\frac{C_{2,ij}}{E_\gamma^2}
            +\frac{C_{3,ij}}{E_\gamma^3}+\frac{C_{4,ij}}{E_\gamma^4}
             \ \ \  (cm^{2}g^{-1})
\label{eq:sandia1}
\end{equation}
with {\it j} changing from 1 to $m_i$,
where $m_i$ is the number of fitting intervals used for element {\it i}.\\
%The photoelectric cross-section is given in centimeters squared per gram.
\par
For the composites or mixtures of N elements the cross-section for $j^{th}$
interval is calculated as:
\begin{equation}
 \mu_{j}  =  \sum_{k=1}^{N} f_k \mu_{j,k}
\label{eq:sandia2}
\end{equation}
where $f_k$ is the fraction by mass of k$^{th}$ element in the material.
Substituting (\ref{eq:sandia1}) into (\ref{eq:sandia2}) one finds that
the cross-section coefficients can be calculated as:
\begin{equation}
 C_{i,total}  =  \sum_{k=1}^{N} f_k C_{i,jk}
\label{eq:sandia3}
\end{equation}
for {\tt i} from 1 to 4.
The macroscopic cross-section is calculated as follows:
\begin{equation}
 \Sigma  =  \rho\mu \ \ \ (cm^{-1})
\end{equation}
where $\rho$ is the medium density.
 
%Coefficients of (\ref{eq:sandia3}) are calculated in routine {\tt GPHXSI}.\\
As follows from (\ref{eq:sandia3}) each compound is decomposed to its
components which should coincide with chemical elements. If this is not
the case i.e. a component has a non-integer atomic number $Z_x$
and is created by a call to \Rind{GSMATE}
then the cross-section constants are calculated
using two elements with $Z_1 = integer(Z_x)$ and $Z_2 = Z_1+1$ applying the
weights $w_1 = Z_2-Z_x$ and $w_2 = Z_x-Z_1$ respectively.
 
\subsection{Materials with ${\tt Z} > 100$}
Originally the parametrisation described below
was developed for the elements with the atomic number {\tt Z} between 5 and 100.
Lacking more accurate data and
assuming that there are no dramatic changes of the cross-section 
behaviour this method
(i.e. \Rind{GPHSIG} function) is used for ${\tt Z} > 100$.
 
The  macroscopic cross-section  for an element is given by
\begin{equation}
\Sigma = \frac{ N \rho \sigma (Z,E_\gamma)}{ A}\ \ \ (cm^{-1})
\end{equation}
and for a compound or a mixture
\begin{equation}
\Sigma =\frac{ N \rho \sum_{i}\sigma (Z_i , E_\gamma)}
        {\sum_{i} n_i A_i }
 =    N \rho \sum_{i}\frac{p_i}{A_i}\sigma ( Z_i ,E_\gamma )\ \ \ (cm^{-1})
\end{equation}
where\\
 
\begin{tabular}[t]{l@{\hspace{1cm}}p{.7\textwidth}}
$ N $        & Avogadro's number                                \\
$ Z(Z_i)$    & atomic number (of i$^{th}$ component) of the medium  \\
$ A(A_i)$    & atomic mass (of i$^{th}$ component) of the medium  \\
$ \rho$      & density                                          \\
$ \sigma$    & cross-section                              \\
$ n_i$       & proportion by number of the i$^{th}$ element in the material
               ($ n_i = W p_i/A_i$ where $p_i$ is the
               corresponding proportion by weight and $W$ is the 
               molecular weight).
\end{tabular}\\
 
The binding energy of the inner shells has been parameterised as:
\begin{equation}
 E_i (Z) = Z^2 (A_i + B_i Z + C_i Z^2 + D_i Z^3 ) \quad  \mbox{MeV}
\end{equation}
where, $ i = K, L_I, L_{II} $ and the constants $A_i, B_i, C_i$ and $D_i$
are tabulated below.\\
\begin{center}\begin{tabular}{|l|r|r|r|r|}
\hline
& \multicolumn{1}{c|}{A}
& \multicolumn{1}{c|}{B}
& \multicolumn{1}{c|}{C}
& \multicolumn{1}{c|}{D} \\
\hline
K        & 0.66644$\cdot 10^{-5}$& 0.22077$\cdot 10^{-6}$& -0.32552$\cdot 10^{-8}$& 0.18199$\cdot 10^{-10}$\\
$L_I$    &-0.29179$\cdot 10^{-6}$& 0.87983$\cdot 10^{-7}$& -0.12589$\cdot 10^{-8}$& 0.69602$\cdot 10^{-11}$\\
$L_{II}$ &-0.68606$\cdot 10^{-6}$& 0.10078$\cdot 10^{-6}$& -0.14496$\cdot 10^{-8}$& 0.78809$\cdot 10^{-11}$\\
\hline
\end{tabular}\end{center}
The photoelectric effect total cross-section has been parameterised as:
\begin{equation}
\sigma(Z,E)=Z^{\alpha} X^{\beta}
  \left\{\begin{array}{l@{}l}
   \begin{array}{lll}
   p_1/Z & + & p_2 X+p_3+p_4 Z+p_5/X+p_6 Z^2 \\
   & +& p_7 Z/X+p_8/X^2+p_9 Z^3  \\
   & +& p_{10} Z^2/X+p_{11} Z/X^2+p_{12}/X^3
   \end{array}
                             &  \mbox{if\quad} E >E_k(Z)  \\
   p_{13}/Z+p_{14}X+p_{15} & \mbox{if\quad } E_{L_I}< E \leq E_k(Z) \\
   p_{16}/Z+p_{17}X+p_{18} & \mbox{if\quad } E_{L_{II}}<E\leq E_{L_I}\\
   p_{19} & \mbox{if\quad } 10 \; keV \leq E \leq E_{L_{II}}
  \end{array}\right.
\label{eq:urban1}
\end{equation}
where $X$ is the ratio of the electron mass to the incident photon energy
and $\sigma$ is expressed in barns/atom.\\
The fit was made over 301 data points chosen between
$5 \leq Z \leq 100\quad \mbox{and} \quad 10 \; keV \leq E \leq 50\; MeV $.
The values of the parameters are put in
a {\tt DATA} statement within the function \Rind{GPHSIG}
which computes the formula (\ref{eq:urban1}).
The accuracy of the fit is estimated to be
\begin{displaymath}
 \frac{\Delta}{\sigma} \leq 
    \left\{
        \begin{array}{ll} 25\% & \mbox{near to the peaks} \\
                          10\% & \mbox{elsewhere.}
         \end{array}
    \right.
\end{displaymath}
 
\section{Photoelectric effect Bank}
 
The banks connected to the photoelectric effect are created during
initialisation for each tracking medium.
{\tt PHXS} bank (data area)
The total cross-section values are stored in {\tt PHOT} bank in energy bins
set within the array {\tt ELOW} (common \Cind{CGMULO}). This bank is pointed by
{\tt JPHOT} link. The first structural link of {\tt PHOT} supports the
{\tt PHXS} bank.
For each energy interval of the cross-section we store the upper limit of the
interval and four constants of equation (\ref{eq:sandia3}).
For each chemical element used to build the medium a {\tt PHFN} bank is created.
This bank, pointed by a structural link of the {\tt PHXS} bank, contains
the data needed for the photoelectric effect final state simulation.
All data word are of the Fortran REAL type.
The descriptions of the {\tt PHXS} and {\tt PHFN} banks are given below.
 
\begin{tabular}{ll}
\multicolumn{2}{c}{{\tt PHXS} bank (data area)} \\
\multicolumn{1}{c}{Word \#} & \multicolumn{1}{c}{Description} \\
\ & \ \\
{\tt 1} & 
\parbox{10cm}{{\tt NZ}-number of chemical elements of the medium} \\
{\tt 2} $\rightarrow$ {\tt NZ+1} & 
\parbox{10cm}{atomic numbers of the elements} \\
{\tt NZ+2} $\rightarrow$ {\tt 2*NZ+1} & 
\parbox{10cm}{not used at present} \\
{\tt 2*NZ+2} $\rightarrow$ {\tt 3*NZ+1} & 
\parbox{10cm}{weights of the cross-section constants} \\
{\tt 3*NZ+2} & 
\parbox{10cm}{{\tt NIT}-number of the cross-section intervals} \\
{\tt 3*NZ+3} $\rightarrow$ {\tt 1+3*NZ+1+5*NIT} & 
\parbox{10cm}{the total cross-section constants}
\end{tabular}
 
\begin{tabular}{ll}
\multicolumn{2}{c}{{\tt PHFN} bank (data area)} \\
\multicolumn{1}{c}{Word \#} & \multicolumn{1}{c}{Description} \\
\ & \ \\
{\tt 1} & \parbox{10cm}{{\tt NIE}-number of intervals for an element} \\
{\tt 2} $\rightarrow$ {\tt NIE*5+1} & 
\parbox{10cm}{the element cross-section constants } \\
{\tt NIE*5+2} & 
\parbox{10cm}{number of shells used. At present always 4 } \\
{\tt NIE*5+3} $\rightarrow$ {\tt NIE*5+6} & 
\parbox{10cm}{binding energies of $K$, $L_I$, $L_{II}$ and $L_{III}$ shells } \\
{\tt NIE*5+7} $\rightarrow$ {\tt NIE*5+10} & 
\parbox{10cm}{probability of the radiative shell decay } \\
{\tt NIE*5+11} & 
\parbox{10cm}{pointer to the radiative decays of $K$ shell {\tt KRD}} \\
{\tt NIE*5+12} & 
\parbox{10cm}{pointer to the radiative decays of $L_I$ shell {\tt L1RD}} \\
{\tt NIE*5+13} & 
\parbox{10cm}{pointer to the radiative decays of $L_{II}$ shell {\tt L2RD}} \\
{\tt NIE*5+14} & 
\parbox{10cm}{pointer to the radiative decays of $L_{III}$ {\tt L3RD}} \\
{\tt NIE*5+15} & 
\parbox{10cm}{pointer to the non-radiative decays of $K$ shell {\tt KNRD}} \\
{\tt NIE*5+16} & 
\parbox{10cm}{pointer to the non-radiative decays of $L_I$ shell {\tt L1NRD}} \\
{\tt NIE*5+17} & 
\parbox{10cm}{pointer to the non-radiative decays of $L_{II}$ shell 
{\tt L2NRD}} \\
{\tt NIE*5+18} & 
\parbox{10cm}{pointer to the non-radiative decays of $L_{III}$ shell 
{\tt L3NRD}} \\
{\tt KRD} & 
\parbox{10cm}{number of $K$ shell radiative decay modes {\tt NRDK}} \\
{\tt KRD+1} $\rightarrow$ {\tt KRD+1+NRDK} & 
\parbox{10cm}{$K$ shell decay mode probability} \\
{\tt KRD+1+NRDK} $\rightarrow$ {\tt KRD+1+2*NRDK} & 
\parbox{10cm}{$K$ shell transition energies} \\
{\tt L1RD} & 
\parbox{10cm}{number of $L_I$ shell radiative 
decay modes {\tt NRDL1}} \\
{\tt L1RD+1} $\rightarrow$ {\tt L1RD+1+NRDL1} & 
\parbox{10cm}{$L_I$ shell decay mode probability} \\
{\tt L1RD+1+NRDL1} $\rightarrow$ {\tt L1RD+1+2*NRDL1} & 
\parbox{10cm}{$L_I$ shell transition energies} \\
{\tt L2RD} & 
\parbox{10cm}{number of $L_I$ shell radiative decay modes {\tt NRDL2}} \\
{\tt L2RD+1} $\rightarrow$ {\tt L2RD+1+NRDL2} & 
\parbox{10cm}{$L_{II}$ shell decay mode probability} \\
{\tt L2RD+1+NRDL2} $\rightarrow$ {\tt L2RD+1+2*NRDL2} & 
\parbox{10cm}{$L_{II}$ shell transition energies} \\
{\tt L3RD} & 
\parbox{10cm}{number of $L_I$ shell radiative decay modes {\tt NRDL3}} \\
{\tt L3RD+1} $\rightarrow$ {\tt L3RD+1+NRDK} & 
\parbox{10cm}{$L_{III}$ shell decay mode probability} \\
{\tt L3RD+1+NRDL3} $\rightarrow$ {\tt L3RD+1+2*NRDL3} & 
\parbox{10cm}{$L_{III}$ shell transition energies} \\
{\tt KNRD} & 
\parbox{10cm}{number of $K$ shell radiative decay modes {\tt RDK = 1}} \\
{\tt KNRD+1} $\rightarrow$ {\tt KNRD+1+RDK} & 
\parbox{10cm}{$K$ shell decay mode probability} \\
{\tt KNRD+1+RDK} $\rightarrow$ {\tt KNRD+1+2*RDK} & 
\parbox{10cm}{$K$ shell transition energies} \\
{\tt L1NRD} & 
\parbox{10cm}{number of $L_I$ shell radiative decay modes {\tt RDL1 = 1}} \\
{\tt L1NRD+1} $\rightarrow$ {\tt L1RD+1+RDL1} & 
\parbox{10cm}{$L_I$ shell decay mode probability} \\
{\tt L1NRD+1+RDL1} $\rightarrow$ {\tt L1RD+1+2*RDL1} & 
\parbox{10cm}{$L_I$ shell transition energies} \\
{\tt L2NRD} & 
\parbox{10cm}{number of $L_I$ shell radiative decay modes {\tt RDL2 = 1}} \\
{\tt L2NRD+1} $\rightarrow$ {\tt L2RD+1+NRDK} & 
\parbox{10cm}{$L_{II}$ shell decay mode probability}
\end{tabular} 

\begin{tabular}{ll}
\multicolumn{2}{c}{{\tt PHFN} bank (data area, continued)} \\
\multicolumn{1}{c}{Word \#} & \multicolumn{1}{c}{Description} \\
\ & \ \\
{\tt L2NRD+1+RDL2} $\rightarrow$ {\tt L2RD+1+2*RDL2} & 
\parbox{10cm}{$L_{II}$ shell transition energies} \\
{\tt L3NRD} & 
\parbox{10cm}{number of $L_I$ shell radiative decay modes {\tt RDL3 = 1}} \\
{\tt L3NRD+1} $\rightarrow$ {\tt L3RD+1+RDL3} & 
\parbox{10cm}{$L_{III}$ shell decay mode probability} \\
{\tt L3NRD+1+RDL3} $\rightarrow$ {\tt L3RD+1+2*NRDL3} & 
\parbox{10cm}{$L_{III}$ shell transition energies}
\end{tabular}
