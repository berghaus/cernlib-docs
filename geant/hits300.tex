%%%%%%%%%%%%%%%%%%%%%%%%%%%%%%%%%%%%%%%%%%%%%%%%%%%%%%%%%%%%%%%%%%%
%                                                                 %
%  GEANT manual in LaTeX form                                     %
%                                                                 %
%  Version 1.00                                                   %
%                                                                 %
%  Last Mod.  9 Jun 1993  1300   MG                               %
%                                                                 %
%%%%%%%%%%%%%%%%%%%%%%%%%%%%%%%%%%%%%%%%%%%%%%%%%%%%%%%%%%%%%%%%%%%
\Origin{R.Brun, W.Gebel}
\Submitted{10.08.84}       \Revised{18.12.93}
\Version{Geant 3.16}       \Routid{HITS300}

\Makehead{Routines to store and retrieve DIGItisations}

\Shubr{GSDIGI}{(ISET,IDET,LTRA,NTRA,NUMBV,KDIGI,IDIG*)}
\begin{DLtt}{MMMMMMMM}
\item[ISET] ({\tt INTEGER}) set number;
\item[IDET] ({\tt INTEGER}) detector number;
\item[LTRA] ({\tt INTEGER}) array of {\tt NTRA} track numbers producing 
this digitisation;
\item[NUMBV] ({\tt INTEGER}) volume numbers corresponding to list {\tt NAMESV} 
of \Rind{GSDET};
\item[KDIGI] ({\tt INTEGER}) array of current digisation elements;
\item[IDIG] ({\tt INTEGER}) stored digitisation number, if 0, digitisation 
has not been stored.
\end{DLtt}
Stores element values for current digitisation
into the data structure {\tt JDIGI}.

\Shubr{GPDIGI}{(CHSET,CHDET)}
\begin{DLtt}{MMMMMMMM}
\item[CHSET] ({\tt CHARACTER*4}) set name,
if {\tt '*'} prints all {\tt JDIGI} banks of all sets;
\item[CHDET] ({\tt CHARACTER*4}) detector name, if {\tt '*'} prints 
digitisations in all detectors of set;
{\tt CHSET}
\end{DLtt}
Prints {\tt JDIGI} banks for detector {\tt CHDET} of set {\tt CHSET}.
 
\Shubr{GFDIGI}{(CHSET,CHDET,NTDIM,NVDIM,NDDIM,NDMAX,NUMVS,
                LTRA*,NTRA*,NUMBV*,KDIGI*,NDIGI*)}
\begin{DLtt}{MMMMMMMM}
\item[CHSET] ({\tt INTEGER}) set name;
\item[CHDET] ({\tt INTEGER}) detector name;
\item[NTDIM] ({\tt INTEGER}) 1$^st$ dimension of {\tt LTRA}, maxmum number of 
tracks contributing to each hit to be returned;
\item[NVDIM] ({\tt INTEGER}) 1$^{st}$ dimension of arrays {\tt NUMVS}, 
{\tt NUMBV}, same as argument {\tt NV} of \Rind{GSDET};
\item[NDDIM] ({\tt INTEGER}) 1$^{st}$ dimension of {\tt KDIGI}, same
as argument {\tt ND} of \Rind{GSDETD};
\item[NDMAX] ({\tt INTEGER}) maximum number of digitisations to be returned,
second dimension of arrays {\tt NTDIM}, {\tt NUMBV} and {\tt KDIGI};
\item[NUMVS] ({\tt INTEGER}) is a 1-dim array of length {\tt NVDIM}
that contains the copy numbers identifying the detector to be selected, all 
0 is interpreted as all copies of detector {\tt CHDET};
\item[LTRA] ({\tt INTEGER}) is a 2-dim array {\tt NTDIM,NDMAX} that contains
for each digitisation the numbers of the tracks which have produced it;
\item[NTRA] ({\tt INTEGER}) is a 1-dim array of length {\tt NDMAX} that contains,
for each digitisation, the number of tracks contributing,
in case this number is greater than {\tt NTDIM}, only the first
{\tt NTDIM} corresponding tracks are returned on {\tt LTRA};
\item[NUMBV] ({\tt INTEGER}) is a 2-dim array {\tt NVDIM,NDMAX} that contains,
for each digitisation, the
list of volume numbers which identify each detector;
\item[KDIGI] ({\tt INTEGER}) is a 2-Dim array {\tt NDDIM,NDMAX} that contains 
the {\tt NDIGI} digitisations returned;
\item[NDIGI] ({\tt INTEGER}) is the total number of digitisations in this 
detector,
in case the total number of digitisations is greater than {\tt NDMAX},
{\tt NDIGI} is set to {\tt NDMAX+1} and only {\tt NDMAX} digitisations are
returned.
\end{DLtt}
Returns the digitisations for the detector {\tt CHDET} identified by the list 
of copy numbers {\tt NUMVS} belonging to set {\tt CHSET}. The maning of
the variables is the following:

\begin{tabular}{lp{10cm}}
\tt KDIGI(1,I) & digitisation element 1 for digitisation number {\tt I} \\
\tt NUMBV(1,I) & first volume number for digitisation number {\tt I} \\
\tt LTRA (1,I) & number of the first track contributing to digitisation 
number {\tt I} \\
\end{tabular}

In the calling routine, the arrays {\tt LTRA, NTRA, NUMVS, NUMBV,
KDIGI} must be dimensioned to:
\begin{verbatim}
      LTRA (NTDIM,NDMAX)
      NTRA (NDMAX)
      NUMVS(NVDIM)
      NUMBV(NVDIM,NDMAX)
      KDIGI(NDDIM,NDMAX)
\end{verbatim}
 
