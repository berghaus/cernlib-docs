%%%%%%%%%%%%%%%%%%%%%%%%%%%%%%%%%%%%%%%%%%%%%%%%%%%%%%%%%%%%%%%%%%%
%                                                                 %
%  GEANT manual in LaTeX form                              %
%                                                                 %
%  Michel Goossens (for translation into LaTeX)                   %
%  Version 1.00                                                   %
%  Last Mod. Jan 24 1991  1300   MG + IB                          %
%                                                                 %
%%%%%%%%%%%%%%%%%%%%%%%%%%%%%%%%%%%%%%%%%%%%%%%%%%%%%%%%%%%%%%%%%%%
\Origin{L.Urb\'{a}n}
\Version{Geant 3.16}\Routid{PHYS440}
\Submitted{26.10.84}  \Revised{16.12.93}
\Makehead{Total cross-section and energy loss for bremsstrahlung by
Muons}
\section{Subroutines}
\Shubr{GBRELA}{}
{\tt GBRELA} fills the tables for continuous energy loss of electrons,
positrons and muons due to bremsstrahlung during initialisation time.
The energy binning is determined by
the array {\tt ELOW} (common \FCind{/CGMULO/}) in the
routine \Rind{GPHYSI}. In the tables, the $dE/dx$ due to bremsstrahlung
is summed with that due to the ionisation. For energy loss
of muons, \Rind{GBRELA} calls the function \Rind{GDRELM}.
The following pointers are used:
\begin{DLtt}{MMMMMMMMMMMMMMMMMM}
\item[JMA = LQ(JMATE-I)] pointer to the I$^{th}$ material;
\item[JEL1 = LQ(JMA-1)]  pointer to dE/dx for electrons;
\item[JEL1+NEK1]           pointer to dE/dx for positrons;
\item[JEL2 = LQ(JMA-2)]  pointer to dE/dx for electrons.
\end{DLtt}
\Rind{GBRELA} is called by \Rind{GPHYSI}.
\Sfunc{GBRELM}{VALUE = GBRELM(Z,T,BCUT)}
\begin{DLtt}{MMMMMMMM}
\item[Z] ({\tt REAL}) atomic number of the material;
\item[T] ({\tt REAL}) kinetic energy of the muon;
\item[BCUT] ({\tt REAL}) {\it soft} bremsstrahlung cut.
\end{DLtt}
\Rind{GBRELM} calculates the muon energy loss due to bremsstrahlung
of photons with energy below {\tt BCUT}, (variable {\tt BCUTM} in the
common \FCind{/GCCUTS/}).
Above this value, the bremsstrahlung process is simulated
explicitly (see {\tt [PHYS441]}) and the energy lost by the muons is
not included in the tables. 
In the tables, the $dE/dx$ due to bremsstrahlung is summed with
the energy lost coming from ionisation, pair production and nuclear interaction.
\Rind{GBRELM} is called by \Rind{GBRELA}.
\Shubr{GBRSGA}{}
{\tt  GBRSGA} calculates the total cross-section for bremsstrahlung.
It tabulates the mean free path
in cm as a function of the medium and the energy. The energy
binning is determined by the array {\tt ELOW} (common \FCind{/CGMULO/}).
The following pointers are used:
\begin{DLtt}{MMMMMMMMMMMMMMMMMM}
\item[JMA = LQ(JMATE-I)]pointer to the I$^{th}$ material;
\item[JBREM = LQ(JMA-9)]pointer to bremsstrahlung cross-sections;
\item[JBREM            ]pointer for \Pem;
\item[JBREM+NEK1       ]pointer for \Pep;
\item[JBREM+2*NEK1     ]pointer for $\mu^+/\mu^-$.
\end{DLtt}
\Rind{GBRSGA} is called during initialisation by \Rind{GPHYSI}.
\Sfunc{GBRSGM}{VALUE = GBRSGM(Z,T,BCUT)}
\begin{DLtt}{MMMMMMMM}
\item[Z] ({\tt REAL}) atomic number of the materian;
\item[T] ({\tt REAL}) kinetic energy of the muon;
\item[BCUT] ({\tt REAL}) {\it soft} bremsstrahlung cut.
\end{DLtt}
\Rind{GBRSGM} calculates the total bremsstrahlung cross-section for
muons when the emitted photon has an energy greater than {\tt BCUT}, 
(variable {\tt BCUTM} in the common \FCind{/GCCUTS/}). It is called
by \Rind{GBRSGA}.
\section{Method}
 
The mean value of the energy lost by the muon due to bremsstrahlung of
photons of energy $<k_{c}$ = {\tt BCUTM} is:
\begin{equation}
\label{eq:phys440-1}
E_{loss}^{brem}(Z,T,k_c ) = \int_{0}^{k_c} k
        \frac{d \sigma (Z,T,k)}{dk}dk
\end{equation}
and the total cross-section for the emission of a photon of energy
$ > k_c $ is
\begin {equation}
\label{eq:phys440-2}
 \sigma (Z,T,k_c ) = \int_{k_c}^{T}
        \frac{d \sigma (Z,T,k)} {dk} dk
\end{equation}
Accurate cross-section formula for the high energy ($T \geq 1$ GeV)
muon bremsstrahlung can be found in \cite{bib-LOHM}.

\subsection{Parameterisation of energy loss and total cross-section}
The cross-sections from \cite{bib-LOHM} have been used to calculate 
{\it data} points for (\ref{eq:phys440-1}) and (\ref{eq:phys440-2}).
These have been parameterised as:
\[ \begin{array}{LcLl}
 \sigma (Z,T,k_c)& =& Z[Z+ \xi_{\sigma}(1+ \gamma \ln Z)]
       \left[\ln \frac{k_{max}}{k_c}\right]^{\alpha}
       F_{\sigma}(Z,X,Y) & \mbox{ in barn} \\
  E_{loss}^{brem} (Z,T,k_c)   & = & Z[Z+\xi_l(1+\delta\ln Z)]
   \left[\ln\frac {k_c} {E} \right]^{\beta}
    F_l(Z,X,Y) & \mbox{ in GeV barn}
\end{array} \]
\[ \begin{array}{LcL@{\hspace{2cm}}LcL}
k_{max} & = & E-0.75 \sqrt {e}m_{\mu}Z^{1/3} & X & = & \ln \frac{E}{m_{\mu}} \\
Y & = & \ln \frac{k_c}{m_{\mu}} & E & = & T+m_{\mu}
\end{array} \]
where  $k_{max}$ is the maximum
possible value of the photon energy.
The functions $F_i(Z,X,Y)$ ($i=\sigma ,l$) are polynomials:
\begin{eqnarray}
\label{eq:phys440-3}
F_i(Z,X,Y) & = & (C_1+C_2X+\cdots +C_6X^5)+(C_7+C_8X+\cdots+C_{12}X^5)Y
\nonumber \\
        & + & \cdots + (C_{31}+C_{32}X+\cdots+C_{36}X^5)Y^5
\nonumber \\
           & + & Z[(C_{37}+C_{38}X+\cdots+C_{40}X^3)+(C_{41}+C_{42}X+
                   \cdots+C_{44}X^3)Y \nonumber \\
           & + & \cdots + (C_{48}+C_{49}X+\cdots+C_{52}X^3)Y^3)]
\end{eqnarray}
A least-squares fit has been performed on more than 2000 points for
$Z = 1, 6, 13, 26, 50, 82, 92$ and 1 GeV $\leq T \leq $ 10 TeV and
10 keV $\leq k_c \leq T $. The resulting values of 
$\xi_{\sigma,l}$, $\gamma$, $\alpha$ ,$C_i$, $\xi_l$, $\delta$ and $\beta$
can be found in the {\tt DATA} statements within the functions
\Rind{GBRSGM} and \Rind{GBRELM} which
compute formulae (\ref{eq:phys440-1}) and (\ref{eq:phys440-2}) respectively.
 
The accuracy of the fit can be estimated as:\\
 
\[ \begin{array}{LcL}
\frac{\Delta \sigma}{\sigma} & = & \left \{
\begin{array}{p{2cm}lL}
$\displaystyle 10-12\%$ & \mbox{if} & T \leq 5 \: \mbox{GeV} \\
$\displaystyle \leq 4\%$ & \mbox{if} & T > 5 \: \mbox{GeV}
\end{array} \right . \\ [0.5cm]
\frac{\Delta E_{loss}^{brem}}{E_{loss}^{brem}} & = & \left \{
\begin{array}{p{2cm}lL}
$\displaystyle 10\%$ & \mbox{if} & T \leq 5 \: \mbox{GeV} \\
$\displaystyle \leq 3\%$ & \mbox{if} & T > 5 \: \mbox{GeV}
\end{array} \right .
\end{array} \]
 
The contribution of the bremsstrahlung to the total energy loss of
the muons is less than 1\% for $ T \leq$ 5 GeV.
 
When $k_{c} \geq k_{max}$, a parameterisation different from 
(\ref{eq:phys440-3}) can be used for the total muon energy loss due to 
bremsstrahlung:
\begin{eqnarray} 
\label{eq:phys440-4}
E_{loss}^{brem} (Z,T) & = & E_{loss}^{brem}(Z,T,k=k_{max}) 
\nonumber\\
& = & Z(Z+1) k_{max}[d_1+(d_2X+d_3Y) + (d_4X^2+d_5 XY+d_6 Y^2) \nonumber \\
& + & \cdots+(d_{22}X^6+d_{23}X^5 Y+\cdots+d_{28}Y^6)]
\end{eqnarray}
where $Y=Z^{1/3}$. 
The accuracy of the formula (\ref{eq:phys440-4}) for $1 \leq Z \leq 100$
is rather good:
\[
\frac{\Delta E_{loss}^{brem}}{E_{loss}^{brem}} = \left \{
\begin{array}{LlL}
\leq 1.5\% & \mbox{if} & T > 1  \; \mbox{GeV} \\
\leq 1\%  & \mbox{if} & T > 5 \; \mbox{GeV}
\end{array} \right .
\]
