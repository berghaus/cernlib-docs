\chapter{Use of the NAG Fortran Library at CERN}
 
\section{Access to NAG Fortran Library programs at CERN}
 
 Access to the NAG Fortran Library is granted on all systems via the
 CERNLIB command:
 
\begin{verbatim}
                     CERNLIB NAGLIB
\end{verbatim}
 
 Then the standard loading procedures must be followed to generate
 an executable:
 
 
\begin{center}
\begin{tabular}{|*{2}{l|}}
\hline
\multicolumn{1}{|c}{\bf SYSTEM}  &
\multicolumn{1}{|c|}{\bf COMMAND} \\
\hline
   \bf    CERNVM        &\Ucom{CERNLIB NAGLIB} \\
                        &\Ucom{ LOAD myobj (NOAUTO} \\[1mm]
\hline
   \bf    VXCERN        &\Ucom{CERNLIB NAGLIB} \\
                        &\Ucom{LINK myobj,'LIB\$'} \\[1mm]
\hline
   \bf UNIX - Bourne/Korn shell &\Ucom{CERNLIB=`cernlib naglib`} \\
                        &\Ucom{load\_command myobj.o \$CERNLIB}\\[1mm]
\hline
   \bf UNIX - C shell &\Ucom{set CERNLIB=`cernlib naglib`} \\
                        &\Ucom{load\_command myobj.o \$CERNLIB}\\[1mm]
\hline
\end{tabular}\\
\end{center}
 
Remember that all floating point arguments to the NAG Library subroutines
must be of type DOUBLE PRECISION. Not using double precision  may lead to
access or memory violation errors which abort your program,
or just give wrong results.
 
\section{Overview of the Fortran Library}
 
Below  we give  a short  overview  of the  chapters included  in the  NAG
Fortran Library.  For  the complete list of all the  routines included in
each chapter,  please refer to  the NAG  reference manuals or  the online
summaries.
 
 
\begin{verbatim}
    Contents of NAG Fortran Library
 
    A02 = Complex Arithmetic
    C02 = Zeros of Polynomials
    C05 = Roots of One or More Transcendental Equations
    C06 = Summation of Series
    D01 = Quadrature
    D02 = Ordinary Differential Equations
     D02M - D02N = Integrators  for Stiff Ordinary Differential Equations
    D03 = Partial Differential Equations
    D04 = Numerical Differentiation
    D05 = Integral Equations
    E01 = Interpolation
    E02 = Curve and Surface Fitting
    E04 = Minimizing or Maximizing a Function
    F   = Linear Algebra
     F01 = Matrix Operations, Including Inversion
     F02 = Eigenvalues and Eigenvectors
     F03 = Determinants
     F04 = Simultaneous Linear Equations
     F05 = Orthogonalisation
     F06 = Linear Algebra Support Routines
     F07 = Linear Equations (LAPACK)
    G01 = Simple Calculations on Statistical Data
    G02 = Correlation and Regression Analysis
    G03 = Multivariate Methods
    G04 = Analysis of Variance
    G05 = Random Number Generators
    G07 = Univariate Estimation
    G08 = Nonparametric Statistics
    G12 = Survival Analysis
    G13 = Time Series Analysis
    H   = Operations Research
    M01 = Sorting
    P01 = Error Trapping
    S   = Approximations of Special Functions
    X01 = Mathematical Constants
    X02 = Machine Constants
    X03 = Innerproducts
    X04 = Input/Output Utilities
    X05 = Date and Time Utilities
\end{verbatim}
 
The routines  of each  of the  above mentioned  chapters have  names like
CCCxyz, where the first three characters are the chapter name and the
last three specify the given routine.
 
The F07 chapter consists of the  LAPACK package, which was introduced at
Mark 15 of the NAG Fortran Library. The chapter contains
subroutines written in Fortran for solving
the most common problems in numerical linear algebra:
systems of linear equations, linear least squares problems,
eigenvalue problems, and singular value problems.
The algorithms and software are structured to achieve high
efficiency on vector processors, high-performance "superscalar"
workstations, and shared-memory multi-processors.
While the LAPACK project has been concerned with high performance
computers, the routines do not compromise efficiency on conventional
machines.
 
NAG  recommends  that the  users  read  the following  minimum  reference
material before calling any library routine:
 
\begin{verbatim}
  (a) Essential Introduction
  (b) Chapter Introduction
  (c) Routine Introduction
  (d) Implementation-specific Users' Note
\end{verbatim}
 
All the  items are included in the NAG Fortran
Library Manuals. Besides, online access to (a) and (d) is provided as described in the
introductory section of this manual.
 
\newpage
 
\section{Access to the Fortran Library examples:
                 the NAGTEST utility.}
 
NAGTEST is an application developed by CERN to manipulate
the example programs offered by NAG.  NAGTEST
 
\begin{itemize}
\item extracts the source codes of the chosen example program;
\item extracts the corresponding input data, when necessary;
\item compiles and runs the example;
\item optionally extracts the program results as supplied by NAG,
      and compares them with the ones just obtained;
\item makes all the relevant files accessible to the user, facili-
      tating adaptations of the examples to specific problems.
\end{itemize}
 
        Online information about the NAGTEST utility can be obtained
via the local HELP facility. The syntax is described in the following
table.
 
\begin{center}
\begin{tabular}{|*{2}{l|}}
\hline
\multicolumn{1}{|c}{\bf SYSTEM}  &
\multicolumn{1}{|c|}{\bf COMMAND} \\
\hline
   \bf    CERNVM        &\Ucom{NAGTEST CCCxyz [ (COMPARE ]}  \\[1mm]
\hline
    \bf   VXCERN        &\Ucom{NAGTEST[/COMPARE] CCCxyz }  \\[1mm]
\hline
    \bf   UNIX          &\Ucom{nagtest [-compare] cccxyz}           \\[1mm]
\hline
\end{tabular}\\
\end{center}
 
where CCCxyz is the name of the routine for which an example is
wanted, following the NAG names scheme explained in section 1.2.
 
 
