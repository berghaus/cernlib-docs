\chapter{Use of the NAG Graphics Library at CERN}
 
\section{Access to NAG Graphics Library programs at CERN}
 
 
This implementation of the NAG  Graphics Library uses the Graphics Kernel
System (GKS) available  at CERN.  The high-level routines as  well as the
utility and interface routines are in chapter J in the NAG classification
scheme.
 
 
The NAG programs themselves are accessed as follows:
 
To make the library (and the  required GKS library) available for loading
type:
 
\begin{center}
\begin{tabular}{|*{2}{l|}}
\hline
\multicolumn{1}{|c}{\bf SYSTEM} &
\multicolumn{1}{|c|}{\bf COMMAND} \\
\hline
     \bf  CERNVM    & \Ucom{CERNLIB NAGLIB(GKS}  \\
\hline
      \bf VXCERN    & \Ucom{CERNLIB NAGLIB/GKS}  \\[1mm]
\hline
\end{tabular}\\
\end{center}
 
Remember
that all floating point arguments to  the NAG Library subroutines must be
of type  DOUBLE PRECISION.  The loading  procedure is similar to  the one
described in the previous chapter, ie. with the commands:
 
\begin{center}
\begin{tabular}{|*{2}{l|}}
\hline
\multicolumn{1}{|c}{\bf CERNVM} &
\multicolumn{1}{|c|}{\bf VXCERN} \\
\hline
\Ucom{CERNLIB NAGLIB/GKS} & \Ucom{CERNLIB NAGLIB(GKS} \\
\Ucom{LOAD myobj (NOAUTO} & \Ucom{LINK myobj,'LIB\$'}  \\[1mm]
\hline
\end{tabular}\\
\end{center}
 
On UNIX systems the Graphics Library may be installed separately
should this be necessary. A GKS library or equivalent would be necessary.
 
 
\section{Getting started with NAG graphics}
 
This section should be read in conjunction with the "GENERAL INTRODUCTION
- Using  the NAG  Graphics Library"  available at  the Computer Science
Library (Bld. 513) for reference.
 
An overview of  the routines included in the NAG  Graphics Library can be
found in Appendix C of this manual.
 
There are 74 examples on the  NAGLIB directories, which show the
use of the  high level NAG graphical subroutines with  names like J06xyf.
The procedure  to run these examples  is explained in section 2.3
and  is very similar to the one followed with the Fortran Library.
 
\section{Access to the Graphics Library examples:
                 the NAGTEST utility}
 
The NAGTEST utility provides in this case the following functionality:
 
\begin{itemize}
\item It extracts the source codes of the chosen example program;
\item It extracts the data for an example program, when necessary;
\item It compiles and runs the example;
\item It gives the user the opportunity of previewing the graphical
      results of the example on the terminal.
\item It creates a GKS metafile and an EPS output of the corresponding
      picture(s) which may be further manipulated with convenient
      utilities.
\item It makes all the relevant files accessible to the users, facili-
      tating adaptations of the examples to specific problems.
\end{itemize}
 
        Online information about the NAGTEST utility can be obtained
via the local HELP facility. The syntax is described in the table below.
 
\begin{center}
\begin{tabular}{|*{2}{l|}}
\hline
\multicolumn{1}{|c}{\bf SYSTEM} &
\multicolumn{1}{|c|}{\bf COMMAND} \\
\hline
      \bf CERNVM        & \Ucom{NAGTEST J06xyz}     \\[1mm]
\hline
      \bf VXCERN        & \Ucom{NAGTEST J06xyz}     \\[1mm]
\hline
\end{tabular}\\
\end{center}
 
where J06xyz is the name of the routine for which an example is
whished, following the NAG names scheme explained before, with all
routines of the Graphics Library beginning with J06.
 
For most of the high level graphical subroutines there is an example with
its corresponding data file on the NAGLIB directories.
 
Until  you get  more experience,  use the  following model,  substituting
different NAG routines to work on your own data as described below:
 
\begin{verbatim}
 NAGTEST J06xyz
\end{verbatim}
 
The user is prompted for the GKS workstation type.  GKS workstation types
for  the terminals  at  CERN are  to  be found  using  "XFIND GKS".
 
For example, to look at the demonstration of the routine J06HGF, type:
 
                     \Ucom{NAGTEST J06HGF}
 
If the  correct terminal type  is given, you will  get the graphics
output  of the
example program for  routine J06HGF  at  the  screen.  Besides,  the
following Program Text and Program Data are automatically accessed:
 
 
\begin{verbatim}
*AJ06HGF
C     J06HGF EXAMPLE PROGRAM TEXT.
C     MARK 3 RE-ISSUE. NAG COPYRIGHT 1989.
C     .. Parameters ..
      INTEGER          NIN, NOUT
      PARAMETER        (NIN=5,NOUT=6)
      INTEGER          MDIM, NMAX, NRWS
      PARAMETER        (MDIM=7,NMAX=15,NRWS=1500)
C     .. Local Scalars ..
      DOUBLE PRECISION PHI, R, THETA, UMAX, UMIN, VMAX, VMIN, ZMAX, ZMIN
      INTEGER          I, IFAIL, J, K, LENGTH, M, N
C     .. Local Arrays ..
      DOUBLE PRECISION HTS(MDIM,NMAX), RWS(NRWS)
      CHARACTER*4      LABELX(MDIM), LABELY(NMAX)
      CHARACTER*17     LABX(MDIM), LABY(NMAX)
C     .. External Subroutines ..
      EXTERNAL         J06AHF, J06HGF, J06VAF, J06WAF, J06WCF, J06WZF,
     *                 XXXXXX
C     .. Executable Statements ..
C
C     Select output channels for error messages
C
      CALL J06VAF(1,NOUT)
C
C     Initialise plotting device
C
      CALL XXXXXX
C
C     Read viewport co-ordinates
C
      READ (NIN,FMT=*)
      READ (NIN,FMT=*) UMIN, UMAX, VMIN, VMAX
C
C     Call NAG Graphical Interface routines to initialise
C     the NAG Graphics and set the viewport.
C     The data region is set by J06HGF
C
      CALL J06WAF
      CALL J06WCF(UMIN,UMAX,VMIN,VMAX)
C
C     Read in M and N, the number of data points
C     on the X and Y axes respectively
C
      READ (NIN,FMT=*) M, N
      IF (M.LE.0 .OR. M.GT.MDIM) THEN
         WRITE (NOUT,FMT=99999)
      ELSE IF (N.LE.0 .OR. N.GT.NMAX) THEN
         WRITE (NOUT,FMT=99998)
      ELSE
C
C        Read in array of heights
C
         READ (NIN,FMT=*) ((HTS(I,J),J=1,N),I=1,M)
C
C        Read in the Z data limits
C
         READ (NIN,FMT=*) ZMIN, ZMAX
C
C        Read in the axes annotation
C
         READ (NIN,FMT=*) LENGTH
         IF (LENGTH.GT.0) THEN
            READ (NIN,FMT=*) (LABELX(K),K=1,M)
            READ (NIN,FMT=*) (LABELY(K),K=1,N)
         ELSE
            READ (NIN,FMT=*) LABX(1)
            READ (NIN,FMT=*) LABY(1)
         END IF
C
C        Read in viewpoint definition
C
         READ (NIN,FMT=*) R, THETA, PHI
C
         IFAIL = 0
         IF (LENGTH.GT.0) THEN
            CALL J06HGF(HTS,MDIM,M,N,R,THETA,PHI,RWS,NRWS,ZMIN,ZMAX,
     *                  LABELX,LABELY,LENGTH,IFAIL)
         ELSE
            CALL J06HGF(HTS,MDIM,M,N,R,THETA,PHI,RWS,NRWS,ZMIN,ZMAX,
     *                  LABX,LABY,LENGTH,IFAIL)
         END IF
C
C        Draw title
C
         CALL J06AHF('J06HGF EXAMPLE PLOT')
      END IF
C
C     Terminate plotting
C
      CALL GPRMPT(1,'READY?',LSTRI,REPLY)
      CALL GCSTOP
      STOP
C
99999 FORMAT (' M is out of range')
99998 FORMAT (' N is out of range')
      END
\end{verbatim}
 
\newpage
 
\begin{verbatim}
J06HGF EXAMPLE PROGRAM DATA
0.0  1.0  0.0  1.0         : VIEWPORT CO-ORDINATES
7    12                    : M, N
1.20 1.40 2.77 0.32 9.98 0.01 2.00 4.12 6.48 8.78 6.44
3.31 1.00 0.01 1.44 1.12 3.45 6.98 1.00 0.10 0.99 1.02
1.01 1.14 1.55 0.11 4.56 0.32 3.00 3.00 1.55 3.33 4.00
7.12 1.55 1.32 1.14 2.71 1.98 3.44 2.54 2.77 0.01 2.50
0.99 5.89 1.23 0.01 1.00 0.05 0.01 0.03 0.50 0.77 0.99
1.45 6.55 9.99 6.50 4.00 0.50 0.00 0.22 0.00 0.10 0.30
0.77 1.50 3.49 3.28 4.99 2.01 0.40 0.10 0.05 0.00 0.00
0.05 0.30 0.99 2.00 3.00 2.00 0.87  : HTS
0.0   0.0                  : ZMIN, ZMAX
0                          : LENGTH
' POLLUTION LEVEL '        : LABX
' NUMBER OF TREES '        : LABY
20.0  20.0  300.0          : R, THETA, PHI
\end{verbatim}
 
 
This and other examples are provided by NAG to demonstrate  the  features
of their Graphics Library, and were tailored for the CERN environment. In
particular, by  means of  the  calls to  routines \Lit{GPRMPT} and
\Lit{GCSTOP},
they terminate more properly than the  original versions.
Another common feature of the NAG example programs  is the initialisation
of plotting device by means of the  routine named \Lit{XXXXXX} (yes: it
is really called this!) by NAG.
The examples call the  routine \Lit{XXXXXX} to initialise the GKS system,
and a version  of this  is provided to  enable the NAG  examples to  run.
We keep the name \Lit{XXXXXX}, but we have
modified  this routine in order  to make a  more proper
initialisation. \Lit{XXXXXX} will direct all input/output to
GKS workstation 1,
which is the terminal type provided by the user. Simutaneously (or if
carriage return is issued instead of providing
the workstation type) all graphics output is written to
a metafile (GKS workstation 2) with logical unit 25 on disk (with the
names \Lit{J06xyf  METAFILE} on VM and \Lit{J06xyf.MET} on VAX). This
metafile can
than be played back using GRVIEW or plotted using GRPLOT.
In \Lit{XXXXXX} we chose the logical unit for the GKS error file
to be the
same as for the NAG error log, namely unit 09. In this way, any GKS
errors will be output to the same file as the NAG error log.
 
The result of this example, as well as the results of all the 74 examples
provided with NAGTEST as run on CERNVM are shown in Appendix D.
 
