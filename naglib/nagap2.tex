\chapter{Essential introduction to the NAG Fortran library}
\begin{center}
   (verbatim copy taken from NAG Introductory Guide - Mark 15)
\end{center}
\begin{verbatim}
ESSENTIAL INTRODUCTION TO THE NAG FORTRAN LIBRARY
 
This document is essential reading for any prospective user of the
Library.
 
Contents:
 
  1. The Library and its Documentation
     1.1. Structure of the Library
     1.2. Structure of the Manual
     1.3. Supplementary Documentation
     1.4. Marks of the Library
     1.5. Implementations of the Library
     1.6. Precision of the Library
     1.7. Library Identification
     1.8. Fortran Language Standards
 
  2. Using the Library
     2.1. General Advice
     2.2. Programming Advice
     2.3. Error-handling and the Parameter IFAIL
     2.4. Input/output in the Library
     2.5. Auxiliary Routines
 
  3. Using the Documentation
     3.1. Using the Manual
     3.2. Structure of Routine Documents
     3.3. Specifications of Parameters
          3.3.1. Classification of parameters
          3.3.2. Constraints and suggested values
          3.3.3. Array parameters
     3.4. Implementation-dependent Information
     3.5. Example Programs and Results
     3.6. Summary for New Users
     3.7. Supplementary Documentation
     3.8. Pre-Mark 14 Routine Documents
 
  4. Contact between Users and NAG
 
  5. Further Information
 
  6. References
 
 
1. The Library and its Documentation
 
1.1. Structure of the Library
 
  The NAG Fortran Library is a comprehensive collection of Fortran 77
  routines for the solution of numerical and statistical problems. The
  word `routine' is used to denote `subroutine' or `function'.
 
  The Library is divided into chapters, each devoted to a branch of
  numerical analysis or statistics. Each chapter has a three-character
  name and a title, e.g.
 
    D01 - Quadrature
 
  Exceptionally two chapters (H and S) have one-character names. (The
  chapters and their names are based on the ACM modified SHARE
  classification index [1].)
 
  All documented routines in the Library have six-character names,
  beginning with the characters of the chapter name, e.g. D01AJF Note
  that the second and third characters are digits, not letters; e.g. 0
  is the digit zero, not the letter O. The last letter of each routine
  name always appears as `F' in the documentation, but may be changed to
  `E' in some single-precision implementations (see Section 1.6).
 
1.2. Structure of the Manual
 
  The NAG Fortran Library Manual is the principal documentation for the
  NAG Fortran Library. It has the same chapter structure as the Library:
  each chapter of routines in the Library has a corresponding chapter
  (of the same name) in the Manual. The chapters occur in alphanumeric
  order. General introductory documents and indexes are placed at the
  beginning of the Manual.
 
  Each chapter consists of the following documents:
 
    Chapter Introduction, e.g. Introduction - D01;
    Chapter Contents, e.g. Contents - D01;
 
  routine documents, one for each documented routine in the chapter. A
  routine document has the same name as the routine which it describes.
  Within each chapter, routine documents occur in alphanumeric order.
  Exceptionally, some chapters (F06, X01, X02), which contain simple
  support routines, do not have individual routine documents; instead,
  all the routines are described together in the Chapter Introduction.
 
  The Manual has been typeset using the typesetting system TSSD [6].
 
1.3. Supplementary Documentation
 
  In addition to the full Manual, NAG provides the following alternative
  forms of documentation, which may be more convenient to use, but do
  not contain all the information and advice which is provided in the
  full Manual:
 
  - the Introductory Guide
  - the Concise Reference
  - the On-line Information Supplement
 
  All these forms of documentation follow the same basic structure
  (ordering, division into chapters) as the Manual. Further details of
  their contents are given in Section 3.7.
 
1.4. Marks of the Library
 
  Periodically a new Mark of the NAG Fortran Library is released: new
  routines are added, corrections or improvements are made to existing
  routines; occasionally routines are withdrawn if they have been
  superseded by improved routines.
 
  At each Mark, the documentation of the Library is updated. You must
  make sure that your documentation has been updated to the same Mark as
  the Library software that you are using.
 
  Marks are numbered, e.g. 12, 13, 14. The current Mark is 15.
 
  The Library software may be updated between Marks to an intermediate
  maintenance level, in order to incorporate corrections. Maintenance
  levels are indicated by a letter following the Mark number, e.g. 15A,
  15B, and so on (Mark 15 documentation supports all these maintenance
  levels).
 
1.5. Implementations of the Library
 
  The NAG Fortran Library is available on many different computer
  systems. For each distinct system, an implementation of the Library is
  prepared by NAG, e.g. the Cray XMP Unicos implementation. The
  implementation is distributed to sites as a tested compiled library.
 
  An implementation is usually specific to a range of machines (e.g. the
  DEC VAX range); it may also be specific to a particular operating
  system, Fortran compiler, or compiler option (e.g. scalar or vector
  mode).
 
  Essentially the same facilities are provided in all implementations of
  the Library, but, because of differences in arithmetic behaviour and
  in the compilation system, routines cannot be expected to give
  identical results on different systems, especially for sensitive
  numerical problems.
 
  The documentation supports all implementations of the Library, with
  the help of a few simple conventions, and a small amount of
  implementation-dependent information, which is published in a separate
  Users' Note for each implementation (see Section 3.4).
 
1.6. Precision of the Library
 
  The NAG Fortran Library is developed in both single precision and
  double precision versions. REAL variables and arrays in the single
  precision version are replaced by DOUBLE PRECISION variables and arrays
  in the double precision version.
 
  On most systems only one precision of the Library is available; the
  precision chosen is that which is considered most suitable in general
  for numerical computation (double precision on most systems).
 
  On some systems both precisions are provided: in this case, the double
  precision routines have names ending in `F' (as in the documentation),
  and the single precision routines have names ending in `E'. Thus in
  DEC VAX/VMS implementations:
 
    D01AJF is a routine in the double precision implementation;
 
    D01AJE is the corresponding routine in the single precision
           implementation.
 
1.7. Library Identification
 
  You must know 4 which implementation, which precision 3 and 4 which
  Mark 3 of the Library you are using or intend to use. To find out
  which implementation, precision and Mark of the Library is available
  at your site, you can run a program which calls the NAG Library
  routine A00AAF (or A00AAE in some single precision implementations).
  This routine has no parameters; it simply outputs text to the NAG
  Library advisory message unit (see Section 2.4). An example of the
  output is:
 
    *** Start of NAG Library implementation details ***
    Implementation title: DEC VAX range (VAX/VMS)
               Precision: FORTRAN double precision
            Product Code: (implementation code FLDVV15D)
                    Mark: 15A
    *** End of NAG Library implementation details ***
 
  (The implementation code can be ignored, except possibly when
  communicating with NAG; see Section 4.)
 
1.8. Fortran Language Standards
 
  All routines in the Library conform to ANSI Standard Fortran 77 [8],
  except for the use of a double precision complex data type (usually
  COMPLEX*16) in some routines in double precision implementations of
  the Library - there is no provision for this data type in the standard.
 
  Many of the routines in the Library were originally written to conform
  to the earlier Fortran 66 standard [7], and their calling sequences
  contain a few parameters which are not strictly necessary in Fortran
  77.
 
 
2. Using the Library
 
2.1. General Advice
 
  A NAG Fortran Library routine cannot be guaranteed to return
  meaningful results, irrespective of the data supplied to it. Care and
  thought must be exercised in:
 
  (a) formulating the problem;
  (b) programming the use of library routines;
  (c) assessing the significance of the results.
 
  The Foreword to the Manual provides some further discussion of points
  (a) and (c); Sections 2.2 to 2.5 are concerned with (b).
 
2.2. Programming Advice
 
  The NAG Fortran Library and its documentation are designed on the
  assumption that users know how to write a calling program in Fortran.
 
  When programming a call to a routine, read the routine document
  carefully, especially the description of the Parameters. This states
  clearly which parameters must have values assigned to them on entry to
  the routine, and which return useful values on exit. See Section 3.3
  for further guidance.
 
  The most common types of programming error in using the Library are:
 
  - incorrect parameters in a call to a Library routine;
  - calling a double precision implementation of the Library from a
    single precision program, or vice versa.
 
  Therefore if a call to a Library routine results in an unexpected
  error message from the system (or possibly from within the Library),
  check the following:
 
    Has the NAG routine been called with the correct number of
    parameters?
 
    Do the parameters all have the correct type?
 
    Have all array parameters been dimensioned correctly?
 
    Is your program in the same precision as the NAG Library routines to
    which your program is being linked?
 
    Have NAG routine names have been modified - if necessary - as
    described in Sections 1.6 and 2.5?
 
  Avoid the use of NAG-type names for your own program units or COMMON
  blocks: in general, do not use names which contain a three-character
  NAG chapter name embedded in them; they may clash with the names of an
  auxiliary routine or COMMON block used by the NAG Library.
 
2.3. Error handling and the Parameter IFAIL
 
  NAG Fortran Library routines may detect various kinds of error,
  failure or warning conditions. Such conditions are handled in a
  systematic way by the Library. They fall roughly into three classes:
 
    (i) an invalid value of a parameter on entry to a routine;
   (ii) a numerical failure during computation (e.g. approximate
        singularity of a matrix, failure of an iteration to converge);
  (iii) a warning that although the computation has been completed, the
        results cannot be guaranteed to be completely reliable.
 
  All three classes are handled in the same way by the Library, and are
  all referred to here simply as errors.
 
  The error-handling mechanism uses the parameter IFAIL, which occurs in
  the calling sequence of most NAG Library routines (almost always it is
  the last parameter). IFAIL serves two purposes:
 
   (i) it allows users to specify what action a Library routine should
       take if it detects an error;
  (ii) it reports the outcome of a call to a Library routine, either
       success (IFAIL = 0) or failure (IFAIL 0, with different values
       indicating different reasons for the failure, as explained in
       Section 6 of the routine document).
 
  For the first purpose IFAIL must be assigned a value before calling
  the routine; since IFAIL is reset by the routine, it must be passed as
  a variable, not as an integer constant. Allowed values on entry are:
 
  IFAIL =  0: an error message is output, and execution is terminated
              (hard failure);
 
  IFAIL = +1: execution continues without any error message;
 
  IFAIL = -1: an error message is output, and execution continues. The
              settings IFAIL = 1 are referred to as `soft failure'.
 
  The safest choice is to set IFAIL to 0, but this is not always
  convenient: some routines return useful results even though a failure
  (in some cases merely a warning) is indicated. However, if IFAIL is
  set to 1 on entry, it is essential for the program to test its value
  on exit from the routine, and to take appropriate action.
 
  The specification of IFAIL in Section 5 of a routine document suggests
  a suitable setting of IFAIL for that routine.
 
  For a full description of the error-handling mechanism, see Chapter
  P01.
 
2.4. Input/output in the Library
 
  Most NAG Library routines perform no output to an external file,
  except possibly to output an error message. All error messages are
  written to a logical error message unit. This unit number (which is
  set by default to 6 in most implementations) can be changed by calling
  the Library routine X04AAF.
 
  Some NAG Library routines may optionally output their final results,
  or intermediate results to monitor the course of computation. All
  output other than error messages is written to a logical advisory
  message unit. This unit number (which is also set by default to 6 in
  most implementations) can be changed by calling the Library routine
  X04ABF. Although it is logically distinct from the error message unit,
  in practice the two unit numbers may be the same.
 
  All output from the Library is formatted.
 
  The only Library routines which perform input from an external file
  are a few option-setting routines in Chapter E04: the unit number is a
  parameter to the routine, and all input is formatted.
 
  You must ensure that the relevant Fortran unit numbers are associated
  with the desired external files, either by an OPEN statement in your
  calling program, or by operating system commands.
 
2.5. Auxiliary Routines
 
  In addition to those Library routines which are documented and are
  intended to be called by users, the Library also contains many
  auxiliary routines. Details of all the auxiliary routines which are
  called directly or indirectly by any documented NAG Library routine,
  are supplied to sites in machine-readable form with the Library
  software.
 
  In general, you need not be concerned with them at all, although you
  may be made aware of their existence if, for example, you examine a
  memory map of an executable program which calls NAG routines. The only
  exception is that when calling some NAG Library routines, you may be
  required or allowed to supply the name of an auxiliary routine from
  the NAG Library as an external procedure parameter. The routine
  documents give the necessary details. In such cases, you only need to
  supply the name of the routine; you never need to know details of its
  parameter-list.
 
  NAG auxiliary routines have names which are similar to the name of the
  documented routine(s) to which they are related, but with last letter
  `Z', `Y', and so on, e.g. D01BAZ is an auxiliary routine called by
  D01BAF. In a single precision implementation in which the names of
  documented routines end in `E', the names of auxiliary routines have
  their first three and last three characters interchanged, e.g. BAZD01
  is an auxiliary routine (corresponding to D01BAZ) called by D01BAE.
 
 
3. Using the Documentation
 
3.1. Using the Manual
 
  The Manual is designed to serve the following functions:
 
  - to give background information about different areas of numerical
    and statistical computation;
  - to advise on the choice of the most suitable NAG Library routine or
    routines to solve a particular problem;
  - to give all the information needed to call a NAG Library routine
    correctly from a Fortran program, and to assess the results.
 
  At the beginning of the Manual are some general introductory
  documents. The following may help you to find the chapter, and
  possibly the routine, which you need to solve your problem:
 
    Contents Summary Mark 15 - a structured list of routines in the
                               Library, by chapter;
    KWIC Index               - a keyword index to chapters and routines;
    GAMS Index               - a list of NAG routines classified
                               according to the GAMS scheme.
 
  Having found a likely chapter or routine, you should read the
  corresponding Chapter Introduction, which gives background information
  about that area of numerical computation, and recommendations on the
  choice of a routine, including indices, tables or decision trees.
 
  When you have chosen a routine, you must consult the routine document.
  Each routine document is essentially self-contained (it may contain
  references to related documents). It includes a description of the
  method, detailed specifications of each parameter, explanations of
  each error exit, remarks on accuracy, and an example program to
  illustrate the use of the routine.
 
3.2. Structure of Routine Documents
 
  Note: at Mark 14 some changes were made to the style and appearance of
  routine documents. If you have a Manual which contains pre-mark 14
  routine documents, you will find that it contains older documents
  which differ in style, although they contain essentially the same
  information. Sections 3.2, 3.3 and 3.5 of this Essential Introduction
  describe the new-style routine documents. Section 3.8 gives some
  details about the old-style documents.
 
  All routine documents have the same structure, consisting of nine
  numbered sections:
 
    1. Purpose
    2. Specification
    3. Description
    4. References
    5. Parameters (see Section 3.3 below)
    6. Error Indicators
    7. Accuracy
    8. Further Comments
    9. Example    (see Section 3.5 below)
 
  In a few documents, Section 5 also includes a description of printed
  output which may optionally be produced by the routine.
 
3.3. Specifications of Parameters
 
  Section 5 of each routine document contains the specification of the
  parameters, in the order of their appearance in the parameter list.
 
3.3.1.  Classification of parameters
 
  Parameters are classified as follows:
 
    Input: you must assign values to these parameters on or before entry
    to the routine, and these values are unchanged on exit from the routine.
 
    Output: you need not assign values to these parameters on or before
    entry to the routine; the routine may assign values to them.
 
    Input/Output: you must assign values to these parameters on or
    before entry to the routine, and the routine may then change these
    values.
 
    Workspace: array parameters which are used as workspace by the
    routine. You must supply arrays of the correct type and dimension,
    but you need not be concerned with their contents.
 
    External Procedure: a subroutine or function which must be supplied
    (e.g. to evaluate an integrand or to print intermediate output).
    Usually it must be supplied as part of your calling program, in
    which case its specification includes full details of its
    parameter-list and specifications of its parameters (all enclosed in
    a box). Its parameters are classified in the same way as those of
    the Library routine, but because you must write the procedure rather
    than call it, the significance of the classification is different:
 
      Input: values may be supplied on entry, which your procedure must
      not change.
 
      Output: you may or must assign values to these parameters before
      exit from your procedure.
 
      Input/Output: values may be supplied on entry, and you may or must
      assign values to them before exit from your procedure.
 
    Occasionally, as mentioned in Section 2.5, the procedure can be
    supplied from the NAG Library, and then you only need to know its name.
 
    User Workspace: array parameters which are passed by the Library
    routine to an external procedure parameter. They are not used by the
    routine, but you may use them to pass information between your
    calling program and the external procedure.
 
    Dummy: a simple variable which is not used by the routine. A
    variable or constant of the correct type must be supplied, but its
    value need not be set. (A dummy parameter is usually a parameter
    which was required by an earlier version of the routine and is
    retained in the parameter-list for compatibility.)
 
3.3.2. Constraints and suggested values
 
  The word `Constraint:' or `Constraints:' in the specification of an
  Input parameter introduces a statement of the range of valid values
  for that parameter, e.g.
 
    Constraint: N > 0.
 
  If the routine is called with an invalid value for the parameter (e.g.
  N = 0), the routine will usually take an error exit, returning a
  non-zero value of IFAIL (see Section 2.3).
 
  In newer documents constraints on parameters of type CHARACTER only
  list uppercase alphabetic characters, e.g.
 
    Constraint: STRING = 'A' or 'B'.
 
  In practice all routines with CHARACTER parameters will permit the use
  of lower case characters.
 
  The phrase `Suggested Value:' introduces a suggestion for a reasonable
  initial setting for an Input parameter (e.g. accuracy or maximum
  number of iterations) in case you are unsure what value to use; you
  should be prepared to use a different setting if the suggested value
  turns out to be unsuitable for your problem.
 
3.3.3. Array parameters
 
  Most array parameters have dimensions which depend on the size of the
  problem. In Fortran terminology they have `adjustable dimensions': the
  dimensions occurring in their declarations are integer variables which
  are also parameters of the Library routine.
 
  For example, a Library routine might have the specification:
 
    SUBROUTINE <name> (M, N, A, B, LDB)
    INTEGER       M, N, A(N), B(LDB,N), LDB
 
  For a one-dimensional array parameter, such as A in this example, the
  specification would begin:
 
    A(N) - INTEGER array.
 
  You must ensure that the dimension of the array, as declared in your
  calling (sub)program, is at least as large as the value you supply for
  N. It may be larger; but the routine uses only the first N elements.
 
  For a two-dimensional array parameter, such as B in the example, the
  specification might be:
 
    B(LDB,N) - INTEGER array.
 
      On entry: the m by n matrix B.
 
  and the parameter LDB might be described as follows:
 
    LDB - INTEGER.                                                 Input
 
      On entry: the first dimension of the array B as declared in the
      (sub)program from which <name> is called.
 
      Constraint: LDB  >= M.
 
  You must supply the first dimension of the array B, as declared in
  your calling (sub)program, through the parameter LDB, even though the
  number of rows actually used by the routine is determined by the
  parameter M. You must ensure that the first dimension of the array is
  at least as large as the value you supply for M. The extra parameter
  LDB is needed because Fortran does not allow information about the
  dimensions of array parameters to be passed automatically to a routine.
 
  You must also ensure that the second dimension of the array, as
  declared in your calling (sub)program, is at least as large as the
  value you supply for N. It may be larger, but the routine only uses
  the first N columns.
 
  A program to call the hypothetical routine used as an example in this
  section might include the statements:
 
    INTEGER AA(100), BB(100,50)
    LDB = 100
    .
    .
    .
    M = 80
    N = 20
    CALL <name>(M,N,AA,BB,LDB)
 
  Fortran requires that the dimensions which occur in array
  declarations, must be greater than zero. Many NAG routines are
  designed so that they can be called with a parameter like N in the
  above example set to 0 (in which case they would usually exit
  immediately without doing anything). If so, the declarations in the
  Library routine would use the `assumed size' array dimension, and
  would be given as:
 
    INTEGER        M, N, A(*), B(LDB,*), LDB
 
  However, the original declaration of an array in your calling program
  must always have constant dimensions, greater than or equal to 1.
 
  Consult an expert or a textbook on Fortran, if you have difficulty in
  calling NAG routines with array parameters.
 
3.4. Implementation-dependent Information
 
  In order to support all implementations of the Library, the Manual has
  adopted a convention of using bold italics to distinguish terms which
  have different interpretations in different implementations.
 
  The most important bold italicised terms are the following; their
  interpretation depends on whether the implementation is in single
  precision or double precision:
 
  real                  means  REAL              or  DOUBLE PRECISION
  complex               means  COMPLEX           or  COMPLEX*16
                                                     (or equivalent)
  basic precision       means  single precision  or  double precision
  additional precision  means  double precision  or  quadruple precision
 
  Another important bold italicised term is which denotes the relative
  precision to which real floating-point numbers are stored in the
  computer, e.g. in an implementation with approximately 16 decimal
  digits of precision, has a value of approximately10**(-16).
 
  The precise value of is given by the function X02AJF. Other functions
  in Chapter X02 return the values of other implementation-dependent
  constants, such as the overflow threshold, or the largest
  representable integer. Refer to the X02 Chapter Introduction for more
  details.
 
  For each implementation of the Library, a separate Users' Note is
  published. This is a short document, revised at each Mark. At most
  installations it is available in machine-readable form. It gives any
  necessary additional information which applies specifically to that
  implementation, in particular:
 
  - the interpretation of bold italicised terms;
  - the values returned by X02 routines;
  - the default unit numbers for output (see Section 2.4);
  - details of name changes for Library routines (see Sections 1.6 and 2.5).
 
3.5. Example Programs and Results
 
  The example program in the last section of each routine document
  illustrates a simple call of the routine. The programs are designed so
  that they can fairly easily be modified, and so serve as the basis for
  a simple program to solve a user's own problem.
 
  Bold italicised terms are used in the printed text of the example
  program, to denote precision-dependent features in the code. The
  correct Fortran code must be substituted before the program can be
  run. In addition to the terms real and complex which were explained in
  Section 3.4, the following are used in the example programs:
 
  Intrinsic Functions: real  means REAL  or DBLE (see Note below)
                       imag  means AIMAG or DIMAG
                       cmplx means CMPLX or DCMPLX
                       conjg means CONJG or DCONJG
  Edit Descriptor:     e     means E     or D    (in FORMAT statements)
  Exponent Letter:     e     means E     or D    (in constants)
 
  Note: in some implementations, the intrinsic function real with a
  complex argument must be interpreted as DREAL rather than DBLE.
 
  For each implementation of the Library, NAG distributes the example
  programs in machine-readable form, with all necessary modifications
  already applied. Many sites make the programs accessible in this form
  to users.
 
  Note that the results from running the example programs may not be
  identical in all implementations, and may not agree exactly with the
  results which are printed in the Manual and which were obtained from
  an Apollo DN3000 double precision implementation (with approximately
  16 digits of precision).
 
  The Users' Note for your implementation will mention any special
  changes which need to be made to the example programs, and any
  significant differences in the results.
 
3.6. Summary for New Users
 
  If you are unfamiliar with the NAG Library and are thinking of using a
  routine from it, please follow these instructions:
 
  (a) read the whole of the Essential Introduction;
  (b) consult the Contents Summary or KWIC Index to choose an
      appropriate chapter or routine;
  (c) read the relevant Chapter Introduction;
  (d) choose a routine, and read the routine document. If the routine
      does not after all meet your needs, return to steps (b) or (c);
  (e) read the Users' Note for your implementation;
  (f) consult local documentation, which should be provided by your
      local support staff, about access to the NAG Library on your
      computing system.
 
  You should now be in a position to include a call to the routine in a
  program, and to attempt to run it. You may of course need to refer
  back to the relevant documentation in the case of difficulties, for
  advice on assessment of results, and so on.
 
  As you become familiar with the Library, some of steps (a) to (f) can
  be omitted, but it is always essential to:
 
  - be familiar with the Chapter Introduction;
  - read the routine document;
  - be aware of the Users' Note for your implementation.
 
3.7. Supplementary Documentation
 
  The Introductory Guide contains all the general introductory
  documents, indexes, chapter introductions and chapter contents, from
  the full Manual. It thus gives background information, and advice on
  choosing the most suitable NAG Library routine; but it does not
  contain any detailed specifications of the routines.
 
  The Concise Reference contains details of the parameter-lists of all
  routines in the Library, and very terse (usually one-line) summaries
  of the specification of each parameter, and of the meaning of each
  error-exit. It is not an adequate substitute for the documentation in
  the full Manual, especially if you are trying to use a routine for the
  first time, but it is intended to be a compact and convenient
  memory-aid for users who have gained some familiarity with the Library.
 
  The On-line Information Supplement is a machine-based `Help' system,
  which describes the subject areas covered by the Library, advises on
  the choice of routines, and gives essential programming details for
  each documented routine. It contains a machine-readable version of
  Sections 1, 2, 5 and 6 of each routine document.
 
  The On-Line Information Supplement is a separate product from the
  Library: consult local documentation to see if it is available at your
  site.
 
3.8. Pre-Mark 14 Routine Documents
 
  You need only read this section if you have an updated Manual, which
  contains pre-mark 14 documents.
 
  You will find that older routine documents appear in a somewhat
  different style, or even several styles if your Manual dates back to
  Mark 7, say. The following are the most important differences between
  the earlier styles and the new style introduced at Mark 14:
 
  - before Mark 12, routine documents had 13 sections: the extra
    sections have either been dropped or merged with the present
    Section 8 (Further Comments);
 
  - in Section 5, parameters were not classified as Input, Output and so
    on; the phrase `Unchanged on exit' was used to indicate an input
    parameter;
 
  - example programs were revised at Mark 12 and again at Mark 14, to
    take advantage of features of Fortran 77: the programs printed in
    older documents do not correspond exactly with those which are now
    distributed to sites in machine-readable form;
 
  - before Mark 12, the printed example programs did not use bold
    italicised terms; they were written in standard single precision
    Fortran;
 
  - before Mark 9, the printed example results were generated on an ICL
    1906A (with approximately 11 digits of precision), and between Marks
    9 and 12 they were generated on an ICL 2900 (with approximately 16
    digits of precision);
 
  - before Mark 13, documents referred to `the appropriate implementation
    document'; this means the same as the `Users' Note for your
    implementation'.
 
 
4. Contact between Users and NAG
 
  For further advice or communication about the NAG Library, you should
  first turn to the staff of your local computer installation. This
  covers such matters as:
 
  - obtaining a copy of the Users' Note for your implementation;
  - obtaining information about local access to the Library;
  - seeking advice about using the Library;
  - reporting suspected errors in routines or documents;
  - making suggestions for new routines or features;
  - purchasing NAG documentation.
 
  Your installation may have advisory and/or information services to
  handle such enquiries. In addition NAG asks each installation mounting
  the Library to nominate a NAG site representative, who may be
  approached directly in the absence of an advisory service. Site
  representatives receive information from NAG about confirmed errors,
  the imminence of updates, and so on, and will forward users' enquiries
  to the appropriate person in the NAG organisation if they cannot be
  dealt with locally. If you are unable to make contact with your local
  site representative, you should write to the address given in the
  Users' Note or to the address given at the head of the Library Manual.
\end{verbatim}
\newpage
\begin{verbatim}
5. Further Information
 
  In the NAG Fortran Library Manual, the document Development of NAG
  gives general information about the NAG project, while Summary of
  Services gives details of other NAG products and services, including
  numerical subroutine libraries in Ada, Pascal and Algol 68.
 
  In addition, references [2], [3], [4], and [5] discuss various aspects
  of the design and development of the NAG Library, and NAG's technical
  policies and organisation.
 
6. References
 
  [1] Collected Algorithms from ACM,
      Index by subject to algorithms, 1960-1976.
 
  [2] FORD, B.
      Transportable Numerical Software.
      Lecture Notes in Computer Science, 142, pp. 128-140, 1982.
 
  [3] FORD, B., BENTLEY, J., DU CROZ, J.J. and HAGUE, S.J.
      The NAG Library machine.
      Software Practice and Experience, 9, 1, pp. 65-72, 1979.
 
  [4] FORD, B. and POOL, J.C.T.
      The Evolving NAG Library Service.
      In: `Sources and Development of Mathematical Software',
      W. Cowell (Ed.).
      Prentice-Hall, Englewood Cliffs, pp. 375-397, 1984.
 
  [5] HAGUE, S.J., NUGENT, S.M. and FORD, B.
      Computer-based Documentation for the NAG Library.
      Lecture Notes in Computer Science, 142, pp. 91-127, 1982.
 
  [6] HOPPER, M.J.
      TSSD, a Typesetting System for Scientific Documents.
      United Kingdom Atomic Energy Authority, Harwell,
      Report AERE-R 8574, 1982.
 
  [7] USA Standard Fortran.
      American National Standards Institute, Publication X3.9, 1966.
 
  [8] American National Standard Fortran.
      American National Standards Institute, Publication X3.9, 1978.
\end{verbatim}
