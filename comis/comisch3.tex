%%%%%%%%%%%%%%%%%%%%%%%%%%%%%%%%%%%%%%%%%%%%%%%%%%%%%%%%%%%%%%%%%%%
%                                                                 %
%   COMIS - Reference Manual -- LaTeX Source                      %
%                                                                 %
%   Chapter 3                                                     %
%                                                                 %
%   Files referenced: none                                        %
%                                                                 %
%   Editor: Michel Goossens / CN-AS                               %
%   Last Mod.: 17 Jan 1993  12:25  mg                             %
%                                                                 %
%%%%%%%%%%%%%%%%%%%%%%%%%%%%%%%%%%%%%%%%%%%%%%%%%%%%%%%%%%%%%%%%%%%
\Filename{H1comiscalling-sequences}
\chapter{Calling sequences}
 
This chapter describes the \COMIS{} interface with a user
and his application program.

\Filename{H2comisinitialization}
\section{Initialization}
 
The first action to perform is to initialize
the \COMIS{} system variables by calling:

\Shubr{CSINIT}{(NWORDS)}

where the input parameter \Lit{NWORDS} is the size of the system common block
\begin{alltt} 
 COMMON /COMIS/CS(NWORDS)
\end{alltt} 
By default \Lit{NWORDS=2000}, which is usually sufficient.

\Filename{H2comis-entry-to-comis}
\section{Entry to \protect\COMIS{}}
 
The three routines described in this section can be used for this purpose:

\Shubr{CSPAUS}{(ENTRY\_PROMPT)}
 
where \Lit{ENTRY_PROMPT} is a character string. In this case
\COMIS{} types the \Lit{ENTRY_PROMPT} and the dialogue is started.

\Shubr{CSTEXT}{(ENTRY\_PROMPT,TEXT)}
 
where input parameters \Lit{ENTRY_PROMPT} and \Lit{TEXT} are
characters strings. 
In this case
\COMIS{} interprets the string ``\Lit{TEXT}'', 
types \Lit{ENTRY_PROMPT} and starts the dialogue.

\Shubr{CSEXEC}{(TEXT,IERR*)}
 
where input parameter \Lit{TEXT} is a character string. In this case
\COMIS{} interprets the given string ``\Lit{TEXT}'' only and control is returned
to the calling routine. 
The output parameter \Lit{IERR} is zero when no errors 
are encountered, non-zero otherwise.
 
During the dialogue session \COMIS{} gives the standard prompt ``\Lit{CS>}''
and the user can:
\begin{UL}
\item enter the system directives( see Chapter~\ref{sec:system-directives});
\item enter the function or the subroutine definition;
\item enter the \COMIS{} main program.
\end{UL}
 
All system directives are executed immediately and
the prompt ``\Lit{CS>}'' is displayed again.
 
If the FUNCTION or the SUBROUTINE statement is entered, \COMIS{} changes
his prompt to ``\Lit{FSD>}'' until the END statement completes the routine
definition. An intermediate code is stored in the internal buffer
and \COMIS{} gives the prompt ``\Lit{CS>}''.
 
During the main program definition \COMIS{} uses the prompt ``\Lit{MND>}''.
The \COMIS{} linker is invoked automatically if the compilation
was error-free. The linker tries to resolve references in the
order: \COMIS{} routines; user's routines.
After this stage the main program executes
and \COMIS{} types the entry prompt
and gives the standard prompt ``\Lit{CS>}''.
 
A \COMIS{} built-in editor (see Chapter~\ref{sec:comiseditor}) is automatically
invoked when a syntax
error is detected. The command ``\Cind{E}'' of the editor causes
the recompilation of the text edited and then the dialogue
is continued. The command ``\Cind{Q}'' of the editor causes the current
routine definition to be skipped and \COMIS{} gives the prompt ``\Lit{CS>}''.
 
The exit from the \COMIS{} dialogue session is performed
by the \Cind{RETURN} statement
in the main program or by an ``empty'' (\Cind{END} or \Lit{#} )
main program definition.
For exiting from any \COMIS{} routine to an application program
the \Cind{QUIT} statement can be used.

\Filename{H2comis-calling-user-routines}
\section{Calling the user routines}
\label{sec:callinguserroutines}
 
In order to invoke the routines compiled by the user \COMIS{}
has to know the address of the called routine. 
The location of the routine can be passed
to the \COMIS{} interpreter through a call to subroutine \Rind{CSEXT}:

\Shubr{CSEXT}{('name1.type,...,nameN.type\num',name1,...,nameN)}

where \Lit{name1,...,nameN} should be declared as
EXTERNAL in the Fortran program.
Possible values for the \Rarg{.type} specifier are:

\begin{DLtt}{1}
\item[D] for a double precision function;
\item[I] for an integer function;
\item[L] for a logical function;
\item[R] for a real function;
\item[S] for a subroutine;
\item[X] for a complex function.
\end{DLtt}

If the type specifier is omitted, then 
type subroutine (\Lit{'.S'}) is assumed.

Data transmission between Fortran and \COMIS{} routines
is done in the usual way using routine parameters and
through COMMON blocks. 
To handle COMMON blocks
the interpreter  has to know the address of the first element
of each such block.
This is specified by using subroutine \Rind{CSCOM} as follows:

\Shubr{CSCOM}{('name1,...,nameN\num',FE1,...,FEN)}

\Lit{FE1,...,FEN} are the first elements of the COMMON blocks with
names \Lit{name1,...,nameN}.
The name \Lit{\$BLANK} should be used for the blank COMMON.
\index{COMMON}\index{COMMON!blank}

The addresses of COMMON blocks which contain character data should be 
specified with a call to routine \Rdef{CSCOMC}, whose
calling sequence is similar to \Rind{CSCOM}.

\Filename{H2comiscalling-comis-from-user-program}
\section{Calling \COMIS{} routines from a user program}
 
The user can call a \COMIS{} routine from a Fortran77 program
using the routine's name or address. The second call is faster, but it
requires the preliminary calculation of the address by

\begin{alltt}
    JADP=CSADDR('NAME')
\end{alltt}
 
\subsection*{Arithmetic parameters}

This section describes routines and functions which can be used
for calling \COMIS{} subroutines and functions which have only 
arithmetic parameters.

\newpage %%%%%%%%%%%%%%%%%%%%%%%%%%%%%%%%%%%%%%%%%%%%%%%%%%%%%%%%%%%%%%%%

\subsubsection*{Calling a \COMIS{} subroutine}
\index{subroutine}

\Shubrii{CSCALL}{('NAME',NPAR,P1,P2,...)}{CSJCAL}{(JADP,NPAR,P1,P2,...)}

where
\begin{DLtt}{1234567}
\item[NAME] is the name of the called \COMIS{} routine;
\item[JADP] is the address of the called \COMIS{} routine;
\item[NPAR] is the number of arguments;
\item[P1,P2,...] are the routine's actual arguments.
\end{DLtt}
 
\subsection*{Using integer \COMIS{} functions}
\index{function!integer}

\Sfuncii{CSICAL}{I = CSICAL('NAME',NPAR,P1,P2,...)}%
        {CSIJCL}{I = CSIJCL(JADP,NPAR,P1,P2,...)}

\subsection*{Using real \COMIS{} functions}
\index{function!real}
 
\Sfuncii{CSRCAL}{R = CSRCAL('NAME',NPAR,P1,P2,...)}%
        {CSRJCL}{R = CSRJCL(JADP,NPAR,P1,P2,...)}

To speed up execution three special functions are available
for the case of routines with one, two and three arguments:

\begin{alltt}
    R = \Rdef{CSR1FN}(JADP,P1)
    R = \Rdef{CSR2FN}(JADP,P1,P2)
    R = \Rdef{CSR3FN}(JADP,P1,P2,P3)
\end{alltt}

\subsection*{Using double precision \COMIS{} functions}
\index{function!double precision}
 
\Sfuncii{CSDCAL}{D = CSDCAL('NAME',NPAR,P1,P2,...)}%
        {CSDJCL}{D = CSDJCL(JADP,NPAR,P1,P2,...)}

\subsection*{Using complex \COMIS{} functions}
\index{function!complex}
 
\Sfuncii{CSCCAL}{Cx = CSCCAL('NAME',NPAR,P1,P2,...)}%
        {CSCJCL}{Cx = CSCJCL(JADP,NPAR,P1,P2,...)}

\subsection*{General parameters}

This section describes routines and functions which can be used
for calling \COMIS{} subroutines and functions which can have
any type of parameters.

\subsubsection*{Calling a \COMIS{} subroutine}
\index{subroutine}
 
\Shubrii{CSSUBR}{(STR,P1,P2,...)}{CSJSUB}{(JADP,STR1,P1,P2,...)}
 
where \Lit{STR} and \Lit{STR1} are character strings which specifies
the name of called \COMIS{} routines and the type of each
parameters in argument list.

\newpage %%%%%%%%%%%%%%%%%%%%%%%%%%%%%%%%%%%%%%%%%%%%%%%%%%%%%%%%%%%%%%%%

\subsection*{Using integer \COMIS{} functions}
\index{function!integer}
 
\Sfuncii{CSIFUN}{I = CSIFUN(STR,P1,P2,...)}%
        {CSIJFN}{I = CSIJFN(J,STR1,P1,P2,...)}

\subsection*{Using real \COMIS{} functions}
\index{function!real}
 
\Sfuncii{CSRFUN}{R = CSRFUN(STR,P1,P2,...)}%
        {CSRJFN}{R = CSRJFN(J,STR1,P1,P2,...)}

\subsection*{Using double precision \COMIS{} functions}
\index{function!double precision} 
\Sfuncii{CSDFUN}{D = CSDFUN(STR,P1,P2,...)}%
        {CSDJFN}{D = CSDJFN(J,STR1,P1,P2,...)}

\subsection*{Using complex \COMIS{} functions}
\index{function!complex}

\Sfuncii{CSCFUN}{Cx = CSCFUN(STR,P1,P2,...)}%
        {CSCJFN}{Cx = CSCJFN(J,STR1,P1,P2,...)}
 
The parameters \Lit{STR} and \Lit{STR1} 
in these routines have the forms

\begin{DLtt}{1234}
\item[STR]   \Lit{name} (parameter type description)
\item[STR1]  (parameter type description)
\end{DLtt}
 
where ``\Lit{name}'' is the name of the \COMIS{} routine 
which is called;
``\Lit{(parameters types description)}'' specifies the type of each
parameter in the argument list.
 
\Examples
\begin{alltt}
 STR = 'NAME(I,R,D,C,E,*12)'

 where  I  stands for INTEGER or LOGICAL
        R         for REAL
        D         for DOUBLE PRECISION
        C         for COMPLEX
        E         for EXTERNAL
        *12       for CHARACTER *n
\end{alltt}
 
If a double precision \COMIS{} function is declared with:
\begin{alltt}
      FUNCTION CSDPF(R,D,C,I)
      DOUBLE CSDPF,D
      REAL R  INTEGER I
      CHARACTER *(*)C
      .....
      END
\end{alltt}
then the call from the user's routine may look like:
\begin{alltt}
      CHARACTER *12 TEXT
      DOUBLE PRECISION D,DP,CSDFUN
      .....
      D = \Rind{CSDFUN}('CSDPF(R,D,*12,I)', R, DP,TEXT,10)
\end{alltt}

\newpage %%%%%%%%%%%%%%%%%%%%%%%%%%%%%%%%%%%%%%%%%%%%%%%%%%%%%%%%%%%%%%%%

\Filename{H2comisXYZ-expressions}
\section{XYZ-expressions}
 
\COMIS{} supports a special class of XYZ-expressions. The XYZ-expression
uses only the variable names X and/or Y and/or Z, e.g.

\begin{alltt}
      X+Y+Z  or X**2+Z**2  or SIN(Y)**2+COS(Y)**2
\end{alltt}
 
An XYZ-expression can be translated by

\Shubr{CSEXPR}{(XYZexpr,JADDR)}

where \Lit{XYZexpr} is an argument of type character, whose contents is an XYZ-expression,
\Lit{JADDR} is the address of an XYZ-expression stored in the  \COMIS{} internal buffer.
 
The XYZ-expression can be evaluated by

\begin{alltt}
      VAL = \Rind{CSRJCL}(JADDR,NPAR,ARG1,ARG2,ARG3)
\end{alltt}

using the standard way to call a \COMIS{} real function.
 
If X is omitted in the XYZ-expression then \Lit{ARG1} must be a dummy argument,
if Y is omitted then \Lit{ARG2} must be a dummy argument, e.g.

\begin{alltt}
  \textrm{XYZ-expression}        \textrm{can be evaluated by}

  SIN(Y)/Y              V = \Rind{CSRJCL}(JADDR,2,dummy,Y)
  X+Z                   V = \Rind{CSRJCL}(JADDR,3,X,dummy,Z)
  Z**2                  V = \Rind{CSRJCL}(JADDR,3,dummy,dummy,Z)
  X**2                  V = \Rind{CSRJCL}(JADDR,1,X)
\end{alltt}

