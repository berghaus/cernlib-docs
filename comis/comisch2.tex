%%%%%%%%%%%%%%%%%%%%%%%%%%%%%%%%%%%%%%%%%%%%%%%%%%%%%%%%%%%%%%%%%%%
%                                                                 %
%   COMIS - Reference Manual -- LaTeX Source                      %
%                                                                 %
%   Chapter 2                                                     %
%                                                                 %
%   Files referenced: none                                        %
%                                                                 %
%   Editor: Michel Goossens / CN-AS                               %
%   Last Mod.: 30 Jun 1993  08:30  mg                             %
%                                                                 %
%%%%%%%%%%%%%%%%%%%%%%%%%%%%%%%%%%%%%%%%%%%%%%%%%%%%%%%%%%%%%%%%%%%
\Filename{H1comiscompiler}
\chapter{The \COMIS{} compiler}
 
This chapter gives a brief description of the \COMIS{} syntax.
Only the main differences between the \COMIS{} syntax and standard
Fortran 77 will be discussed.
 
The \COMIS{} system language is an almost full implementation of Fortran 77.
As the user is working in a realtime environment, some simplifications
of the language syntax (in addition to the standard one) are also
available: short forms for key-words
(\Lit{CHAR} instead of \Lit{CHARACTER} and so on) and a free source code format.
 
The main differences between the \COMIS{} syntax and standard
Fortran77 are:
\begin{UL}
  \item \COMIS{} does not accept:
      \begin{UL}
        \item character functions;
        \item statement functions;
        \item INTRINSIC statements;
        \item ENTRY statements;
        \item BLOCK DATA statements.
      \end{UL}
  \item \COMIS{} does not allow a character type variable to be equivalenced
        with another character variable.
  \item A type declaration statement, which define the data type of
        common block name's, must precede the COMMON statement.
  \item The logical operators .EQV. and .NEQV. are not included
        in \COMIS{}.
  \item \COMIS{} does not allow a character expression and
        an alternate return specifiers as actual arguments.
\end{UL}

\Filename{H2comis-source-format}
\section{The source format}
 
The \COMIS{} source code is essentially free format. Extensions are:

\begin{UL}
  \item Statements may start at any position.
  \item More than one statement may be given per source line.
        In this case they should be separated by blank(s),
        or optionally by ';'.
  \item A statement may occupy more than one source line without any
        special marks.
  \item Source lines may be up to 80 characters long. Thus,
        source line numbering in columns 73-80 is not available.
  \item A symbolic name is a string of letters, digits and underscore (\Lit{_}).
        The symbolic name may have any length, but significant characters
        are the first eight only.
\end{UL}

\subsection*{Note}

\begin{UL}
\item Blanks are  meaningful in \COMIS{}.
\item Comment lines are marked by \Lit{'*'} or \Lit{'C'} in column one.
      An exclamation mark (!) starts an inline comment, except when it appears as
      the first non-blank character on a line.
\item Some extentions may cause a conflict with the conventions of Fortran~77
      (see the \Cind{\excl{}FORTRAN} directive in Chapter~\ref{sec:system-directives}).
\end{UL}

\section{Data types and constants}
 
The following data types are supported:
\begin{alltt}
 INTEGER, REAL, DOUBLE PRECISION, LOGICAL, CHARACTER, COMPLEX
\end{alltt}
 
There are seven types of constants: Integer, Real, Double precision,
Logical, Complex, Character string and Hollerith.
 
Constants may be placed directly in the source code,
or they can be accessed
by names assigned to them with PARAMETER statements.

\Filename{H2comis-declarations}
\section{Declarations}
\label{sec:declarations}
 
\subsection*{Data type declarations}
\index{data type}
\begin{alltt}
      full form                  short form
      ---------                  ----------
      INTEGER                    INT
      REAL                       REAL
      DOUBLE PRECISION           DOUBLE
      LOGICAL                    LOG
      CHARACTER                  CHAR
      CHARACTER*(*)              CHAR*(*)
      CHARACTER *n               CHAR *n
      COMPLEX                    COMPLEX
\end{alltt}

\subsection*{DIMENSION statement}
\index{DIMENSION}

In F77 the upper bound of the last dimension may be
specified as '*'. \COMIS{} does not allow it.
 
\subsection*{EQUIVALENCE statement or EQU}
\index{EQUIVALENCE}\index{EQU}
 
The short form of this statement is EQU.

In F77 a character type variable may be equivalenced
with another character variable. \COMIS{} does not allow this.

\subsection*{DATA statement}
\index{BLOCK DATA}\index{DATA}\index{DO!implicit}
 
In F77 named common blocks may be initialized by means of a
DATA statement in a BLOCK DATA subprogram. \COMIS{} does not
support BLOCK DATA subprogram, but DATA statement may be
used to initialize common blocks in any \COMIS{} routines.
 
Note that \COMIS{} does not support an implied DO list in a DATA
statement to initialize the elements of an array.

\subsection*{IMPLICIT statement or IMP}
\index{IMPLICIT}\index{IMP}
 
This statement is the same as the standard one.
\COMIS{} supports the \Lit{IMPLICIT NONE} statement also.
 
\subsection*{SAVE statement}
\index{SAVE}

All data are saved in \COMIS{} .

In F77 a named common block name (preceded and followed
by a slash) is allowed in SAVE lists. \COMIS{} does not allow this.
 
\subsection*{INTRINSIC statement}
\index{INTRINSIC}

\begin{tabular}{|c|}\hline
This statement is at present not supported by \COMIS{}.\\\hline
\end{tabular}
 
\subsection*{COMMON statement or COM}
\index{COMMON}\index{COM}
 
This statement is the same as the standard one.
 
The COMMON statement defines the position and the order of the user's
variables and/or arrays in storage . The common block with the same
name must be present in the application program (see also section
\ref{sec:callinguserroutines}).
 
\subsection*{USE statement}
\index{USE}

\begin{alltt}
 USE list
\end{alltt}
 
\index{COMMON}
In any \COMIS{} routines one can have access to variables'
names,  defined in COMMON declarations in
the main \COMIS{} program (see section \ref{sec:COMISmain}). 
This can be done by
the USE statement, in which one should declare the names of
a corresponding COMMON block.
The name \Lit{BLANK\$} should be declared in the list for access
to variable names of a non labeled COMMON.
 
\subsection*{VECTOR statement}
\index{VECTOR}
 
This statement can be used in a \PAW{} version of \COMIS{} only.
 
The declaration \Lit{VECTOR vector_name} may be used inside a \COMIS{}
routine to address a \KUIP{} vector. 
If the vector does not exist,
it is created with the specifications provided by the declared
dimension.
The vectors \Lit{x} and \Lit{y} defined in the example on 
page~\pageref{sec:comis-with-paw} show how this works.

\section{Other language elements}

\subsection{Arithmetic, logical and character string expressions}
\index{expression!arithmetic}
\index{expression!logical}
\index{expression!character}
\index{arithmetic expression}
\index{logical expression}
\index{character expression}

The following operations are available:

\begin{alltt}
    +  -  *  /  **  //
   .LT.   or   <
   .LE.   or   <=
   .EQ.   or   ==
   .NE.   or   /=
   .GE.   or   >=
   .GT.   or   >
   .OR.  .AND.  .NOT.
\end{alltt}

The logical operators \Lit{.EQV.} and \Lit{.NEQV.} are not included
in \COMIS{}.
 
\subsection{Control statements}
\index{control statement}\index{statement!control}
 
\subsection*{Loops}
\index{loop}\index{DO}\index{DO!WHILE}\index{DO!ENDDO}\index{DO!OD}
\index{WHILE}\index{ENDDO}\index{OD}
 
\Lit{DO} loops with index and \Lit{DO WHILE} (logical expression) constructions
are  provided.
Loops may have the \Lit{DO-ENDDO} or \Lit{DO-OD} forms or may be labeled
in a Fortran-like manner. For the indexed loops the index may be of
an integer or a real type only.
 
\subsection{IF statements}
\index{IF}\index{IF!ELSE}\index{IF!ELSEIF}\index{IF!FI}\index{IF!ENDIF}
\index{ELSE}\index{ELSEIF}\index{FI}\index{ENDIF}

\COMIS{} supports both the standard F77 syntax
and a form ending in \Lit{FI} for the \Lit{IF} statement.
\begin{alltt}
     IF...[ELSEIF]...[ELSE]...ENDIF
\textrm{or}
     IF...[ELESIF]...[ELSE]...FI
\end{alltt} 
 
\subsection*{GOTO statements or GO}
\index{GOTO}\index{ASSIGN}
 
Jumps to constant labels, assigned GOTOs in connection with
ASSIGN statements, and computed GOTOs are provided.

F77 allows the use of a list with optional labels on an assigned GOTO statement;
this facility is not supported in \COMIS.
 
\subsection*{CALL statement}
\index{CALL}
 
This statement calls a user or a \COMIS{} routine.

\COMIS{} supports three forms of the CALL statement:

\begin{alltt}
  1.   CALL subr_name [( arg_list)]
  2.   subr_name ( arg_list)
  3.1  subr_name  arg_list
  3.2  subr_name;
\end{alltt}

where \Lit{subr_name} is the name of the subroutine which may be:

\begin{UL}
  \item an already defined \COMIS{} routine;
  \item a user supplied routine: if no \COMIS{} routine with the specified name 
        is found, then \COMIS{} will search for a user routines with that
        name (see also section \ref{sec:callinguserroutines}).
\end{UL}

The semi-colon character ``;'' is significant in case \Lit{3.2} above!
 
\subsection*{INCLUDE directive}
\index{INCLUDE}

An INCLUDE directive is available and has one of the following forms:

\begin{alltt}
  1.   INCLUDE 'name'
  2.   INCLUDE  name
\end{alltt}
 
When \Lit{name} is given between quotes (case \Lit{1.} above),
then the name is taken literally, while in case \Lit{2.} 
the string \Lit{name} is converted to uppercase.

Note that \COMIS{} does not allow recursive INCLUDE directives.

\subsection{Other statements}
\index{CONTINUE}\index{PAUSE}\index{STOP}\index{RETURN}\index{QUIT}\index{END}
\index{RET}\index{CON}\index{\num}
\begin{DLtt}{1234567809}
  \item[CONTINUE]            the short form of this statement is  CON
  \item[RETURN]        the short form of this statement is  RET
  \item[QUIT]          return from \COMIS{} to application program
  \item[END]           the short form of this statement is ``\num''
\end{DLtt}

\subsection{Input/output statements}
\index{input}\index{output}
\index{OPEN}\index{WRITE}\index{PRINT}\index{READ}
\index{ENDFILE}\index{BACKSPACE}\index{REWIND}\index{CLOSE}\index{REQUIRE}
\index{INPUT}\index{TYPE}
 
The full set of F77 input/output statements is implemented.
\COMIS{} provides four types of input/output statements:
\begin{UL}
  \item Sequential input/output statements
  \item Direct access input/output statements
  \item List directed input/output statements
  \item Internal data set input/output statements.
\end{UL}
 
A list of \COMIS{} input/output source statements follows:
\begin{alltt}
 OPEN, WRITE, PRINT, READ, ENDFILE, BACKSPACE, REWIND, CLOSE, INQUIRE.
\end{alltt}
 
The \COMIS{} extentions are:
\begin{UL}
\item INPUT statement  - input from a terminal in free format.
\item TYPE  statement  - output to a terminal in free format.
\end{UL}
 
\subsection*{FORMAT statement}
\index{FORMAT}
 
The FORMAT statement has the form:
\begin{alltt}
 FORMAT( f1 [,f2 [,...,fn]])
\end{alltt}
where \Lit{f1,f2,...,fn} are format codes.
 
The current version of \COMIS{} does not support
\Lit{kP, S, SP, SS, BN} and \Lit{BZ} edit descriptors.

The length of a format specification cannot exceed 256 characters.
 
\subsection*{INQUIRE statement}
\index{INQUIRE}
 
The INQUIRE statement has one of the forms:
\begin{alltt}
       INQUIRE(dsns,iflist)
       INQUIRE(  ns,iflist)
\end{alltt}

\begin{DLtt}{123456}
\item[dsns]    data set name specifier of the form \Lit{FILE= dsn};
\item[ns]      data set reference number of the form \Lit{UNIT= ns};
\item[iflist]  list as defined in Fortran 77.
\end{DLtt}
 
You cannot omit \Lit{UNIT=ns} on a \COMIS{} inquire statement.
 
\subsection*{INPUT statement or INP}
\index{INPUT}\index{INP}
 
The INPUT statement has the form
\begin{alltt}
     INPUT list
\end{alltt}
 
This statement inputs values from the terminal; the user is prompted
with the list element name.
If the user press a carriage return key, the current value is not
changed, otherwise the constant
typed in becomes the new value of a list element. If a list element value
is a constant then this constant is simply typed out.
 
\subsection*{TYPE statement}
\index{TYPE}
 
The TYPE statement has the form
\begin{alltt}
     TYPE list
\end{alltt}
 
This statement types values of the lists elements in free format.
 
\Example
If the user runs the next simple \COMIS{} program:
\begin{alltt}
       CHARACTER A*4
       J=1
       INPUT 'CHARACTER *4',A, 'J HAS VALUE 1',J,Z
       TYPE A,J,Z
       END
\end{alltt}
The dialogue will be
 
\begin{alltt}
'CHARACTER*4'
*C  A=\Ucom{abcd<cr>}
'J HAS VALUE 1'
*I  J=\Ucom{<cr>}
*R  Z=\Ucom{3.14<cr>}
*T  A = ABCD     J = 1     Z = 3.140000
\end{alltt}

\section{Programs}
\index{program}

\subsection{Main program}
\label{sec:COMISmain}
\index{PROGRAM}
 
The \COMIS{} main program syntax is:

\begin{alltt}
   [PROGRAM name]
   [Declaration statements]
   [Executable statements]
   END
\end{alltt}
 
The main program is executed immediately whenever its
definition is finished. The \COMIS{} system ``remembers''
all declarations of common blocks issued in the main
program. These declarations should not be repeated
each time the main program is redefined.
 
Moreover these declarations are valid for other
subprograms through the USE statement 
(see section \ref{sec:declarations})
This does not mean that you cannot enter the new COMMON 
declarations during the main program redefinition.

\subsection{Functions and subroutines}
\index{\Colon}
 
The colon character ``\Colon'' is the short form of key words FUNCTION and SUBROUTINE.
\index{FUNCTION}\index{SUBROUTINE}%
Actual arguments may be constants, variables, arithmetic
expressions, arrays, arrays elements or subroutines names.

\COMIS{} does not allow a character expression and
alternate return specifiers as an actual arguments.
 
The \COMIS{} extensions are:

\begin{UL}
\item The number of routine's arguments may vary and
      can be obtained inside the called routine.
\item An argument can be a sequence of statements enclosed in
      square brackets.
\item An argument can be omited.
\end{UL}

For details see Appendix~\ref{sec:obtain-values-for-actual-paramers}.
 
\COMIS{} does not accept statement functions, functions
of a character type and an ENTRY statement.
\index{ENTRY}
 
The full set of intrinsic functions supplied in the \COMIS{} system
excepting \Lit{CHAR, LLT, LLE, LGT, LGE}.
