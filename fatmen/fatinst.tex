%%%%%%%%%%%%%%%%%%%%%%%%%%%%%%%%%%%%%%%%%%%%%%%%%%%%%%%%%%%%%%%%%%%
%                                                                 %
%   FATMEN User Guide and Reference manual                        %
%                                                                 %
%   Fatmen Part 4: Installaion Guide                              %
%                                                                 %
%   This document needs the following external EPS files:         %
%   none                                                          %
%                                                                 %
%   Editor: Michel Goossens / CN-AS                               %
%   Last Mod.:  7 June 1993 13:20 mg                              %
%                                                                 %
%%%%%%%%%%%%%%%%%%%%%%%%%%%%%%%%%%%%%%%%%%%%%%%%%%%%%%%%%%%%%%%%%%%

\Filename{H1Fatmeninstallation-general-hints}
\chapter{General hints}
\Filename{H2Fatmeninstallation-pam-file-availability}
\section{Availability of PAM files, libraries and FATMEN shell}
\par
The FATMEN package is installed as part of the CERN program library.
If you have a version of the CERN program library corresponding
to CERN Computer Newsletter 197 or later, you should have FATMEN
on your system.
\index{Library}
\index{PAM}
\index{server}
\par
\index{EXEC}
\index{FM}
\index{shell}
\par
The standard installation (CNL 201 and after) creates the FATMEN
callable interface (part of PACKLIB), the FATMEN shell and the
FATMEN server.
\Filename{H2Fatmeninstallation-Using-ZEBRA-HBOOK-with-FATMEN}
\section{Using ZEBRA, HBOOK etc. with FATMEN}
\subsection{The size of the users' store}
\par
\index{ZEBRA}
The FATMEN package creates a division of type 'P' (package)
in the store that the user declares in the call to
FMINIT.(See on Page~\pageref{FMINIT}) This division
is declared with NW=10000 and NWMAX=100000. (See the ZEBRA users'
guide\cite{bib-ZEBRA} for details of the routine MZDIV).
Thus, the users store should be sufficiently large to accomodate
this division.
\subsection{Using HBOOK and FATMEN}
\index{HBOOK}
\par
If the user also wishes to use HBOOK, ZEBRA should be initialised
first followed by a call to HLIMIT with a negative argument. This will
indicate to HBOOK that ZEBRA has already been initialised.
\par
Note that the FATMEN FORTRAN routines do not save and restore the
current directory within an RZ file. 
\subsection{Calling MZWIPE}
\index{MZWIPE}
\par
It is normally safe to call MZWIPE to clean divisions in the store
which contains the FATMEN division with the following exception:
\begin{UL}
\item
Do not use MZWIPE to delete all package divisions
({\tt IXWIPE=IXSTOR+23}). This will delete all banks created with
FMLIFT (see Page~\pageref{FMLIFT}) and not saved using
FMPUT (see Page~\pageref{FMPUT}).
\end{UL}
\Filename{H2Fatmeninstallatio-Using-FATMEN-without-TMS}
\section{Using FATMEN without a Tape Management System}
\par
\index{TMS}
\index{VMTAPE}
The only Tape Management Systems currently supported by FATMEN
are the HEPVM TMS, and VMTAPE. These are selected by the PATCHY statements
\begin{XMP}
+USE,TMS.

and

+USE,VMTAPE.

respectively.
\end{XMP}
\par
\index{TMS}
The Tape Management System is responsible for maintaining information
on tape volumes, such as availability, label type etc. When installed
without the TMS flag, FATMEN takes the default values for each
media type as defined at installation time, or by a call to the
routine FMEDIA (or the MEDIA command in the shell).
\par
To see the current defaults, type the command MEDIA in the shell.
\par
As well as tailoring the default values for each media type,
a user exit, FMUTMS, can be provided to override the default
values for a given volume. See the description of the FMUTMS 
routine in the user interface section of this manual for more details.
\Filename{H1Fatmeninstallation-installing-fatmen}
\chapter{Installing FATMEN}
\par
As described above, the installation of the FATMEN software as part
of the standard CERN program library installation. FATMEN relies
heavily on the program library, and so this is a pre-requisite
for its usage.
\par
The standard program library installation procedure
will generate the FATMEN FORTRAN interface (in PACKLIB),
and three modules. These are the shell (FM), the server (FATSRV)
and a program to create a new, empty RZ file for use with
FATMEN (MKFATNEW).
\index{mkfatnew}
\index{FATNEW}
\index{Creating a new catalogue}

{\tt mkfatnew} is just a simple script that calls the {\tt FATNEW}
program. For example, on VAX/VMS systems the following maybe used.
\begin{XMPt}{Example MKFATNEW command file}
$! fatsys:==CERN ! For example
$! fatgrp:==LHC  ! For example
$
$  type/nopage sys$input

Please give the name of the FATMEN system. This name forms
the top-level of the FATMEN catalogue, e.g. //CERN

$eod
$  inquire/nopunc ans "FATSYS? "
$  if ans.eqs."" then ans = "CERN"
$  fatsys==ans - "//"
$  type/nopage sys$input

Please give the name of the FATMEN group.

$eod
$  inquire/nopunc ans "FATGRP? "
$  if ans.eqs."" then exit
$  fatgrp==ans - "FM"
$  write sys$output ""
$  inquire/nopunc fatdir "Directory where FATMEN catalogue should reside? "
$  olddir = f$environment("DEFAULT")
$  set default 'fatdir'
$  create/directory [.todo]
$  create/directory [.tovm]          ! Only at CERN !!!
$  create/directory [.done]
$  set file/protection=w:rw todo.dir
$  fatman="FM''fatgrp'"
$  set file/acl=(id='fatman',access=read+write,options=default) todo.dir
$  set file/protection=w:rw tovm.dir ! Only at CERN !!!
$  set file/protection=w:rw done.dir
$  run cern:[pro.exe]fatnew
$  set default 'olddir'
\end{XMPt}
\Filename{H2Fatmeninstallation-Installing-FATMEN-on-new-machine}
\section{Installing FATMEN on a new machine}
\par
The first step is the installation of the CERN program libraries.
Once this has been achieved, FATMEN should be configured as below.
\subsection{Access to data}
\par
Access to data through FATMEN is currently supported on:
\begin{OL}
\item
VM/CMS systems running the HEPVM software, in particular SETUP and STAGE.
\item
VM/CMS systems without the HEPVM software are supported
provided that the CERN REXX local function package as well as the
GIME and DROP execs are installed. 
\index{Local function}
\index{REXX local function package}
\index{DROP}
\index{GIME}
\index{RXLOCFN}
\index{REXX}
\index{HEPVM}
\item 
\index{VMBATCH}
\index{VMTAPE}
VM/CMS systems running VMBATCH and/or VMTAPE, provided that the
REXX local function package as well as the
GIME and DROP execs are installed. 
\item
\index{VAXTAP}
VAX/VMS systems. The CERN Program Library Package VAXTAP is required
for access to tapes.
\item
\index{SHIFT}
Unix systems. Tape support is currently only possible on the CERN Cray
system, the CERN SHIFT project and the
L3 Apollo network at CERN.
\par
CERN is developing generalised Unix tape software (staging and tape
mounting) which will be used to provide tape support on all other
Unix systems. This software is based upon the existing Cray and
SHIFT syntax and so the FATMEN interface is already written.
\item
MVS systems. Only FORTRAN I/O is supported in the current version.
The IBM FORTRAN routine FILEINF is used whereever possible,
together with the CERN Program Library Package FTPACK, written
by R. Matthews.
\end{OL}
\par
On all other systems some modifications to the code are required for
data access. These modifications are restricted to the routines
FMOPEN, FMCLOS and FMCOPY.
\subsection{Using the FATMEN catalogue}
\par
The FATMEN catalogue is simply a FORTRAN direct access file, processed
using the CERN ZEBRA package. Users access the file in read-only mode,
with updates being sent to and applied by a server (one per experiment).
Updates are sent as ZEBRA FZ files (exchange mode, ASCII mapping).
\subsection{Configuring FATMEN}
\subsubsection{VM/CMS systems}
\par
Machines that do not run the HEPVM software may use FATMEN
provided that they install the GIME and DROP execs, and
the REXX local function package. To install the REXX local
function package, simply copy the files RXLOCFN SEGMENT and
RXLOCFN MODULE to your local machine. They will be
automatically loaded when needed. Some functions
in this package will return a null-string or zero unless
certain HEPVM CP modifications are installed. However, all of the
functions required for FATMEN will work without any problem.
\par
If you require any of the above software and do not have
access to CERNVM (or any other machine running the HEPVM
software), simply sent a mail message to JAMIE@CERNVM. 
\par
Upon initialisation the FATMEN software issues an 'EXEC GIME FMgroup',
e.g. 'EXEC GIME FML3'. Thus, if the FATMEN catalogue for L3 is NOT
kept on the 191 disk of FML3, an appropriate
\index{names file}
\index{GIME}
entry in the GIMEUSER, GIMEGRP or GIMESYS NAMES file is required.
An additional entry is also required in the normal NAMES file
so that the updates can be correctly sent to the appropriate service
machine.
\par
Systems running VMBATCH should activate the necessary code by
putting
\begin{XMP}
+USE,VMBATCH.
\end{XMP}
into the installation cradle for FATMEN. 
(Different code is required to obtain the username, jobname
and account field in a VMBATCH job to that used in HEPVM batch).
\par
Systems running VMTAPE should activate the necessary code by putting
\begin{XMP}
+USE,VMTAPE.
\end{XMP}
This will provide access to tape files.
If user tapes are catalogued in the VMTAPE catalogue (TMC),
select also
\begin{XMP}
+USE,VMTMC.
\end{XMP}
\par
Systems running both VMTAPE and VMBATCH may simply add
\begin{XMP}
+USE,VMCENTER.
\end{XMP}
\subsubsection{VAX/VMS systems}
\par
Upon initialisation, the client looks for a file in the directory
specified by the symbol FMgroup, e.g. FML3.
If the FATMEN catalogue is kept in
DISK\$FAT:\lsb FATMEN.L3\rsb, 
one should type
\begin{XMP}
FML3:==DISK$FAT:\lsb FATMEN.L3\rsb
\end{XMP}
This directory must contain the subdirectories
\lsb .TODO\rsb , \lsb .DONE\rsb . The \lsb .TODO \rsb directory
must be writeable by all members of the L3 collaboration (in this example).
\par
If the catalogue is on a disk that has disk quotas enabled, then an ACL
must be established on the directory \lsb .TODO\rsb as in the example
below. In this example, the top level directory is \lsb FMCDF \rsb  
belonging to user FMCDF. The identifier CDF\_EXPERIMENT is held by
all members of the CDF experiment and is an identifier with the
RESOURCE attribute. Setting up the directory in this manner allows
all CDF users to send updates to the FATMEN server for their experiment
and avoids the need for entries in the quota file for every CDF user
for the disk in question. 
Only entries for FMCDF and CDF\_EXPERIMENT are required.
\par
If disk quotas are not enabled on the volume in question, then 
allowing group write access to the \lsb .TODO \rsb    
directory is sufficient.
\begin{XMP}
(IDENTIFIER=FMCDF,ACCESS=READ+WRITE+EXECUTE+DELETE+CONTROL)
(IDENTIFIER=CDF_EXPERIMENT,ACCESS=READ+WRITE+EXECUTE)
(IDENTIFIER=FMCDF,OPTIONS=DEFAULT,ACCESS=READ+WRITE+EXECUTE+DELETE+CONTROL)
(IDENTIFIER=CDF_EXPERIMENT,OPTIONS=DEFAULT,ACCESS=READ+WRITE+EXECUTE)
\end{XMP}
\par
The configuration for the server is performed by the
sample command file shown below.
\par
For the server only, a variable FATSYS must be
set to the appropriate name, should the catalogue name not be //CERN
\begin{XMP}
$!
$! Example FATSERV.COM
$!
$ dd = f$cvtime(,,"WEEKDAY")
$ tt = f$time()
$ hh = f$trnlnm("SYS$NODE")
$ write sys$output ""
$ write sys$output "FATSERV starting at ''dd' ''tt' on ''hh'"
$ write sys$output ""
$ !
$ ! Set FATMEN system
$ !
$ FATSYS:==CERN
$ !
$ ! Set FATMEN group
$ !
$ FATGRP:==FMCNDIV
$ !
$ ! Set FATMEN wakeup interval in seconds
$ !
$ ! Define FMCNDIV if not defined in system login
$ !
$ FMCNDIV:==DISK\$CERN:\lsb FMCNDIV\rsb
$ FMWAKEUP:==30
$ !
$ ! Set FATMEN log level
$ !
$ FMLOGL:==3
$ write sys$output -
"FATMEN group set to ''FATGRP', wakeup interval is ''FMWAKEUP' seconds"
$ run cern:\lsb pro.exe\rsb fatsrv
\end{XMP}
\index{remote staging}
\index{multi-file staging}
\index{VAXTAP}

See the \Rind{VAXTAP}~\cite{bib-VAXTAP} long writeup for details on configuring
the VAX tape handling software, in particular for remote and
multi-file staging.
\subsubsection{Unix systems}
\par
Upon initialisation, the client looks for a file in the directory
specified by the variable FMexperiment, e.g. FML3. Thus,
if the FATMEN catalogue is kept in /users/fatmen/l3, the variable
FML3 should be set to this path:
\begin{XMP}
FML3=/users/fatmen/l3; export FML3
\end{XMP}
This directory must contain the subdirectories
todo and done. The todo directory
must be writeable by all members of the L3 collaboration (in this example).
\par
For the server only, a variable FATSYS must be
set to the appropriate name, should the catalogue name not be //CERN
\begin{XMP}
FATSYS=DESY;export FATSYS
\end{XMP}
\par
The configuration for the server is performed by the
sample script shown below.
\begin{XMP}
#!/bin/sh
#
# Example FATSERV script.
#
t="date"
h="hostname"
echo
echo FATSERV starting at $t on $h
echo
FATGRP=FML3   ;export FATGRP
FATSYS=CERN   ;export FATSYS
FMWAKEUP=10;   export FMWAKEUP
FML3=/fatmen/fml3; export FML3
echo FATMEN group set to $FATGRP , wakeup interval is $FMWAKEUP seconds
echo
/cern/pro/bin/fatsrv
\end{XMP}
\Filename{H1Fatmeninstallation-remote-access-to-fatmen-catalogue}
\chapter{Remote access to the FATMEN catalogue}
\index{remote catalogues}
\index{catalogue access}
\par
Remote access to the FATMEN catalogue is currently supported
over DECnet (between VAX/VMS systems only) NFS, AFS and CSPACK.
CSPACK access has not yet been tested on VM/CMS or MVS systems.
\Filename{H2Fatmeninstallation-DECnet-access-to-FATMEN-catalogues}
\section{DECnet access to FATMEN catalogues}
\par
\index{DECnet}
To access a remote catalogue over DECnet, simply include
the node specification in the symbol definition. e.g.
\begin{XMP}
FMCDF:==FNALF::USR$ROOT37:[FMCDF]
\end{XMP}
\Filename{H2Fatmeninstallation-NFS-access-to-FATMEN-catalogues}
\section{NFS access to FATMEN catalogues}
\par
\index{NFS}
To access a remote catalogue over NFS, simply define
the required environmental variable as in the example
below.
\begin{XMP}
FMOPAL=/fatmen/fmopal; export FMOPAL

df /fatmen
Filesystem    Total KB    free %used   iused %iused Mounted on
fatcat:/fatmen  409600  138704   66%   -       -    /fatmen

\end{XMP}

\Filename{H2Fatmeninstallation-AFS access-to-FATMEN-catalogues}
\section{AFS access to FATMEN catalogues}
\index{AFS}
To access a remote catalogue over AFS, one simply
has to define the appropriate environment variable
to point to the directory where the catalogue resides,
e.g.

\begin{XMP}
export FML3=/afs/cern.ch/fatmen/fml3
\end{XMP}

\Filename{H2Fatmeninstallation-CSPACK-access-to-FATMEN-catalogues}
\section{CSPACK access to FATMEN catalogues}
\index{CSPACK}

If FATMEN has been installed with the {\tt CSPACK}
option enabled (\Lit{+USE,CSPACK} in the PATCHY step of
the library build), then remote catalogues may also
be accessed and updated using {\tt CSPACK}. The
standard {\tt ZSERV} must be installed on the remote
system on which the catalogue resides, as described in
the {\tt CSPACK}~\cite{bib-CSPACK} manual. One then defines
the symbol (VAX/VMS) or environmental variable (UNIX)
that points to the directory where the catalogue resides
with the syntax {\tt node:path}.

Thus, to enable {\tt CSPACK} access to the {\tt L3}
catalogue residing on the node {\tt fatcat}, the
appropriate {\tt C shell} definition would be

\begin{XMP}
setenv FML3 fatcat:/fatmen/fml3
\end{XMP}

\Filename{H2Fatmeninstallation-FATCAT-dedicated-FATMEN-server-at-CERN}
\section{FATCAT - the dedicated FATMEN server at CERN}
\index{fatcat}

At CERN, the FATMEN catalogues are stored on the node FATCAT.
This is currently an RS6000 which is only used for FATMEN catalogue
management. The recommended way of accessing a FATMEN catalogue at
CERN is the following:
\begin{OL}
\item
NFS mount the \Lit{/fatmen} file system
\item
Define environmental variables to point to the directory of interest
\end{OL}

An example of how this may be done is shown below.

\begin{XMPt}{NFS mounting the /fatmen file system on a Unix platform}

mount fatcat:/fatmen /fatmen

\end{XMPt}

\begin{XMPt}{NFS mounting the /fatmen file system on a VMS platform}

$NFSMOUNT fatcat:"/fatmen" fatmen

\end{XMPt}

\begin{XMPt}{Defining the environmental variables on a Unix system}
#/bin/ksh
for i in /fatmen/fm*
   do
     typeset -u fatgrp
     fatpath=$i
     fatgrp=`basename $i`
     echo Setting $fatgrp to $fatpath ...
     eval $fatgrp=$fatpath;export $fatgrp
   done
\end{XMPt}

\begin{XMPt}{Defining the environmental variables on a VAX/VMS system}
$ loop:
$ fatman = f$search("FATMEN:[000000]FM*.DIR")
$ if fatman .eqs. "" then exit
$ fatman = f$parse(fatman,,,"NAME")
$ fatdir = "FATMEN:[''fatman']"
$ write sys$output "Setting ''fatman' to ''fatdir'..."
$ 'fatman' :== 'fatdir'
$ goto loop
\end{XMPt}

\Filename{H1Fatmeninstallation-distributing-catalogue-upfates}
\chapter{Distribution of catalogue updates}
\index{catalogue}
\index{updates}
\index{servers}
\index{distribution of updates}
\index{remote catalogues}
\index{DECnet}
\index{bitnet}
\index{TCPIP}
\par
It is unfortunately at least unrealisitic and more likely impossible
to access a single catalogue world-wide and so one typically
has multiple copies of the catalogue. These catalogues
are automatically keep up to date by the FATMEN servers.
The updates may be sent over Bitnet, DECnet or TCP/IP.
However, the servers must be configured so that they
know to which nodes updates should be sent. In addition,
it may be desirable to maintain subsets of the catalogue
at remote sites. For example, information on raw data
may remain at the laboratory where the data is taken
whereas all DST information is sent to all collaborating
institutes. 
\par
\index{NAMEFIND}
\index{NAMEFD}
\index{names file}
The configuration file that is used has the same format
on all systems. This format is that of a VM/CMS names file.
On non-VM systems, these files are processed using the
FORTRAN callable routine, NAMEFD, available in the CERN
program library KERNLIB.
\Filename{H2Fatmeninstallation-Configuring-servers-on-VM-systems}
\section{Configuring servers on VM systems}
\par
This is performed by adding entries to the NAMES file
of the servers in question. For each remote server
one can define up to 16 generic name patterns, each
of which may include wild-cards. For more details
see the example below. Updates may be sent to
remote IBM machines, VAXes connected via Interlink
and Unix machines using the CSPACK software. 
\par
Note that the default for VM systems is to send
the updates using the SENDFILE exec. This can be
overridden for individual entries either by sending
the updates to a gateway machine, as described later,
or by specifying an alternate exec, as for the entry
FMVAX.
\begin{XMP}
:nick.FATSERVERS
               :list FMSACLAY FATCAT FMIN2P3 FMRAL FMVAX

:nick.FMVAX
               :userid.fmsmc
               :node.vxcrna
               :DIR1.//cern/smc/dst/*
               :exec.fat2vax

:nick.FMSACLAY
               :userid.fmsmc
               :node.frsac12
               :DIR1.//cern/smc/dst/*
 
:nick.FMIN2P3
               :userid.fmsmc
               :node.frcpn11
               :DIR1.//cern/smc/dst/*
 
:nick.FMRAL
               :userid.fmsmc
               :node.ukacrl
               :DIR1.//cern/smc/dst/*
 
:nick.FATCAT
               :userid.fmfatcat
               :node.cernvm
               :DIR1.//cern/smc/dst/*
 
\end{XMP}
\par
In the preceeding example, all updates referring to generic
names beginning //CERN/SMC are sent to the server defined
by the name FATCAT, whereas only updates referring to
names beginning //CERN/SMC/DST are sent to the
servers at SACLAY, IN2P3 and RAL.
Note the the FMFATCAT account is in fact a gateway
into the Unix-world (using the program in PATCH FATCAT
in the FATMEN PAM file and that FMVAX is an account of
a VAX linked to the IBM via Interlink, but this is completely
transparent to the server.
\par
On systems other than CERNVM, updates that are to be sent to
non-VM systems should pass via a gateway service machine.
\begin{XMP}
:nick.FATDESY  :userid.R01H1
               :node.dhhdesy3
               :DIR1.//cern/h1/*
               :protocol.mvsjob
               :gateway.fmgate at cernvm

:nick.FATVAX   :userid.fmh1
               :node.vxdesy
               :dir1.//cern/h1/*
               :protocol.tcpip
               :gateway.fmgate at cernvm
               :receive.yes
\end{XMP}
\par
In the above example, updates destined for FATDESY and FATVAX
are not send directly, but pass via an intermediate service machine.
wakes up at predefined intervals and transfers any files to the specified
remote nodes. In the case of the entry FATVAX, files are also
retrieved from the remote system.
\subsection{Transferring updates to VAX/VMS systems via Interlink}
\par
This method should only be used at CERN. For other systems,
a gateway service machine should be used. See the description
below for setting up a gateway service machine.
\par
Files sent from IBM VM/CMS systems via the Interlink VM-DECnet
gateway will arrive in the default directory for the FAL DECnet
object, typically
\begin{XMP}
SYS\$SPECIFIC:\lsb FAL\$SERVER \rsb
\end{XMP}
\par
By default, the files will be named FATMEN.RDRFILE. At CERN
a small modification has been made to the Interlink software
so that the files arrive as VAXUSER.FILENAME\_FILETYPE,
thus files sent to FMOPAL would be called
\begin{XMP}
FMOPAL.FATMEN_RDRFILE
\end{XMP}
\par
These files can be copied to the correct directory using the
command file FATRL.COM, included in the PATCH DCL of the FATMEN
pamfile.
\Filename{H2Fatmeninstallation-Configuring-servers-on-VMS-MVS-and-Unix-systems}
\section{Configuring servers on VMS, MVS and Unix systems}
\par
The names file configuration is exactly the same on VMS, MVS and
Unix systems as on VM nodes. The processing of the names file
is performed by a FORTRAN program, FATSEND,  which uses the CERN program
library routine NAMEFD and the updates are transferred
using the CSPACK routines. 
\par
For each remote server, a subdirectory should be created.
(If the subdirectories are note created, they will be
created as required by the server).
In the case of the example names file shown below, we
would have the subdirectories \lsb .TOD0SG01 \rsb,
\lsb .TOD0SF01 \rsb and so on (VAX/VMS systems). 
The FATMEN server copies all updates into each of these
subdirectories and the program FATSEND copies the updates
to the specified remote nodes, using either TCP/IP or DECnet,
and deletes those files that are successfully copied.
DECnet transfers are only possible between VAX/VMS systems.
\par
On MVS systems, the protocol BITNET is also valid for transmission
of updates to remote VM systems. This is done using the TSO
transmit command.
\par
An example names file is shown below.
\begin{XMP}
:nick.FATSERVERS
               :list FMSGI FMVAX 

:nick.FMSGI    :userid.fmd0
               :node.d0sg01
               :protocl.tcpip
               :dir1//fnal/d0/*
               :receive.yes 
               :queue:USR$ROOT37:[FMD0.TODO]

:nick.FMVAX    :userid.fmd0
               :node.d0sf01
               :protocl.tcpip
               :dir1//fnal/d0/*
\end{XMP}

In this example, servers connecting the the machine \Lit{D0SG01}
will transmit updates in both directions, rather than the
default, which is to send updates only.
This option is useful for the transmission of updates
to a machine that does not accept incoming connections,
as is the case with VAX systems running the UCX TCP/IP software,
the Cray at CERN and MVS systems.

If the tag \Lit{:queue} is specified, the updates are placed in
the specified directory. If not, they are placed in the subdirectory
\Lit{todo} of the account that is used to performed the transfer.

If a subdirectory TOVM exists, the FATMEN server will attempt to send
all updates via Interlink to CERNVM. This option is only intended for
use at CERN.
\Filename{H2Fatmeninstallation-Using-gateway-service-machine-on-VM-systems}
\section{Using a gateway service machine on VM systems}
\par
To send updates from a VM node into a non-VM system,
use the :gateway tag to define a gateway service machine.
The VM service machine will send all updates to this
gateway machine, which in turn will transmit them by
the specified protocol at regular intervals using the
program FATSEND.
\par
Valid protocols are TCPIP, BITNET and MVSJOB.
TCPIP, which is the default, results in the files
being transferred using the CSPACK routines, as
described above. One should normally also set the
:receive.yes option so that updates are also retrieved
from these remote systems.
\par
Specfiying BITNET together with a gateway machine permits
updates to be sent to remote nodes at specified
intervals, rather than immediately.
\par
Although it is possible to use SENDFILE to send files to
MVS nodes, these files are generally unreadable and
hence the preferred method is to specify the protocol
MVSJOB. This will result in an IEBGENER job being submitted
to the remote node, which will copy the updates to a temporary
file in the input queue of the specified server.
The usual restrictions on remote job submission apply.
\Filename{H1Fatmeninstallation-fatsend-program}
\chapter{The Program FATSEND}
\par
This program is generated using the command MAKEPACK FATSEND.
Alternatively, one may use a job such as the following:
\begin{XMP}

VAX/VMS systems

$if p1 .nes. "" then goto 'p1
$ y:
$ypatchy cern:[pro.pam]ZEBRA.pam fatsend.for :go 
+use,qcde.
+use,vaxvms,*fatsend.
+exe.
+pam,11,r=qcde.
+pam,12,t=c,a. fatmen.cards
+quit
$ f:
$for fatsend
$ l:
$link fatsend,'lib$,sys$input/opt
sys$library:vaxcrtl/shareable
sys$system:sys.stb
multinet_socket_library/share
$ exit

Unix systems (Use f77 instead of xlf on non-AIX machines)

ypatchy /cern/pro/pam/ZEBRA.pam fatsend.f :go <<!
+use,qcde.
+use,ibmrt,*fatsend.
+exe.
+pam,11,r=qcde.
+pam,12,t=c,a. fatmen.cards
+quit
!
xlf -c -q extname -q charlen=32000 fatsend.f 
xlf fatsend.o -L/cern/new/lib -lpacklib -lc -o fatsend

\end{XMP}
\par
The program can be tailored by setting the variables FMLOGL and FMWAKEUP, e.g.
\begin{XMP}

$fmlogl:==3
$fmwakeup:==3600 ! Once an hour

\end{XMP}
\par
On Unix systems, the tailoring is done using environmental variables.
The following example is for the C shell.
\begin{XMP}

setenv fmlogl 3
setenv fmwakeup 3600 

\end{XMP}
On VM/CMS systems, the SETENV command should be used, as in
\begin{XMP}
SETENV FMWAKEUP 3600
\end{XMP}
\par
On MVS systems no equivalent to environmental variables exists.
Therefore names files entries are used, as shown below.
\begin{XMP}
:nick.fatsrv  :wakeup.60   :logl.0

:nick.fatsend :wakeup.3600 :logl.3

:nick.FMSGI
               :mvsid.r01d0
               :userid.fmd0
               :node.d0sg01
               :protocl.tcpip
               :dir1//fnal/d0/*
\end{XMP}
\Filename{H1Fatmeninstallation-installing-vaxtap}
\chapter{Installing VAXTAP for tape access on VAX/VMS systems}
\index{VAXTAP}
\par
FATMEN interfaces to the CERN Program Library package VAXTAP
to provide tape support on VAX/VMS systems. The long writeup
should also be consulted.~\cite{bib-VAXTAP}
\par
The package is installed by running a command file that can
be generated by the following PATCHY run:
\begin{XMP}
$YPATCHY CERN:[PRO.PAM]VAXTAP.PAM INSTALL.COM :GO
+USE,INSTALL,T=EXE.
+PAM.
+QUIT
\end{XMP}
\par
Once this command file has been extracted, installation of the package
proceeds by typing 
\begin{XMP}
@INSTALL
\end{XMP}
and answering the questions. If VAXTAP is to be installed on a system
without access to the HEPVM Tape Management System, as is likely
to be the case when installing it outside CERN, answer NO to the first
question. Answering A or ALL to the second question will cause
the installation to complete without any further dialogue.
\par
When the command file completes, the file SETUP\_STARTUP.COM should
be edited and tailored.
\begin{XMP}
$ !---------------------------------------------------------------------------*
$ !
$ !      Startup command file for SETUP/STAGE/LABELDUMP
$ !      Modify logical name definitions as required for your node.
$ !---------------------------------------------------------------------------*
$ !
$ !
$ !
$ !     Create lnm table for SETUP information ...
$       create/name_table/parent=lnm$system_table/prot=w:wred lnm$setup
$ !
$ !      Define directory for .EXE files
$      define/system setup_exe cern_root:[exe]
$ !
$ !      Allow usage of tapes interactively
$ !
$      define/system setup_enabled "INTERACTIVE"
$ !
$ !      Disallow specific users from using tapes (useful to stop troublemakers)
$ !
$ !      define/system setup_notapes "DECNET,CERNET"
$ !
$ !      Allow tapes in these batch queues
$ !
$!      define/system setup_queues "SYS$TAPES"
$!      define/system setup_queues "SYS$BATCH,SYS$TAPES"
$!      define/system setup_queues "*" ! all queues
$ !
$ !      Set up lists of available device types
$ !
$      define/system setup_tk50s "VSDD18$MKA700:"
$      define/system setup_8200s "UXDDB1$MUB0:"
$      define/system setup_exabytes setup_8200s ! Can also have aliases...
$ !
$ !      Allow tape staging
$ !
$      define/system stage_tapes "YES"
$ !      Must also ensure that DISK$STAGE exists...
\end{XMP}
\par
See also the description of the FMEDIA routine and the MEDIA shell
command for information on configuring generic device names in
FATMEN. The generic device names used by VAXTAP must match those
used by FATMEN. Thus, if the generic device type for a given
medium is set to DAT, the logical name SETUP\_DATS must point
to a list of valid device names.
\par
See also the installation instructions in PATCH DOC of the VAXTAP
pam file.
\Filename{H1Fatmeninstallation-VM-service-machines}
\chapter{The VM FATMEN service machines}
\par
These machines should normally be running
in disconnected mode and autologged
at system startup time. These machines run a FORTRAN program which
calls an EXEC to issue WAKEUP upon the arrival of new RDR files.
These are then read in, processed, then WAKEUP is called again.
\Filename{H2Fatmeninstallation-Setting-up-new-service-machine}
\section{Setting up a new service machine}
\par
The service machine should have the name FM{\bf throng}, e.g
\index{throng}
{\bf throng}
\Rind{FMALEPH}, \Rind{FMCHARM2}, \Rind{FMCPLEAR}, \Rind{FMDELPHI}, etc.
\footnote{The FATMEN servers may have any names, provided
that suitable NAMES file entries are established.}
The account created requires a 191 disk of sufficient size
to maintain the expected file catalogue information. Some 800 bytes
are required per file catalogue entry, thus for a catalogue containing
15000 files, about 20 cylinders are required. A 193 disk is required for
maintaining journal information.
This disk should be at least 5 cylinders.
\begin{table}[h]
\caption{Mini-disks required for FATMEN service machines}
\begin{center}\begin{tabular}{|l|p{.8\linewidth}|}
\hline
191    & Disk for FATMEN RZ file, log file and \Lit{PROFILE EXEC}.
         The profile exec contains only one line - \Lit{'EXEC FATPROF'}
         20 cylinders is normally sufficient for an initial allocation for this
         disk.  \\
192    & Link to the disk containing the server code and EXECs.
         At CERN, these are kept on the 191 disk of userid FATMEN.\\
193    &  Disk for keeping journal files. About 5 cylinders are normally
         required. These files are created automatically by the server
          following each update and are named \Lit{FATyyjjj FZhhmmss}, 
         e.g. \Lit{FAT90001 FZ120000} for a file created at 
         12:00:00 on January 1st, 1990.\\
\hline
\end{tabular} \end{center}\end{table}
\par
On CERNVM, the FATMEN service machines are monitored and controlled
by the FATONE machine. Any new service machines must be registered
by running the FATMEN exec on the 191 disk of this machine, as shown
below.
\begin{XMPt}{Modifying the list of FATMEN servers}
defaults list fatmen       /* list current list of FATMEN servers */
FMALEPH FMCNDIVR FMDELPHI FMCHARM2 FML3 FMOPAL FMCPLEAR CPDELPHI
defaults set fatmen fmep03 /* add new FATMEN server */
fatmen (edit               /* edit list of FATMEN servers */
\end{XMPt}
\Filename{H2Fatmeninstallation-Generating-FATMEN-EXECs}
\section{Generating the FATMEN EXECs}
\par
The EXECs used by FATMEN can be generated using the following PATCHY
cradle:
\begin{XMP}
+EXE.
+ASM, 21,R=!./*BEGIN ! EXEC */
+USE,REXX.
+PAM,11.
+QUIT.
\end{XMP}
\par
The resultant file should be processed using the \Lit{SPLITFIL} exec
as shown below:
\begin{XMP}
\Ucom{splitfil fatrexx rexx}
-----> Split FATREXX REXX A1 into pieces
Generate FM EXEC A
Generate FATSERV EXEC A
Generate FAT2CERN EXEC A
Generate FAT4WARD EXEC A
Generate FATJOURN EXEC A
Generate FATUSE EXEC A
Generate FATLOG EXEC A
Generate PURGE EXEC A
\end{XMP}

These EXECs should reside on the FATMEN 191 disk.
\Filename{H2Fatmeninstallation-Monitoring-FATMEN-servers}
\section{Monitoring the FATMEN servers}

{\bf Privileged users} may send the FATMEN servers
management commands to monitor their progress, check that they are active
etc. Examples of use are:
\begin{XMP}
Jamie@Cernvm;
\Ucom{TELL FMDDDIVR QSPOOL}
12:44:53  * MSG FROM FMDDDIVR: I HAVE 0 FILES IN MY RDR
Jamie@Cernvm;
\Ucom{TELL FMDDDIVR HELLO}
12:44:58  * MSG FROM FMDDDIVR: HELLO AND HOW ARE YOU TODAY?
Jamie@Cernvm;
* See how much disk space CHARM2 service machine has used
\Ucom{TELL FMCHARM2 QDISK A}
12:45:19 * MSG FROM FMCHARM2: LABEL  VDEV M STAT CYL TYPE BLKSIZE FILES  BLKS USED-(%) BLKS LEFT BLK TOTAL
12:45:19 * MSG FROM FMCHARM2: CHARM2 191  A R/W   20 3380    4096     6       35-01         2961      3000
\end{XMP}
\begin{table}[h]
\caption{List of commands currently supported by the FATMEN service machine}
\begin{center} \begin{tabular} {|>{\tt}l|p{.8\linewidth}|}
\hline
HELLO    & Check whether server is active.
         If the server does not respond immediately, it means that it
         is either down, or processing an update. As multiple updates
         can be grouped together, it can happen that the server does not
         respond for a considerable period of time to a \Lit{HELLO} message. \\
HELP     & Displays this list of commands. \\
STOP     & Stop the server, but does not log it off. \\
QDISK    & Issue a \Lit{QUERY DISK} command and send the output back
         to the originator of the message. \\
QSPOOL   & Return the number of RDR files that the service
         machine currently has in its reader. \\
DROP     & Cause the server to call the \Lit{DROP}
         exec to drop the specified link. \\
GIME     & Cause an \Lit{EXEC GIME} to be issued.\\
CLOSE    & Close the console log and spool it to the owner of
         the service machine. \\
NEWLOG   & Cause the server to open a new log file. \\
PURGE    & Purge (delete) journal files. May be sent automatically
          whenever a backup of the FATMEN RZ file has been taken.\\
*FATLOG* & This command is issued automatically from the FATMEN user code to log
           FATMEN activity.\\
LOGOFF   & Shutdown the service machines prior to system shutdown or 
          other scheduled interruption, such as ORACLE backup.\\
\hline
\end{tabular}\end{center}\end{table}
\subsection{Names file entries for the FATMEN Servers}
\par
Each FATMEN server should have at least two entries in their NAMES file.
Those on CERNVM may also have a third, FATSERVERS, which points to
the list of remote servers who should automatically receive updates
from CERNVM.
\par
The two entries that are always required are FATOWNERS and FATOPERATORS.
At the present time, usernames in either of these lists are allowed
to issue commands in the above table, and receive SMSGs when the servers
stop.
\begin{XMPt}{Example of \Lit{NAMES} file for the FATMEN server}
:nick.FATOPERATORS                :list.fatop1 fatop2
:nick.FATOP1                      :userid.console
                                  :node.cernvm
:nick.FATOP2                      :userid.opsutil
                                  :node.cernvm
:nick.FATOWN1                     :userid.fatone
                                  :node.cernvm
:nick.FATOWN2                     :userid.wojcik
                                  :node.frcpn11
:nick.FATOWNERS                   :list.fatown1 fatown2 jamie
\end{XMPt}

\Filename{H2Fatmeninstallation-Generating-ORACLE-tables}
\section{Generating the ORACLE tables}
\par
\index{group}
\index{ORACLE}
The ORACLE tables for the new group are created by editting the
SQL statements below, replacing {\bf throng}
by the group name in question. Then, type
\Lit{SQLPLUS user/password @file}, where \Lit{file} is
the name of the file containing these statements.
These SQL commands are in the patch \Lit{FATSQL} on the FATMEN pamfile.
 
\begin{Fighere}
\caption{Creation of the ORACLE table for a new throng}
\label{FORATAB}
\begin{minipage}[t]{.494\linewidth}
\begin{XMP}
INSERT INTO Fatmen VALUES ('CERN', 'throng')
/
REM ***   SPECIFIC TABLES FOR throng
REM ***   Note: 1) CHAR(240) => Oracle V5.1
REM ***         2) TMS later substitutes Volumes_
 
CREATE TABLE GNames_throng ( GName CHAR(240) NOT NULL,
                             GN# NUMBER NOT NULL)
/
CREATE TABLE Files_throng ( File# NUMBER NOT NULL,
                            GN# NUMBER NOT NULL,
                            Copylevel NUMBER(2) NOT NULL,
                            Location NUMBER NOT NULL,
                            Hostname CHAR(8) NOT NULL,
                            Fullname CHAR(240) NOT NULL,
                            Hosttype CHAR(16),
                            Opersys CHAR(12),
                            Fileformat CHAR(4) NOT NULL,
                            Userformat CHAR(4),
                            Startrec# NUMBER,
                            Endrec# NUMBER,
                            Startblk# NUMBER,
                            Endblk# NUMBER,
                            Recformat CHAR(4),
                            Reclength NUMBER,
                            Blklength NUMBER,
                            Creation DATE,
                            Catalogation DATE,
                            Lastaccess DATE,
                            Active CHAR(1) NOT NULL,
                            Creatorname CHAR(8),
                            Creatoraccount CHAR(8),
                            Creatornode CHAR(8),
                            Creatorjob CHAR(8),
                            Protection NUMBER(2),
\end{XMP}
\end{minipage}\hfill
\begin{minipage}[t]{.494\linewidth}
\begin{XMP}
                            Userword0 NUMBER,
                            Userword1 NUMBER,
                            Userword2 NUMBER,
                            Userword3 NUMBER,
                            Userword4 NUMBER,
                            Userword5 NUMBER,
                            Userword6 NUMBER,
                            Userword7 NUMBER,
                            Userword8 NUMBER,
                            Userword9 NUMBER,
                            Sysword0 NUMBER,
                            Sysword1 NUMBER,
                            Sysword2 NUMBER,
                            Sysword3 NUMBER,
                            Sysword4 NUMBER,
                            Sysword5 NUMBER,
                            Sysword6 NUMBER,
                            Sysword7 NUMBER,
                            Sysword8 NUMBER,
                            Sysword9 NUMBER,
                            Comments CHAR(80),
                            Mediatype CHAR(1))
/
CREATE TABLE FXV_throng ( File# NUMBER NOT NULL,
                          Fileseq# NUMBER NOT NULL,
                          Vol# NUMBER NOT NULL,
                          Volseq# NUMBER)
/
CREATE TABLE Volumes_throng ( Vol# NUMBER NOT NULL,
                              VSN CHAR(6) NOT NULL,
                              VID CHAR(6) NOT NULL,
                              VIDprefix NUMBER,
                              Density NUMBER)
/
\end{XMP}
\end{minipage}
\end{Fighere}
\Filename{H2Fatmeninstallation-Generating-SQLDS-tables}
\section{Generating the SQL/DS tables}
\index{group}
\index{SQL/DS}

The SQL/DS tables for the new group are created by editting the
SQL statements below, replacing {\bf throng}
by the group name in question.

\begin{Fighere}
\caption{Creation of the SQL/DS tables for a new throng}
\label{FSQLTAB}
\begin{minipage}[t]{.498\linewidth}
\begin{XMP}
INSERT INTO Fatmen VALUES ('CERN', 'throng')
 
CREATE TABLE GNames_throng ( GName CHAR(240) NOT NULL,   -
                             GN# INTEGER NOT NULL)
CREATE TABLE Files_throng ( File# INTEGER NOT NULL,      -
                            GN# INTEGER NOT NULL,        -
                            Copylevel INTEGER NOT NULL,  -
                            Location INTEGER NOT NULL,   -
                            Hostname CHAR(8) NOT NULL,   -
                            Fullname CHAR(240) NOT NULL, -
                            Hosttype CHAR(16),           -
                            Opersys CHAR(12),            -
                            Fileformat CHAR(4) NOT NULL, -
                            Userformat CHAR(4),          -
                            Startrec# INTEGER,           -
                            Endrec# INTEGER,             -
                            Startblk# INTEGER,           -
                            Endblk# INTEGER,             -
                            Recformat CHAR(4),           -
                            Reclength INTEGER,           -
                            Blklength INTEGER,           -
                            Creation DATE,               -
                            Catalogation DATE,           -
                            Lastaccess DATE,             -
                            Active CHAR(1) NOT NULL      -
                            Creatorname CHAR(8),         -
                            Creatoraccount CHAR(8),      -
                            Creatornode CHAR(8),         -
                            Creatorjob CHAR(8),          -
                            Protection INTEGER(2))
\end{XMP}
\end{minipage}\hfill
\begin{minipage}[t]{.498\linewidth}
\begin{XMP}
Alter table files_throng add (Userword0 INTEGER,         -                                  -
                              Userword1 INTEGER,         -
                              Userword2 INTEGER,         -
                              Userword3 INTEGER,         -
                              Userword4 INTEGER,         -
                              Userword5 INTEGER,         -
                              Userword6 INTEGER,         -
                              Userword7 INTEGER,         -
                              Userword8 INTEGER,         -
                              Userword9 INTEGER,         -
                              Sysword0 INTEGER,          -
                              Sysword1 INTEGER,          -
                              Sysword2 INTEGER,          -
                              Sysword3 INTEGER,          -
                              Sysword4 INTEGER,          -
                              Sysword5 INTEGER,          -
                              Sysword6 INTEGER,          -
                              Sysword7 INTEGER,          -
                              Sysword8 INTEGER,          -
                              Sysword9 INTEGER,          -
                              Comments CHAR(80),         -
                              Mediatype CHAR(1))
CREATE TABLE FXV_throng ( File# INTEGER NOT NULL,        -
                          Fileseq# INTEGER NOT NULL,     -
                          Vol# INTEGER NOT NULL,         -
                          Volseq# INTEGER)
CREATE TABLE Volumes_throng ( Vol# INTEGER NOT NULL,     -
                              VSN CHAR(6) NOT NULL,      -
                              VID CHAR(6) NOT NULL,      -
                              VIDprefix INTEGER,         -
                              Density INTEGER)
\end{XMP}
\end{minipage}
\end{Fighere}

\Filename{H1Fatmeninstallation-restoring-RZ-files-from-oracle}
\chapter{Restoring the RZ files from ORACLE or SQL/DS}

Data can be restored from ORACLE or SQL/DS in one of two ways: either by
directly recreating the FATMEN RZ file, or by sending each entry
as an update in FZ format to the RDR of the virtual machine.
The former is useful if the entire file is lost or corrupt, the
latter for recovering individual entries or for sending the
recovered data to a remote system.

\Filename{H2Fatmeninstallation-Recreating-FATMEN-RZfile-directly}
\section{Recreating the FATMEN RZ file directly}

To restore the complete RZ file from ORACLE or SQL/DS,
logon to the service
machine of the group in question and type RESTORE.
This exec currently extracts all
active.\footnote{Active files are those not marked for delete. File entries
marked for delete are be recovered by resetting the active flag to \Lit{Y}.}
information from the ORACLE SQL/DS tables.

\Filename{H2Fatmeninstallation-Extracting-information-from-ORACLE-SQLDS-as-FZupdates}
\section{Extracting information from ORACLE or SQL/DS as FZ updates}

If a small number of entries are to be restored, or the restored
file is to be sent to a remote site, it may be convenient to
send each entry as an FZ file to the server, with disturbing
its normal operation. The program in PATCH FATO2F on the FATMEN
PAM can be used for this purpose. The program reads a group name
followed by a list of generic names from FORTRAN unit 5, i.e. the
terminal or program stack unless overridden by a FILEDEF command.
The generic names input may include a wild-card as shown below.
The comments, delimited by an exclamation mark (!), should not
be included in a real file and are only used here to help explain
the various options. The routine FMUPDT (see on Page~\pageref{FMUPDT})
may be used to group the updates together if required.
\begin{XMPt}{Example of a file for restoring from ORACLE into FZ}
CHARM2                                 ! Initialise for group CHARM2
//CERN/CHARM2/SPECIAL/R4248            ! Recover an individual entry
//CERN/CHARM2/TEST/*                   ! Recover a complete tree
\end{XMPt}
\Filename{H1Fatmeninstallation-fatmen-code}
\chapter{The FATMEN code}
\Filename{H2Fatmeninstallation-Structure-FATMEN-PAM}
\section{Structure of the FATMEN PAM file}
\par
Cradles to generate the FATMEN FORTRAN from the FATMEN pamfile are
maintained on the FATMEN pamfile. In general, it is sufficient to
{\tt +USE} the pilot patch plus the target machine,
e.g. +USE,*FATLIB,IBM. In addition, the switch CZ must be selected
with {\tt +USE,CZ.} for all machines which will access
a central file catalogue using the ZEBRA server.
\index{ZEBRA server}
\index{ZS ZEBRA server}
\begin{table}[h]
\caption{Description of the cradles in the FATMEN PAM file}
\begin{center} \begin{tabular}{|l|p{.8\linewidth}|}
\hline
*FATSQL   &FORTSQL for the CERNVM service machines.
          The routines must be preprocessed using the PCC exec before 
         compilation.\\
*FATUSER &Sample program on the PAM file. \\
*FATO2Z  &Code to restore the RZ files from the ORACLE database. \\
*FATSRV  &Code for the VM service machines. \\
*FMCDF   &Command definition file (CDF) for the FATMEN shell.
         It must be processed with the KUIPC command before 
         compilation.\\
*FMKUIP  &Code for the FATMEN shell.
          To generate the FATMEN shell, the output of FMINT and 
          the CDF file,
          preprocessed using the KUIPC command, must also be included.
          The CDF file is in patch
          FMCDF on the FATMEN pam.\\
*FATLIB  &Code for the FORTRAN callable interface, maintained in FATLIB TXTLIB.
          It is also required by the FATMEN shell.  \\
\hline
\end{tabular} \end{center}\end{table}
\Filename{H2Fatmeninstallation-Installing-FATMEN-software}
\section{Installing the FATMEN software}
\par
Installation of the FATMEN software consists of two steps:
\begin{OL}
\item
Installation of the FATMEN library (part of PACKLIB).
\item
Creation of the FATMEN shell program.
\end{OL}
\subsection{Installation of FATMEN on CERNVM}
\subsubsection{Generating the FATLIB library}
\par
The extraction of the code is made with the following PATCHY cradle:
\begin{XMP}
+EXE.
+USE,QCDE.
+USE,*FATLIB,IBM.
+PAM,11, R=QCDE, T=A.ZEBRA
+PAM,12.       , T=A.FATMEN
+QUIT.
\end{XMP}
followed by the standard procedure to generate the TXTLIB, namely
VFORT asm, EDITLIB asm, TXT GEN FATLIB asm.
\subsubsection{The FATMEN module (for the command line interface)}
\par
The extraction of the code is made with the following PATCHY cradle:
\begin{XMP}
+EXE.
+ASM,31.
+USE,*FMKUIP,IBM.
+USE,FMCDF,T=EXE,DIV.
+PAM,11, T=A.FATMEN
+QUIT.
\end{XMP}
which generates two files: the standard ASM file with the FORTRAN code
and the diverted ASM2 file containing input to the KUIP processor.
Both the ASM code and the output from the KUIP processor must be
compiled and linked to build the FATMEN module.
\par
The following exec has been used at CERN
to perform the complete operation:
\begin{XMP}
/**/
 Address Command
 'EXEC ZPATCHY FATMEN ( CRADLE(FATMEN) ASM(FATMEN) ASM2(FMCDF CDF)'
 
 'KUIPC FMCDF'  /* To generate the FORTRAN equivalent of the CDF file */
 
 'EXEC VFORT FATMEN'
 'EXEC VFORT FMCDF'
 'EXEC VFORT ENDMODU'         /* Dummy routine to "squeeze" the module*/
 'EXEC CERNLIB FATLIB ( LINK'
 
 'LOAD FATMEN FMCDF ( CLEAR NOAUTO'
 'INCLUDE ENDMODU'
 'GENMOD FATMEN ( TO ENDMODU'
 Exit
/*BEGIN ENDMODU FORTRAN */
      BLOCK DATA ENDMODU
      END
\end{XMP}
\subsubsection{\protect\label{HORACLE}Processing the ORACLE routines for the 
FATMEN server}
The ORACLE
FORTSQL routines can be generated using the following cradles:
\begin{XMP}
+EXE.
+ASM, 21,R=!./*BEGIN ! FORTSQL */
+USE,ORACLE.
+USE,*FATSQL.
+PAM,11.
+QUIT.
\end{XMP}
The output file should then be processed by the SPLITFIL command,
which will create a separate file for each routine.
These can then be pre-processed by the PCC exec as follows:
\begin{XMP}
splitfile fatsql fortsql
-----> Split FATSQL FORTSQL A1 into pieces
Generate BLANKDEK FORTSQL A
Generate FMLOGI FORTSQL A
Generate FODEL FORTSQL A
Generate FOGET FORTSQL A
Generate FOPUT FORTSQL A
FAT3@cernvm;
pcc iname=fmlogi host=fortran
 
ORACLE Precompiler: Version 1.2.13.10 - Production on Thu Feb 22 17:32:04 1990
 
Copyright (c) 1987, Oracle Corporation, California, USA.  All rights reserved.
 
 
Precompiling FMLOGI.FORTSQL
FAT3@cernvm;
vfort fmlogi
VFORT compilation options in effect are: CHARLEN(32767) FLAG(W) GOSTMT OPT(2) 
MAP NOSDUMP.
File 'FMLOGI FORTRAN A1' will be processed.
VS FORTRAN VERSION 2 ENTERED.  17:32:07
 
**FMLOGI** END OF COMPILATION 1 ******
 
VS FORTRAN VERSION 2 EXITED.   17:32:10
 
FAT3@cernvm;
\end{XMP}
\subsubsection{Generating the code for the FATMEN server}
This code can be generated with the following cradle:
\begin{XMP}
+USE,IBM,*SQL,*FATSRV.
+EXE.
+PAM,11, R=QCDE, T=A.  ZEBRA PAM *
+PAM,12,  T=A. FATMEN PAM *
+QUIT.
\end{XMP}
The server module can then be built as follows:
\begin{XMP}
CERNLIB V5ORAFIX V5ORACLE FATLIB
LOAD FATSRV OSDDAT OSDU2OS (CLEAR NOMAP NODUP NOAUTO)
GENMOD FATSRV
\end{XMP}
\subsubsection{Generating the FATMEN server for remote VM systems}
This is performed in the same way as on CERNVM with the following
exception:
\begin{UL}
\item
If neither ORACLE or SQL/DS are to be used, the \begin{XMP}+USE,*SQL\end{XMP}
statement should be omitted.
\end{UL}
If ORACLE is to be used, the procedure for generating the FORTSQL
routines is described on Page~\pageref{HORACLE}
should be followed. If SQL/DS is to be used, the procedure
below should be followed.
\subsubsection{\protect\label{HSQLDS}Processing the SQLDS routines for the 
FATMEN server}
The SQL/DS
FORSQL routines can be generated using the following cradles:
\begin{XMP}
+EXE.
+ASM, 21,R=!./*BEGIN ! FORSQL */
+USE,SQLDS.
+USE,*FATSQL.
+PAM,11.
+QUIT.
\end{XMP}
The output file should then be processed by the SPLITFILE command,
which will create a separate file for each routine.
These can then be pre-processed by the SQLIFY exec as follows:
\begin{XMP}
splitfile fatsql forsql
-----> Split FATSQL FORSQL A1 into pieces
Generate BLANKDEK FORSQL A
Generate FMLOGI FORSQL A
Generate FODEL FORSQL A
Generate FOGET FORSQL A
Generate FOPUT FORSQL A
FAT3@Cernvm;
sqluser
SQL195 ( 01A8  R ) RR
FAT3@Cernvm;
sqlify fmlogi
ARI0717I START SQLPREP EXEC: 02/23/90 10:27:47 SET
ARI0320I THE DEFAULT DATABASE NAME IS SQLDBA.
ARI0663I FILEDEFS IN EFFECT ARE:
SYSIN    DISK     FMLOGI   FORSQL   A1
SYSPRINT DISK     FMLOGI   LISTPREP A1
SYSPUNCH DISK     FMLOGI   FORTRAN  A1
ARISQLLD DISK     ARISQLLD LOADLIB  R1
ARI0713I PREPROCESSOR ARIPRPF CALLED WITH THE FOLLOWING PARAMETERS:
........ PREP=FMLOGI, NOCHECK,ISOLATION(CS)
ARI0710E ERROR(S) OCCURRED DURING SQLPREP EXEC PROCESSING.
ARI0796I END SQLPREP EXEC: 02/23/90 10:27:48 SET
FAT3@Cernvm(00008);
FAT3@Cernvm;
vfort fmlogi
CRNVFT030I VFORT compilation options in effect are: CHARLEN(32767) FLAG(W) 
GOSTMT OPT(2) MAP NOSDUMP NOPRINT.
CRNVFT000I File 'FMLOGI FORTRAN A1' will be processed.
VS FORTRAN VERSION 2 ENTERED.  10:27:55
 
**FMLOGI** END OF COMPILATION 1 ******
 
**SQLINT** END OF COMPILATION 2 ******
 
VS FORTRAN VERSION 2 EXITED.   10:27:55
 
/**********************************************************************/
/*                                                                    */
/*  Title  : Invoke SQL preprocessor                                  */
/*  ======                                                            */
/*                                                                    */
/*  Format : SQLIFY   fn <ft>                                         */
/*  ======               $ASMSQL                                      */
/*                                                                    */
/*  Author : J. Wood, Systems Group, CCD, RAL, 10/04/86               */
/*  ======                                                            */
/*                                                                    */
/**********************************************************************/
Address Command
Signal on HALT ; buf=0
 
Arg fn ft fm . '(' 'USERID' user pwd .
 
If user='' Then foruser=''
   Else Do
      user='STRIP'(user)
      foruser='USERID='user'/'pwd','
      End   /* Else Do */
 
 
ret=0
 
If fm = '' Then fm = 'A'
 
Select
   When fn='' Then Do
      Say 'Format is: SQLIFY file_name <file_type>'
      Exit 4
      End   /* Do When fn='' */
   When Abbrev('PLISQL',ft,1) Then ft='PLISQL'
   When Abbrev('ASMSQL',ft,1) Then ft='ASMSQL'
   When Abbrev('FORSQL',ft,1) Then ft='FORSQL'
   Otherwise Do
      ft='*'
      'MAKEBUF' ; buf=rc
      'LFILE' fn ft '* ( FIFO'
      n=Queued()
      reply=1
      j=0
      type.1='PLISQL'
      mode.1='A'
      typlist='/PLISQL/ASMSQL/FORSQL'
      Do i=1 To n
         Pull fn1 ft1 fm1 .
         loctyp='/'||Strip(ft1)
         If Pos(loctyp,typlist)>0 Then Do
            Do k=1 To j
               If ft1=type.k & fm1=mode.k Then Iterate i
               End   /* Do k=1 ... */
            j=j+1
            type.j=ft1
            mode.j=fm1
            End   /* Do If Pos ... */
         End   /* Do n */
 
      If j>1 Then Do k=1 While 1
         Say 'There is more than one file of filename' fn
         Say 'Select by number:'
         Do i=1 To j
            Say i':' fn type.i mode.i
            End   /* Do i= ... */
         Pull reply
         Select
            When ^Datatype(reply,'W') Then Iterate k
            When reply >j | reply < 1 Then Iterate k
            Otherwise Leave k
            End   /* Select */
         End   /* Do k=1 ... */
 
      ft=type.reply
      fm=mode.reply
      End   /* Do Otherwise */
   End   /* Select */
 
'STATE' fn ft fm
If rc^=0 Then Do
   Say fn ft fm 'does not exist'
   ret=rc
   Signal HALT
   End   /* Do If rc^=0 */
rc='CPUSH'('PRT')
'CP SPOOL 00E TO' Userid()
 
lang = left(ft,3)
if ft = 'FORSQL' then lang = 'FORTRAN'
'EXEC SQLPREP' lang 'PP(PREP='fn','foruser,
                 'NOCHECK,ISOLATION(CS))',
                 'SYSIN('fn ft fm')'
ret=rc
rc='CPOP'('PRT')
HALT:
If buf^=0 Then 'DROPBUF' buf
Exit ret
\end{XMP}
\Filename{H2Fatmeninstallation-Tailoring-FATMEN-shell}
\section{Tailoring the FATMEN shell}
\index{KUIP}
\index{shell}
\index{Tailoring the FATMEN shell}
\index{Modifying the FATMEN shell}
\par
The FATMEN shell may be tailored by
\begin{OL}
\item
The use of KUIP macros
\item
Modifying the FATMEN CDF file, e.g. adding extra commands
\end{OL}
\subsection{KUIP macros}
\par
KUIP macros consist of files containing FATMEN or KUIP system commands.
A trivial example might contain commands such as
\begin{XMPt}{Example macro CDOPAL KUMAC}
MESS 'Initialise FATMEN for OPAL'
INIT OPAL
MESS 'Change directory to PROD/PASS3/FILT/P1R1039L044'
CD PROD/PASS3/FILT/P1R1039L044
MESS 'Count the number of files in this directory'
FC
\end{XMPt}

\par
The output of the above macro is given below:
\begin{XMPt}{Result of executing CDOPAL KUMAC}
FAT3@cernvm;
fm
Type INIT to initialise FATMEN>
exec cdopal
 
FMINIT.  Initialisation of FATMEN package
FATMEN   1.05  900312 13.00  CERN PROGRAM LIBRARY FATMEN=Q123
         This version created on      900312  at        1600
Linked to FMOPAL               mode Z
FAOPEN : for FARZ on Unit    1 opened File CERN FATRZ
Current Working Directory = //CERN/OPAL
Current Working Directory = //CERN/OPAL/PROD/PASS3/FILT/P1R1039L044
Files:    5
FM>
\end{XMPt}

Further details can be found in the KUIP Long writeup<BIBREF REFID=KUIP>.
\subsection{Adding commands to the FATMEN shell}
\par
Extra commands maybe added to the FATMEN shell by
\begin{OL}
\item
Creating a Command Definition File (CDF)
\item
Merging this with the FATMEN CDF file (Patch FMCDF on the FATMEN PAM)
\item
Processing the resultant CDF file using the command KUIPC
\item
Compiling the FORTRAN file produced by the KUIPC command
\item
Linking with the FATMEN library
\end{OL}
\par
For example, we could add the command XLS which would
\begin{OL}
\item
Accept a generic name which could contain wild cards at any point
\item
Find all matching files
\item
Display the file entries
\end{OL}
(In fact, the ls command has been modified to do just this!)
\begin{XMPt}{CDF file for XLS command}
>COMMAND XLS
>GUIDENCE
Use the XLS command to perform an extended LS command.
Syntax: XLS path options
Options:
  A - list all attributes, except DZSHOW (option Z).
  C - display comment field associated with file
  F - list file attributes, such as start/end record and block
  G - list the full generic name of each file
  K - list keys associated with this file (copy level, media type, location)
  L - list logical attributes, such as FATMEN file format
      (ZEBRA exchange etc.)
  M - list media attributes, such as VSN, VID, file sequence number for tape
      files, host type and operating system for disk files.
  N - lists dataset name on disk/tape of this file
  O - list owner, node and job of creator etc.
  P - list physical attributes, such as record format etc.
  S - lists security details of this file (protection)
  T - list date and time of creation, last access etc.
  U - list user words.
  Z - dump ZEBRA bank with DZSHOW.
>ACTION FMXLSC
>PARAMETERS
+
FILE 'File or pathname' C D='CURRENT_DIRECTORY'
OPTN 'Options'          C D=' '
\end{XMPt}
\par
Our action routine, FMXLSC, might look like the following:
\begin{XMPt}{Example action routine, FMXLSC}
      SUBROUTINE FMXLSC
      PARAMETER     (MAXFIL=10000)
      PARAMETER     (LKEYFA=10)
      DIMENSION     KEYS(LKEYFA,MAXFIL)
      CHARACTER*255 FILES(MAXFIL)
      CHARACTER*255 PATH
      CHARACTER*25  CHOPT
*
      CALL KUGETC(PATH,LPATH)
      CALL KUGETC(CHOPT,LCHOPT)
*
      CALL FMLIST(PATH(1:LPATH),FILES,KEYS,NFOUND,MAXFIL,IRC)
*
      DO 10 I=1,NFOUND
10    CALL FMSHOW(FILES(I),LBANK,CHOPT(1:LCHOPT),IRC)
*
      END
\end{XMPt}

\Filename{H1Fatmen-Monitoring-information}
\chapter{Monitoring information}

\Filename{H2Fatmen-Monitoring-introduction}
\section{Introduction}

{\tt FATMEN} maintains three types of monitoring information:

\begin{OL}
\item
Information recorded in the {\tt FATMEN} catalogue
\item
Information logged per file access
\item
Information logged per session
\end{OL}

\subsection{Monitoring information in the {\tt FATMEN} catalogue}

The monitoring information that is stored in the {\tt FATMEN} catalogued
consists of the {\tt file size}, the {\tt date} and {\tt time} of
last access and the {\tt number of file accesses}.

This information is stored per entry at the offsets {\tt MFSZFA},
{\tt MLATFA} and {\tt MUSCFA}. See page~\pageref{BANK-OFFSETS}
for a description of the bank offsets.

This information can be histogrammed with the example program
{\tt FATPLOT}, included in the {\tt FATMEN} source file
in patch {\tt EXAMPLE}, deck {\tt FATPLOT} and reproduced below.

\begin{XMPt}{Histogramming monitoring information}
*=======================================================================
*
*  Example of using HBOOK to plot various FATMEN catalogue values.
*  This program histograms the file size, number of days since last
*  access, medium type etc.
*=======================================================================
      PARAMETER (LURCOR=200000)
      COMMON/CRZT/IXSTOR,IXDIV,IFENCE(2),LEV,LEVIN,BLVECT(LURCOR)
      DIMENSION    LQ(999),IQ(999),Q(999)
      EQUIVALENCE (IQ(1),Q(1),LQ(9)),(LQ(1),LEV)
*
      COMMON /USRLNK/LUSRK1,LUSRBK,LUSRLS
*
      COMMON /QUEST/IQUEST(100)
      CHARACTER*8   THRONG
+CDE,FATPARA.
+CDE,FATBUG.
      EXTERNAL      UROUT
*
*
*     Initialise ZEBRA
*
      CALL MZEBRA(-3)
      CALL MZSTOR(IXSTOR,'/CRZT/','Q',IFENCE,LEV,BLVECT(1),BLVECT(1),
     +            BLVECT(5000),BLVECT(LURCOR))
      CALL MZLOGL(IXSTOR,-3)
 
*
* *** Define user division and link area like:
*
      CALL MZDIV  (IXSTOR, IXDIV, 'USERS', 50000, LURCOR, 'L')
      CALL MZLINK (IXSTOR, '/USRLNK/', LUSRK1, LUSRLS, LUSRK1)
*
*     Units for FATMEN RZ/FZ files
*
      LUNRZ = 1
      LUNFZ = 2
      CALL GETENVF('THRONG',THRONG)
      LTH = LENOCC(THRONG)
*
*     Initialise FATMEN
*
      CALL FMINIT(IXSTOR,LUNRZ,LUNFZ,'//CERN/'//THRONG(1:LTH),IRC)
      CALL FMLOGL(0)
*
*     Initialise HBOOK
*
      CALL HLIMIT(-20000)
*
*     Book histograms
*
      CALL HBOOK1(1,'File Size (MB)',50,0.,200.,0.)
      CALL HBOOK1(2,'Number of accesses',50,0.,50.,0.)
      CALL HBOOK1(3,'Number days since last access',50,0.,300.,0.)
      CALL HBOOK1(4,'Number days since catalogued',50,0.,300.,0.)
      CALL HBOOK1(5,'Number days since created',50,0.,300.,0.)
      CALL HBOOK1(6,'Medium',5,0.,5.,0.)
      CALL HIDOPT(0,'BLAC')
*
*     Loop over all files
*
      CALL FMLOOP('//CERN/*/*',-1,UROUT,IRC)
*
*     Print and store the histograms
*
      CALL HPRINT(0)
      CALL HRPUT(0,'FATTUPLE.'//THRONG(1:LTH),'N')
*
*     Terminate cleanly
*
      CALL FMEND(IRC)
      END
 
      SUBROUTINE UROUT(PATH,KEYS,IRC)
+CDE,FATPARA.
      PARAMETER (LURCOR=200000)
      COMMON/CRZT/IXSTOR,IXDIV,IFENCE(2),LEV,LEVIN,BLVECT(LURCOR)
      DIMENSION    LQ(999),IQ(999),Q(999)
      EQUIVALENCE (IQ(1),Q(1),LQ(9)),(LQ(1),LEV)
      CHARACTER*(*) PATH
      PARAMETER     (LKEYFA=10)
      DIMENSION     KEYS(LKEYFA)
      DIMENSION     NDAYS(3)
      COMMON/QUEST/IQUEST(100)
      IRC   = 0
      LBANK = 0
      LP    = LENOCC(PATH)
      CALL FMGETK(PATH(1:LP),LBANK,KEYS,IRC)
*
*     Fill histograms
*
      IF(IQ(LBANK+MFSZFA).NE.0)
     +CALL HFILL(1,FLOAT(IQ(LBANK+MFSZFA)),0.,1.)
      IF(IQ(LBANK+MUSCFA).NE.0)
     +CALL HFILL(2,FLOAT(IQ(LBANK+MUSCFA)),0.,1.)
      CALL FMDAYS(PATH(1:LP),LBANK,KEYS,NDAYS,' ',IRC)
      CALL HFILL(3,FLOAT(NDAYS(3)),0.,1.)
      CALL HFILL(4,FLOAT(NDAYS(2)),0.,1.)
      CALL HFILL(5,FLOAT(NDAYS(1)),0.,1.)
      CALL HFILL(6,FLOAT(IQ(LBANK+MMTPFA)),0.,1.)
      CALL MZDROP(IXSTOR,LBANK,' ')
      END
\end{XMPt}

\Filename{H2Fatmen-Monitoring-fileaccess}
\section{Monitoring information logged per file access}

For each call to \Rind{FMOPEN}, an update is sent to
the {\tt FATMEN} server to update the last access date
and time, use count etc. as described above.
This logging record also records additional information,
which is stored in the logs of the servers.

This information consists of the following quantities:

\begin{DLtt}{1234567890}
\item[CHFNFA]The fully qualified name of the actual file that
was accessed. 
\item[IHOWFA]A bit pattern indicating how the file was accessed.
\begin{DLtt}{1234567890}
\item[JLOCFA=1]Local disk (standard file system)
\item[JSFSFA=2]VM shared file system
\item[JMSCFA=3]MSCP (VAXcluster)
\item[JAFSFA=4]Andrew file system
\item[JOSFFA=5]OSF distributed file system
\item[JDFSFA=6]DEC DFS
\item[JNFSFA=7]Sun NFS
\item[JDECFA=8]DECnet
\item[JCSPFA=9]CSPACK server
\item[JFPKFA=10]FPACK server
\item[JRFIFA=11]RFIO
\item[JSTGFA=31]Stage required
\item[JTPMFA=32]TPMNT (=not staged)
\end{DLtt}
\item[ITIMFA]The time in seconds spent in the routine 
\Rind{FMOPEN} (including the time for {\tt STAGE} operations etc.)
\end{DLtt}

In addition, the username and nodename from which the file
was accessed is logged, together with the generic name
and {\tt FATMEN} keys (location code, media type and data representation).
These permit further information to be extracted from the {\tt FATMEN}
catalogue if required.

\Filename{H2Fatmen-Monitoring-session-logging}
\section{Session logging}

At the end of each {\tt FATMEN} session, {\bf provided} that a
call to \Rind{FMEND} is made, a log record is written to
the server. The server ignores this record (apart from
counting the number that it receives, unless installed
with the flag {\tt +USE,FATLOG.}

If installed with this flag, the logging records are
collected into a daily summary for subsequent processing.

The logging records consists of the following blocks:

\begin{OL}
\item
Hollerith block
\begin{DLtt}{1234567}
\item[KFMSYS]FATMEN system (e.g. CERN)
\item[KFMGRP]FATMEN group (e.g. DELPHI)
\item[KFMTIT]Title of FATMEN source file
\item[KFMUSR]User name
\item[KFMHST]Host name
\item[KFMTYP]Host type
\item[KFMOS]Host operating system
\end{DLtt}
\item
MB counts
\begin{DLtt}{1234567}
\item[KFMMBR]Number of MB read
\item[KFMMBW]Number of MB written
\item[KFZMBR]Number of MB read with ZEBRA FZ
\item[KFZMBW]Number of MB written with ZEBRA FZ
\item[KFMMBC]Number of MB copied
\item[KFMMBN]Number of MB copied over the network
\item[KFMMBQ]Number of MB queued for copy (e.g. {\tt CHEOPS})
\end{DLtt}
\item
Dates and times (packed with \Rind{FMPKTM})
\begin{DLtt}{1234567}
\item[KFMIDQ]Date and time of PATCHY installation job
\item[KFMIDS]Date and time of start of FATMEN session
\item[KFMIDE]Date and time of end of FATMEN session
\end{DLtt}
\item
Catalogue modifications
\begin{DLtt}{1234567}
\item[KFMADD]Number of disk entries added (\Rind{FMADDD})
\item[KFMADL]Number of links added (\Rind{FMADDL})
\item[KFMADT]Number of tape entries added (\Rind{FMADDT})
\item[KFMMDR]Number of \Rind{MKDIR} commands
\item[KFMRDR]Number of \Rind{RMDIR} commands
\item[KFMRLN]Number of \Rind{RMLN} commands
\item[KFMRTR]Number of \Rind{RMTREE} commands
\item[KFMRMF]Number of \Rind{RM} commands
\item[KFMCPF]Number of \Rind{CP} commands
\item[KFMMVF]Number of \Rind{MV} commands
\item[KFMMOD]Number of \Rind{MODIFY} commands
\item[KFMTCH]Number of \Rind{TOUCH} commands
\end{DLtt}
\item
File accesses
\begin{DLtt}{1234567}
\item[KFMOPN]Number of files opened, e.g. by \Rind{FMOPEN}
\item[KFMCLS]Number of files closed, e.g. by \Rind{FMCLOS}
\item[KFMCPY]Number of files copied, e.g. by \Rind{FMCOPY}
\item[KFMCPQ]Number of files queued for copy, e.g. by \Rind{FMCOPQ}
\item[KFMCPN]Number of files copied over the network, e.g. by \Rind{FMRCOP}
\end{DLtt}
\item
SYSREQ and TMS operations
\begin{DLtt}{1234567}
\item[KFMSRQ]Number of calls to SYSREQ (\Rind{FMSREQ})
\item[KFMQVL]Number of {\tt QVOL} commands (\Rind{FMQVOL})
\item[KFMAVL]Number of volumes allocated (\Rind{FMALLO})
\item[KFMASP]Number of space allocations (\Rind{FMGVOL} or \Rind{FMGVID})
\item[KFMPOL]Number of {\tt TRANSFER} commands (pool operations, e.g. \Rind{FMPOOL}
\item[KFMLCK]Number of volumes locked, e.g. \Rind{FMLOCK}
\item[KFMULK]Number of volumes unlocked, e.g. \Rind{FMULOK}
\item[KFMDTG]Number of tags deleted, e.g. \Rind{FMTAGS}
\item[KFMGTG]Number of tags obtained, e.g. \Rind{FMTAGS}
\item[KFMSTG]Number of tags set, e.g. \Rind{FMTAGS}
\end{DLtt}
\item
Catalogue processing
\begin{DLtt}{1234567}
\item[KFMBNK]Number of banks read from the catalogue (\Rind{FMRZIN})
\item[KFMGET]Number of banks read with default selection (\Rind{FMGET})
\item[KFMGTK]Number of banks user selected (\Rind{FMGETK})
\item[KFMSHW]Number of calls to \Rind{FMSHOW}
\item[KFMSCN]Number of calls to \Rind{FMSCAN}
\item[KFMLOP]Number of calls to \Rind{FMLOOP}
\item[KFMLDR]Number of calls to \Rind{FMLDIR}
\item[KFMLFL]Number of calls to \Rind{FMLFIL}
\item[KFMSRT]Number of calls to \Rind{FMSORT}
\item[KFMRNK]Number of calls to \Rind{FMRANK}
\item[KFMSLK]Number of calls to \Rind{FMSELK}
\item[KFMMTC]Number of calls to \Rind{FMATCH}
\end{DLtt}
\end{OL}
