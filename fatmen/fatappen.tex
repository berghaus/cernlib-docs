%%%%%%%%%%%%%%%%%%%%%%%%%%%%%%%%%%%%%%%%%%%%%%%%%%%%%%%%%%%%%%%%%%%
%                                                                 %
%   FATMEN User Guide and Reference manual                        %
%                                                                 %
%   Fatmen Part 5: Appendices                                     %
%                                                                 %
%   This document needs the following external EPS files:         %
%   none                                                          %
%                                                                 %
%   Editor: Michel Goossens / CN-AS                               %
%   Last Mod.: 19 January 1993 11:50 mg                           %
%                                                                 %
%%%%%%%%%%%%%%%%%%%%%%%%%%%%%%%%%%%%%%%%%%%%%%%%%%%%%%%%%%%%%%%%%%%
\Filename{H1fatmenappendix-fatcat-server}
\chapter{The fatcat server}
\Filename{H2fatmenappendix-overview-fatcat-files-and-directories}
\section{Overview of FATCAT files and directories}

At CERN, a dedicated system is used to host FATMEN servers and
the associated catalogues. This system is currently an IBM RS6000
with node name {\tt fatcat}. No user accounts exist on the machine -
the {\tt /fatmen} file system is {\tt exported} and {\tt NFS-mounted}
on all systems requiring access. At a later date, {\tt OSF-DFS} will
be used for catalogue access.

The {\tt /fatmen} file system contains a subdirectory for each
experiment, as shown below.

\begin{XMPt}{/fatmen directories}
[fatcat] (209) ls -F /fatmen
fmaleph/    fmcplear/   fmna31/     fmatlas/    fmna44/     fmna52/     fmcadd/     
fmdelphi/   fmnomad/    fmcharm2/   fmopal/     fmrd5/      fmchorus/   fmkeops/    
fmrd6/      fml3/       fmsmc/      fmcndiv/       
\end{XMPt}

Each of these directories contains the following files or directories:
\begin{DLtt}{123456789AB}
\item[cern.fatrz]The FATMEN catalogue for the experiment in question.
\item[done]A directory where the server places update files after processing them.
\item[fatserv]The script that runs the server for this experiment.
\item[fatserv.log]The log file from the server process.
\item[fatsrv]The server program (in fact a link to the server program).
\item[fml3.names]A names file {\tt fm}experiment.names, e.g. fml3.names as here.
\item[todo]A directory where updates are written by FATMEN clients (users) and which
is scanned regularly by the server process.
\item[tovm]Updates to be transferred to {\tt CERNVM} are placed here. This
directory is optional, and requires server software on the VM node.
\end{DLtt}

In addition, one may also have the following configuration files:

\begin{DLtt}{123456789012345}
\item[fatmen.loccodes]A list of FATMEN location codes (integers) and associated
text, e.g. 1=CERN Computer Centre.
\item[fatmen.acl]An ACL file determining which users on which nodes may update
the specified subdirectory structure.
\item[fatmen.updates]A file controlling for which users and paths monitoring
information is to be sent to the server.
\item[fatmen.accounts]A list of account aliases.
\end{DLtt}

Furthermore, one needs a subdirectory for each remote server.
In the particular case of {\tt fatcat} there are no remote servers,
put the directories are called {\tt to}node, e.g. {\tt tovxcrna}.
Provided that the names file is correctly configured, all
missing subdirectories are automatically created by the server,
with the exception of {\tt tovm} and {\tt todo}. {\tt todo} must
be given the permission {\tt o+w} to allow all users the possibility
to make catalogue updates. (With {\tt AFS} one could use ACLs to
provide better security).

\Filename{H2fatmenappendix-manage-servers}
\section{Managing the servers}

The FATMEN servers are started automatically at boot-time.
This is done by adding the following line to the file {\tt /etc/inittab}.

\begin{XMP}
fatmen:2:wait:/etc/rc.fatmen > /dev/console 2>&1 # Start Fatmen
\end{XMP}

The file {\tt rc.fatmen} is show below:

\begin{XMPt}{rc.fatmen}
#!/bin/sh
#
#               Start FATMEN servers
#
#
if [ -x /u/jamie/bin/fatstart ]
then
        echo Start FATMEN servers ...
        su - jamie /u/jamie/bin/fatstart 2>&1
fi
\end{XMPt}

Normally, no other operation is required. Should it be necessary to
stop the servers, the following script may be used:

\begin{XMPt}{fatstop}
#!/bin/ksh
#
#   Stop all FATMEN servers that are running and create a
#   file 'restart_fat' in the CWD that can be used to restart them.
#
stop=" "
run=" "
nolog=" "
noscr=" "
b="."
d=`date`
#
#   Ensure that variables are defined...
#
if [ -f restart_fat ]
   echo Remove old restart_fat file...
   then rm -i restart_fat
fi
 
for i in /fatmen/fm*
   do
 
typeset -u fatgrp
typeset -l fatman
fatpath=$i
fatgrp=`basename $i`
fatman=$fatgrp
eval $fatgrp=$fatpath;export $fatgrp
#
# and stop those servers that are running...
#
if [ -x ${i}/fatserv ]
   then
#
# does a log file exist?
#
   if [ -f /fatmen/${fatgrp}.log ]
      then
      echo Log file exists for ${fatgrp} - looking for existing process
      log=${log}${b}${fatgrp}
      pid=`cat /fatmen/${fatgrp}.log | awk '{printf "%s\\n",$13}'`
      if (test $pid)
         then
         echo Looking for server process for $fatgrp
         if(ps -ae  | grep -s $pid )
            then
            echo FATSRV running PID = $pid
            run=${run}${b}${fatgrp}
            echo rm /fatmen/${fatman}/todo/signal.stop >> restart_fat
            echo Server stopped at $d > /fatmen/${fatman}/todo/signal.stop
            else
            echo No existing server found for $fatgrp
            echo Removing old log file...
            rm   /fatmen/${fatgrp}.log
            if [ -f ${i}/todo/signal.stop ]
               then echo signal.stop file found!
               rm ${i}/todo/signal.stop
               echo '(removed)'
            fi
         fi
      fi
   fi
fi
 
done
 
echo
echo Log files found for $log | tr '.' ' '
echo Servers already running for $run | tr '.' ' '
echo fatstart >> restart_fat
if [ -f restart_fat_fat ]
   then chmod +x restart_fat_fat
   echo restart reservers by typing restart_fat
fi
\end{XMPt}

Should a server abend, it can be restarted (after curing the problem) by
rerunning the {\tt fatstart} script that is invoked at boottime.

\Filename{H2fatmenappendix-monitor-servers}
\section{Monitoring the servers}

A simple script may be used to look for active servers and their
CPU usage. The fact that each server runs its 'own' {\tt fatsrv}
module is used by this script, which reduces to three lines:

\begin{XMPt}{fatps}
echo 'FATMEN server                                       Elapsed     CPU time   %CPU'
echo '==============================================================================='
ps -aef -F "args,etime,time,pcpu" | grep "/fatsrv" | sort +2 -r
\end{XMPt}

One may also inspect the log files, e.g. {\tt tail -f fatserv.log}.

\Filename{H1fatmenappendix-catalogue-recovery}
\chapter{Catalogue recovery}

On rare occasions a FATMEN catalogue may become corrupted. 
To recover, one can simply restore the most recent backup
copy and reapply all journal files since the time of the
backup. (Simply copy or move them from the {\tt done}
directory to the {\tt todo} directory. The time stamp
in the filenames will ensure that they are processed
in the correct order).

Alternatively, one can attempt to 'repair' a catalogue.
This may be necessary if the corruption is not noticed
for some time. This can happen if a systematic check of
the catalogue is not made before every backup as corruption
typically affects only a few entries which may well not
be accessed for some time.

Repairing a corrupted catalogue consists of the following
steps:
\begin{OL}
\item
Identifying which directories and/or entries are corrupted
\item
Copying the catalogue skipping these directories and/or entries
\item
Recovering the corrupted information from journal files or
from a backup copy of the catalogue
\end{OL}

\Filename{H2fatmenappendix-find-corrupted-entries}
\section{Finding the corrupted entries}

Normally, the corrupted entries are either reported
by a user (e.g. I cannot access files in this directory),
or are found by the server. Should the server abend,
it will automatically send a mail message to the FATMEN
manager. (Tailor the {\tt fatabend} script as appropriate).
The server log will then show which directory and/or entries
were giving problems. Should this information be unavailable,
one can find the corrupted entries by running the 
program {\tt fatloop2}, as shown below (for a Unix system).

\begin{XMPt}{Running FATLOOP2}

export FMLOGL=1
export FATSYS=CERN
export FATGRP=L3
fatloop2 > fatloop2.log

\end{XMPt}

This program attempts to retrieve each catalogue entry in
turn. Should a directory or catalogue entry be corrupted,
then it will terminate abnormally (via ZFATAL). Thus
it is good practice to run this program on a regular basis,
e.g. before a periodic backup.

An example of the log is shown below.

\begin{XMPt}{Output of the FATLOOP2 program}

 FATMEN system defaulted to CERN
 FATMEN group:  L3

 FMINIT.  Initialisation of FATMEN package
 FATMEN   1.81/07 930203 08:50 CERN PROGRAM LIBRARY FATMEN=Q123
          This version created on      930203  at         852
 FMLOGL. setting log level to           1
 Get: //CERN/L3/CDREMM/CC132563
 Get: //CERN/L3/CDREMM/CC132563
 Get: //CERN/L3/CDREMM/CC132563
 Get: //CERN/L3/CDREMM/CC132564

 ...

 Get: //CERN/L3/PROD/DATA/SDRETT/CC02H8G2
 Get: //CERN/L3/PROD/DATA/SDRETT/CC02H8IU
 Get: //CERN/L3/PROD/DATA/SDRETT/CC02HBJ6
 Get: //CERN/L3/PROD/DATA/SDRETT/CC02HBGE
 Get: //CERN/L3/PROD/DATA/SDRETT/CC02HBJ6
 Get: //CERN/L3/PROD/DATA/SDRETT/CC02HBGE
 Get: //CERN/L3/PROD/DATA/SDRETT/CC02HBLY

 !!!!! ZFATAL called from MZGAR1
              called from FZIMTB

 !!!!! ZFATAL reached from MZGAR1    for Case=  1

          IQUEST(11) = *********         DFE0035F   ��^C_

          Current Store number =  0  (JQDIVI=19)
1ZEBRA SYSTEM Post-Mortem from ZPOSTM.

 /QUEST/
              0            1         3835           47            0            1         1030           10
     1297762113   1378951200   -538967201            0           19    981191424        12309       930127
           1428          180            0         1019          644          645          646          647
            648          649          650          651          652          653          654          655
            656          657         3742         3836          203          204          205          206
       00000000     00000001     00000EFB     0000002F     00000000     00000001     00000406     0000000A
       4D5A4741     52312020     DFE0035F     00000000     00000013     3A7BCB00     00003015     000E314F
       00000594     000000B4     00000000     000003FB     00000284     00000285     00000286     00000287
       00000288     00000289     0000028A     0000028B     0000028C     0000028D     0000028E     0000028F
       00000290     00000291     00000E9E     00000EFC     000000CB     000000CC     000000CD     000000CE

\end{XMPt}

\Filename{H2fatmenappendix-recover-corrupted-entries}
\section{Recovering from corrupted entries}

Once the bad directories or entries have been identified,
one can make a new catalogue using a special version of
the {\tt RTOX/RFRX} tools. Here we simply exclude directories
and/or entries that have been identified as bad.

In the above example, we clearly see that there are problems 
in the directory {\tt //CERN/L3/PROD/DATA/SDRETT}
We can now use the FATMEN shell to localise the corrupted entries.

First, we use the {\tt ls} command to list all files in
this directory.

\begin{XMPt}{Using the ls command to locate corrupted entries}

ls -w output=sdrett.log

\end{XMPt}

The server log has told use that the first bad entry is 
{\tt CC02HBLY}. As the catalogue entries take 163 words
and the record length is 1024 words, we would expect 
7 corrupted entries if an entire record was corrupt.
That means that we should be able to retrieve the entry
{\tt CC02HJ0M} correctly. This we can test using the
{\tt ls} command, e.g.

\begin{XMP}

ls CC02HJ0M -a

\end{XMP}

However, if we attempt to retrieve the previous entry, we
get the following:

\begin{XMP}
FM> ls CC02HJ66 -a
 
 Directory: //CERN/L3/PROD/DATA/SDRETT
 
 FMRZIN. Error in RZIN for directory //CERN/L3/PROD/DATA/SDRETT
 Object:
 Key serial number =   1670 filename = CC02HJ66             data repr. =   0 media type =  2 location code =      1
 
 Generic filename: CC02HJ66
 Data repr.:  0 Media type: 2 Location code:      1 File serial number:   1670
 Device group: CT1     
 Comment: 
 Start record:          0 End record:          0
 Start block :          0 End block :          0
 File format:  user format: 
 VSN:  VID:  FSEQ:    0
 Fileid:      
 Created by:   ACCT:  on node:  by job: 
 RECFM:   LRECL:     0 BLKSIZE:     0 Filesize:     0 Use count:     0
 File protection mask:          00000000
 Date and time of creation:          0    0
 Date and time catalogued:           0    0
 Date and time last accessed:        0    0
 User words:  00000000 00000000 00000000 00000000 00000000
              00000000 00000000 00000000 00000000 00000000

\end{XMP}

We can extract all of the good entries in this directory
by using the {\tt extract} or {\tt touch} commands.

\begin{XMPt}{Extract good entries from a corrupted directory}

FM>ls output=sdrett.log
FM>edit sdrett.log

(delete the lines corresponding to bad entries)
(add the command 'touch' to the beginning of every line)
(save as sdrett.kumac)

FM>update 9999 9999 0
FM>exec sdrett

\end{XMPt}

\Filename{H2fatmenappendix-skipping-bad-directories}
\section{Skipping bad directories}

As described above, we can skip bad directories by modifying
the program {\tt RTOX}.

\begin{XMPt}{Extracting the source for RTOX}
ypatchy /cern/new/src/car/zebra.car r2x.f :go <<!
+use,qcde.
+use,qmuix,ibmrt,*rtox.
+use,p=rz,d=rzcdes.
+use,p=rz,d=rztofz.
+use,p=rz,d=rztof1.
+use,p=rztofrfz,d=blankdek.
+exe.
+pam,11,t=c.
+quit
!
\end{XMPt}

We then modify the routine {\tt RZTOFZ} to exclude the directories
that contain corrupted entries.

\begin{XMPt}{Modifying RTOX}
CDECK  ID>, RZTOFZ. 
      SUBROUTINE RZTOFZ(LUNFZ,CHOPT)
*
************************************************************************
*
*        Copy the CWD tree to a sequential FZ file
*        The FZ file must have been declared with FZOPEN
* Input:
*   LUNFZ   Logical unit number of the FZ sequential access file
*   CHOPT   default save only the highest cycle to LUNFZ
*           'C' save all cycles
*
* Called by <USER>
*
*  Author  : R.Brun DD/US/PD
*  Written : 14.05.86
*  Last mod: 26.06.92 JDS - protect against RZPAFF problems
*
************************************************************************
      PARAMETER      (IQDROP=25, IQMARK=26, IQCRIT=27, IQSYSX=28)
      COMMON /QUEST/ IQUEST(100)
      COMMON /ZVFAUT/IQVID(2),IQVSTA,IQVLOG,IQVTHR(2),IQVREM(2,6)
      COMMON /ZEBQ/  IQFENC(4), LQ(100)
                              DIMENSION    IQ(92),        Q(92)
                              EQUIVALENCE (IQ(1),LQ(9)), (Q(1),IQ(1))
      COMMON /MZCA/  NQSTOR,NQOFFT(16),NQOFFS(16),NQALLO(16), NQIAM
     +,              LQATAB,LQASTO,LQBTIS, LQWKTB,NQWKTB,LQWKFZ
     +,              MQKEYS(3),NQINIT,NQTSYS,NQM99,NQPERM,NQFATA,NQCASE
     +,              NQTRAC,MQTRAC(48)
                                       EQUIVALENCE (KQSP,NQOFFS(1))
      COMMON /MZCB/  JQSTOR,KQT,KQS,  JQDIVI,JQDIVR
     +,              JQKIND,JQMODE,JQDIVN,JQSHAR,JQSHR1,JQSHR2,NQRESV
     +,              LQSTOR,NQFEND,NQSTRU,NQREF,NQLINK,NQMINR,LQ2END
     +,              JQDVLL,JQDVSY,NQLOGL,NQSNAM(6)
                                       DIMENSION    IQCUR(16)
                                       EQUIVALENCE (IQCUR(1),LQSTOR)
      COMMON /MZCC/  LQPSTO,NQPFEN,NQPSTR,NQPREF,NQPLK,NQPMIN,LQP2E
     +,              JQPDVL,JQPDVS,NQPLOG,NQPNAM(6)
     +,              LQSYSS(10), LQSYSR(10), IQTDUM(22)
     +,              LQSTA(21), LQEND(20), NQDMAX(20),IQMODE(20)
     +,              IQKIND(20),IQRCU(20), IQRTO(20), IQRNO(20)
     +,              NQDINI(20),NQDWIP(20),NQDGAU(20),NQDGAF(20)
     +,              NQDPSH(20),NQDRED(20),NQDSIZ(20)
     +,              IQDN1(20), IQDN2(20),      KQFT, LQFSTA(21)
                                       DIMENSION    IQTABV(16)
                                       EQUIVALENCE (IQTABV(1),LQPSTO)
C
      COMMON /RZCL/  LTOP,LRZ0,LCDIR,LRIN,LROUT,LFREE,LUSED,LPURG
     +,              LTEMP,LCORD,LFROM
                   EQUIVALENCE (LQRS,LQSYSS(7))
C
      PARAMETER (NLPATM=100)
      COMMON /RZDIRN/NLCDIR,NLNDIR,NLPAT
      COMMON /RZDIRC/CHCDIR(NLPATM),CHNDIR(NLPATM),CHPAT(NLPATM)
      CHARACTER*16   CHNDIR,    CHCDIR,    CHPAT
C
      COMMON /RZCH/  CHWOLD,CHL
      CHARACTER*255  CHWOLD,CHL
C
      PARAMETER (KUP=5,KPW1=7,KNCH=9,KDATEC=10,KDATEM=11,KQUOTA=12,
     +           KRUSED=13,KWUSED=14,KMEGA=15,KIRIN=17,KIROUT=18,
     +           KRLOUT=19,KIP1=20,KNFREE=22,KNSD=23,KLD=24,KLB=25,
     +           KLS=26,KLK=27,KLF=28,KLC=29,KLE=30,KNKEYS=31,
     +           KNWKEY=32,KKDES=33,KNSIZE=253,KEX=6,KNMAX=100)
C
      CHARACTER*(*) CHOPT
      DIMENSION ISD(NLPATM),NSD(NLPATM),IHDIR(4)
*
*-----------------------------------------------------------------------
*
      IQUEST(1)=0
      IQ1=0
      IF(LQRS.EQ.0)GO TO 99
*
      CALL UOPTC(CHOPT,'C',IOPTC)
      NLPAT0=NLPAT
      DO 5 I=1,NLPAT0
         CHPAT(I)=CHCDIR(I)
   5  CONTINUE
      ITIME=0
      CALL RZCDIR(CHWOLD,'R')
*
*        Garbage collection in user short range divisions
*        in primary store
*
      CALL MZGARB(21,0)
*
*            Write general header
*
      IHDIR(1)=12345
      IHDIR(2)=NLPAT0
      CALL FZOUT(LUNFZ,JQPDVS,0,1,'Z',1,2,IHDIR)
      IF(IQUEST(1).NE.0)THEN
         IQ1=IQUEST(1)
         GO TO 90
      ENDIF
*
*            Set CWD to the current level
*
  10  CONTINUE
      IF(ITIME.NE.0)THEN
         CALL RZPAFF(CHPAT,NLPAT,CHL)
      IF(IQUEST(1).NE.0)THEN
         IQ1=IQUEST(1)
         NLPAT=NLPAT-1
         GO TO 20
      ENDIF
         CALL RZCDIR(CHL,' ')
      ENDIF
      ISD(NLPAT)=0
      NSD(NLPAT)=IQ(KQSP+LCDIR+KNSD)
*
*     Skip bad directories
*
      if(chl(1:lchl).eq.'//RZ/L3/PROD/DATA/SDRETT') goto 20
*
*            Write current directory
*
      CALL RZTOF1(LUNFZ,IOPTC)
      IF(IQUEST(1).NE.0)THEN
         IQ1=IQUEST(1)
         NLPAT=NLPAT-1
         GO TO 20
      ENDIF
*
*            Process possible down directories
*
  20  ISD(NLPAT)=ISD(NLPAT)+1
      IF(ISD(NLPAT).LE.NSD(NLPAT))THEN
         NLPAT=NLPAT+1
         LS=IQ(KQSP+LCDIR+KLS)
         IH=LS+7*(ISD(NLPAT-1)-1)
         CALL ZITOH(IQ(KQSP+LCDIR+IH),IHDIR,4)
         CALL UHTOC(IHDIR,4,CHPAT(NLPAT),16)
         ITIME=ITIME+1
         GO TO 10
      ELSE
         NLPAT=NLPAT-1
         IF(NLPAT.GE.NLPAT0)THEN
            LUP=LQ(KQSP+LCDIR+1)
            CALL MZDROP(JQPDVS,LCDIR,' ')
            LCDIR=LUP
            GO TO 20
         ENDIF
      ENDIF
*
*             Write final trailer
*
      NLPAT=NLPAT0
      CALL FZOUT(LUNFZ,JQPDVS,0,1,'Z',1,1,99)
      IF(IQUEST(1).NE.0)THEN
         IQ1=IQUEST(1)
         GO TO 90
      ENDIF
      LCORD=LQ(KQSP+LTOP-4)
      IF(LCORD.NE.0)THEN
         CALL MZDROP(JQPDVS,LCORD,'L')
         LCORD=0
      ENDIF
*
*            Reset CWD
*
  90  CONTINUE
      CALL RZCDIR(CHWOLD,' ')
      IF(IQ1.NE.0.AND.IQUEST(1).EQ.0)IQUEST(1)=1
*
  99  RETURN
      END
\end{XMPt}

We now perform the following steps:

\begin{OL}
\item
Convert the catalogue to sequential (FZ) format,
skipping bad directories in the process
\item
Convert the sequential file back to RZ format.
If this is done on VM, the program {\tt FATFROMX}
should be used in preference to {\tt RFRX}
\item
Extract all good entries in the bad directories, as
shown above
\item
Move this file to the input directory ({\tt /todo}) of
the server
\end{OL}

Now we have almost completed the catalogue recovery,
with the exception of the corrupted entries.

\index{FATFROMX}
\index{preformatting}
When creating an RZ file on a VM system, it is important
that
\begin{itemize}
\item
The RZ file is mode 6 (update in place)
\item
Records are preformatted so that the file is not extended
but updated in place
\end{itemize}

This is done using the program {\tt FATFROMX}.

The {\tt PARAMETER} NPRE should be set as follows:

\begin{XMPt}{Obtaining an appropriate value for {\tt NPRE}}
/* Calculate NPRE. This assumes a default blocksize of 4096
   bytes, both for the mini-disk and the FATMEN RZ file     */

npre = "QDISK"("191","BLKTOT") * .85

\end{XMPt}




\Filename{H2fatmenappendix-restoring-corrupted-entries}
\section{Restoring the corrupted entries}

In the example above we are now searching for only 7
catalogue entries. Here we have the following possibilities:
\begin{OL}
\item
Is there a copy of the catalogue on another node in which
these entries exist?
\item
Do we have a backup copy, e.g. on tape, in which these entries 
exist?
\item
Can we find the journal files corresponding to these entries?
\end{OL}

We can scan journal files for the entries we need as follows.

\begin{XMPt}{Printing the headers of the journal files}

for i in $FML3/done
   do
      fathead $i >> l3.journal
   done

\end{XMPt}

We then simply {\tt grep} the file {\tt l3.journal} for the
entries in question and then moved the corresponding files
to the {\tt /todo} directory.

\Filename{H1fatmenappendix-cheops-interface}
\chapter{CHEOPS interface}

If FATMEN has been installed with the \Lit{CHEOPS} option,
users may request that files are copied between sites via
the CHEOPS system. Requests will only be accepted at sites
which also have the correct server configuration.

On VM/CMS systems, this requires a special service machine
with username \Lit{FMCHEOPS}. This service machine receives
copy requests from users on the local VM/CMS system and forwards
them to the local CHEOPS server.

On Unix and VAX/VMS systems, an environmental variable/symbol
\Lit{FMCHEOPS} must be defined. This points to a directory
into which the copy requests will be automatically written
by the FATMEN software.

The following steps are involved:
\begin{OL}
\item
When a copy request is made, e.g. using the shell command \Rind{COPY}
with the \Lit{TRANSPORT=CHEOPS} parameter or the \Rind{FMCOPY}
subroutine with the \Lit{K} option, the FATMEN catalogue is updated
with an entry corresponding to the destination file. The comment
field contains the text
\begin{XMP}
Copy request queue to CHEOPS on YYMMDD at HHMM
\end{XMP}
\item
Once the request has been processed by the CHEOPS server,
the comment is changed to
\begin{XMP}
copy successfully queued to CHEOPS.
\end{XMP}
\item
If there are errors processing the request, the comment
becomes
\begin{XMP}
ERROR: <cheops error message>
\end{XMP}
\item
Finally, once the transfer has been performed, the original
comment as specified by the user is restored.
\item
If the actual transfer failed, the reason for the failure
is instead written into the comment string of the corresponding
FATMEN entry.
\end{OL}

\Filename{H2fatmenappendix-build-fatmen-cheops-interface-on-unix}
\section{Building the FATMEN/CHEOPS interface on a Unix system}
The following script may be used to build the FATMEN/CHEOPS
interface on a Unix system.
\begin{XMPt}{Script to build the FATMEN/CHEOPS interface}
ypatchy /cern/pro/src/car/zebra.car cheops2f.f :go <<!
+use,qcde.
+use,ibmrt,*cheops2f.
+exe.
+pam,11,r=qcde,t=c.
+pam,12,t=c,a. fatmen.cards
+quit
!
xlf -q extname cheops2f.f -L/cern/pro/lib -lpacklib -o cheops2f
\end{XMPt}
This server may be run with the following script.
\begin{XMPt}{Running the server}
#!/bin/sh
#
# script to run the FATMEN/CHEOPS interface.
#
t=`date`
h=`hostname`
echo
echo FATMEN/CHEOPS server starting at $t on $h
echo
FMWAKEUP=30;   export FMWAKEUP
FMLOGL=3;      export FMLOGL
FMCHEOPS=/fatmen/fmcheops; export FMCHEOPS
echo Wakeup interval is $FMWAKEUP seconds
echo
$HOME/fatmen/cheops2f
\end{XMPt}
\Filename{H2fatmenappendix-build-FMCHEOPS-server-on-vmcms}
\section{Building the FMCHEOPS server on a VM/CMS system}
The FMCHEOPS server can be built using the following EXEC.
\begin{XMPt}{FMCHEOPS EXEC}
/*       BUILD THE FMCHEOPS SERVER     */                                      
'EXEC YPATCHY PAM1="ZEBRA CAR *" PAM2="FATMEN CARDS" ',                        
    'CRADLE="FATKEOPS CRADLE" ASM="FATKEOPS FORTRAN A1"'                       
'EXEC VFORT FATKEOPS'                                                          
'GIME TCPVMA'                                                                  
'CERNLIB PACKLIB!NEW COMMTXT IBMLIB CMSLIB EDCBASE ( LINK'                     
'GLOBAL LOADLIB VSF2LOAD EDCLINK'                                              
'LOAD FATKEOPS'                                                                
'GENMOD FATKEOPS'
\end{XMPt}
The server should be set up as for other FATMEN servers,
i.e. it should have a \Lit{NAMES} file defining who may
issue privileged commands, as follows:
\begin{XMPt}{Example of a NAMES file for the FMCHEOPS server}
:nick.FATOPERATORS
               :list.fatop1 fatop2
                                 
:nick.FATOP1                                                                   
               :userid.console
               :node.cernvm 
                           
:nick.FATOP2              
               :userid.opsutil
               :node.cernvm 
                                                                               
:nick.FATOWN1                                                                  
               :userid.fatone
               :node.cernvm 
                                                                               
:nick.FATOWN2              
               :userid.hrrcr
               :node.cernvm
                          
:nick.FATOWN3            
               :userid.jamie
               :node.cernvm
                          
:nick.FATOWNERS          
               :list.fatown1 fatown2 fatown3
                                           
\end{XMPt}
The server may be run using the following EXEC.
\begin{XMPt}{Running the FMCHEOPS server}
/*                      F A T _ S T A R T                     */               
Address Command                                                                
   If QCONSOLE("DISCO") then nop                                               
                        else do                                                
                        Say "+++ Type #CP DISC to disconnect +++"              
                        Say "+++ Type #CP DISC to disconnect +++"              
                        Say "+++ Type #CP DISC to disconnect +++"              
                        end                                                    
'IDENTIFY (STACK'                                                              
Parse pull . . locnode .                                                       
                                                                               
If locnode = 'CERNVM' then do                                                  
                                                                               
/* Start FATCAT FORTRAN Program, which calls FATSERV EXEC */                   
   'GLOBALV SELECT *EXEC GET PWD'                                              
   'GLOBALV SELECT *EXEC GET USER'                                             
   'FATKEOPS'                                                                  
   end                                                                         
                                                                              
   else do                                                                     
                                                                               
   Say "FMCHEOPS can only be run on CERNVM..."                                  
   end                                                                         
\end{XMPt}
\Filename{H1fatmenappendix-cheops-interface-to-fatmen}
\chapter{CHEOPS interface to the FATMEN system}
The CHEOPS software uses the information passed from FATMEN
to stage in files pending transfer to the requested remote site.
Similar, once the file has been transferred, a request is made
through the FATMEN software to stage the file out to tape.

To simplify the interface between the CHEOPS software, written
in C, and the FATMEN package, written primarily in FORTRAN,
a special interface has been provided. This interface is intended
only for use by the appropriate CHEOPS software and should not
be called from user programs.
\Filename{H2fatmenappendix-stage-file}
\section{Stage in a file}
\Shubr{FMSTGI}{(GENAME,
     CFQNFA,CHSNFA,ICPLFA,IMTPFA,ILOCFA,CHSTFA,CHOSFA,
     CVSNFA,CVIDFA,IVIPFA,IDENFA,IVSQFA,IFSQFA,ISRDFA,
     IERDFA,ISBLFA,IEBLFA,CRFMFA,IRLNFA,IBLNFA,CFLFFA,
     CFUTFA,ICRTFA,ICTTFA,ILATFA,CCURFA,CCIDFA,CCNIFA,
     CCJIFA,IFPRFA,ISYWFA,IUSWFA,CUCMFA,
     CHLINK,CHOPT,IRC)}
\begin{DLtt}{1234567}
\item[GENAM]Character variable of maximum length 255 specifying the generic
name of the file to be staged.
\item[CFQNFA]Character variable of maximum length 256 specifying the
fully qualified dataset name of the file to be staged.
\item[CHSNFA]Character variable of maximum length 8 specifying the
host on which the file resides (disk files).
\item[ICPLFA]Integer variable specifying the copy level or data representation
of the file to be staged.
\item[IMTPFA]Integer variable specifying the media type code of the file to
be staged.
\item[ILOCFA]Integer variable specifying the location code of the file to be
staged.
\item[CHSTFA]Character variable of maximum length 16 specifying the
host type of the system on which the file resides.
\item[CHOSFA]Character variable of maximum length 12 specifying the 
operation system of the host on which the file resides.
\item[CVSNFA]Character variable of maximum length 8 specifying the
volume serial number of the tape on which the file resides.
\item[CVIDFA]Character variable of maximum length 8 specifying the
visual identifier of the volume on which the file resides.
\item[IVIPFA]Integer VID prefix of the volume
\item[IDENFA]Integer density
\item[IVSQFA]Integer volume sequence number
\item[IFSQFA]Integer file sequence number
\item[ISRDFA]Integer start record number
\item[IERDFA]Integer end record number
\item[ISBLFA]Integer start block number
\item[IEBLFA]Integer end block number
\item[CRFMFA]Character variable of length 4 specifying the record format
\item[IRLNFA]Integer record length in 32 bit words
\item[IBLNFA]Integer block length in 32 bit words
\item[CFLFFA]Character variable of length 4 specifying the FATMEN file format
\item[CFUTFA]Character variable of length 4 specifying the user defined file type
\item[ICRTFA]Integer containing the file creation date and time in packed format
\item[ICTTFA]Integer containing the date and time that the file was catalogued in packed format
\item[ILATFA]Integer containing the date and time of last access in packed format
\item[CCURFA]Character variable of length 8 specifying the user name of the
file creator
\item[CCIDFA]Character variable of length 8 specifying the account of the
creator
\item[CCNIFA]Character variable of length 8 specifying the node on which
the file was created
\item[CCJIFA]Character variable of length 8 specifying the name of the job
that created the file
\item[IFPRFA]Integer file protection mask
\item[ISYWFA]Integer vector of length 10 containing the system words
\item[IUSWFA]Integer vector of length 10 containing the user words
\item[CUCMFA]Character variable of length 80 containing the user comment
\item[CHLINK]Character variable specifying the link should point to the
staged file
\item[CHOPT]Character options
\item[IRC]Integer return code
\end{DLtt}
\Filename{H2fatmenappendix-stage-out-file}
\subsection{Stage out a file}
\Shubr{FMSTGO}{(GENAME,
     CFQNFA,CHSNFA,ICPLFA,IMTPFA,ILOCFA,CHSTFA,CHOSFA,
     CVSNFA,CVIDFA,IVIPFA,IDENFA,IVSQFA,IFSQFA,ISRDFA,
     IERDFA,ISBLFA,IEBLFA,CRFMFA,IRLNFA,IBLNFA,CFLFFA,
     CFUTFA,ICRTFA,ICTTFA,ILATFA,CCURFA,CCIDFA,CCNIFA,
     CCJIFA,IFPRFA,ISYWFA,IUSWFA,CUCMFA,
     CHLINK,CHOPT,IRC)}
\begin{DLtt}{1234567}
\item[GENAM]Character variable of maximum length 255 specifying the generic
name of the file to be staged.
\item[CFQNFA]Character variable of maximum length 256 specifying the
fully qualified dataset name of the file to be staged.
\item[CHSNFA]Character variable of maximum length 8 specifying the
host on which the file resides (disk files).
\item[ICPLFA]Integer variable specifying the copy level or data representation
of the file to be staged.
\item[IMTPFA]Integer variable specifying the media type code of the file to
be staged.
\item[ILOCFA]Integer variable specifying the location code of the file to be
staged.
\item[CHSTFA]Character variable of maximum length 16 specifying the
host type of the system on which the file resides.
\item[CHOSFA]Character variable of maximum length 12 specifying the 
operation system of the host on which the file resides.
\item[CVSNFA]Character variable of maximum length 8 specifying the
volume serial number of the tape on which the file resides.
\item[CVIDFA]Character variable of maximum length 8 specifying the
visual identifier of the volume on which the file resides.
\item[IVIPFA]Integer VID prefix of the volume
\item[IDENFA]Integer density
\item[IVSQFA]Integer volume sequence number
\item[IFSQFA]Integer file sequence number
\item[ISRDFA]Integer start record number
\item[IERDFA]Integer end record number
\item[ISBLFA]Integer start block number
\item[IEBLFA]Integer end block number
\item[CRFMFA]Character variable of length 4 specifying the record format
\item[IRLNFA]Integer record length in 32 bit words
\item[IBLNFA]Integer block length in 32 bit words
\item[CFLFFA]Character variable of length 4 specifying the FATMEN file format
\item[CFUTFA]Character variable of length 4 specifying the user defined file type
\item[ICRTFA]Integer containing the file creation date and time in packed format
\item[ICTTFA]Integer containing the date and time that the file was catalogued in packed format
\item[ILATFA]Integer containing the date and time of last access in packed format
\item[CCURFA]Character variable of length 8 specifying the user name of the
file creator
\item[CCIDFA]Character variable of length 8 specifying the account of the
creator
\item[CCNIFA]Character variable of length 8 specifying the node on which
the file was created
\item[CCJIFA]Character variable of length 8 specifying the name of the job
that created the file
\item[IFPRFA]Integer file protection mask
\item[ISYWFA]Integer vector of length 10 containing the system words
\item[IUSWFA]Integer vector of length 10 containing the user words
\item[CUCMFA]Character variable of length 80 containing the user comment
\item[CHLINK]Character variable specifying the link should point to the
staged file
\item[CHOPT]Character options
\item[IRC]Integer return code
\end{DLtt}

\Filename{H1fatmenappendix-security-issues}
\chapter{Security issues}

Limited security precautions are provided in FATMEN. They are as follows:

\begin{OL}
\item
There are no restrictions on access to catalogue information,
other than those provided at the file system level. That is,
any use may inspect any catalogue. 
\item
Directories may only be deleted if they are completely empty,
i.e. contain no subdirectories and no files.
\item
File entries may only be deleted by their creator. The creator
is identified by the account field.
\item
Protection may be established on directory trees, as explained below.
\end{OL}

\Filename{H2fatmenappendix-restricting-read-access-to-catalogue-info}
\section{Restricting read access to catalogue information}

If such restrictions are required, they must be performed at the
file system level. For example, on VAX/VMS systems one may use
access control to restrict read access to specific users
or users with a given rights identifier.

\Filename{H2fatmenappendix-access-files-catalogues-in-fatmen}
\section{Access to files catalogued in FATMEN}

No restrictions on file access are provided by FATMEN itself.
Volume access restrictions may be set up using the Tape Management System,
if required. For disk files, the access restrictions of the file
system again apply.
\Filename{H2fatmenappendix-access-control-lists}
\section{Access control lists}
\index{ACL}
\index{security}
\index{protection}
It is possible to restrict write access to the FATMEN catalogue
using an ACL file. The ACL file consists of 3 fields.
\begin{OL}
\item
User name
\item
Node name
\item
Path name
\end{OL}
Any of the fields may be wild-carded. When an update operation is
attempted, FATMEN checks the username, nodename and pathname 
against those defined in the file \Lit{FATMEN.ACL}. This file
resides in the directory or on the mini-disk containing the
catalogue for which the update is being attempted. Thus, in the
case of the CDF experiment, the file will be in the directory
pointed to by the symbol or variable \Lit{FMCDF}.

Lines beginning with an exclamation mark, an asterix, a hash or
slash asterix treated as comments.

\begin{XMPt}{Example of an ACL file}

/*              FATMEN.ACL file for CDF experiment at Fermilab         */

! ! !
! ! ! CDF FATMEN Superusers: can modify any directory
! ! !
LINGFENG * //FNAL/CDF
! ! !
! ! ! CDF FATMEN Test users: can modify the subtree //FNAL/CDF/FATMEN
! ! !
CDF_FATM * //FNAL/CDF/FATMEN    
! ! !
! ! ! CDF FATMEN General users: can modify <user> tree.
! ! ! e.g. FRODO can modify //FNAL/CDF/USERS/FRODO
! ! !
<user> * //FNAL/CDF/USERS/<user>
! ! !
! ! ! CDF FATMEN General users: user ID does not match user name:
! ! !
GPYEH * //FNAL/CDF/USERS/YEH
! ! !
! ! ! CDF FATMEN TOP group Superusers:
! ! !
GPYEH * //FNAL/CDF/TOP
! ! !
! ! ! CDF FATMEN TOP DILEPTON subgroup Superusers:
! ! !
LUC * //FNAL/CDF/TOP/DILEPTON
CHIKA * //FNAL/CDF/TOP/DILEPTON
CENYI * //FNAL/CDF/TOP/DILEPTON
\end{XMPt}
\Filename{H2fatmenappendix-account-aliases}
\section{Account aliases}
\index{account}
In addition to the above protection, catalogue entries may only
be deleted by users whose account matches that of the catalogue
entry. However, as shown in the following example, accounts are
often not common across different systems. 
\begin{XMP}
FM> ls -o                                                                   
 
 Directory: //CERN/CNDIV/TAPES
 
 Generic filename: XTRACE91
 Created by:  JAMIE    ACCT: JDS$CT   on node: CERNVM   by job: JAMIE   
 
 Generic filename: XTRACE92
 Created by:  jamie    ACCT: JDSCT    on node: zfatal   by job: jamie-15
 
\end{XMP}
To cater for this situation,
account aliasing is possible. An alias file, \Lit{FATMEN.ACCOUNTS},
consists of 2 fields.
\begin{OL}
\item
Account name
\item
List of aliases
\end{OL}
Note that this scheme is not designed for large numbers of users
with different accounts on each machine. It is intended for use
by a small number of production managers.
\begin{XMPt}{Example of an account alias file}
EMC$XV=CCA$XV,PRO$XV                                                            
CCA$XV=EMC$XV,PRO$XV                                                            
PRO$XV=CCA$XV,EMC$XV                                                            
\end{XMPt}

\Filename{H2fatmenappendix-update-control}
\section{Update control}

\index{updates}
\index{update}

By default, all file accesses that use the routine \Rind{FMOPEN}
or the shell \Rind{FIND} or \Rind{MAKE} commands result in
a logging record being sent to the servers. This record contains
information which includes how the file was accessed, the time taken to access
the file and updates the usage count and last access date and time.

Such updates can be controlled by a file named {\tt FATMEN.UPDATES}
which has the same format as
the {\tt FATMEN.ACL} file. If such a file exists, then the logging
information will only be sent if there is an appropriate line
in this file.
\begin{XMPt}{Example of an update control file}

!
! Log monitoring information only for production users
!
DELPROD * //CERN/DELPHI
!
! but also for users accessing their private files
!
<user> * //CERN/DELPHI/USERS/<user>

\end{XMPt}
\Filename{H1fatmenappendix-tms-at-cern}
\chapter{TMS at CERN}
\par
The following information describes the TMS as installed at CERN.
In general, the description is also valid for outside sites where
the TMS is also installed, such as the Rutherford Appleton Laboratory
in the UK, where the product originated, and IN2P3 and Saclay in France.
\par
The Tape Management System (TMS) has been installed at CERN
to manage the tapes and cartridges used at CERN. Information on some 450,000
volumes is held in a central database and may be queried and, in some cases,
updated by users.
\par
Note that FATMEN automatically issues TMS commands where necessary,
and so a detailed knowledge of the TMS and the command syntax is
not required. However, it may be useful to have a general overview
of the system.
{\bf Note that all access to TMS information through FATMEN is via
a generic name, and not a tape number.}
\par 
The TMS system runs in a special service machine and, as is the case for other
service  machines, does not accept commands directly but rather through a
special interface. For other service machines the interface is SMSG, but TMS
uses instead the SYSREQ interface (again from RAL) which handles automatic
continuati of responses to queries and which allows (following modifications at
CERN) queries from other CERN machines.
\par
The SYSREQ interface is also available on remote machines, such as the
central VAXcluster VXCERN, the Cray, the Siemens, SHIFT etc. This
remote interface runs over TCP/IP and is contained in the CERN Program
Library entry CSPACK. 

A TMS query may be issued as follows:
\begin{XMP}
   SYSREQ TMS QUERY VID YZ0001
\end{XMP}
and similarly for  other TMS commands.  Note that descriptions  of TMS commands
in the various help  files generally omit the "SYSREQ TMS",  but this is always
needed.

A list of all TMS commands is available via the command \underline{XFIND TMS} 
on CERNVM.  On remote systems, this information is available
through the www browser. To access the XFIND information directly, 
use a script such as
\begin{XMP}
# XFIND script
www "http://crnvmc/FIND" $@
\end{XMP} 
and then use the command \underline{XFIND TMS} as on CERNVM.
\par
In either case you can  then find information on a specific command
of interest.  To  find information about a particular TMS  command directly use
the  commands \underline{XFIND TMS  command}, e.g.
\begin{XMP}
xfind tms query
\end{XMP}

\Filename{H2fatmenappendix-TMS-volume-organisation}
\section{Volume Organisation in TMS}

The logical  organisation of  volumes within TMS  follows closely  the physical
organisation   and  is  therefore  based  around  volume  prefixes.   For  each
different  volume  prefix   a  number  of  TMS  LIBRARIES   are  defined,  each
corresponding to a  possible location for volumes in that  series.  Most series
will have the libraries/locations (where xx is the two lette VID prefix)
\begin{XMPt}{Examples of TMS libraries}
 
xx_VAULT the location  for volumes physically in the tape  vault.  Only volumes
         in these libraries will be able to mounted on the central systems.
 
xx_REST  a general location for volumes not in the vault
 
\end{XMPt}
\par
In  some cases  there will  be libraries  such as  xx\_EXPT for  volumes at  the
experiment - but  this depends on the  way an individual experiment  chooses to
manage its own volumes.

Certain FATMEN routines require a TMS library as an argument, such
as \Rind{FMALLO}, which allocates a volume from a named pool (see below)
from a given library.

\Filename{H2fatmenappendix-TMS-volume-ownership}
\section{Volume Ownership in TMS} 

As well as recording the location of a volume TMS also holds ownership details.
Ownership of  a volume  is by  ACCOUNT and  OWNER.  ACCOUNT  is the  usual CERN
accounting group, gg  - all Delphi volumes,  for example, are owned  by account
XX.

The OWNER  field in the  TMS database is a  little more special,  however.  All
registered CERN computer users have an individual user code, or UUU, and if the
owner  field  is  a three  character  UUU  then  the  volume is  owned  by  the
corresponding individual.

If the owner field  is longer than 3 characters this  then it cannot correspond
to a  UUU and so  no real person  owns the volume,  instead the owner  field is
being used  to group  the volume  into a  POOL for  a particular  purpose.  For
example  most experiments  wish to  maintain a  pool  of volumes  which can  be
allocated to  users on  request. These  volumes can  be given  the TMS  'owner'
GG\_USER and  it is then  easy to see  which volumes remain  in the pool  and to
choose one for allocation - perhaps  with the GETPOOL command, or better,
using the routine \Rind{FMALLO}.  The FATMEN shell command
{\bf ALLOCATE} may also be used to allocate volumes. {\bf Note that
FATMEN only retains information on volumes by generic name.
That is, there must be at least one file in the FATMEN catalogue
on a given volume}.
Another example
would be  to have  the 'owner'  for all DST  volumes as  GG\_DST so  users could
easily list the available  DSTs (and even for a given  location using the QUERY
CONTENTS command).
\par
Volumes may be moved between pools using the routine \Rind{FMPOOL}.
\Filename{H2fatmenappendix-TMS-access-rules}
\section{Volume Access Rules in TMS}
\par 
TMS always permits read access to all volumes by all users.
{\bf N.B. this may not be the case outside CERN!}
\par 
Write access is normally allowed to all members of the account group owning the
volume, but access  may be restricted by using PROTECTION  GROUPS. A protection
group allows access to be limited to  a small number of individuals or extended
to  members of  other groups.   The QUERY  VID command  gives the  name of  the
protection  group governing  write access  to a  volume and  the QUERY  PROTECT
command can  be used to  see the users permitted  access by a  given protection
group.
\par 
The owner of a  volume (or an authorised user for  'pool' volumes) may globally
disable write access using the LOCK command  - a software equivalent of the "Do
not write on this tape" sticker. This functionality is provided
by the routine \Rind{FMLOCK}. Volumes may be unlocked using the routine
\Rind{FMULOK}. LOCK and UNLOCK commands also exist in the FATMEN shell.
\Filename{H2fatmenappendix-TMS-create-pool}
\section{Creating TMS pools for use with FATMEN}
\par
The following TMS commands may be used to create TMS pools for use
with the routine \Rind{FMALLO} or the shell {\tt ALLOCATE} command.
Note that one may also allocate tapes from these pools by issuing
TMS commands directly without causing any conflict.
\par
Note that these commands require privilege and may normally only
be issued by the tape librarian.
\par
If the tapes are already known to the TMS, the following commands
should be used.
\begin{XMPt}{Creating a TMS pool}

PROTECT poolname ACCOUNT gg CREATE

PROTECT poolname ACCOUNT gg GRANT AUTHORITY CONTROL ACCOUNT gg
                                                 or USERID  uuu

TRANSFER VID vid1 - vid2 FROMUSER owner TO USERID poolname

\end{XMPt}
\par
The {\tt poolname} is the name of the pool, e.g. {\tt XU\_FREE}.
{\tt gg} and {\tt uuu} have their usual meanings. The two
{\tt GRANT AUTHORITY} commands differ in that the first permits
and member of the group {\tt gg} to allocate tapes from this pool,
where as the second enables only a specified user.
\par
If the tape volumes are not yet known to the TMS, they must be
entered using the {\tt ENTER} command. At this time, one may
also specify the {\tt OWNER} and hence avoid the {\tt TRANSFER}
command described above. For example, if the tapes are to be
in pool {\tt XX\_FREE}, then they should be entered with
{\tt USERID XX\_FREE}.
However, the {\tt PROTECT} commands are still required.    

\Filename{H2fatmenappendix-TMS-return-codes}
\section{TMS return codes}
\begin{verbatim}
#ifndef STNDMESG_H_INCLUDED
/************************************************************/
/*                                                          */
/* Object : Definition of the  standard  messages used  by  */
/* =======  various standard parts of the TMS system.  For  */
/*          inclusion in local message repositories         */
/*                                                          */
/* Attrib : TMS system header file.                         */
/* =======                                                  */
/*                                                          */
/* Author : (c)  Jonathan  Wood,  Systems  Group, CCD, RAL, */
/* =======  5_Nov_1988                                      */
/*                                                          */
/* ANSI C Version : Martin Ellerker, IBM Support Section,   */
/* ===============  Software Group, CERN.                   */
/*                  28_April_1993                           */
/*                                                          */
/************************************************************/
/*  File:          "stndmesg.h"                             */
/*  Dependencies:  "message.h"                              */
/************************************************************/
#define STNDMESG_H_INCLUDED 1
 
#include "message.h"
 
 
#ifndef EXCLUDE_convert_msgs
    #define EXCLUDE_convert_msgs 1
 
    QMSG(1,"Input fileid is missing")
    QMSG(2,"Unable to open input file")
    QMSG(3,"\nSyntax{%d:%d}. Expected '%s' "\
"found instead '%.*s'")
    QMSG(4,"%*.*s%c")
    QMSG(5,"Unable to open output file")
    QMSG(6,"Processed %d clauses in %d records to "\
"output file '%s'")
    QMSG(7,"Processing input file %s")
    QMSG(8,"Processed %d clauses in %d records "\
"with %d errors")
 
    #endif
 
#ifndef EXCLUDE_parser_msgs
    #define EXCLUDE_parser_msgs 1
 
    QMSG(20,"Syntax Error")
    QMSG(21,"Found '%.*s' where expected")
    QMSG(22,"%c<%-8.8s>")
    QMSG(23,"%c'%-8.8s'")
    QMSG(24,"%c %-8.8s ")
    QMSG(25,"Parser Error: %s :OPR %-8.8s")
    QMSG(27,"Parser Error: %s :OPT %-8.8s")
    QMSG(28,"Syntax fallacy at token: %-8.8s")
    QMSG(29,"Unable to load DLCS syntax file '%s'")
    QMSG(30,"Unexpected clause type during load at block %d")
 
    #endif
 
#ifndef EXCLUDE_index_msgs
    #define EXCLUDE_index_msgs 1
    QMSG(35,"Rebuild attempted for indices on: %.*s code %d")
    QMSG(36,"Routine for index rebuild on: %.*s not provided")
    #endif
 
#ifndef EXCLUDE_util_msgs
    #define EXCLUDE_util_msgs 1
 
    QMSG(40,"Getcor failure for %d * %d bytes")
    QMSG(41,"Message buffer too small: Msg number %5.5d")
    QMSG(42,"Host command complete rc=%d")
 
    #endif
 
 
#ifndef EXCLUDE_comms_msgs
    #define EXCLUDE_comms_msgs 1
 
    QMSG(45,"**** Possible command retry: may have been "\
"executed")
    QMSG(50,"Schedule table error. Item number %d."\
"      SQLCODE %d")
 
 
    #endif
 
#ifndef EXCLUDE_sched_msgs
    #define EXCLUDE_sched_msgs 1
 
    QMSG(55,"Recurrence interval too large!")
    QMSG(56,"Permission Denied: Scheduled request %d")
    QMSG(57,"Command too long to schedule!")
 
    QMSG(58,"Reqnum   Userid   Command  Last_Date       "\
"Next_date       Recursion")
    QMSG(59,"-------- -------- -------- --------------- "\
"--------------- -----------------")
    QMSG(60,"%8.1d %-8.8s %-8.8s %15.6f %15.6f %8.8s %-8.8s")
    QMSG(61,"%.*s %1.8d")
    QMSG(62,"%.*s")
 
    #endif
 
#ifndef EXCLUDE_feature_msgs
    #define EXCLUDE_feature_msgs 1
 
    QMSG(65,"Invalid feature type: %.*s")
    QMSG(66,"Unsupported feature: %.*s")
    QMSG(67,"Feature change rejected: %.*s")
    QMSG(68,"%-8.8s")
    QMSG(69,"%.*s %-8.8s")
 
    #endif
 
#ifndef EXCLUDE_secure_msgs
    #define EXCLUDE_secure_msgs 1
 
    QMSG(70,"Permission denied: Command %.*s")
    QMSG(71,"Not you again?")
    QMSG(72,"You are becoming a nuisance")
    QMSG(73,"Your activity is being logged")
    QMSG(74,"Desist!  You have been warned")
    QMSG(75,"Your file has been passed to "\
"the Fraud Squad")
 
    QMSG(76,"Authorisation not found")
    QMSG(77,"Authorisation already exists")
    QMSG(78,"%-9.8s")
    QMSG(79,"%.*s %-8.8s")
 
 
    #endif
 
#ifndef EXCLUDE_dbase_msgs
    #define EXCLUDE_dbase_msgs 1
 
    QMSG(80,"Severe SQL error SQLCODE=%d Attempting recovery")
    QMSG(81,"Severe SQL error SQLCODE=%d Unable to recover")
    QMSG(82,"SQL Diagnostic error report:\n"\
" SQLCAID: %-8.8s SQLCABC: %d SQLCODE: %d SQLERRP: %-8.8s")
    QMSG(83," SQLERRM: %.*s")
    QMSG(84," SQLERRD: %d %d %d %d %d %d")
    QMSG(85," SQLEXT : %-5.5s")
    QMSG(86," SQLWARN: 0123456789A\n          %-11.11s")
 
    #endif
 
#ifndef EXCLUDE_main_msgs
    #define EXCLUDE_main_msgs 1
 
    QMSG(87,"Installation configuration complete")
    QMSG(88,"Installation configuration already complete")
    QMSG(89,"Initialisation already complete")
    QMSG(90,"**** %.*s: Sub-Command implementation deferred")
    QMSG(91,"**** %s stopped by %-8.8s on %-8.8s")
    QMSG(92,"**** %.*s: Command implementation deferred")
    QMSG(93,"**** %s initialisation complete")
    QMSG(94,"**** Command processor restart: %d")
    QMSG(95,"%s restarting at %.24s in '%s' mode")
    QMSG(96,"**** New command accepted at %.24s from "\
"%-8.8s (%-8.8s) on %-8.8s:\n%.*s")
    QMSG(97,"**** Completion code %d at %.24s")
    QMSG(98,"%s terminating at %.24s with code: %d")
    QMSG(99,"**** The %s is temporarily unavailable")
 
    #endif
 
#endif                            /* ifndef x_H_INCLUDED     */
 
/* UPDATES APPLIED AT 09:43:45 ON 12 AUG 1993
UUU$GG
*         STNDME$H UUU$GG   D1 REL191  08/12/93   09:40:06
*/

#ifndef APPLMESG_H_INCLUDED
/************************************************************/
/*                                                          */
/* Object : Definition  of the  message repository for the  */
/* =======  application server program.  Insert application */
/*          specific messages.                              */
/*                                                          */
/* Attrib : Application system Header file.                 */
/* =======                                                  */
/*                                                          */
/* Author : (c)  Jonathan  Wood,  Systems  Group, CCD, RAL, */
/* =======  5_Nov_1988                                      */
/*                                                          */
/************************************************************/
/*  File:          "applmesg.h"                             */
/*  Included by:   "message.h"                              */
/************************************************************/
#define APPLMESG_H_INCLUDED 1
 
QMSG(101,"%s")
 
/*************************************************************/
/*  Msgs for QTAPE command                                   */
/*************************************************************/
QMSG(102,"%-8.8s %9.1ld %-8.8s %-8.8s %8.1ld %8.1ld")
QMSG(103,"Unknown volume : %-6.6s")
 
/*************************************************************/
/*  Msgs for QUERY VID command                               */
/*************************************************************/
QMSG(104,"Volume(s) not found")
QMSG(105," %-8.8s")
QMSG(106," %-6.6s")
QMSG(107," %-3.3s")
QMSG(108," %-1.1s")
QMSG(109," %-12.12s")
QMSG(110," %015.6f")
QMSG(111," %8.1ld")
QMSG(112," %5.1hd")
QMSG(113,"%.*s %.6s - %.6s %.*s")
QMSG(114,"%-6.6s")
QMSG(115," %8.8s")
 
/*************************************************************/
/*  Msgs for LIBRARY and QUERY LIBRARY commands              */
/*************************************************************/
QMSG(119,"Library not empty: %-.8s")
QMSG(120,"Duplicate library name: %-.8s")
QMSG(121,"Unknown library: %-.8s")
QMSG(122,"Library not empty: %-.8s")
QMSG(123,"Library size change invalid: %-8.8s")
QMSG(125,"Library  Czar     Group    R A L S M Target   "\
"Retain Racks  Slots    Spare")
QMSG(126,"-------- -------- -------- - - - - - -------- "\
"------ ------ -------- --------")
QMSG(127,"%-8.8s %-8.8s %-8.8s %c %c %c %c %c %-8.8s "\
"%6.1ld %6.1ld %8.1ld %8.1ld")
QMSG(128,"%.*s")
QMSG(129,"%.*s %-8.8s %-8.8s")
 
/*************************************************************/
/*  Msgs for RACK    command                                 */
/*************************************************************/
QMSG(130,"Slot(s) in use: %-8.8s %1.1d %1.1d")
QMSG(131,"Unracked library: %-8.8s")
QMSG(132,"Unknown rack: %-8.8s %1.1d")
QMSG(133,"Invalid status change: %-8.8s %1.1d %1.1d")
QMSG(134,"Invalid split/join for rack: %-8.8s %1.1d %1.1d")
QMSG(135,"Rack too large: %-8.8s %1.1d %1.1d")
QMSG(136,"Slot was not in use: %-8.8s %8.1d")
QMSG(137,"Invalid device model: %-6.6s")
 
/*************************************************************/
/*  Msgs for Rule validation functions.                      */
/*************************************************************/
QMSG(139,"Permission Denied: Account %-8.8s")
QMSG(140,"Permission Denied: Volume %-6.6s")
QMSG(141,"Permission Denied: No volumes found")
QMSG(142,"Permission Denied: Library %-8.8s not found")
QMSG(143,"Permission Denied: Library %-8.8s")
QMSG(144,"Permission Denied: Group %-8.8s Account %-8.8s")
 
/*************************************************************/
/*  Msgs for Group Manipulation functions                    */
/*************************************************************/
QMSG(145,"Permission Denied: Volume account mismatch")
QMSG(146,"Permission Denied: Group exists")
QMSG(147,"Permission Denied: Group in use")
 
/*************************************************************/
/*  Msgs for Query Group(s) functions                        */
/*************************************************************/
QMSG(148,"*Account ")
QMSG(149,"*Userid  ")
QMSG(150,"*World   ")
QMSG(151,"%-9.8s")
QMSG(152,"%-7.6s")
QMSG(153,"%.*s %c %-8.8s")
 
/*************************************************************/
/*  Msgs for Query RACK     function                         */
/*************************************************************/
QMSG(154,"Library  Start    End      Slots    Spare    "\
"Model  Status")
QMSG(155,"-------- -------- -------- -------- -------- "\
"------ ------")
QMSG(156,"%-8.8s %8.1d %8.1d %8.1d %8.1d %-6.6s %c")
QMSG(157,"%.*s %8.1d ")
 
/*************************************************************/
/*  Msgs for MOVE           functions                        */
/*************************************************************/
QMSG(160,"Insufficient space: Library %-8.8s")
QMSG(161,"Move initiated for %1.1d volumes out of %1.1d tried")
QMSG(162,"Logical library: %-8.8s")
QMSG(163,"Insufficient space in library")
 
/*************************************************************/
/*  Msgs for the REQUEST queue manipulation functions        */
/*************************************************************/
QMSG(170,"Duplicate action request for vid: %-6.6s")
QMSG(171,"Request not found: vid %-6.6s")
QMSG(172,"Permission denied: Request on vid %-6.6s")
 
QMSG(173,"A C Target_1 Target_2 Requestr Jobname  Owner    "\
"Date            Jobid   ")
QMSG(174,"- - -------- -------- -------- -------- -------- "\
"--------------- --------")
QMSG(175,"%c %c %-8.8s %8.1d %-8.8s %-8.8s %-8.8s %15.6f"\
" %-8.8s")
QMSG(176,"%.*s %8.1d")
 
QMSG(177,"%1.1d requests of type '%.*s' found")
QMSG(178,"%-6.6s %-8.8s %8.1d %c %-8.8s %8.1d "\
"%-8.8s %15.6f %-8.8s")
QMSG(179,"%.*s %-8.8s %1.1d %-8.8s %1.1d "\
"%-6.6s %1.1d")
QMSG(180,"VID    Cur_lib  Cur_slot C Target_1 Target_2 "\
"Userid   Date            Target_3")
QMSG(181,"------ -------- -------- - -------- -------- "\
"-------- --------------- --------")
 
/*************************************************************/
/*  Msgs for the MOUNT   command and subcommands             */
/*************************************************************/
QMSG(182,"Volume locked read-only: %-6.6s")
QMSG(183,"Operating system unknown: %-8.8s")
QMSG(184,"Volume preallocated: %-6.6s")
QMSG(185,"Volume being deleted: %-6.6s")
QMSG(186,"Volume absent: %-6.6s")
QMSG(187,"Volume held: %-6.6s")
QMSG(188,"Volume unknown to TMS: %-6.6s")
QMSG(189,"Volume not in %.*s: %-6.6s")
QMSG(190,"Volume busy: %-6.6s")
QMSG(191,"Volume mount not (yet) active: %-6.6s")
QMSG(192,"Volume ring status changed: %-6.6s")
QMSG(193,"%-6.6s %-8.8s %8.1d %-6.6s %-3.3s %8.8s %-8.8s "\
"%8.1d %8.1d %1.1c")
QMSG(194,"Attempting auto-move to library: %-8.8s")
QMSG(195,"Volume is not of type '%.*s': %-6.6s")
 
/*************************************************************/
/*  Msgs for the CLEAN   command                             */
/*************************************************************/
QMSG(196,"Clean initiated for %1.1d volumes out of %1.1d "\
"tried")
 
/*************************************************************/
/*  Msgs for the ENTER/ISSUE commands                        */
/*************************************************************/
QMSG(200,"Rejected: Unknown device type: Model %-6.6s "\
"Max_den %-6.6s Length %d")
QMSG(201,"Rejected: Invalid density: %-6.6s")
QMSG(202,"Rejected: Invalid VSN range %-.6s-%-.6s")
QMSG(203,"Rejected: VID not the same as VSN")
QMSG(204,"Rejected: VID wraparound")
QMSG(205,"Rejected: Only %d slots available for %d volumes")
QMSG(206,"Rejected: %d VIDs in range already exist")
QMSG(207,"Rejected: Invalid Account(%-8.8s) or "\
"Userid(%-8.8s)")
QMSG(208,"Rejected: Congested VID range - pick another!")
QMSG(209,"Rejected: Ran out of slots with %d volumes to go")
QMSG(210,"Rejected: Unexpectedly entered no volumes!!")
QMSG(211,"Rejected: Insufficient supply of requested "\
"device type")
QMSG(212,"Rejected: Unknown allocation series: %-8.8s")
QMSG(213,"Rejected: Limit of %d volumes exceeded")
QMSG(214,"Rejected: Foreign batch not found: Model %.6s "\
"Max_den %-6.6s Length %d")
 
/*************************************************************/
/*  Msgs for the TAG     command                             */
/*************************************************************/
QMSG(220,"%d")
QMSG(221,"%.*s")
 
/*************************************************************/
/*  Msgs for the REMOVE and RETURN commands                  */
/*************************************************************/
QMSG(225,"Remove initiated for %1.1d volumes out of %1.1d "\
"tried")
QMSG(226,"Return initiated for %1.1d volumes out of %1.1d "\
"tried")
QMSG(227,"Volume is not absent: %6.6s")
 
/*************************************************************/
/*  Msgs for the DEVTYPE and QUERY DEVTYPES commands         */
/*************************************************************/
QMSG(230,"Unknown devtype: Model %-6.6s Max_den %d "\
"Length %d")
QMSG(231,"Duplicate devtype: Model %-6.6s Max_den %d "\
"Length %d")
QMSG(232,"Devtype in use: Model %-6.6s Max_den %d "\
"Length %d")
 
QMSG(233,"Model   Max_den   Length     Code Sort Group    "\
"P Max_strp Strp_siz")
QMSG(234,"------ -------- -------- -------- ---- -------- "\
"- -------- --------")
QMSG(235,"%6.6s %8.8s %8.1d %8.1d %4.4s %8.8s %c %8.1d %8.1d")
QMSG(236,"%.*s : %.6s %8.8s %d")
 
/*************************************************************/
/*  Msgs for the SUPPLY  and QUERY SUPPLY   commands         */
/*************************************************************/
QMSG(240,"Supply order requested not found")
QMSG(241,"Only %d volumes are still outstanding for this "\
"supply order")
QMSG(242,"Supply order already exists")
 
QMSG(243,"Reqnum   Batchnum Model    Maxden   Length S "\
"   Total     Left  Broken  Date")
QMSG(244,"-------- -------- ------ -------- -------- - "\
"-------- -------- -------- --------")
QMSG(245,"%-8.8s %8.1d %-6.6s %8.8s %8.1d %c %8.1d %8.1d "\
"%8.1d %8.8d")
 
QMSG(246,"%.*s %d")
QMSG(247,"%.*s %d %d")
 
QMSG(248,"Reqnum   Batchnum Model    Maxden   Length S "\
"   Total     Cost Userid   Manufact")
QMSG(249,"%-8.8s %8.1d %-6.6s %8.8s %8.1d %c %8.1d %8.1d "\
"%-8.8s %-8.*s")
 
QMSG(250,"Supply batch %d not found")
QMSG(251,"Batch %d is not in supply")
QMSG(252,"Batch %d is not HELD")
QMSG(253,"Supply batch only has %d volumes left")
QMSG(254,"Supply batch only has %d broken volumes")
QMSG(255,"Batch %d is in use")
 
QMSG(258,"Density value must not be greater than MaxDensity")
 
 
QMSG(260,"Volume %-6.6s not Free or Held")
QMSG(261," Volume %-6.6s")
QMSG(262,"Transfer from %-8.8s to %-8.8s and account "\
"unchanged")
QMSG(263,"Non existant new group: %-8.8s Account %-8.8s "\
"Volume %-6.6s")
QMSG(264,"No matching volumes found belonging to %-8.8s")
QMSG(265,"Permission Denied: Operand %.*s")
 
/*************************************************************/
/*  Msgs for the QUERY CONTENTS command.                     */
/*************************************************************/
QMSG(270,"Empty library: %-8.8s")
QMSG(271,"Library      Slot Type VID    Model  Owner    "\
"Account  S")
QMSG(272,"-------- -------- ---- ------ ------ -------- "\
"-------- -")
QMSG(273,"%-8.8s %8.1d Home %-6.6s %-6.6s %-8.8s %-8.8s %c")
QMSG(274,"%-8.8s %8.1d Move %-6.6s %-6.6s %-8.8s %-8.8s %c")
QMSG(275,"%.*s %-6.6s %8.1d")
QMSG(276,"No volumes found in given slot range")
QMSG(277,"Slots for library %-8.8s between %8.1d and %8.1d")
 
/*************************************************************/
/*  Msgs for the OPSYSTEM and QUERY OPSYSTEM commands        */
/*************************************************************/
QMSG(280,"Unknown Operating system: %-8.8s")
QMSG(281,"Duplicate Operating system: %-8.8s")
QMSG(282,"System   Generic  Userid   SysType  Enq Scr "\
"Location")
QMSG(283,"-------- -------- -------- -------- --- --- "\
"------------------------------")
QMSG(284,"%-8.8s %-8.8s %-8.8s %-8.8s  %c   %c  %.*s")
QMSG(285,"%.*s %.8s %.8s")
QMSG(286,"Requests outstanding from system %-8.8s")
QMSG(287,"Require location identifier")
 
/*************************************************************/
/*  Messages for the LOGICAL family of commands              */
/*************************************************************/
QMSG(290,"Library not logical: %-8.8s")
QMSG(291,"Duplicate cartridge: %-8.8s")
QMSG(292,"Unknown cartridge: %-8.8s")
QMSG(293,"Stripe range invalid: %1.1d-%1.1d")
QMSG(294,"Stripe(s) in use: %-8.8s %1.1d-%1.1d")
 
QMSG(295,"Cart     VID    Stripe_1 Stripe_2 Last_written    "\
"S Userid   Account  Library")
QMSG(296,"-------- ------ -------- -------- --------------- "\
"- -------- -------- --------")
QMSG(297,"%-8.8s %-6.6s %8.1d %8.1d %15.6f "\
"%c %-8.8s %-8.8s %-8.8s")
QMSG(298,"%.*s %-6.6s")
 
QMSG(299,"Cart     Model  Volumes  Userid   Account  "\
"Library      Slot Group")
QMSG(300,"-------- ------ -------- -------- -------- "\
"-------- -------- --------")
QMSG(301,"%-8.8s %-6.6s %8.1d %-8.8s %-8.8s "\
"%-8.8s %8.1d %-8.8s")
QMSG(302,"%.*s %-8.8s %-8.8s")
 
/*************************************************************/
/* For the QVOL command                                     */
/*************************************************************/
QMSG(310,"%-6.6s Volume Not Known. %25.1c")
QMSG(311,"%-6.6s %-8.8s %8.8ld %-6.6s %8.8s %1.1c %1.1c "\
"%1.1c %1.1c %1.1c %1.1c %8.8s %5.5ld %5.5ld %-3.3s")
QMSG(313,"Unknown Generic Operating System: %-8.8s")
QMSG(314,"Unknown Operating System: %-8.8s")
QMSG(315,"Volume unmountable")
QMSG(316,"Ignoring option: %-8.8s")
 
/*************************************************************/
/* For the ALLOCATE command                                 */
/*************************************************************/
QMSG(320,"Can not allocate zero tapes.")
QMSG(321,"Allocated %d tape%sof device code %d")
QMSG(322,"Allocated %d tape%sfrom batch %d.")
 
/*************************************************************/
/* For the Transparent Tape Staging Interface.              */
/*************************************************************/
QMSG(330,"rc = %d  from TMSPURGE of vid: %6.6s")
QMSG(331,"Error flushing tape in Transparent Tape Stager. "\
"Probable stage out pending.")
QMSG(400,"Exchange initiated for vid: %6.6s")
 
/*************************************************************/
/*  Msgs for the CZAR/MANAGER command                        */
/*************************************************************/
QMSG(1076,"Czar/Account/Node not found")
QMSG(1077,"Czar/Account/Node already exists")
QMSG(1078,"%-9.8s")        /* Not used */
QMSG(1079,"%.*s %-8.8s %-8.8s %-8.8s")
QMSG(1080,"%.*s %-8.8s %-8.8s")
QMSG(1081,"%.*s %-8.8s")
QMSG(1121,"Unknown Czar   : %-.8s")
QMSG(1122,"Unknown Account: %-.8s")
QMSG(1125,"Manager  Account  Node     Set")
QMSG(1126,"-------- -------- -------- ---")
QMSG(1127,"%-8.8s %-8.8s %-8.8s  %c")
QMSG(1128,"%.*s")
QMSG(1129,"%.*s %-8.8s %-8.8s")
QMSG(1130,"TMS authorization granted to %-8.8s for "\
"account %-8.8s node %-8.8s")
QMSG(1131,"TMS authorization revoked from %-8.8s for "\
"account %-8.8s node %-8.8s")
QMSG(1140,"Permission Denied: Account %-8.8s Node %-8.8s")
 
/*************************************************************/
/*  Msgs for the Frequently Used Commands algorithm          */
/*************************************************************/
QMSG(1200,"**** Vector initialisation failure")
QMSG(1205,"Command    Location    Count MilliSec   "\
"    Next Previous   Syntax")
QMSG(1206,"--------   -------- -------- --------   "\
"-------- -------- --------")
QMSG(1207,"%-8.8s   %8x %8lu %8lu   %8x %8x %8x")
QMSG(1208,"%-8.8s %6lu")
QMSG(1210,"%.*s %-8.8s")
 
/*************************************************************/
/*  Message for the QUERY FREE1ST SUBCOMMAND                 */
/*************************************************************/
QMSG(2001,"First free slot: %d")
 
/*************************************************************/
/*  Msgs for the FLUSH SYSTEM command                        */
/*************************************************************/
QMSG(2010,"All MOUNT requests flushed")
QMSG(2011,"No MOUNT requests found")
 
/*************************************************************/
/*  Msgs for the extended AUTHUSER command                   */
/*************************************************************/
QMSG(2076,"%-8.8s")
QMSG(2077,"%-8.8s by %-8.8s")
QMSG(2078,"%-8.8s AT %-8.8s")
QMSG(2079,"%.*s %-8.8s %-8.8s")
QMSG(2080,"%.*s FROM %-8.8s")
 
/*************************************************************/
/*  Msgs for the extended QUERY COMMAND command              */
/*************************************************************/
QMSG(2090,"General commands for %-8.8s at %-8.8s")
QMSG(2091,"All possible commands for %-8.8s at %-8.8s")
QMSG(2092,"Authorised commands for %-8.8s at %-8.8s")
QMSG(2095,"------- -------- --- -------- -- --------")
QMSG(2096,"--- -------- -------- --- -------- -- --------")
QMSG(2097,"---------- -------- --- -------- -- --------")
QMSG(2100,"General command %-8.8s not found")
QMSG(2101,"Command %-8.8s not found")
QMSG(2102,"Authorised command %-8.8s not found")
 
/*************************************************************/
/*  Msgs for the extended QUERY LIBRARY command              */
/*************************************************************/
QMSG(2125,"Library  Czar     Group    Slots    Spare    "\
"Location")
QMSG(2126,"-------- -------- -------- -------- -------- "\
"------------------------------")
QMSG(2127,"%-8.8s %-8.8s %-8.8s %8.1ld %8.1ld %.*s")
QMSG(2129,"%.*s %-8.8s %-8.8s ( LOCID")
 
/*************************************************************/
/*  Msgs for the extended unit address field in MOUNT        */
/*************************************************************/
QMSG(2175,"%c %c %-8.8s %8.8s %-8.8s %-8.8s %-8.8s "\
"%15.6f %-8.8s")
QMSG(2178,"%-6.6s %-8.8s %8.1d %c %-8.8s %8.8s "\
"%-8.8s %15.6f %-8.8s")
 
/*************************************************************/
/*  Msgs for the extended checking in the MOUNT command      */
/*************************************************************/
QMSG(2190," Volume in use on this system: %6.6s")
QMSG(2193," Request for volume %6.6s creates deadlock")
 
/*************************************************************/
/*  Msgs for the GENERIC and QUERY GENERIC commands          */
/*************************************************************/
QMSG(2200,"No Generic type for model %-6.6s library %-8.8s "\
"density %d system %-8.8s")
QMSG(2201,"Generic type for model %-6.6s library %-8.8s "\
"density %d system %-8.8s already exists")
QMSG(2202,"Model  Library  Density  SysType  Name  ")
QMSG(2203,"------ -------- -------- -------- ------")
QMSG(2204,"%-6.6s %-8.8s %8.8s %-8.8s %-6.6s")
QMSG(2205,"%.*s %-6.6s %-8.8s %8.8s %-8.8s")
 
/*************************************************************/
/*  Msgs for the FINDVIDS command                            */
/*************************************************************/
QMSG(2210,"FINDVIDS USERID %-8.8s VID %-6.6s TEXT   "\
"LIKE %.*s")
QMSG(2211,"FINDVIDS USERID %-8.8s VID %-6.6s BINARY "\
"LIKE %.*s")
QMSG(2212,"FINDVIDS USERID %-8.8s VID %-6.6s VOLINFO "\
"LIKE %.*s")
QMSG(2213,"FINDVIDS LIBRARY %-8.8s VID %-6.6s TEXT   "\
"LIKE %.*s")
QMSG(2214,"FINDVIDS LIBRARY %-8.8s VID %-6.6s BINARY "\
"LIKE %.*s")
QMSG(2215,"FINDVIDS LIBRARY %-8.8s VID %-6.6s VOLINFO "\
"LIKE %.*s")
QMSG(2216,"%-6.6s %.*s")
 
/*************************************************************/
/*  Msgs for the UPDATE command                              */
/*************************************************************/
QMSG(2400,"No new label information given")
QMSG(2401,"Label for VID %-6.6s: Type=%-3.3s;  VSN=%-6.6s;  "\
"Density=%8.8s")
 
/*************************************************************/
/*  Msgs for the DENSITY mapping command                     */
/*************************************************************/
QMSG(2500,"Model  Density  Code     P")
QMSG(2501,"------ -------- -------- -")
QMSG(2502,"%-6.6s %-8.8s %8.d %c")
QMSG(2503,"%.*s %-6.6s %-8.8s %8.d")
QMSG(2504,"Cannot update Primary Key Field")
QMSG(2505,"Mapping already exists")
QMSG(2506,"Mapping does not exist")
QMSG(2507,"Density %8.8s invalid for model %6.6s")
QMSG(2510,"Cannot delete Primary Entry")
QMSG(2511,"Cannot update code for Primary Entry")
 
/*************************************************************/
/*  Msgs for the LOCATION mapping command                    */
/*                                                           */
/*  Warning --> Location command relies on 2505 & 2506 above */
/*************************************************************/
QMSG(2520,"Code   Location Description")
QMSG(2521,"------ ------------------------------")
QMSG(2522,"%-6.d %.*s")
QMSG(2523,"%.*s %-6.d")
 
/*************************************************************/
/*  Msgs for the extended FEATURE QUERY ALERTEE command      */
/*************************************************************/
QMSG(9068,"%-8.8s at %-8.8s")
QMSG(9069,"%.*s %-8.8s AT %-8.8s")
 
/*************************************************************/
/*  Msgs for general Code debugging facilities               */
/*************************************************************/
QMSG(9900,"SQL Debug: %-8.8s  ->  %8.8x.%8.8x")
QMSG(9910,"PDate: %8.8d")
QMSG(9911,"PTime: %8.8d")
QMSG(9912,"DateTime: %f")
QMSG(9920,"Sqlcode: %d")
QMSG(9921,"__%s.%d")
QMSG(9922,"R%d code %d")
 
#endif
 
/* UPDATES APPLIED AT 09:36:29 ON 12 AUG 1993
UNITADDR * T. Cass      22 Mar 91 * Store MOUNT unit addr as char strin
*         APPLMESG UNITADDR D1 REL191  06/07/93   14:20:05
VOLINFO  * M. Ellerker   7 Jul 93 * Find Volinfo for user/lib
*         APPLMESG VOLINFO  D1 REL191  07/07/93   14:18:15
REORG10  * M. Ellerker  23 Jul 93 * Only allow GENERIC OR SYSTEM
*         APPLMESG REORG10  D1 REL191  08/06/93   14:30:20
DENMAP   * M. Ellerker   6 Aug 93 * Forgot this one !
*         APPLMESG DENMAP   D1 REL191  08/06/93   17:28:17
*/
\end{verbatim}

\Filename{H1fatmenappendix-summary-fatmen-system}
\chapter{Summary of the FATMEN system}

\begin{Tabhere}
\caption{FATMEN command line interface}
\begin{center}
\begin{tabular}{|>{\tt}p{5cm}r|>{\tt}p{5cm}r|}
\hline
\rm\bf Command           & \bf page                &
\rm\bf Command           & \bf page                \\
\hline
ALLOCATE                 & \pageref{ALLOCATE}      &
ADDDISK                  & \pageref{ADDDISK}       \\
ADDTAPE                  & \pageref{ADDTAPE}       &
CD                       & \pageref{CD}            \\
CLR                      & \pageref{CLR}           &
CP                       & \pageref{CP}            \\
COPY                     & \pageref{COPY}          &
DIR                      & \pageref{DIR}           \\
DUMP                     & \pageref{DUMP}          &
END                      & \pageref{END}           \\
EXIT                     & \pageref{EXIT}          &
EXTRACT                  & \pageref{EXTRACT}       \\
FC                       & \pageref{FC}            &
FIND                     & \pageref{FIND}          \\
GIME                     & \pageref{GIME}          &
INIT                     & \pageref{INIT}          \\
LD                       & \pageref{LD}            &
LOCK                     & \pageref{LOCK}          \\
LOGLEVEL                 & \pageref{LOGLEVEL}      &
LS                       & \pageref{LS}            \\
MAKE                     & \pageref{MAKE}          &
MEDIA                    & \pageref{MEDIA}         \\
MKDIR                    & \pageref{MKDIR}         &
MODIFY                   & \pageref{MODIFY}        \\
                         &                         &
MV                       & \pageref{MV}            \\
PWD                      & \pageref{PWD}           &
QUIT                     & \pageref{QUIT}          \\
RM                       & \pageref{RM}            &
RMDIR                    & \pageref{RMDIR}         \\
RMTREE                   & \pageref{RMTREE}        &
SEARCH                   & \pageref{SEARCH}        \\
TAG                      & \pageref{TAG}           &
TOUCH                    & \pageref{TOUCH}         \\
TREE                     & \pageref{TREE}          &
UNLOCK                   & \pageref{UNLOCK}        \\
UPDATE                   & \pageref{UPDATE}        &
VERSION                  & \pageref{VERSION}       \\
VIEW                     & \pageref{VIEW}          &
ZOOM                     & \pageref{ZOOM}          \\
SET/COPYLEVEL            & \pageref{SETCOPYLEVEL}  &
SET/MEDIATYPE            & \pageref{SETMEDIATYPE}  \\
SET/LOCATION             & \pageref{SETLOCATION}   &
SET/USERWORDS            & \pageref{SETUSERWORDS}  \\
SET/DATAREP              & \pageref{SETDATAREP}    &
SET/SOURCE               & \pageref{SETSOURCE}     \\
SET/DESTINATION          & \pageref{SETDESTINATION}& 
SHOW/SOURCE              & \pageref{SHOWSOURCE}    \\
SHOW/DESTINATION         & \pageref{SHOWDESTINATION}& 
SHOW/DATAREP             & \pageref{SHOWDATAREP}   \\
SHOW/COPYLEVEL           & \pageref{SHOWCOPYLEVEL} &
SHOW/LOCATION            & \pageref{SHOWLOCATION}  \\
SHOW/MEDIATYPE           & \pageref{SHOWMEDIATYPE} &
SHOW/USERWORDS           & \pageref{SHOWUSERWORDS} \\
\hline
\end{tabular}
\end{center}
\end{Tabhere}
\clearpage

\extrarowheight0pt
\begin{longtable}{|p{.9\linewidth}r|} 
\caption{FATMEN Routine calling sequences}                 \\
\hline
 \bf Function                           &                  \\
 \quad\bf Description                   & \bf page         \\
\hline
\endfirsthead
\caption[]{FATMEN Routine calling sequences (cont.)}       \\
\hline
 \bf Function                           &                  \\
 \quad\bf Description                   & \bf page         \\
\hline
\endhead
\hline
\endfoot
Initialise FATMEN system & \\
  \quad\small\tt CALL FMSTRT(LUNRZ,LUNFZ,CHFAT,IRC*) & \pageref{FMSTRT} \\
Access a dataset & \\
  \quad\small\tt CALL FMFILE(LUN,GENAM,CHOPT,IRC*) & \pageref{FMFILE} \\
Deaccess a dataset & \\
  \quad\small\tt CALL FMFEND(LUN,GENAM,CHOPT,IRC*) & \pageref{FMFEND} \\
Add a tape file & \\
  \quad\small\tt CALL FMADDT(argument-list) & \pageref{FMADDT} \\
Add a disk file & \\
  \quad\small\tt CALL FMADDD(argument-list) & \pageref{FMADDD} \\
Return information on FATMEN entry & \\
  \quad\small\tt CALL FMPEEK(GENAM,IVECT,CHOPT,IRC*) & \pageref{FMPEEK} \\
Add entry to catalogue & \\
  \quad\small\tt CALL FMPOKE(GENAM,IVECT,CHOPT,IRC*) & \pageref{FMPOKE} \\
Initialise FATMEN system & \\
  \quad\small\tt CALL FMINIT(IXSTOR*,LUNRZ,LUNFZ,DBNAME,IRC*) & \pageref{FMINIT} \\
Terminate FATMEN package & \\
  \quad\small\tt CALL FMEND(IRC*) & \pageref{FMEND} \\
Set logging level of FATMEN package & \\
  \quad\small\tt CALL FMLOGL(LEVEL) & \pageref{FMLOGL} \\
Control updating mode & \\
  \quad\small\tt CALL FMUPDT(MAX,NGROUP,IFLAG,IRC*) & \pageref{FMUPDT} \\
Purge old entries from catalogue & \\
  \quad\small\tt CALL FMPURG(PATH,KEYSEL,MAXSIZ,MINACC,MAXDAYS,MINCPS,LUNPUR,CHOPT,IRC*) & \pageref{FMPURG} \\
Get information on named file & \\
  \quad\small\tt CALL FMGET(GENAM,LBANK*,KEYS*,IRC*) & \pageref{FMGET} \\
Get information on named file with key selection & \\
  \quad\small\tt CALL FMGETK(GENAM,LBANK*,*KEYS*,IRC*) & \pageref{FMGETK} \\
Add entry to FATMEN catalogue & \\
  \quad\small\tt CALL FMPUT(GENAM,LBANK,IRC*) & \pageref{FMPUT} \\
Modify existing entry & \\
  \quad\small\tt CALL FMMOD(GENAM,LBANK,IFLAG,IRC*) & \pageref{FMMOD} \\
Create a new FATMEN bank & \\
  \quad\small\tt CALL FMBOOK(GENAM,KEYS*,LADDR*,*LSUP*,JBIAS,IRC*) & \pageref{FMBOOK} \\
Remove entry from FATMEN catalogue & \\
  \quad\small\tt CALL FMRM(GENAM,LBANK*,KEYS,IRC*) & \pageref{FMRM} \\
Make directory & \\
  \quad\small\tt CALL FMKDIR(CHDIR,IRC*) & \pageref{FMKDIR} \\
Set contents of FATMEN bank & \\
  \quad\small\tt CALL FMFILL(GENAM,*LBANK*,*KEYS*,CHOPT,IRC*) & \pageref{FMFILL} \\
Insert character data into FATMEN bank & \\
  \quad\small\tt CALL FMPUTC(LBANK,STRING,ISTART,NCH,IRC*) & \pageref{FMPUTC} \\
Read character data from FATMEN bank & \\
  \quad\small\tt CALL FMGETC(LBANK,STRING*,ISTART,NCH,IRC*) & \pageref{FMGETC} \\
Insert integer vector into FATMEN bank & \\
  \quad\small\tt CALL FMPUTV(LBANK,IVECT,ISTART,NWORDS,IRC*) & \pageref{FMPUTV} \\
Read integer vector from FATMEN bank & \\
  \quad\small\tt CALL FMGETV(LBANK,IVECT*,ISTART,NWORDS,IRC*) & \pageref{FMGETV} \\
Insert integer value into FATMEN bank & \\
  \quad\small\tt CALL FMPUTI(LBANK,IVAL,IOFF,IRC*) & \pageref{FMPUTI} \\
Read integer value from FATMEN bank & \\
  \quad\small\tt CALL FMGETI(LBANK,IVAL*,IOFF,IRC*) & \pageref{FMGETI} \\
Find existing dataset and associate with logical unit & \\
  \quad\small\tt CALL FMFIND(GENAM,DDNAME,*LBANK*,IRC*) & \pageref{FMFIND} \\
Create new dataset & \\
  \quad\small\tt CALL FMMAKE(GENAM,DDNAME,*LBANK*,IRC*) & \pageref{FMMAKE} \\
Open a dataset for read or write & \\
  \quad\small\tt CALL FMOPEN(GENAM,DDNAME,*LBANK*,CHOPT,IRC*) & \pageref{FMOPEN} \\
Close file opened via FATMEN & \\
  \quad\small\tt CALL FMCLOS(GENAM,DDNAME,LBANK,CHOPT,IRC*) & \pageref{FMCLOS} \\
Copy a dataset and update the FATMEN catalogue & \\
  \quad\small\tt CALL FMCOPY(GN1,*LBANK1*,*KEYS1*,GN2,*LBANK2*,*KEYS2*,CHOPT,IRC*) & \pageref{FMCOPY} \\
Check whether generic name already exists & \\
  \quad\small\tt CALL FMEXST(GENAM,IRC*) & \pageref{FMEXST} \\
List files in specified directory & \\
  \quad\small\tt CALL FMLS(GENAM,CHOPT,IRC*) & \pageref{FMLS} \\
Display contents of FATMEN bank & \\
  \quad\small\tt CALL FMSHOW(GENAM,*LBANK*,*KEYS*,CHOPT,IRC*) & \pageref{FMSHOW} \\
Count file names & \\
  \quad\small\tt CALL FMFILC(GENAM,NFILES*,IRC*) & \pageref{FMFILC} \\
Scan FATMEN directory structure & \\
  \quad\small\tt CALL FMSCAN(PATH,NLEVEL,UROUT,IRC*) & \pageref{FMSCAN} \\
Loop through FATMEN file names & \\
  \quad\small\tt CALL FMLOOP(GENAM,UROUT,IRC*) & \pageref{FMLOOP} \\
Return directory names in directory structure & \\
  \quad\small\tt CALL FMLDIR(PATH,DIRS*,NFOUND,MAXDIR,*ICONT*,IRC*) & \pageref{FMLDIR} \\
Return file names in directory structure & \\
  \quad\small\tt CALL FMLFIL(GENAM,FILES*,KEYS*,NFOUND,MAXFIL,JCONT,IRC*) & \pageref{FMLFIL} \\
Sort file names and keys & \\
  \quad\small\tt CALL FMSORT(FILES,KEYS,NFILES,JSORT*,IRC*) & \pageref{FMSORT} \\
Match file name against pattern & \\
  \quad\small\tt CALL FMATCH(CHFILE,MATCH,IRC*) & \pageref{FMATCH} \\
Match multiple names against pattern & \\
  \quad\small\tt CALL FMMANY(MATCH,FILES,NFILES,NMATCH*,IRC*) & \pageref{FMMANY} \\
Print contents of FATMEN keys vector & \\
  \quad\small\tt CALL FMPKEY(KEYS,NKEYS) & \pageref{FMPKEY} \\
Select files using the FATMEN keys & \\
  \quad\small\tt CALL FMSELK(GENAM,INKEYS,OUKEYS*,NKEYS*,MAXKEY,IRC*) & \pageref{FMSELK} \\
Select files using the FATMEN bank information & \\
  \quad\small\tt CALL FMSELB(GENAM,INKEYS,NKEYS,UEXIT,ISEL*,IRC*) & \pageref{FMSELB} \\
Select files using keys matrix & \\
  \quad\small\tt CALL FMSELM(GENAM,LBANK*,KEYS*,KEYM,NKEY,CHOPT,IRC*) & \pageref{FMSELM} \\
Compare FATMEN entries & \\
  \quad\small\tt CALL FMCOMP(GENAM1,*LBANK1*,*KEYS1*,GENAM2,*LBANK2*,*KEYS2*,IRC*) & \pageref{FMCOMP} \\
Print user words and comment & \\
  \quad\small\tt CALL FMUPRT(GENAM,LBANK,KEYS,IUSER,COMM,IRC*) & \pageref{FMUPRT} \\
User selection & \\
  \quad\small\tt CALL FMUSEL(GENAM,LBANK,KEYS,IRC*) & \pageref{FMUSEL} \\
Allocate new piece of media & \\
  \quad\small\tt CALL FMALLO(MEDIA,DENS,COMPACT,LIB,POOL,LBANK,CHOPT,VSN*,VID*,IRC*) & \pageref{FMALLO} \\
Move volumes between TMS pools & \\
  \quad\small\tt CALL FMPOOL(GENAM,LBANK,KEYS,CHPOOL,CHOPT,IRC) & \pageref{FMPOOL} \\
Obtain volume characteristics & \\
  \quad\small\tt CALL FMQVOL(GENAM,LBANK,KEYS,LIB*,MODEL*,DENS*,MNTTYP*,LABTYP*,IRC*) & \pageref{FMQVOL} \\
Obtain media information & \\
  \quad\small\tt CALL FMQMED(GENAM,*LBANK*,*KEYS*,IMEDIA*,IROBOT*,IRC*) & \pageref{FMQMED} \\
Set default media information & \\
  \quad\small\tt CALL FMEDIA(MFMMED,MFMTYP,MFMGEN,MFMSIZ,MFMDEN,MFMMNT,MFMLAB,NMEDIA,IRC*) & \pageref{FMEDIA} \\
Get, Set or Delete TMS Tags & \\
  \quad\small\tt CALL FMTAGS(GENAM,*LBANK*,*KEYS*,*CHTAG*,CHOPT,IRC*) & \pageref{FMTAGS} \\
Declare location codes to FATMEN & \\
  \quad\small\tt CALL FMSETL(LOC,NLOC,IRC*) & \pageref{FMSETL} \\
Declare media types to FATMEN & \\
  \quad\small\tt CALL FMSETM(MTP,NMTP,IRC*) & \pageref{FMSETM} \\
Declare copy levels to FATMEN & \\
  \quad\small\tt CALL FMSETC(CPL,NCPL,IRC*) & \pageref{FMSETC} \\
Declare selection matrix and options to FATMEN & \\
  \quad\small\tt CALL FMSETK(KEYM,NK,CHOPT,IRC*) & \pageref{FMSETK} \\
Modify user words & \\
  \quad\small\tt CALL FMMODU(PATH,UFORM,UVECT,UCOMM,CHOPT,IRC*) & \pageref{FMMODU} \\
Declare logical units to FATMEN & \\
  \quad\small\tt CALL FMSETU(LUN,NLUN,IRC*) & \pageref{FMSETU} \\
Get a free logical unit & \\
  \quad\small\tt CALL FMGLUN(LUN*,IRC*) & \pageref{FMGLUN} \\
Free a logical unit & \\
  \quad\small\tt CALL FMFLUN(LUN*,IRC*) & \pageref{FMFLUN} \\
Verify bank contents & \\
  \quad\small\tt CALL FMVERI(GENAM,LBANK,KEYS,CHOPT,IRC*) & \pageref{FMVERI} \\
Pack date and time. & \\
  \quad\small\tt CALL FMPKTM(IDATE,ITIME,IPACK*,IRC*) & \pageref{FMPKTM} \\
Unpack date and time. & \\
  \quad\small\tt CALL FMUPTM(IDATE*,ITIME*,IPACK,IRC*) & \pageref{FMUPTM} \\
Pack date and time for VAX format. & \\
  \quad\small\tt CALL FMPKVX(CHDATE,IDATE,ITIME,IPACK*,IRC*) & \pageref{FMPKVX} \\
Unpack date and time for VAX format. & \\
  \quad\small\tt CALL FMUPVX(CHDATE*,IDATE*,ITIME*,IPACK,IRC*) & \pageref{FMUPVX} \\
Return file names in specified directory & \\
  \quad\small\tt CALL FMFNMS(PATH,FILES*,KEYS*,NKEYS*,MAXKEY,IRC*) & \pageref{FMFNMS} \\
Return file names in directory structure & \\
  \quad\small\tt CALL FMLIST(PATH,FILES*,KEYS*,NFOUND,MAXFIL,IRC*) & \pageref{FMLIST} \\
Obtain names of subdirectories in specified tree & \\
  \quad\small\tt CALL FMTREE(PATH,SUBDIR*,NLEVEL,NFOUND*,MAXDIR,IRC*) & \pageref{FMTREE} \\
User routine to allocate new piece of media & \\
  \quad\small\tt CALL FUALLO(MEDIA,VSN*,VID*,IRC*) & \pageref{FUALLO} \\
Create a new FATMEN bank & \\
  \quad\small\tt CALL FMLIFT(GENAM,KEYS*,MEDIA,CHOPT,IRC*) & \pageref{FMLIFT} \\
Get the address of a FATMEN bank & \\
  \quad\small\tt CALL FMLINK(GENAM,LBANK*,CHOPT,IRC*) & \pageref{FMLINK} \\
Obtain volume characteristics & \\
  \quad\small\tt CALL FMQTMS(VID,LIB*,MODEL*,DENS*,MNTTYP*,LABTYP*,IRC*) & \pageref{FMQTMS} \\
\end{longtable}
 
\clearpage

\label{BANK-OFFSETS}
\extrarowheight2pt
%\arraycolsep3pt
\newlength{\Tonecol}
\setlength{\Tonecol}{\linewidth}
\addtolength{\Tonecol}{-\arraycolsep}
\addtolength{\Tonecol}{-\arrayrulewidth}
\newlength{\myla}\settowidth{\myla}{\bf Date and timexx}
\newlength{\mylb}\setlength{\mylb}{\linewidth}
\addtolength{\mylb}{-\myla}
\addtolength{\mylb}{-.21\linewidth}
\addtolength{\mylb}{-6\arraycolsep}
\addtolength{\mylb}{-4\arrayrulewidth}
\begin{longtable}{|p{\myla}|>{\tt}p{.21\linewidth}|p{\mylb}|}
\caption{Bank offsets and datatypes}                                     \\ 
\hline
\multicolumn{1}{|c}{\bf Function}    &
\multicolumn{1}{c}{\bf Data type}   &
\multicolumn{1}{c|}{\bf Description}                                     \\
\hline
\endfirsthead
\caption[]{Bank offsets and datatypes (cont.)}                           \\
\hline
\multicolumn{1}{|c}{\bf Function}    &
\multicolumn{1}{c}{\bf Data type}   &
\multicolumn{1}{c|}{\bf Description}                                     \\
\hline
\endhead
\hline
\endfoot
\bf Key Definitions               &
\tt MKSRFA (I*4)                  &
An integer assigned by the FATMEN system to permit differentiation between
entries with the same generic name.
It is not normally specified by the user, unless a particular entry is
required, e.g. in an {\tt RM} command.                                    \\
\cline{2-3}
& \tt MKFNFA (H*20)              &
The part of the generic name following the last slash, such as
{\tt RUN123} in the name {\tt //CERN/DELPHI/TEST/RUN123}                  \\
\cline{2-3}
& \tt MKCLFA (I*4)               &
The 'copy level', where 0=original, 1=copy, 2=copy of a copy etc.
Some experiments prefer to use the copy level to indicate the data
representation of the data pointed to by that entry, where {\tt 1=IEEE},
{\tt 2=IBM}, {\tt 3=VAX}, {\tt 4=byte-swapped IEEE} (e.g. DECstation),
{\tt 5=CRAY} \\
\cline{2-3}
& \tt MKLCFA (I*4)               &
The location code, which is used to find the nearest or most
suitable copy of a dataset. The location code corresponds to a site or LAN,
because it is only at this level that a meaningful choice can be made.
Thus it is only a first level selection, to filter out all
opies at a given location, such as CERN, RAL etc.                         \\
\cline{2-3}
& \tt MKMTFA (I*4)               &
The type of medium on which the dataset resides. \newline
{\tt 1=DISK, 2=3480, 3=3420, 4=8200 (Exabyte)}.                           \\
\hline
\bf Generic file description    &
\tt MFQNFA (H*256)$^{*+}$         &
This points to the fully qualified dataset name that the file has on
disk or tape. For VM files, the file name should be encoded in the
form {\tt <user.addr>fn.ft} where {\tt addr}, if omitted, defaults to 191.\\
\cline{2-3}
& \tt MHSNFA (H*8)$^{*+}$        &
The host on which the file resides, or through which it is accessed,
in the case of a tape file.                                               \\
\cline{2-3}
& \tt MCPLFA (I*4)$^{*+}$        &
The 'copy level', where 0=original, 1=copy, 2=copy of a copy etc.         \\
\cline{2-3}
& \tt MLOCFA (I*4)$^{*+}$        &
The location code, which is used to find the nearest or most
suitable copy of a dataset.The location code corresponds to a site or LAN,
because it is only at this level that a meaningful choice can be made.
Thus it is only a first level selection, to filter out all
copies at a given location, such as CERN, RAL etc.                        \\
\cline{2-3}
& \tt MMTPFA (I*4)$^{*+}$        &
The type of medium on which the dataset resides. \newline
{\tt 1=DISK, 2=3480, 3=3420, 4=8200 (Exabyte)}.                           \\
\hline
\bf Disk description             &
\tt MHSTFA (H*16)$^+$            &
The host type, e.g. IBM 3090.
This field is filled in by the FATMEN system.                             \\
\cline{2-3}
& \tt MHOSFA (H*12)$^+$          &
The host operating system, such as {\tt VM/XA CMS 5.5.}
This field is filled in by the FATMEN system.                             \\
\hline
\bf Tape description             &
\tt MVSNFA (H*8)$^*$             &
The {\bf volume serial number} or magnetically recorded label on the tape.\\
%\cline{2-3}
& \tt MVIDFA (H*8)$^*$           &
The {\bf visual identifier} or contents of the sticky label on the tape.  \\
\cline{2-3}
& \tt MVIPFA (I*4)               &
The VID prefix - internal representation.                                 \\
\cline{2-3}
& \tt MDENFA (I*4)$^+$           &
The density of the medium, such as 6250, 38000.
Normally, such details are retained by the Tape Management System.        \\
\cline{2-3}
& \tt MVSQFA (I*4)$^+$           &
The volume sequence number, only used for multi-volume files.
(Not yet supported).                                                      \\
\cline{2-3}
& \tt MFSQFA (I*4)$^*$           &
The file sequence number, for multi-file tapes.                           \\
\hline
\bf File description\footnotemark[1]
                                 &
\tt MSRDFA (I*4)                 & The start record in the file.          \\
\cline{2-3}
& \tt MERDFA (I*4)               & The end record in the file.            \\
\cline{2-3}
& \tt MSBLFA (I*4)               & The start block in the file.           \\
\cline{2-3}
& \tt MEBLFA (I*4)               & The end block in the file.             \\
\hline
\bf Logical description           &
\tt MFLFFA (H*4)$^{*+}$          &
The FATMEN defined file format.
The following definitions are recognised by the FATMEN system:
\begin{DLtt}{MMMM}
\item[FA]ZEBRA ASCII
\item[FZ]ZEBRA native
\item[FXN]ZEBRA native, exchange file format
\item[FX]ZEBRA exchange
\item[FFX]ZEBRA exchange, FORTRAN I/O
\item[RZ]ZEBRA RZ
\item[RX]ZEBRA RZ, exchange format
\item[EP]EPIO
\item[AS]normal ASCII
\item[DA]FORTRAN direct access dataset
\item[UN]unknown
\item[FPT]FPACK text
\item[FPS]FPACK sequential-access
\item[FPD]FPACK direct-access
\item[FPK]FPACK keyed-access
\item[FPO]FPACK ordered
\item[YBB]YBOS binary
\item[YBD]YBOS direct-access
\end{DLtt}
\vspace*{-\baselineskip}                                                   \\
\cline{2-3}
& \tt MFUTFA (H*4)               & The user defined file type.            \\
\hline
\bf Physical description         &
\tt MRFMFA (H*4)                 &
The record format, e.g. {\tt VBS, FB} etc.                                \\
\cline{2-3}
& \tt MRLNFA (I*4)               & The record length, in 4-byte words.    \\
\cline{2-3}
& \tt MBLNFA (I*4)               & The block length, in 4-byte words.     \\
%\cline{2-3}
& \tt MFSZFA (I*4)               &
The file size in Megabytes, rounded up to the next integer.
A Megabyte is defined as {\tt (1,024)**2} bytes.                          \\
\cline{2-3}
& \tt MUSCFA (I*4)               &
The number of times that the file has been accessed (use count).
A call to FMOPEN with the option W, or to FMMAKE sets the use count to 1.
A call to FMOPEN with the option R, or to FMFIND increments the count.    \\
\hline
\bf Date and time\footnotemark[2] &
  \tt MCRTFA (I*4)$^+$           & Creation date and time.                \\
\cline{2-3}
& \tt MCTTFA (I*4)$^+$           & Date and time of cataloging.           \\
\cline{2-3}
& \tt MLATFA (I*4)               &
Date and time of last access (when available).                            \\
\hline
\bf Creator information           &
 \tt MCURFA (H*8)$^+$            & The user name of the creator.          \\
\cline{2-3}
& \tt MCIDFA (H*8)$^+$           & The account
({\tt UUUGG} or {\tt UUU\$GG}) of the creator (CERNVM or VAX/VMS).
For Unix systems, the NFSID is used.                                      \\
\cline{2-3}
& \tt MCNIFA (H*8)$^+$           & The node on which the file was created.
This is not necessarily the node from which it was catalogued.            \\
\cline{2-3}
& \tt MCJIFA (H*8)$^+$           &
The job or process name which created the file.                           \\
\hline
\bf File protection               & \tt MFPRFA (I*4)                &
The file protection flag, currently unused.                               \\
\hline
\bf System area                   & \tt MSYWFA(10) (I*4)            &
Pointer to the first of ten system defined words
(reserved for future use).                                                \\
\hline
\bf User area                    & \tt MUSWFA(10) (I*4)            &
Pointer to the first of ten user defined words.                           \\
\cline{2-3}
& \tt MUCMFA (H*80)$^+$          & A user-defined comment.                \\
\hline
\multicolumn{3}{|p{\the\Tonecol}|}{
\small
\begin{flushleft}
\quad $^+$\ Items filled
in automatically by the routine \Rind{FMBOOK} or \Rind{FMLIFT} (See
pages~\pageref{FMBOOK} and \pageref{FMLIFT}).\\
\quad $^*$\ Mandatory items (defined as \Lit{NOT NULL} 
in the ORACLE/SQL tables).\\
All fields may be overwritten
by the user, but care must be exercised to ensure that the new
values are coded in the correct format.
In particular, for hollerith
values, the entire field must first be set to blanks to avoid problems
if the new value is shorter than the default setting.
The routines \Rind{FMPUTC} and \Rind{FMPUTI} (see on Page~\pageref{FMPUTC} and
on Page~\pageref{FMPUTI}) may be used to simplify bank modification.
See the routine on Page~\pageref{FMVERI} for information on how
to check the contents of a bank.\\[5mm]
{\bf Notes}\\
\quad\footnotesize$^1$\ The file description information will
normally be left as zero, or set to the first and last record and block
numbers. They could also be used to indicate a subset of a file.\\
\quad$^2$\ The date  and time fields are stored
packed, using routine \Rind{FMPKTM}. 
They can be unpacked using routine \Rind{FMUPTM}.
\end{flushleft}}
\end{longtable}
\footnotetext[1]{The file description information will
normally be left as zero, or set to the first and last record and block
numbers. They could also be used to indicate a subset of a file.}
\footnotetext[2]{The date  and time fields are stored
packed, using routine \Rind{FMPKTM}. 
They can be unpacked using routine \Rind{FMUPTM}.}

\clearpage

\subsection{Disk file name}
\index{disk filename}
\par
The disk file name should be entered in the format of the host
operating system. For VM/CMS systems, the convention
{\tt <user.address> filename.filetype} is used.
If {\tt address} is omitted, {\tt user} may then be any
valid entry in a {\tt GIMEUSER}, {\tt GIMEGRP} or {\tt GIMESYS} names file.
If not found in a GIME names file, the default address 191 will be used.
\par
An extract from a GIME names file is given below:
\begin{XMP}
:NICK.L3DSTS  :USERID.L3MAXI
              :ADDR.222
              :LINKMODE.RR
              :FILEMODE.E
\end{XMP}
Using this example, files such as {\tt<L3DSTS>RUN1.DST} would be assumed
to be on the 222 disk of user L3MAXI.
\par
For VAX/VMS systems, the full file name, including disk and directory
names should be entered.
\subsection{Disk files and VAXclusters}
\index{VAXcluster}
\index{cluster}
\index{cluster alias}
\index{DECnet cluster alias}
\par
If the FATMEN software finds that the current node is in the
same cluster as the node specified in the FATMEN catalogue
and the disk specified is available, it will treat the entry
as if the file were on the current node.
\begin{XMPt}{Example of valid disk file names for the cluster VXCERN}
*     Create a valid entry in an existing FATMEN bank
      CALL UCTOH('VXCRNA',IQ(LFAT+MHSNFA),4,6)
*     or we could use the VAXcluster alias rather than current node name
      CALL UCTOH('VXCERN',IQ(LFAT+MHSNFA),4,6)
*     Create a valid disk file name
*     Note that logical name for disk is unique and consistant HEP-wide!
      CALL UCTOH('DELPHI$DK123:<DELPROD.MCDSTS>RUN567',
     +            IQ(LFAT+MFQNFA,4,35)
\end{XMPt}

\newpage

\subsection{Disk files and DFS or NFS}
\index{DFS}
\index{NFS}

The FATMEN system can use DFS\cite{bib-DFS} or
NFS\cite{bib-NFS} to access a remote disk file,
if the disk or file system on which it resides is mounted locally.
Users are strongly recommended to adopt a convention for DFS and NFS
names, so that the same name is used throughout a LAN. For DFS it is
recommended that the disk in question have the same logical name
on the system to which it is directly attached as on those
where it is mounted via DFS. Thus, a disk on the central cluster
with logical name VXCERN\$DISK999 should be mounted remotely via
DFS with the same logical name. If the FATMEN system finds a DFS
device of the corresponding name on the local system, it will
assume that the physical device to which it points is indeed the correct
one. {\bf N.B. the FATMEN software currently considers only the
'system' logical name table ({\tt LNM\$SYSTEM\_TABLE}).}
\begin{XMPt}{Example of mounting a remote disk via DFS}
$!Mount the VXCERN disk DELPHI$DK123
$!The 'access-point' is defined on VXCERN and points
$!to the disk in the previous example. This will allow
$!users on the current node to access data on this disk
$!that is catalogued in FATMEN transparently.
$RUN SYS$SYSTEM:DFS$CONTROL
DFS>mount access-point DELPHI$DK123 /SYSTEM
DFS>exit
\end{XMPt}
\par
If the dataset name in the FATMEN entry (MFQNFA) starts with a \$,
then on Unix systems FATMEN will assume that the text that follows
up to the first slash character is an environmental variable and 
attempt to translate it. An example is given below.
\begin{XMP}
 FM> ls C01 -gn # List full generic name and corresponding dataset name
 
 //CERN/OPAL/DDST/PASS4/FYZ1/P19R1973/C01
 Fileid:      $OPALDATA/fyz1/p19r1973.c01
 
 Files:    1
 FM>quit

[zfatal] > echo $OPALDATA                      
/u/ws/ddst
[zfatal] >
[zfatal] >
[zfatal] > telnet shift1

...

shift1 [2] printenv OPALDATA
/user/ws/data
shift1 [3]
\end{XMP}
\subsection{Location code}
\index{location code}
\par
This is an integer code indicating the LAN or site where the
data resides. Currently, the only supported LAN is CERN, for which
the location code is defined as 1. This code is used by the FATMEN
system for a first pass selection of the best copy of a data set.
The correspondance between the location code and the LAN will be
stored in the FATMEN database itself.
\subsection{Username (MCURFA)}
\index{username (MCURFA)}
\par
This is obtained through QUERY USERID on VM/CMS systems, the
\$GETJPI system service on VAX/VMS and the getpwuid function on Unix.
\subsection{Jobname (MCJIFA)}
\index{jobname (MCJIFA)}
\par
This is obtained through JOBNAM (CERN Program Library Entry Z100)
on VM/CMS systems, the \$GETQUI system service on VAX/VMS and the
getpwuid function on Unix.
\subsection{Account (MCIDFA)}
\index{account (MCIDFA)}
\par
This is obtained through the QUERY ACCOUNT command on VM/CMS, the \$GETUAI
system service on VAX/VMS or the getuid function on Unix.
\subsection{Host name (MHSTFA, MCNIFA)}
\index{host name (MHSTFA, MCNIFA)}
\par
This is obtained through the QUERY USERID command on VM/CMS and the
\$GETSYI system service on VAX/VMS.
\subsection{Host type (MHSTFA)}
\index{host type (MHSTFA)}
\par
This is obtained through the QUERY CMSLEVEL command on VM/CMS and
the \$GETSYI system service on VAX.
\subsection{Host Operating System (MHOSFA)}
\index{host operating system (MHOSFA)}
\par
This is obtained using the CP QUERY USERID command on VM/CMS
systems and the \$GETSYI system service on VAX/VMS machines.

