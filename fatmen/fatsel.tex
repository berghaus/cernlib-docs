\section{Routines to tailor FATMEN selection}
\par
By default, FATMEN makes no check on location code or copy
level when accessing a dataset or listing a catalogue entry.
When attempting to access a dataset, FATMEN searches
for a copy first on disk, then 3480 cartridge, and so on.
The following routines may be used to modify the selection
procedure.
\par
If a call to FMSETL or FMSETC is made, only entries with
a location code or copy level that match one of those
specified will be accessible (unless the key serial number
is given explicitly) or visible via a call to FMLS.
\par
The routine FMSETM permits not only the range of medium
types to be set but also the order.
For example, a call to FMSETM with the following values
will limit the selection to medium types 3, 5, 1 and 2,
in that order.
\begin{XMP}
      NMTP = 4
      MTP(1) = 3
      MTP(2) = 5
      MTP(3) = 1
      MTP(4) = 2
\end{XMP}
\par
Alternatively, the routine FMSETK can be used. This allows the
user to declare a keys matrix and character option which will
be used in all future file selections. That is, it overrides
both the FATMEN default selection and that set by the routines
FMSETC, FMSETL and FMSETM.
\par
The difference between these three approaches is explained below.
With the default FATMEN selection, no check is made on copy level
or location code. FATMEN takes in turn media types 1 to 4 and
looks for an accessible dataset. The first one found is taken,
with the proviso that robotically mounted tapes are given preference
over manually mounted tapes.
\par
The FMSETC, FMSETL and FMSETM routines allow the user to specify
allowed values for the copy level and location code, and the
order in which the different media types are to be processed.
This allows media types that are not in the default selection
to be processed, and the order to be changed. 
However, the user is unable to indicate that, for example,
a location code or 31 is preferable to one of 21, within
a given media type.
\par
The FMSETK routine provides the user with more control.
One might wish to set the order to be native mode copy
on disk, native mode copy on 3480 robotically mounted
cartridge, exchange mode copy on disk, exchange mode
copy on 3480 cartridge. 
\subsection{Declare location codes to FATMEN}
\Sbox{FMSETL}{CALL FMSETL(LOC,NLOC,IRC*)}
\index{Location code}
\begin{DLtt}{1234567}
\item[LOC]
Array of length NLOC containing a list of location codes
to be used in FATMEN selection of an entry.
\item [NLOC]
The number of location codes declared, maximum 99
\item[IRC]
Integer variable in which the return code is returned.
\end{DLtt}
\subsection{Declare media types to FATMEN}
\Sbox{FMSETM}{CALL FMSETM(MTP,NMTP,IRC*)}
\index{Media type}
\begin{DLtt}{1234567}
\item[MTP]
Array of length NMTP containing a list of media types
to be used in FATMEN selection of an entry.
\item[NMTP]
The number of media types declared, maximum 99
\item[IRC]
Integer variable in which the return code is returned.
\end{DLtt}
\subsection{Declare copy levels to FATMEN}
\Sbox{FMSETC}{CALL FMSETC(CPL,NCPL,IRC*)}
\index{Copy level}
\begin{DLtt}{1234567}
\item[CPL]
Array of length NCPL containing a list of copy levels
to be used in FATMEN selection of an entry.
\item[NCPL]
The number of copy levels declared, maximum 99
\item[IRC]
Integer variable in which the return code is returned.
\end{DLtt}
\subsection{Declare selection matrix and options to FATMEN}
\Sbox{FMSETK}{CALL FMSETK(KEYM,NK,CHOPT,IRC*)}
\index{Key selection}
\index{Key matrix}
\begin{DLtt}{1234567}
\item[KEYM]
Matrix of size (LKEYFA,NK) containing an ordered
list of key vectors 
to be used in FATMEN selection of an entry.
\item[NK]
The number of copy levels declared, maximum 99
\item[CHOPT]
The options to be used when selecting a dataset,
as defined by the routine FMSELM.
\item[IRC]
Integer variable in which the return code is returned.
\end{DLtt}
\par
See the description of the FMSELM routine for more information
on how the key matrix is used in dataset selection.
\subsection{Select files using keys matrix}
\Sbox{FMSELM}{CALL FMSELM(GENAM,LBANK*,KEYS*,KEYM,NKEY,CHOPT,IRC*)}
\begin{DLtt}{1234567}
\item[GENAM]
Character variable of maximum length 255 to specify the generic name.
\item[LBANK]
Integer variable in which the address of the FATMEN bank is returned.
\item[KEY]
Integer array of length 10 in which the keys vector associated with
the bank at LBANK is returned.
\item[KEYM]
Integer matrix of size (10,NKEY) containing the ordered selection
criteria.
\item[NKEY]
Integer variable to specify the second dimension of
the matrix KEYM.
\item[CHOPT]
Character variable containing the user specified options.
\item[IRC]
Integer variable in which the return code is returned.
\end{DLtt}
\par
This routine takes each row of the input matrix KEYM in turn,
and attempts to find a matching dataset. If no matching dataset
is found, it then proceeds to the next row.
Currently, checks are only made on media type (MKMTFA), location code
(MKLCFA) and copy level (MKCLFA). A value of -1 indicates that no
check on this item is to be made.
\par
The following options are available:
\begin{XMP}
   I - issue FORTRAN inquire for disk datasets (with full support
       for SHIFT files, Unix environmental variables, VM mini-disks,
       SFS pools etc.)
   N - check host name for disk datasets against current host
   M - require that tape datasets reside on manually mounted volumes
   R - require that tape datasets reside on robotically mounted volumes
\end{XMP}
\begin{XMPt}{Example of using the FMSELM routine}
*     Argument declarations
      PARAMETER (LKEYFA=10)
      PARAMETER (NKEY=3)
      DIMENSION KEYS(LKEYFA),KEYM(LKEYFA,NKEY)
*
*     Define search criteria: first, disk dataset in location 31,
*        with no check on copy level,
*                             next, dataset on media type 2 in location 1,
*                             next, dataset on any media in any location 
*                             with copy level 1
*
      KEYM(MKMTFA,1) = 1
      KEYM(MKCLFA,1) = -1
      KEYM(MKLCFA,1) = 31

      KEYM(MKMTFA,2) = 2
      KEYM(MKCLFA,2) = -1
      KEYM(MKLCFA,2) = 1

      KEYM(MKMTFA,3) = -1
      KEYM(MKCLFA,3) = 1
      KEYM(MKLCFA,3) = -1
*
*     Options NM: No check on host name for disk files,
*                 Manually mounted tape required
*
      CALL FMSELM('//CERN/OPAL/DDST/PASS3/FYZ1/P20/R02222C01',
     +   LBANK,KEYS,KEYM,NK,'NM',IRC)

      PRINT *,'Return code from FMSELM = ',IRC

      IF(IRC.EQ.0)
     +   CALL FMSHOW('//CERN/OPAL/DDST/PASS3/FYZ1/P20/R02222C01',
     +                LBANK,KEYS,'A',IRC)


\end{XMPt}
