%%%%%%%%%%%%%%%%%%%%%%%%%%%%%%%%%%%%%%%%%%%%%%%%%%%%%%%%%%%%%%%%%%%
%                                                                 %
%   FATMEN User Guide and Reference manual                        %
%                                                                 %
%   Fatmen Master driving LaTeX source file                       %
%                                                                 %
%   This document needs the following external EPS files:         %
%   See the respective source files which are included            %
%                                                                 %
%   Editor: Michel Goossens / CN-AS                               %
%   Last Mod.: 12 Sep  1991   mg                                  %
%                                                                 %
%%%%%%%%%%%%%%%%%%%%%%%%%%%%%%%%%%%%%%%%%%%%%%%%%%%%%%%%%%%%%%%%%%%
\documentstyle[epsf,longtable,screen]{cernman}
 
\setlongtables
\newcommand{\Example}{\par{\bf Example:}\par}% Example
\newcommand{\Examples}{\par{\bf Examples:}\par}% Examples
\renewcommand{\Remark}{\par{\bf Remark:}\par}% Reamrks
\newcommand{\Remarks}{\par{\bf Remarks:}\par}% Reamrks
\renewcommand{\Iind}[1]{\protect\index{HIDOPT!referenced}
\protect\index{Options of HIDOPT!#1}\protect\index{#1 option}}
\renewcommand{\Rind}[1]{{\tt\bf#1}\protect\index{#1!referenced}}
\newcommand{\Rindno}[1]{\protect\index{#1!referenced}}% no print
\renewcommand{\Rdef}[1]{{\tt\bf#1}\protect\index{#1!defined}}
 
\newcommand{\Sboxz}[2]% #1 abbreviation #2 contents(no definition)
{\par\leavevmode\vspace*{\baselineskip}
\fbox{\rule[-1ex]{0mm}{3ex}\quad\tt\bf#2\quad}
\par\index{#1!referenced}
}% ***** end of \newcommand{\Sbox}
 
\renewcommand{\Sbox}[2]% #1 abbreviation #2 contents
{\par\leavevmode\vspace*{\baselineskip}
\fbox{\rule[-1ex]{0mm}{3ex}\quad\tt\bf#2\quad}
\par\label{#1}\index{#1!defined}\Rkeep{#1}{#2}%
}% ***** end of \newcommand{\Sbox}
 
% Definition of two routines in same box separated by ``and''
\newcommand{\Sboxii}[4]% #1 abb. 1 #2 abb. 2 #3 contents 1 #4 cont. 2
{\par\leavevmode\vspace*{\baselineskip}
\fbox{\rule[-1ex]{0mm}{3ex}\quad\tt\bf#3\quad{\rm and}\quad#4\quad}
\par\label{#1}\index{#1!defined}\label{#2}\index{#2!defined}%
\Rkeep{#1}{#3}\Rkeep{#2}{#4}%
}% ***** end of \newcommand{\Sboxii}
 
%%%%%%%%%%%% Environment DL   %%%%%%%%%%%%%%%%%%%%%%%%%%%%%%%%%%%%%
\renewenvironment{DL}[1]{% The parameter is the width of the term
\begin{list}{\phantom{\bf#1\quad}}
   {\settowidth{\labelwidth}{\bf#1\quad}% define global width
    \setlength{\leftmargin}{\labelwidth}% set global width
    \setlength{\itemindent}{0pt}% set global width
    \setlength{\labelsep}{5pt}% horizontal separation between term and item
    \setlength{\itemsep}{1pt}% vertical separation between two items
    \setlength{\parsep}{0pt}% vertical separation two paragraphs in an item
    \setlength{\topsep}{.2\baselineskip}% vertical separation text/list
    \renewcommand{\makelabel}[1]{\bf##1\hfil}}}% get parameter from item
{\end{list}}% ***** end of environment{DL}
 
%%%%%%%%%%%% Font definitions (NFSS) %%%%%%%%%%%%%%%%%%%%%%%%%%%%%%%%%
\ifx\selectfont\undefined
\newcommand{\Ttt}{\tt}
\newcommand{\Stt}[2]{\tt}
\else
% NFSS definitions
\let\oldtt\tt
\def\tt{\oldtt\series{c}\size{9}{11pt}\selectfont}
\newcommand{\Ttt}{\tt}
\newcommand{\Stt}[2]{\oldtt\size{#1}{#2}\selectfont}
\renewcommand{\ttbf}% take one size smaller than current font
{{\tt\size{9}{11pt}\selectfont}}%
\fi
 
\setcounter{secnumdepth}{1}
\setcounter{tocdepth}{2}
 
\begin{document}
%  ==================== Front material ============================
%%%%%%%%%%%%%%%%%%%%%%%%%%%%%%%%%%%%%%%%%%%%%%%%%%%%%%%%%%%%%%%%%%%
%                                                                 %
%   FATMEN User Guide and Reference manual                        %
%                                                                 %
%   Front Material                                                %
%                                                                 %
%   Editor: Michel Goossens / CN-AS                               %
%   Last Mod.:  2 Nov 1993 17:40 mg                               %
%                                                                 %
%%%%%%%%%%%%%%%%%%%%%%%%%%%%%%%%%%%%%%%%%%%%%%%%%%%%%%%%%%%%%%%%%%%

%%%%%%%%%%%%%%%%%%%%%%%%%%%%%%%%%%%%%%%%%%%%%%%%%%%%%%%%%%%%%%%%%%%%
%    Tile page                                                     %
%%%%%%%%%%%%%%%%%%%%%%%%%%%%%%%%%%%%%%%%%%%%%%%%%%%%%%%%%%%%%%%%%%%%
\def\Ptitle#1{\special{ps: /Printstring (#1) def}
                       \epsfbox{cnastit.eps}}
 
\begin{titlepage}
\vspace*{-23mm}%
\mbox{\epsfig{file=/usr/local/lib/tex/ps/cern15.eps,height=30mm}}%
\hfill
\raise8mm\hbox{\Large\bf CERN Program Library Long Writeups Q123}
\hfill\mbox{}
\begin{center}
  \mbox{}\\[6mm]
  \mbox{\Ptitle{FATMEN}}\\[2cm]
  {\LARGE Distributed}\\[1cm]
  {\LARGE File and Tape Management System}\\[2cm]
  {\LARGE Version 1.90}\\[3cm]
  {\Large Application Software Group}\\[6mm]
  {\Large Computing and Networks Division}\\[2cm]
\end{center}
\vfill%
\begin{center}\Large CERN Geneva, Switzerland\end{center}
\end{titlepage}

\Filename{H1Preface}

\Filename{H2Copyright}

%%%%%%%%%%%%%%%%%%%%%%%%%%%%%%%%%%%%%%%%%%%%%%%%%%%%%%%%%%%%%%%%%%%%
%    Copyright  page                                               %
%%%%%%%%%%%%%%%%%%%%%%%%%%%%%%%%%%%%%%%%%%%%%%%%%%%%%%%%%%%%%%%%%%%%
\thispagestyle{empty}
\framebox[\textwidth][t]{\hfill\begin{minipage}{0.96\textwidth}%
\vspace*{3mm}
\begin{center}Copyright Notice\end{center}
\parskip\baselineskip

\parskip.6\baselineskip
CERN Program Library entry \textbf{Q123}

\textbf{FATMEN -- File and Tape Management System}

\copyright{} Copyright CERN, Geneva 1995
 
Copyright and any other appropriate legal protection of these
computer programs and associated documentation reserved in all
countries of the world.
 
These programs or documentation may not be reproduced by any
method without prior written consent of the Director-General
of CERN or his delegate.
 
Permission for the usage of any programs described herein is
granted apriori to those scientific institutes associated with
the CERN experimental program or with whom CERN has concluded
a scientific collaboration agreement.
 
Requests for information should be addressed to:
\vspace*{-.5\baselineskip}
\begin{center}
\tt\begin{tabular}{l}
CERN Program Library Office              \\
CERN-CN Division                         \\
CH-1211 Geneva 23                        \\
Switzerland                              \\
Tel.      +41 22 767 4951                \\
Fax.      +41 22 767 7155                \\
Bitnet:   CERNLIB@CERNVM                 \\
DECnet:   VXCERN::CERNLIB (node 22.190)  \\
Internet: CERNLIB@CERNVM.CERN.CH
\end{tabular}
\end{center}
\vspace*{2mm}
\end{minipage}\hfill}%end of minipage in framebox
\vspace{6mm}
 
{\bf Trademark notice: All trademarks appearing in this guide are acknowledged as such.}
\vfill

\begin{tabular}{l@{\quad}l@{\quad}>{\small\tt}l}
{\em Contact Person\/}:        & Jamie Shiers /CN    & (JAMIE\atsign CERNVM.CERN.CH)\\[1mm]
{\em Technical Realization\/}: & Michel Goossens /CN & (GOOSSENS\atsign CERNVM.CERN.CH)\\[2cm]
\textem{Edition -- February 1995}
\end{tabular}
\newpage

\Filename{H2Prelimininary-remarks}

%%%%%%%%%%%%%%%%%%%%%%%%%%%%%%%%%%%%%%%%%%%%%%%%%%%%%%%%%%%%%%%%%%%%
%    Introductory material                                         %
%%%%%%%%%%%%%%%%%%%%%%%%%%%%%%%%%%%%%%%%%%%%%%%%%%%%%%%%%%%%%%%%%%%%
\pagenumbering{roman}
\setcounter{page}{1}

\section*{Preliminary remarks}

This {\bf Complete Reference} of
the FATMEN system (for \textbf{F}ile \textbf{a}nd
\textbf{T}ape \textbf{M}anagement: \textbf{E}xperimental
\textbf{N}eeds), consists of four parts:
\begin{OL}
\item An \textbf{overview} of the system.
\item A \textbf{step by step} tutorial introduction.
\item A \textbf{user guide}, describing each command in detail.
\item An \textbf{installation and management guide}.
\end{OL}

The FATMEN system is implemented on various mainframes and personal
workstations. In particular versions exist for IBM~VM/CMS, IBM~MVS,
VAX/VMS and various Unix-like platforms, such as APOLLO, HP/UX, Cray Unicos,
Ultrix, IBM RS6000, Silicon Graphics and SUN.

\begin{center}
\fbox{\parbox{12cm}{Throughout this manual, commands to be \textbf{entered}
are {\tt\underline{underlined}}}}
\end{center}

\index{underlining}
\index{user input}

\Filename{H2Acknowledgement}
\section*{Acknowledgements}

The FATMEN package has undergone continuous evolution since its
first introduction in 1989. During this time many people   
have contributed to the system,
through discussions or by providing code and assistance.

The FATMEN system depends upon a number of other packages, such
as the Tape Management System and numerous tape staging subsystems.
The help of the authors and maintainers of such systems is 
gratefully acknowledged.

This document has been produced using \LaTeX~\cite{bib-LATEX}
with the \Lit{cernman} style option, developed at CERN. 
A compressed PostScript file \Lit{fatmen.ps.gz}, 
containing a complete printable version
of this manual, can be obtained from any CERN machine
by anonymous ftp as follows
(commands to be typed by the user are underlined)\footnote{%
If your site does not carry the gnu \Lit{gzip} utility you can get the
uncompressed file by dropping the \Lit{.gz} suffix from the
\Ucom{get} command, and also skipping the \Ucom{binary}
specification below.}:

\begin{XMP}
    \underline{ftp asisftp.cern.ch}
    Trying 128.141.202.89...
    Connected to asisftp.cern.ch.
    220 asis01 FTP server (Version 6.10 ...) ready.
    Name (asis01:username): \underline{anonymous}
    Password: \underline{your\_{}mailaddress}
    230 Guest login ok, access restrictions apply.
    ftp> \underline{cd cernlib/doc/ps.dir}
    ftp> \underline{binary}
    ftp> \underline{get fatmen.ps.gz}    ! one page per physical page
    ftp> \underline{get fatmen2.ps.gz}   ! two pages per physical page
    ftp> \underline{quit}
\end{XMP}


\Filename{H2prel-related-documents}
\section*{Related Documents}
\par This document can be complemented by the following documents:
\begin{UL}
\item ORACLE User Guide~\cite{bib-ORACLE}
\item KUIP - Kit for a User Interface Package~\cite{bib-KUIP}
\item ZEBRA - Data Structure Management System~\cite{bib-ZEBRA}
\item CSPACK - Client Server package~\cite{bib-CSPACK}
\item HEPDB - Database management package~\cite{bib-HEPDB}
\item The FATMEN Report - DD/89/15~\cite{bib-FATREP}
\item TMS - The CERN Tape Management System (to be published)
\item The MUSCLE Report - DD/88/1~\cite{bib-MUSCLE}
\item Computing at CERN in the 1990s~\cite{bib-NGB}
\end{UL}

%%%%%%%%%%%%%%%%%%%%%%%%%%%%%%%%%%%%%%%%%%%%%%%%%%%%%%%%%%%%%%%%%%%%
%    Tables of contents ...                                        %
%%%%%%%%%%%%%%%%%%%%%%%%%%%%%%%%%%%%%%%%%%%%%%%%%%%%%%%%%%%%%%%%%%%%
\newpage
\tableofcontents
\newpage
\listoffigures
\listoftables

%  ==================== Body of text ==============================
\pagenumbering{arabic}
\setcounter{page}{1}
\part{FATMEN -- Overview}
\include{fatover}
\part{FATMEN -- Tutorial}
%%%%%%%%%%%%%%%%%%%%%%%%%%%%%%%%%%%%%%%%%%%%%%%%%%%%%%%%%%%%%%%%%%%
%                                                                 %
%   FATMEN User Guide and Reference manual                        %
%                                                                 %
%   Fatmen Part 2: Tutorial                                       %
%                                                                 %
%   This document needs the following external EPS files:         %
%   none                                                          %
%                                                                 %
%   Editor: Michel Goossens / CN-AS                               %
%   Last Mod.:  7 June 1993 11:45 mg                              %
%                                                                 %
%%%%%%%%%%%%%%%%%%%%%%%%%%%%%%%%%%%%%%%%%%%%%%%%%%%%%%%%%%%%%%%%%%%

\Filename{H1Fatmentutorial-introduction}
\chapter{A tutorial introduction to FATMEN}
\Filename{H2Fatmentutorial-what-is-fatmen}
\section{What is FATMEN?}

The FATMEN package is a distributed File and Tape management system.
It provides transparent access to file catalogue
information and to data with no source code changes required
when moving from one host to another. This is independent 
of the location of the data or of the host operating system.
\par
All data is referenced by a Unix-like name, the so-called generic-name.
The FATMEN system is responsible for matching this name against
a physical dataset.
\Filename{H2Fatmentutorial-how-fatmen-helps}
\section{How does FATMEN help?}
\par
FATMEN removes the necessity of having to know details such as which
tape or disk a file is on. In addition, commands or subroutine calls to
access data are identical on all supported systems 
(currently VM/CMS, MVS, VAX/VMS and Unix). 
Thus, one can take the same code from machine to machine and access
data in a transparent manner, although the underlying mechanism
or even the copy of the data may be completely different.
\par
In addition, through the FATMEN catalogue, users may list characteristic
of a data set, such as a comment field, user words, file format etc.
which otherwise would have to be recorded in a log book or NEWS item.
\Filename{H2Fatmentutorial-explanation-of-terms}
\section{Explanation of terms}
\par
When using the FATMEN system, all data is referenced by a name known 
as the {\bf generic-name}. 
The generic name has the form
\begin{XMP}
//catalogue/experiment/dir1/dir2/.../dirn/filename
\end{XMP}
where
the slash character (/) is a directory delimiter, as for Unix file names,
{\bf catalogue} indicates in which catalogue the file resides
\index{catalogue}
{\bf experiment} is the name of the experiment to which the file
belongs
The rest of the file name is free format, although its total length
may not exceed 255 bytes and each component may not exceed
20 characters. Typically, experiments will have conventions
for the generic-name, but the sort of information that you might
want to include in the generic-name is
\begin{UL}
\item
Real data, simulated (possibly also technical run, cosmics etc.)
\item 
Beam particle (e.g. pi+, pi- etc.)
\item
Beam energy
\item
Target (for fixed target experiments)
\item
Period and run
\item
Magnet setting, if relevant
\item
Number of the pass through the reconstruction chain
\item
Level, i.e. DST, RAWDATA, etc.
\end{UL}
\par
Examples of catalogue names are CERN, for all CERN experiments, DESY, for experiments
based at DESY, and so on. Examples of experiment name are L3, H1, CDF.
\par
Note that the same generic-name can be used for more than one file. 
In this case, the files are all assumed to contain the same data,
but may well reside on different media or in different locations.
The file format may differ, for example, one copy might be in
Zebra FZ native format and another in exchange format.
We will see later how different entries with the same generic-name
may be selected or listed.     
\par
Associated with each generic-name are a catalogue entry and a key vector
which contains important information that can be used to make a first
pass selection. The catalogue entry is in fact a Zebra bank stored
in a Zebra RZ file, and the keys are the normal RZ keys.
However, for all practical purposes it is not necessary to know 
the structure of the catalogue in such detail, particularly when using
the FATMEN shell (interactive interface) or the so-called 'novice'
FORTRAN callable interface, which both hide Zebra completely.
\par
The key vector contains the filename, i.e. the part of the generic-name
following the last slash, and information on the mediatype on which the
data resides, the location of the file and the so-called copy level.
By using these keys, it is possible to make a first pass selection
of a file, or to view only a subset of a catalogue in a very efficient
manner. For example, when working at ones home laboratory say in the United
States, one is probably not interested in looking at catalogue entries
corresponding to files located several thousand kilometers away in
CERN, still less in trying to access them over the network.
\par
\index{Key definitions}
\index{KEYS}
\index{Media type definitions}
The meanings of the various keys are experiment defined, except for the
media type, which is defined as follows:
\begin{XMP}
1: disk
2: 3480 cartridge
3: 3420 tape
4: 8200 Exabyte cartridge
\end{XMP}
\subsection{The location code}
\index{Location type definitions}
\par
The location code is one piece of information available to FATMEN
to select the best available source of data.
The following convention is used by OPAL:
\index{location codes}
\index{fatmen.loccodes}
\begin{XMP}
         0=Cern Vault     CERNVM VXCERN CRAY SHIFT etc      
         1=Cern Vault                                      
         2=Cern Vault                                     
        11=VXOPON         OPAL Online Vax cluster        
        12=Online         OPAL (apollo) online facilities
        21=VXOPOF         OPAL Offline cluster           
        31=SHIFT          SHIFT disk and archive storage
     33101=Saclay         Active cartridges            
     33901=Saclay         'obsolete' cartridges       
     44501=UKACRL         Active cartridges          
     44901=UKACRL         'obsolete' cartridges     
\end{XMP}
\par
Even if the location code is not set, FATMEN will still be able to
find 'the best' copy of a file. However, it is much more efficient
to restrict the search by specifying one or more location codes,
as this results in less I/O to the FATMEN catalogue and, more importantly,
less queries to the Tape Management System (TMS).
\subsection{The copy level}
\index{Copy level definitions}
\par
Initially, the copy level was defined as follows:
\begin{XMP}
     0       original
     1       copy of an original
     2       copy of a copy
     ...
\end{XMP}
\par
In fact, this has been of limited use and it is now more
commonly used to indicate the data representation. The
following definitions are used by OPAL and correspond to
those used by the FPACK package developed at DESY.
\begin{XMP}
     1       IEEE floating point format (used on Unix workstations and 
                                         for Zebra exchange data format) 
     2       IBM  floating point format
     3       VAX  floating point format
     4       IEEE floating point format, but byte-swapped
     5       Cray floating point format
     4       IEEE floating point format, but byte-swapped
\end{XMP}
\Filename{H2Fatmentutorial-using-the-catalogue}
\section{Using the FATMEN catalogue - a simple example}
\par
The FATMEN catalogue can be used as a general purpose
catalogue, for example, to store information about
the records, cassettes and CDs that someone owns.
\par
To create a catalogue, we just type
\index{mkfatnew}
\index{FATNEW}
\index{Creating a new catalogue}
\begin{XMP}
mkfatnew
\end{XMP}
We then give the name \Lit{MUSIC} as the name of 
the FATMEN system, and \Lit{CLASSICAL} as the
name of the experiment. This will create a file
called \Lit{MUSIC.FATRZ} and all generic names will
start \Lit{//MUSIC/CLASSICAL}.
\par
We can then catalogue our collection by composer,
making directories such as \Lit{BACH}, \Lit{BEETHOVEN}, 
\Lit{CHOPIN} etc.
\begin{XMP}
fm
FM>mkdir BACH
FM>mkdir BEETHOVEN
FM>mkdir CHOPIN
\end{XMP}
\par
We may well group the music under the type of work, such
as \Lit{CONCERTOS}, \Lit{SONATOS}, \Lit{SYMPHONIES}.
We then have very readable generic-names,
e.g. \Lit{//MUSIC/CLASSICAL/BEETHOVEN/SYMPHONIES/NUMBER9},
could correspond to Beethoven's 9th Symphony.
Note that, using the FATMEN shell, 
we would never need to type the \Lit{//MUSIC/CLASSICAL}
as this would be our 'home' directory. We can even
return to it by typing \Lit{cd \$HOME}.
\par
We can now type commands such as:
\begin{XMP}
FM>ls beethoven/symphonies/(7:9) | List works 7-9 inclusive
FM>ls b*/c*/*                   | All works by composers
                                  with names beginning B
                                  and categories beginning C.
FM>ls */* -gc | Display the full name and comment field of 
                every entry
\end{XMP}
\par
We can, of course, use other fields of the catalogue, such
as the media type. Here we will use 1 to mean CD, 2 for 
cassette and 3 for LP. Thus, by default, FATMEN will preferentially
find a CD before a cassette before an LP. 
We could also define the location code, such as
\begin{XMP}
1 : home
2 : car
3 : boat
4 : pad in St. Tropez
\end{XMP}
\par
Then, we can limit our searching to a subset of the catalogue.
\begin{XMP}
FM>set/location 3
FM>ls */* -gc | Now we just see what we have on our boat
\end{XMP}
\par
If we get a new DAT player, we may want to add this,
say as media type 4. We can then redefine the search
order with:
\begin{XMP}
FM>set/media 1,4,2,3 | Access DAT after CD but before everything else
\end{XMP}
\par
There are also other fields that could be of interest,
such as the performance date, for which we could use
the creation date field, the performers, e.g.
Berlin Philharmonic, and the conducter. 
We could then use the search command to list all entries
performed by the London Sympony Orchestra (for example).
\par
Thus, we can make selections according to 
\begin{UL}
\item
The generic-name, e.g. BACH\_*/*, to list all works
by the BACH family.
\item
The FATMEN keys, such as the location-code, media-type
or copy-level (which could be important for convential
cassettes!)
\item
The catalogue entry itself, such as performance date,
composer, record company etc.
\end{UL}
\par
We will see later how to use the same features to 
manage physics data, which is of course our main concern.
\Filename{H2Fatmentutorial-starting to work-with-fatmen}
\section{Starting to work with the FATMEN system}
\par
Before you can start to work with the FATMEN system, your group must
have been entered into the FATMEN system as detailed in the installation
guide section of this manual. This involves
\begin{UL}
\item
Creating a set of ORACLE or SQL/DS tables for your experiment
{\bf N.B.This is only necessary if you wish to have the
FATMEN catalogued backed up into a relational database.
This option is typically only used at CERN.}
\item
Creating a service machine to manage the FATMEN database
\item
Registering this machine with the FATMEN master service machine
\end{UL}
\par
To get this done, just send a mail to JAMIE@CERNVM.
\Filename{H2Fatmentutorial-adding-information}
\section{Adding information to the file catalogue}
\par
Once the above steps have been taken, one can procede to catalogue
data in the database.
This can be done through either the FORTRAN callable interface
or the FATMEN shell. We will demonstrate the use of both and show
how the FATMEN system can then be used to access the data referenced
by these catalogue entries.
\subsection{Adding existing data to the FATMEN catalogue}
\par
The following examples show how existing data may be catalogued
within FATMEN. The examples are taken from the Fermilab CDF 
experiment.
\par
Information on existing data is kept in flat files, as shown below.
\begin{XMP}
R015881E.TOP04   |CED984|FNALH   |MUA21|04-SEP-1989:06:53:19.60|GOOD|
R015882B.TOP04   |CED984|FNALH   |MUA21|04-SEP-1989:08:59:03.96|GOOD|
R015882C.TOP04   |CED984|FNALH   |MUA21|04-SEP-1989:10:25:58.47|GOOD|
R015882D.TOP04   |CED984|FNALH   |MUA21|04-SEP-1989:11:50:21.63|GOOD|
R015898A.TOP04   |CED984|FNALH   |MUA21|04-SEP-1989:13:26:08.62|GOOD|
R015910A.TOP04   |CED984|FNALH   |MUA21|04-SEP-1989:14:13:17.86|GOOD|
R015911A.TOP04   |CED984|FNALH   |MUA21|04-SEP-1989:15:43:56.84|GOOD|
\end{XMP}
where the fields are dataset name (filename), tape number (vsn),
nodename, device name on which the tape was written, date and time
and status.
\par
The easiest way to add this information is to write a small command
file that converts the file into a KUMAC file. This can be done
as follows.
\begin{XMP}
$       open/read in top04.tape
$       open/write out addfat.kumac
$loop:
$       read/end=eof in line
$       line=f$edit(line,"TRIM,COMPRESS,COLLAPSE")
$       dsn =f$element(0,"|",line)
$       vsn =f$element(1,"|",line)
$       node=f$element(2,"|",line)
$       comm=f$element(5,"|",line)
$       fname=f$element(0,".",dsn)
$       temp =dsn - fname - "."
$       gname="//fnal/cdf/dst/" + temp + "/" + fname
$       write out "ADD/TAPE ''vsn' ''vsn' 1 ''gname' ''dsn' _"
$       write out "YBB 0 ''node' ''comm' F 0 0 200 3"
$       goto loop
$eof:   close in
$       close out
\end{XMP}
This generates the following KUMAC file.
\begin{XMP}
ADD/TAPE CED984 CED984 1 //fnal/cdf/dst/TOP04/R015880A R015880A.TOP04 _
UN 0 FNALH GOOD F 0 0 200 3
ADD/TAPE CEH060 CEH060 1 //fnal/cdf/dst/TOP04/R015881B R015881B.TOP04 _
UN 0 FNALH GOOD F 0 0 200 3
ADD/TAPE CED984 CED984 1 //fnal/cdf/dst/TOP04/R015881C R015881C.TOP04 _
UN 0 FNALH GOOD F 0 0 200 3
ADD/TAPE CED984 CED984 1 //fnal/cdf/dst/TOP04/R015881D R015881D.TOP04 _
UN 0 FNALH GOOD F 0 0 200 3
ADD/TAPE CED984 CED984 1 //fnal/cdf/dst/TOP04/R015881E R015881E.TOP04 _
UN 0 FNALH GOOD F 0 0 200 3
ADD/TAPE CED984 CED984 1 //fnal/cdf/dst/TOP04/R015882B R015882B.TOP04 _
UN 0 FNALH GOOD F 0 0 200 3
ADD/TAPE CED984 CED984 1 //fnal/cdf/dst/TOP04/R015882C R015882C.TOP04 _
UN 0 FNALH GOOD F 0 0 200 3
ADD/TAPE CED984 CED984 1 //fnal/cdf/dst/TOP04/R015882D R015882D.TOP04 _
UN 0 FNALH GOOD F 0 0 200 3
ADD/TAPE CED984 CED984 1 //fnal/cdf/dst/TOP04/R015898A R015898A.TOP04 _
UN 0 FNALH GOOD F 0 0 200 3
ADD/TAPE CED984 CED984 1 //fnal/cdf/dst/TOP04/R015910A R015910A.TOP04 _
UN 0 FNALH GOOD F 0 0 200 3
ADD/TAPE CED984 CED984 1 //fnal/cdf/dst/TOP04/R015911A R015911A.TOP04 _
UN 0 FNALH GOOD F 0 0 200 3
\end{XMP}
To add this to FATMEN we now type
\begin{XMP}
\Ucom{EXEC ADDFAT}
\end{XMP}
from within the FATMEN shell.
\par
We could also do the same thing using the FORTRAN interface.
This would have the advantage that we could also convert
the date and time of creation from VAX format and add that
to the catalogue, as is shown in the following FORTRAN
program.
\begin{XMPt}{The ADDTEST fortran program}
      PROGRAM ADDTEST
*
*     Add stuff to FATMEN catalogue, using CDF tape log files
*     A similar function can be performed by using ADDFAT.COM,
*     followed by ADDFAT.KUMAC in the FM shell.
*
      CHARACTER*256 GENAME,DSN,CHLINE
      CHARACTER*80  COMM
      CHARACTER*8   HOST
      CHARACTER*23  VAXDAT
      CHARACTER*6   VSN
      CHARACTER*4   FFORM,RECFM,CHOPT
      CHARACTER*3   MONTHS(12),CHMON
      DATA          MONTHS( 1)/'JAN'/,MONTHS( 2)/'FEB'/,
     +              MONTHS( 3)/'MAR'/,MONTHS( 4)/'APR'/,
     +              MONTHS( 5)/'MAY'/,MONTHS( 6)/'JUN'/,
     +              MONTHS( 7)/'JUL'/,MONTHS( 8)/'AUG'/,
     +              MONTHS( 9)/'SEP'/,MONTHS(10)/'OCT'/,
     +              MONTHS(11)/'NOV'/,MONTHS(12)/'DEC'/
*
* Start of FATMEN sequence FATPARA
*
** ***     Data set bank mnemonics
*
*
*          Keys
      PARAMETER ( MKSRFA= 1, MKFNFA= 2, MKCLFA=7, MKMTFA=8
     1           ,MKLCFA= 9, MKNBFA=10, NKDSFA=10 )
*
** ***     Bank offsets
*
      PARAMETER ( MFQNFA=  1, MHSNFA= 65, MCPLFA= 67, MMTPFA= 68
     1           ,MLOCFA= 69, MHSTFA= 70, MHOSFA= 74
     2           ,MVSNFA= 77, MVIDFA= 79, MVIPFA= 81, MDENFA= 82
     3           ,MVSQFA= 83, MFSQFA= 84, MSRDFA= 85, MERDFA= 86
     4           ,MSBLFA= 87, MEBLFA= 88, MRFMFA= 89, MRLNFA= 90
     5           ,MBLNFA= 91, MFLFFA= 92, MFUTFA= 93, MCRTFA= 94
     6           ,MCTTFA= 95, MLATFA= 96, MCURFA= 97, MCIDFA= 99
     7           ,MCNIFA=101, MCJIFA=103, MFPRFA=105, MSYWFA=106
     8           ,MUSWFA=116, MUCMFA=126, NWDSFA=145
     9           ,MFSZFA=MSYWFA,MUSCFA=MSYWFA+1)
*
* End of FATMEN sequence FATPARA
*
      COMMON /USRLNK/LUSRK1,LUSRBK,LADDBK,LUSRLS

      DIMENSION     NFAT(NWDSFA)

*
*     Initialise FATMEN & Zebra...
*
      CALL FMSTRT(1,2,'//FNAL/CDF',IRC)
      CALL FMLOGL(0)
*
*     Open the data file...
*
      OPEN(3,FORM='FORMATTED',STATUS='OLD')
*
*     Now process the data...
*
10    CONTINUE
      READ(3,'(A)',END=99) CHLINE
*     PRINT *,'Processing ',CHLINE(1:LENOCC(CHLINE))

*123456789_123456789_123456789_123456789_123456789_123456789_123456789_
*R015880A.TOP04   |CED984|FNALH   |MUA21|04-SEP-1989:01:31:48.11|GOOD

*
*     Convert date and time...
*
      VAXDAT = CHLINE(41:60)
      READ(VAXDAT,'(I2,1X,A3,1X,I4,1X,I2,1X,I2)') 
     +     IDAY,CHMON,IYEA,IHOU,IMIN
      IMON   = ICNTH(CHMON,MONTHS,12)

      IYEA   = MOD(IYEA,1900)

      ID     = IYEA*10000 + IMON*100 + IDAY
      IT     = IHOU*100   + IMIN
*
*     and pack for insertion into FATMEN bank...
*
      CALL FMPKTM(ID,IT,IP,IRC)
      
      GENAME = '//FNAL/CDF/DST/'//CHLINE(10:14)//'/'//CHLINE(1:8)

      VSN    = CHLINE(19:24)

      DSN    = CHLINE(1:14)

      HOST   = CHLINE(26:31)

      IFILE  = 1

      CHOPT  = 'N'
*
*     Change later to YBB...
*
      FFORM  = 'UN'

      ICOPY  = 0

      RECFM  = 'V'

      COMM   = CHLINE(65:68) // ' ' // CHLINE(35:39)
 
      CALL FMADDT(GENAME,VSN,VSN,IFILE,
     +            DSN,FFORM,ICOPY,HOST,RECFM,0,0,0,0,COMM,
     +            IVECT,CHOPT,IRC)

*
*     Now get back the bank into a vector and modify the fields
*     that we could not via FMADDT
*
      CALL FMPEEK(GENAME,NFAT,' ',IRC)
*
*     Media type 3 = 3420
*
      NFAT(MMTPFA) = 3
*
*     these should be zero anyway...
*
      CALL VZERO(NFAT(MUSWFA),10)
*
*     Creation date...
*
      NFAT(MCRTFA) = IP

      CALL FMPOKE(GENAME,NFAT,'P',IRC)

      GOTO 10

99    CLOSE(3)
*
      CALL FMEND(IRC)
      PRINT *,'Return code ',IRC,' from FMEND'
*
      END
\end{XMPt}

\subsection{Adding and referencing data using the FATMEN shell}

Let us assume that the tapes have volume serial numbers (VSN - the
magnetically recorded label) JS1001 to JS1010 inclusive. The visual
identifiers (VID) are different - CIN136 to CIN145 inclusive.
As this type of VID implies, these are actually IBM 3480 cartridges,
rather than the older open reel tape.
The file identifier of all the datasets is the same - RAWDATA.
In order to catalogue these files we must first establish a table
linking each file or tape with a generic-name.
Let us assume that the generic-name starts with
\Lit{//CERN/OPAL/LEPD/RAWD/P51989}. 
This implies that
the data are real LEP data (as opposed to simulated data), in
raw data format, i.e. no filtering or reconstruction. The data
were recorded in period 5 in 1989.

The FATMEN shell does not allow all fields of the file catalogue
to be entered. We are able to specify the following fields:
\begin{OL}
\item
Volume sequence number (VSN)
\item
Visual identifier (VID)
\item
File sequence number (FSEQ)
\item
File identifier or dataset name (DSN)
\item
File format (FX, FZ etc.)
\item
Copy level (is this a copy or an original?)
\item
Host name
\item
Comment
\end{OL}
\par
Suppose we now enter this information into a disk file, each line
representing a different tape. The resultant file might look
like the following:
\begin{XMP}
JS1001 CIN136 1 RAWDATA FX 0 VXOPON 'Early physics run - some dectectors out'
JS1002 CIN137 1 RAWDATA FX 0 VXOPON 'Early physics run - some dectectors out'
JS1003 CIN138 1 RAWDATA FX 0 VXOPON 'Early physics run - some dectectors out'
JS1004 CIN139 1 RAWDATA FX 0 VXOPON 'Early physics run - some dectectors out'
JS1005 CIN140 1 RAWDATA FX 0 VXOPON 'Early physics run - some dectectors out'
JS1006 CIN141 1 RAWDATA FX 0 VXOPON 'Early physics run - some dectectors out'
JS1007 CIN142 1 RAWDATA FX 0 VXOPON 'Early physics run - some dectectors out'
JS1008 CIN143 1 RAWDATA FX 0 VXOPON 'Early physics run - some dectectors out'
JS1009 CIN144 1 RAWDATA FX 0 VXOPON 'Early physics run - some dectectors out'
JS1010 CIN145 1 RAWDATA FX 0 VXOPON 'Early physics run - some dectectors out'
\end{XMP}
\par
As we first wish to add this data using the FATMEN shell,
we must edit the file to include the generic-name of each dataset.
To do this, we will create a file with filetype KUMAC, so that
we can then execute it directly from within the FATMEN shell.
First, we include a command to change the current directory
to \Lit{//CERN/OPAL/LEPD/RAWD/P51989}. 
This is done using the command \Ucom{cd}.
 Note that there is no need to type the
\Lit{//CERN/OPAL} as we enter at this level by default.
Then, between the file sequence number and dataset name, we insert
the generic-name. For simplity, we will refer to these files
as FILE1 to FILE10. 
We also add the command \Ucom{ADD/TAPE} at
the beginning of each line to instruct the FATMEN shell to add this
information to the database.
\begin{XMPt}{Resultant file \Lit{TESTFAT KUMAC} (or \Lit{testfat.kumac})}
cd lepd/rawd/p51989

ADD/TAPE JS1001 CIN136 1 TAPE1 RAWDATA FX 0 VXOPON 'Early physics run - some detectors out'
ADD/TAPE JS1002 CIN137 1 TAPE2 RAWDATA FX 0 VXOPON 'Early physics run - some detectors out'
ADD/TAPE JS1003 CIN138 1 TAPE3 RAWDATA FX 0 VXOPON 'Early physics run - some detectors out'
ADD/TAPE JS1004 CIN139 1 TAPE4 RAWDATA FX 0 VXOPON 'Early physics run - some detectors out'
ADD/TAPE JS1005 CIN140 1 TAPE5 RAWDATA FX 0 VXOPON 'Early physics run - some detectors out'
ADD/TAPE JS1006 CIN141 1 TAPE6 RAWDATA FX 0 VXOPON 'Early physics run - some detectors out'
ADD/TAPE JS1007 CIN142 1 TAPE7 RAWDATA FX 0 VXOPON 'Early physics run - some detectors out'
ADD/TAPE JS1008 CIN143 1 TAPE8 RAWDATA FX 0 VXOPON 'Early physics run - some detectors out'
ADD/TAPE JS1009 CIN144 1 TAPE9 RAWDATA FX 0 VXOPON 'Early physics run - some detectors out'
ADD/TAPE JS1010 CIN145 1 TAPE10 RAWDATA FX 0 VXOPON 'Early physics run -some dectectors out'
\end{XMPt}
We can now procede to run this macro, by typing the following commands:
\begin{XMP}
\Ucom{FM}
\Ucom{EXEC TESTFAT}
\Ucom{END}
\end{XMP}
Now that these files have been catalogued, we can look at the catalogue
information using the {\tt\underline{ls}} command.
This command allows
us to list the various files and there attributes. For example, the
command {\tt\underline{ls tape\% -a}}
would list all details (option a)
for the files tape1-tape9. Tape10 would not be listed as the \% character
only matches against a single character. We must use the syntax
{\tt\underline{ls tape* -a}} to see also this file.
\par
Of perhaps more interest is the ability to be able to access the data
itself by using the generic name. This is performed by using the
{\tt find} command. Thus, we could type the commands
\begin{XMP}
{\bf\it
fm
cd lepd/rawd/p51989
find tape7 iofile13
end
}
\end{XMP}
\par
This would initiate an input tape stage of the volume matching the 
generic name
\begin{XMP}
//CERN/OPAL/LEPD/RAWD/P51989/TAPE7
\end{XMP}
\par
Once the stage has completed, a FORTRAN
program can read the data on {\ttsc IOFILE13}.
\subsection{Adding data using the FORTRAN callable interface}
\par
The following example shows how data may be added to the
catalogue using the FORTRAN interface.
The \Rind{FMADDD} and \Rind{FMADDT} routines provide the same functionality
as the shell ADD/DISK and ADD/TAPE commands. 
\begin{XMPt}{Adding information to the catalogue}
      PROGRAM ADDTEST
      CHARACTER*256 GENAME,DSN
      CHARACTER*80  COMM
      CHARACTER*8   HOST,CHUSER
      CHARACTER*4   FFORM,RECFM,CHOPT
*
*     Sequence FATPARA from PATCH FATCDES on FATMEN pam
*
+CDE,FATPARA.

      CALL FMSTRT(1,2,'//CERN/CNDIV',IRC)
      CALL FMLOGL(3)
 
      GENAME = '//CERN/CNDIV/JAMIE/ULF'
      DSN    = '<JAMIE.192>BOX.SET'
      FFORM  = 'AS'
      HOST   = 'CERNVM'
      RECFM  = 'V'
      COMM   = 'ADDED VIA NEW ADDTEST FORTRAN'
      CHOPT  = 'N'
*
*     CHOPT = N : do not add this entry to the catalogue
*     this allows us to modify other fields before sending
*     the update to the server.
* 
      CALL FMADDD(GENAME,DSN,FFORM,0,HOST,RECFM,80,11,483,2,COMM,
     +            IVECT,CHOPT,IRC)
*
*     Update user file format - this field is not accessible
*     directly via FMADDD
*
*     Two ways of doing this are shown. 
*     
+SELF,IF=PEEK.
*
*     Get contents of bank into vector IVECT
*
      CALL FMPEEK(GENAME,IVECT,' ',IRC)
*
*     Update user file format
*
      CALL UCTOH('NFF',IVECT(MFUTFA),4,3)
*
*     Copy vector back and update catalogue
*
      CALL FMPOKE(GENAME,IVECT,'P',IRC)
+SELF,IF=-PEEK.
      CALL FMPUTC(-1,'NFF',MFUTFA,3,IRC)
*
*     Now update the catalogue
*         Options: N - ignore IVECT
*                  P - add entry via FMPUT
*
      CALL FMPOKE(GENAME,IVECT,'NP',IRC)
+SELF.
      CALL FMEND(IRC)
      PRINT *,'Return code ',IRC,' from FMEND'
*
      END
\end{XMPt}
\Filename{H2Fatmentutorial-data-access-using-fortran}
\section{Access to data using the FORTRAN callable interface}
\par
The following example shows how a dataset may be accessed
through the FORTRAN interface.
\begin{XMPt}{Access to data}
      CHARACTER*80 GENAME
      COMMON /QUEST/IQUEST(100)
*
*     FATMEN keys. These will be returned by the call to FMOPEN
*
      PARAMETER (LKEYFA=10)
      DIMENSION KEY(LKEYFA)
*
*     Units for reading the FATMEN catalogue and for
*     sending updates to the server
*
      LUNRZ = 1
      LUNFZ = 2
*    
*     Initialise FATMEN novice interface
*
      CALL FMSTRT(LUNRZ,LUNFZ,'//CERN/OPAL',IRC)
*
      GENAME = '//CERN/OPAL/DDST/PASS3/FYZ1/P18R1929/C01'
      LG = LENOCC(GENAME)
*
*     Access and open the file. R indicates Read mode
*     and F instructs FMFILE to issue the appropriate
*     call to FZFILE for this dataset.
* 
      LBANK = 0
      CALL FMFILE(11,GENAME(1:LG),'RF',IRC)
      IF(IRC.NE.0) THEN
         PRINT *,'Return code ',IRC,' from FMFILE'
         GOTO 10
      ELSE
*
*     Now process the file, reading just the FZ headers
*
         CALL READRZ(11)
      ENDIF
*
*     Now close this file, issuing FZENDI (option E),
*     drop staging disk (option D)
*
      CALL FMFEND(11,GENAME(1:LG),'ED',IRC)
      IF(IRC.NE.0) PRINT *,'Return code ',IRC,' from FMFEND'
1     CONTINUE
10    CONTINUE
*
      END
 
      SUBROUTINE READFZ(LUN)
      COMMON/QUEST/IQUEST(100)
      CHARACTER*8  DELTIM
      DIMENSION    IUHEAD(400)
      DIMENSION    IOCR(100)
      PARAMETER (JBIAS=2)
 
      NREC = 0
      CALL FMRTIM(DELTIM)
      CALL TIMED(T)
   1  CONTINUE
      NHEAD = 400
      IXDIV = 0
      CALL FZIN(LUN,IXDIV,LSUP,JBIAS,'S',NHEAD,IUHEAD)
      IF(IQUEST(1).LT.4) THEN
         NREC = NREC + 1
         GOTO 1
      ENDIF
 
      PRINT *,'READFZ. end after ',NREC,' records, IQUEST(1) = ',
     +        IQUEST(1)
      CALL FMRTIM(DELTIM)
      CALL TIMED(T)
      PRINT *,'FMFZLP. Elapsed time = ',DELTIM,
     +        ' CP time = ',T,' sec.'
      END
 
 
\end{XMPt}
\Filename{H2Fatmentutorial-step-by-step-data-access}
\section{Access to data step by step}
\par
The following examples go through some of the various
options available when accessing a dataset.
The simplest example as already been given, and
is access to a dataset using the FATMEN shell.
\par
We just type
\begin{XMP}

FM> find generic-name unit

\end{XMP}
\par
If unit is numeric, it will be interpreted as a FORTRAN
unit number. Thus, after typing
\begin{XMP}

find //cern/cndiv/jamie/test 11

\end{XMP}
we would have the following.
\begin{UL}
\item
On VM systems, a FILEDEF on unit FT11F001 pointing to the
dataset referenced by //cern/cndiv/jamie. Any mini-disk
links and accesses or stage operations would have been
performed automatically, so that a subsequent FORTRAN
program could just open unit 11 and read.
\item
On VMS systems, a logical name FOR011. Once again,
our FORTRAN program can just open unit 11 and read.
\item
On Unix systems, a soft link fort.11.
\end{UL}
\par
One could also specify a unit such as VM11F001, if
the file were to be read by VMIO, or any character
string that would be used as a logical name or
symbolic link, e.g.
\begin{XMP}
find //cern/cndiv/jamie/test mydat
\end{XMP}
\par
Our FORTRAN program could then open the named file MYDAT.  
(On VM or MVS systems, this corresponds to the DDNAME, 
rather than the file name.) 
\par
Normally, we will wish to issue the FIND command from
FORTRAN, as this is more powerful. Although a routine
\Rind{FMFIND} exists, we will describe the \Rind{FMOPEN} routine,
which includes all of the functionality of \Rind{FMFIND} and
more.
\par
Our call to \Rind{FMOPEN} looks like
\begin{XMP}
      CALL FMOPEN('//CERN/CNDIV/JAMIE/TEST',CHLUN,LBANK,CHOPT,IRC)
\end{XMP}
\par
Here, the generic name and unit are specified exactly as in the
shell, and are both of type character. This permits
the use of units such as VM11F001, MYDAT and so on. Again,
if we call \Rind{FMOPEN} with
\begin{XMP}
      CALL FMOPEN('//CERN/CNDIV/JAMIE/TEST','11',LBANK,CHOPT,IRC)
\end{XMP}
the correct FORTRAN name on the system in question would be used.
The exception to this is when the generic name in question points
to a file which should not be processed by FORTRAN. In this case,
\Rind{FMOPEN} will automatically perform the correct operation.
For example, EPIO files on the IBM should be read with IOPACK,
hence \Rind{FMOPEN} will build a DDNAME of IOFILE11. (This was also
true for Zebra exchange format files on the IBM prior to
version 3.67, when FORTRAN I/O became the default).
\par
Unlike the shell, \Rind{FMOPEN} will not only perform operations such
as staging the file as required, but will also issue
the correct OPEN. This may be overridden by the character
option parameter CHOPT.
\subsection{FMOPEN options}
\par
In the case of files to be processed by the Zebra FZ or RZ package,
we can ask \Rind{FMOPEN} to perform the FZFILE or RZFILE call.
For example,
\begin{XMP}
      CALL FMOPEN('//CERN/CNDIV/JAMIE/TEST','11',LBANK,'F',IRC)
\end{XMP}
\par
where F indicates that a call to FZFILE is to be issued. FATMEN obtains
the correct parameters for the call to FZFILE from the catalogue.
\par
Another interesting option is the S option. This will instruct FATMEN
to update the catalogue with the file size. This is useful for tape
files, as future accesses will request a staging disk of the correct
size, and hence use the system more efficiently. 
\par
We can also request that an automatic duplicate of the file is made
into the robot at CERN (SMCF), via the D option. This option requires
a pool of tapes gg\_FAT1, e.g. XX\_FAT1, in the TMS. Naturally, if a robot
copy already exists a new one will not be made.
\subsection{Access to tape data}
\index{tape data}
\index{configuration files}
\index{SETUP names file}
\index{names file}
\index{SETUP logical names}
\index{VAXTAP}
\index{SHIFT}
\par
For a dataset on a tape volume to be accessible, a device capable
of reading or writing the volume must exist on the local node,
or on a server node, in the case of remote access to data or VAXcluster systems.
\par
Suppose we wished to read the dataset with generic name 
{\tt //CERN/CNDIV/JAMIE/TEST}. A copy of this dataset might
exist on tape volume {\tt 123456}. If this volume required a device
of tape {\tt CT1}, the FATMEN software will attempt to determine
if such a device exists on the local node. Where remote tape access
is available, it will check if such a device exists on one of the
server nodes. This is done as follows.
\subsubsection{VM/CMS systems running HEPVM software}
A entry must exist in the file {\tt SETUP~NAMES}
for the device {\tt CT1} (in this example). {\bf (Not yet implemented)}
\subsubsection{VAX/VMS systems running VAXTAP software}
A logical name in the system table {\tt SETUP\_CT1S} (in this example)
must exist for locally attached tapes. Served tapes must be defined
in the file {\tt SETUP\_EXE:TPSERV.CONF}
\subsubsection{Cray Unicos systems}
An entry for the corresponding device must exist
in the file {\tt /etc/shift.conf}.
\subsubsection{Unix systems running the SHIFT tape software}
An entry for the corresponding
device must exist in the file {\tt /etc/shift.conf} or in the file {\tt /etc/TPCONFIG}.
\subsection{Access to remote data}
\index{remote data}
\index{remote access}
\index{L3 Stage}
\index{SHIFT}
\par
The above examples will work on both local and remote data, without
any change. Note that a call to \Rind{FMOPEN} on an Apollo in Helsinki will
not result in a cartridge being mounted in the robot at CERN.
Remote access to tape data must be explicitly enabled. It is currently
enabled in the SHIFT facility, where the data is staged via the CRAY,
and on the L3 Apollos, where the data is staged through LEPICS.
\par
Access to remote disk data requires explicit selection of a copy of 
a dataset. This may be done as shown below.
\begin{XMPt}{Example of using the \protect\Rind{FMSELK} routine}
*     Argument declarations
      PARAMETER (LKEYFA=10)
      PARAMETER (MAXKEY=999)
      DIMENSION INKEYS(LKEYFA),OUKEYS(LKEYFA,MAXKEY)
*     The following statements will select all datasets
*     with copy level (MKCLFA) of 1 (i.e. a copy of an original file),
*     media type of 1 (i.e. disk) and location code of 1 (i.e. CERN)
      INKEYS(MKCLFA) = 1
      INKEYS(MKMTFA) = 1
      INKEYS(MKLCFA) = 1
      CALL FMSELK('//CERN/CNDIV/JAMIE/TEST',INKEYS,OUKEYS,NFOUND,MAXKEY,IRC)
      IF(NFOUND.GT.0) THEN
*     Just take the first one found which matches
         CALL FMGETK('//CERN/CNDIV/JAMIE/TEST',LBANK,OUKEYS(1,1),IRC)
*     Now pass the bank to FMOPEN.
         CALL FMOPEN('//CERN/CNDIV/JAMIE/TEST','11',LBANK,'F',IRC)
      ENDIF
\end{XMPt}
\par
In the current version this dataset would only be accessible if mounted
via NFS.
\Filename{H2Fatmentutorial-relationships}
\section{The relationship between generic names, keys vectors and Zebra banks}
\par
As explained above, there can be multiple copies of a file with the same 
generic name. Typically, these copies will reside on different media,
in different locations or have different data representations.
\par
An entry exists in the FATMEN catalogue for each copy of a file.
This entry consists of a Zebra bank and an associated vector
known as the KEYS vector. One can use the keys vector, explicitly
or implicitly, to select a particular copy of a file. More details
on how this is done are given below.
\par
Many of the FATMEN callable routines have the generic name, Zebra
bank address and keys vector as arguments. If the bank address
is zero, the FATMEN software retrieves the bank corresponding to
the specified generic name from the FATMEN catalogue. In the case of
multiple entries for the same generic name, the entry which is returned
is determined by the rules described below. The exception to this
rule is the routine \Rind{FMGETK}, which returns the entry corresponding
to the key vector specified.
\subsection{The FATMEN selection rules}
\par
By default, the FATMEN selection is as follows:
\begin{OL}
\item
The catalogue is scanned for entries matching a given generic
name. No check is made on location code, or copy level,
but the different media types are processed in numerical order
1 - 4. (Media types 1 - 4 are disk, 3480 cartridge, 3420 tape
and Exabyte 8200 cassette respectively).
\item
If a entry is found for a given media type that is accessible,
that entry is taken and the search stops.
\item
For disk files, the host name in the catalogue must match
the current host name for the file to be deemed accessible.
\item
In the case of VAXclusters, the node name in the catalogue
may be the VAXcluster alias or the name of any member of 
the cluster.
\item
In the case of tape files, an entry on a robotically mounted 
volume is taken in preference over one on a manually mounted
volume.
\item
If the interface to the Tape Management System (TMS) is enabled,
the volume must exist and be in an active vault.
\end{OL}
\par
Note that for systems such as Apollo, SHIFT etc. the node name
that the FATMEN software uses can be set using an environmental
variable. Thus on the various SHIFT nodes at CERN (shift1, shd01 etc.)
the node name is set to SHIFT.
\par
One can set ranges of valid location codes and copy levels using
the routines \Rind{FMSETL} and \Rind{FMSETC} respectively. (The corresponding
shell commands are \Lit{SET/LOCATION} and \Lit{SET/COPYLEVEL}). If ranges
for either of these keys are set, then only entries with
keys that match will be considered for selection. 
Note that setting a range of location codes can result in
significantly faster selection time in the case of multiple
entries, particurly if multiple TMS queries are avoided.
\par
For example, if one makes 30 copies of every DST tape for
export to outside laboratories, one can avoid up to 29 
TMS queries by using different location codes at these
sites.
\par
In addition to ranges of location codes and copy levels,
one may also set ranges of media types with \Rind{FMSETM} or
\Lit{SET/MEDIATYPE}. Here, the order is also significant,
thus \Lit{SET/MEDIATYPE 2,4,1} will look first on 3480 cartridge,
then Exabyte 8200 cassette and finally disk.
\par
In some cases, a more powerful selection technique is needed.
For example, one may want to set the search order to
\begin{OL}
\item
Local disk, native data format
\item
Robotically mounted tape, native data format
\item
Local disk, exchange data format
\item
Robotically mounted tape, exchange data format
\end{OL}
to avoid the overhead of conversion between data representation
types. This can be achieved using the FORTRAN routine \Rind{FMSETK}.
This routine is described further in the user guide section of
this manual.
\Filename{H2Fatmentutorial-dataset-copies}
\section{Using FATMEN to make copies of datasets}
\par
FATMEN can make copies of a dataset with automatic update of the catalogue.
This is available both through the shell and the FORTRAN interface.
The shell version provides an interface to the file transfer
routines of CSPACK (the same as those used by ZFTP) and hence
permits the network transfer of files. The files are transferred
from disk to disk. If the local file resides on tape, it is first
staged to disk. For reasons of program size this facility
is not yet enabled on VM or MVS machines, but just for VMS and Unix
systems. An interface to remote tapes is not yet provided.

Both the shell COPY command and the FORTRAN routine \Rind{FMCOPY} permit
conversion of data representation and record format during
copy. That is, one may copy an input Zebra FZ binary exchange
file to an output Zebra FZ native file, or an input file
with VBS format to an output file with record format U.

\newpage

The following example shows the use of \Rind{FMCOPY}. This example
was written for DELPHI, but similar programs are in use by
CPLEAR, L3 and OPAL.

\begin{XMPt}{Example of copy data with \protect\Rind{FMCOPY}}
***********************************************************************
* PROGRAM DELRCOPY                                                    *
* ================                                                    *
* Make copies of all files corresponding to generic names on unit 10  *
* into the robot (SMCF) allocating volumes from pool XX_RAWD          *
***********************************************************************
      PARAMETER (LURCOR=100000)
      COMMON/CRZT/IXSTOR,IXDIV,IFENCE(2),LEV,LEVIN,BLVECT(LURCOR)
      DIMENSION    LQ(999),IQ(999),Q(999)
      EQUIVALENCE (IQ(1),Q(1),LQ(9)),(LQ(1),LEV)
      COMMON /USRLNK/LUSRK1,LUSRBK,LUSRLS
      COMMON /QUEST/IQUEST(100)
*
* Start of FATMEN sequence FATPARA
*
** ***     Data set bank mnemonics
*
*          Keys
      PARAMETER ( MKSRFA= 1, MKFNFA= 2, MKCLFA=7, MKMTFA=8
     1           ,MKLCFA= 9, MKNBFA=10, NKDSFA=10 )
*
** ***     Bank offsets
*
      PARAMETER ( MFQNFA=  1, MHSNFA= 65, MCPLFA= 67, MMTPFA= 68
     1           ,MLOCFA= 69, MHSTFA= 70, MHOSFA= 74
     2           ,MVSNFA= 77, MVIDFA= 79, MVIPFA= 81, MDENFA= 82
     3           ,MVSQFA= 83, MFSQFA= 84, MSRDFA= 85, MERDFA= 86
     4           ,MSBLFA= 87, MEBLFA= 88, MRFMFA= 89, MRLNFA= 90
     5           ,MBLNFA= 91, MFLFFA= 92, MFUTFA= 93, MCRTFA= 94
     6           ,MCTTFA= 95, MLATFA= 96, MCURFA= 97, MCIDFA= 99
     7           ,MCNIFA=101, MCJIFA=103, MFPRFA=105, MSYWFA=106
     8           ,MUSWFA=116, MUCMFA=126, NWDSFA=145
     9           ,MFSZFA=MSYWFA,MUSCFA=MSYWFA+1)
 
* End of FATMEN sequence FATPARA
      CHARACTER*6  DENS
      CHARACTER*8  LIB
      CHARACTER*4  LABTYP
      CHARACTER*1  MNTTYP
      CHARACTER*8  MODEL
      CHARACTER*7  ROBMAN(2)
      DATA         ROBMAN(1)/'-Robot '/,ROBMAN(2)/'-Manual'/
      PARAMETER (LKEYFA=10)
      PARAMETER (MAXFIL=1000)
      DIMENSION KEYS(LKEYFA,MAXFIL)
      DIMENSION JSORT(MAXFIL)
      DIMENSION KEYSIN(LKEYFA),KEYSOU(LKEYFA,MAXFIL)
      CHARACTER*255 FILES(MAXFIL)
      CHARACTER     THRONG*8, DSN*8, VSN*6, VID*6
      CHARACTER*80  TOPDIR,GENAME
      CHARACTER*26  CHOPT
*
*     Initialise ZEBRA
*
      CALL MZEBRA(-3)
      CALL MZSTOR(IXSTOR,'/CRZT/','Q',IFENCE,LEV,BLVECT(1),BLVECT(1),
     +            BLVECT(5000),BLVECT(LURCOR))
      CALL MZLOGL(IXSTOR,-3)
*
* *** Define user division and link area like:
*
      CALL MZDIV  (IXSTOR, IXDIV, 'USERS', 50000, LURCOR, 'L')
      CALL MZLINK (IXSTOR, '/USRLNK/', LUSRK1, LUSRLS, LUSRK1)
*
*     Units for FATMEN RZ/FZ files
*
      LUNRZ = 1
      LUNFZ = 2
*
*     Initialise FATMEN
*
      CALL FMINIT(IXSTOR,LUNRZ,LUNFZ,'//CERN/DELPHI',IRC)
      CALL FMLOGL(1)
      IDEBFA = 2
      NPROC  = 0
 
10    CONTINUE
      CALL TIMEL(T)
      IF(T.LT.100) THEN
         PRINT *,'Stopping due to time limit'
         GOTO 99
      ENDIF
      READ(10,9001,END=99) GENAME
9001  FORMAT(A80)
*     GENAME = FILES(JSORT(I))
      LGN = LENOCC(GENAME)
      PRINT *,'Processing ',GENAME(1:LGN)
      IROBOT = 0
*
*     First, check that a robot copy does not already exist
*
      CALL VBLANK(KEYSIN(2),5)
      LFN = INDEXB(GENAME(1:LGN),'/') + 1
*
*     Don't compare copy level or location code
*
      KEYSIN(MKCLFA) = -1
      KEYSIN(MKLCFA) = -1
*
*     Restrict search to 3480s
*
      KEYSIN(MKMTFA) = 2
      CALL FMSELK(GENAME(1:LGN),KEYSIN,KEYSOU,NMATCH,MAXFIL,IRC)
      IF(IRC.NE.0) THEN
         PRINT *,'Return code ',IRC,' from FMSELK'
         PRINT *,'Skipping ',GENAME(1:LGN)
         GOTO 10
      ENDIF
      IF(IDEBFA.GE.2)
     +PRINT *,'Found ',nmatch,' matches for media type 2'
      DO 30 J=1,NMATCH
      CALL FMQVOL(GENAME(1:LGN),LBANKR,KEYSOU(1,J),
     +            LIB,MODEL,DENS,MNTTYP,LABTYP,IC)
 
      IF(INDEX(LIB,'*Unknown') .NE.0) THEN
         IF(IDEBFA.GE.0) PRINT *,'Cannot determine mount type'
      ENDIF
 
      IF(MNTTYP.EQ.'R') THEN
         IF(IDEBFA.GE.0) PRINT *,'Robot copy already exists'
         IROBOT = 1
      ENDIF
 
30    CONTINUE
      IF(IROBOT.EQ.0) THEN
         PRINT *,'Robot copy does not exist for ',GENAME(1:LGN)
*
*     Now make a robot copy
*     First, allocate a new tape
*
         CALL FMALLO('3480','38K',' ','SMCF_1','XX_RAWD',
     +               LBANKR,' ',VSN,VID,IRC)
         IF(IRC.NE.0) THEN
            PRINT *,'Cannot allocate new tape - STOP!'
            GOTO 99
         ENDIF
*
*     Display the bank
*
         CALL FMSHOW(GENAME(1:LGN),LBANKR,KEYSOU(1,J),'A',IRC)
         LZERO = 0
         CALL VZERO(KEYSIN,10)
*
*     and make the copy
*
         CALL FMLOGL(3)
         CALL FMCOPY(GENAME(1:LGN),LZERO,KEYSIN,
     +               GENAME(1:LGN),LBANKR,KEYSOU(1,J),' ',IRC)
         IF(IRC.NE.0) THEN
            PRINT *,'Error from FMCOPY - STOP!'
            GOTO 99
         ENDIF
         CALL MZDROP(IXSTOR,LBANKR,' ')
         CALL MZDROP(IXSTOR,LZERO,' ')
         NPROC = NPROC + 1
      ENDIF
      GOTO 10
99    CONTINUE
      PRINT *,'Made ',NPROC,' copies, ',T,' seconds left'
*
*     Terminate cleanly
*
      CALL FMEND(IRC)
      END
\end{XMPt}
\newpage
\Filename{H2Fatmentutorial-using-tms-tag-info}
\section{Using the TMS tag information}
\par
When FATMEN is installed with the TMS option, access to the TMS
tag information is possible. The TMS keeps two tags for each volume.
As FATMEN already contains a fair amount of user information in the
catalogue, it is not recommended that these tags be used to store
further such information. Firstly, the TMS tags are only available
for data on tape and secondly, access to the TMS tags is slower
and less efficient than to information in the FATMEN catalogue.
It can be useful to set the text tag to the generic name.
If, for example, the generic name 
\begin{XMP}
//cern/l3/prod/data/sdsutt/cc00ftgu
\end{XMP}
points to the volume XP3088 one could set the text tag on
this volume to be 
\begin{XMP}
//cern/l3/prod/data/sdsutt/cc00ftgu
\end{XMP}
This means that if we find the tape XP3088 somewhere and want to
know what is on it, we can query the TMS and have a pointer
to the entry in the FATMEN catalogue.
Alternatively, one could use the command
\begin{XMP}
SEARCH */* VID=XP3088
\end{XMP}
\par
The TMS text tag can be set to the generic name with the shell command
\begin{XMP}
tag //cern/l3/prod/data/sdsutt/cc00ftgu -s
\end{XMP}
\Filename{H2Fatmentutorial-using-tape-pools-in-tms}
\section{Using tape pools within the TMS}
\par
The following is the recommended method of using tape pools
within the TMS.
\begin{UL}
\item
When a new tape is required, it is first allocated out of the
specified pool using the subroutine \Rind{FMALLO}.
\item
The job that allocates this volume should then write the data and,
if successful, move the volume to another named pool, optionally
write-locking it.
\item
If the job fails, it should return the volume to the pool from which
it came.
\item
When the data is no longer required, the entry can be deleted from
the catalogue with the tape volume automatically write-enabled
and freed using the shell rm command.
\end{UL} 
\par
Thus, a FORTRAN program might look like the following.
\begin{XMPt}{Example of using TMS tape pools}
*
*     Allocate a new 3480 from the pool XX_FREE
*
      CALL FMALLO('3480','38K',' ','SMCF_1','XX_FREE',
     +LBANK,' ',VSN,VID,IRC)
*
*     Now write the data ...
*
...
...      
*
*     OK - write-lock the tape and move to XX_DSTS
*     set TMS tag to generic name
*
10    CONTINUE
      CALL FMPOOL(GENAM,LBANK,KEYS,'XX_DSTS','LS',IRC)
      RETURN
*
*     Error - free tape
*     delete TMS tag 
*
20    CONTINUE
      CALL FMPOOL(GENAM,LBANK,KEYS,'XX_FREE','D',IRC)
      RETURN
      END
\end{XMPt}
\Filename{H2Fatmentutorial-protecting-banks-with-link-areas}
\section{Using a Zebra link area to protect the addresses of FATMEN banks}
\index{Link area}
\label{LAREA}
\par
The addresses of FATMEN banks obtained from the FATMEN RZ file, e.g.
using the routines \Rind{FMGET}, \Rind{FMGETK} are stored in a link area by the
FATMEN software. However, only the most recent bank address is saved -
these routines will DROP any previous bank upon entry.
Should you wish to retain the addresses of multiple banks, the
following method may be used.
\begin{XMPt}{Example of using a link area}
      COMMON /USRLNK/LUSRK1,LBANKS,LUSRLS
      DIMENSION LBANKS(NBANKS)
      CHARACTER*255 GENAM(NBANKS)
      PARAMETER (LKEYFA=10)
      DIMENSION KEYS(LKEYFA)
+CDE,FATBANK.
*
      CALL MZLINK (IXSTOR, '/USRLNK/', LUSRK1, LUSRLS, LUSRK1)
*
      DO 10 I=1,NBANKS
      CALL FMGET(GENAM(I)(1:LENOCC(GENAM(I))),LBANK,KEYS,IRC)
      CALL ZSHUNT(IXSTOR,LBANK,LBANKS(I),1,0)
*
*     FMGET will DROP any bank at LTDSFA - including the one
*     we just got!
*
      LTDSFA = 0
 
10    CONTINUE
\end{XMPt}
\Filename{H2Fatmentutorial-using-fmallo}
\section{Using the routine FMALLO to allocate a tape}

The following example is taken from the CERN specific routine \Rind{FMSMCF}.
This routine is called from \Rind{FMOPEN} if the option D is specified.
It performs the following functions:
\begin{OL}
\item
Checks to see if a copy of the specified dataset already exists in the
robot (SMCF)
\item
If not, a tape is allocated from the special pool {\tt gg\_FAT1}, e.g. 
{\tt WS\_FAT1.}
\item
The \Rind{FMCOPY} routine is then invoked to copy the data and update the catalogue.
\end{OL}
\par
As soon as the catalogue has been updated, subsequent accesses will by
default take the copy in the robot.
\begin{XMPt}{Using \protect\Rind{FMALLO} to allocate a tape}
      SUBROUTINE FMSMCF(GENAME,LBANK,IRC)
*
*     Routine to make a copy of the dataset STAGEd in into the robot
*     using FMCOPY option 'S' (STAGE CHANGE)
*
+CDE,FATBANK.
+CDE,FATPARA.
+CDE,TMSDEF.
      CHARACTER*(*) GENAME
      CHARACTER*6   VSN,VID,CHACC
      PARAMETER (LKEYFA=10)
      DIMENSION KEYS(LKEYFA),KEYSR(LKEYFA)
      PARAMETER       (MAXKEY=1000)
      DIMENSION KEYSOU(LKEYFA,MAXKEY),KEYSIN(LKEYFA)
      INTEGER   FMACNT
*
      LGN = LENOCC(GENAME)
*
*     Save old bank address
*
      LOLDFA = LBANK
      LTDSFA = 0
*
*     First, check that a robot copy does not already exist
*
      CALL UCOPY(KEYS,KEYSIN,10)
*
*     Don't compare copy level or location code
*
      KEYSIN(MKCLFA) = -1
      KEYSIN(MKLCFA) = -1
      CALL FMSELK(GENAME(1:LGN),KEYSIN,KEYSOU,NMATCH,MAXKEY,IRC)
      IF(IDEBFA.GE.2)
     +PRINT *,'FMSMCF. found ',nmatch,' matches for media type 2'
      DO 10 I=1,NMATCH
      CALL FMGETK(GENAME(1:LGN),LBANKR,KEYSOU(1,I),IRC)
      CALL UHTOC(IQ(L+KOFUFA+MVIDFA),4,VID,6)
      LVID = LENOCC(VID)
      CALL FMQTMS(VID(1:LVID),LIB,MODEL,DENS,MNTTYP,LABTYP,IC)
      IF(MNTTYP.EQ.'R') THEN
         IF(IDEBFA.GE.0) PRINT *,'FMSMCF. robot copy already exists'
         RETURN
      ENDIF
10    CONTINUE
*
*     Lift new bank for the robot copy
*
      CALL FMLIFT(GENAME(1:LGN),KEYSR,'DISK',' ',IRC)
      CALL FMLINK(GENAME(1:LGN),LBANKR,' ',IRC)
*
*     Blindly copy old bank into new...
*
      CALL UCOPY(IQ(LBANK+KOFUFA+MFQNFA),IQ(LBANKR+KOFUFA+MFQNFA),
     +           NWDSFA)
*
*     and the keys...
*
      CALL UCOPY(KEYS,KEYSR,10)
*
*     Set last access date, date of cataloging and use count
*
      CALL DATIME(IDATE,ITIME)
      CALL FMPKTM(IDATE,ITIME,IPACK,IRC)
      IQ(LBANKR+KOFUFA+MCTTFA) = IPACK
      IQ(LBANKR+KOFUFA+MLATFA) = IPACK
      IQ(LBANKR+KOFUFA+MUSCFA) = 1
*
*     Now, allocate new tape
*
      IC = FMACNT(CHACC)
      CALL FMALLO('3480','38K',' ','SMCF_1',CHACC(5:6)//'_FAT1',
     +LBANKR,' ',VSN,VID,IRC)
      IF(IRC.NE.0) THEN
         IF(IDEBFA.GE.0) PRINT *,'FMSMCF. Cannot allocate robot tape'
         RETURN
         ELSE
         IF(IDEBFA.GE.0) PRINT *,'FMSMCF. allocated ',VSN,' ',VID,
     +                           ' (VSN/VID)'
      ENDIF
*
*     Do the copy
*
      CALL FMCOPY(GENAME,LOLDFA,KEYS,GENAME,LBANKR,KEYSR,'S',IRC)
      IF(IRC.NE.0) PRINT *,'FMSMCF. return code ',IRC,' from FMCOPY'
*
*     Restore bank address
*
      LBANK = LOLDFA
      END
 
\end{XMPt}

\Filename{H2Fatmentutorial-processing-multiple-entries}
\section{Processing multiple entries}

One may frequently wish to perform the same operation on
multiple datasets, or on multiple catalogue entries.
For example, we may wish to reset the user words
for all entries corresponding to Monte Carlo data.
This can be done in a simple way by

\begin{UL}
\item Using the \Rind{FMLFIL} routine to generate a list of entries.
\item Loop over all entries found
\item Read each entry from the catalogue
\item Replace the user vector
\item Update the catalogue
\end{UL}

The following example shows how this can be done using
the novice interface.

\begin{XMPt}{Modifying the user words}
      PARAMETER     (MAXFIL=100)
      PARAMETER     (LKEYFA=10)
      CHARACTER*255 CHFILES(MAXFIL),GENAME
      DIMENSION     KEYS(LKEYFA,MAXFIL)
*
** ***     Data set bank mnemonics
*
*          Keys
      PARAMETER ( MKSRFA= 1, MKFNFA= 2, MKCLFA=7, MKMTFA=8
     1           ,MKLCFA= 9, MKNBFA=10, NKDSFA=10 )
*
** ***     Bank offsets
*
      PARAMETER ( MFQNFA=  1, MHSNFA= 65, MCPLFA= 67, MMTPFA= 68
     1           ,MLOCFA= 69, MHSTFA= 70, MHOSFA= 74
     2           ,MVSNFA= 77, MVIDFA= 79, MVIPFA= 81, MDENFA= 82
     3           ,MVSQFA= 83, MFSQFA= 84, MSRDFA= 85, MERDFA= 86
     4           ,MSBLFA= 87, MEBLFA= 88, MRFMFA= 89, MRLNFA= 90
     5           ,MBLNFA= 91, MFLFFA= 92, MFUTFA= 93, MCRTFA= 94
     6           ,MCTTFA= 95, MLATFA= 96, MCURFA= 97, MCIDFA= 99
     7           ,MCNIFA=101, MCJIFA=103, MFPRFA=105, MSYWFA=106
     8           ,MUSWFA=116, MUCMFA=126, NWDSFA=145
     9           ,MFSZFA=MSYWFA,MUSCFA=MSYWFA+1)

      PARAMETER (LURCOR=200000)
      COMMON/FAT/IXSTOR,IXDIV,IFENCE(2),LEV,LEVIN,BLVECT(LURCOR)
      DIMENSION    LQ(999),IQ(999),Q(999)
      EQUIVALENCE (IQ(1),Q(1),LQ(9)),(LQ(1),LEV)
*
      COMMON /QUEST/IQUEST(100)
      DIMENSION     IVECT(10)
*
*     Initialise FATMEN and Zebra
*
      LUNRZ = 1
      LUNFZ = 2
      CALL FMSTRT(LUNRZ,LUNFZ,'//CERN/OPAL',IRC)
 
      GENAME = '//CERN/OPAL/SIMD/DDST/PASS3/*/*'
      LG = LENOCC(GENAME)
*
*     Find all entries that match 
*
      ICONT = 0
      IFLAG = 1
10    CONTINUE
      CALL FMLFIL(GENAME(1:LG),CHFILES,KEYS,NFOUND,MAXFIL,ICONT,IRC)
*
      DO 20 J=1,NFOUND
      LF = LENOCC(CHFILES(J))
      LBANK = 0
      PRINT *,'Processing ',CHFILES(J)(1:LF)
*
*     Read entry from catalogue
*
      CALL FMGETK(CHFILES(J)(1:LF),LBANK,KEYS(1,J),IRC) 
*
*     here we could add checks on the bank contents
*
*     Store vector IVECT at offset MUSWFA, length 10
*
      CALL FMPUTV(LBANK,IVECT,MUSWFA,10,IRC)
*
*     and write back to the catalogue
*
      CALL FMMOD(CHFILES(J)(1:LF),LBANK,IFLAG,IRC)
*
*     drop bank
*
      CALL MZDROP(IXSTOR,LBANK,' ')
20    CONTINUE

      IF(ICONT.NE.0) GOTO 10

      CALL FMEND(IRC)
      END
\end{XMPt}

The following example finds all generic-names that match
a pattern containing wild cards, and then deletes all 
entries corresponding to tapes that are mounted robotically.
The tape volumes are write-enabled and moved to a TMS pool
so that they can be allocated for future use.

\begin{XMPt}{Example of processing multiple entries in FORTRAN}
      PARAMETER (LURCOR=200000)                                         
      COMMON/CRZT/IXSTOR,IXDIV,IFENCE(2),LEV,LEVIN,BLVECT(LURCOR)       
      DIMENSION    LQ(999),IQ(999),Q(999)                               
      EQUIVALENCE (IQ(1),Q(1),LQ(9)),(LQ(1),LEV)                        
*                                                                       
      COMMON /USRLNK/LUSRK1,LUSRBK,LUSRLS                               
*                                                                       
      COMMON /QUEST/IQUEST(100)                                         
*                                                                       
* Start of FATMEN sequence FATPARA                                      
*                                                                       
** ***     Data set bank mnemonics                                      
*                                                                       
*          Keys                                                         
      PARAMETER ( MKSRFA= 1, MKFNFA= 2, MKCLFA=7, MKMTFA=8              
     1           ,MKLCFA= 9, MKNBFA=10, NKDSFA=10 )                     
*                                                                       
** ***     Bank offsets                                                 
*                                                                       
      PARAMETER ( MFQNFA=  1, MHSNFA= 65, MCPLFA= 67, MMTPFA= 68        
     1           ,MLOCFA= 69, MHSTFA= 70, MHOSFA= 74                    
     2           ,MVSNFA= 77, MVIDFA= 79, MVIPFA= 81, MDENFA= 82        
     3           ,MVSQFA= 83, MFSQFA= 84, MSRDFA= 85, MERDFA= 86        
     4           ,MSBLFA= 87, MEBLFA= 88, MRFMFA= 89, MRLNFA= 90        
     5           ,MBLNFA= 91, MFLFFA= 92, MFUTFA= 93, MCRTFA= 94        
     6           ,MCTTFA= 95, MLATFA= 96, MCURFA= 97, MCIDFA= 99        
     7           ,MCNIFA=101, MCJIFA=103, MFPRFA=105, MSYWFA=106        
     8           ,MUSWFA=116, MUCMFA=126, NWDSFA=145                    
     9           ,MFSZFA=MSYWFA,MUSCFA=MSYWFA+1)                        
                                                                        
* End of FATMEN sequence FATPARA                                        
      CHARACTER*6  DENS                                                 
      CHARACTER*8  LIB                                                  
      CHARACTER*4  LABTYP                                               
      CHARACTER*1  MNTTYP                                               
      CHARACTER*8  MODEL                                                
      CHARACTER*7  ROBMAN(2)                                            
      DATA         ROBMAN(1)/'-Robot '/,ROBMAN(2)/'-Manual'/            
      PARAMETER (LKEYFA=10)                                             
      PARAMETER (MAXFIL=3000)                                           
      DIMENSION KEYS(LKEYFA,MAXFIL)                                     
      CHARACTER*255 FILES(MAXFIL)                                       
      CHARACTER*8   THRONG                                              
      CHARACTER*255 TOPDIR                                              
      CHARACTER*26  CHOPT                                               
      CHARACTER*8   DSN                                                 
*                                                                       
*                                                                       
*     Initialise ZEBRA                                                  
*                                                                       
      CALL MZEBRA(-3)                                                   
      CALL MZSTOR(IXSTOR,'/CRZT/','Q',IFENCE,LEV,BLVECT(1),BLVECT(1),   
     +            BLVECT(5000),BLVECT(LURCOR))                          
      CALL MZLOGL(IXSTOR,-3)                                            
                                                                        
*                                                                       
* *** Define user division and link area like:                          
*                                                                       
      CALL MZDIV  (IXSTOR, IXDIV, 'USERS', 50000, LURCOR, 'L')          
      CALL MZLINK (IXSTOR, '/USRLNK/', LUSRK1, LUSRLS, LUSRK1)          
*                                                                       
*     Units for FATMEN RZ/FZ files                                      
*                                                                       
      LUNRZ = 1                                                         
      LUNFZ = 2                                                         
*                                                                       
*     Initialise FATMEN                                                 
*                                                                       
      CALL FMINIT(IXSTOR,LUNRZ,LUNFZ,'//CERN/delphi',IRC)               
      CALL FMLOGL(1)                                                    
*                                                                       
*     Get list of file names                                            
*                                                                       
      JCONT = 0                                                         
1     CONTINUE                                                          
      CALL FMLFIL('//CERN/DELPHI/P01_*/RAWD/NONE/Y90V00/E*/L*/*',       
     +FILES,KEYS,NFOUND,MAXFIL,JCONT,IRC)                               
*
*     IRC = -1 indicates that there are more files found than
*     fit in FILES(MAXFIL). Calling FMLFIL again with JCONT^=0 will
*     return the next MAXFIL matches
*
      IF(IRC.EQ.-1) THEN                                                
        JCONT = 1                                                       
      ELSE                                                              
        JCONT = 0                                                       
      ENDIF                                                             
                                                                        
      PRINT *,NFOUND,' files found'                                     
                                                                        
      DO 10 I=1,NFOUND                                            
      LENF = LENOCC(FILES(I))                                           
      PRINT *,'Processing ',FILES(I)(1:LENF)                            
      LBANK = 0                                                         
      CALL FMQMED(FILES(I)(1:LENF),LBANK,KEYS(1,I),IMEDIA,IROBOT,IRC)   
      IF(IROBOT.NE.1) GOTO 10                                           
*
*     Display media information and the full generic-name for this entry
*
      CALL FMSHOW(FILES(I)(1:LENF),LBANK,KEYS(1,I),'MG',IRC)            
*
*     Unlock (write-enable) corresponding tape volume
*
      CALL FMULOK(FILES(I)(1:LENF),LBANK,KEYS(1,I),' ',IRC)             
      IF(IRC.NE.0) THEN                                                 
         PRINT *,'Return code ',IRC,' from FMULOK for ',                
     +   FILES(I)(1:LENF)                                               
         GOTO 10                                                        
      ENDIF                                                             
*
*     Move to pool XX_DSTS. 
*
      CALL FMPOOL(FILES(I)(1:LENF),LBANK,KEYS(1,I),                     
     +            'XX_RAWD',' ',IRC)                                    
      IF(IRC.NE.0) THEN                                                 
         PRINT *,'Return code ',IRC,' from FMPOOL for ',                
     +   FILES(I)(1:LENF)                                               
         GOTO 10                                                        
      ENDIF                                                             
*
*     and remove the entry from the FATMEN catalogue
*
      CALL FMRM(FILES(I)(1:LENF),LBANK,KEYS(1,I),IRC)                   
      IF(IRC.NE.0) THEN                                                 
         PRINT *,'Return code ',IRC,' from FMRM for ',                  
     +   FILES(I)(1:LENF)                                               
         GOTO 10                                                        
      ENDIF                                                             
10    CONTINUE                                                          
*
*     any more files?
*
      IF(JCONT.NE.0) GOTO 1                                             
*                                                                       
*     Terminate cleanly                                                 
*                                                                       
      CALL FMEND(IRC)                                                   
                                                                        
      END                                                               
                                                                        
\end{XMPt}
\Filename{H2Fatmentutorial-deleting-multiple-files}
\section{Deleting multiple files using the FATMEN shell}
\par
Neither the FORTRAN routine \Rind{FMRM} nor the shell commmand \Ucom{rm}
accept wild-cards in the generic-name. This is deliberate
and protects against mis-use such as
\begin{XMP}
   \Ucom{rm *}

or worse

   \Ucom{rm */*}
\end{XMP}
which would remove all files in all directories,
subject to the normal protection rules,  and generate
a lot of work for the servers.

Multiple files can be deleted as shown in the previous FORTRAN
example, or via a 2-stage operation in the shell.
If, for example, one wished to delete all files belonging to
JAMIE, one could type
\begin{XMPt}{Using the SEARCH command to prepare a macro}
FM>SEARCH */* USER=JAMIE -D OUTPUT=DELETE.KUMAC
FM>EXEC DELETE
\end{XMPt}
\par
To approximate to the previous FORTRAN example, one could use
\begin{XMPt}{Using the SEARCH command to delete robot tapes}
FM>SEARCH //CERN/DELPHI/P01_*/RAWD/NONE/Y90V00/E*/L*/* VID=I* -D _
   OUTPUT=DELROB.KUMAC
FM>EDIT DELROB | Now add -FU to each line
FM>EXEC DELROB
\end{XMPt}
\par
Note that the removal of entries from the catalogue should
always be done with extreme care.
 
\Filename{H2Fatmentutorial-access-tms-tag-info}
\section{Access to TMS tag information}
\par
The following program shows how the TMS tags associated
with a tape volume may be accessed. As usual in the FATMEN
system, all access is based on the generic-name.
\begin{XMPt}{Access to TMS tag information from FORTRAN}
      PARAMETER (LURCOR=200000)
      COMMON/CRZT/IXSTOR,IXDIV,IFENCE(2),LEV,LEVIN,BLVECT(LURCOR)
      DIMENSION    LQ(999),IQ(999),Q(999)
      EQUIVALENCE (IQ(1),Q(1),LQ(9)),(LQ(1),LEV)
*
      COMMON /USRLNK/LUSRK1,LUSRBK,LUSRLS
*
      COMMON /QUEST/IQUEST(100)
*
* Start of FATMEN sequence FATPARA
*
** ***     Data set bank mnemonics
*
*          Keys
      PARAMETER ( MKSRFA= 1, MKFNFA= 2, MKCLFA=7, MKMTFA=8
     1           ,MKLCFA= 9, MKNBFA=10, NKDSFA=10 )
*
** ***     Bank offsets
*
      PARAMETER ( MFQNFA=  1, MHSNFA= 65, MCPLFA= 67, MMTPFA= 68
     1           ,MLOCFA= 69, MHSTFA= 70, MHOSFA= 74
     2           ,MVSNFA= 77, MVIDFA= 79, MVIPFA= 81, MDENFA= 82
     3           ,MVSQFA= 83, MFSQFA= 84, MSRDFA= 85, MERDFA= 86
     4           ,MSBLFA= 87, MEBLFA= 88, MRFMFA= 89, MRLNFA= 90
     5           ,MBLNFA= 91, MFLFFA= 92, MFUTFA= 93, MCRTFA= 94
     6           ,MCTTFA= 95, MLATFA= 96, MCURFA= 97, MCIDFA= 99
     7           ,MCNIFA=101, MCJIFA=103, MFPRFA=105, MSYWFA=106
     8           ,MUSWFA=116, MUCMFA=126, NWDSFA=145
     9           ,MFSZFA=MSYWFA,MUSCFA=MSYWFA+1)
 
* End of FATMEN sequence FATPARA
      PARAMETER (LKEYFA=10)
      DIMENSION KEYS(LKEYFA)
      CHARACTER*80  GENAM
      CHARACTER*255 CHTAGS
*
*
*     Initialise ZEBRA
*
      CALL MZEBRA(-3)
      CALL MZSTOR(IXSTOR,'/CRZT/','Q',IFENCE,LEV,BLVECT(1),BLVECT(1),
     +            BLVECT(5000),BLVECT(LURCOR))
      CALL MZLOGL(IXSTOR,-3)
 
*
* *** Define user division and link area like:
*
      CALL MZDIV  (IXSTOR, IXDIV, 'USERS', 50000, LURCOR, 'L')
      CALL MZLINK (IXSTOR, '/USRLNK/', LUSRK1, LUSRLS, LUSRK1)
*
*     Units for FATMEN RZ/FZ files
*
      LUNRZ = 1
      LUNFZ = 2
*
*     Initialise FATMEN
*
      CALL FMINIT(IXSTOR,LUNRZ,LUNFZ,'//CERN/CNDIV',IRC)
      CALL FMLOGL(0)
*
*     Get list of file names
*
      GENAM = '//CERN/CNDIV/JAMIE/OUT'
      LG    = LENOCC(GENAM)
      CHTAGS = 'Archive tape for FATMEN source'
*
*     Set tag - default is Text tag
*
      CALL FMTAGS(GENAM(1:LG),LBANK,KEYS,CHTAGS,'S',IRC)
*
*     Read back and print tag
*
      CALL FMTAGS(GENAM(1:LG),LBANK,KEYS,CHTAGS,'G',IRC)
      PRINT *,IRC,CHTAGS(1:LENOCC(CHTAGS))
*
*     Terminate cleanly
*
      CALL FMEND(IRC)
 
      END
 
\end{XMPt}

\Filename{H2Fatmentutorial-runtime-tailoring}
\section{Run time tailoring of the FATMEN system}

Certain features of the FATMEN system may be reconfigured at run-time,
both via the FORTRAN callable interface or interactive shell.
A description of what can be tailored is given below.
\subsection{Host name and account fields}

The FATMEN software attempts to determine various information, such as the
current host name and account, by calling system routines. In some cases, 
it may be desirable to override the values returned. This can be done by
setting variables (environmental variables in Unix, global symbols in VAX/VMS). 

For example, the SHIFT facility at CERN is composed of a number of systems
(shift1, shd01 etc.). By default, a file created on a given system
would not be accessible on another as the node name check would fail. 
To override this, the variable \Rind{FMHOST} is set, 
as below.
\begin{XMP}
\Ucom{setenv FMHOST shift}
\end{XMP}

The account field can be set in a similar way, e.g.
\begin{XMP}
FMACNT:==JDSCT
\end{XMP}
for a VAX/VMS system.
\subsection{Media attributes}
\par
If FATMEN has been installed using the TMS flag, media attributes are obtained
from the TMS (Tape Management System). (A preliminary interface to the VMTAPE
package also exists and is selected via the flag VMTAPE).
\par
On other systems, the defaults can be overridden globally, or on a per volume basis.
Ideally, the media attributes for a given site should be automatically selected
via installation flags. However, the \Rind{FMEDIA} callable routine and MEDIA shell command
permit the default values to be overridden at any time.
\par
For example, to change the generic device name of media type 2, which defaults
at CERN to CT1, the following command could be used.
\begin{XMP}
MEDIA 2 3480 TA90
\end{XMP}
\subsection{Tailoring the FATMEN selection}
\index{selection}
\index{KEYS selection}
\par
By default, all entries in the FATMEN catalogue are visible and may
be examined using the shell command ls. One may limit the range
of entries that can be seen by defining a list of location codes,
copy levels and media types. These lists also affect those files
that may be accessed, and the selection procedure itself.
\par
For example, an experiment which computes at several laboratories will
probably have copies of the tapes containing the DST information at
each site. The shell command
\begin{XMP}
set/location 1-3
\end{XMP}
would prevent any catalogue entries with a location code outside the range
1-3 from being visible. This is particularly useful for large collaborations
that typically have many copies of each dataset.
\par
It is also important to set the correct range for efficiency in data access.
In the case where an experiment has multiple copies of a file on different tapes,
each at a different laboratory, FATMEN will normally issue a TMS query
for each volume to see if it is accessible. Setting the location code
correctly reduces the number of TMS queries and hence improves data access time.
\par
In the case of the location code or copy level, only those entries with
a value in the defined range will be visible or accessible.
In the case of the medium type, the order of the values is also important.
By default, FATMEN first looks for a disk file, then a copy on a 3480 cartridge and so
on. The shell command
\begin{XMP}
set/media 5,1,2
\end{XMP}
would cause FATMEN to first look for a copy on an Exabyte 8500 cartridge,
then disk and finally a 3480 cassette, assuming the default medium attributes.

\Filename{H2Fatmentutorial-plotting-onfo}
\section{Plotting information from the FATMEN catalogue}

The following program shows how certain information, such as the
file size, number of days since last access, etc. can be histogrammed
using HBOOK and saved in a file for further processing with PAW.

Before running, the variable THRONG should be set to the FATMEN
group that is to be processed, as shown below.
\begin{XMPt}{Setting the \Lit{THRONG} variable on various systems}
Unix systems:        THRONG=OPAL;export THRONG {\rm (Bourne and Korn shells)}
-------------        setenv THRONG OPAL        {\rm (C shell)}

VMS systems:         throng==opal
------------

VM/CMS systems:      SETENV THRONG OPAL
---------------
\end{XMPt}
\begin{XMPt}{Plotting FATMEN information with HBOOK}
      PARAMETER (LURCOR=200000)                                         
      COMMON/CRZT/IXSTOR,IXDIV,IFENCE(2),LEV,LEVIN,BLVECT(LURCOR)       
      DIMENSION    LQ(999),IQ(999),Q(999)                               
      EQUIVALENCE (IQ(1),Q(1),LQ(9)),(LQ(1),LEV)                        
*                                                                       
      COMMON /USRLNK/LUSRK1,LUSRBK,LUSRLS                               
*                                                                       
      COMMON /QUEST/IQUEST(100)                                         
      CHARACTER*8   THRONG
*                                                                       
* Start of FATMEN sequence FATPARA                                      
*                                                                       
** ***     Data set bank mnemonics                                      
*                                                                       
*          Keys                                                         
      PARAMETER ( MKSRFA= 1, MKFNFA= 2, MKCLFA=7, MKMTFA=8              
     1           ,MKLCFA= 9, MKNBFA=10, NKDSFA=10 )                     
*                                                                       
** ***     Bank offsets                                                 
*                                                                       
      PARAMETER ( MFQNFA=  1, MHSNFA= 65, MCPLFA= 67, MMTPFA= 68        
     1           ,MLOCFA= 69, MHSTFA= 70, MHOSFA= 74                    
     2           ,MVSNFA= 77, MVIDFA= 79, MVIPFA= 81, MDENFA= 82        
     3           ,MVSQFA= 83, MFSQFA= 84, MSRDFA= 85, MERDFA= 86        
     4           ,MSBLFA= 87, MEBLFA= 88, MRFMFA= 89, MRLNFA= 90        
     5           ,MBLNFA= 91, MFLFFA= 92, MFUTFA= 93, MCRTFA= 94        
     6           ,MCTTFA= 95, MLATFA= 96, MCURFA= 97, MCIDFA= 99        
     7           ,MCNIFA=101, MCJIFA=103, MFPRFA=105, MSYWFA=106        
     8           ,MUSWFA=116, MUCMFA=126, NWDSFA=145                    
     9           ,MFSZFA=MSYWFA,MUSCFA=MSYWFA+1)                        
                                                                        
* End of FATMEN sequence FATPARA                                        
*KEEP,FATBUG.
      COMMON /FATUSE/ IDEBFA, IDIVFA, IKDRFA, KOFSFA, KOFUFA, LBFXFA
     +              , LSAVFA, LTOPFA, LBBKFA, LBGNFA, LTDSFA, LBDSFA
     +              , LPRTFA, NTOPFA, LUFZFA, IOUPFA, IOBKFA, IODSFA
     +              , LLNLFA, LLNHFA
*KEND.
      CHARACTER*8   DSN                                                 
      EXTERNAL      UROUT
*                                                                       
*                                                                       
*     Initialise ZEBRA                                                  
*                                                                       
      CALL MZEBRA(-3)                                                   
      CALL MZSTOR(IXSTOR,'/CRZT/','Q',IFENCE,LEV,BLVECT(1),BLVECT(1),   
     +            BLVECT(5000),BLVECT(LURCOR))                          
      CALL MZLOGL(IXSTOR,-3)                                            
                                                                        
*                                                                       
* *** Define user division and link area like:                          
*                                                                       
      CALL MZDIV  (IXSTOR, IXDIV, 'USERS', 50000, LURCOR, 'L')          
      CALL MZLINK (IXSTOR, '/USRLNK/', LUSRK1, LUSRLS, LUSRK1)          
*                                                                       
*     Units for FATMEN RZ/FZ files                                      
*                                                                       
      LUNRZ = 1                                                         
      LUNFZ = 2                                                         
      CALL GETENVF('THRONG',THRONG)
      LTH = LENOCC(THRONG)
*                                                                       
*     Initialise FATMEN                                                 
*                                                                       
      CALL FMINIT(IXSTOR,LUNRZ,LUNFZ,'//CERN/'//THRONG(1:LTH),IRC)
      CALL FMLOGL(0)
*
*     Initialise HBOOK
*
      CALL HLIMIT(-20000)
*
*     Book histograms
*
      CALL HBOOK1(1,'File Size (MB)',50,0.,200.,0.)
      CALL HBOOK1(2,'Number of accesses',50,0.,50.,0.)
      CALL HBOOK1(3,'Number days since last access',50,0.,300.,0.)
      CALL HBOOK1(4,'Number days since catalogued',50,0.,300.,0.)
      CALL HBOOK1(5,'Number days since created',50,0.,300.,0.)
      CALL HBOOK1(6,'Medium',5,0.,5.,0.)
      CALL HIDOPT(0,'BLAC')
*
*     Loop over all files
*
      CALL FMLOOP('//CERN/*/*',-1,UROUT,IRC)
*
*     Print and store the histograms
*
      CALL HPRINT(0)
      CALL HRPUT(0,'FATTUPLE.'//THRONG(1:LTH),'N')
*
*     Terminate cleanly
*
      CALL FMEND(IRC)
      END

      SUBROUTINE UROUT(PATH,KEYS,IRC)
*                                                                       
* Start of FATMEN sequence FATPARA                                      
*                                                                       
** ***     Data set bank mnemonics                                      
*                                                                       
*          Keys                                                         
      PARAMETER ( MKSRFA= 1, MKFNFA= 2, MKCLFA=7, MKMTFA=8              
     1           ,MKLCFA= 9, MKNBFA=10, NKDSFA=10 )                     
*                                                                       
** ***     Bank offsets                                                 
*                                                                       
      PARAMETER ( MFQNFA=  1, MHSNFA= 65, MCPLFA= 67, MMTPFA= 68        
     1           ,MLOCFA= 69, MHSTFA= 70, MHOSFA= 74                    
     2           ,MVSNFA= 77, MVIDFA= 79, MVIPFA= 81, MDENFA= 82        
     3           ,MVSQFA= 83, MFSQFA= 84, MSRDFA= 85, MERDFA= 86        
     4           ,MSBLFA= 87, MEBLFA= 88, MRFMFA= 89, MRLNFA= 90        
     5           ,MBLNFA= 91, MFLFFA= 92, MFUTFA= 93, MCRTFA= 94        
     6           ,MCTTFA= 95, MLATFA= 96, MCURFA= 97, MCIDFA= 99        
     7           ,MCNIFA=101, MCJIFA=103, MFPRFA=105, MSYWFA=106        
     8           ,MUSWFA=116, MUCMFA=126, NWDSFA=145                    
     9           ,MFSZFA=MSYWFA,MUSCFA=MSYWFA+1)                        
                                                                        
* End of FATMEN sequence FATPARA                                        
      PARAMETER (LURCOR=200000)                                         
      COMMON/CRZT/IXSTOR,IXDIV,IFENCE(2),LEV,LEVIN,BLVECT(LURCOR)       
      DIMENSION    LQ(999),IQ(999),Q(999)                               
      EQUIVALENCE (IQ(1),Q(1),LQ(9)),(LQ(1),LEV)                        
      CHARACTER*(*) PATH
      PARAMETER     (LKEYFA=10)
      DIMENSION     KEYS(LKEYFA)
      DIMENSION     NDAYS(3)
      COMMON/QUEST/IQUEST(100)
      IRC   = 0
      LBANK = 0
      LP    = LENOCC(PATH)
      CALL FMGETK(PATH(1:LP),LBANK,KEYS,IRC)
*
*     Fill histograms
*
      IF(IQ(LBANK+MFSZFA).NE.0)
     +CALL HFILL(1,FLOAT(IQ(LBANK+MFSZFA)),0.,1.)
      IF(IQ(LBANK+MUSCFA).NE.0)
     +CALL HFILL(2,FLOAT(IQ(LBANK+MUSCFA)),0.,1.)
      CALL FMDAYS(PATH(1:LP),LBANK,KEYS,NDAYS,' ',IRC)
      CALL HFILL(3,FLOAT(NDAYS(3)),0.,1.)
      CALL HFILL(4,FLOAT(NDAYS(2)),0.,1.)
      CALL HFILL(5,FLOAT(NDAYS(1)),0.,1.)
      CALL HFILL(6,FLOAT(IQ(LBANK+MMTPFA)),0.,1.)
      CALL MZDROP(IXSTOR,LBANK,' ')
      END
\end{XMPt}

\part{FATMEN -- User Guide}
\include{fatuser}
\part{FATMEN -- Installation and Management Guide}
%%%%%%%%%%%%%%%%%%%%%%%%%%%%%%%%%%%%%%%%%%%%%%%%%%%%%%%%%%%%%%%%%%%
%                                                                 %
%   FATMEN User Guide and Reference manual                        %
%                                                                 %
%   Fatmen Part 4: Installaion Guide                              %
%                                                                 %
%   This document needs the following external EPS files:         %
%   none                                                          %
%                                                                 %
%   Editor: Michel Goossens / CN-AS                               %
%   Last Mod.:  7 June 1993 13:20 mg                              %
%                                                                 %
%%%%%%%%%%%%%%%%%%%%%%%%%%%%%%%%%%%%%%%%%%%%%%%%%%%%%%%%%%%%%%%%%%%

\Filename{H1Fatmeninstallation-general-hints}
\chapter{General hints}
\Filename{H2Fatmeninstallation-pam-file-availability}
\section{Availability of PAM files, libraries and FATMEN shell}
\par
The FATMEN package is installed as part of the CERN program library.
If you have a version of the CERN program library corresponding
to CERN Computer Newsletter 197 or later, you should have FATMEN
on your system.
\index{Library}
\index{PAM}
\index{server}
\par
\index{EXEC}
\index{FM}
\index{shell}
\par
The standard installation (CNL 201 and after) creates the FATMEN
callable interface (part of PACKLIB), the FATMEN shell and the
FATMEN server.
\Filename{H2Fatmeninstallation-Using-ZEBRA-HBOOK-with-FATMEN}
\section{Using ZEBRA, HBOOK etc. with FATMEN}
\subsection{The size of the users' store}
\par
\index{ZEBRA}
The FATMEN package creates a division of type 'P' (package)
in the store that the user declares in the call to
FMINIT.(See on Page~\pageref{FMINIT}) This division
is declared with NW=10000 and NWMAX=100000. (See the ZEBRA users'
guide\cite{bib-ZEBRA} for details of the routine MZDIV).
Thus, the users store should be sufficiently large to accomodate
this division.
\subsection{Using HBOOK and FATMEN}
\index{HBOOK}
\par
If the user also wishes to use HBOOK, ZEBRA should be initialised
first followed by a call to HLIMIT with a negative argument. This will
indicate to HBOOK that ZEBRA has already been initialised.
\par
Note that the FATMEN FORTRAN routines do not save and restore the
current directory within an RZ file. 
\subsection{Calling MZWIPE}
\index{MZWIPE}
\par
It is normally safe to call MZWIPE to clean divisions in the store
which contains the FATMEN division with the following exception:
\begin{UL}
\item
Do not use MZWIPE to delete all package divisions
({\tt IXWIPE=IXSTOR+23}). This will delete all banks created with
FMLIFT (see Page~\pageref{FMLIFT}) and not saved using
FMPUT (see Page~\pageref{FMPUT}).
\end{UL}
\Filename{H2Fatmeninstallatio-Using-FATMEN-without-TMS}
\section{Using FATMEN without a Tape Management System}
\par
\index{TMS}
\index{VMTAPE}
The only Tape Management Systems currently supported by FATMEN
are the HEPVM TMS, and VMTAPE. These are selected by the PATCHY statements
\begin{XMP}
+USE,TMS.

and

+USE,VMTAPE.

respectively.
\end{XMP}
\par
\index{TMS}
The Tape Management System is responsible for maintaining information
on tape volumes, such as availability, label type etc. When installed
without the TMS flag, FATMEN takes the default values for each
media type as defined at installation time, or by a call to the
routine FMEDIA (or the MEDIA command in the shell).
\par
To see the current defaults, type the command MEDIA in the shell.
\par
As well as tailoring the default values for each media type,
a user exit, FMUTMS, can be provided to override the default
values for a given volume. See the description of the FMUTMS 
routine in the user interface section of this manual for more details.
\Filename{H1Fatmeninstallation-installing-fatmen}
\chapter{Installing FATMEN}
\par
As described above, the installation of the FATMEN software as part
of the standard CERN program library installation. FATMEN relies
heavily on the program library, and so this is a pre-requisite
for its usage.
\par
The standard program library installation procedure
will generate the FATMEN FORTRAN interface (in PACKLIB),
and three modules. These are the shell (FM), the server (FATSRV)
and a program to create a new, empty RZ file for use with
FATMEN (MKFATNEW).
\index{mkfatnew}
\index{FATNEW}
\index{Creating a new catalogue}

{\tt mkfatnew} is just a simple script that calls the {\tt FATNEW}
program. For example, on VAX/VMS systems the following maybe used.
\begin{XMPt}{Example MKFATNEW command file}
$! fatsys:==CERN ! For example
$! fatgrp:==LHC  ! For example
$
$  type/nopage sys$input

Please give the name of the FATMEN system. This name forms
the top-level of the FATMEN catalogue, e.g. //CERN

$eod
$  inquire/nopunc ans "FATSYS? "
$  if ans.eqs."" then ans = "CERN"
$  fatsys==ans - "//"
$  type/nopage sys$input

Please give the name of the FATMEN group.

$eod
$  inquire/nopunc ans "FATGRP? "
$  if ans.eqs."" then exit
$  fatgrp==ans - "FM"
$  write sys$output ""
$  inquire/nopunc fatdir "Directory where FATMEN catalogue should reside? "
$  olddir = f$environment("DEFAULT")
$  set default 'fatdir'
$  create/directory [.todo]
$  create/directory [.tovm]          ! Only at CERN !!!
$  create/directory [.done]
$  set file/protection=w:rw todo.dir
$  fatman="FM''fatgrp'"
$  set file/acl=(id='fatman',access=read+write,options=default) todo.dir
$  set file/protection=w:rw tovm.dir ! Only at CERN !!!
$  set file/protection=w:rw done.dir
$  run cern:[pro.exe]fatnew
$  set default 'olddir'
\end{XMPt}
\Filename{H2Fatmeninstallation-Installing-FATMEN-on-new-machine}
\section{Installing FATMEN on a new machine}
\par
The first step is the installation of the CERN program libraries.
Once this has been achieved, FATMEN should be configured as below.
\subsection{Access to data}
\par
Access to data through FATMEN is currently supported on:
\begin{OL}
\item
VM/CMS systems running the HEPVM software, in particular SETUP and STAGE.
\item
VM/CMS systems without the HEPVM software are supported
provided that the CERN REXX local function package as well as the
GIME and DROP execs are installed. 
\index{Local function}
\index{REXX local function package}
\index{DROP}
\index{GIME}
\index{RXLOCFN}
\index{REXX}
\index{HEPVM}
\item 
\index{VMBATCH}
\index{VMTAPE}
VM/CMS systems running VMBATCH and/or VMTAPE, provided that the
REXX local function package as well as the
GIME and DROP execs are installed. 
\item
\index{VAXTAP}
VAX/VMS systems. The CERN Program Library Package VAXTAP is required
for access to tapes.
\item
\index{SHIFT}
Unix systems. Tape support is currently only possible on the CERN Cray
system, the CERN SHIFT project and the
L3 Apollo network at CERN.
\par
CERN is developing generalised Unix tape software (staging and tape
mounting) which will be used to provide tape support on all other
Unix systems. This software is based upon the existing Cray and
SHIFT syntax and so the FATMEN interface is already written.
\item
MVS systems. Only FORTRAN I/O is supported in the current version.
The IBM FORTRAN routine FILEINF is used whereever possible,
together with the CERN Program Library Package FTPACK, written
by R. Matthews.
\end{OL}
\par
On all other systems some modifications to the code are required for
data access. These modifications are restricted to the routines
FMOPEN, FMCLOS and FMCOPY.
\subsection{Using the FATMEN catalogue}
\par
The FATMEN catalogue is simply a FORTRAN direct access file, processed
using the CERN ZEBRA package. Users access the file in read-only mode,
with updates being sent to and applied by a server (one per experiment).
Updates are sent as ZEBRA FZ files (exchange mode, ASCII mapping).
\subsection{Configuring FATMEN}
\subsubsection{VM/CMS systems}
\par
Machines that do not run the HEPVM software may use FATMEN
provided that they install the GIME and DROP execs, and
the REXX local function package. To install the REXX local
function package, simply copy the files RXLOCFN SEGMENT and
RXLOCFN MODULE to your local machine. They will be
automatically loaded when needed. Some functions
in this package will return a null-string or zero unless
certain HEPVM CP modifications are installed. However, all of the
functions required for FATMEN will work without any problem.
\par
If you require any of the above software and do not have
access to CERNVM (or any other machine running the HEPVM
software), simply sent a mail message to JAMIE@CERNVM. 
\par
Upon initialisation the FATMEN software issues an 'EXEC GIME FMgroup',
e.g. 'EXEC GIME FML3'. Thus, if the FATMEN catalogue for L3 is NOT
kept on the 191 disk of FML3, an appropriate
\index{names file}
\index{GIME}
entry in the GIMEUSER, GIMEGRP or GIMESYS NAMES file is required.
An additional entry is also required in the normal NAMES file
so that the updates can be correctly sent to the appropriate service
machine.
\par
Systems running VMBATCH should activate the necessary code by
putting
\begin{XMP}
+USE,VMBATCH.
\end{XMP}
into the installation cradle for FATMEN. 
(Different code is required to obtain the username, jobname
and account field in a VMBATCH job to that used in HEPVM batch).
\par
Systems running VMTAPE should activate the necessary code by putting
\begin{XMP}
+USE,VMTAPE.
\end{XMP}
This will provide access to tape files.
If user tapes are catalogued in the VMTAPE catalogue (TMC),
select also
\begin{XMP}
+USE,VMTMC.
\end{XMP}
\par
Systems running both VMTAPE and VMBATCH may simply add
\begin{XMP}
+USE,VMCENTER.
\end{XMP}
\subsubsection{VAX/VMS systems}
\par
Upon initialisation, the client looks for a file in the directory
specified by the symbol FMgroup, e.g. FML3.
If the FATMEN catalogue is kept in
DISK\$FAT:\lsb FATMEN.L3\rsb, 
one should type
\begin{XMP}
FML3:==DISK$FAT:\lsb FATMEN.L3\rsb
\end{XMP}
This directory must contain the subdirectories
\lsb .TODO\rsb , \lsb .DONE\rsb . The \lsb .TODO \rsb directory
must be writeable by all members of the L3 collaboration (in this example).
\par
If the catalogue is on a disk that has disk quotas enabled, then an ACL
must be established on the directory \lsb .TODO\rsb as in the example
below. In this example, the top level directory is \lsb FMCDF \rsb  
belonging to user FMCDF. The identifier CDF\_EXPERIMENT is held by
all members of the CDF experiment and is an identifier with the
RESOURCE attribute. Setting up the directory in this manner allows
all CDF users to send updates to the FATMEN server for their experiment
and avoids the need for entries in the quota file for every CDF user
for the disk in question. 
Only entries for FMCDF and CDF\_EXPERIMENT are required.
\par
If disk quotas are not enabled on the volume in question, then 
allowing group write access to the \lsb .TODO \rsb    
directory is sufficient.
\begin{XMP}
(IDENTIFIER=FMCDF,ACCESS=READ+WRITE+EXECUTE+DELETE+CONTROL)
(IDENTIFIER=CDF_EXPERIMENT,ACCESS=READ+WRITE+EXECUTE)
(IDENTIFIER=FMCDF,OPTIONS=DEFAULT,ACCESS=READ+WRITE+EXECUTE+DELETE+CONTROL)
(IDENTIFIER=CDF_EXPERIMENT,OPTIONS=DEFAULT,ACCESS=READ+WRITE+EXECUTE)
\end{XMP}
\par
The configuration for the server is performed by the
sample command file shown below.
\par
For the server only, a variable FATSYS must be
set to the appropriate name, should the catalogue name not be //CERN
\begin{XMP}
$!
$! Example FATSERV.COM
$!
$ dd = f$cvtime(,,"WEEKDAY")
$ tt = f$time()
$ hh = f$trnlnm("SYS$NODE")
$ write sys$output ""
$ write sys$output "FATSERV starting at ''dd' ''tt' on ''hh'"
$ write sys$output ""
$ !
$ ! Set FATMEN system
$ !
$ FATSYS:==CERN
$ !
$ ! Set FATMEN group
$ !
$ FATGRP:==FMCNDIV
$ !
$ ! Set FATMEN wakeup interval in seconds
$ !
$ ! Define FMCNDIV if not defined in system login
$ !
$ FMCNDIV:==DISK\$CERN:\lsb FMCNDIV\rsb
$ FMWAKEUP:==30
$ !
$ ! Set FATMEN log level
$ !
$ FMLOGL:==3
$ write sys$output -
"FATMEN group set to ''FATGRP', wakeup interval is ''FMWAKEUP' seconds"
$ run cern:\lsb pro.exe\rsb fatsrv
\end{XMP}
\index{remote staging}
\index{multi-file staging}
\index{VAXTAP}

See the \Rind{VAXTAP}~\cite{bib-VAXTAP} long writeup for details on configuring
the VAX tape handling software, in particular for remote and
multi-file staging.
\subsubsection{Unix systems}
\par
Upon initialisation, the client looks for a file in the directory
specified by the variable FMexperiment, e.g. FML3. Thus,
if the FATMEN catalogue is kept in /users/fatmen/l3, the variable
FML3 should be set to this path:
\begin{XMP}
FML3=/users/fatmen/l3; export FML3
\end{XMP}
This directory must contain the subdirectories
todo and done. The todo directory
must be writeable by all members of the L3 collaboration (in this example).
\par
For the server only, a variable FATSYS must be
set to the appropriate name, should the catalogue name not be //CERN
\begin{XMP}
FATSYS=DESY;export FATSYS
\end{XMP}
\par
The configuration for the server is performed by the
sample script shown below.
\begin{XMP}
#!/bin/sh
#
# Example FATSERV script.
#
t="date"
h="hostname"
echo
echo FATSERV starting at $t on $h
echo
FATGRP=FML3   ;export FATGRP
FATSYS=CERN   ;export FATSYS
FMWAKEUP=10;   export FMWAKEUP
FML3=/fatmen/fml3; export FML3
echo FATMEN group set to $FATGRP , wakeup interval is $FMWAKEUP seconds
echo
/cern/pro/bin/fatsrv
\end{XMP}
\Filename{H1Fatmeninstallation-remote-access-to-fatmen-catalogue}
\chapter{Remote access to the FATMEN catalogue}
\index{remote catalogues}
\index{catalogue access}
\par
Remote access to the FATMEN catalogue is currently supported
over DECnet (between VAX/VMS systems only) NFS, AFS and CSPACK.
CSPACK access has not yet been tested on VM/CMS or MVS systems.
\Filename{H2Fatmeninstallation-DECnet-access-to-FATMEN-catalogues}
\section{DECnet access to FATMEN catalogues}
\par
\index{DECnet}
To access a remote catalogue over DECnet, simply include
the node specification in the symbol definition. e.g.
\begin{XMP}
FMCDF:==FNALF::USR$ROOT37:[FMCDF]
\end{XMP}
\Filename{H2Fatmeninstallation-NFS-access-to-FATMEN-catalogues}
\section{NFS access to FATMEN catalogues}
\par
\index{NFS}
To access a remote catalogue over NFS, simply define
the required environmental variable as in the example
below.
\begin{XMP}
FMOPAL=/fatmen/fmopal; export FMOPAL

df /fatmen
Filesystem    Total KB    free %used   iused %iused Mounted on
fatcat:/fatmen  409600  138704   66%   -       -    /fatmen

\end{XMP}

\Filename{H2Fatmeninstallation-AFS access-to-FATMEN-catalogues}
\section{AFS access to FATMEN catalogues}
\index{AFS}
To access a remote catalogue over AFS, one simply
has to define the appropriate environment variable
to point to the directory where the catalogue resides,
e.g.

\begin{XMP}
export FML3=/afs/cern.ch/fatmen/fml3
\end{XMP}

\Filename{H2Fatmeninstallation-CSPACK-access-to-FATMEN-catalogues}
\section{CSPACK access to FATMEN catalogues}
\index{CSPACK}

If FATMEN has been installed with the {\tt CSPACK}
option enabled (\Lit{+USE,CSPACK} in the PATCHY step of
the library build), then remote catalogues may also
be accessed and updated using {\tt CSPACK}. The
standard {\tt ZSERV} must be installed on the remote
system on which the catalogue resides, as described in
the {\tt CSPACK}~\cite{bib-CSPACK} manual. One then defines
the symbol (VAX/VMS) or environmental variable (UNIX)
that points to the directory where the catalogue resides
with the syntax {\tt node:path}.

Thus, to enable {\tt CSPACK} access to the {\tt L3}
catalogue residing on the node {\tt fatcat}, the
appropriate {\tt C shell} definition would be

\begin{XMP}
setenv FML3 fatcat:/fatmen/fml3
\end{XMP}

\Filename{H2Fatmeninstallation-FATCAT-dedicated-FATMEN-server-at-CERN}
\section{FATCAT - the dedicated FATMEN server at CERN}
\index{fatcat}

At CERN, the FATMEN catalogues are stored on the node FATCAT.
This is currently an RS6000 which is only used for FATMEN catalogue
management. The recommended way of accessing a FATMEN catalogue at
CERN is the following:
\begin{OL}
\item
NFS mount the \Lit{/fatmen} file system
\item
Define environmental variables to point to the directory of interest
\end{OL}

An example of how this may be done is shown below.

\begin{XMPt}{NFS mounting the /fatmen file system on a Unix platform}

mount fatcat:/fatmen /fatmen

\end{XMPt}

\begin{XMPt}{NFS mounting the /fatmen file system on a VMS platform}

$NFSMOUNT fatcat:"/fatmen" fatmen

\end{XMPt}

\begin{XMPt}{Defining the environmental variables on a Unix system}
#/bin/ksh
for i in /fatmen/fm*
   do
     typeset -u fatgrp
     fatpath=$i
     fatgrp=`basename $i`
     echo Setting $fatgrp to $fatpath ...
     eval $fatgrp=$fatpath;export $fatgrp
   done
\end{XMPt}

\begin{XMPt}{Defining the environmental variables on a VAX/VMS system}
$ loop:
$ fatman = f$search("FATMEN:[000000]FM*.DIR")
$ if fatman .eqs. "" then exit
$ fatman = f$parse(fatman,,,"NAME")
$ fatdir = "FATMEN:[''fatman']"
$ write sys$output "Setting ''fatman' to ''fatdir'..."
$ 'fatman' :== 'fatdir'
$ goto loop
\end{XMPt}

\Filename{H1Fatmeninstallation-distributing-catalogue-upfates}
\chapter{Distribution of catalogue updates}
\index{catalogue}
\index{updates}
\index{servers}
\index{distribution of updates}
\index{remote catalogues}
\index{DECnet}
\index{bitnet}
\index{TCPIP}
\par
It is unfortunately at least unrealisitic and more likely impossible
to access a single catalogue world-wide and so one typically
has multiple copies of the catalogue. These catalogues
are automatically keep up to date by the FATMEN servers.
The updates may be sent over Bitnet, DECnet or TCP/IP.
However, the servers must be configured so that they
know to which nodes updates should be sent. In addition,
it may be desirable to maintain subsets of the catalogue
at remote sites. For example, information on raw data
may remain at the laboratory where the data is taken
whereas all DST information is sent to all collaborating
institutes. 
\par
\index{NAMEFIND}
\index{NAMEFD}
\index{names file}
The configuration file that is used has the same format
on all systems. This format is that of a VM/CMS names file.
On non-VM systems, these files are processed using the
FORTRAN callable routine, NAMEFD, available in the CERN
program library KERNLIB.
\Filename{H2Fatmeninstallation-Configuring-servers-on-VM-systems}
\section{Configuring servers on VM systems}
\par
This is performed by adding entries to the NAMES file
of the servers in question. For each remote server
one can define up to 16 generic name patterns, each
of which may include wild-cards. For more details
see the example below. Updates may be sent to
remote IBM machines, VAXes connected via Interlink
and Unix machines using the CSPACK software. 
\par
Note that the default for VM systems is to send
the updates using the SENDFILE exec. This can be
overridden for individual entries either by sending
the updates to a gateway machine, as described later,
or by specifying an alternate exec, as for the entry
FMVAX.
\begin{XMP}
:nick.FATSERVERS
               :list FMSACLAY FATCAT FMIN2P3 FMRAL FMVAX

:nick.FMVAX
               :userid.fmsmc
               :node.vxcrna
               :DIR1.//cern/smc/dst/*
               :exec.fat2vax

:nick.FMSACLAY
               :userid.fmsmc
               :node.frsac12
               :DIR1.//cern/smc/dst/*
 
:nick.FMIN2P3
               :userid.fmsmc
               :node.frcpn11
               :DIR1.//cern/smc/dst/*
 
:nick.FMRAL
               :userid.fmsmc
               :node.ukacrl
               :DIR1.//cern/smc/dst/*
 
:nick.FATCAT
               :userid.fmfatcat
               :node.cernvm
               :DIR1.//cern/smc/dst/*
 
\end{XMP}
\par
In the preceeding example, all updates referring to generic
names beginning //CERN/SMC are sent to the server defined
by the name FATCAT, whereas only updates referring to
names beginning //CERN/SMC/DST are sent to the
servers at SACLAY, IN2P3 and RAL.
Note the the FMFATCAT account is in fact a gateway
into the Unix-world (using the program in PATCH FATCAT
in the FATMEN PAM file and that FMVAX is an account of
a VAX linked to the IBM via Interlink, but this is completely
transparent to the server.
\par
On systems other than CERNVM, updates that are to be sent to
non-VM systems should pass via a gateway service machine.
\begin{XMP}
:nick.FATDESY  :userid.R01H1
               :node.dhhdesy3
               :DIR1.//cern/h1/*
               :protocol.mvsjob
               :gateway.fmgate at cernvm

:nick.FATVAX   :userid.fmh1
               :node.vxdesy
               :dir1.//cern/h1/*
               :protocol.tcpip
               :gateway.fmgate at cernvm
               :receive.yes
\end{XMP}
\par
In the above example, updates destined for FATDESY and FATVAX
are not send directly, but pass via an intermediate service machine.
wakes up at predefined intervals and transfers any files to the specified
remote nodes. In the case of the entry FATVAX, files are also
retrieved from the remote system.
\subsection{Transferring updates to VAX/VMS systems via Interlink}
\par
This method should only be used at CERN. For other systems,
a gateway service machine should be used. See the description
below for setting up a gateway service machine.
\par
Files sent from IBM VM/CMS systems via the Interlink VM-DECnet
gateway will arrive in the default directory for the FAL DECnet
object, typically
\begin{XMP}
SYS\$SPECIFIC:\lsb FAL\$SERVER \rsb
\end{XMP}
\par
By default, the files will be named FATMEN.RDRFILE. At CERN
a small modification has been made to the Interlink software
so that the files arrive as VAXUSER.FILENAME\_FILETYPE,
thus files sent to FMOPAL would be called
\begin{XMP}
FMOPAL.FATMEN_RDRFILE
\end{XMP}
\par
These files can be copied to the correct directory using the
command file FATRL.COM, included in the PATCH DCL of the FATMEN
pamfile.
\Filename{H2Fatmeninstallation-Configuring-servers-on-VMS-MVS-and-Unix-systems}
\section{Configuring servers on VMS, MVS and Unix systems}
\par
The names file configuration is exactly the same on VMS, MVS and
Unix systems as on VM nodes. The processing of the names file
is performed by a FORTRAN program, FATSEND,  which uses the CERN program
library routine NAMEFD and the updates are transferred
using the CSPACK routines. 
\par
For each remote server, a subdirectory should be created.
(If the subdirectories are note created, they will be
created as required by the server).
In the case of the example names file shown below, we
would have the subdirectories \lsb .TOD0SG01 \rsb,
\lsb .TOD0SF01 \rsb and so on (VAX/VMS systems). 
The FATMEN server copies all updates into each of these
subdirectories and the program FATSEND copies the updates
to the specified remote nodes, using either TCP/IP or DECnet,
and deletes those files that are successfully copied.
DECnet transfers are only possible between VAX/VMS systems.
\par
On MVS systems, the protocol BITNET is also valid for transmission
of updates to remote VM systems. This is done using the TSO
transmit command.
\par
An example names file is shown below.
\begin{XMP}
:nick.FATSERVERS
               :list FMSGI FMVAX 

:nick.FMSGI    :userid.fmd0
               :node.d0sg01
               :protocl.tcpip
               :dir1//fnal/d0/*
               :receive.yes 
               :queue:USR$ROOT37:[FMD0.TODO]

:nick.FMVAX    :userid.fmd0
               :node.d0sf01
               :protocl.tcpip
               :dir1//fnal/d0/*
\end{XMP}

In this example, servers connecting the the machine \Lit{D0SG01}
will transmit updates in both directions, rather than the
default, which is to send updates only.
This option is useful for the transmission of updates
to a machine that does not accept incoming connections,
as is the case with VAX systems running the UCX TCP/IP software,
the Cray at CERN and MVS systems.

If the tag \Lit{:queue} is specified, the updates are placed in
the specified directory. If not, they are placed in the subdirectory
\Lit{todo} of the account that is used to performed the transfer.

If a subdirectory TOVM exists, the FATMEN server will attempt to send
all updates via Interlink to CERNVM. This option is only intended for
use at CERN.
\Filename{H2Fatmeninstallation-Using-gateway-service-machine-on-VM-systems}
\section{Using a gateway service machine on VM systems}
\par
To send updates from a VM node into a non-VM system,
use the :gateway tag to define a gateway service machine.
The VM service machine will send all updates to this
gateway machine, which in turn will transmit them by
the specified protocol at regular intervals using the
program FATSEND.
\par
Valid protocols are TCPIP, BITNET and MVSJOB.
TCPIP, which is the default, results in the files
being transferred using the CSPACK routines, as
described above. One should normally also set the
:receive.yes option so that updates are also retrieved
from these remote systems.
\par
Specfiying BITNET together with a gateway machine permits
updates to be sent to remote nodes at specified
intervals, rather than immediately.
\par
Although it is possible to use SENDFILE to send files to
MVS nodes, these files are generally unreadable and
hence the preferred method is to specify the protocol
MVSJOB. This will result in an IEBGENER job being submitted
to the remote node, which will copy the updates to a temporary
file in the input queue of the specified server.
The usual restrictions on remote job submission apply.
\Filename{H1Fatmeninstallation-fatsend-program}
\chapter{The Program FATSEND}
\par
This program is generated using the command MAKEPACK FATSEND.
Alternatively, one may use a job such as the following:
\begin{XMP}

VAX/VMS systems

$if p1 .nes. "" then goto 'p1
$ y:
$ypatchy cern:[pro.pam]ZEBRA.pam fatsend.for :go 
+use,qcde.
+use,vaxvms,*fatsend.
+exe.
+pam,11,r=qcde.
+pam,12,t=c,a. fatmen.cards
+quit
$ f:
$for fatsend
$ l:
$link fatsend,'lib$,sys$input/opt
sys$library:vaxcrtl/shareable
sys$system:sys.stb
multinet_socket_library/share
$ exit

Unix systems (Use f77 instead of xlf on non-AIX machines)

ypatchy /cern/pro/pam/ZEBRA.pam fatsend.f :go <<!
+use,qcde.
+use,ibmrt,*fatsend.
+exe.
+pam,11,r=qcde.
+pam,12,t=c,a. fatmen.cards
+quit
!
xlf -c -q extname -q charlen=32000 fatsend.f 
xlf fatsend.o -L/cern/new/lib -lpacklib -lc -o fatsend

\end{XMP}
\par
The program can be tailored by setting the variables FMLOGL and FMWAKEUP, e.g.
\begin{XMP}

$fmlogl:==3
$fmwakeup:==3600 ! Once an hour

\end{XMP}
\par
On Unix systems, the tailoring is done using environmental variables.
The following example is for the C shell.
\begin{XMP}

setenv fmlogl 3
setenv fmwakeup 3600 

\end{XMP}
On VM/CMS systems, the SETENV command should be used, as in
\begin{XMP}
SETENV FMWAKEUP 3600
\end{XMP}
\par
On MVS systems no equivalent to environmental variables exists.
Therefore names files entries are used, as shown below.
\begin{XMP}
:nick.fatsrv  :wakeup.60   :logl.0

:nick.fatsend :wakeup.3600 :logl.3

:nick.FMSGI
               :mvsid.r01d0
               :userid.fmd0
               :node.d0sg01
               :protocl.tcpip
               :dir1//fnal/d0/*
\end{XMP}
\Filename{H1Fatmeninstallation-installing-vaxtap}
\chapter{Installing VAXTAP for tape access on VAX/VMS systems}
\index{VAXTAP}
\par
FATMEN interfaces to the CERN Program Library package VAXTAP
to provide tape support on VAX/VMS systems. The long writeup
should also be consulted.~\cite{bib-VAXTAP}
\par
The package is installed by running a command file that can
be generated by the following PATCHY run:
\begin{XMP}
$YPATCHY CERN:[PRO.PAM]VAXTAP.PAM INSTALL.COM :GO
+USE,INSTALL,T=EXE.
+PAM.
+QUIT
\end{XMP}
\par
Once this command file has been extracted, installation of the package
proceeds by typing 
\begin{XMP}
@INSTALL
\end{XMP}
and answering the questions. If VAXTAP is to be installed on a system
without access to the HEPVM Tape Management System, as is likely
to be the case when installing it outside CERN, answer NO to the first
question. Answering A or ALL to the second question will cause
the installation to complete without any further dialogue.
\par
When the command file completes, the file SETUP\_STARTUP.COM should
be edited and tailored.
\begin{XMP}
$ !---------------------------------------------------------------------------*
$ !
$ !      Startup command file for SETUP/STAGE/LABELDUMP
$ !      Modify logical name definitions as required for your node.
$ !---------------------------------------------------------------------------*
$ !
$ !
$ !
$ !     Create lnm table for SETUP information ...
$       create/name_table/parent=lnm$system_table/prot=w:wred lnm$setup
$ !
$ !      Define directory for .EXE files
$      define/system setup_exe cern_root:[exe]
$ !
$ !      Allow usage of tapes interactively
$ !
$      define/system setup_enabled "INTERACTIVE"
$ !
$ !      Disallow specific users from using tapes (useful to stop troublemakers)
$ !
$ !      define/system setup_notapes "DECNET,CERNET"
$ !
$ !      Allow tapes in these batch queues
$ !
$!      define/system setup_queues "SYS$TAPES"
$!      define/system setup_queues "SYS$BATCH,SYS$TAPES"
$!      define/system setup_queues "*" ! all queues
$ !
$ !      Set up lists of available device types
$ !
$      define/system setup_tk50s "VSDD18$MKA700:"
$      define/system setup_8200s "UXDDB1$MUB0:"
$      define/system setup_exabytes setup_8200s ! Can also have aliases...
$ !
$ !      Allow tape staging
$ !
$      define/system stage_tapes "YES"
$ !      Must also ensure that DISK$STAGE exists...
\end{XMP}
\par
See also the description of the FMEDIA routine and the MEDIA shell
command for information on configuring generic device names in
FATMEN. The generic device names used by VAXTAP must match those
used by FATMEN. Thus, if the generic device type for a given
medium is set to DAT, the logical name SETUP\_DATS must point
to a list of valid device names.
\par
See also the installation instructions in PATCH DOC of the VAXTAP
pam file.
\Filename{H1Fatmeninstallation-VM-service-machines}
\chapter{The VM FATMEN service machines}
\par
These machines should normally be running
in disconnected mode and autologged
at system startup time. These machines run a FORTRAN program which
calls an EXEC to issue WAKEUP upon the arrival of new RDR files.
These are then read in, processed, then WAKEUP is called again.
\Filename{H2Fatmeninstallation-Setting-up-new-service-machine}
\section{Setting up a new service machine}
\par
The service machine should have the name FM{\bf throng}, e.g
\index{throng}
{\bf throng}
\Rind{FMALEPH}, \Rind{FMCHARM2}, \Rind{FMCPLEAR}, \Rind{FMDELPHI}, etc.
\footnote{The FATMEN servers may have any names, provided
that suitable NAMES file entries are established.}
The account created requires a 191 disk of sufficient size
to maintain the expected file catalogue information. Some 800 bytes
are required per file catalogue entry, thus for a catalogue containing
15000 files, about 20 cylinders are required. A 193 disk is required for
maintaining journal information.
This disk should be at least 5 cylinders.
\begin{table}[h]
\caption{Mini-disks required for FATMEN service machines}
\begin{center}\begin{tabular}{|l|p{.8\linewidth}|}
\hline
191    & Disk for FATMEN RZ file, log file and \Lit{PROFILE EXEC}.
         The profile exec contains only one line - \Lit{'EXEC FATPROF'}
         20 cylinders is normally sufficient for an initial allocation for this
         disk.  \\
192    & Link to the disk containing the server code and EXECs.
         At CERN, these are kept on the 191 disk of userid FATMEN.\\
193    &  Disk for keeping journal files. About 5 cylinders are normally
         required. These files are created automatically by the server
          following each update and are named \Lit{FATyyjjj FZhhmmss}, 
         e.g. \Lit{FAT90001 FZ120000} for a file created at 
         12:00:00 on January 1st, 1990.\\
\hline
\end{tabular} \end{center}\end{table}
\par
On CERNVM, the FATMEN service machines are monitored and controlled
by the FATONE machine. Any new service machines must be registered
by running the FATMEN exec on the 191 disk of this machine, as shown
below.
\begin{XMPt}{Modifying the list of FATMEN servers}
defaults list fatmen       /* list current list of FATMEN servers */
FMALEPH FMCNDIVR FMDELPHI FMCHARM2 FML3 FMOPAL FMCPLEAR CPDELPHI
defaults set fatmen fmep03 /* add new FATMEN server */
fatmen (edit               /* edit list of FATMEN servers */
\end{XMPt}
\Filename{H2Fatmeninstallation-Generating-FATMEN-EXECs}
\section{Generating the FATMEN EXECs}
\par
The EXECs used by FATMEN can be generated using the following PATCHY
cradle:
\begin{XMP}
+EXE.
+ASM, 21,R=!./*BEGIN ! EXEC */
+USE,REXX.
+PAM,11.
+QUIT.
\end{XMP}
\par
The resultant file should be processed using the \Lit{SPLITFIL} exec
as shown below:
\begin{XMP}
\Ucom{splitfil fatrexx rexx}
-----> Split FATREXX REXX A1 into pieces
Generate FM EXEC A
Generate FATSERV EXEC A
Generate FAT2CERN EXEC A
Generate FAT4WARD EXEC A
Generate FATJOURN EXEC A
Generate FATUSE EXEC A
Generate FATLOG EXEC A
Generate PURGE EXEC A
\end{XMP}

These EXECs should reside on the FATMEN 191 disk.
\Filename{H2Fatmeninstallation-Monitoring-FATMEN-servers}
\section{Monitoring the FATMEN servers}

{\bf Privileged users} may send the FATMEN servers
management commands to monitor their progress, check that they are active
etc. Examples of use are:
\begin{XMP}
Jamie@Cernvm;
\Ucom{TELL FMDDDIVR QSPOOL}
12:44:53  * MSG FROM FMDDDIVR: I HAVE 0 FILES IN MY RDR
Jamie@Cernvm;
\Ucom{TELL FMDDDIVR HELLO}
12:44:58  * MSG FROM FMDDDIVR: HELLO AND HOW ARE YOU TODAY?
Jamie@Cernvm;
* See how much disk space CHARM2 service machine has used
\Ucom{TELL FMCHARM2 QDISK A}
12:45:19 * MSG FROM FMCHARM2: LABEL  VDEV M STAT CYL TYPE BLKSIZE FILES  BLKS USED-(%) BLKS LEFT BLK TOTAL
12:45:19 * MSG FROM FMCHARM2: CHARM2 191  A R/W   20 3380    4096     6       35-01         2961      3000
\end{XMP}
\begin{table}[h]
\caption{List of commands currently supported by the FATMEN service machine}
\begin{center} \begin{tabular} {|>{\tt}l|p{.8\linewidth}|}
\hline
HELLO    & Check whether server is active.
         If the server does not respond immediately, it means that it
         is either down, or processing an update. As multiple updates
         can be grouped together, it can happen that the server does not
         respond for a considerable period of time to a \Lit{HELLO} message. \\
HELP     & Displays this list of commands. \\
STOP     & Stop the server, but does not log it off. \\
QDISK    & Issue a \Lit{QUERY DISK} command and send the output back
         to the originator of the message. \\
QSPOOL   & Return the number of RDR files that the service
         machine currently has in its reader. \\
DROP     & Cause the server to call the \Lit{DROP}
         exec to drop the specified link. \\
GIME     & Cause an \Lit{EXEC GIME} to be issued.\\
CLOSE    & Close the console log and spool it to the owner of
         the service machine. \\
NEWLOG   & Cause the server to open a new log file. \\
PURGE    & Purge (delete) journal files. May be sent automatically
          whenever a backup of the FATMEN RZ file has been taken.\\
*FATLOG* & This command is issued automatically from the FATMEN user code to log
           FATMEN activity.\\
LOGOFF   & Shutdown the service machines prior to system shutdown or 
          other scheduled interruption, such as ORACLE backup.\\
\hline
\end{tabular}\end{center}\end{table}
\subsection{Names file entries for the FATMEN Servers}
\par
Each FATMEN server should have at least two entries in their NAMES file.
Those on CERNVM may also have a third, FATSERVERS, which points to
the list of remote servers who should automatically receive updates
from CERNVM.
\par
The two entries that are always required are FATOWNERS and FATOPERATORS.
At the present time, usernames in either of these lists are allowed
to issue commands in the above table, and receive SMSGs when the servers
stop.
\begin{XMPt}{Example of \Lit{NAMES} file for the FATMEN server}
:nick.FATOPERATORS                :list.fatop1 fatop2
:nick.FATOP1                      :userid.console
                                  :node.cernvm
:nick.FATOP2                      :userid.opsutil
                                  :node.cernvm
:nick.FATOWN1                     :userid.fatone
                                  :node.cernvm
:nick.FATOWN2                     :userid.wojcik
                                  :node.frcpn11
:nick.FATOWNERS                   :list.fatown1 fatown2 jamie
\end{XMPt}

\Filename{H2Fatmeninstallation-Generating-ORACLE-tables}
\section{Generating the ORACLE tables}
\par
\index{group}
\index{ORACLE}
The ORACLE tables for the new group are created by editting the
SQL statements below, replacing {\bf throng}
by the group name in question. Then, type
\Lit{SQLPLUS user/password @file}, where \Lit{file} is
the name of the file containing these statements.
These SQL commands are in the patch \Lit{FATSQL} on the FATMEN pamfile.
 
\begin{Fighere}
\caption{Creation of the ORACLE table for a new throng}
\label{FORATAB}
\begin{minipage}[t]{.494\linewidth}
\begin{XMP}
INSERT INTO Fatmen VALUES ('CERN', 'throng')
/
REM ***   SPECIFIC TABLES FOR throng
REM ***   Note: 1) CHAR(240) => Oracle V5.1
REM ***         2) TMS later substitutes Volumes_
 
CREATE TABLE GNames_throng ( GName CHAR(240) NOT NULL,
                             GN# NUMBER NOT NULL)
/
CREATE TABLE Files_throng ( File# NUMBER NOT NULL,
                            GN# NUMBER NOT NULL,
                            Copylevel NUMBER(2) NOT NULL,
                            Location NUMBER NOT NULL,
                            Hostname CHAR(8) NOT NULL,
                            Fullname CHAR(240) NOT NULL,
                            Hosttype CHAR(16),
                            Opersys CHAR(12),
                            Fileformat CHAR(4) NOT NULL,
                            Userformat CHAR(4),
                            Startrec# NUMBER,
                            Endrec# NUMBER,
                            Startblk# NUMBER,
                            Endblk# NUMBER,
                            Recformat CHAR(4),
                            Reclength NUMBER,
                            Blklength NUMBER,
                            Creation DATE,
                            Catalogation DATE,
                            Lastaccess DATE,
                            Active CHAR(1) NOT NULL,
                            Creatorname CHAR(8),
                            Creatoraccount CHAR(8),
                            Creatornode CHAR(8),
                            Creatorjob CHAR(8),
                            Protection NUMBER(2),
\end{XMP}
\end{minipage}\hfill
\begin{minipage}[t]{.494\linewidth}
\begin{XMP}
                            Userword0 NUMBER,
                            Userword1 NUMBER,
                            Userword2 NUMBER,
                            Userword3 NUMBER,
                            Userword4 NUMBER,
                            Userword5 NUMBER,
                            Userword6 NUMBER,
                            Userword7 NUMBER,
                            Userword8 NUMBER,
                            Userword9 NUMBER,
                            Sysword0 NUMBER,
                            Sysword1 NUMBER,
                            Sysword2 NUMBER,
                            Sysword3 NUMBER,
                            Sysword4 NUMBER,
                            Sysword5 NUMBER,
                            Sysword6 NUMBER,
                            Sysword7 NUMBER,
                            Sysword8 NUMBER,
                            Sysword9 NUMBER,
                            Comments CHAR(80),
                            Mediatype CHAR(1))
/
CREATE TABLE FXV_throng ( File# NUMBER NOT NULL,
                          Fileseq# NUMBER NOT NULL,
                          Vol# NUMBER NOT NULL,
                          Volseq# NUMBER)
/
CREATE TABLE Volumes_throng ( Vol# NUMBER NOT NULL,
                              VSN CHAR(6) NOT NULL,
                              VID CHAR(6) NOT NULL,
                              VIDprefix NUMBER,
                              Density NUMBER)
/
\end{XMP}
\end{minipage}
\end{Fighere}
\Filename{H2Fatmeninstallation-Generating-SQLDS-tables}
\section{Generating the SQL/DS tables}
\index{group}
\index{SQL/DS}

The SQL/DS tables for the new group are created by editting the
SQL statements below, replacing {\bf throng}
by the group name in question.

\begin{Fighere}
\caption{Creation of the SQL/DS tables for a new throng}
\label{FSQLTAB}
\begin{minipage}[t]{.498\linewidth}
\begin{XMP}
INSERT INTO Fatmen VALUES ('CERN', 'throng')
 
CREATE TABLE GNames_throng ( GName CHAR(240) NOT NULL,   -
                             GN# INTEGER NOT NULL)
CREATE TABLE Files_throng ( File# INTEGER NOT NULL,      -
                            GN# INTEGER NOT NULL,        -
                            Copylevel INTEGER NOT NULL,  -
                            Location INTEGER NOT NULL,   -
                            Hostname CHAR(8) NOT NULL,   -
                            Fullname CHAR(240) NOT NULL, -
                            Hosttype CHAR(16),           -
                            Opersys CHAR(12),            -
                            Fileformat CHAR(4) NOT NULL, -
                            Userformat CHAR(4),          -
                            Startrec# INTEGER,           -
                            Endrec# INTEGER,             -
                            Startblk# INTEGER,           -
                            Endblk# INTEGER,             -
                            Recformat CHAR(4),           -
                            Reclength INTEGER,           -
                            Blklength INTEGER,           -
                            Creation DATE,               -
                            Catalogation DATE,           -
                            Lastaccess DATE,             -
                            Active CHAR(1) NOT NULL      -
                            Creatorname CHAR(8),         -
                            Creatoraccount CHAR(8),      -
                            Creatornode CHAR(8),         -
                            Creatorjob CHAR(8),          -
                            Protection INTEGER(2))
\end{XMP}
\end{minipage}\hfill
\begin{minipage}[t]{.498\linewidth}
\begin{XMP}
Alter table files_throng add (Userword0 INTEGER,         -                                  -
                              Userword1 INTEGER,         -
                              Userword2 INTEGER,         -
                              Userword3 INTEGER,         -
                              Userword4 INTEGER,         -
                              Userword5 INTEGER,         -
                              Userword6 INTEGER,         -
                              Userword7 INTEGER,         -
                              Userword8 INTEGER,         -
                              Userword9 INTEGER,         -
                              Sysword0 INTEGER,          -
                              Sysword1 INTEGER,          -
                              Sysword2 INTEGER,          -
                              Sysword3 INTEGER,          -
                              Sysword4 INTEGER,          -
                              Sysword5 INTEGER,          -
                              Sysword6 INTEGER,          -
                              Sysword7 INTEGER,          -
                              Sysword8 INTEGER,          -
                              Sysword9 INTEGER,          -
                              Comments CHAR(80),         -
                              Mediatype CHAR(1))
CREATE TABLE FXV_throng ( File# INTEGER NOT NULL,        -
                          Fileseq# INTEGER NOT NULL,     -
                          Vol# INTEGER NOT NULL,         -
                          Volseq# INTEGER)
CREATE TABLE Volumes_throng ( Vol# INTEGER NOT NULL,     -
                              VSN CHAR(6) NOT NULL,      -
                              VID CHAR(6) NOT NULL,      -
                              VIDprefix INTEGER,         -
                              Density INTEGER)
\end{XMP}
\end{minipage}
\end{Fighere}

\Filename{H1Fatmeninstallation-restoring-RZ-files-from-oracle}
\chapter{Restoring the RZ files from ORACLE or SQL/DS}

Data can be restored from ORACLE or SQL/DS in one of two ways: either by
directly recreating the FATMEN RZ file, or by sending each entry
as an update in FZ format to the RDR of the virtual machine.
The former is useful if the entire file is lost or corrupt, the
latter for recovering individual entries or for sending the
recovered data to a remote system.

\Filename{H2Fatmeninstallation-Recreating-FATMEN-RZfile-directly}
\section{Recreating the FATMEN RZ file directly}

To restore the complete RZ file from ORACLE or SQL/DS,
logon to the service
machine of the group in question and type RESTORE.
This exec currently extracts all
active.\footnote{Active files are those not marked for delete. File entries
marked for delete are be recovered by resetting the active flag to \Lit{Y}.}
information from the ORACLE SQL/DS tables.

\Filename{H2Fatmeninstallation-Extracting-information-from-ORACLE-SQLDS-as-FZupdates}
\section{Extracting information from ORACLE or SQL/DS as FZ updates}

If a small number of entries are to be restored, or the restored
file is to be sent to a remote site, it may be convenient to
send each entry as an FZ file to the server, with disturbing
its normal operation. The program in PATCH FATO2F on the FATMEN
PAM can be used for this purpose. The program reads a group name
followed by a list of generic names from FORTRAN unit 5, i.e. the
terminal or program stack unless overridden by a FILEDEF command.
The generic names input may include a wild-card as shown below.
The comments, delimited by an exclamation mark (!), should not
be included in a real file and are only used here to help explain
the various options. The routine FMUPDT (see on Page~\pageref{FMUPDT})
may be used to group the updates together if required.
\begin{XMPt}{Example of a file for restoring from ORACLE into FZ}
CHARM2                                 ! Initialise for group CHARM2
//CERN/CHARM2/SPECIAL/R4248            ! Recover an individual entry
//CERN/CHARM2/TEST/*                   ! Recover a complete tree
\end{XMPt}
\Filename{H1Fatmeninstallation-fatmen-code}
\chapter{The FATMEN code}
\Filename{H2Fatmeninstallation-Structure-FATMEN-PAM}
\section{Structure of the FATMEN PAM file}
\par
Cradles to generate the FATMEN FORTRAN from the FATMEN pamfile are
maintained on the FATMEN pamfile. In general, it is sufficient to
{\tt +USE} the pilot patch plus the target machine,
e.g. +USE,*FATLIB,IBM. In addition, the switch CZ must be selected
with {\tt +USE,CZ.} for all machines which will access
a central file catalogue using the ZEBRA server.
\index{ZEBRA server}
\index{ZS ZEBRA server}
\begin{table}[h]
\caption{Description of the cradles in the FATMEN PAM file}
\begin{center} \begin{tabular}{|l|p{.8\linewidth}|}
\hline
*FATSQL   &FORTSQL for the CERNVM service machines.
          The routines must be preprocessed using the PCC exec before 
         compilation.\\
*FATUSER &Sample program on the PAM file. \\
*FATO2Z  &Code to restore the RZ files from the ORACLE database. \\
*FATSRV  &Code for the VM service machines. \\
*FMCDF   &Command definition file (CDF) for the FATMEN shell.
         It must be processed with the KUIPC command before 
         compilation.\\
*FMKUIP  &Code for the FATMEN shell.
          To generate the FATMEN shell, the output of FMINT and 
          the CDF file,
          preprocessed using the KUIPC command, must also be included.
          The CDF file is in patch
          FMCDF on the FATMEN pam.\\
*FATLIB  &Code for the FORTRAN callable interface, maintained in FATLIB TXTLIB.
          It is also required by the FATMEN shell.  \\
\hline
\end{tabular} \end{center}\end{table}
\Filename{H2Fatmeninstallation-Installing-FATMEN-software}
\section{Installing the FATMEN software}
\par
Installation of the FATMEN software consists of two steps:
\begin{OL}
\item
Installation of the FATMEN library (part of PACKLIB).
\item
Creation of the FATMEN shell program.
\end{OL}
\subsection{Installation of FATMEN on CERNVM}
\subsubsection{Generating the FATLIB library}
\par
The extraction of the code is made with the following PATCHY cradle:
\begin{XMP}
+EXE.
+USE,QCDE.
+USE,*FATLIB,IBM.
+PAM,11, R=QCDE, T=A.ZEBRA
+PAM,12.       , T=A.FATMEN
+QUIT.
\end{XMP}
followed by the standard procedure to generate the TXTLIB, namely
VFORT asm, EDITLIB asm, TXT GEN FATLIB asm.
\subsubsection{The FATMEN module (for the command line interface)}
\par
The extraction of the code is made with the following PATCHY cradle:
\begin{XMP}
+EXE.
+ASM,31.
+USE,*FMKUIP,IBM.
+USE,FMCDF,T=EXE,DIV.
+PAM,11, T=A.FATMEN
+QUIT.
\end{XMP}
which generates two files: the standard ASM file with the FORTRAN code
and the diverted ASM2 file containing input to the KUIP processor.
Both the ASM code and the output from the KUIP processor must be
compiled and linked to build the FATMEN module.
\par
The following exec has been used at CERN
to perform the complete operation:
\begin{XMP}
/**/
 Address Command
 'EXEC ZPATCHY FATMEN ( CRADLE(FATMEN) ASM(FATMEN) ASM2(FMCDF CDF)'
 
 'KUIPC FMCDF'  /* To generate the FORTRAN equivalent of the CDF file */
 
 'EXEC VFORT FATMEN'
 'EXEC VFORT FMCDF'
 'EXEC VFORT ENDMODU'         /* Dummy routine to "squeeze" the module*/
 'EXEC CERNLIB FATLIB ( LINK'
 
 'LOAD FATMEN FMCDF ( CLEAR NOAUTO'
 'INCLUDE ENDMODU'
 'GENMOD FATMEN ( TO ENDMODU'
 Exit
/*BEGIN ENDMODU FORTRAN */
      BLOCK DATA ENDMODU
      END
\end{XMP}
\subsubsection{\protect\label{HORACLE}Processing the ORACLE routines for the 
FATMEN server}
The ORACLE
FORTSQL routines can be generated using the following cradles:
\begin{XMP}
+EXE.
+ASM, 21,R=!./*BEGIN ! FORTSQL */
+USE,ORACLE.
+USE,*FATSQL.
+PAM,11.
+QUIT.
\end{XMP}
The output file should then be processed by the SPLITFIL command,
which will create a separate file for each routine.
These can then be pre-processed by the PCC exec as follows:
\begin{XMP}
splitfile fatsql fortsql
-----> Split FATSQL FORTSQL A1 into pieces
Generate BLANKDEK FORTSQL A
Generate FMLOGI FORTSQL A
Generate FODEL FORTSQL A
Generate FOGET FORTSQL A
Generate FOPUT FORTSQL A
FAT3@cernvm;
pcc iname=fmlogi host=fortran
 
ORACLE Precompiler: Version 1.2.13.10 - Production on Thu Feb 22 17:32:04 1990
 
Copyright (c) 1987, Oracle Corporation, California, USA.  All rights reserved.
 
 
Precompiling FMLOGI.FORTSQL
FAT3@cernvm;
vfort fmlogi
VFORT compilation options in effect are: CHARLEN(32767) FLAG(W) GOSTMT OPT(2) 
MAP NOSDUMP.
File 'FMLOGI FORTRAN A1' will be processed.
VS FORTRAN VERSION 2 ENTERED.  17:32:07
 
**FMLOGI** END OF COMPILATION 1 ******
 
VS FORTRAN VERSION 2 EXITED.   17:32:10
 
FAT3@cernvm;
\end{XMP}
\subsubsection{Generating the code for the FATMEN server}
This code can be generated with the following cradle:
\begin{XMP}
+USE,IBM,*SQL,*FATSRV.
+EXE.
+PAM,11, R=QCDE, T=A.  ZEBRA PAM *
+PAM,12,  T=A. FATMEN PAM *
+QUIT.
\end{XMP}
The server module can then be built as follows:
\begin{XMP}
CERNLIB V5ORAFIX V5ORACLE FATLIB
LOAD FATSRV OSDDAT OSDU2OS (CLEAR NOMAP NODUP NOAUTO)
GENMOD FATSRV
\end{XMP}
\subsubsection{Generating the FATMEN server for remote VM systems}
This is performed in the same way as on CERNVM with the following
exception:
\begin{UL}
\item
If neither ORACLE or SQL/DS are to be used, the \begin{XMP}+USE,*SQL\end{XMP}
statement should be omitted.
\end{UL}
If ORACLE is to be used, the procedure for generating the FORTSQL
routines is described on Page~\pageref{HORACLE}
should be followed. If SQL/DS is to be used, the procedure
below should be followed.
\subsubsection{\protect\label{HSQLDS}Processing the SQLDS routines for the 
FATMEN server}
The SQL/DS
FORSQL routines can be generated using the following cradles:
\begin{XMP}
+EXE.
+ASM, 21,R=!./*BEGIN ! FORSQL */
+USE,SQLDS.
+USE,*FATSQL.
+PAM,11.
+QUIT.
\end{XMP}
The output file should then be processed by the SPLITFILE command,
which will create a separate file for each routine.
These can then be pre-processed by the SQLIFY exec as follows:
\begin{XMP}
splitfile fatsql forsql
-----> Split FATSQL FORSQL A1 into pieces
Generate BLANKDEK FORSQL A
Generate FMLOGI FORSQL A
Generate FODEL FORSQL A
Generate FOGET FORSQL A
Generate FOPUT FORSQL A
FAT3@Cernvm;
sqluser
SQL195 ( 01A8  R ) RR
FAT3@Cernvm;
sqlify fmlogi
ARI0717I START SQLPREP EXEC: 02/23/90 10:27:47 SET
ARI0320I THE DEFAULT DATABASE NAME IS SQLDBA.
ARI0663I FILEDEFS IN EFFECT ARE:
SYSIN    DISK     FMLOGI   FORSQL   A1
SYSPRINT DISK     FMLOGI   LISTPREP A1
SYSPUNCH DISK     FMLOGI   FORTRAN  A1
ARISQLLD DISK     ARISQLLD LOADLIB  R1
ARI0713I PREPROCESSOR ARIPRPF CALLED WITH THE FOLLOWING PARAMETERS:
........ PREP=FMLOGI, NOCHECK,ISOLATION(CS)
ARI0710E ERROR(S) OCCURRED DURING SQLPREP EXEC PROCESSING.
ARI0796I END SQLPREP EXEC: 02/23/90 10:27:48 SET
FAT3@Cernvm(00008);
FAT3@Cernvm;
vfort fmlogi
CRNVFT030I VFORT compilation options in effect are: CHARLEN(32767) FLAG(W) 
GOSTMT OPT(2) MAP NOSDUMP NOPRINT.
CRNVFT000I File 'FMLOGI FORTRAN A1' will be processed.
VS FORTRAN VERSION 2 ENTERED.  10:27:55
 
**FMLOGI** END OF COMPILATION 1 ******
 
**SQLINT** END OF COMPILATION 2 ******
 
VS FORTRAN VERSION 2 EXITED.   10:27:55
 
/**********************************************************************/
/*                                                                    */
/*  Title  : Invoke SQL preprocessor                                  */
/*  ======                                                            */
/*                                                                    */
/*  Format : SQLIFY   fn <ft>                                         */
/*  ======               $ASMSQL                                      */
/*                                                                    */
/*  Author : J. Wood, Systems Group, CCD, RAL, 10/04/86               */
/*  ======                                                            */
/*                                                                    */
/**********************************************************************/
Address Command
Signal on HALT ; buf=0
 
Arg fn ft fm . '(' 'USERID' user pwd .
 
If user='' Then foruser=''
   Else Do
      user='STRIP'(user)
      foruser='USERID='user'/'pwd','
      End   /* Else Do */
 
 
ret=0
 
If fm = '' Then fm = 'A'
 
Select
   When fn='' Then Do
      Say 'Format is: SQLIFY file_name <file_type>'
      Exit 4
      End   /* Do When fn='' */
   When Abbrev('PLISQL',ft,1) Then ft='PLISQL'
   When Abbrev('ASMSQL',ft,1) Then ft='ASMSQL'
   When Abbrev('FORSQL',ft,1) Then ft='FORSQL'
   Otherwise Do
      ft='*'
      'MAKEBUF' ; buf=rc
      'LFILE' fn ft '* ( FIFO'
      n=Queued()
      reply=1
      j=0
      type.1='PLISQL'
      mode.1='A'
      typlist='/PLISQL/ASMSQL/FORSQL'
      Do i=1 To n
         Pull fn1 ft1 fm1 .
         loctyp='/'||Strip(ft1)
         If Pos(loctyp,typlist)>0 Then Do
            Do k=1 To j
               If ft1=type.k & fm1=mode.k Then Iterate i
               End   /* Do k=1 ... */
            j=j+1
            type.j=ft1
            mode.j=fm1
            End   /* Do If Pos ... */
         End   /* Do n */
 
      If j>1 Then Do k=1 While 1
         Say 'There is more than one file of filename' fn
         Say 'Select by number:'
         Do i=1 To j
            Say i':' fn type.i mode.i
            End   /* Do i= ... */
         Pull reply
         Select
            When ^Datatype(reply,'W') Then Iterate k
            When reply >j | reply < 1 Then Iterate k
            Otherwise Leave k
            End   /* Select */
         End   /* Do k=1 ... */
 
      ft=type.reply
      fm=mode.reply
      End   /* Do Otherwise */
   End   /* Select */
 
'STATE' fn ft fm
If rc^=0 Then Do
   Say fn ft fm 'does not exist'
   ret=rc
   Signal HALT
   End   /* Do If rc^=0 */
rc='CPUSH'('PRT')
'CP SPOOL 00E TO' Userid()
 
lang = left(ft,3)
if ft = 'FORSQL' then lang = 'FORTRAN'
'EXEC SQLPREP' lang 'PP(PREP='fn','foruser,
                 'NOCHECK,ISOLATION(CS))',
                 'SYSIN('fn ft fm')'
ret=rc
rc='CPOP'('PRT')
HALT:
If buf^=0 Then 'DROPBUF' buf
Exit ret
\end{XMP}
\Filename{H2Fatmeninstallation-Tailoring-FATMEN-shell}
\section{Tailoring the FATMEN shell}
\index{KUIP}
\index{shell}
\index{Tailoring the FATMEN shell}
\index{Modifying the FATMEN shell}
\par
The FATMEN shell may be tailored by
\begin{OL}
\item
The use of KUIP macros
\item
Modifying the FATMEN CDF file, e.g. adding extra commands
\end{OL}
\subsection{KUIP macros}
\par
KUIP macros consist of files containing FATMEN or KUIP system commands.
A trivial example might contain commands such as
\begin{XMPt}{Example macro CDOPAL KUMAC}
MESS 'Initialise FATMEN for OPAL'
INIT OPAL
MESS 'Change directory to PROD/PASS3/FILT/P1R1039L044'
CD PROD/PASS3/FILT/P1R1039L044
MESS 'Count the number of files in this directory'
FC
\end{XMPt}

\par
The output of the above macro is given below:
\begin{XMPt}{Result of executing CDOPAL KUMAC}
FAT3@cernvm;
fm
Type INIT to initialise FATMEN>
exec cdopal
 
FMINIT.  Initialisation of FATMEN package
FATMEN   1.05  900312 13.00  CERN PROGRAM LIBRARY FATMEN=Q123
         This version created on      900312  at        1600
Linked to FMOPAL               mode Z
FAOPEN : for FARZ on Unit    1 opened File CERN FATRZ
Current Working Directory = //CERN/OPAL
Current Working Directory = //CERN/OPAL/PROD/PASS3/FILT/P1R1039L044
Files:    5
FM>
\end{XMPt}

Further details can be found in the KUIP Long writeup<BIBREF REFID=KUIP>.
\subsection{Adding commands to the FATMEN shell}
\par
Extra commands maybe added to the FATMEN shell by
\begin{OL}
\item
Creating a Command Definition File (CDF)
\item
Merging this with the FATMEN CDF file (Patch FMCDF on the FATMEN PAM)
\item
Processing the resultant CDF file using the command KUIPC
\item
Compiling the FORTRAN file produced by the KUIPC command
\item
Linking with the FATMEN library
\end{OL}
\par
For example, we could add the command XLS which would
\begin{OL}
\item
Accept a generic name which could contain wild cards at any point
\item
Find all matching files
\item
Display the file entries
\end{OL}
(In fact, the ls command has been modified to do just this!)
\begin{XMPt}{CDF file for XLS command}
>COMMAND XLS
>GUIDENCE
Use the XLS command to perform an extended LS command.
Syntax: XLS path options
Options:
  A - list all attributes, except DZSHOW (option Z).
  C - display comment field associated with file
  F - list file attributes, such as start/end record and block
  G - list the full generic name of each file
  K - list keys associated with this file (copy level, media type, location)
  L - list logical attributes, such as FATMEN file format
      (ZEBRA exchange etc.)
  M - list media attributes, such as VSN, VID, file sequence number for tape
      files, host type and operating system for disk files.
  N - lists dataset name on disk/tape of this file
  O - list owner, node and job of creator etc.
  P - list physical attributes, such as record format etc.
  S - lists security details of this file (protection)
  T - list date and time of creation, last access etc.
  U - list user words.
  Z - dump ZEBRA bank with DZSHOW.
>ACTION FMXLSC
>PARAMETERS
+
FILE 'File or pathname' C D='CURRENT_DIRECTORY'
OPTN 'Options'          C D=' '
\end{XMPt}
\par
Our action routine, FMXLSC, might look like the following:
\begin{XMPt}{Example action routine, FMXLSC}
      SUBROUTINE FMXLSC
      PARAMETER     (MAXFIL=10000)
      PARAMETER     (LKEYFA=10)
      DIMENSION     KEYS(LKEYFA,MAXFIL)
      CHARACTER*255 FILES(MAXFIL)
      CHARACTER*255 PATH
      CHARACTER*25  CHOPT
*
      CALL KUGETC(PATH,LPATH)
      CALL KUGETC(CHOPT,LCHOPT)
*
      CALL FMLIST(PATH(1:LPATH),FILES,KEYS,NFOUND,MAXFIL,IRC)
*
      DO 10 I=1,NFOUND
10    CALL FMSHOW(FILES(I),LBANK,CHOPT(1:LCHOPT),IRC)
*
      END
\end{XMPt}

\Filename{H1Fatmen-Monitoring-information}
\chapter{Monitoring information}

\Filename{H2Fatmen-Monitoring-introduction}
\section{Introduction}

{\tt FATMEN} maintains three types of monitoring information:

\begin{OL}
\item
Information recorded in the {\tt FATMEN} catalogue
\item
Information logged per file access
\item
Information logged per session
\end{OL}

\subsection{Monitoring information in the {\tt FATMEN} catalogue}

The monitoring information that is stored in the {\tt FATMEN} catalogued
consists of the {\tt file size}, the {\tt date} and {\tt time} of
last access and the {\tt number of file accesses}.

This information is stored per entry at the offsets {\tt MFSZFA},
{\tt MLATFA} and {\tt MUSCFA}. See page~\pageref{BANK-OFFSETS}
for a description of the bank offsets.

This information can be histogrammed with the example program
{\tt FATPLOT}, included in the {\tt FATMEN} source file
in patch {\tt EXAMPLE}, deck {\tt FATPLOT} and reproduced below.

\begin{XMPt}{Histogramming monitoring information}
*=======================================================================
*
*  Example of using HBOOK to plot various FATMEN catalogue values.
*  This program histograms the file size, number of days since last
*  access, medium type etc.
*=======================================================================
      PARAMETER (LURCOR=200000)
      COMMON/CRZT/IXSTOR,IXDIV,IFENCE(2),LEV,LEVIN,BLVECT(LURCOR)
      DIMENSION    LQ(999),IQ(999),Q(999)
      EQUIVALENCE (IQ(1),Q(1),LQ(9)),(LQ(1),LEV)
*
      COMMON /USRLNK/LUSRK1,LUSRBK,LUSRLS
*
      COMMON /QUEST/IQUEST(100)
      CHARACTER*8   THRONG
+CDE,FATPARA.
+CDE,FATBUG.
      EXTERNAL      UROUT
*
*
*     Initialise ZEBRA
*
      CALL MZEBRA(-3)
      CALL MZSTOR(IXSTOR,'/CRZT/','Q',IFENCE,LEV,BLVECT(1),BLVECT(1),
     +            BLVECT(5000),BLVECT(LURCOR))
      CALL MZLOGL(IXSTOR,-3)
 
*
* *** Define user division and link area like:
*
      CALL MZDIV  (IXSTOR, IXDIV, 'USERS', 50000, LURCOR, 'L')
      CALL MZLINK (IXSTOR, '/USRLNK/', LUSRK1, LUSRLS, LUSRK1)
*
*     Units for FATMEN RZ/FZ files
*
      LUNRZ = 1
      LUNFZ = 2
      CALL GETENVF('THRONG',THRONG)
      LTH = LENOCC(THRONG)
*
*     Initialise FATMEN
*
      CALL FMINIT(IXSTOR,LUNRZ,LUNFZ,'//CERN/'//THRONG(1:LTH),IRC)
      CALL FMLOGL(0)
*
*     Initialise HBOOK
*
      CALL HLIMIT(-20000)
*
*     Book histograms
*
      CALL HBOOK1(1,'File Size (MB)',50,0.,200.,0.)
      CALL HBOOK1(2,'Number of accesses',50,0.,50.,0.)
      CALL HBOOK1(3,'Number days since last access',50,0.,300.,0.)
      CALL HBOOK1(4,'Number days since catalogued',50,0.,300.,0.)
      CALL HBOOK1(5,'Number days since created',50,0.,300.,0.)
      CALL HBOOK1(6,'Medium',5,0.,5.,0.)
      CALL HIDOPT(0,'BLAC')
*
*     Loop over all files
*
      CALL FMLOOP('//CERN/*/*',-1,UROUT,IRC)
*
*     Print and store the histograms
*
      CALL HPRINT(0)
      CALL HRPUT(0,'FATTUPLE.'//THRONG(1:LTH),'N')
*
*     Terminate cleanly
*
      CALL FMEND(IRC)
      END
 
      SUBROUTINE UROUT(PATH,KEYS,IRC)
+CDE,FATPARA.
      PARAMETER (LURCOR=200000)
      COMMON/CRZT/IXSTOR,IXDIV,IFENCE(2),LEV,LEVIN,BLVECT(LURCOR)
      DIMENSION    LQ(999),IQ(999),Q(999)
      EQUIVALENCE (IQ(1),Q(1),LQ(9)),(LQ(1),LEV)
      CHARACTER*(*) PATH
      PARAMETER     (LKEYFA=10)
      DIMENSION     KEYS(LKEYFA)
      DIMENSION     NDAYS(3)
      COMMON/QUEST/IQUEST(100)
      IRC   = 0
      LBANK = 0
      LP    = LENOCC(PATH)
      CALL FMGETK(PATH(1:LP),LBANK,KEYS,IRC)
*
*     Fill histograms
*
      IF(IQ(LBANK+MFSZFA).NE.0)
     +CALL HFILL(1,FLOAT(IQ(LBANK+MFSZFA)),0.,1.)
      IF(IQ(LBANK+MUSCFA).NE.0)
     +CALL HFILL(2,FLOAT(IQ(LBANK+MUSCFA)),0.,1.)
      CALL FMDAYS(PATH(1:LP),LBANK,KEYS,NDAYS,' ',IRC)
      CALL HFILL(3,FLOAT(NDAYS(3)),0.,1.)
      CALL HFILL(4,FLOAT(NDAYS(2)),0.,1.)
      CALL HFILL(5,FLOAT(NDAYS(1)),0.,1.)
      CALL HFILL(6,FLOAT(IQ(LBANK+MMTPFA)),0.,1.)
      CALL MZDROP(IXSTOR,LBANK,' ')
      END
\end{XMPt}

\Filename{H2Fatmen-Monitoring-fileaccess}
\section{Monitoring information logged per file access}

For each call to \Rind{FMOPEN}, an update is sent to
the {\tt FATMEN} server to update the last access date
and time, use count etc. as described above.
This logging record also records additional information,
which is stored in the logs of the servers.

This information consists of the following quantities:

\begin{DLtt}{1234567890}
\item[CHFNFA]The fully qualified name of the actual file that
was accessed. 
\item[IHOWFA]A bit pattern indicating how the file was accessed.
\begin{DLtt}{1234567890}
\item[JLOCFA=1]Local disk (standard file system)
\item[JSFSFA=2]VM shared file system
\item[JMSCFA=3]MSCP (VAXcluster)
\item[JAFSFA=4]Andrew file system
\item[JOSFFA=5]OSF distributed file system
\item[JDFSFA=6]DEC DFS
\item[JNFSFA=7]Sun NFS
\item[JDECFA=8]DECnet
\item[JCSPFA=9]CSPACK server
\item[JFPKFA=10]FPACK server
\item[JRFIFA=11]RFIO
\item[JSTGFA=31]Stage required
\item[JTPMFA=32]TPMNT (=not staged)
\end{DLtt}
\item[ITIMFA]The time in seconds spent in the routine 
\Rind{FMOPEN} (including the time for {\tt STAGE} operations etc.)
\end{DLtt}

In addition, the username and nodename from which the file
was accessed is logged, together with the generic name
and {\tt FATMEN} keys (location code, media type and data representation).
These permit further information to be extracted from the {\tt FATMEN}
catalogue if required.

\Filename{H2Fatmen-Monitoring-session-logging}
\section{Session logging}

At the end of each {\tt FATMEN} session, {\bf provided} that a
call to \Rind{FMEND} is made, a log record is written to
the server. The server ignores this record (apart from
counting the number that it receives, unless installed
with the flag {\tt +USE,FATLOG.}

If installed with this flag, the logging records are
collected into a daily summary for subsequent processing.

The logging records consists of the following blocks:

\begin{OL}
\item
Hollerith block
\begin{DLtt}{1234567}
\item[KFMSYS]FATMEN system (e.g. CERN)
\item[KFMGRP]FATMEN group (e.g. DELPHI)
\item[KFMTIT]Title of FATMEN source file
\item[KFMUSR]User name
\item[KFMHST]Host name
\item[KFMTYP]Host type
\item[KFMOS]Host operating system
\end{DLtt}
\item
MB counts
\begin{DLtt}{1234567}
\item[KFMMBR]Number of MB read
\item[KFMMBW]Number of MB written
\item[KFZMBR]Number of MB read with ZEBRA FZ
\item[KFZMBW]Number of MB written with ZEBRA FZ
\item[KFMMBC]Number of MB copied
\item[KFMMBN]Number of MB copied over the network
\item[KFMMBQ]Number of MB queued for copy (e.g. {\tt CHEOPS})
\end{DLtt}
\item
Dates and times (packed with \Rind{FMPKTM})
\begin{DLtt}{1234567}
\item[KFMIDQ]Date and time of PATCHY installation job
\item[KFMIDS]Date and time of start of FATMEN session
\item[KFMIDE]Date and time of end of FATMEN session
\end{DLtt}
\item
Catalogue modifications
\begin{DLtt}{1234567}
\item[KFMADD]Number of disk entries added (\Rind{FMADDD})
\item[KFMADL]Number of links added (\Rind{FMADDL})
\item[KFMADT]Number of tape entries added (\Rind{FMADDT})
\item[KFMMDR]Number of \Rind{MKDIR} commands
\item[KFMRDR]Number of \Rind{RMDIR} commands
\item[KFMRLN]Number of \Rind{RMLN} commands
\item[KFMRTR]Number of \Rind{RMTREE} commands
\item[KFMRMF]Number of \Rind{RM} commands
\item[KFMCPF]Number of \Rind{CP} commands
\item[KFMMVF]Number of \Rind{MV} commands
\item[KFMMOD]Number of \Rind{MODIFY} commands
\item[KFMTCH]Number of \Rind{TOUCH} commands
\end{DLtt}
\item
File accesses
\begin{DLtt}{1234567}
\item[KFMOPN]Number of files opened, e.g. by \Rind{FMOPEN}
\item[KFMCLS]Number of files closed, e.g. by \Rind{FMCLOS}
\item[KFMCPY]Number of files copied, e.g. by \Rind{FMCOPY}
\item[KFMCPQ]Number of files queued for copy, e.g. by \Rind{FMCOPQ}
\item[KFMCPN]Number of files copied over the network, e.g. by \Rind{FMRCOP}
\end{DLtt}
\item
SYSREQ and TMS operations
\begin{DLtt}{1234567}
\item[KFMSRQ]Number of calls to SYSREQ (\Rind{FMSREQ})
\item[KFMQVL]Number of {\tt QVOL} commands (\Rind{FMQVOL})
\item[KFMAVL]Number of volumes allocated (\Rind{FMALLO})
\item[KFMASP]Number of space allocations (\Rind{FMGVOL} or \Rind{FMGVID})
\item[KFMPOL]Number of {\tt TRANSFER} commands (pool operations, e.g. \Rind{FMPOOL}
\item[KFMLCK]Number of volumes locked, e.g. \Rind{FMLOCK}
\item[KFMULK]Number of volumes unlocked, e.g. \Rind{FMULOK}
\item[KFMDTG]Number of tags deleted, e.g. \Rind{FMTAGS}
\item[KFMGTG]Number of tags obtained, e.g. \Rind{FMTAGS}
\item[KFMSTG]Number of tags set, e.g. \Rind{FMTAGS}
\end{DLtt}
\item
Catalogue processing
\begin{DLtt}{1234567}
\item[KFMBNK]Number of banks read from the catalogue (\Rind{FMRZIN})
\item[KFMGET]Number of banks read with default selection (\Rind{FMGET})
\item[KFMGTK]Number of banks user selected (\Rind{FMGETK})
\item[KFMSHW]Number of calls to \Rind{FMSHOW}
\item[KFMSCN]Number of calls to \Rind{FMSCAN}
\item[KFMLOP]Number of calls to \Rind{FMLOOP}
\item[KFMLDR]Number of calls to \Rind{FMLDIR}
\item[KFMLFL]Number of calls to \Rind{FMLFIL}
\item[KFMSRT]Number of calls to \Rind{FMSORT}
\item[KFMRNK]Number of calls to \Rind{FMRANK}
\item[KFMSLK]Number of calls to \Rind{FMSELK}
\item[KFMMTC]Number of calls to \Rind{FMATCH}
\end{DLtt}
\end{OL}

%  ==================== Appendixes ================================
\begin{appendix}
%%%%%%%%%%%%%%%%%%%%%%%%%%%%%%%%%%%%%%%%%%%%%%%%%%%%%%%%%%%%%%%%%%%
%                                                                 %
%   FATMEN User Guide and Reference manual                        %
%                                                                 %
%   Fatmen Part 5: Appendices                                     %
%                                                                 %
%   This document needs the following external EPS files:         %
%   none                                                          %
%                                                                 %
%   Editor: Michel Goossens / CN-AS                               %
%   Last Mod.: 19 January 1993 11:50 mg                           %
%                                                                 %
%%%%%%%%%%%%%%%%%%%%%%%%%%%%%%%%%%%%%%%%%%%%%%%%%%%%%%%%%%%%%%%%%%%
\Filename{H1fatmenappendix-fatcat-server}
\chapter{The fatcat server}
\Filename{H2fatmenappendix-overview-fatcat-files-and-directories}
\section{Overview of FATCAT files and directories}

At CERN, a dedicated system is used to host FATMEN servers and
the associated catalogues. This system is currently an IBM RS6000
with node name {\tt fatcat}. No user accounts exist on the machine -
the {\tt /fatmen} file system is {\tt exported} and {\tt NFS-mounted}
on all systems requiring access. At a later date, {\tt OSF-DFS} will
be used for catalogue access.

The {\tt /fatmen} file system contains a subdirectory for each
experiment, as shown below.

\begin{XMPt}{/fatmen directories}
[fatcat] (209) ls -F /fatmen
fmaleph/    fmcplear/   fmna31/     fmatlas/    fmna44/     fmna52/     fmcadd/     
fmdelphi/   fmnomad/    fmcharm2/   fmopal/     fmrd5/      fmchorus/   fmkeops/    
fmrd6/      fml3/       fmsmc/      fmcndiv/       
\end{XMPt}

Each of these directories contains the following files or directories:
\begin{DLtt}{123456789AB}
\item[cern.fatrz]The FATMEN catalogue for the experiment in question.
\item[done]A directory where the server places update files after processing them.
\item[fatserv]The script that runs the server for this experiment.
\item[fatserv.log]The log file from the server process.
\item[fatsrv]The server program (in fact a link to the server program).
\item[fml3.names]A names file {\tt fm}experiment.names, e.g. fml3.names as here.
\item[todo]A directory where updates are written by FATMEN clients (users) and which
is scanned regularly by the server process.
\item[tovm]Updates to be transferred to {\tt CERNVM} are placed here. This
directory is optional, and requires server software on the VM node.
\end{DLtt}

In addition, one may also have the following configuration files:

\begin{DLtt}{123456789012345}
\item[fatmen.loccodes]A list of FATMEN location codes (integers) and associated
text, e.g. 1=CERN Computer Centre.
\item[fatmen.acl]An ACL file determining which users on which nodes may update
the specified subdirectory structure.
\item[fatmen.updates]A file controlling for which users and paths monitoring
information is to be sent to the server.
\item[fatmen.accounts]A list of account aliases.
\end{DLtt}

Furthermore, one needs a subdirectory for each remote server.
In the particular case of {\tt fatcat} there are no remote servers,
put the directories are called {\tt to}node, e.g. {\tt tovxcrna}.
Provided that the names file is correctly configured, all
missing subdirectories are automatically created by the server,
with the exception of {\tt tovm} and {\tt todo}. {\tt todo} must
be given the permission {\tt o+w} to allow all users the possibility
to make catalogue updates. (With {\tt AFS} one could use ACLs to
provide better security).

\Filename{H2fatmenappendix-manage-servers}
\section{Managing the servers}

The FATMEN servers are started automatically at boot-time.
This is done by adding the following line to the file {\tt /etc/inittab}.

\begin{XMP}
fatmen:2:wait:/etc/rc.fatmen > /dev/console 2>&1 # Start Fatmen
\end{XMP}

The file {\tt rc.fatmen} is show below:

\begin{XMPt}{rc.fatmen}
#!/bin/sh
#
#               Start FATMEN servers
#
#
if [ -x /u/jamie/bin/fatstart ]
then
        echo Start FATMEN servers ...
        su - jamie /u/jamie/bin/fatstart 2>&1
fi
\end{XMPt}

Normally, no other operation is required. Should it be necessary to
stop the servers, the following script may be used:

\begin{XMPt}{fatstop}
#!/bin/ksh
#
#   Stop all FATMEN servers that are running and create a
#   file 'restart_fat' in the CWD that can be used to restart them.
#
stop=" "
run=" "
nolog=" "
noscr=" "
b="."
d=`date`
#
#   Ensure that variables are defined...
#
if [ -f restart_fat ]
   echo Remove old restart_fat file...
   then rm -i restart_fat
fi
 
for i in /fatmen/fm*
   do
 
typeset -u fatgrp
typeset -l fatman
fatpath=$i
fatgrp=`basename $i`
fatman=$fatgrp
eval $fatgrp=$fatpath;export $fatgrp
#
# and stop those servers that are running...
#
if [ -x ${i}/fatserv ]
   then
#
# does a log file exist?
#
   if [ -f /fatmen/${fatgrp}.log ]
      then
      echo Log file exists for ${fatgrp} - looking for existing process
      log=${log}${b}${fatgrp}
      pid=`cat /fatmen/${fatgrp}.log | awk '{printf "%s\\n",$13}'`
      if (test $pid)
         then
         echo Looking for server process for $fatgrp
         if(ps -ae  | grep -s $pid )
            then
            echo FATSRV running PID = $pid
            run=${run}${b}${fatgrp}
            echo rm /fatmen/${fatman}/todo/signal.stop >> restart_fat
            echo Server stopped at $d > /fatmen/${fatman}/todo/signal.stop
            else
            echo No existing server found for $fatgrp
            echo Removing old log file...
            rm   /fatmen/${fatgrp}.log
            if [ -f ${i}/todo/signal.stop ]
               then echo signal.stop file found!
               rm ${i}/todo/signal.stop
               echo '(removed)'
            fi
         fi
      fi
   fi
fi
 
done
 
echo
echo Log files found for $log | tr '.' ' '
echo Servers already running for $run | tr '.' ' '
echo fatstart >> restart_fat
if [ -f restart_fat_fat ]
   then chmod +x restart_fat_fat
   echo restart reservers by typing restart_fat
fi
\end{XMPt}

Should a server abend, it can be restarted (after curing the problem) by
rerunning the {\tt fatstart} script that is invoked at boottime.

\Filename{H2fatmenappendix-monitor-servers}
\section{Monitoring the servers}

A simple script may be used to look for active servers and their
CPU usage. The fact that each server runs its 'own' {\tt fatsrv}
module is used by this script, which reduces to three lines:

\begin{XMPt}{fatps}
echo 'FATMEN server                                       Elapsed     CPU time   %CPU'
echo '==============================================================================='
ps -aef -F "args,etime,time,pcpu" | grep "/fatsrv" | sort +2 -r
\end{XMPt}

One may also inspect the log files, e.g. {\tt tail -f fatserv.log}.

\Filename{H1fatmenappendix-catalogue-recovery}
\chapter{Catalogue recovery}

On rare occasions a FATMEN catalogue may become corrupted. 
To recover, one can simply restore the most recent backup
copy and reapply all journal files since the time of the
backup. (Simply copy or move them from the {\tt done}
directory to the {\tt todo} directory. The time stamp
in the filenames will ensure that they are processed
in the correct order).

Alternatively, one can attempt to 'repair' a catalogue.
This may be necessary if the corruption is not noticed
for some time. This can happen if a systematic check of
the catalogue is not made before every backup as corruption
typically affects only a few entries which may well not
be accessed for some time.

Repairing a corrupted catalogue consists of the following
steps:
\begin{OL}
\item
Identifying which directories and/or entries are corrupted
\item
Copying the catalogue skipping these directories and/or entries
\item
Recovering the corrupted information from journal files or
from a backup copy of the catalogue
\end{OL}

\Filename{H2fatmenappendix-find-corrupted-entries}
\section{Finding the corrupted entries}

Normally, the corrupted entries are either reported
by a user (e.g. I cannot access files in this directory),
or are found by the server. Should the server abend,
it will automatically send a mail message to the FATMEN
manager. (Tailor the {\tt fatabend} script as appropriate).
The server log will then show which directory and/or entries
were giving problems. Should this information be unavailable,
one can find the corrupted entries by running the 
program {\tt fatloop2}, as shown below (for a Unix system).

\begin{XMPt}{Running FATLOOP2}

export FMLOGL=1
export FATSYS=CERN
export FATGRP=L3
fatloop2 > fatloop2.log

\end{XMPt}

This program attempts to retrieve each catalogue entry in
turn. Should a directory or catalogue entry be corrupted,
then it will terminate abnormally (via ZFATAL). Thus
it is good practice to run this program on a regular basis,
e.g. before a periodic backup.

An example of the log is shown below.

\begin{XMPt}{Output of the FATLOOP2 program}

 FATMEN system defaulted to CERN
 FATMEN group:  L3

 FMINIT.  Initialisation of FATMEN package
 FATMEN   1.81/07 930203 08:50 CERN PROGRAM LIBRARY FATMEN=Q123
          This version created on      930203  at         852
 FMLOGL. setting log level to           1
 Get: //CERN/L3/CDREMM/CC132563
 Get: //CERN/L3/CDREMM/CC132563
 Get: //CERN/L3/CDREMM/CC132563
 Get: //CERN/L3/CDREMM/CC132564

 ...

 Get: //CERN/L3/PROD/DATA/SDRETT/CC02H8G2
 Get: //CERN/L3/PROD/DATA/SDRETT/CC02H8IU
 Get: //CERN/L3/PROD/DATA/SDRETT/CC02HBJ6
 Get: //CERN/L3/PROD/DATA/SDRETT/CC02HBGE
 Get: //CERN/L3/PROD/DATA/SDRETT/CC02HBJ6
 Get: //CERN/L3/PROD/DATA/SDRETT/CC02HBGE
 Get: //CERN/L3/PROD/DATA/SDRETT/CC02HBLY

 !!!!! ZFATAL called from MZGAR1
              called from FZIMTB

 !!!!! ZFATAL reached from MZGAR1    for Case=  1

          IQUEST(11) = *********         DFE0035F   ��^C_

          Current Store number =  0  (JQDIVI=19)
1ZEBRA SYSTEM Post-Mortem from ZPOSTM.

 /QUEST/
              0            1         3835           47            0            1         1030           10
     1297762113   1378951200   -538967201            0           19    981191424        12309       930127
           1428          180            0         1019          644          645          646          647
            648          649          650          651          652          653          654          655
            656          657         3742         3836          203          204          205          206
       00000000     00000001     00000EFB     0000002F     00000000     00000001     00000406     0000000A
       4D5A4741     52312020     DFE0035F     00000000     00000013     3A7BCB00     00003015     000E314F
       00000594     000000B4     00000000     000003FB     00000284     00000285     00000286     00000287
       00000288     00000289     0000028A     0000028B     0000028C     0000028D     0000028E     0000028F
       00000290     00000291     00000E9E     00000EFC     000000CB     000000CC     000000CD     000000CE

\end{XMPt}

\Filename{H2fatmenappendix-recover-corrupted-entries}
\section{Recovering from corrupted entries}

Once the bad directories or entries have been identified,
one can make a new catalogue using a special version of
the {\tt RTOX/RFRX} tools. Here we simply exclude directories
and/or entries that have been identified as bad.

In the above example, we clearly see that there are problems 
in the directory {\tt //CERN/L3/PROD/DATA/SDRETT}
We can now use the FATMEN shell to localise the corrupted entries.

First, we use the {\tt ls} command to list all files in
this directory.

\begin{XMPt}{Using the ls command to locate corrupted entries}

ls -w output=sdrett.log

\end{XMPt}

The server log has told use that the first bad entry is 
{\tt CC02HBLY}. As the catalogue entries take 163 words
and the record length is 1024 words, we would expect 
7 corrupted entries if an entire record was corrupt.
That means that we should be able to retrieve the entry
{\tt CC02HJ0M} correctly. This we can test using the
{\tt ls} command, e.g.

\begin{XMP}

ls CC02HJ0M -a

\end{XMP}

However, if we attempt to retrieve the previous entry, we
get the following:

\begin{XMP}
FM> ls CC02HJ66 -a
 
 Directory: //CERN/L3/PROD/DATA/SDRETT
 
 FMRZIN. Error in RZIN for directory //CERN/L3/PROD/DATA/SDRETT
 Object:
 Key serial number =   1670 filename = CC02HJ66             data repr. =   0 media type =  2 location code =      1
 
 Generic filename: CC02HJ66
 Data repr.:  0 Media type: 2 Location code:      1 File serial number:   1670
 Device group: CT1     
 Comment: 
 Start record:          0 End record:          0
 Start block :          0 End block :          0
 File format:  user format: 
 VSN:  VID:  FSEQ:    0
 Fileid:      
 Created by:   ACCT:  on node:  by job: 
 RECFM:   LRECL:     0 BLKSIZE:     0 Filesize:     0 Use count:     0
 File protection mask:          00000000
 Date and time of creation:          0    0
 Date and time catalogued:           0    0
 Date and time last accessed:        0    0
 User words:  00000000 00000000 00000000 00000000 00000000
              00000000 00000000 00000000 00000000 00000000

\end{XMP}

We can extract all of the good entries in this directory
by using the {\tt extract} or {\tt touch} commands.

\begin{XMPt}{Extract good entries from a corrupted directory}

FM>ls output=sdrett.log
FM>edit sdrett.log

(delete the lines corresponding to bad entries)
(add the command 'touch' to the beginning of every line)
(save as sdrett.kumac)

FM>update 9999 9999 0
FM>exec sdrett

\end{XMPt}

\Filename{H2fatmenappendix-skipping-bad-directories}
\section{Skipping bad directories}

As described above, we can skip bad directories by modifying
the program {\tt RTOX}.

\begin{XMPt}{Extracting the source for RTOX}
ypatchy /cern/new/src/car/zebra.car r2x.f :go <<!
+use,qcde.
+use,qmuix,ibmrt,*rtox.
+use,p=rz,d=rzcdes.
+use,p=rz,d=rztofz.
+use,p=rz,d=rztof1.
+use,p=rztofrfz,d=blankdek.
+exe.
+pam,11,t=c.
+quit
!
\end{XMPt}

We then modify the routine {\tt RZTOFZ} to exclude the directories
that contain corrupted entries.

\begin{XMPt}{Modifying RTOX}
CDECK  ID>, RZTOFZ. 
      SUBROUTINE RZTOFZ(LUNFZ,CHOPT)
*
************************************************************************
*
*        Copy the CWD tree to a sequential FZ file
*        The FZ file must have been declared with FZOPEN
* Input:
*   LUNFZ   Logical unit number of the FZ sequential access file
*   CHOPT   default save only the highest cycle to LUNFZ
*           'C' save all cycles
*
* Called by <USER>
*
*  Author  : R.Brun DD/US/PD
*  Written : 14.05.86
*  Last mod: 26.06.92 JDS - protect against RZPAFF problems
*
************************************************************************
      PARAMETER      (IQDROP=25, IQMARK=26, IQCRIT=27, IQSYSX=28)
      COMMON /QUEST/ IQUEST(100)
      COMMON /ZVFAUT/IQVID(2),IQVSTA,IQVLOG,IQVTHR(2),IQVREM(2,6)
      COMMON /ZEBQ/  IQFENC(4), LQ(100)
                              DIMENSION    IQ(92),        Q(92)
                              EQUIVALENCE (IQ(1),LQ(9)), (Q(1),IQ(1))
      COMMON /MZCA/  NQSTOR,NQOFFT(16),NQOFFS(16),NQALLO(16), NQIAM
     +,              LQATAB,LQASTO,LQBTIS, LQWKTB,NQWKTB,LQWKFZ
     +,              MQKEYS(3),NQINIT,NQTSYS,NQM99,NQPERM,NQFATA,NQCASE
     +,              NQTRAC,MQTRAC(48)
                                       EQUIVALENCE (KQSP,NQOFFS(1))
      COMMON /MZCB/  JQSTOR,KQT,KQS,  JQDIVI,JQDIVR
     +,              JQKIND,JQMODE,JQDIVN,JQSHAR,JQSHR1,JQSHR2,NQRESV
     +,              LQSTOR,NQFEND,NQSTRU,NQREF,NQLINK,NQMINR,LQ2END
     +,              JQDVLL,JQDVSY,NQLOGL,NQSNAM(6)
                                       DIMENSION    IQCUR(16)
                                       EQUIVALENCE (IQCUR(1),LQSTOR)
      COMMON /MZCC/  LQPSTO,NQPFEN,NQPSTR,NQPREF,NQPLK,NQPMIN,LQP2E
     +,              JQPDVL,JQPDVS,NQPLOG,NQPNAM(6)
     +,              LQSYSS(10), LQSYSR(10), IQTDUM(22)
     +,              LQSTA(21), LQEND(20), NQDMAX(20),IQMODE(20)
     +,              IQKIND(20),IQRCU(20), IQRTO(20), IQRNO(20)
     +,              NQDINI(20),NQDWIP(20),NQDGAU(20),NQDGAF(20)
     +,              NQDPSH(20),NQDRED(20),NQDSIZ(20)
     +,              IQDN1(20), IQDN2(20),      KQFT, LQFSTA(21)
                                       DIMENSION    IQTABV(16)
                                       EQUIVALENCE (IQTABV(1),LQPSTO)
C
      COMMON /RZCL/  LTOP,LRZ0,LCDIR,LRIN,LROUT,LFREE,LUSED,LPURG
     +,              LTEMP,LCORD,LFROM
                   EQUIVALENCE (LQRS,LQSYSS(7))
C
      PARAMETER (NLPATM=100)
      COMMON /RZDIRN/NLCDIR,NLNDIR,NLPAT
      COMMON /RZDIRC/CHCDIR(NLPATM),CHNDIR(NLPATM),CHPAT(NLPATM)
      CHARACTER*16   CHNDIR,    CHCDIR,    CHPAT
C
      COMMON /RZCH/  CHWOLD,CHL
      CHARACTER*255  CHWOLD,CHL
C
      PARAMETER (KUP=5,KPW1=7,KNCH=9,KDATEC=10,KDATEM=11,KQUOTA=12,
     +           KRUSED=13,KWUSED=14,KMEGA=15,KIRIN=17,KIROUT=18,
     +           KRLOUT=19,KIP1=20,KNFREE=22,KNSD=23,KLD=24,KLB=25,
     +           KLS=26,KLK=27,KLF=28,KLC=29,KLE=30,KNKEYS=31,
     +           KNWKEY=32,KKDES=33,KNSIZE=253,KEX=6,KNMAX=100)
C
      CHARACTER*(*) CHOPT
      DIMENSION ISD(NLPATM),NSD(NLPATM),IHDIR(4)
*
*-----------------------------------------------------------------------
*
      IQUEST(1)=0
      IQ1=0
      IF(LQRS.EQ.0)GO TO 99
*
      CALL UOPTC(CHOPT,'C',IOPTC)
      NLPAT0=NLPAT
      DO 5 I=1,NLPAT0
         CHPAT(I)=CHCDIR(I)
   5  CONTINUE
      ITIME=0
      CALL RZCDIR(CHWOLD,'R')
*
*        Garbage collection in user short range divisions
*        in primary store
*
      CALL MZGARB(21,0)
*
*            Write general header
*
      IHDIR(1)=12345
      IHDIR(2)=NLPAT0
      CALL FZOUT(LUNFZ,JQPDVS,0,1,'Z',1,2,IHDIR)
      IF(IQUEST(1).NE.0)THEN
         IQ1=IQUEST(1)
         GO TO 90
      ENDIF
*
*            Set CWD to the current level
*
  10  CONTINUE
      IF(ITIME.NE.0)THEN
         CALL RZPAFF(CHPAT,NLPAT,CHL)
      IF(IQUEST(1).NE.0)THEN
         IQ1=IQUEST(1)
         NLPAT=NLPAT-1
         GO TO 20
      ENDIF
         CALL RZCDIR(CHL,' ')
      ENDIF
      ISD(NLPAT)=0
      NSD(NLPAT)=IQ(KQSP+LCDIR+KNSD)
*
*     Skip bad directories
*
      if(chl(1:lchl).eq.'//RZ/L3/PROD/DATA/SDRETT') goto 20
*
*            Write current directory
*
      CALL RZTOF1(LUNFZ,IOPTC)
      IF(IQUEST(1).NE.0)THEN
         IQ1=IQUEST(1)
         NLPAT=NLPAT-1
         GO TO 20
      ENDIF
*
*            Process possible down directories
*
  20  ISD(NLPAT)=ISD(NLPAT)+1
      IF(ISD(NLPAT).LE.NSD(NLPAT))THEN
         NLPAT=NLPAT+1
         LS=IQ(KQSP+LCDIR+KLS)
         IH=LS+7*(ISD(NLPAT-1)-1)
         CALL ZITOH(IQ(KQSP+LCDIR+IH),IHDIR,4)
         CALL UHTOC(IHDIR,4,CHPAT(NLPAT),16)
         ITIME=ITIME+1
         GO TO 10
      ELSE
         NLPAT=NLPAT-1
         IF(NLPAT.GE.NLPAT0)THEN
            LUP=LQ(KQSP+LCDIR+1)
            CALL MZDROP(JQPDVS,LCDIR,' ')
            LCDIR=LUP
            GO TO 20
         ENDIF
      ENDIF
*
*             Write final trailer
*
      NLPAT=NLPAT0
      CALL FZOUT(LUNFZ,JQPDVS,0,1,'Z',1,1,99)
      IF(IQUEST(1).NE.0)THEN
         IQ1=IQUEST(1)
         GO TO 90
      ENDIF
      LCORD=LQ(KQSP+LTOP-4)
      IF(LCORD.NE.0)THEN
         CALL MZDROP(JQPDVS,LCORD,'L')
         LCORD=0
      ENDIF
*
*            Reset CWD
*
  90  CONTINUE
      CALL RZCDIR(CHWOLD,' ')
      IF(IQ1.NE.0.AND.IQUEST(1).EQ.0)IQUEST(1)=1
*
  99  RETURN
      END
\end{XMPt}

We now perform the following steps:

\begin{OL}
\item
Convert the catalogue to sequential (FZ) format,
skipping bad directories in the process
\item
Convert the sequential file back to RZ format.
If this is done on VM, the program {\tt FATFROMX}
should be used in preference to {\tt RFRX}
\item
Extract all good entries in the bad directories, as
shown above
\item
Move this file to the input directory ({\tt /todo}) of
the server
\end{OL}

Now we have almost completed the catalogue recovery,
with the exception of the corrupted entries.

\index{FATFROMX}
\index{preformatting}
When creating an RZ file on a VM system, it is important
that
\begin{itemize}
\item
The RZ file is mode 6 (update in place)
\item
Records are preformatted so that the file is not extended
but updated in place
\end{itemize}

This is done using the program {\tt FATFROMX}.

The {\tt PARAMETER} NPRE should be set as follows:

\begin{XMPt}{Obtaining an appropriate value for {\tt NPRE}}
/* Calculate NPRE. This assumes a default blocksize of 4096
   bytes, both for the mini-disk and the FATMEN RZ file     */

npre = "QDISK"("191","BLKTOT") * .85

\end{XMPt}




\Filename{H2fatmenappendix-restoring-corrupted-entries}
\section{Restoring the corrupted entries}

In the example above we are now searching for only 7
catalogue entries. Here we have the following possibilities:
\begin{OL}
\item
Is there a copy of the catalogue on another node in which
these entries exist?
\item
Do we have a backup copy, e.g. on tape, in which these entries 
exist?
\item
Can we find the journal files corresponding to these entries?
\end{OL}

We can scan journal files for the entries we need as follows.

\begin{XMPt}{Printing the headers of the journal files}

for i in $FML3/done
   do
      fathead $i >> l3.journal
   done

\end{XMPt}

We then simply {\tt grep} the file {\tt l3.journal} for the
entries in question and then moved the corresponding files
to the {\tt /todo} directory.

\Filename{H1fatmenappendix-cheops-interface}
\chapter{CHEOPS interface}

If FATMEN has been installed with the \Lit{CHEOPS} option,
users may request that files are copied between sites via
the CHEOPS system. Requests will only be accepted at sites
which also have the correct server configuration.

On VM/CMS systems, this requires a special service machine
with username \Lit{FMCHEOPS}. This service machine receives
copy requests from users on the local VM/CMS system and forwards
them to the local CHEOPS server.

On Unix and VAX/VMS systems, an environmental variable/symbol
\Lit{FMCHEOPS} must be defined. This points to a directory
into which the copy requests will be automatically written
by the FATMEN software.

The following steps are involved:
\begin{OL}
\item
When a copy request is made, e.g. using the shell command \Rind{COPY}
with the \Lit{TRANSPORT=CHEOPS} parameter or the \Rind{FMCOPY}
subroutine with the \Lit{K} option, the FATMEN catalogue is updated
with an entry corresponding to the destination file. The comment
field contains the text
\begin{XMP}
Copy request queue to CHEOPS on YYMMDD at HHMM
\end{XMP}
\item
Once the request has been processed by the CHEOPS server,
the comment is changed to
\begin{XMP}
copy successfully queued to CHEOPS.
\end{XMP}
\item
If there are errors processing the request, the comment
becomes
\begin{XMP}
ERROR: <cheops error message>
\end{XMP}
\item
Finally, once the transfer has been performed, the original
comment as specified by the user is restored.
\item
If the actual transfer failed, the reason for the failure
is instead written into the comment string of the corresponding
FATMEN entry.
\end{OL}

\Filename{H2fatmenappendix-build-fatmen-cheops-interface-on-unix}
\section{Building the FATMEN/CHEOPS interface on a Unix system}
The following script may be used to build the FATMEN/CHEOPS
interface on a Unix system.
\begin{XMPt}{Script to build the FATMEN/CHEOPS interface}
ypatchy /cern/pro/src/car/zebra.car cheops2f.f :go <<!
+use,qcde.
+use,ibmrt,*cheops2f.
+exe.
+pam,11,r=qcde,t=c.
+pam,12,t=c,a. fatmen.cards
+quit
!
xlf -q extname cheops2f.f -L/cern/pro/lib -lpacklib -o cheops2f
\end{XMPt}
This server may be run with the following script.
\begin{XMPt}{Running the server}
#!/bin/sh
#
# script to run the FATMEN/CHEOPS interface.
#
t=`date`
h=`hostname`
echo
echo FATMEN/CHEOPS server starting at $t on $h
echo
FMWAKEUP=30;   export FMWAKEUP
FMLOGL=3;      export FMLOGL
FMCHEOPS=/fatmen/fmcheops; export FMCHEOPS
echo Wakeup interval is $FMWAKEUP seconds
echo
$HOME/fatmen/cheops2f
\end{XMPt}
\Filename{H2fatmenappendix-build-FMCHEOPS-server-on-vmcms}
\section{Building the FMCHEOPS server on a VM/CMS system}
The FMCHEOPS server can be built using the following EXEC.
\begin{XMPt}{FMCHEOPS EXEC}
/*       BUILD THE FMCHEOPS SERVER     */                                      
'EXEC YPATCHY PAM1="ZEBRA CAR *" PAM2="FATMEN CARDS" ',                        
    'CRADLE="FATKEOPS CRADLE" ASM="FATKEOPS FORTRAN A1"'                       
'EXEC VFORT FATKEOPS'                                                          
'GIME TCPVMA'                                                                  
'CERNLIB PACKLIB!NEW COMMTXT IBMLIB CMSLIB EDCBASE ( LINK'                     
'GLOBAL LOADLIB VSF2LOAD EDCLINK'                                              
'LOAD FATKEOPS'                                                                
'GENMOD FATKEOPS'
\end{XMPt}
The server should be set up as for other FATMEN servers,
i.e. it should have a \Lit{NAMES} file defining who may
issue privileged commands, as follows:
\begin{XMPt}{Example of a NAMES file for the FMCHEOPS server}
:nick.FATOPERATORS
               :list.fatop1 fatop2
                                 
:nick.FATOP1                                                                   
               :userid.console
               :node.cernvm 
                           
:nick.FATOP2              
               :userid.opsutil
               :node.cernvm 
                                                                               
:nick.FATOWN1                                                                  
               :userid.fatone
               :node.cernvm 
                                                                               
:nick.FATOWN2              
               :userid.hrrcr
               :node.cernvm
                          
:nick.FATOWN3            
               :userid.jamie
               :node.cernvm
                          
:nick.FATOWNERS          
               :list.fatown1 fatown2 fatown3
                                           
\end{XMPt}
The server may be run using the following EXEC.
\begin{XMPt}{Running the FMCHEOPS server}
/*                      F A T _ S T A R T                     */               
Address Command                                                                
   If QCONSOLE("DISCO") then nop                                               
                        else do                                                
                        Say "+++ Type #CP DISC to disconnect +++"              
                        Say "+++ Type #CP DISC to disconnect +++"              
                        Say "+++ Type #CP DISC to disconnect +++"              
                        end                                                    
'IDENTIFY (STACK'                                                              
Parse pull . . locnode .                                                       
                                                                               
If locnode = 'CERNVM' then do                                                  
                                                                               
/* Start FATCAT FORTRAN Program, which calls FATSERV EXEC */                   
   'GLOBALV SELECT *EXEC GET PWD'                                              
   'GLOBALV SELECT *EXEC GET USER'                                             
   'FATKEOPS'                                                                  
   end                                                                         
                                                                              
   else do                                                                     
                                                                               
   Say "FMCHEOPS can only be run on CERNVM..."                                  
   end                                                                         
\end{XMPt}
\Filename{H1fatmenappendix-cheops-interface-to-fatmen}
\chapter{CHEOPS interface to the FATMEN system}
The CHEOPS software uses the information passed from FATMEN
to stage in files pending transfer to the requested remote site.
Similar, once the file has been transferred, a request is made
through the FATMEN software to stage the file out to tape.

To simplify the interface between the CHEOPS software, written
in C, and the FATMEN package, written primarily in FORTRAN,
a special interface has been provided. This interface is intended
only for use by the appropriate CHEOPS software and should not
be called from user programs.
\Filename{H2fatmenappendix-stage-file}
\section{Stage in a file}
\Shubr{FMSTGI}{(GENAME,
     CFQNFA,CHSNFA,ICPLFA,IMTPFA,ILOCFA,CHSTFA,CHOSFA,
     CVSNFA,CVIDFA,IVIPFA,IDENFA,IVSQFA,IFSQFA,ISRDFA,
     IERDFA,ISBLFA,IEBLFA,CRFMFA,IRLNFA,IBLNFA,CFLFFA,
     CFUTFA,ICRTFA,ICTTFA,ILATFA,CCURFA,CCIDFA,CCNIFA,
     CCJIFA,IFPRFA,ISYWFA,IUSWFA,CUCMFA,
     CHLINK,CHOPT,IRC)}
\begin{DLtt}{1234567}
\item[GENAM]Character variable of maximum length 255 specifying the generic
name of the file to be staged.
\item[CFQNFA]Character variable of maximum length 256 specifying the
fully qualified dataset name of the file to be staged.
\item[CHSNFA]Character variable of maximum length 8 specifying the
host on which the file resides (disk files).
\item[ICPLFA]Integer variable specifying the copy level or data representation
of the file to be staged.
\item[IMTPFA]Integer variable specifying the media type code of the file to
be staged.
\item[ILOCFA]Integer variable specifying the location code of the file to be
staged.
\item[CHSTFA]Character variable of maximum length 16 specifying the
host type of the system on which the file resides.
\item[CHOSFA]Character variable of maximum length 12 specifying the 
operation system of the host on which the file resides.
\item[CVSNFA]Character variable of maximum length 8 specifying the
volume serial number of the tape on which the file resides.
\item[CVIDFA]Character variable of maximum length 8 specifying the
visual identifier of the volume on which the file resides.
\item[IVIPFA]Integer VID prefix of the volume
\item[IDENFA]Integer density
\item[IVSQFA]Integer volume sequence number
\item[IFSQFA]Integer file sequence number
\item[ISRDFA]Integer start record number
\item[IERDFA]Integer end record number
\item[ISBLFA]Integer start block number
\item[IEBLFA]Integer end block number
\item[CRFMFA]Character variable of length 4 specifying the record format
\item[IRLNFA]Integer record length in 32 bit words
\item[IBLNFA]Integer block length in 32 bit words
\item[CFLFFA]Character variable of length 4 specifying the FATMEN file format
\item[CFUTFA]Character variable of length 4 specifying the user defined file type
\item[ICRTFA]Integer containing the file creation date and time in packed format
\item[ICTTFA]Integer containing the date and time that the file was catalogued in packed format
\item[ILATFA]Integer containing the date and time of last access in packed format
\item[CCURFA]Character variable of length 8 specifying the user name of the
file creator
\item[CCIDFA]Character variable of length 8 specifying the account of the
creator
\item[CCNIFA]Character variable of length 8 specifying the node on which
the file was created
\item[CCJIFA]Character variable of length 8 specifying the name of the job
that created the file
\item[IFPRFA]Integer file protection mask
\item[ISYWFA]Integer vector of length 10 containing the system words
\item[IUSWFA]Integer vector of length 10 containing the user words
\item[CUCMFA]Character variable of length 80 containing the user comment
\item[CHLINK]Character variable specifying the link should point to the
staged file
\item[CHOPT]Character options
\item[IRC]Integer return code
\end{DLtt}
\Filename{H2fatmenappendix-stage-out-file}
\subsection{Stage out a file}
\Shubr{FMSTGO}{(GENAME,
     CFQNFA,CHSNFA,ICPLFA,IMTPFA,ILOCFA,CHSTFA,CHOSFA,
     CVSNFA,CVIDFA,IVIPFA,IDENFA,IVSQFA,IFSQFA,ISRDFA,
     IERDFA,ISBLFA,IEBLFA,CRFMFA,IRLNFA,IBLNFA,CFLFFA,
     CFUTFA,ICRTFA,ICTTFA,ILATFA,CCURFA,CCIDFA,CCNIFA,
     CCJIFA,IFPRFA,ISYWFA,IUSWFA,CUCMFA,
     CHLINK,CHOPT,IRC)}
\begin{DLtt}{1234567}
\item[GENAM]Character variable of maximum length 255 specifying the generic
name of the file to be staged.
\item[CFQNFA]Character variable of maximum length 256 specifying the
fully qualified dataset name of the file to be staged.
\item[CHSNFA]Character variable of maximum length 8 specifying the
host on which the file resides (disk files).
\item[ICPLFA]Integer variable specifying the copy level or data representation
of the file to be staged.
\item[IMTPFA]Integer variable specifying the media type code of the file to
be staged.
\item[ILOCFA]Integer variable specifying the location code of the file to be
staged.
\item[CHSTFA]Character variable of maximum length 16 specifying the
host type of the system on which the file resides.
\item[CHOSFA]Character variable of maximum length 12 specifying the 
operation system of the host on which the file resides.
\item[CVSNFA]Character variable of maximum length 8 specifying the
volume serial number of the tape on which the file resides.
\item[CVIDFA]Character variable of maximum length 8 specifying the
visual identifier of the volume on which the file resides.
\item[IVIPFA]Integer VID prefix of the volume
\item[IDENFA]Integer density
\item[IVSQFA]Integer volume sequence number
\item[IFSQFA]Integer file sequence number
\item[ISRDFA]Integer start record number
\item[IERDFA]Integer end record number
\item[ISBLFA]Integer start block number
\item[IEBLFA]Integer end block number
\item[CRFMFA]Character variable of length 4 specifying the record format
\item[IRLNFA]Integer record length in 32 bit words
\item[IBLNFA]Integer block length in 32 bit words
\item[CFLFFA]Character variable of length 4 specifying the FATMEN file format
\item[CFUTFA]Character variable of length 4 specifying the user defined file type
\item[ICRTFA]Integer containing the file creation date and time in packed format
\item[ICTTFA]Integer containing the date and time that the file was catalogued in packed format
\item[ILATFA]Integer containing the date and time of last access in packed format
\item[CCURFA]Character variable of length 8 specifying the user name of the
file creator
\item[CCIDFA]Character variable of length 8 specifying the account of the
creator
\item[CCNIFA]Character variable of length 8 specifying the node on which
the file was created
\item[CCJIFA]Character variable of length 8 specifying the name of the job
that created the file
\item[IFPRFA]Integer file protection mask
\item[ISYWFA]Integer vector of length 10 containing the system words
\item[IUSWFA]Integer vector of length 10 containing the user words
\item[CUCMFA]Character variable of length 80 containing the user comment
\item[CHLINK]Character variable specifying the link should point to the
staged file
\item[CHOPT]Character options
\item[IRC]Integer return code
\end{DLtt}

\Filename{H1fatmenappendix-security-issues}
\chapter{Security issues}

Limited security precautions are provided in FATMEN. They are as follows:

\begin{OL}
\item
There are no restrictions on access to catalogue information,
other than those provided at the file system level. That is,
any use may inspect any catalogue. 
\item
Directories may only be deleted if they are completely empty,
i.e. contain no subdirectories and no files.
\item
File entries may only be deleted by their creator. The creator
is identified by the account field.
\item
Protection may be established on directory trees, as explained below.
\end{OL}

\Filename{H2fatmenappendix-restricting-read-access-to-catalogue-info}
\section{Restricting read access to catalogue information}

If such restrictions are required, they must be performed at the
file system level. For example, on VAX/VMS systems one may use
access control to restrict read access to specific users
or users with a given rights identifier.

\Filename{H2fatmenappendix-access-files-catalogues-in-fatmen}
\section{Access to files catalogued in FATMEN}

No restrictions on file access are provided by FATMEN itself.
Volume access restrictions may be set up using the Tape Management System,
if required. For disk files, the access restrictions of the file
system again apply.
\Filename{H2fatmenappendix-access-control-lists}
\section{Access control lists}
\index{ACL}
\index{security}
\index{protection}
It is possible to restrict write access to the FATMEN catalogue
using an ACL file. The ACL file consists of 3 fields.
\begin{OL}
\item
User name
\item
Node name
\item
Path name
\end{OL}
Any of the fields may be wild-carded. When an update operation is
attempted, FATMEN checks the username, nodename and pathname 
against those defined in the file \Lit{FATMEN.ACL}. This file
resides in the directory or on the mini-disk containing the
catalogue for which the update is being attempted. Thus, in the
case of the CDF experiment, the file will be in the directory
pointed to by the symbol or variable \Lit{FMCDF}.

Lines beginning with an exclamation mark, an asterix, a hash or
slash asterix treated as comments.

\begin{XMPt}{Example of an ACL file}

/*              FATMEN.ACL file for CDF experiment at Fermilab         */

! ! !
! ! ! CDF FATMEN Superusers: can modify any directory
! ! !
LINGFENG * //FNAL/CDF
! ! !
! ! ! CDF FATMEN Test users: can modify the subtree //FNAL/CDF/FATMEN
! ! !
CDF_FATM * //FNAL/CDF/FATMEN    
! ! !
! ! ! CDF FATMEN General users: can modify <user> tree.
! ! ! e.g. FRODO can modify //FNAL/CDF/USERS/FRODO
! ! !
<user> * //FNAL/CDF/USERS/<user>
! ! !
! ! ! CDF FATMEN General users: user ID does not match user name:
! ! !
GPYEH * //FNAL/CDF/USERS/YEH
! ! !
! ! ! CDF FATMEN TOP group Superusers:
! ! !
GPYEH * //FNAL/CDF/TOP
! ! !
! ! ! CDF FATMEN TOP DILEPTON subgroup Superusers:
! ! !
LUC * //FNAL/CDF/TOP/DILEPTON
CHIKA * //FNAL/CDF/TOP/DILEPTON
CENYI * //FNAL/CDF/TOP/DILEPTON
\end{XMPt}
\Filename{H2fatmenappendix-account-aliases}
\section{Account aliases}
\index{account}
In addition to the above protection, catalogue entries may only
be deleted by users whose account matches that of the catalogue
entry. However, as shown in the following example, accounts are
often not common across different systems. 
\begin{XMP}
FM> ls -o                                                                   
 
 Directory: //CERN/CNDIV/TAPES
 
 Generic filename: XTRACE91
 Created by:  JAMIE    ACCT: JDS$CT   on node: CERNVM   by job: JAMIE   
 
 Generic filename: XTRACE92
 Created by:  jamie    ACCT: JDSCT    on node: zfatal   by job: jamie-15
 
\end{XMP}
To cater for this situation,
account aliasing is possible. An alias file, \Lit{FATMEN.ACCOUNTS},
consists of 2 fields.
\begin{OL}
\item
Account name
\item
List of aliases
\end{OL}
Note that this scheme is not designed for large numbers of users
with different accounts on each machine. It is intended for use
by a small number of production managers.
\begin{XMPt}{Example of an account alias file}
EMC$XV=CCA$XV,PRO$XV                                                            
CCA$XV=EMC$XV,PRO$XV                                                            
PRO$XV=CCA$XV,EMC$XV                                                            
\end{XMPt}

\Filename{H2fatmenappendix-update-control}
\section{Update control}

\index{updates}
\index{update}

By default, all file accesses that use the routine \Rind{FMOPEN}
or the shell \Rind{FIND} or \Rind{MAKE} commands result in
a logging record being sent to the servers. This record contains
information which includes how the file was accessed, the time taken to access
the file and updates the usage count and last access date and time.

Such updates can be controlled by a file named {\tt FATMEN.UPDATES}
which has the same format as
the {\tt FATMEN.ACL} file. If such a file exists, then the logging
information will only be sent if there is an appropriate line
in this file.
\begin{XMPt}{Example of an update control file}

!
! Log monitoring information only for production users
!
DELPROD * //CERN/DELPHI
!
! but also for users accessing their private files
!
<user> * //CERN/DELPHI/USERS/<user>

\end{XMPt}
\Filename{H1fatmenappendix-tms-at-cern}
\chapter{TMS at CERN}
\par
The following information describes the TMS as installed at CERN.
In general, the description is also valid for outside sites where
the TMS is also installed, such as the Rutherford Appleton Laboratory
in the UK, where the product originated, and IN2P3 and Saclay in France.
\par
The Tape Management System (TMS) has been installed at CERN
to manage the tapes and cartridges used at CERN. Information on some 450,000
volumes is held in a central database and may be queried and, in some cases,
updated by users.
\par
Note that FATMEN automatically issues TMS commands where necessary,
and so a detailed knowledge of the TMS and the command syntax is
not required. However, it may be useful to have a general overview
of the system.
{\bf Note that all access to TMS information through FATMEN is via
a generic name, and not a tape number.}
\par 
The TMS system runs in a special service machine and, as is the case for other
service  machines, does not accept commands directly but rather through a
special interface. For other service machines the interface is SMSG, but TMS
uses instead the SYSREQ interface (again from RAL) which handles automatic
continuati of responses to queries and which allows (following modifications at
CERN) queries from other CERN machines.
\par
The SYSREQ interface is also available on remote machines, such as the
central VAXcluster VXCERN, the Cray, the Siemens, SHIFT etc. This
remote interface runs over TCP/IP and is contained in the CERN Program
Library entry CSPACK. 

A TMS query may be issued as follows:
\begin{XMP}
   SYSREQ TMS QUERY VID YZ0001
\end{XMP}
and similarly for  other TMS commands.  Note that descriptions  of TMS commands
in the various help  files generally omit the "SYSREQ TMS",  but this is always
needed.

A list of all TMS commands is available via the command \underline{XFIND TMS} 
on CERNVM.  On remote systems, this information is available
through the www browser. To access the XFIND information directly, 
use a script such as
\begin{XMP}
# XFIND script
www "http://crnvmc/FIND" $@
\end{XMP} 
and then use the command \underline{XFIND TMS} as on CERNVM.
\par
In either case you can  then find information on a specific command
of interest.  To  find information about a particular TMS  command directly use
the  commands \underline{XFIND TMS  command}, e.g.
\begin{XMP}
xfind tms query
\end{XMP}

\Filename{H2fatmenappendix-TMS-volume-organisation}
\section{Volume Organisation in TMS}

The logical  organisation of  volumes within TMS  follows closely  the physical
organisation   and  is  therefore  based  around  volume  prefixes.   For  each
different  volume  prefix   a  number  of  TMS  LIBRARIES   are  defined,  each
corresponding to a  possible location for volumes in that  series.  Most series
will have the libraries/locations (where xx is the two lette VID prefix)
\begin{XMPt}{Examples of TMS libraries}
 
xx_VAULT the location  for volumes physically in the tape  vault.  Only volumes
         in these libraries will be able to mounted on the central systems.
 
xx_REST  a general location for volumes not in the vault
 
\end{XMPt}
\par
In  some cases  there will  be libraries  such as  xx\_EXPT for  volumes at  the
experiment - but  this depends on the  way an individual experiment  chooses to
manage its own volumes.

Certain FATMEN routines require a TMS library as an argument, such
as \Rind{FMALLO}, which allocates a volume from a named pool (see below)
from a given library.

\Filename{H2fatmenappendix-TMS-volume-ownership}
\section{Volume Ownership in TMS} 

As well as recording the location of a volume TMS also holds ownership details.
Ownership of  a volume  is by  ACCOUNT and  OWNER.  ACCOUNT  is the  usual CERN
accounting group, gg  - all Delphi volumes,  for example, are owned  by account
XX.

The OWNER  field in the  TMS database is a  little more special,  however.  All
registered CERN computer users have an individual user code, or UUU, and if the
owner  field  is  a three  character  UUU  then  the  volume is  owned  by  the
corresponding individual.

If the owner field  is longer than 3 characters this  then it cannot correspond
to a  UUU and so  no real person  owns the volume,  instead the owner  field is
being used  to group  the volume  into a  POOL for  a particular  purpose.  For
example  most experiments  wish to  maintain a  pool  of volumes  which can  be
allocated to  users on  request. These  volumes can  be given  the TMS  'owner'
GG\_USER and  it is then  easy to see  which volumes remain  in the pool  and to
choose one for allocation - perhaps  with the GETPOOL command, or better,
using the routine \Rind{FMALLO}.  The FATMEN shell command
{\bf ALLOCATE} may also be used to allocate volumes. {\bf Note that
FATMEN only retains information on volumes by generic name.
That is, there must be at least one file in the FATMEN catalogue
on a given volume}.
Another example
would be  to have  the 'owner'  for all DST  volumes as  GG\_DST so  users could
easily list the available  DSTs (and even for a given  location using the QUERY
CONTENTS command).
\par
Volumes may be moved between pools using the routine \Rind{FMPOOL}.
\Filename{H2fatmenappendix-TMS-access-rules}
\section{Volume Access Rules in TMS}
\par 
TMS always permits read access to all volumes by all users.
{\bf N.B. this may not be the case outside CERN!}
\par 
Write access is normally allowed to all members of the account group owning the
volume, but access  may be restricted by using PROTECTION  GROUPS. A protection
group allows access to be limited to  a small number of individuals or extended
to  members of  other groups.   The QUERY  VID command  gives the  name of  the
protection  group governing  write access  to a  volume and  the QUERY  PROTECT
command can  be used to  see the users permitted  access by a  given protection
group.
\par 
The owner of a  volume (or an authorised user for  'pool' volumes) may globally
disable write access using the LOCK command  - a software equivalent of the "Do
not write on this tape" sticker. This functionality is provided
by the routine \Rind{FMLOCK}. Volumes may be unlocked using the routine
\Rind{FMULOK}. LOCK and UNLOCK commands also exist in the FATMEN shell.
\Filename{H2fatmenappendix-TMS-create-pool}
\section{Creating TMS pools for use with FATMEN}
\par
The following TMS commands may be used to create TMS pools for use
with the routine \Rind{FMALLO} or the shell {\tt ALLOCATE} command.
Note that one may also allocate tapes from these pools by issuing
TMS commands directly without causing any conflict.
\par
Note that these commands require privilege and may normally only
be issued by the tape librarian.
\par
If the tapes are already known to the TMS, the following commands
should be used.
\begin{XMPt}{Creating a TMS pool}

PROTECT poolname ACCOUNT gg CREATE

PROTECT poolname ACCOUNT gg GRANT AUTHORITY CONTROL ACCOUNT gg
                                                 or USERID  uuu

TRANSFER VID vid1 - vid2 FROMUSER owner TO USERID poolname

\end{XMPt}
\par
The {\tt poolname} is the name of the pool, e.g. {\tt XU\_FREE}.
{\tt gg} and {\tt uuu} have their usual meanings. The two
{\tt GRANT AUTHORITY} commands differ in that the first permits
and member of the group {\tt gg} to allocate tapes from this pool,
where as the second enables only a specified user.
\par
If the tape volumes are not yet known to the TMS, they must be
entered using the {\tt ENTER} command. At this time, one may
also specify the {\tt OWNER} and hence avoid the {\tt TRANSFER}
command described above. For example, if the tapes are to be
in pool {\tt XX\_FREE}, then they should be entered with
{\tt USERID XX\_FREE}.
However, the {\tt PROTECT} commands are still required.    

\Filename{H2fatmenappendix-TMS-return-codes}
\section{TMS return codes}
\begin{verbatim}
#ifndef STNDMESG_H_INCLUDED
/************************************************************/
/*                                                          */
/* Object : Definition of the  standard  messages used  by  */
/* =======  various standard parts of the TMS system.  For  */
/*          inclusion in local message repositories         */
/*                                                          */
/* Attrib : TMS system header file.                         */
/* =======                                                  */
/*                                                          */
/* Author : (c)  Jonathan  Wood,  Systems  Group, CCD, RAL, */
/* =======  5_Nov_1988                                      */
/*                                                          */
/* ANSI C Version : Martin Ellerker, IBM Support Section,   */
/* ===============  Software Group, CERN.                   */
/*                  28_April_1993                           */
/*                                                          */
/************************************************************/
/*  File:          "stndmesg.h"                             */
/*  Dependencies:  "message.h"                              */
/************************************************************/
#define STNDMESG_H_INCLUDED 1
 
#include "message.h"
 
 
#ifndef EXCLUDE_convert_msgs
    #define EXCLUDE_convert_msgs 1
 
    QMSG(1,"Input fileid is missing")
    QMSG(2,"Unable to open input file")
    QMSG(3,"\nSyntax{%d:%d}. Expected '%s' "\
"found instead '%.*s'")
    QMSG(4,"%*.*s%c")
    QMSG(5,"Unable to open output file")
    QMSG(6,"Processed %d clauses in %d records to "\
"output file '%s'")
    QMSG(7,"Processing input file %s")
    QMSG(8,"Processed %d clauses in %d records "\
"with %d errors")
 
    #endif
 
#ifndef EXCLUDE_parser_msgs
    #define EXCLUDE_parser_msgs 1
 
    QMSG(20,"Syntax Error")
    QMSG(21,"Found '%.*s' where expected")
    QMSG(22,"%c<%-8.8s>")
    QMSG(23,"%c'%-8.8s'")
    QMSG(24,"%c %-8.8s ")
    QMSG(25,"Parser Error: %s :OPR %-8.8s")
    QMSG(27,"Parser Error: %s :OPT %-8.8s")
    QMSG(28,"Syntax fallacy at token: %-8.8s")
    QMSG(29,"Unable to load DLCS syntax file '%s'")
    QMSG(30,"Unexpected clause type during load at block %d")
 
    #endif
 
#ifndef EXCLUDE_index_msgs
    #define EXCLUDE_index_msgs 1
    QMSG(35,"Rebuild attempted for indices on: %.*s code %d")
    QMSG(36,"Routine for index rebuild on: %.*s not provided")
    #endif
 
#ifndef EXCLUDE_util_msgs
    #define EXCLUDE_util_msgs 1
 
    QMSG(40,"Getcor failure for %d * %d bytes")
    QMSG(41,"Message buffer too small: Msg number %5.5d")
    QMSG(42,"Host command complete rc=%d")
 
    #endif
 
 
#ifndef EXCLUDE_comms_msgs
    #define EXCLUDE_comms_msgs 1
 
    QMSG(45,"**** Possible command retry: may have been "\
"executed")
    QMSG(50,"Schedule table error. Item number %d."\
"      SQLCODE %d")
 
 
    #endif
 
#ifndef EXCLUDE_sched_msgs
    #define EXCLUDE_sched_msgs 1
 
    QMSG(55,"Recurrence interval too large!")
    QMSG(56,"Permission Denied: Scheduled request %d")
    QMSG(57,"Command too long to schedule!")
 
    QMSG(58,"Reqnum   Userid   Command  Last_Date       "\
"Next_date       Recursion")
    QMSG(59,"-------- -------- -------- --------------- "\
"--------------- -----------------")
    QMSG(60,"%8.1d %-8.8s %-8.8s %15.6f %15.6f %8.8s %-8.8s")
    QMSG(61,"%.*s %1.8d")
    QMSG(62,"%.*s")
 
    #endif
 
#ifndef EXCLUDE_feature_msgs
    #define EXCLUDE_feature_msgs 1
 
    QMSG(65,"Invalid feature type: %.*s")
    QMSG(66,"Unsupported feature: %.*s")
    QMSG(67,"Feature change rejected: %.*s")
    QMSG(68,"%-8.8s")
    QMSG(69,"%.*s %-8.8s")
 
    #endif
 
#ifndef EXCLUDE_secure_msgs
    #define EXCLUDE_secure_msgs 1
 
    QMSG(70,"Permission denied: Command %.*s")
    QMSG(71,"Not you again?")
    QMSG(72,"You are becoming a nuisance")
    QMSG(73,"Your activity is being logged")
    QMSG(74,"Desist!  You have been warned")
    QMSG(75,"Your file has been passed to "\
"the Fraud Squad")
 
    QMSG(76,"Authorisation not found")
    QMSG(77,"Authorisation already exists")
    QMSG(78,"%-9.8s")
    QMSG(79,"%.*s %-8.8s")
 
 
    #endif
 
#ifndef EXCLUDE_dbase_msgs
    #define EXCLUDE_dbase_msgs 1
 
    QMSG(80,"Severe SQL error SQLCODE=%d Attempting recovery")
    QMSG(81,"Severe SQL error SQLCODE=%d Unable to recover")
    QMSG(82,"SQL Diagnostic error report:\n"\
" SQLCAID: %-8.8s SQLCABC: %d SQLCODE: %d SQLERRP: %-8.8s")
    QMSG(83," SQLERRM: %.*s")
    QMSG(84," SQLERRD: %d %d %d %d %d %d")
    QMSG(85," SQLEXT : %-5.5s")
    QMSG(86," SQLWARN: 0123456789A\n          %-11.11s")
 
    #endif
 
#ifndef EXCLUDE_main_msgs
    #define EXCLUDE_main_msgs 1
 
    QMSG(87,"Installation configuration complete")
    QMSG(88,"Installation configuration already complete")
    QMSG(89,"Initialisation already complete")
    QMSG(90,"**** %.*s: Sub-Command implementation deferred")
    QMSG(91,"**** %s stopped by %-8.8s on %-8.8s")
    QMSG(92,"**** %.*s: Command implementation deferred")
    QMSG(93,"**** %s initialisation complete")
    QMSG(94,"**** Command processor restart: %d")
    QMSG(95,"%s restarting at %.24s in '%s' mode")
    QMSG(96,"**** New command accepted at %.24s from "\
"%-8.8s (%-8.8s) on %-8.8s:\n%.*s")
    QMSG(97,"**** Completion code %d at %.24s")
    QMSG(98,"%s terminating at %.24s with code: %d")
    QMSG(99,"**** The %s is temporarily unavailable")
 
    #endif
 
#endif                            /* ifndef x_H_INCLUDED     */
 
/* UPDATES APPLIED AT 09:43:45 ON 12 AUG 1993
UUU$GG
*         STNDME$H UUU$GG   D1 REL191  08/12/93   09:40:06
*/

#ifndef APPLMESG_H_INCLUDED
/************************************************************/
/*                                                          */
/* Object : Definition  of the  message repository for the  */
/* =======  application server program.  Insert application */
/*          specific messages.                              */
/*                                                          */
/* Attrib : Application system Header file.                 */
/* =======                                                  */
/*                                                          */
/* Author : (c)  Jonathan  Wood,  Systems  Group, CCD, RAL, */
/* =======  5_Nov_1988                                      */
/*                                                          */
/************************************************************/
/*  File:          "applmesg.h"                             */
/*  Included by:   "message.h"                              */
/************************************************************/
#define APPLMESG_H_INCLUDED 1
 
QMSG(101,"%s")
 
/*************************************************************/
/*  Msgs for QTAPE command                                   */
/*************************************************************/
QMSG(102,"%-8.8s %9.1ld %-8.8s %-8.8s %8.1ld %8.1ld")
QMSG(103,"Unknown volume : %-6.6s")
 
/*************************************************************/
/*  Msgs for QUERY VID command                               */
/*************************************************************/
QMSG(104,"Volume(s) not found")
QMSG(105," %-8.8s")
QMSG(106," %-6.6s")
QMSG(107," %-3.3s")
QMSG(108," %-1.1s")
QMSG(109," %-12.12s")
QMSG(110," %015.6f")
QMSG(111," %8.1ld")
QMSG(112," %5.1hd")
QMSG(113,"%.*s %.6s - %.6s %.*s")
QMSG(114,"%-6.6s")
QMSG(115," %8.8s")
 
/*************************************************************/
/*  Msgs for LIBRARY and QUERY LIBRARY commands              */
/*************************************************************/
QMSG(119,"Library not empty: %-.8s")
QMSG(120,"Duplicate library name: %-.8s")
QMSG(121,"Unknown library: %-.8s")
QMSG(122,"Library not empty: %-.8s")
QMSG(123,"Library size change invalid: %-8.8s")
QMSG(125,"Library  Czar     Group    R A L S M Target   "\
"Retain Racks  Slots    Spare")
QMSG(126,"-------- -------- -------- - - - - - -------- "\
"------ ------ -------- --------")
QMSG(127,"%-8.8s %-8.8s %-8.8s %c %c %c %c %c %-8.8s "\
"%6.1ld %6.1ld %8.1ld %8.1ld")
QMSG(128,"%.*s")
QMSG(129,"%.*s %-8.8s %-8.8s")
 
/*************************************************************/
/*  Msgs for RACK    command                                 */
/*************************************************************/
QMSG(130,"Slot(s) in use: %-8.8s %1.1d %1.1d")
QMSG(131,"Unracked library: %-8.8s")
QMSG(132,"Unknown rack: %-8.8s %1.1d")
QMSG(133,"Invalid status change: %-8.8s %1.1d %1.1d")
QMSG(134,"Invalid split/join for rack: %-8.8s %1.1d %1.1d")
QMSG(135,"Rack too large: %-8.8s %1.1d %1.1d")
QMSG(136,"Slot was not in use: %-8.8s %8.1d")
QMSG(137,"Invalid device model: %-6.6s")
 
/*************************************************************/
/*  Msgs for Rule validation functions.                      */
/*************************************************************/
QMSG(139,"Permission Denied: Account %-8.8s")
QMSG(140,"Permission Denied: Volume %-6.6s")
QMSG(141,"Permission Denied: No volumes found")
QMSG(142,"Permission Denied: Library %-8.8s not found")
QMSG(143,"Permission Denied: Library %-8.8s")
QMSG(144,"Permission Denied: Group %-8.8s Account %-8.8s")
 
/*************************************************************/
/*  Msgs for Group Manipulation functions                    */
/*************************************************************/
QMSG(145,"Permission Denied: Volume account mismatch")
QMSG(146,"Permission Denied: Group exists")
QMSG(147,"Permission Denied: Group in use")
 
/*************************************************************/
/*  Msgs for Query Group(s) functions                        */
/*************************************************************/
QMSG(148,"*Account ")
QMSG(149,"*Userid  ")
QMSG(150,"*World   ")
QMSG(151,"%-9.8s")
QMSG(152,"%-7.6s")
QMSG(153,"%.*s %c %-8.8s")
 
/*************************************************************/
/*  Msgs for Query RACK     function                         */
/*************************************************************/
QMSG(154,"Library  Start    End      Slots    Spare    "\
"Model  Status")
QMSG(155,"-------- -------- -------- -------- -------- "\
"------ ------")
QMSG(156,"%-8.8s %8.1d %8.1d %8.1d %8.1d %-6.6s %c")
QMSG(157,"%.*s %8.1d ")
 
/*************************************************************/
/*  Msgs for MOVE           functions                        */
/*************************************************************/
QMSG(160,"Insufficient space: Library %-8.8s")
QMSG(161,"Move initiated for %1.1d volumes out of %1.1d tried")
QMSG(162,"Logical library: %-8.8s")
QMSG(163,"Insufficient space in library")
 
/*************************************************************/
/*  Msgs for the REQUEST queue manipulation functions        */
/*************************************************************/
QMSG(170,"Duplicate action request for vid: %-6.6s")
QMSG(171,"Request not found: vid %-6.6s")
QMSG(172,"Permission denied: Request on vid %-6.6s")
 
QMSG(173,"A C Target_1 Target_2 Requestr Jobname  Owner    "\
"Date            Jobid   ")
QMSG(174,"- - -------- -------- -------- -------- -------- "\
"--------------- --------")
QMSG(175,"%c %c %-8.8s %8.1d %-8.8s %-8.8s %-8.8s %15.6f"\
" %-8.8s")
QMSG(176,"%.*s %8.1d")
 
QMSG(177,"%1.1d requests of type '%.*s' found")
QMSG(178,"%-6.6s %-8.8s %8.1d %c %-8.8s %8.1d "\
"%-8.8s %15.6f %-8.8s")
QMSG(179,"%.*s %-8.8s %1.1d %-8.8s %1.1d "\
"%-6.6s %1.1d")
QMSG(180,"VID    Cur_lib  Cur_slot C Target_1 Target_2 "\
"Userid   Date            Target_3")
QMSG(181,"------ -------- -------- - -------- -------- "\
"-------- --------------- --------")
 
/*************************************************************/
/*  Msgs for the MOUNT   command and subcommands             */
/*************************************************************/
QMSG(182,"Volume locked read-only: %-6.6s")
QMSG(183,"Operating system unknown: %-8.8s")
QMSG(184,"Volume preallocated: %-6.6s")
QMSG(185,"Volume being deleted: %-6.6s")
QMSG(186,"Volume absent: %-6.6s")
QMSG(187,"Volume held: %-6.6s")
QMSG(188,"Volume unknown to TMS: %-6.6s")
QMSG(189,"Volume not in %.*s: %-6.6s")
QMSG(190,"Volume busy: %-6.6s")
QMSG(191,"Volume mount not (yet) active: %-6.6s")
QMSG(192,"Volume ring status changed: %-6.6s")
QMSG(193,"%-6.6s %-8.8s %8.1d %-6.6s %-3.3s %8.8s %-8.8s "\
"%8.1d %8.1d %1.1c")
QMSG(194,"Attempting auto-move to library: %-8.8s")
QMSG(195,"Volume is not of type '%.*s': %-6.6s")
 
/*************************************************************/
/*  Msgs for the CLEAN   command                             */
/*************************************************************/
QMSG(196,"Clean initiated for %1.1d volumes out of %1.1d "\
"tried")
 
/*************************************************************/
/*  Msgs for the ENTER/ISSUE commands                        */
/*************************************************************/
QMSG(200,"Rejected: Unknown device type: Model %-6.6s "\
"Max_den %-6.6s Length %d")
QMSG(201,"Rejected: Invalid density: %-6.6s")
QMSG(202,"Rejected: Invalid VSN range %-.6s-%-.6s")
QMSG(203,"Rejected: VID not the same as VSN")
QMSG(204,"Rejected: VID wraparound")
QMSG(205,"Rejected: Only %d slots available for %d volumes")
QMSG(206,"Rejected: %d VIDs in range already exist")
QMSG(207,"Rejected: Invalid Account(%-8.8s) or "\
"Userid(%-8.8s)")
QMSG(208,"Rejected: Congested VID range - pick another!")
QMSG(209,"Rejected: Ran out of slots with %d volumes to go")
QMSG(210,"Rejected: Unexpectedly entered no volumes!!")
QMSG(211,"Rejected: Insufficient supply of requested "\
"device type")
QMSG(212,"Rejected: Unknown allocation series: %-8.8s")
QMSG(213,"Rejected: Limit of %d volumes exceeded")
QMSG(214,"Rejected: Foreign batch not found: Model %.6s "\
"Max_den %-6.6s Length %d")
 
/*************************************************************/
/*  Msgs for the TAG     command                             */
/*************************************************************/
QMSG(220,"%d")
QMSG(221,"%.*s")
 
/*************************************************************/
/*  Msgs for the REMOVE and RETURN commands                  */
/*************************************************************/
QMSG(225,"Remove initiated for %1.1d volumes out of %1.1d "\
"tried")
QMSG(226,"Return initiated for %1.1d volumes out of %1.1d "\
"tried")
QMSG(227,"Volume is not absent: %6.6s")
 
/*************************************************************/
/*  Msgs for the DEVTYPE and QUERY DEVTYPES commands         */
/*************************************************************/
QMSG(230,"Unknown devtype: Model %-6.6s Max_den %d "\
"Length %d")
QMSG(231,"Duplicate devtype: Model %-6.6s Max_den %d "\
"Length %d")
QMSG(232,"Devtype in use: Model %-6.6s Max_den %d "\
"Length %d")
 
QMSG(233,"Model   Max_den   Length     Code Sort Group    "\
"P Max_strp Strp_siz")
QMSG(234,"------ -------- -------- -------- ---- -------- "\
"- -------- --------")
QMSG(235,"%6.6s %8.8s %8.1d %8.1d %4.4s %8.8s %c %8.1d %8.1d")
QMSG(236,"%.*s : %.6s %8.8s %d")
 
/*************************************************************/
/*  Msgs for the SUPPLY  and QUERY SUPPLY   commands         */
/*************************************************************/
QMSG(240,"Supply order requested not found")
QMSG(241,"Only %d volumes are still outstanding for this "\
"supply order")
QMSG(242,"Supply order already exists")
 
QMSG(243,"Reqnum   Batchnum Model    Maxden   Length S "\
"   Total     Left  Broken  Date")
QMSG(244,"-------- -------- ------ -------- -------- - "\
"-------- -------- -------- --------")
QMSG(245,"%-8.8s %8.1d %-6.6s %8.8s %8.1d %c %8.1d %8.1d "\
"%8.1d %8.8d")
 
QMSG(246,"%.*s %d")
QMSG(247,"%.*s %d %d")
 
QMSG(248,"Reqnum   Batchnum Model    Maxden   Length S "\
"   Total     Cost Userid   Manufact")
QMSG(249,"%-8.8s %8.1d %-6.6s %8.8s %8.1d %c %8.1d %8.1d "\
"%-8.8s %-8.*s")
 
QMSG(250,"Supply batch %d not found")
QMSG(251,"Batch %d is not in supply")
QMSG(252,"Batch %d is not HELD")
QMSG(253,"Supply batch only has %d volumes left")
QMSG(254,"Supply batch only has %d broken volumes")
QMSG(255,"Batch %d is in use")
 
QMSG(258,"Density value must not be greater than MaxDensity")
 
 
QMSG(260,"Volume %-6.6s not Free or Held")
QMSG(261," Volume %-6.6s")
QMSG(262,"Transfer from %-8.8s to %-8.8s and account "\
"unchanged")
QMSG(263,"Non existant new group: %-8.8s Account %-8.8s "\
"Volume %-6.6s")
QMSG(264,"No matching volumes found belonging to %-8.8s")
QMSG(265,"Permission Denied: Operand %.*s")
 
/*************************************************************/
/*  Msgs for the QUERY CONTENTS command.                     */
/*************************************************************/
QMSG(270,"Empty library: %-8.8s")
QMSG(271,"Library      Slot Type VID    Model  Owner    "\
"Account  S")
QMSG(272,"-------- -------- ---- ------ ------ -------- "\
"-------- -")
QMSG(273,"%-8.8s %8.1d Home %-6.6s %-6.6s %-8.8s %-8.8s %c")
QMSG(274,"%-8.8s %8.1d Move %-6.6s %-6.6s %-8.8s %-8.8s %c")
QMSG(275,"%.*s %-6.6s %8.1d")
QMSG(276,"No volumes found in given slot range")
QMSG(277,"Slots for library %-8.8s between %8.1d and %8.1d")
 
/*************************************************************/
/*  Msgs for the OPSYSTEM and QUERY OPSYSTEM commands        */
/*************************************************************/
QMSG(280,"Unknown Operating system: %-8.8s")
QMSG(281,"Duplicate Operating system: %-8.8s")
QMSG(282,"System   Generic  Userid   SysType  Enq Scr "\
"Location")
QMSG(283,"-------- -------- -------- -------- --- --- "\
"------------------------------")
QMSG(284,"%-8.8s %-8.8s %-8.8s %-8.8s  %c   %c  %.*s")
QMSG(285,"%.*s %.8s %.8s")
QMSG(286,"Requests outstanding from system %-8.8s")
QMSG(287,"Require location identifier")
 
/*************************************************************/
/*  Messages for the LOGICAL family of commands              */
/*************************************************************/
QMSG(290,"Library not logical: %-8.8s")
QMSG(291,"Duplicate cartridge: %-8.8s")
QMSG(292,"Unknown cartridge: %-8.8s")
QMSG(293,"Stripe range invalid: %1.1d-%1.1d")
QMSG(294,"Stripe(s) in use: %-8.8s %1.1d-%1.1d")
 
QMSG(295,"Cart     VID    Stripe_1 Stripe_2 Last_written    "\
"S Userid   Account  Library")
QMSG(296,"-------- ------ -------- -------- --------------- "\
"- -------- -------- --------")
QMSG(297,"%-8.8s %-6.6s %8.1d %8.1d %15.6f "\
"%c %-8.8s %-8.8s %-8.8s")
QMSG(298,"%.*s %-6.6s")
 
QMSG(299,"Cart     Model  Volumes  Userid   Account  "\
"Library      Slot Group")
QMSG(300,"-------- ------ -------- -------- -------- "\
"-------- -------- --------")
QMSG(301,"%-8.8s %-6.6s %8.1d %-8.8s %-8.8s "\
"%-8.8s %8.1d %-8.8s")
QMSG(302,"%.*s %-8.8s %-8.8s")
 
/*************************************************************/
/* For the QVOL command                                     */
/*************************************************************/
QMSG(310,"%-6.6s Volume Not Known. %25.1c")
QMSG(311,"%-6.6s %-8.8s %8.8ld %-6.6s %8.8s %1.1c %1.1c "\
"%1.1c %1.1c %1.1c %1.1c %8.8s %5.5ld %5.5ld %-3.3s")
QMSG(313,"Unknown Generic Operating System: %-8.8s")
QMSG(314,"Unknown Operating System: %-8.8s")
QMSG(315,"Volume unmountable")
QMSG(316,"Ignoring option: %-8.8s")
 
/*************************************************************/
/* For the ALLOCATE command                                 */
/*************************************************************/
QMSG(320,"Can not allocate zero tapes.")
QMSG(321,"Allocated %d tape%sof device code %d")
QMSG(322,"Allocated %d tape%sfrom batch %d.")
 
/*************************************************************/
/* For the Transparent Tape Staging Interface.              */
/*************************************************************/
QMSG(330,"rc = %d  from TMSPURGE of vid: %6.6s")
QMSG(331,"Error flushing tape in Transparent Tape Stager. "\
"Probable stage out pending.")
QMSG(400,"Exchange initiated for vid: %6.6s")
 
/*************************************************************/
/*  Msgs for the CZAR/MANAGER command                        */
/*************************************************************/
QMSG(1076,"Czar/Account/Node not found")
QMSG(1077,"Czar/Account/Node already exists")
QMSG(1078,"%-9.8s")        /* Not used */
QMSG(1079,"%.*s %-8.8s %-8.8s %-8.8s")
QMSG(1080,"%.*s %-8.8s %-8.8s")
QMSG(1081,"%.*s %-8.8s")
QMSG(1121,"Unknown Czar   : %-.8s")
QMSG(1122,"Unknown Account: %-.8s")
QMSG(1125,"Manager  Account  Node     Set")
QMSG(1126,"-------- -------- -------- ---")
QMSG(1127,"%-8.8s %-8.8s %-8.8s  %c")
QMSG(1128,"%.*s")
QMSG(1129,"%.*s %-8.8s %-8.8s")
QMSG(1130,"TMS authorization granted to %-8.8s for "\
"account %-8.8s node %-8.8s")
QMSG(1131,"TMS authorization revoked from %-8.8s for "\
"account %-8.8s node %-8.8s")
QMSG(1140,"Permission Denied: Account %-8.8s Node %-8.8s")
 
/*************************************************************/
/*  Msgs for the Frequently Used Commands algorithm          */
/*************************************************************/
QMSG(1200,"**** Vector initialisation failure")
QMSG(1205,"Command    Location    Count MilliSec   "\
"    Next Previous   Syntax")
QMSG(1206,"--------   -------- -------- --------   "\
"-------- -------- --------")
QMSG(1207,"%-8.8s   %8x %8lu %8lu   %8x %8x %8x")
QMSG(1208,"%-8.8s %6lu")
QMSG(1210,"%.*s %-8.8s")
 
/*************************************************************/
/*  Message for the QUERY FREE1ST SUBCOMMAND                 */
/*************************************************************/
QMSG(2001,"First free slot: %d")
 
/*************************************************************/
/*  Msgs for the FLUSH SYSTEM command                        */
/*************************************************************/
QMSG(2010,"All MOUNT requests flushed")
QMSG(2011,"No MOUNT requests found")
 
/*************************************************************/
/*  Msgs for the extended AUTHUSER command                   */
/*************************************************************/
QMSG(2076,"%-8.8s")
QMSG(2077,"%-8.8s by %-8.8s")
QMSG(2078,"%-8.8s AT %-8.8s")
QMSG(2079,"%.*s %-8.8s %-8.8s")
QMSG(2080,"%.*s FROM %-8.8s")
 
/*************************************************************/
/*  Msgs for the extended QUERY COMMAND command              */
/*************************************************************/
QMSG(2090,"General commands for %-8.8s at %-8.8s")
QMSG(2091,"All possible commands for %-8.8s at %-8.8s")
QMSG(2092,"Authorised commands for %-8.8s at %-8.8s")
QMSG(2095,"------- -------- --- -------- -- --------")
QMSG(2096,"--- -------- -------- --- -------- -- --------")
QMSG(2097,"---------- -------- --- -------- -- --------")
QMSG(2100,"General command %-8.8s not found")
QMSG(2101,"Command %-8.8s not found")
QMSG(2102,"Authorised command %-8.8s not found")
 
/*************************************************************/
/*  Msgs for the extended QUERY LIBRARY command              */
/*************************************************************/
QMSG(2125,"Library  Czar     Group    Slots    Spare    "\
"Location")
QMSG(2126,"-------- -------- -------- -------- -------- "\
"------------------------------")
QMSG(2127,"%-8.8s %-8.8s %-8.8s %8.1ld %8.1ld %.*s")
QMSG(2129,"%.*s %-8.8s %-8.8s ( LOCID")
 
/*************************************************************/
/*  Msgs for the extended unit address field in MOUNT        */
/*************************************************************/
QMSG(2175,"%c %c %-8.8s %8.8s %-8.8s %-8.8s %-8.8s "\
"%15.6f %-8.8s")
QMSG(2178,"%-6.6s %-8.8s %8.1d %c %-8.8s %8.8s "\
"%-8.8s %15.6f %-8.8s")
 
/*************************************************************/
/*  Msgs for the extended checking in the MOUNT command      */
/*************************************************************/
QMSG(2190," Volume in use on this system: %6.6s")
QMSG(2193," Request for volume %6.6s creates deadlock")
 
/*************************************************************/
/*  Msgs for the GENERIC and QUERY GENERIC commands          */
/*************************************************************/
QMSG(2200,"No Generic type for model %-6.6s library %-8.8s "\
"density %d system %-8.8s")
QMSG(2201,"Generic type for model %-6.6s library %-8.8s "\
"density %d system %-8.8s already exists")
QMSG(2202,"Model  Library  Density  SysType  Name  ")
QMSG(2203,"------ -------- -------- -------- ------")
QMSG(2204,"%-6.6s %-8.8s %8.8s %-8.8s %-6.6s")
QMSG(2205,"%.*s %-6.6s %-8.8s %8.8s %-8.8s")
 
/*************************************************************/
/*  Msgs for the FINDVIDS command                            */
/*************************************************************/
QMSG(2210,"FINDVIDS USERID %-8.8s VID %-6.6s TEXT   "\
"LIKE %.*s")
QMSG(2211,"FINDVIDS USERID %-8.8s VID %-6.6s BINARY "\
"LIKE %.*s")
QMSG(2212,"FINDVIDS USERID %-8.8s VID %-6.6s VOLINFO "\
"LIKE %.*s")
QMSG(2213,"FINDVIDS LIBRARY %-8.8s VID %-6.6s TEXT   "\
"LIKE %.*s")
QMSG(2214,"FINDVIDS LIBRARY %-8.8s VID %-6.6s BINARY "\
"LIKE %.*s")
QMSG(2215,"FINDVIDS LIBRARY %-8.8s VID %-6.6s VOLINFO "\
"LIKE %.*s")
QMSG(2216,"%-6.6s %.*s")
 
/*************************************************************/
/*  Msgs for the UPDATE command                              */
/*************************************************************/
QMSG(2400,"No new label information given")
QMSG(2401,"Label for VID %-6.6s: Type=%-3.3s;  VSN=%-6.6s;  "\
"Density=%8.8s")
 
/*************************************************************/
/*  Msgs for the DENSITY mapping command                     */
/*************************************************************/
QMSG(2500,"Model  Density  Code     P")
QMSG(2501,"------ -------- -------- -")
QMSG(2502,"%-6.6s %-8.8s %8.d %c")
QMSG(2503,"%.*s %-6.6s %-8.8s %8.d")
QMSG(2504,"Cannot update Primary Key Field")
QMSG(2505,"Mapping already exists")
QMSG(2506,"Mapping does not exist")
QMSG(2507,"Density %8.8s invalid for model %6.6s")
QMSG(2510,"Cannot delete Primary Entry")
QMSG(2511,"Cannot update code for Primary Entry")
 
/*************************************************************/
/*  Msgs for the LOCATION mapping command                    */
/*                                                           */
/*  Warning --> Location command relies on 2505 & 2506 above */
/*************************************************************/
QMSG(2520,"Code   Location Description")
QMSG(2521,"------ ------------------------------")
QMSG(2522,"%-6.d %.*s")
QMSG(2523,"%.*s %-6.d")
 
/*************************************************************/
/*  Msgs for the extended FEATURE QUERY ALERTEE command      */
/*************************************************************/
QMSG(9068,"%-8.8s at %-8.8s")
QMSG(9069,"%.*s %-8.8s AT %-8.8s")
 
/*************************************************************/
/*  Msgs for general Code debugging facilities               */
/*************************************************************/
QMSG(9900,"SQL Debug: %-8.8s  ->  %8.8x.%8.8x")
QMSG(9910,"PDate: %8.8d")
QMSG(9911,"PTime: %8.8d")
QMSG(9912,"DateTime: %f")
QMSG(9920,"Sqlcode: %d")
QMSG(9921,"__%s.%d")
QMSG(9922,"R%d code %d")
 
#endif
 
/* UPDATES APPLIED AT 09:36:29 ON 12 AUG 1993
UNITADDR * T. Cass      22 Mar 91 * Store MOUNT unit addr as char strin
*         APPLMESG UNITADDR D1 REL191  06/07/93   14:20:05
VOLINFO  * M. Ellerker   7 Jul 93 * Find Volinfo for user/lib
*         APPLMESG VOLINFO  D1 REL191  07/07/93   14:18:15
REORG10  * M. Ellerker  23 Jul 93 * Only allow GENERIC OR SYSTEM
*         APPLMESG REORG10  D1 REL191  08/06/93   14:30:20
DENMAP   * M. Ellerker   6 Aug 93 * Forgot this one !
*         APPLMESG DENMAP   D1 REL191  08/06/93   17:28:17
*/
\end{verbatim}

\Filename{H1fatmenappendix-summary-fatmen-system}
\chapter{Summary of the FATMEN system}

\begin{Tabhere}
\caption{FATMEN command line interface}
\begin{center}
\begin{tabular}{|>{\tt}p{5cm}r|>{\tt}p{5cm}r|}
\hline
\rm\bf Command           & \bf page                &
\rm\bf Command           & \bf page                \\
\hline
ALLOCATE                 & \pageref{ALLOCATE}      &
ADDDISK                  & \pageref{ADDDISK}       \\
ADDTAPE                  & \pageref{ADDTAPE}       &
CD                       & \pageref{CD}            \\
CLR                      & \pageref{CLR}           &
CP                       & \pageref{CP}            \\
COPY                     & \pageref{COPY}          &
DIR                      & \pageref{DIR}           \\
DUMP                     & \pageref{DUMP}          &
END                      & \pageref{END}           \\
EXIT                     & \pageref{EXIT}          &
EXTRACT                  & \pageref{EXTRACT}       \\
FC                       & \pageref{FC}            &
FIND                     & \pageref{FIND}          \\
GIME                     & \pageref{GIME}          &
INIT                     & \pageref{INIT}          \\
LD                       & \pageref{LD}            &
LOCK                     & \pageref{LOCK}          \\
LOGLEVEL                 & \pageref{LOGLEVEL}      &
LS                       & \pageref{LS}            \\
MAKE                     & \pageref{MAKE}          &
MEDIA                    & \pageref{MEDIA}         \\
MKDIR                    & \pageref{MKDIR}         &
MODIFY                   & \pageref{MODIFY}        \\
                         &                         &
MV                       & \pageref{MV}            \\
PWD                      & \pageref{PWD}           &
QUIT                     & \pageref{QUIT}          \\
RM                       & \pageref{RM}            &
RMDIR                    & \pageref{RMDIR}         \\
RMTREE                   & \pageref{RMTREE}        &
SEARCH                   & \pageref{SEARCH}        \\
TAG                      & \pageref{TAG}           &
TOUCH                    & \pageref{TOUCH}         \\
TREE                     & \pageref{TREE}          &
UNLOCK                   & \pageref{UNLOCK}        \\
UPDATE                   & \pageref{UPDATE}        &
VERSION                  & \pageref{VERSION}       \\
VIEW                     & \pageref{VIEW}          &
ZOOM                     & \pageref{ZOOM}          \\
SET/COPYLEVEL            & \pageref{SETCOPYLEVEL}  &
SET/MEDIATYPE            & \pageref{SETMEDIATYPE}  \\
SET/LOCATION             & \pageref{SETLOCATION}   &
SET/USERWORDS            & \pageref{SETUSERWORDS}  \\
SET/DATAREP              & \pageref{SETDATAREP}    &
SET/SOURCE               & \pageref{SETSOURCE}     \\
SET/DESTINATION          & \pageref{SETDESTINATION}& 
SHOW/SOURCE              & \pageref{SHOWSOURCE}    \\
SHOW/DESTINATION         & \pageref{SHOWDESTINATION}& 
SHOW/DATAREP             & \pageref{SHOWDATAREP}   \\
SHOW/COPYLEVEL           & \pageref{SHOWCOPYLEVEL} &
SHOW/LOCATION            & \pageref{SHOWLOCATION}  \\
SHOW/MEDIATYPE           & \pageref{SHOWMEDIATYPE} &
SHOW/USERWORDS           & \pageref{SHOWUSERWORDS} \\
\hline
\end{tabular}
\end{center}
\end{Tabhere}
\clearpage

\extrarowheight0pt
\begin{longtable}{|p{.9\linewidth}r|} 
\caption{FATMEN Routine calling sequences}                 \\
\hline
 \bf Function                           &                  \\
 \quad\bf Description                   & \bf page         \\
\hline
\endfirsthead
\caption[]{FATMEN Routine calling sequences (cont.)}       \\
\hline
 \bf Function                           &                  \\
 \quad\bf Description                   & \bf page         \\
\hline
\endhead
\hline
\endfoot
Initialise FATMEN system & \\
  \quad\small\tt CALL FMSTRT(LUNRZ,LUNFZ,CHFAT,IRC*) & \pageref{FMSTRT} \\
Access a dataset & \\
  \quad\small\tt CALL FMFILE(LUN,GENAM,CHOPT,IRC*) & \pageref{FMFILE} \\
Deaccess a dataset & \\
  \quad\small\tt CALL FMFEND(LUN,GENAM,CHOPT,IRC*) & \pageref{FMFEND} \\
Add a tape file & \\
  \quad\small\tt CALL FMADDT(argument-list) & \pageref{FMADDT} \\
Add a disk file & \\
  \quad\small\tt CALL FMADDD(argument-list) & \pageref{FMADDD} \\
Return information on FATMEN entry & \\
  \quad\small\tt CALL FMPEEK(GENAM,IVECT,CHOPT,IRC*) & \pageref{FMPEEK} \\
Add entry to catalogue & \\
  \quad\small\tt CALL FMPOKE(GENAM,IVECT,CHOPT,IRC*) & \pageref{FMPOKE} \\
Initialise FATMEN system & \\
  \quad\small\tt CALL FMINIT(IXSTOR*,LUNRZ,LUNFZ,DBNAME,IRC*) & \pageref{FMINIT} \\
Terminate FATMEN package & \\
  \quad\small\tt CALL FMEND(IRC*) & \pageref{FMEND} \\
Set logging level of FATMEN package & \\
  \quad\small\tt CALL FMLOGL(LEVEL) & \pageref{FMLOGL} \\
Control updating mode & \\
  \quad\small\tt CALL FMUPDT(MAX,NGROUP,IFLAG,IRC*) & \pageref{FMUPDT} \\
Purge old entries from catalogue & \\
  \quad\small\tt CALL FMPURG(PATH,KEYSEL,MAXSIZ,MINACC,MAXDAYS,MINCPS,LUNPUR,CHOPT,IRC*) & \pageref{FMPURG} \\
Get information on named file & \\
  \quad\small\tt CALL FMGET(GENAM,LBANK*,KEYS*,IRC*) & \pageref{FMGET} \\
Get information on named file with key selection & \\
  \quad\small\tt CALL FMGETK(GENAM,LBANK*,*KEYS*,IRC*) & \pageref{FMGETK} \\
Add entry to FATMEN catalogue & \\
  \quad\small\tt CALL FMPUT(GENAM,LBANK,IRC*) & \pageref{FMPUT} \\
Modify existing entry & \\
  \quad\small\tt CALL FMMOD(GENAM,LBANK,IFLAG,IRC*) & \pageref{FMMOD} \\
Create a new FATMEN bank & \\
  \quad\small\tt CALL FMBOOK(GENAM,KEYS*,LADDR*,*LSUP*,JBIAS,IRC*) & \pageref{FMBOOK} \\
Remove entry from FATMEN catalogue & \\
  \quad\small\tt CALL FMRM(GENAM,LBANK*,KEYS,IRC*) & \pageref{FMRM} \\
Make directory & \\
  \quad\small\tt CALL FMKDIR(CHDIR,IRC*) & \pageref{FMKDIR} \\
Set contents of FATMEN bank & \\
  \quad\small\tt CALL FMFILL(GENAM,*LBANK*,*KEYS*,CHOPT,IRC*) & \pageref{FMFILL} \\
Insert character data into FATMEN bank & \\
  \quad\small\tt CALL FMPUTC(LBANK,STRING,ISTART,NCH,IRC*) & \pageref{FMPUTC} \\
Read character data from FATMEN bank & \\
  \quad\small\tt CALL FMGETC(LBANK,STRING*,ISTART,NCH,IRC*) & \pageref{FMGETC} \\
Insert integer vector into FATMEN bank & \\
  \quad\small\tt CALL FMPUTV(LBANK,IVECT,ISTART,NWORDS,IRC*) & \pageref{FMPUTV} \\
Read integer vector from FATMEN bank & \\
  \quad\small\tt CALL FMGETV(LBANK,IVECT*,ISTART,NWORDS,IRC*) & \pageref{FMGETV} \\
Insert integer value into FATMEN bank & \\
  \quad\small\tt CALL FMPUTI(LBANK,IVAL,IOFF,IRC*) & \pageref{FMPUTI} \\
Read integer value from FATMEN bank & \\
  \quad\small\tt CALL FMGETI(LBANK,IVAL*,IOFF,IRC*) & \pageref{FMGETI} \\
Find existing dataset and associate with logical unit & \\
  \quad\small\tt CALL FMFIND(GENAM,DDNAME,*LBANK*,IRC*) & \pageref{FMFIND} \\
Create new dataset & \\
  \quad\small\tt CALL FMMAKE(GENAM,DDNAME,*LBANK*,IRC*) & \pageref{FMMAKE} \\
Open a dataset for read or write & \\
  \quad\small\tt CALL FMOPEN(GENAM,DDNAME,*LBANK*,CHOPT,IRC*) & \pageref{FMOPEN} \\
Close file opened via FATMEN & \\
  \quad\small\tt CALL FMCLOS(GENAM,DDNAME,LBANK,CHOPT,IRC*) & \pageref{FMCLOS} \\
Copy a dataset and update the FATMEN catalogue & \\
  \quad\small\tt CALL FMCOPY(GN1,*LBANK1*,*KEYS1*,GN2,*LBANK2*,*KEYS2*,CHOPT,IRC*) & \pageref{FMCOPY} \\
Check whether generic name already exists & \\
  \quad\small\tt CALL FMEXST(GENAM,IRC*) & \pageref{FMEXST} \\
List files in specified directory & \\
  \quad\small\tt CALL FMLS(GENAM,CHOPT,IRC*) & \pageref{FMLS} \\
Display contents of FATMEN bank & \\
  \quad\small\tt CALL FMSHOW(GENAM,*LBANK*,*KEYS*,CHOPT,IRC*) & \pageref{FMSHOW} \\
Count file names & \\
  \quad\small\tt CALL FMFILC(GENAM,NFILES*,IRC*) & \pageref{FMFILC} \\
Scan FATMEN directory structure & \\
  \quad\small\tt CALL FMSCAN(PATH,NLEVEL,UROUT,IRC*) & \pageref{FMSCAN} \\
Loop through FATMEN file names & \\
  \quad\small\tt CALL FMLOOP(GENAM,UROUT,IRC*) & \pageref{FMLOOP} \\
Return directory names in directory structure & \\
  \quad\small\tt CALL FMLDIR(PATH,DIRS*,NFOUND,MAXDIR,*ICONT*,IRC*) & \pageref{FMLDIR} \\
Return file names in directory structure & \\
  \quad\small\tt CALL FMLFIL(GENAM,FILES*,KEYS*,NFOUND,MAXFIL,JCONT,IRC*) & \pageref{FMLFIL} \\
Sort file names and keys & \\
  \quad\small\tt CALL FMSORT(FILES,KEYS,NFILES,JSORT*,IRC*) & \pageref{FMSORT} \\
Match file name against pattern & \\
  \quad\small\tt CALL FMATCH(CHFILE,MATCH,IRC*) & \pageref{FMATCH} \\
Match multiple names against pattern & \\
  \quad\small\tt CALL FMMANY(MATCH,FILES,NFILES,NMATCH*,IRC*) & \pageref{FMMANY} \\
Print contents of FATMEN keys vector & \\
  \quad\small\tt CALL FMPKEY(KEYS,NKEYS) & \pageref{FMPKEY} \\
Select files using the FATMEN keys & \\
  \quad\small\tt CALL FMSELK(GENAM,INKEYS,OUKEYS*,NKEYS*,MAXKEY,IRC*) & \pageref{FMSELK} \\
Select files using the FATMEN bank information & \\
  \quad\small\tt CALL FMSELB(GENAM,INKEYS,NKEYS,UEXIT,ISEL*,IRC*) & \pageref{FMSELB} \\
Select files using keys matrix & \\
  \quad\small\tt CALL FMSELM(GENAM,LBANK*,KEYS*,KEYM,NKEY,CHOPT,IRC*) & \pageref{FMSELM} \\
Compare FATMEN entries & \\
  \quad\small\tt CALL FMCOMP(GENAM1,*LBANK1*,*KEYS1*,GENAM2,*LBANK2*,*KEYS2*,IRC*) & \pageref{FMCOMP} \\
Print user words and comment & \\
  \quad\small\tt CALL FMUPRT(GENAM,LBANK,KEYS,IUSER,COMM,IRC*) & \pageref{FMUPRT} \\
User selection & \\
  \quad\small\tt CALL FMUSEL(GENAM,LBANK,KEYS,IRC*) & \pageref{FMUSEL} \\
Allocate new piece of media & \\
  \quad\small\tt CALL FMALLO(MEDIA,DENS,COMPACT,LIB,POOL,LBANK,CHOPT,VSN*,VID*,IRC*) & \pageref{FMALLO} \\
Move volumes between TMS pools & \\
  \quad\small\tt CALL FMPOOL(GENAM,LBANK,KEYS,CHPOOL,CHOPT,IRC) & \pageref{FMPOOL} \\
Obtain volume characteristics & \\
  \quad\small\tt CALL FMQVOL(GENAM,LBANK,KEYS,LIB*,MODEL*,DENS*,MNTTYP*,LABTYP*,IRC*) & \pageref{FMQVOL} \\
Obtain media information & \\
  \quad\small\tt CALL FMQMED(GENAM,*LBANK*,*KEYS*,IMEDIA*,IROBOT*,IRC*) & \pageref{FMQMED} \\
Set default media information & \\
  \quad\small\tt CALL FMEDIA(MFMMED,MFMTYP,MFMGEN,MFMSIZ,MFMDEN,MFMMNT,MFMLAB,NMEDIA,IRC*) & \pageref{FMEDIA} \\
Get, Set or Delete TMS Tags & \\
  \quad\small\tt CALL FMTAGS(GENAM,*LBANK*,*KEYS*,*CHTAG*,CHOPT,IRC*) & \pageref{FMTAGS} \\
Declare location codes to FATMEN & \\
  \quad\small\tt CALL FMSETL(LOC,NLOC,IRC*) & \pageref{FMSETL} \\
Declare media types to FATMEN & \\
  \quad\small\tt CALL FMSETM(MTP,NMTP,IRC*) & \pageref{FMSETM} \\
Declare copy levels to FATMEN & \\
  \quad\small\tt CALL FMSETC(CPL,NCPL,IRC*) & \pageref{FMSETC} \\
Declare selection matrix and options to FATMEN & \\
  \quad\small\tt CALL FMSETK(KEYM,NK,CHOPT,IRC*) & \pageref{FMSETK} \\
Modify user words & \\
  \quad\small\tt CALL FMMODU(PATH,UFORM,UVECT,UCOMM,CHOPT,IRC*) & \pageref{FMMODU} \\
Declare logical units to FATMEN & \\
  \quad\small\tt CALL FMSETU(LUN,NLUN,IRC*) & \pageref{FMSETU} \\
Get a free logical unit & \\
  \quad\small\tt CALL FMGLUN(LUN*,IRC*) & \pageref{FMGLUN} \\
Free a logical unit & \\
  \quad\small\tt CALL FMFLUN(LUN*,IRC*) & \pageref{FMFLUN} \\
Verify bank contents & \\
  \quad\small\tt CALL FMVERI(GENAM,LBANK,KEYS,CHOPT,IRC*) & \pageref{FMVERI} \\
Pack date and time. & \\
  \quad\small\tt CALL FMPKTM(IDATE,ITIME,IPACK*,IRC*) & \pageref{FMPKTM} \\
Unpack date and time. & \\
  \quad\small\tt CALL FMUPTM(IDATE*,ITIME*,IPACK,IRC*) & \pageref{FMUPTM} \\
Pack date and time for VAX format. & \\
  \quad\small\tt CALL FMPKVX(CHDATE,IDATE,ITIME,IPACK*,IRC*) & \pageref{FMPKVX} \\
Unpack date and time for VAX format. & \\
  \quad\small\tt CALL FMUPVX(CHDATE*,IDATE*,ITIME*,IPACK,IRC*) & \pageref{FMUPVX} \\
Return file names in specified directory & \\
  \quad\small\tt CALL FMFNMS(PATH,FILES*,KEYS*,NKEYS*,MAXKEY,IRC*) & \pageref{FMFNMS} \\
Return file names in directory structure & \\
  \quad\small\tt CALL FMLIST(PATH,FILES*,KEYS*,NFOUND,MAXFIL,IRC*) & \pageref{FMLIST} \\
Obtain names of subdirectories in specified tree & \\
  \quad\small\tt CALL FMTREE(PATH,SUBDIR*,NLEVEL,NFOUND*,MAXDIR,IRC*) & \pageref{FMTREE} \\
User routine to allocate new piece of media & \\
  \quad\small\tt CALL FUALLO(MEDIA,VSN*,VID*,IRC*) & \pageref{FUALLO} \\
Create a new FATMEN bank & \\
  \quad\small\tt CALL FMLIFT(GENAM,KEYS*,MEDIA,CHOPT,IRC*) & \pageref{FMLIFT} \\
Get the address of a FATMEN bank & \\
  \quad\small\tt CALL FMLINK(GENAM,LBANK*,CHOPT,IRC*) & \pageref{FMLINK} \\
Obtain volume characteristics & \\
  \quad\small\tt CALL FMQTMS(VID,LIB*,MODEL*,DENS*,MNTTYP*,LABTYP*,IRC*) & \pageref{FMQTMS} \\
\end{longtable}
 
\clearpage

\label{BANK-OFFSETS}
\extrarowheight2pt
%\arraycolsep3pt
\newlength{\Tonecol}
\setlength{\Tonecol}{\linewidth}
\addtolength{\Tonecol}{-\arraycolsep}
\addtolength{\Tonecol}{-\arrayrulewidth}
\newlength{\myla}\settowidth{\myla}{\bf Date and timexx}
\newlength{\mylb}\setlength{\mylb}{\linewidth}
\addtolength{\mylb}{-\myla}
\addtolength{\mylb}{-.21\linewidth}
\addtolength{\mylb}{-6\arraycolsep}
\addtolength{\mylb}{-4\arrayrulewidth}
\begin{longtable}{|p{\myla}|>{\tt}p{.21\linewidth}|p{\mylb}|}
\caption{Bank offsets and datatypes}                                     \\ 
\hline
\multicolumn{1}{|c}{\bf Function}    &
\multicolumn{1}{c}{\bf Data type}   &
\multicolumn{1}{c|}{\bf Description}                                     \\
\hline
\endfirsthead
\caption[]{Bank offsets and datatypes (cont.)}                           \\
\hline
\multicolumn{1}{|c}{\bf Function}    &
\multicolumn{1}{c}{\bf Data type}   &
\multicolumn{1}{c|}{\bf Description}                                     \\
\hline
\endhead
\hline
\endfoot
\bf Key Definitions               &
\tt MKSRFA (I*4)                  &
An integer assigned by the FATMEN system to permit differentiation between
entries with the same generic name.
It is not normally specified by the user, unless a particular entry is
required, e.g. in an {\tt RM} command.                                    \\
\cline{2-3}
& \tt MKFNFA (H*20)              &
The part of the generic name following the last slash, such as
{\tt RUN123} in the name {\tt //CERN/DELPHI/TEST/RUN123}                  \\
\cline{2-3}
& \tt MKCLFA (I*4)               &
The 'copy level', where 0=original, 1=copy, 2=copy of a copy etc.
Some experiments prefer to use the copy level to indicate the data
representation of the data pointed to by that entry, where {\tt 1=IEEE},
{\tt 2=IBM}, {\tt 3=VAX}, {\tt 4=byte-swapped IEEE} (e.g. DECstation),
{\tt 5=CRAY} \\
\cline{2-3}
& \tt MKLCFA (I*4)               &
The location code, which is used to find the nearest or most
suitable copy of a dataset. The location code corresponds to a site or LAN,
because it is only at this level that a meaningful choice can be made.
Thus it is only a first level selection, to filter out all
opies at a given location, such as CERN, RAL etc.                         \\
\cline{2-3}
& \tt MKMTFA (I*4)               &
The type of medium on which the dataset resides. \newline
{\tt 1=DISK, 2=3480, 3=3420, 4=8200 (Exabyte)}.                           \\
\hline
\bf Generic file description    &
\tt MFQNFA (H*256)$^{*+}$         &
This points to the fully qualified dataset name that the file has on
disk or tape. For VM files, the file name should be encoded in the
form {\tt <user.addr>fn.ft} where {\tt addr}, if omitted, defaults to 191.\\
\cline{2-3}
& \tt MHSNFA (H*8)$^{*+}$        &
The host on which the file resides, or through which it is accessed,
in the case of a tape file.                                               \\
\cline{2-3}
& \tt MCPLFA (I*4)$^{*+}$        &
The 'copy level', where 0=original, 1=copy, 2=copy of a copy etc.         \\
\cline{2-3}
& \tt MLOCFA (I*4)$^{*+}$        &
The location code, which is used to find the nearest or most
suitable copy of a dataset.The location code corresponds to a site or LAN,
because it is only at this level that a meaningful choice can be made.
Thus it is only a first level selection, to filter out all
copies at a given location, such as CERN, RAL etc.                        \\
\cline{2-3}
& \tt MMTPFA (I*4)$^{*+}$        &
The type of medium on which the dataset resides. \newline
{\tt 1=DISK, 2=3480, 3=3420, 4=8200 (Exabyte)}.                           \\
\hline
\bf Disk description             &
\tt MHSTFA (H*16)$^+$            &
The host type, e.g. IBM 3090.
This field is filled in by the FATMEN system.                             \\
\cline{2-3}
& \tt MHOSFA (H*12)$^+$          &
The host operating system, such as {\tt VM/XA CMS 5.5.}
This field is filled in by the FATMEN system.                             \\
\hline
\bf Tape description             &
\tt MVSNFA (H*8)$^*$             &
The {\bf volume serial number} or magnetically recorded label on the tape.\\
%\cline{2-3}
& \tt MVIDFA (H*8)$^*$           &
The {\bf visual identifier} or contents of the sticky label on the tape.  \\
\cline{2-3}
& \tt MVIPFA (I*4)               &
The VID prefix - internal representation.                                 \\
\cline{2-3}
& \tt MDENFA (I*4)$^+$           &
The density of the medium, such as 6250, 38000.
Normally, such details are retained by the Tape Management System.        \\
\cline{2-3}
& \tt MVSQFA (I*4)$^+$           &
The volume sequence number, only used for multi-volume files.
(Not yet supported).                                                      \\
\cline{2-3}
& \tt MFSQFA (I*4)$^*$           &
The file sequence number, for multi-file tapes.                           \\
\hline
\bf File description\footnotemark[1]
                                 &
\tt MSRDFA (I*4)                 & The start record in the file.          \\
\cline{2-3}
& \tt MERDFA (I*4)               & The end record in the file.            \\
\cline{2-3}
& \tt MSBLFA (I*4)               & The start block in the file.           \\
\cline{2-3}
& \tt MEBLFA (I*4)               & The end block in the file.             \\
\hline
\bf Logical description           &
\tt MFLFFA (H*4)$^{*+}$          &
The FATMEN defined file format.
The following definitions are recognised by the FATMEN system:
\begin{DLtt}{MMMM}
\item[FA]ZEBRA ASCII
\item[FZ]ZEBRA native
\item[FXN]ZEBRA native, exchange file format
\item[FX]ZEBRA exchange
\item[FFX]ZEBRA exchange, FORTRAN I/O
\item[RZ]ZEBRA RZ
\item[RX]ZEBRA RZ, exchange format
\item[EP]EPIO
\item[AS]normal ASCII
\item[DA]FORTRAN direct access dataset
\item[UN]unknown
\item[FPT]FPACK text
\item[FPS]FPACK sequential-access
\item[FPD]FPACK direct-access
\item[FPK]FPACK keyed-access
\item[FPO]FPACK ordered
\item[YBB]YBOS binary
\item[YBD]YBOS direct-access
\end{DLtt}
\vspace*{-\baselineskip}                                                   \\
\cline{2-3}
& \tt MFUTFA (H*4)               & The user defined file type.            \\
\hline
\bf Physical description         &
\tt MRFMFA (H*4)                 &
The record format, e.g. {\tt VBS, FB} etc.                                \\
\cline{2-3}
& \tt MRLNFA (I*4)               & The record length, in 4-byte words.    \\
\cline{2-3}
& \tt MBLNFA (I*4)               & The block length, in 4-byte words.     \\
%\cline{2-3}
& \tt MFSZFA (I*4)               &
The file size in Megabytes, rounded up to the next integer.
A Megabyte is defined as {\tt (1,024)**2} bytes.                          \\
\cline{2-3}
& \tt MUSCFA (I*4)               &
The number of times that the file has been accessed (use count).
A call to FMOPEN with the option W, or to FMMAKE sets the use count to 1.
A call to FMOPEN with the option R, or to FMFIND increments the count.    \\
\hline
\bf Date and time\footnotemark[2] &
  \tt MCRTFA (I*4)$^+$           & Creation date and time.                \\
\cline{2-3}
& \tt MCTTFA (I*4)$^+$           & Date and time of cataloging.           \\
\cline{2-3}
& \tt MLATFA (I*4)               &
Date and time of last access (when available).                            \\
\hline
\bf Creator information           &
 \tt MCURFA (H*8)$^+$            & The user name of the creator.          \\
\cline{2-3}
& \tt MCIDFA (H*8)$^+$           & The account
({\tt UUUGG} or {\tt UUU\$GG}) of the creator (CERNVM or VAX/VMS).
For Unix systems, the NFSID is used.                                      \\
\cline{2-3}
& \tt MCNIFA (H*8)$^+$           & The node on which the file was created.
This is not necessarily the node from which it was catalogued.            \\
\cline{2-3}
& \tt MCJIFA (H*8)$^+$           &
The job or process name which created the file.                           \\
\hline
\bf File protection               & \tt MFPRFA (I*4)                &
The file protection flag, currently unused.                               \\
\hline
\bf System area                   & \tt MSYWFA(10) (I*4)            &
Pointer to the first of ten system defined words
(reserved for future use).                                                \\
\hline
\bf User area                    & \tt MUSWFA(10) (I*4)            &
Pointer to the first of ten user defined words.                           \\
\cline{2-3}
& \tt MUCMFA (H*80)$^+$          & A user-defined comment.                \\
\hline
\multicolumn{3}{|p{\the\Tonecol}|}{
\small
\begin{flushleft}
\quad $^+$\ Items filled
in automatically by the routine \Rind{FMBOOK} or \Rind{FMLIFT} (See
pages~\pageref{FMBOOK} and \pageref{FMLIFT}).\\
\quad $^*$\ Mandatory items (defined as \Lit{NOT NULL} 
in the ORACLE/SQL tables).\\
All fields may be overwritten
by the user, but care must be exercised to ensure that the new
values are coded in the correct format.
In particular, for hollerith
values, the entire field must first be set to blanks to avoid problems
if the new value is shorter than the default setting.
The routines \Rind{FMPUTC} and \Rind{FMPUTI} (see on Page~\pageref{FMPUTC} and
on Page~\pageref{FMPUTI}) may be used to simplify bank modification.
See the routine on Page~\pageref{FMVERI} for information on how
to check the contents of a bank.\\[5mm]
{\bf Notes}\\
\quad\footnotesize$^1$\ The file description information will
normally be left as zero, or set to the first and last record and block
numbers. They could also be used to indicate a subset of a file.\\
\quad$^2$\ The date  and time fields are stored
packed, using routine \Rind{FMPKTM}. 
They can be unpacked using routine \Rind{FMUPTM}.
\end{flushleft}}
\end{longtable}
\footnotetext[1]{The file description information will
normally be left as zero, or set to the first and last record and block
numbers. They could also be used to indicate a subset of a file.}
\footnotetext[2]{The date  and time fields are stored
packed, using routine \Rind{FMPKTM}. 
They can be unpacked using routine \Rind{FMUPTM}.}

\clearpage

\subsection{Disk file name}
\index{disk filename}
\par
The disk file name should be entered in the format of the host
operating system. For VM/CMS systems, the convention
{\tt <user.address> filename.filetype} is used.
If {\tt address} is omitted, {\tt user} may then be any
valid entry in a {\tt GIMEUSER}, {\tt GIMEGRP} or {\tt GIMESYS} names file.
If not found in a GIME names file, the default address 191 will be used.
\par
An extract from a GIME names file is given below:
\begin{XMP}
:NICK.L3DSTS  :USERID.L3MAXI
              :ADDR.222
              :LINKMODE.RR
              :FILEMODE.E
\end{XMP}
Using this example, files such as {\tt<L3DSTS>RUN1.DST} would be assumed
to be on the 222 disk of user L3MAXI.
\par
For VAX/VMS systems, the full file name, including disk and directory
names should be entered.
\subsection{Disk files and VAXclusters}
\index{VAXcluster}
\index{cluster}
\index{cluster alias}
\index{DECnet cluster alias}
\par
If the FATMEN software finds that the current node is in the
same cluster as the node specified in the FATMEN catalogue
and the disk specified is available, it will treat the entry
as if the file were on the current node.
\begin{XMPt}{Example of valid disk file names for the cluster VXCERN}
*     Create a valid entry in an existing FATMEN bank
      CALL UCTOH('VXCRNA',IQ(LFAT+MHSNFA),4,6)
*     or we could use the VAXcluster alias rather than current node name
      CALL UCTOH('VXCERN',IQ(LFAT+MHSNFA),4,6)
*     Create a valid disk file name
*     Note that logical name for disk is unique and consistant HEP-wide!
      CALL UCTOH('DELPHI$DK123:<DELPROD.MCDSTS>RUN567',
     +            IQ(LFAT+MFQNFA,4,35)
\end{XMPt}

\newpage

\subsection{Disk files and DFS or NFS}
\index{DFS}
\index{NFS}

The FATMEN system can use DFS\cite{bib-DFS} or
NFS\cite{bib-NFS} to access a remote disk file,
if the disk or file system on which it resides is mounted locally.
Users are strongly recommended to adopt a convention for DFS and NFS
names, so that the same name is used throughout a LAN. For DFS it is
recommended that the disk in question have the same logical name
on the system to which it is directly attached as on those
where it is mounted via DFS. Thus, a disk on the central cluster
with logical name VXCERN\$DISK999 should be mounted remotely via
DFS with the same logical name. If the FATMEN system finds a DFS
device of the corresponding name on the local system, it will
assume that the physical device to which it points is indeed the correct
one. {\bf N.B. the FATMEN software currently considers only the
'system' logical name table ({\tt LNM\$SYSTEM\_TABLE}).}
\begin{XMPt}{Example of mounting a remote disk via DFS}
$!Mount the VXCERN disk DELPHI$DK123
$!The 'access-point' is defined on VXCERN and points
$!to the disk in the previous example. This will allow
$!users on the current node to access data on this disk
$!that is catalogued in FATMEN transparently.
$RUN SYS$SYSTEM:DFS$CONTROL
DFS>mount access-point DELPHI$DK123 /SYSTEM
DFS>exit
\end{XMPt}
\par
If the dataset name in the FATMEN entry (MFQNFA) starts with a \$,
then on Unix systems FATMEN will assume that the text that follows
up to the first slash character is an environmental variable and 
attempt to translate it. An example is given below.
\begin{XMP}
 FM> ls C01 -gn # List full generic name and corresponding dataset name
 
 //CERN/OPAL/DDST/PASS4/FYZ1/P19R1973/C01
 Fileid:      $OPALDATA/fyz1/p19r1973.c01
 
 Files:    1
 FM>quit

[zfatal] > echo $OPALDATA                      
/u/ws/ddst
[zfatal] >
[zfatal] >
[zfatal] > telnet shift1

...

shift1 [2] printenv OPALDATA
/user/ws/data
shift1 [3]
\end{XMP}
\subsection{Location code}
\index{location code}
\par
This is an integer code indicating the LAN or site where the
data resides. Currently, the only supported LAN is CERN, for which
the location code is defined as 1. This code is used by the FATMEN
system for a first pass selection of the best copy of a data set.
The correspondance between the location code and the LAN will be
stored in the FATMEN database itself.
\subsection{Username (MCURFA)}
\index{username (MCURFA)}
\par
This is obtained through QUERY USERID on VM/CMS systems, the
\$GETJPI system service on VAX/VMS and the getpwuid function on Unix.
\subsection{Jobname (MCJIFA)}
\index{jobname (MCJIFA)}
\par
This is obtained through JOBNAM (CERN Program Library Entry Z100)
on VM/CMS systems, the \$GETQUI system service on VAX/VMS and the
getpwuid function on Unix.
\subsection{Account (MCIDFA)}
\index{account (MCIDFA)}
\par
This is obtained through the QUERY ACCOUNT command on VM/CMS, the \$GETUAI
system service on VAX/VMS or the getuid function on Unix.
\subsection{Host name (MHSTFA, MCNIFA)}
\index{host name (MHSTFA, MCNIFA)}
\par
This is obtained through the QUERY USERID command on VM/CMS and the
\$GETSYI system service on VAX/VMS.
\subsection{Host type (MHSTFA)}
\index{host type (MHSTFA)}
\par
This is obtained through the QUERY CMSLEVEL command on VM/CMS and
the \$GETSYI system service on VAX.
\subsection{Host Operating System (MHOSFA)}
\index{host operating system (MHOSFA)}
\par
This is obtained using the CP QUERY USERID command on VM/CMS
systems and the \$GETSYI system service on VAX/VMS machines.


\end{appendix}
%  ==================== Backmaterial ==============================
\begin{theindex}

  \item access data, 99
  \item accessing existing tape data, 102
  \item account (MCIDFA), 88
  \item ADDDISK
    \subitem defined, 89
  \item ADDTAPE
    \subitem defined, 89
  \item alias, 2
  \item ALLOCATE
    \subitem defined, 88

  \indexspace

  \item bank offsets, 84
  \item bitnet, 112

  \indexspace

  \item catalogue, 4, 13, 112
  \item CD
    \subitem defined, 89
  \item CDF Command Definition File, 1
  \item Changes, 39
  \item CLI, 5, 98
  \item CLR
    \subitem defined, 90
  \item cluster, 87
  \item cluster alias, 87
  \item command
    \subitem ADDDISK, 89
    \subitem ADDTAPE, 89
    \subitem ALLOCATE, 88
    \subitem CD, 89
    \subitem CLR, 90
    \subitem COPY, 90
    \subitem CP, 90
    \subitem DIR, 91
    \subitem DUMP, 91
    \subitem END, 91
    \subitem EXIT, 91
    \subitem FC, 91
    \subitem FIND, 91
    \subitem GIME, 91
    \subitem INIT, 91
    \subitem LD, 92
    \subitem LOCK, 92
    \subitem LOGLEVEL, 92
    \subitem MAKE, 93
    \subitem MEDIA, 93
    \subitem MKDIR, 94
    \subitem MV, 94
    \subitem PWD, 94
    \subitem QUIT, 94
    \subitem RM, 94
    \subitem RMDIR, 95
    \subitem RMTREE, 95
    \subitem SEARCH, 95
    \subitem SETCOPYLEVEL, 97
    \subitem SETLOCATION, 98
    \subitem SETMEDIATYPE, 98
    \subitem SHOWCOPYLEVEL, 98
    \subitem SHOWLOCATION, 98
    \subitem SHOWMEDIATYPE, 98
    \subitem TAG, 96
    \subitem TOUCH, 97
    \subitem TREE, 97
    \subitem UPDATE, 97
    \subitem VERSION, 97
    \subitem VIEW, 97
  \item command abbreviation, 1
  \item Command Definition File (CDF), 1
  \item command line interface, 5, 98
  \item continuation lines, 99
  \item COPY
    \subitem defined, 90
  \item Copy level, 72
  \item CP
    \subitem defined, 90
  \item creating new tape data, 102
  \item CSPACK, 2, 6

  \indexspace

  \item data access, 99
  \item data staging, 102
  \item DBUPTM, 74
  \item DECnet, 112
  \item DECnet cluster alias, 87
  \item DFS, 87
  \item DIR
    \subitem defined, 91
  \item disk filename, 86
  \item distribution of updates, 112
  \item DROP, 108
  \item DUMP
    \subitem defined, 91
  \item DUMPTAPE, 91

  \indexspace

  \item END
    \subitem defined, 91
  \item EPIO, 7
  \item errors, 39
  \item EXEC, 107
  \item existing data, 102
  \item EXIT
    \subitem defined, 91

  \indexspace

  \item FATMEN Keys, 84
  \item FC
    \subitem defined, 91
  \item FIND
    \subitem defined, 91
  \item FM, 107
  \item FMADDD
    \subitem defined, 41
  \item FMADDT
    \subitem defined, 41
  \item FMALLO
    \subitem defined, 64
  \item FMATCH
    \subitem defined, 60
  \item FMBOOK
    \subitem defined, 47
    \subitem referenced, 135
  \item FMCLOS
    \subitem defined, 53
  \item FMCOMP
    \subitem defined, 62
  \item FMCOPY
    \subitem defined, 53
  \item FMEDIA
    \subitem defined, 69
  \item FMEND
    \subitem defined, 43
  \item FMEXST
    \subitem defined, 54
  \item FMFEND
    \subitem defined, 40
  \item FMFILC
    \subitem defined, 56
  \item FMFILE
    \subitem defined, 40
  \item FMFILL
    \subitem defined, 48
  \item FMFIND
    \subitem defined, 51
  \item FMFLUN
    \subitem defined, 73
  \item FMFNMS
    \subitem defined, 75
  \item FMGET
    \subitem defined, 45
  \item FMGETC
    \subitem defined, 49
  \item FMGETI
    \subitem defined, 51
  \item FMGETK
    \subitem defined, 46
  \item FMGETV
    \subitem defined, 50
  \item FMGLUN
    \subitem defined, 73
  \item FMINIT
    \subitem defined, 43
  \item FMKDIR
    \subitem defined, 48
  \item FMLDIR
    \subitem defined, 58
  \item FMLFIL
    \subitem defined, 58
  \item FMLIFT
    \subitem defined, 78
    \subitem referenced, 135
  \item FMLINK
    \subitem defined, 78
  \item FMLIST
    \subitem defined, 76
  \item FMLOGL
    \subitem defined, 44
  \item FMLOOP
    \subitem defined, 57
  \item FMLS
    \subitem defined, 54
  \item FMMAKE
    \subitem defined, 52
  \item FMMANY
    \subitem defined, 60
  \item FMMOD
    \subitem defined, 46
  \item FMMODU
    \subitem defined, 72
  \item FMOPEN
    \subitem defined, 52
  \item FMPEEK
    \subitem defined, 42
  \item FMPKEY
    \subitem defined, 61
  \item FMPKTM
    \subitem defined, 74
    \subitem referenced, 135
  \item FMPKVX
    \subitem defined, 74
  \item FMPOKE
    \subitem defined, 42
  \item FMPOOL
    \subitem defined, 64
  \item FMPURG
    \subitem defined, 44
  \item FMPUT
    \subitem defined, 46
  \item FMPUTC
    \subitem defined, 49
    \subitem referenced, 135
  \item FMPUTI
    \subitem defined, 50
    \subitem referenced, 135
  \item FMPUTV
    \subitem defined, 50
  \item FMQMED
    \subitem defined, 67
  \item FMQTMS
    \subitem defined, 78
  \item FMQVOL
    \subitem defined, 65
  \item FMRM
    \subitem defined, 47
  \item FMSCAN
    \subitem defined, 56
  \item FMSELB
    \subitem defined, 61
  \item FMSELK
    \subitem defined, 61
  \item FMSETC
    \subitem defined, 72
  \item FMSETL
    \subitem defined, 72
  \item FMSETM
    \subitem defined, 72
  \item FMSETU
    \subitem defined, 73
  \item FMSHOW
    \subitem defined, 55
  \item FMSORT
    \subitem defined, 59
  \item FMSTRT
    \subitem defined, 39
  \item FMTAGS
    \subitem defined, 71
  \item FMTREE
    \subitem defined, 76
  \item FMUPDT
    \subitem defined, 44
  \item FMUPRT
    \subitem defined, 63
  \item FMUPTM
    \subitem defined, 74
    \subitem referenced, 135
  \item FMUPVX
    \subitem defined, 75
  \item FMUSEL
    \subitem defined, 63
  \item FMVERI
    \subitem defined, 73
  \item FORTRAN callable interface, 5
  \item FORTRAN program, 79
  \item forwarding updates, 8
  \item FUALLO
    \subitem defined, 77
  \item FZ, 8

  \indexspace

  \item generic name, 4
  \item GIME, 108
    \subitem defined, 91
  \item group, 120

  \indexspace

  \item HBOOK, 107
  \item HEPVM, 108
  \item History, 39
  \item history file, 1, 2
  \item host name (MHSTFA, MCNIFA), 88
  \item host operating system (MHOSFA), 88
  \item host type (MHSTFA), 88

  \indexspace

  \item INIT
    \subitem defined, 91
  \item IQUEST, 39

  \indexspace

  \item jobname (MCJIFA), 88

  \indexspace

  \item Keys, 61
  \item keys, 84
  \item KUIP, 1, 88, 99, 127

  \indexspace

  \item LD
    \subitem defined, 92
  \item Library, 107
  \item Link area, 28, 45, 46
  \item Location code, 72
  \item location code, 88
  \item LOCK
    \subitem defined, 92
  \item Logical unit, 73
  \item Logical units, 73
  \item LOGLEVEL
    \subitem defined, 92
  \item LS
    \subitem defined, 92

  \indexspace

  \item macro, 1, 2
  \item MAKE
    \subitem defined, 93
  \item Manual, 67
  \item MEDIA
    \subitem defined, 93
  \item Media type, 72
  \item menu, 2
  \item MKDIR
    \subitem defined, 94
  \item Modifying the FATMEN shell, 127
  \item MV
    \subitem defined, 94
  \item MZWIPE, 107

  \indexspace

  \item NAMEFD, 112
  \item NAMEFIND, 112
  \item names file, 112
  \item new data, 102
  \item NFS, 6, 87
  \item NOWAIT, 51

  \indexspace

  \item online help, 1
  \item ORACLE, 2, 8, 120

  \indexspace

  \item PAM, 107
  \item parameter, 2
  \item parameters, 84
  \item PWD
    \subitem defined, 94

  \indexspace

  \item QUIT
    \subitem defined, 94

  \indexspace

  \item remote catalogues, 112
  \item remote servers, 8
  \item Return codes, 39
  \item REXX, 108
  \item RM
    \subitem defined, 94
  \item RMDIR
    \subitem defined, 95
  \item RMTREE
    \subitem defined, 95
  \item Robot, 67
  \item RXLOCFN, 108
  \item RZ, 8

  \indexspace

  \item SEARCH
    \subitem defined, 95
  \item server, 107
  \item servers, 112
  \item SETCOPYLEVEL
    \subitem defined, 97
  \item SETLOCATION
    \subitem defined, 98
  \item SETMEDIATYPE
    \subitem defined, 98
  \item shell, 98, 107, 127
  \item SHIFT, 108
  \item SHOWCOPYLEVEL
    \subitem defined, 98
  \item SHOWLOCATION
    \subitem defined, 98
  \item SHOWMEDIATYPE
    \subitem defined, 98
  \item SQL/DS, 120
  \item STAGE, 5
  \item staging data, 102
  \item starting the FATMEN shell, 98
  \item SYSREQ, 2, 5, 102

  \indexspace

  \item TAG
    \subitem defined, 96
  \item Tags, 71
  \item Tailoring the FATMEN shell, 127
  \item tape, 102
  \item Tape Management System, 2
  \item TAPEDUMP, 91
  \item TCP/IP, 5
  \item TCPIP, 112
  \item TCPREQ, 2, 5
  \item throng, 118
  \item TMS, 2, 5, 65, 78, 92, 102, 107
  \item TOUCH
    \subitem defined, 97
  \item TREE
    \subitem defined, 97

  \indexspace

  \item underlining, i
  \item UPDATE
    \subitem defined, 97
  \item updates, 112
  \item User exits, 63
  \item user input, i
  \item username (MCURFA), 88

  \indexspace

  \item VAXcluster, 87
  \item VAXTAP, 108, 117
  \item VERSION
    \subitem defined, 97
  \item VID, 64, 77
  \item VIEW
    \subitem defined, 97
  \item VMBATCH, 108
  \item VMTAPE, 107, 108
  \item VSN, 64, 77

  \indexspace

  \item WRITE-LOCK, 92

  \indexspace

  \item XTAPE, 91

  \indexspace

  \item ZEBRA, 1, 7, 8, 39, 84, 107
  \item ZEBRA server, 6, 123
  \item ZS ZEBRA server, 6, 123

\end{theindex}

%!--  FATMEN  User Guide  -  Bibliography file -->
%!--  Last Mod. 22 Nov. 12.00   jds+mg         -->
%!--                                           -->
\begin{thebibliography}{99}
\bibitem{FATMEN}
Shiers, J.D.
1991
FATMEN - Distributed File and Tape Management 
CERN Program Library Q123
\bibitem{KUIP}
Brun R., Zanarini P.
1988
KUIP - Kit for a User Interface Package
CERN Program Library I202
\bibitem{PATCHY}
Klein H.J., Zoll J.
1988
PATCHY Reference Manual
CERN Program Library L400
\bibitem{TMS}
Cass, A.
1991
CERN TMS Users Guide
\bibitem{POSTSCR}
Adobe Systems Incorporated
1988
PostScript Language - Reference Manual
Addison-Wesley Publishing Company
\bibitem{LaTeX}
\LaTeX, document preparation system,
Leslie Lamport, Addison and Wesley, 1986
\bibitem{ZEBRA}
Brun R., Zoll J.
1989
ZEBRA - Data Structure Management System
CERN Program Library Q100
\bibitem{ZEBRAFZ}
Zoll J.
1989
ZEBRA - Reference Manual - FZ Sequential I/O
CERN Program Library Q100
\bibitem{CMZ}
Brun M., Brun R., Rademakers F.
1989
CMZ - Code Management system using Zebra
Proceedings of Conference on Computing in High Energy Physics, Oxford April 1989
\bibitem{TCPAW1}
Segal B.
The TCPAW Package
1989
Internal DD report. To be published
\bibitem{TCPAW2}
Brun R.
Segal B.
A distributed Physics Analysis workbench
1989
CERN/DD report. To be published
\bibitem{ORACLE}
Oracle Corporation
ORACLE Overview and Introduction to SQL
1985
ORACLE Corporation
\bibitem{CSPACK}
CERN
CN/AS, CN/SW
CERN Program Library entry Q124
1991
\bibitem{FATREP}
CERN
C.Curran et al.
The FATMEN Report - File and Tape Management: Experimental Needs
1989
\bibitem{MUSCLE}
D.O.Williams et al.
The Muscle Report DD/88/1
CERN
1988
\bibitem{NGB}
J.J. Thresher et al.
Computing at CERN in the 1990s
CERN
1989
\bibitem{EPIO}
H. Grote, I. Mclaren
EPIO \ Experimental Physics Input Output Package
CERN
1981
\bibitem{DFS}
Digital Equipment Corporation
VAX Distributed File Service
Digital Equipment Corporation
1987
\bibitem{NFS}
Sun Microsystems
Network File System Version 2
Sun Microsystems
1987
\bibitem{DELOFB}
L. Palermo, N. van Eijndhoven
Delphi Offline Book-keeping Users' Guide
DELPHI
1989
\bibitem{HBOOK}
Brun R., Lienart D.
1988
HBOOK User Guide - Version 4
CERN Program Library Y250
\bibitem{PAW}
Brun R., Couet O., Vandoni C., Zanarini P.
1989
PAW Long Writeup
CERN Program Library Q121
\end{thebibliography}

\end{document}
