% -*- LaTeX -*-

\documentstyle[11pt,bezier,twoside,a4p,makeidx]{report}

\def\documentlabel#1{\gdef\@documentlabel{#1}}
\makeindex
\gdef\@documentlabel{\tt Preliminary draft}
 
\documentlabel{CERN/SL/90-13 (AP) \\ (Rev. 4)}

% *******************************************************************
 
\pagestyle{headings}
 
% heading for the \chapter command.
\makeatletter
\def\@makechapterhead#1{{
   \parindent 0pt\raggedright\LARGE \bf
   \ifnum \c@secnumdepth >\m@ne
      \@chapapp{} \thechapter.\ \ \fi
   #1\par\nobreak\vskip 20pt
}}
 
% heading for the \chapter* command.
\def\@makeschapterhead#1{{
   \parindent 0pt\raggedright\LARGE \bf
   #1\par\nobreak\vskip 20pt
}}
 
% titles for the lists of figures and tables:
\def\listoffigures{
   \section*{List of Figures}
   \@starttoc{lof}
}
\def\listoftables{
   \section*{List of Tables}
   \@starttoc{lot}
}
 
\makeatother
 
% description lists:
\def\mylist{\list{}{
   \setlength{\labelwidth}{2.3cm}
   \setlength{\leftmargin}{2.5cm}
   \let\makelabel\mylabel}
}
 
\let\endmylist\endlist
 
\def\mylabel#1{#1\hfill}
 
\def\eqref#1{Eqn.~\ref{#1}}
\def\epem{e${}^+\)-e${}^-$}
 
% automatic indexing of keywords:
\def\ttitem#1{\item[{\tt #1}]\index{#1@{\tt #1}}}
\def\ttindex#1{{\tt #1}\index{#1@{\tt #1}}}
\def\ttnindex#1{\index{#1@{\tt #1}}}
 
% MAD command specifications:
\newcommand{\mybox}[1]{
   \begin{quote}
      \tt
      \fbox{
         \begin{minipage}{0.95\textwidth}
            \begin{tabbing}
               #1
            \end{tabbing}
         \end{minipage}
      }
   \end{quote}
}
 
% MAD command examples:
\newcommand{\myxmp}[1]{
   \begin{quote}
      \tt
      \begin{tabbing}
         #1
      \end{tabbing}
   \end{quote}
}

% Some Mathematical Symbols and Operators
\def\sign{\mathop{\mathrm{sign}}\nolimits}
\def\Tr{\mathop{\mathrm{Tr}}\nolimits}
\def\ad{{\sl Ad\/}}
\def\adphsp{$\{{:}f{:}\}$}
\def\eqdef{\stackrel{def}{=}}
\def\gappeq{\stackrel{>}{\approx}}
\def\lieop#1{{:}{#1}{:}}
\def\lietran#1{e^{{:}{#1}{:}}}
\def\map#1{$\cal #1$}
\def\numclass#1{\mathrm\bf{#1}}
\def\order{order}
\def\phsp{\{Z\}}
\def\pbkt#1#2{\sum_{k=1}^3\left(
  {\partial #1\over\partial q_k}{\partial #2\over\partial p_k} -
  {\partial #1\over\partial p_k}{\partial #2\over\partial q_k}
\right)}

\def\aux{\left(p_x^2+p_y^2+\frac{p_t^2}{\beta^2\gamma^2}\right)}
\def\solaux{\left((p_x+ky)^2+(p_y-kx)^2+\frac{p_t^2}{\beta^2\gamma^2}\right)}

\def\vbar{\overline{v}}
\def\wbar{\overline{w}}

% `array' environment using display style
\long\def\eqarray#1{\vcenter{\let\\=\cr\openup1\jot
  \halign{&\strut$\,\displaystyle{##}\;$\hfil\crcr#1\crcr}}}

% `array' environment using display style for matrices
\long\def\myarray#1{\null\vcenter{\let\\=\cr\openup1\jot
  \halign{&\strut\hfil$\>\displaystyle{##}\>$\hfil\crcr#1\crcr}}}

% left align displayed equations
\def\[#1\]{$$\indent\leftline{$\displaystyle{#1}$}\hss$$}
 
\tabskip=0pt plus 1fill
\begin{document}
\setlength{\evensidemargin}{\oddsidemargin}
\pagenumbering{roman}
\begin{titlepage}
\begin{center}\normalsize
EUROPEAN ORGANISATION FOR NUCLEAR RESEARCH
\end{center}
\vskip 0.7cm
\begin{flushright}
\@documentlabel \\                   % document label
\end{flushright}
\vskip 2.3cm
\begin{center}\LARGE                 % document title
{\bf The MAD Program} \\
(Methodical Accelerator Design) \\
Version 8.13/8 \\
{\bf User's Reference Manual}
\end{center}
\vskip 1.5em
\begin{center}                       % authors
Hans Grote \\
F. Christoph Iselin
\vskip 2em
{\large Abstract}
\end{center}
\begin{quotation}
MAD is a tool for charged-particle optics in
alternating-gradient accelerators and beam lines.
It can handle very large and very small accelerators
and solves various problems on such machines.
\par Features include:
Linear lattice parameter calculation,
linear lattice matching,
transfer matrix matching,
survey calculations,
error definitions,
closed orbit correction,
particle tracking,
chromatic effects and resonances,
file services,
subroutines,
intra-beam scattering,
Lie-algebraic analysis,
debugging services.
Most of the Lie-algebraic algorithms used have been developed
at Maryland University by Alex Dragt and his collaborators.
\par The present revision of this manual is valid for Version 8.13 of MAD.
It contains the following enhancements:
Option to allow additive error definitions and corrector settings,
option to save machine structure as a line definition,
COMMENT command,
improved synchrotron radiation
(separate switches in tracking for damping and quantum effects),
correct rendering of tunes for large momentum deviation,
generation of track tables in selected observation points.
\end{quotation}
\vfill
\begin{center}
Geneva, Switzerland \\
\today
\end{center}
\end{titlepage}
\begin{center}
The MAD program contains the following copyright note:
\end{center}
\vspace{1em}
\framebox[\textwidth]{
\begin{minipage}{0.95\textwidth}\tt
\vspace{1em}
\begin{center}
CERN \\
\vspace{1em}
EUROPEAN ORGANISATION FOR NUCLEAR RESEARCH
\end{center}
\vspace{1em}
\thispagestyle{empty}
\begin{tabular}{ll}
Program name: &MAD --- Methodical Accelerator Design \\
 \\
CERN program library entry:&T5001 \\
 \\
Authors or contacts:&Hans Grote, F. Christoph Iselin \\
                    &SL Division \\
                    &CERN \\
                    &CH-1211 GENEVA 23 \\
                    &SWITZERLAND \\
                    &Tel. [0041] (022) 767 36 57 \\
                    &FCI at CERNVM.BITNET \\
\end{tabular}
\vspace{1em}
\begin{raggedright}\par
Copyright CERN, Geneva 1990 - Copyright and any other
appropriate legal protection of this computer program and
associated documentation reserved in all countries of the
world.
\vspace{1cm}\par
Organisations collaborating with CERN
may receive this program and
documentation freely and without charge.
\vspace{1cm}\par
CERN undertakes no obligation for the maintenance of this
program, nor responsibility for its correctness,
and accepts no liability whatsoever resulting from its use.
\vspace{1cm}\par
Program and documentation are provided solely for the use of
the organisation to which they are distributed.
\vspace{1cm}\par
This program may not be copied or otherwise distributed
without permission. This message must be retained on this and
any other authorised copies.
\vspace{1cm}\par
The material cannot be sold. CERN should be given credit in
all references.
\end{raggedright}
\vfill
\end{minipage}}
\clearpage
\setcounter{page}{1}
\chapter*{Preface}
\par The MAD framework should make it easy to add new features
in the form of program modules.
The authors of MAD hope that such modules will also be contributed
and documented by others.
Future updates to this manual will be published in the form of
complete chapters which either replace existing ones,
or document new features.
As required, the table of contents and the indices will be revised.
The contributions of other authors are acknowledged in the relevant
chapters.
\par Misprints and obscurity are almost inevitable in a manual
of this size.
Comments from readers are therefore most welcome.
They may also be sent to one of the following BITNET addresses:
\begin{verbatim}
FCI   at CERNVM.CERN.CH.bitnet
HANSG at CERNVM.CERN.CH.bitnet
KEIL  at CERNVM.CERN.CH.bitnet
\end{verbatim}
\cleardoublepage

\tableofcontents
\listoffigures
\listoftables

\cleardoublepage
 
\pagenumbering{arabic}
\chapter{Conventions}
\label{S-CONVENT}
\index{conventions}
 
\section{Reference System}
\label{S-REFER}
\index{reference!system}
\index{reference!orbit}
\index{local coordinates}
\index{coordinates!local}
The accelerator and/or beam line to be studied is described as
a sequence of beam elements placed sequentially along a reference
orbit.
The reference orbit is the path of a charged particle having the
central design momentum of the accelerator through idealised magnets
with no fringe fields (see Figure~\ref{F-REF}).
 
\begin{figure}[]
\centering
\setlength{\unitlength}{1pt}
\begin{picture}(400,250)
% axes
\thicklines
\put(180,40){\vector(0,1){150}}
\put(190,180){\makebox(0,0){\(y\)}}
\put(280,0){\vector(-1,1){160}}
\put(120,150){\makebox(0,0){\(x\)}}
\thinlines
\put(180,100){\circle*{4}}
% coordinates
\put(150,130){\line(0,1){60}}
\put(180,160){\line(-1,1){30}}
\put(150,130){\circle*{4}}
\put(150,190){\circle*{4}}
\put(180,160){\circle*{4}}
% radii of curvature and centre
\put(280,0){\vector(-3,1){172}}
\put(160,30){\makebox(0,0){\(\rho\)}}
\put(280,0){\vector(0,1){142}}
\put(290,60){\makebox(0,0){\(\rho\)}}
\put(280,0){\circle*{4}}
\put(295,15){\vector(-1,-1){12}}
\put(295,15){\makebox(0,0)[bl]{\shortstack{centre of \\curvature}}}
% actual orbit
\thicklines
\bezier{150}(0,100)(75,165)(150,190)
\put(150,190){\vector(3,1){18}}
\put(159,200){\makebox(0,0)[br]{\(d\vec r\)}}
\bezier{150}(150,190)(240,220)(320,220)
\put(320,220){\vector(1,0){4}}
\put(80,180){\vector(1,-1){12}}
\put(80,180){\makebox(0,0)[br]{\shortstack{actual \\orbit}}}
% actual orbit
\bezier{150}(40,0)(100,60)(180,100)
\bezier{150}(180,100)(260,140)(340,160)
\put(340,160){\vector(4,1){4}}
\put(320,165){\makebox{\(s\)}}
\put(320,135){\vector(-1,1){12}}
\put(320,135){\makebox(0,0)[tl]{\shortstack{reference \\orbit}}}
\end{picture}
\caption{Local Reference System}
\label{F-REF}
\end{figure}
 
The reference orbit\index{orbit!reference} consists of a series of
straight line segments and circular arcs.
It is defined under the assumption that all elements are
perfectly aligned.
The accompanying tripod of the reference orbit spans
a local curvilinear right handed coordinate system \((x, y, s)\).
The local \(s\)-axis is the tangent to the reference orbit.
The two other axes are perpendicular to the reference orbit and
are labelled \(x\) (in the bend plane)
and \(y\) (perpendicular to the bend plane).
 
\section{Closed Orbit}
\label{S-CLORB}
\index{closed orbit}\index{orbit!closed}
\index{energy!error}
\index{momentum!error}
\index{local!reference}
Due to various errors like misalignment errors, field errors,
fringe fields etc.,
the closed orbit does not coincide with the reference orbit.
It also changes with the momentum error.
The closed orbit is described with respect to the reference orbit,
using the local reference system \((x, y, s)\).
It is evaluated including any nonlinear effects.
\par MAD also computes the betatron and synchrotron oscillations
with respect to the closed orbit.
Results are given in the local \((x, y, s)\)-system defined
by the reference orbit.
 
\section{Global Reference System}
\label{S-LAYOUT}
\index{global!reference}
\index{global!coordinates}
\index{coordinates!global}
\index{local!origin}
The reference orbit of the accelerator is uniquely defined by the
sequence of physical elements.
The local reference system \((x, y, s)\) may thus be referred
to a global Cartesian coordinate system \((X, Y, Z)\)
(see Figure~\ref{F-GLOB}).
The positions between beam elements are numbered \(0,\ldots,i,\ldots,n\).
The local reference system \((x_{i}, y_{i}, z_{i})\)
at position \(i\),
i.e. the displacement and direction of the reference orbit
with respect to the system \((X, Y, Z)\) are defined by
three displacements \((X_{i}, Y_{i}, Z_{i})\)
and three angles \((\theta_{i}, \phi_{i}, \psi_{i})\)
The above quantities are defined more precisely as follows:
 
\begin{figure}[ht]%                                         1.2
\centering
\setlength{\unitlength}{1pt}
\begin{picture}(400,270)
% global axes
\thicklines
\put(20,150){\vector(2,-1){280}}
\put(290,35){\makebox(0,0){\(Z\)}}
\put(20,100){\vector(3,1){360}}
\put(370,200){\makebox(0,0){\(X\)}}
\put(80,0){\vector(0,1){270}}
\put(70,240){\makebox(0,0){\(Y\)}}
%local axes
\put(133.3,0){\vector(1,3){90}}
\put(213.3,260){\makebox(0,0){\(x\)}}
\put(300,150){\vector(-2,1){180}}
\put(150,215){\makebox(0,0){\(y\)}}
\put(0,100){\vector(2,1){270}}
\put(260,240){\makebox(0,0){\(s\)}}
% projection of s onto ZX
\thinlines
\put(80,120){\circle*{4}}
\put(200,200){\circle*{4}}
\put(0,110){\line(1,0){290}}
\put(300,110){\makebox(0,0)[l]{\shortstack{projection of \(s\) \\
onto \(ZX\)-plane}}}
\put(20,110){\circle*{4}}
\put(50,110){\circle*{4}}
\put(100,110){\circle*{4}}
% displacement of local system
\put(140,140){\line(2,-1){60}}
\put(160,128){\makebox(0,0)[tr]{\(Z\)}}
\put(140,90){\line(3,1){60}}
\put(190,100){\makebox(0,0)[tl]{\(X\)}}
\put(200,110){\line(0,1){90}}
\put(205,140){\makebox(0,0)[l]{\(Y\)}}
\put(140,90){\circle*{4}}
\put(140,140){\circle*{4}}
\put(200,110){\circle*{4}}
% intersection of xy and ZX
\put(130,0){\line(1,1){230}}
\put(135,5){\circle*{4}}
\put(193.3,63.3){\circle*{4}}
\put(240,110){\circle*{4}}
\put(286.7,156.7){\circle*{4}}
\put(335,205){\circle*{4}}
\thicklines
\put(180,20){\vector(-1,1){12}}
\put(180,20){\makebox(0,0)[tl]{\shortstack{intersection of \\
\(xy\) and \(ZX\) planes}}}
% reference orbit
\bezier{80}(140,150)(170,185)(200,200)
\bezier{80}(200,200)(230,215)(260,220)
\put(260,220){\makebox(0,0)[l]{\shortstack{reference \\orbit}}}
% roll angle
\bezier{30}(160,30)(160,40)(150,50)
\put(152,48){\vector(-1,1){2}}
\put(150,30){\makebox(0,0){\(\psi\)}}
\put(140,30){\makebox(0,0)[br]{roll angle}}
% pitch angle
\bezier{20}(60,110)(60,120)(55,125)
\put(57,123){\vector(-1,2){2}}
\put(50,118){\makebox(0,0){\(\phi\)}}
\put(40,125){\makebox(0,0)[br]{pitch angle}}
% azimuth
\bezier{20}(130,95)(140,100)(140,110)
\put(140,105){\vector(0,1){5}}
\put(130,105){\makebox(0,0){\(\theta\)}}
\put(115,95){\makebox(0,0)[t]{azimuth}}
\end{picture}
\caption{Global Reference System}
\label{F-GLOB}
\end{figure}
 
\begin{mylist}
\ttitem{X}
Displacement of the local origin in \(X\)-direction.

\ttitem{Y}
Displacement of the local origin in \(Y\)-direction.

\ttitem{Z}
Displacement of the local origin in \(Z\)-direction.

\ttitem[THETA]
\index{rotation angle}
\index{angle!rotation}
\index{azimuth}
Angle of rotation (azimuth) about the global \(Y\)-axis,
between the global \(Z\)-axis and the projection
of the reference orbit onto the \((Z, X)\)-plane.
A positive angle \(\theta\) forms a right-hand screw with the
\(Y\)-axis.

\ttitem[PHI]
\index{elevation angle}
Elevation angle,
i.e. the angle between the reference orbit and its projection
onto the \((Z, X)\)-plane.
A positive angle \(\phi\) correspond to increasing \(Y\).
If only horizontal bends are present,
the reference orbit remains in the \((Z, X)\)-plane.
In this case \(\phi\) is always zero.

\ttitem[PSI]
\index{roll angle}
\index{tilt angle}
Roll angle about the local \(s\)-axis,
i.e. the angle between the intersection
of the \((x, y)\) and \((Z, X)\)-planes and the local \(x\)-axis.
A positive angle \(\psi\) forms a right-hand screw with the \(s\)-axis.
\end{mylist}

The angles \((\theta, \phi, \psi)\) are {\it not} the Euler angles.
The reference orbit starts at the origin and points by default
in the direction of the positive \(Z\)-axis.
\index{local axes}\index{axes!local}
\index{global axes}\index{axes!global}
The initial local axes \((x, y, s)\) coincide with the global axes
\((X, Y, Z)\) in this order.
The six quantities
\((X_{0}, Y_{0}, Z_{0}, \theta_{0}, \phi_{0}, \psi_{0})\)
thus all have zero initial values by default.
The program user may however specify different initial conditions.
\par Internally the displacement is described by a vector \(V\),
\index{displacement}
\index{orientation}
and the orientation by a unitary matrix \(W\).
The column vectors of \(W\) are the unit vectors spanning
the local coordinate axes in the order \((x, y, s)\).
\(V\) and \(W\) have the values
\[
V=\left(\myarray{
   X \\
   Y \\
   Z
}\right),
\qquad
W=\Theta\Phi\Psi
\]
where
\[
   \Theta=\left(\myarray{
      \cos\theta &  0 &  \sin\theta \\
        0         &  1 &   0 \\
      -\sin\theta &  0 &  \cos\theta
   }\right),
   \quad
   \Phi=\left(\myarray{
       1 &  0        &  0 \\
       0 &  \cos\phi &  \sin\phi \\
       0 & -\sin\phi &  \cos\phi
   }\right),
   \quad
   \Psi=\left(\myarray{
       \cos\psi & -\sin\psi &  0 \\
       \sin\psi &  \cos\psi &  0 \\
       0        &  0        &  1
   }\right).
\]
The reference orbit should be closed and it should not be twisted.
This means that the displacement of the local reference system
must be periodic with the revolution frequency of the accelerator,
while the position angles must be periodic (modulo \(2\pi\))
with the revolution frequency.
If \(\psi\) is not periodic (modulo \(2\pi\)),
coupling effects are introduced.
When advancing through a beam element,
MAD computes \(V_{i}\) and \(W_{i}\)
by the recurrence relations
\[
   V_{i}=W_{i-1}R_{i}+V_{i-1},
   \qquad
   W_{i}=W_{i-1}S_{i}
\]
The vector \(R_{i}\) is the displacement and the matrix \(S_{i}\) is
the rotation of the local reference system at the exit of the element
\(i\) with respect to the entrance of the same element.
The values of \(R\) and \(S\) are listed below for each physical
element type.

\subsection{Straight Elements}
In straight elements the local reference system is simply
translated along the local \(s\)-axis by the length of the
element.
This is true for
\begin{itemize}
\item Drift spaces,\index{drift}
\item Quadrupoles,\index{quadrupole}
\item Sextupoles,\index{sextupole}
\item Octupoles,\index{octupole}
\item Solenoids,\index{solenoid}
\item RF cavities,\index{RF cavity}\index{cavity}
\item Electrostatic separators,\index{separator}\index{electrostatic separator}
\item Closed orbit correctors,\index{corrector}
\item Beam monitors.\index{monitor}
\end{itemize}
The corresponding \(R\) and \(S\) are
\[
   R=\left(\myarray{
      0 \\
      0 \\
      L
   }\right),
   \qquad
   S=\left(\myarray{
      1 & 0 & 0 \\
      0 & 1 & 0 \\
      0 & 0 & 1 \\
   }\right).
\]
A rotation of the element about the \(s\)-axis has no effect
on \(R\) and \(S\),
since the rotations of the reference system before and after the
element cancel.

\subsection{Bending Magnets}
\index{bending magnet}
\index{dipole}
Bending magnets have a curved reference orbit.
For both rectangular and sector bending magnets
\[
   R=\left(\myarray{
      \rho(\cos\alpha-1) \\
      0 \\
      \rho\sin\alpha
   }\right),
   \qquad
   S=\left(\myarray{
       \cos\alpha & 0 & -\sin\alpha \\
       0          & 1 &  0 \\
       \sin\alpha & 0 &  \cos\alpha
   }\right),
\]
where \(\alpha\) is the bend angle.\index{bend!angle}
A positive bend angle represents a bend to the right,
i.e. towards negative \(x\) values.
For sector bending magnets,
the bend radius\index{bend!radius} is given by \(\rho=L/\alpha\),
and for rectangular bending magnets it has the value
\(\rho=L/(2\sin(\alpha/2))\).
If the magnet is rotated about the \(s\)-axis by an angle \(\psi\),
\(R\) and \(S\) are transformed by
\[
   \overline{R}=TR,
   \qquad
   \overline{S}=TST^{-1}
\]
where \(T\) is the rotation matrix
\[
   T=\left(\myarray{
       \cos\psi & -\sin\psi &  0 \\
       \sin\psi &  \cos\psi &  0 \\
       0        &  0        &  1 \\
   }\right).
\]
The special value \(\psi=\pi/2\) represents
a bend down.

\subsection{Rotation of the Reference System}
\index{rotation}
\index{reference!system}
For a rotation of the reference system by an angle \(\psi\) about
the beam (\(s\)) axis:
\[
   S=\left(\myarray{
       \cos\psi & -\sin\psi &  0 \\
       \sin\psi &  \cos\psi &  0 \\
       0        &  0        &  1 \\
   }\right),
\]
while for a rotation of the reference system by an angle \(\theta\)
about the vertical (\(y\)) axis:
\[
   S=\left(\myarray{
       \cos\theta &  0 & -\sin\theta \\
       0          &  1 &  0 \\
       \sin\theta &  0 &  \cos\theta
   }\right).
\]
In both cases the displacement \(R\) is zero.

\subsection{Elements which do not Change the Local Reference}
\ttindex{MARKER} elements do not affect the reference orbit.
They are ignored for geometry calculations.
 
\section{Sign Conventions for Magnetic Fields}
\label{S-FIELD}
\index{magnetic field}\index{field!expansion}\index{field!magnetic}
The MAD program uses the following Taylor expansion~\cite{B-BRO72}
for the field on the mid-plane (\(y=0\)):
\[
   B_{y}(x,0)=\sum_{n=0}^{\infty}\frac{B_{n}x^{n}}{n!}
\]
Note the factorial in the denominator.
The field coefficients have the following meaning:
\begin{mylist}
\item[\(B_0\)]
Dipole field, with a positive value in the positive \(y\) direction;
a positive field bends a positively charged particle to the right.
\item[\(B_1\)]
Quadrupole coefficient \(B_1=\partial B_y/\partial x\);
a positive value corresponds to horizontal focussing of a positively
charged particle.
\item[\(B_2\)]
Sextupole coefficient \(B_2=\partial^2B_y/\partial x^2\).
\item[\(B_3\)]
Octupole coefficient \(B_3=\partial^3B_y/\partial x^3\).
\end{mylist}
Using this expansion and the curvature \(h\) of the reference orbit,
the longitudinal component of the vector potential to order 4 is:
\index{vector potential}\index{potential}
\[
\eqarray{
   A_{s}=&+&B_{0}\left(x-\frac{hx^{2}}{2(1+hx)}\right)
   &+&B_{1}\left(\frac{1}{2}(x^{2}-y^{2})-\frac{h}{6}x^{3}
      +\frac{h^{2}}{24}(4x^{4}-y^{4})+\ldots\right) \\
   &+&B_{2}\left(\frac{1}{6}(x^{3}-3xy^{2})-\frac{h}{24}(x^{4}-y^{4})
      +\ldots\right)
   &+&B_{3}\left(\frac{1}{24}(x^{4}-6x^{2}y^{2}+y^{4})+\ldots\right)
    + \ldots
}
\]
Taking \({\rm curl}A_{s}\) in curvilinear coordinates
the field components can be computed as
\index{field!components}
\[
\eqarray{
B_{x}(x,y)=
  &+&B_{1}\left(y+\frac{h^{2}}{6}y^{3}+\ldots\right) \\
  &+&B_{2}\left(xy-\frac{h^{3}}{6}y^{3}+\ldots\right)
  &+&B_{3}\left(\frac{1}{6}(3x^{2}y-y^{3})+\ldots\right) + \ldots \\
B_{y}(x,y)=&+&B_{0}
  &+&B_{1}\left(x-\frac{h}{2}y^{2}+\frac{h^{2}}{2}xy^{2}+\ldots\right) \\
  &+&B_{2}\left(\frac{1}{2}(x^{2}-y^{2})-\frac{h}{2}xy^{2}+\ldots\right)
  &+&B_{3}\left(\frac{1}{6}(x^{3}-3xy^{2})+\ldots\right) + \ldots
}
\]
It can be easily verified that \(\mathrm{curl}\vec{B}\) and
\(\mathrm{div}\vec{B}\) are both zero to the order of the \(B_3\) term.
Introducing the magnetic rigidity \(B\rho\),
\index{multipole coefficients}\index{field!components}
the multipole coefficients are computed as
\(K_n=eB_n/p_0=B_n/B\rho\).
 
\section{Variables Used in MAD}
\label{S-VARIA}
\index{variables}
\index{units}
For each variable the physical units are listed in square brackets.

\subsection{Canonical Variables Describing Orbits}
\label{S-CANON}
\index{canonical variables}\index{variables!canonical}
MAD uses the following canonical variables
to describe the motion of particles:

\begin{mylist}
\ttitem{X}
\index{horizontal position}\index{position!horizontal}
Horizontal position \(x\) of the (closed) orbit,
referred to the ideal orbit [m].

\ttitem{PX}
\index{canonical momentum}\index{momentum!canonical}
Horizontal canonical momentum \(p_{x}\) of the (closed) orbit
referred to the ideal orbit, divided by the reference momentum:
\(\hbox{\tt PX}=p_x/p_0\), [1].

\ttitem{Y}
\index{vertical position}\index{position!vertical}
Vertical position \(y\) of the (closed) orbit,
referred to the ideal orbit [m].

\ttitem{PY}
\index{canonical momentum}\index{momentum!canonical}
Vertical canonical momentum \(p_{y}\) of the (closed) orbit
referred to the ideal orbit, divided by the reference momentum:
\(\hbox{\tt PY}=p_y/p_0\), [1].

\ttitem{T}
\index{time lag}\index{lag (time)}
Velocity of light times the negative time difference
with respect to the reference particle:
\(\hbox{\tt T}=-c\Delta t\), [m].
A positive {\tt T} means that the particle arrives ahead
of the reference particle.

\ttitem{PT}
\index{energy error}\index{error!energy}
Energy error, divided by the reference momentum times the velocity
of light:
\(\hbox{\tt PT}=\delta =\Delta E/p_{0}c\), [1].
This value is only non-zero when synchrotron motion is present.
It describes the deviation of the particle from the orbit of a
particle with the momentum error {\tt DELTAP}.

\ttitem{DELTAP}
\index{momentum error}\index{error!momentum}
Difference of the reference momentum and the design momentum,
divided by the reference momentum:
\(\hbox{\tt DELTAP}=\delta_s =\Delta p_s/p_0\), [1].
This quantity is used to normalize all element strengths.
See Chapter~\ref{S-treat} for explanation.
\end{mylist}

The independent variable is:

\begin{mylist}
\ttitem{S}
\index{arc length}
Arc length along the reference orbit, [m].
\end{mylist}

\par In the limit of fully relativistic particles (\(v=c,\ E=pc\)),
\index{relativistic}
the variables {\tt T},~{\tt PT} used here agree with \(s, \delta\)
used in TRANSPORT~\cite{B-BRO80}.
This means that {\tt T} becomes the negative path length difference,
while {\tt PT} becomes the fractional momentum error.
The reference momentum \(p_0\) must be constant
in order to keep the system canonical.

\subsection{Normalised Variables and other Derived Quantities}
\label{S-NORM}
\index{normalised variables}\index{variables!normalised}

\begin{mylist}
\ttitem{XN}
\index{horizontal position}\index{position!horizontal}
The normalised horizontal displacement
\(x_{n}=\Re E_1^T S Z\) [\(\mathrm{m}^{-1/2}\)].

\ttitem{PXN}
\index{momentum}\index{momentum!normalised}
The normalised horizontal transverse momentum
\(p_{xn}=\Im E_1^T S Z\) [\(\mathrm{m}^{-1/2}\)].

\ttitem{WX}
\index{horizontal invariant}\index{invariant!horizontal}
The horizontal Courant-Snyder invariant
\(W_{x}=x_{n}^{2}+p_{xn}^{2}\) [\(\mathrm{m}^{-1}\)].

\ttitem{PHIX}
\index{horizontal phase}\index{phase!horizontal}
The horizontal phase
\(\phi_{x}=-\arctan(p_{xn}/x_{n}) / 2\pi\) [1].

\ttitem{YN}
\index{vertical position}\index{position!vertical}
The normalised vertical displacement
\(y_{n}=\Re E_2^T S Z\) [\(\mathrm{m}^{-1/2}\)].

\ttitem{PYN}
\index{momentum}\index{momentum!normalised}
The normalised vertical transverse momentum
\(p_{Yxn}=\Im E_21^T S Z\) [\(\mathrm{m}^{-1/2}\)].

\ttitem{WY}
\index{vertical invariant}\index{invariant!vertical}
The vertical Courant-Snyder invariant
\(W_{y}=y_{n}^{2}+p_{yn}^{2}\) [\(\mathrm{m}^{-1}\)].

\ttitem{PHIY}
\index{vertical phase}\index{phase!vertical}
The vertical phase
\(\phi_{y}=-\arctan(p_{yn}/y_{n}) / 2\pi\).

\ttitem{TN}
\index{vertical position}\index{position!vertical}
The normalised vertical displacement
\(t_{n}=\Re E_3^T S Z\) [\(\mathrm{m}^{-1/2}\)].

\ttitem{PTN}
\index{momentum}\index{momentum!normalised}
The normalised vertical transverse momentum
\(p_{tn}=\Im E_3^T S Z\) [\(\mathrm{m}^{-1/2}\)].

\ttitem{WT}
\index{longitudinal invariant}\index{invariant!longitudinal}
The longitudinal invariant
\(W_{y}=t_n^2 + p_{tn}^2\) [\(\mathrm{m}^{-1}\)].

\ttitem{PHIT}
\index{vertical phase}\index{phase!vertical}
The vertical phase
\(\phi_{t}=-\arctan(p_{tn}/t_{n}) / 2\pi\) [1].
\end{mylist}

in the above formulas \(Z\) is the phase space vector
\[
Z = \left(
\begin{array}{l} x \\ p_x \\ y \\ p_y \\ t \\ p_t
\end{array} \right),
\]
the matrix \(S\) is the ``symplectic unit matrix''
\[
S = \left( \begin{array}{cccccc}
 0 & 1 & 0 & 0 & 0 & 0 \\
-1 & 0 & 0 & 0 & 0 & 0 \\
 0 & 0 & 0 & 1 & 0 & 0 \\
 0 & 0 &-1 & 0 & 0 & 0 \\
 0 & 0 & 0 & 0 & 0 & 1 \\
 0 & 0 & 0 & 0 &-1 & 0
\end{array} \right ),
\]
and the vectors \(E_i\) are the three complex eigenvectors.


\subsection{Linear Lattice Functions (Optical Functions)}
\label{S-LINLAT}
\index{lattice functions}\index{functions!lattice}
Several MAD commands refer to linear lattice functions.
Since MAD uses the canonical momenta \(p_{x}, p_{y}\)
instead of the slopes \(x', y'\),
their definitions differ slightly from those
in Reference~\cite{B-COU58}.
The linear lattice functions are known to MAD under the following names:

\begin{mylist}
\ttitem{BETX}
Amplitude function \(\beta_{x}\), [m].

\ttitem{ALFX}
Correlation function
\(\alpha_{x}=-(1/2)(\partial\beta_{x}/\partial s)\), [1].

\ttitem{MUX}
Phase function \(\mu_{x}\), [\(2\pi\)].

\ttitem{DX}
Dispersion of \(x\), i.e.
\(D_{x}=\partial x/\partial\delta\), [m].

\ttitem{DPX}
Dispersion of \(p_{x}/p_{0}\), i.e.
\(D_{p_{x}}=(1/p_{0})(\partial p_{x}/\partial\delta)\), [1].

\ttitem{BETY}
Amplitude function \(\beta_{y}\), [m].

\ttitem{ALFY}
Correlation function
\(\alpha_{y}=-(1/2)(\partial\beta_{y}/\partial s)\), [1].

\ttitem{MUY}
Phase function \(\mu_{y}\), [\(2\pi\)].

\ttitem{DY}
Dispersion of \(y\), i.e.
\(D_{y}=\partial y/\partial\delta\), [m].

\ttitem{DPY}
Dispersion of \(p_{y}/p_{0}\), i.e.
\(D_{p_{y}}=(1/p_{0})(\partial p_{y}/\partial\delta)\), [1].
\end{mylist}

\subsection{Chromatic Functions}
\label{S-CHROM}
\index{chromatic functions}\index{functions!chromatic}
Several MAD commands refer to the chromatic functions.
Since MAD uses the canonical momenta \(p_{x}, p_{y}\)
instead of the slopes \(x', y'\),
their definitions differ slightly from those
in Reference~\cite{B-MON79}.
The chromatic functions are known to MAD under the following names:

\begin{mylist}
\ttitem{WX}
Chromatic amplitude function
\(W_{x}=\sqrt{a_{x}^{2}+b_{x}^{2}}\), [1], where
\[
   b_{x}=\frac{1}{\beta_{x}}\frac{d\beta_{x}}{d\delta},
   \qquad
   a_{x}=\frac{d\alpha_{x}}{d\delta}
        -\frac{\alpha_{x}}{\beta_{x}}\frac{d\beta_{x}}{d\delta}.
\]

\ttitem{PHIX}
Chromatic phase function
\(\Phi_{x}=\arctan(a_{x}/b_{x})\), [\(2\pi\)].

\ttitem{DMUX}
Chromatic derivative of phase function
\(d\mu_{x}/d\delta\), [\(2\pi\)].

\ttitem{DDX}
Chromatic derivative of dispersion
\(dD_{x}/d\delta=d^{2}x/d\delta^{2}\), [m].

\ttitem{DDPX}
Chromatic derivative of dispersion
\(dD_{p_{x}}/d\delta=(1/p_{0})(d^{2}p_{x}/d\delta^{2})\), [1].

\ttitem{WY}
Chromatic amplitude function
\(W_{y}=\sqrt{a_{y}^{2}+b_{y}^{2}}\), [1], where
\[
   b_{y}=\frac{1}{\beta_{y}}\frac{d\beta_{y}}{d\delta},
   \qquad
   a_{y}=\frac{d\alpha_{y}}{d\delta}
      -\frac{\alpha_{y}}{\beta_{y}}\frac{d\beta_{y}}{d\delta}.
\]

\ttitem{PHIY}
Chromatic phase function
\(\Phi_{y}=\arctan(a_{y}/b_{y})\), [\(2\pi\)].

\ttitem{DMUY}
Chromatic derivative of phase function
\(d\mu_{y}/d\delta\), [\(2\pi\)].

\ttitem{DDY}
Chromatic derivative of dispersion
\(dD_{y}/d\delta=d^{2}x/d\delta^{2}\), [m].

\ttitem{DDPY}
Chromatic derivative of dispersion
\(dD_{p_{y}}/d\delta=(1/p_{0})(d^{2}p_{x}/d\delta^{2})\), [1].
\end{mylist}
 
\subsection{Variables in the TUNES Tables}
\label{S-TUNES}
The \ttindex{TUNES} option of the \ttindex{TWISS} command produces a table
which contains the following variables:

\begin{mylist}
\ttitem{DELTAP}
\index{energy error}\index{error!energy}
\index{momentum error}\index{error!momentum}
Energy difference,
divided by the reference momentum times the velocity of light:
\(\hbox{\tt DELTAP}=\delta_s =\Delta E/p_{0}c\), [1].

\ttitem{ALFA}
The momentum compaction \(\alpha = \frac{\Delta C /C}{\delta}\) [1].

\ttitem{GAMMATR}
The transition energy \(\gamma_{tr} = E_{tr} / m_0c^2\) [1].

\ttitem{QX}
The horizontal tune [1].

\ttitem{QY}
The vertical tune [1].

\ttitem{XIX}
The horizontal chromaticity \(\partial Q_x / \partial \delta\) [1].

\ttitem{XIY}
The vertical chromaticity \(\partial Q_y / \partial \delta\) [1].

\ttitem{XRMS}
The horizontal r.m.s. closed orbit deviation [m].

\ttitem{YRMS}
The vertical r.m.s. closed orbit deviation [m].

\ttitem{XMAX}
The absolute maximum of horizontal closed orbit deviation [m].

\ttitem{YMAX}
The absolute maximum of vertical closed orbit deviation [m].

\ttitem{BXMAX}
The largest horizontal \(\beta_x\) [m].

\ttitem{BYMAX}
The largest vertical \(\beta_y\) [m].

\ttitem{DXMAX}
The absolutely largest horizontal dispersion [m].

\ttitem{DYMAX}
The absolutely largest vertical dispersion [m].
\end{mylist}
 
\subsection{Variables in the TRACK Table}
\label{S-TRTAB}
\index{track table}\index{table!track}
The command \ttindex{RUN} writes tables with the following variables:

\begin{mylist}
\ttitem{X}
\index{horizontal position}\index{position!horizontal}
Horizontal position \(x\) of the orbit, referred to the ideal orbit [m].

\ttitem{PX}
\index{canonical momentum}\index{momentum!canonical}
Horizontal canonical momentum \(p_{x}\) of the orbit
referred to the ideal orbit, divided by the reference momentum:
\(\hbox{\tt PX}=p_{x}/p_{0}\), [1].

\ttitem{Y}
\index{vertical position}\index{position!vertical}
Vertical position \(y\) of the orbit, referred to the ideal orbit [m].

\ttitem{PY}
\index{canonical momentum}\index{momentum!canonical}
Vertical canonical momentum \(p_{y}\) of the orbit
referred to the ideal orbit, divided by the reference momentum:
\(\hbox{\tt PY}=p_{y}/p_{0}\), [1].

\ttitem{T}
\index{time lag}\index{lag!time}
Velocity of light times the negative time difference
with respect to the reference particle:
\(\hbox{\tt T}=-c\Delta t\), [m].
A positive {\tt T} means that the particle arrives ahead
of the reference particle.

\ttitem{DELTAP}
\index{energy error}\index{error!energy}
\index{momentum error}\index{error!momentum}
Energy difference,
divided by the reference momentum times the velocity of light:
\(\hbox{\tt DELTAP}=\delta =\Delta p/p_sc\), [1].
\end{mylist}
 
\section{Physical Units Used}
\label{S-PHYSUN}
\index{units}
\index{physical units}
Throughout the computations MAD uses international
(SI, Syst\`eme International) units.
These units are summarised in Table~\ref{T-UNIT}.
\begin{table}[ht]
\label{T-UNIT}
\caption{Physical Units used by MAD}
\vspace{1ex}
\centering
\begin{tabular}{|l|l|}
\hline
Quantity &Unit \\
\hline
\index{length}
\index{angle}
\index{quadrupole}
\index{multipole}
\index{voltage}
\index{field}
\index{electric field}
\index{frequency}
\index{RF}
\index{energy}
\index{mass}
\index{momentum}
\index{current}
\index{charge}
\index{impedance}
\index{emittance}
\index{power}
\index{high order modes}
Length &m (metres) \\
Angle &rad (radians) \\
Quadrupole coefficient &\(\mathrm{m}^{-2}\) \\
Multipole coefficient, 2n poles &\(\mathrm{m}^{-n}\) \\
Electric voltage &MV (Megavolts) \\
Electric field strength &MV/m \\
Frequency &MHz (Megahertz) \\
Phase angles &multiples of \(2\pi\) \\
Particle energy &GeV \\
Particle mass &GeV/c\({}^2\) \\
Particle momentum &GeV/c \\
Beam current &A (Amp\`eres) \\
Particle charge &e (elementary charges) \\
Impedances &M\(\Omega\) (Megohms) \\
Emittances &\(\pi\cdot\)m \\
RF power &MW (Megawatts) \\
Higher mode loss factor &V/pc \\
\hline
\end{tabular}
\end{table}
 
\section{Wild Card Patterns}
\label{S-WILD}
\index{wild-card}
Some commands allow selection of items via ``wild-card'' strings.
Such a pattern string {\em must} be enclosed in single or double quotes;
and the case of letters is significant.
The meaning of special characters is similar to the UNIX \ttindex{grep}
utility:
\begin{mylist}
\ttitem{.}
Stands for an arbitrary character,
\ttitem{[\(c_1c_2c_3...\)]}
Stands for one character belonging to the string contained in brackets
(example: {\tt [abc]} means one of {\tt a, b, c}).
\ttitem{[\(c_1-c_2c_3-c_4...\)]}
Stands for ranges of characters
(example: {\tt [a-zA-Z]} means any letter).
\ttitem{*}
Allows zero or more repetitions of the preceding item
(example: {\tt [A-Z]*} means zero or more upper-case letters).
\item[\(\backslash c_1\)]
Removes the special meaning of character \(c_1\).
\end{mylist}
All other characters stand for themselves.
Examples:
\ttnindex{PATTERN}
\myxmp{
REMOVE,PATTERN="D.." \\
SAVE,PATTERN="K.*QD.*\(\backslash\).R1"
}
The first command removes all elements whose names have exactly three
characters and begin with the letter {\tt D}.
The second command saves definitions beginning with the letter {\tt K},
containing the string {\tt QD}, and ending with the string {\tt .R1}.
The two occurrences of {\tt .*} each stand for an arbitrary
number (including zero) of any character,
and the occurrence {\tt \(\backslash\).} stands for a literal period.
 
\section{Save Sequence as a Line, SAVELINE Command}
A representation of the range set in the latest {\tt USE} can
be saved as an equivalent {\tt LINE} definition using the command
\mybox{
SAVELINE,NAME=line-name,FILENAME=filename
}
where
\begin{mylist}
\ttitem{NAME}
is the name to be given on the generated {\tt LINE} definition,
\ttitem{FILENAME}
is the name of the file to be written.
\end{mylist}
The currently active range of the latest {\tt USE} command will be
written together with all parameter and element definitions.
The output may contain a lot of irrelevant information,
as no attempt is made to output only the definitions needed to
regenerate an equivalent machine.

\chapter{Commands and Statements}
\label{S-COMMAND}
\begin{table}[ht]
\label{T-COMMAND}
\caption{Utility Commands}
\index{utility commands}\index{commands!utility}
\vspace{1ex}
\centering
\begin{tabular}{|l|p{0.6\textwidth}|l|}
\hline
Name &Meaning &Section \\
\hline
\ttindex{HELP}&Help on command names &\ref{S-HELP} \\
\ttindex{SHOW}&Help on defined names &\ref{S-SHOW} \\
\ttindex{TITLE}&Define page header for output &\ref{S-TITLE} \\
\ttindex{STOP}&End program run &\ref{S-STOP} \\
\ttindex{OPTION}&Set command options &\ref{S-OPTION} \\
\tt :=&Define parameter dependencies &\ref{S-PARAM} \\
\ttindex{SET}&Set parameter value &\ref{S-SET} \\
\ttindex{VALUE}&Show parameter values &\ref{S-VALUE} \\
\ttindex{SYSTEM}&Execute operating system command &\ref{S-SYSTEM} \\
\hline
\end{tabular}
\end{table}
 
\index{format!command}\index{command!format}
\index{blanks}
\index{input!lines}
Input for MAD is free format.
Blank characters or blank input lines do not affect program execution.
Input lines are normally printed on the \ttindex{ECHO} file as soon as they
are read, but this feature can be turned off for long input files.
In the following descriptions,
words in \ttindex{lower case} stand for syntactic units
which are to be replaced by actual text.
\ttindex{UPPER CASE} is used for keywords or names.
These must be entered as shown.
Items enclosed in braces ({\tt \{\ldots\}}) may be repeated
\index{braces}
any number of times, including zero times.
\index{vertical bar}
A vertical bar (\(|\))\index{bar}\index{\("|\)} represents an alternative.
 
\section{Statements, Attributes, Options, Comments}
\label{S-STATMT}
\index{statement}
\index{command}
\index{attribute}
\index{command!attribute}
\index{comments}
The input file consists of a sequence of commands,
also known as statements.
Normally a statement occupies a single input line.
Several statements may be placed on the same line,
if they are separated by semicolons (\ttindex{;}).
A longer statement can be continued on several input lines.
All but the last line of the statement are then terminated
\index{continuation}\index{command!continuation}
by an ampersand (\ttindex{\&}).
Blank input lines,
or input lines beginning with an exclamation mark are accepted
as comment lines.
When an ampersand ({\tt\&}) or an exclamation mark
({\tt !}\index{"!@{\tt "!}}) is found
on an input line,
the remaining characters of the line are skipped,
After an ampersand ({\tt\&}) MAD expects continuation of the statement
on the next input line,
while an exclamation mark ({\tt!}) terminates the statement.
An input line cannot be longer than 80 characters;
the length of one command is limited at 4000 non-blank characters.
\par The general format for a command is
\mybox{
label: keyword \{,attribute\}
}
It has three parts:

\begin{mylist}
\ttitem{label}
\index{command!label}\index{label}
Gives a name to the stored command.
A {\tt label} is required for a definition statement.

\item[\tt keyword]
\index{keyword}
\index{command!keyword}
Identifies the action desired.

\item[\tt attribute]
\index{attribute}\index{command!attribute}
Most commands require attributes for their operation.
These are normally entered in the form
\mybox{
attribute-name=attribute-value
}
The {\tt attribute-name} selects the attribute,
and {\tt attribute-value} gives it a value.
\index{attribute!name}
\index{attribute!value}
\end{mylist}

The format of {\tt label} and {\tt keyword}
is described in Section~\ref{S-LABEL}.
 
In many cases the input can be abbreviated by omitting
{\tt attribute-name} and the equals sign.
The {\tt attribute-values} must then be entered in the order
in which attributes appear in the command dictionary.
For some attributes it is sufficient to enter the name only.
The attribute is then given a default value taken from the
command dictionary.
Example: {\tt TILT} attribute for various magnets.
 
Whenever this makes sense, an attribute can be left out altogether.
To avoid overloading the syntax descriptions,
this usage is not indicated.
 
The command dictionary also defines one of the following types
for each command attributes:
\begin{itemize}
\item Name attribute (see Section~\ref{S-NAMAT}),
\item String attribute (see Section~\ref{S-STRAT}),
\item Logical attribute (see Section~\ref{S-LOGAT}),
\item Integer attribute (see Section~\ref{S-INTAT}),
\item Real expression (see Section~\ref{S-FLTAT}),
\item Deferred expression (see Section~\ref{S-DEFAT}),
\item Constraint (see Section~\ref{S-CONAT}),
\item Variable name (see Section~\ref{S-VARAT}),
\item Line definition (see Section~\ref{S-LINAT}),
\item Range selection (see Section~\ref{S-OBSAT}),
\end{itemize}
When a command is entered with a {\tt label},
\index{label}\index{command!label}
MAD keeps it in memory.
This allows repeated execution of the same command
by just entering its label.
If the label of such a command appears
together with new attributes,
the attributes will be replaced first:
\myxmp{
xxxxxxxxxxxxxxxxxxxxxxxxxxxxx\=\kill
QF: QUADRUPOLE,L=1,K1=0.01   \>! first definition of QF \\
QF,L=2                       \>! redefinition of QF, new length \\
TW1: TWISS,BETX=1,BETY=1     \>! first execution of TW1 \\
                             \>! with BETX=1, BETY=1 \\
TW1,BETX=2,BETY=3            \>! second execution of  TW1 \\
                             \>! with BETX=2, BETY=3
}
 
\section{Getting Help}
\subsection{HELP Command}
\label{S-HELP}
\index{keyword!help}
A user who is uncertain about the attributes of a command
should try the \ttindex{HELP} command
\mybox{
HELP,keyword
}
The program then prints the list of attributes for the command
{\tt keyword} with their type and,
if they exist, their defaults and limits.
\ttindex{HELP} entered alone lists all available commands.
Examples:
\myxmp{
XXXXXXXXXXXX\=\kill
HELP        \>! List all valid keywords \\
HELP,TWISS  \>! List attributes of "TWISS"
}
 
\subsection{SHOW Command}
\label{S-SHOW}
\index{label!show}
\index{name!show}
The \ttindex{SHOW} statement displays the current values
of a command, element, line, or parameter.
\mybox{
SHOW,name
}
It displays the named command or definition on the \ttindex{ECHO} file
in the same format as the original input statement used
to create the item viewed.
\ttindex{SHOW} entered alone lists all known definitions.
Examples:
\myxmp{
XXXXXXXXX\=\kill
SHOW     \>! Show all defined names \\
SHOW,QD  \>! Show definition of QD
}
 
\section{Identifiers or Labels}
\label{S-LABEL}
\index{keyword!format}
\index{identifier format}
\index{label!format}\index{format!label}
\index{name!format}\index{format!name}
A label begins with a letter, followed by up to fifteen letters,
digits, decimal points ({\tt .}), primes ({\tt '}),
or underscores ({\tt \(\_\)}).
Characters beyond the sixteenth are dropped,
and the resulting sequence must be unique.
Other special characters are allowed,
if they are enclosed in single or double quotes.
It makes no difference which type of quotes is used,
as long as the same are used at both ends.
The preferred form is double quotes.
 
A label may refer to a keyword, an element,
a beam-line, a parameter, etc.
Once it is defined, a reference can be abbreviated by dropping trailing
characters, as long as the abbreviation fits only one defined label.
Note that abbreviations may introduce ambiguities which cannot
always be resolved correctly by MAD.
\index{abbreviations}
 
MAD maintains two name spaces,
one for keywords, and one for user-defined names.
Thus no conflict arises if a user-defined name is the same
as a predefined keyword.
 
\section{Command Attributes}
\subsection{Name Attributes}
\index{name}\index{attribute!name}
\label{S-NAMAT}
A name attribute often selects one of a set of options:
\myxmp{
xxxxxxxxxxxxxxxxxxxxxxxxx\= \kill
RUN,METHOD=TRANSPORT     \>! track by the "TRANSPORT" method
}
It may also refer to a user-defined object:
\myxmp{
xxxxxxxxxxxxxxxxxxxxxxxxx\= \kill
ARCHIVE,TABLE=TWISS      \>! archive the table "TWISS"
}
 
\subsection{String Attributes}
\label{S-STRAT}
\index{string}\index{attribute!string}
A string attribute makes alphanumeric information available,
e.g. a title or a file name.
Examples:
\myxmp{
TITLE,"This is a title for the program run" \\
POOLDUMP,FILENAME="pool.dump.file"
}
 
\subsection{Logical Attributes}
\label{S-LOGAT}
\index{logical}\index{attribute!logical}
\index{flag}
\index{TRUE@{\tt .TRUE.}}\index{FALSE@{\tt .FALSE.}}
Many commands in MAD require the setting of logical values (flags)
to represent the on/off state of an option.
A logical value {\tt flag} can be set in several ways:
\mybox{
\ flag | flag=.YES. | flag=.TRUE. | flag=.T. | flag=.ON.
}
It can be reset by any of the following:
\mybox{
-flag | flag=.NO. | flag=.FALSE. | flag=.F. | flag=.OFF.
}
Example:
\myxmp{
xxxxxxxxxxxxxxxxxxxxxxxxx\= \kill
USE,...,SYMM             \>! the beam line is symmetric
}
The default for a logical flag is always {\tt .FALSE.}.
 
\subsection{Integer Attributes}
\label{S-INTAT}
\index{integer}\index{attribute!integer}
An integer attribute usually denotes a count.
Example:
\myxmp{
xxxxxxxxxxxxxxxxxxxxxxxxx\= \kill
USE,...,SUPER=4          \>! there are four superperiods
}
 
\subsection{Real Expressions}
\label{S-FLTAT}
\index{expression}\index{attribute!real}
\index{real}
To facilitate the definition of interdependent quantities,
any real value can be entered as an arithmetic expression.
When a value used in an expression is redefined by the user
or changed in a matching process,
the expression is reevaluated.
Expression definitions may be entered in any order.
MAD evaluates them in the correct order before it performs
any computation.
At evaluation time all operands used must have values assigned.
 
\subsection{Operators in Arithmetic Expressions}
An expression can be formed using the following operators:
\begin{itemize}
\item {\em Arithmetic operators:}
\index{operator}
\begin{mylist}
\ttitem{+}
Addition,
\ttitem{-}
Subtraction,
\ttitem{*}
Multiplication,
\ttitem{/}
Division,
\ttitem{\^\ }
Exponentiation.
\end{mylist}
\item {\em Ordinary functions:}
\index{function}
\begin{mylist}
\ttitem{SQRT(X)}
Square root,
\ttitem{LOG(X)}
Logarithm,
\ttitem{EXP(X)}
Exponential,
\ttitem{SIN(X)}
Trigonometric sine,
\ttitem{COS(X)}
Trigonometric cosine,
\ttitem{TAN(X)}
Trigonometric tangent,
\ttitem{ASIN(X)}
Arc sine,
\ttitem{ABS(X)}
Absolute value,
\ttitem{MAX(X,Y)}
Maximum of two values,
\ttitem{MIN(X,Y)}
Minimum of two values.
\end{mylist}
\item {\em Random number generators:}
\index{random!generator}
\begin{mylist}
\ttitem{RANF()}
Random number, uniformly distributed in [0,1],
\ttitem{GAUSS()}
Random number, Gaussian distribution with unit standard deviation,
\ttitem{TGAUSS(X)}
Random number, Gaussian distribution with unit standard deviation,
truncated at {\tt X} standard deviations,
\ttitem{USER0()}
Random number, user-defined distribution without arguments,
\ttitem{USER1(X)}
Random number, user-defined distribution with one argument,
\ttitem{USER2(X,Y)}
Random number, user-defined distribution with two arguments.
\end{mylist}
\end{itemize}
Parentheses indicate operator precedence if required.
Constant sub-expressions are evaluated immediately,
and the result is stored as a constant.
Exponentiation is not directly available.
However, it may be performed by the identity
\(A^{B}=\exp(B*\log(A))\).
 
In an ordinary expression a random generator
is only permitted if it has no variable arguments.
A random number is then generated at definition time
and stored as a constant.
Example:
\myxmp{
xxxxxxxxxxxxxxxxxxxxxxxxx\= \kill
D:DRIFT,L=0.01*RANF()    \>! a drift space with random length, \\
                         \>! generated at definition time. \\
E:DRIFT,L=USER1(X)       \>! not permitted, if X is a variable
}
 
\subsection{Operands in Arithmetic Expressions}
\index{operand}
An expression may contain the following operands:
\begin{itemize}
\item {\em Literal constants:}
\index{constant}
Numerical values are entered like FORTRAN constants.
Real values are accepted in INTEGER or REAL format.
The use of a decimal exponent, marked by the letter {\tt D} or {\tt E},
is permitted.
Examples:
\myxmp{
1, 10.35, 5E3, 314.1592E-2
}
\item {\em Symbolic constants:}
\index{symbolic constant}
\index{constant}
MAD recognises the mathematical and physical constants
listed in Table~\ref{T-SYMB}.
Their names should not be used for user-defined labels.
 
Additional symbolic constants may be defined to simplify
their repeated use in statements and expressions.
The \ttindex{CONSTANT} command
\mybox{
label: CONSTANT=constant-expression
}
defines a constant with the name {\tt label}.
{\tt Label} must be unique.
An existing symbolic constant can be redefined,
but it cannot change in a matching procedure.
A reference to a constant is immediately replaced by its value.
Thus redefinition of a constant does not affect any
preceding definitions.
Example:
\myxmp{
IN: CONSTANT=0.0254
}
\begin{table}[ht]
\caption{Symbolic Constants}
\vspace{1ex}
\label{T-SYMB}
\index{symbolic constant}
\centering
\begin{tabular}{|l|l|c|l|}
\hline
symbol     & name           &Value used       0       &unit \\
\hline
\(\pi\)    &\ttindex{PI    }&3.1415926535898         &1 \\
\(2\pi\)   &\ttindex{TWOPI }&6.2831853071796         &1 \\
\(180/\pi\)&\ttindex{DEGRAD}&57.295779513082         &\({}^\circ\)/rad \\
\(\pi/180\)&\ttindex{RADDEG}&0.017453292519943       &rad/\({}^\circ\) \\
\(e\)      &\ttindex{E     }&2.7182818284590         &1 \\
\(m_e\)    &\ttindex{EMASS }&0.51099906*10\({}^{-3}\)&GeV \\
\(m_p\)    &\ttindex{PMASS }&0.93827231              &GeV \\
\(c\)      &\ttindex{CLIGHT}&2.99792458*10\({}^8\)   &m/s \\
\hline
\end{tabular}
\end{table}
\item {\em Parameter labels:}
\index{parameter!label}
Often a set of numerical values depends
on a common variable parameter.
Such a parameter must be defined according
to Section~\ref{S-PARAM}.
When its name is used in an expression,
MAD uses the current value of the parameter.
Example:
\myxmp{
xxxxxxxx\= \kill
        \>X:=1.0 \\
D1:     \>DRIFT,L=X \\
D2:     \>DRIFT,L=2.0-X
}
When the value of {\tt X} is changed,
the lengths of the drift spaces are recalculated as
{\tt X} and {\tt 2-X} respectively.
\item {\em Element or command attributes:}
\index{element!attribute}
\index{command!attribute}
\index{attribute}
In arithmetic expressions the attributes of physical elements
or commands can occur as operands.
They are named respectively by
\mybox{
element-name[attribute-name] \\
command-name[attribute-name]
}
Values are assigned to attributes in element definitions or commands.
Example:
\myxmp{
xxxxxxxx\= \kill
D1:     \>DRIFT,L=1.0 \\
D2:     \>DRIFT,L=2.0-D1[L]
}
{\tt D1[L]} refers to the length {\tt L} of the drift space {\tt D1}.
\end{itemize}
 
\subsection{Deferred Expressions and Random Values}
\label{S-DEFAT}
\index{expression}\index{attribute!deferred}
\index{deferred expression}
\index{random values}
The definition of random machine imperfections requires evaluation
of expressions containing random functions.
These are not evaluated like other expressions before a command
begins execution, but sampled as required from the distributions
indicated when errors are generated.
Such an expression is known as a {\em deferred expression}.
Its value cannot occur as an operand in another expression.
Example:
\myxmp{
xxxxxxxx\= \kill \\
ERROR:  \>EALIGN,range,DX=SIGMA*GAUSS()
}
All elements in {\tt range} are assigned independent random
displacements sampled from a Gaussian distribution
with standard deviation {\tt SIGMA}.
The quantity {\tt ERROR[DX]} must not occur as an operand
in another expression.
 
\subsection{Constraints}
\label{S-CONAT}
\index{constraint}
In matching it is desired to specify equality constraints,
as well as lower and upper limits for a quantity.
MAD accepts the following forms of constraints:
\mybox{
xxxxxxxxxxxxxxxxxxxxxxxxxxxxxxxxxxx\= \kill
name=expression                    \>! equality constraint \\
name<expression                    \>! upper limit \\
name>expression                    \>! lower limit \\
name<expression,name>expression    \>! both upper and lower limit \\
                                   \>! for the same name
}
 
\subsection{Variable Names}
\label{S-VARAT}
\index{variable}
A variable name can have one of the formats:
\mybox{
parameter-name \\
command-name[attribute-name] \\
element-name[attribute-name]
}
The first format refers to the value of the global parameter
{\tt parameter-name} (see Section~\ref{S-PARAM}),
\index{parameter!name}
\index{attribute!name}
the second format refers to the attribute {\tt attribute-name}
of the command {\tt command-name}
or element {\tt element-name} respectively.
 
\section{TITLE Statement}
\label{S-TITLE}
\index{page header}
The \ttindex{TITLE} statement
\mybox{
TITLE,S=page-header
}
expects a string {\tt page-header},
enclosed in single or double quotes,
which will be used as a title for subsequent output pages.
Before a {\tt TITLE} statement is encountered, the page header is blank.
It can be redefined at any time.
 
\section{STOP Statement}
\label{S-STOP}
The \ttindex{STOP} statement
\mybox{
STOP
}
terminates execution of the program.
Any statement following the {\tt STOP} statement is ignored.
 
\section{OPTION Statement}
\label{S-OPTION}
\index{command!option}
The {\tt OPTIONS} command (former {\tt SETOPTS} statement)
controls global command execution:
\mybox{
OPTION, \=RESET,INTER,ECHO,TRACE,DOUBLE,VERIFY,WARN,DEBUG,\& \\
        \>INFO,SYMPL,\& \\
        \>KEYWORD=integer,COMMAND=integer,DEFINE=integer,\& \\
        \>EXPRESS=integer,LINE=integer,TELL
}
The first seven logical flags activate or deactivate execution options:
\begin{mylist}
\ttitem{RESET}
If true, all options are reset to their default values listed below
before setting.
\ttitem{INTER}
Normally MAD determines itself whether it runs interactively.
On some computer systems this may not be possible;
the user may then set this flag manually.
\ttitem{ECHO}
Controls printing of an echo of input lines on the file {\tt ECHO}.
\ttitem{TRACE}
When the {\tt TRACE} option is on,
MAD writes additional information on the
{\tt ECHO} file for each executable command. This information includes
the command name, elapsed CPU time before and after the command,
and the CPU time used for the command.
\ttitem{DOUBLE}
When a table is created
and the {\tt DOUBLE} option is on,
the table will be built in double precision.
Otherwise the table will be built in single precision.
This option has no effect in the single-precision version of MAD.
\ttitem{VERIFY}
If this option is on, MAD gives a message for each undefined variable.
\ttitem{WARN}
If this option is turned off, MAD suppresses all warning messages.
\ttitem{INFO}
If this option is turned off, MAD suppresses all information messages. 
\ttitem{SYMPL}
If this option is turned off, MAD suppresses matrix symplectification.
\ttitem{DEBUG}
This option is reserved for the programmer.
\end{mylist}
The next five integer flags control debugging output. Each of them
can have four different values:
\begin{mylist}
\item[0]
No output.
\item[1]
Output in "comprehensive" format.
\item[2]
Output by the ZEBRA dump routines.
\item[3]
Output in both formats.
\end{mylist}
The relevant output flags are:
\begin{mylist}
\ttitem{KEYWORD}
Dump every new keyword definition.
\ttitem{COMMAND}
Dump every executable command before execution.
\ttitem{DEFINE}
Dump every new element or parameter definition.
\ttitem{EXPRESS}
Dump arithmetic expressions when they are decoded or evaluated.
\ttitem{LINE}
Dump every each new line definition.
\end{mylist}
One attribute controls the closed orbit finder:
\begin{mylist}
\ttitem{COFACT}
The former parameter {\tt COFACT} is now ignored.
\end{mylist}
The last attribute requests listing of the current settings:
\begin{mylist}
\ttitem{TELL}
If true, the current settings are listed.
\end{mylist}
Example:
\myxmp{
OPTION,-ECHO,TELL
}
Turns off the command echo print-out and lists the current settings.
If contradicting options are entered, the last option given prevails.
Example:
\myxmp{
OPTION,ECHO,-ECHO
}
turns off the {\tt ECHO} option.
The default settings at program start-up time are listed in
Table~\ref{T-OPT}.
\begin{table}[ht]
\caption{Initial Defaults for command options}
\vspace{1ex}
\label{T-OPT}
\index{command!option}\index{option!command}
\centering
\begin{tabular}{|l|l||l|l||l|l||l|l|}
\hline
option &setting &option &setting &option &setting &option &setting \\
\hline
{\tt INTER}  &on &{\tt ECHO}   &on & {\tt TRACE} &off &{\tt DOUBLE} &off \\
{\tt KEYWORD}&0  &{\tt COMMAND}&0  & {\tt DEFINE}&0   &{\tt EXPRESS}&0 \\
{\tt LINE}   &0  &             &   &             &    &             & \\
\hline
\end{tabular}
\end{table}
 
\section{Parameter Statements}
\index{parameter|(}
\index{expression}\index{definition!parameter}
\index{relation}\index{parameter!name}

\subsection{Relations between Variable Parameters}
\label{S-PARAM}
A relation is established between variables by the statement
\mybox{
parameter-name:=expression
}
\index{parameter!name}
\index{expression}
It creates a new parameter {\tt parameter-name}
and discards any old parameter with the same name.
Its value depends on all quantities occurring
in {\tt expression} on the right-hand side.
Whenever an operand in {\tt expression} changes,
a new value is calculated.
The definition may be thought of as a mathematical equation;
but MAD is not able to solve the equation for a quantity on the
right-hand side.
Example:
\myxmp{
GEV:=100 \\
BEAM,ENERGY=GEV
}
Circular definitions are not allowed (but see Section~\ref{S-SET}):
\myxmp{
xxxxxxxxxx\=\kill
X:=X+1    \>! X cannot be equal to X+1 \\
A:=B \\
B:=A      \>! A and B are equal, but of unknown value
}
 
\subsection{Assignment to Parameters}
\label{S-SET}
\index{assignment!parameter}
\index{parameter!name}
\index{expression}
A value is assigned to a parameter by the \ttindex{SET} statement
\mybox{
SET,VARIABLE=parameter-name,VALUE=expression
}
This statement acts like a FORTRAN assignment.
If the parameter {\tt parameter-name} does not yet exist,
it is created.
Then the {\tt expression} is evaluated,
and the result is assigned to the parameter {\tt parameter-name}.
Finally the {\tt expression} is discarded.
Therefore a sequence like the following is permitted:
\myxmp{
xxxxxxxxxxxxxxxxxxxxxxxxx\= \kill
\ldots                   \>! some definitions \\
USE,line \\
X:=0                     \>! create parameter X with value zero \\
                         \>! could also use "SET,X,0" \\
DO,TIMES=10              \>! repeat ten times \\
xx\=xxxxxxxxxxxxxxxxxxxxxxx\= \kill
  \>TWISS                  \>! uses X=0, 0.01, ..., 0.10 \\
  \>SET,X,X+.01            \>! increment parameter X by 0.01 \\
ENDDO
}
 
\subsection{Output of Parameters}
\label{S-VALUE}
\index{expression}\index{parameter!value}
The \ttindex{VALUE} statement
\mybox{
VALUE,VALUE=expression\{,expression\}
}
evaluates up to five {\tt expressions} using the most recent values of
any operands and prints the results on the {\tt ECHO} file.
Example:
\myxmp{
P1:=5 \\
P2:=7 \\
VALUE,P1*P2-3
}
After echoing the command, this prints:
\myxmp{
AAVALU.  Value of expression "P1*P2-3" is     32.00000000
}
The main use of this commands is for printing a quantity
which depends on matched attributes.
It allows use of MAD as a programmable calculator.
One may also tabulate functions.
\index{parameter|)}
 
\section{SYSTEM: Execute Operating System Command}
\label{S-SYSTEM}
\index{operating system command}
During an interactive MAD session the command \ttindex{SYSTEM}
allows to execute operating system commands.
After execution of the system command, successful or not,
control returns to MAD.
At present this command is only available under UNIX or VM/CMS.
Its format is:
\mybox{
SYSTEM,"string"
}
where string is a valid operating system command.
 
\subsection{SYSTEM Command under UNIX}
Most \ttindex{UNIX} commands can be issued directly. Example:
\myxmp{
SYSTEM,"ls -l"
}
causes a listing of the current directory in long form on the terminal.
 
\subsection{SYSTEM Command under VM/CMS}
If the \index{VM/CMS} system command refers to an EXEC file the string
must begin with the word EXEC.
Synonyms or abbreviations are not always accepted.
It is recommended to use standard CMS command names
and to spell them out in full.
Examples:
\myxmp{
SYSTEM,"ERASE TESTDATA MAD A" \\
SYSTEM,"EXEC FILELIST * * D" \\
SYSTEM,"XEDIT TESTDATA MAD A"
}

\clearpage
\section{COMMENT/ENDCOMMENT Statements}
Commands bracketed between {\tt COMMENT} and {\tt ENDCOMMENT} are not
executed, but skipped.
These commands may be nested.
Example:
\myxmp{
XXXXXXXXXX\=\kill
EMIT      \>! executed \\
COMMENT \\
SURVEY    \>! not executed (nesting level 1) \\
COMMENT \\
SURVEY    \>! not executed (nesting level 2) \\
ENDCOMMENT \\
TWISS     \>! still not executed (nesting level 1) \\
ENDCOMMENT \\
TWISS     \>! executed again
}

\chapter{Physical Elements and Markers}
\label{S-ELMDEF}
\index{physical element}
\index{element}
 
\begin{table}[ht]
\label{T-ELMDEF}
\caption{Element Definition Commands}
\vspace{1ex}
\centering
\begin{tabular}{|l|p{0.6\textwidth}|l|}
\hline
Name &Meaning &Section \\
\hline
\ttindex{MARKER}&Marker for beam observation &\ref{S-MARK} \\
\ttindex{DRIFT}&Drift space &\ref{S-DRIFT} \\
\ttindex{SBEND}&Sector bending magnet &\ref{S-BEND} \\
\ttindex{RBEND}&Rectangular bending magnet &\ref{S-BEND} \\
\ttindex{QUADRUPOLE}&Quadrupole &\ref{S-QUAD} \\
\ttindex{SEXTUPOLE}&Sextupole &\ref{S-SEXT} \\
\ttindex{OCTUPOLE}&Octupole &\ref{S-OCT} \\
\ttindex{MULTIPOLE}&Thin multipole &\ref{S-MULT} \\
\ttindex{SOLENOID}&Solenoid &\ref{S-SOLO} \\
\ttindex{HKICKER}&Horizontal orbit corrector &\ref{S-KICK} \\
\ttindex{VKICKER}&Vertical orbit corrector &\ref{S-KICK} \\
\ttindex{KICKER}&Corrector for both planes &\ref{S-KICK} \\
\ttindex{RFCAVITY}&RF cavity &\ref{S-RF} \\
\ttindex{ELSEPARATOR}&electrostatic separator &\ref{S-ELSE} \\
\ttindex{HMONITOR}&Horizontal orbit position monitor &\ref{S-MONI} \\
\ttindex{VMONITOR}&Vertical orbit position monitor &\ref{S-MONI} \\
\ttindex{MONITOR}&Orbit position monitor (both planes) &\ref{S-MONI} \\
\ttindex{INSTRUMENT}&Space for beam instrumentation &\ref{S-MONI} \\
\ttindex{ECOLLIMATOR}&Elliptic collimator &\ref{S-COLL} \\
\ttindex{RCOLLIMATOR}&Rectangular collimator &\ref{S-COLL} \\
\ttindex{YROT}&Rotation about vertical axis &\ref{S-ROT} \\
\ttindex{SROT}&Rotation about longitudinal axis &\ref{S-ROT} \\
\ttindex{BEAMBEAM}&Beam-beam interaction &\ref{S-BB} \\
\ttindex{MATRIX}&Arbitrary matrix &\ref{S-MATRIX} \\
\ttindex{LUMP}&Concatenation of elements &\ref{S-LUMP} \\
\hline
\end{tabular}
\end{table}
 
\section{Input Format}
\index{element!format}\index{definition!element}
\index{format!element}
\label{S-ELMFORM}
All physical elements are defined by statements of the form
\mybox{
label: keyword [,TYPE=name] \{,attribute\}
}
Example:
\myxmp{
QF: QUADRUPOLE,TYPE=MQ,L=1.8,K1=0.015832
}
where
\begin{mylist}
\item[\tt label]
\index{element!label}\index{element!name}\index{label!element}
A name to be given to the element (in the example {\tt QF}),
\item[\tt keyword]
\index{element!keyword}\index{keyword!element}
An element type keyword (in the example {\tt QUADRUPOLE}),
\ttitem{TYPE}
\index{element!type}
A label to be attached to the element.
It denotes the ``engineering type'' as defined
             documentlabelin earlier versions of MAD,
and may be used for selection of elements in various commands
like error definitions.
(in the example {\tt MQ}).
\ttitem{attribute}
\index{element!attribute}\index{attribute!element}
Normally has the form
\mybox{
attribute-name=attribute-value
}
{\tt Attribute-name} selects the attribute,
as defined for the element type keyword
(in the example {\tt L} and {\tt K1}),
and {\tt attribute-value} gives it a value
(in the example 1.8 and 0.015832).
\end{mylist}
Omitted attributes are assigned a default value, normally zero.
For some attributes it is permitted to enter their name only.
In this case the attributes are assigned a special default value
(Example: {\tt TILT} angles for magnets).
If such usage is allowed,
it is indicated with the respective attribute.
 
\section{Marker Definitions}
\label{S-MARK}
\mybox{
label: MARKER,TYPE=name
}
The simplest element which can occur in a beam line is the \ttindex{MARKER}.
It has no effect on the beam,
but it allows one to identify a position in the beam line,
for example to apply a matching constraint.
A marker has only the {\tt TYPE} attribute:
Example:
\myxmp{
M27: MARKER,TYPE=MM
}
 
\section{Drift Space}
\label{S-DRIFT}
\mybox{
label: DRIFT,TYPE=name,L=real
}
A \ttindex{DRIFT} space has one real attribute:
\begin{mylist}
\ttitem{L}
The drift length (default:~0~m)
\end{mylist}
Examples:
\myxmp{
DR1: DRIFT,L=1.5 \\
DR2: DRIFT,L=DR1[L],TYPE=DRF
}
The length of {\tt DR2} will always be equal to the length of {\tt DR1}.
The reference system for a drift space is shown in Figure~\ref{F-DRF}.
 
\begin{figure}[ht]
\centering
\setlength{\unitlength}{1pt}
\begin{picture}(400,100)
\thinlines
% axes
\put(150,50){\circle{8}}\put(150,50){\circle*{2}}
\put(140,40){\makebox(0,0){\(y_1\)}}
\put(250,50){\circle{8}}\put(250,50){\circle*{2}}
\put(260,40){\makebox(0,0){\(y_2\)}}
\put(100,50){\line(1,0){46}}
\put(154,50){\line(1,0){92}}
\put(254,50){\vector(1,0){46}}
\put(290,40){\makebox(0,0){\(s\)}}
\put(150,0){\line(0,1){46}}
\put(150,54){\vector(0,1){46}}
\put(140,90){\makebox(0,0){\(x_1\)}}
\put(250,0){\line(0,1){46}}
\put(250,54){\vector(0,1){46}}
\put(260,90){\makebox(0,0){\(x_2\)}}
% magnet outline
\thicklines
\put(150,54){\line(0,1){26}}
\put(150,46){\line(0,-1){26}}
\put(250,54){\line(0,1){26}}
\put(250,46){\line(0,-1){26}}
\put(150,20){\line(1,0){100}}
\put(150,80){\line(1,0){100}}
\put(200,2){\vector(1,0){50}}
\put(200,2){\vector(-1,0){50}}
\put(200,10){\makebox(0,0){L}}
\end{picture}
\caption{Reference System for Straight Beam Elements}
\label{F-DRF}
\end{figure}
 
\section{Bending Magnets}
\label{S-BEND}
\index{bending magnet}
\index{dipole}
Two different type keywords are recognised for bending magnets,
they are distinguished only by the reference system used:
\begin{mylist}
\ttitem{RBEND}
Rectangular bending magnet (Reference see Figure~\ref{F-RBND}),
\ttitem{SBEND}
Sector bending magnet (Reference see Figure~\ref{F-SBND}).
\end{mylist}
 
\begin{figure}[ht]
\centering
\setlength{\unitlength}{1pt}
\begin{picture}(400,215)
% axes
\thinlines
\put(150,150){\circle{8}}\put(150,150){\circle*{2}}
\put(160,140){\makebox(0,0){\(y_1\)}}
\put(250,150){\circle{8}}\put(250,150){\circle*{2}}
\put(240,140){\makebox(0,0){\(y_2\)}}
\put(74,124.7){\vector(3,1){72}}
\put(84,135){\makebox(0,0){\(s_1\)}}
\put(254,148.7){\vector(3,-1){72}}
\put(316,135){\makebox(0,0){\(s_2\)}}
\put(200,0){\vector(-1,3){48.7}}
\put(165,75){\makebox(0,0){\(\rho\)}}
\put(148.7,154){\vector(-1,3){18}}
\put(118,206){\makebox(0,0){\(x_1\)}}
\put(200,0){\vector(1,3){48.7}}
\put(235,75){\makebox(0,0){\(\rho\)}}
\put(251.3,154){\vector(1,3){18}}
\put(282,206){\makebox(0,0){\(x_2\)}}
\bezier{20}(190.5,28.5)(200,31.7)(209.5,28.5)
\put(200,20){\makebox(0,0){\(\alpha\)}}
\put(154,150){\line(1,0){92}}
\put(200,150){\circle*{4}}
\put(200,150){\vector(0,1){60}}
\put(210,200){\makebox(0,0){\(x\)}}
\put(150,154){\line(0,1){44}}
\put(150,146){\line(0,-1){46}}
\put(151,154){\line(1,4){11}}
\put(250,154){\line(0,1){44}}
\put(250,146){\line(0,-1){46}}
\put(249,154){\line(-1,4){11}}
% magnet outline
\thicklines
\put(200,102){\vector(-1,0){50}}
\put(200,102){\vector(1,0){50}}
\put(200,110){\makebox(0,0){L}}
\put(151,154){\line(1,4){6}}
\put(149,146){\line(-1,-4){6}}
\put(249,154){\line(-1,4){6}}
\put(251,146){\line(1,-4){6}}
\put(157,178){\line(1,0){86}}
\put(143,122){\line(1,0){114}}
\bezier{10}(150,195)(155.5,195)(160.9,193.7)
\put(155.5,195){\vector(3,-1){5.4}}
\put(150,205){\makebox(0,0)[l]{\(e_1\)}}
\bezier{10}(250,195)(244.5,195)(239.1,193.7)
\put(244.5,195){\vector(-3,-1){5.4}}
\put(250,205){\makebox(0,0)[r]{\(e_2\)}}
\end{picture}
\caption[Reference System for a Rectangular Bending Magnet]%
{Reference System for a Rectangular Bending Magnet;
the signs of pole-face rotations are positive as shown.}
\label{F-RBND}
\end{figure}
 
\begin{figure}[ht]%                                         3.3
\centering
\setlength{\unitlength}{1pt}
\begin{picture}(400,215)
% axes
\thinlines
\put(150,150){\circle{8}}\put(150,150){\circle*{2}}
\put(160,140){\makebox(0,0){\(y_1\)}}
\put(250,150){\circle{8}}\put(250,150){\circle*{2}}
\put(240,140){\makebox(0,0){\(y_2\)}}
\put(74,124.7){\vector(3,1){72}}
\put(84,135){\makebox(0,0){\(s_1\)}}
\put(254,148.7){\vector(3,-1){72}}
\put(316,135){\makebox(0,0){\(s_2\)}}
\put(200,0){\vector(-1,3){48.7}}
\put(165,75){\makebox(0,0){\(\rho\)}}
\put(148.7,154){\vector(-1,3){18}}
\put(118,206){\makebox(0,0){\(x_1\)}}
\put(200,0){\vector(1,3){48.7}}
\put(235,75){\makebox(0,0){\(\rho\)}}
\put(251.3,154){\vector(1,3){18}}
\put(282,206){\makebox(0,0){\(x_2\)}}
\bezier{20}(190.5,28.5)(200,31.7)(209.5,28.5)
\put(200,20){\makebox(0,0){\(\alpha\)}}
\put(200,158.8){\circle*{4}}
\put(200,158.8){\vector(0,1){50}}
\put(210,200){\makebox(0,0){\(r\)}}
\put(151,154){\line(1,4){10}}
\put(249,154){\line(-1,4){10}}
% magnet outline
\thicklines
\bezier{100}(154,151.3)(200,166.7)(246,151.3)
\put(162,154){\vector(-3,-1){8}}
\put(238,154){\vector(3,-1){8}}
\put(210,168){\makebox(0,0){L}}
\put(151,154){\line(1,4){6}}
\put(149,146){\line(-1,-4){6}}
\put(249,154){\line(-1,4){6}}
\put(251,146){\line(1,-4){6}}
\bezier{90}(157,178)(200,188.4)(243,178)
\bezier{110}(143,122)(200,148.6)(257,122)
\bezier{20}(137.4,187.9)(149.1,191.5)(159.7,188.8)
\put(153.7,190.8){\vector(3,-1){6}}
\put(150,180){\makebox(0,0){\(e_1\)}}
\bezier{20}(262.6,187.9)(250.9,191.5)(240.3,188.8)
\put(246.3,190.8){\vector(-3,-1){6}}
\put(250,180){\makebox(0,0){\(e_2\)}}
\end{picture}
\caption[Reference System for a Sector Bending Magnet]%
{Reference System for a Sector Bending Magnet;
the signs of pole-face rotations are positive as shown.}
\label{F-SBND}
\end{figure}
 
\mybox{
SBEND, \=TYPE=name,L=real,ANGLE=real,K1=real,E1=real,E2=real,\& \\
       \>TILT=real,K2=real,H1=real,K2=real,HGAP=real,FINT=real,\& \\
       \>K3=real \\
RBEND, \>TYPE=name,L=real,ANGLE=real,K1=real,E1=real,E2=real,\& \\
       \>TILT=real,K2=real,H1=real,K2=real,HGAP=real,FINT=real,\& \\
       \>K3=real
}
For both types,
the following first-order attributes are permitted:
\begin{mylist}
\ttitem{L}
The length of the magnet (default:~0~m).
For a rectangular magnet the length is measured along a straight line,
while for a sector magnet it is the arc length of the reference orbit.
\ttitem{ANGLE}
The bend angle (default:~0~rad).
A positive bend angle represents a bend to the right,
i.e. towards negative \(x\) values.
\ttitem{K1}
The quadrupole coefficient
\(K_{1}=(1/B\rho) (\partial B_{y}/\partial x)\).
The default is~0~\(\mathrm{m}^{-2}\).
A positive quadrupole strength implies horizontal focussing
of positively charged particles.
\ttitem{E1}
The rotation angle for the entrance pole face
(default:~0~rad).
\ttitem{E2}
The rotation angle for the exit pole face
(default:~0~rad).
\ttitem{FINT}
\index{fringe field}
\index{field!fringe integral}
The field integral, see~\cite{B-BRO72} and below. The default value is 0.
\ttitem{HGAP}
The half gap of the magnet (default:~0~m).
\end{mylist}
The pole face rotation angles are referred to the magnet model
(see Figure~\ref{F-RBND} and Figure~\ref{F-SBND}).
The quantities {\tt FINT} and {\tt HGAP} specify
the finite extent of the fringe fields~\cite{B-BRO72}.
There they are defined as follows:
\[
K_1 = \hbox{\tt FINT} = \int_{-\infty}^{\infty}
\frac{B_y(s) (B_0 - B_y(s))}{g \cdot B_0^2} ds,
\qquad
g = 2 \cdot \hbox{\tt HGAP}.
\]
The default values of zero corresponds to the hard-edge approximation,
i.e. a rectangular field distribution.
For other approximations, enter the correct value of the half gap,
and one of the following values for {\tt FINT}:
\begin{center}
\begin{tabular}{ll}
Linear drop-off of the field           &1/6 \\
Clamped ``Rogowski'' fringing field    &0.4 \\
Un-clamped ``Rogowski'' fringing field  &0.7 \\
``Square-edged'' non-saturating magnet &0.45
\end{tabular}
\end{center}
Entering the keyword {\tt FINT} alone sets the integral to 0.5.
This is a reasonable average of the above values.
The magnet may be rotated about the longitudinal axis by use of the
following parameter:
\begin{mylist}
\ttitem{TILT}
The roll angle about the longitudinal axis (default:~0~rad,
a positive bend angle then denotes a bend to the right).
A vertical bend is defined by entering {\tt TILT} with no value;
this implies a roll of \(\pi/2\)~rad,
i.e. a positive bend angle denotes a deflection down.
A positive angle represents a clockwise rotation.
The following second-order attributes are permitted:
\ttitem{K2}
The sextupole coefficient
\(K_{2}=(1/B\rho) (\partial^{2} B_{y})/(\partial x^{2})\)
(default:~0~\(\mathrm{m}^{-3}\)).
\ttitem{H1}
The curvature of the entrance pole face (default:~0~\(\mathrm{m}^{-1}\)).
\ttitem{H2}
The curvature of the exit pole face (default:~0~\(\mathrm{m}^{-1}\)).
A positive pole face curvature induces a negative sextupole component;
i.e. for positive {\tt H1} and {\tt H2}
the centres of curvature of the pole faces are placed inside the magnet.
\end{mylist}
One third-order parameter is accepted,
but at present it is ignored:
\begin{mylist}
\ttitem{K3}
The octupole coefficient
\(K_{3}=(1/B\rho) (\partial^{3} B_{y}/\partial x^{3})\)
(default:~0~\(\mathrm{m}^{-4}\)).
\end{mylist}
Examples:
\myxmp{
xxxxxxxxxxxxxxxxxxxxxxxxxxxxxxxxxxxxxxxxxxx\= \kill
BR: RBEND,L=5.5,ANGLE=+0.001               \>! Deflection to the right \\
BD: SBEND,L=5.5,ANGLE=+0.001,TILT          \>! Deflection down \\
BL: SBEND,L=5.5,ANGLE=-0.001               \>! Deflection to the left \\
BU: SBEND,L=5.5,ANGLE=-0.001,TILT          \>! Deflection up
}
 
\section{Quadrupole}
\label{S-QUAD}
\mybox{
label: QUADRUPOLE,TYPE=name,L=real,K1=real,TILT=real
}
A \ttindex{QUADRUPOLE} has three real attributes:
\begin{mylist}
\ttitem{L}
The quadrupole length (default:~0~m).
\ttitem{K1}
The quadrupole coefficient
\(K_{1}=(1/B\rho) (\partial B_{y}/\partial x)\).
The default is~0~\(\mathrm{m}^{-2}\).
A positive quadrupole strength implies horizontal
focussing of positively charged particles.
\ttitem{TILT}
The roll angle about the longitudinal axis (default:~0~rad,
i.~e.~a normal quadrupole).
A skewed quadrupole is defined by entering {\tt TILT} with no value;
this implies a roll of \(\pi/4\)~rad about the \(s\)-axis.
A positive angle represents a clockwise rotation.
\end{mylist}
Example:
\myxmp{
QF: QUADRUPOLE,L=1.5,K1=0.001
}
The reference system for a quadrupole is shown in Figure~\ref{F-DRF}.
 
\section{Sextupole}
\label{S-SEXT}
\mybox{
label: SEXTUPOLE,TYPE=name,L=real,K2=real,TILT=real
}
A \ttindex{SEXTUPOLE} has three real attributes:
\begin{mylist}
\ttitem{L}
The sextupole length (default:~0~m).
\ttitem{K2}
The sextupole coefficient
\(K_{2}=(1/B\rho) (\partial^{2} B_{y}/\partial x^{2})\)
(default:~0~\(\mathrm{m}^{-3}\)).
\ttitem{TILT}
The roll angle about the longitudinal axis (default:~0~rad,
i.~e.~a normal sextupole).
A skewed sextupole is defined by entering {\tt TILT} with no value;
this implies a roll of \(\pi/6\)~rad about the \(s\)-axis.
A positive angle represents a clockwise rotation.
\end{mylist}
Example:
\myxmp{
S: SEXTUPOLE,L=0.4,K2=0.00134
}
The reference system for a sextupole is shown in Figure~\ref{F-DRF}.
 
\section{Octupole}
\label{S-OCT}
\mybox{
label: OCTUPOLE,TYPE=name,L=real,K3=real,TILT=real
}
An \ttindex{OCTUPOLE} has three real attributes:
\begin{mylist}
\ttitem{L}
The octupole length (default:~0~m).
\ttitem{K3}
The octupole coefficient
\(K_{3}=(1/B\rho) (\partial^{3} B_{y}/\partial x^{3})\)
(default:~0~\(\mathrm{m}^{-4}\)).
\ttitem{TILT}
The roll angle about the longitudinal axis (default:~0~rad,
i.~e.~a normal octupole).
A skewed octupole is defined by entering {\tt TILT} with no value;
this implies a roll of \(\pi/8\)~rad about the \(s\)-axis.
A positive angle represents a clockwise rotation.
\end{mylist}
Example:
\myxmp{
O3: OCTUPOLE,L=0.3,K3=0.543
}
The reference system for a octupole is shown in Figure~\ref{F-DRF}.
Octupoles are normally treated as thin lenses,
except when tracking by Lie-algebraic methods.
 
\section{General Thin Multipole}
\label{S-MULT}
\mybox{
label: \=MULTIPOLE,TYPE=name,LRAD=real,\& \\
       \>K0L=real,T0=real,K1L=real,T1=real,\& \\
       \>K2L=real,T2=real,K3L=real,T3=real,\& \\
       \>K4L=real,T4=real,K5L=real,T5=real,\& \\
       \>K6L=real,T6=real,K7L=real,T7=real,\& \\
       \>K8L=real,T8=real,K9L=real,T9=real
}
A \ttindex{MULTIPOLE} is thin lens of arbitrary order, including a dipole:
\begin{mylist}
\ttitem{LRAD}
A fictitious length, which is only used to compute synchrotron
radiation effects.
\ttitem{KnL}
The multipole coefficient of order \(n\):
\(K_{n}L=(L/B\rho) (\partial^{n} B_{y}/\partial x^{n})\).
(default:~0~\(\mathrm{m}^{-n}\)).
This is the multipole coefficient integrated over the length
of the multipole.
The digit \(n\) may take the values \(0\leq n\leq 9\).
The number of poles of the component is \((2n + 2)\).
The most important error components of quadrupoles are:
{\tt K5L}, the twelve-pole, and {\tt K9L}, the twenty-pole.
Superposition of several multipole components is permitted.
\ttitem{Tn}
The roll angle for the multipole component of order \(n\)
about the \(s\)-axis
(default:~0~rad, i.~e.~a normal multipole).
A skewed multipole is defined by entering {\tt Tn} with no value;
this implies a roll of \(\pi/(2n + 2)\)~rad about the \(s\)-axis.
A positive angle represents a clockwise rotation.
\end{mylist}
The multipole element has no length.
Example:
\myxmp{
M27: MULTIPOLE,K3L=0.0001,T3,K2L=0.0001
}
A multipole with no dipole component has no effect on the reference
orbit,
i.e. the reference system at its exit is the same as at its entrance.
If it includes a dipole component,
it has the same effect on the reference orbit as a dipole
with zero length and deflection angle {\tt K0L}.
 
\section{Solenoid}
\label{S-SOLO}
\mybox{
label: SOLENOID,TYPE=name,L=real,KS=real
}
A \ttindex{SOLENOID} has two real attributes:
\begin{mylist}
\ttitem{L}
The length of the solenoid (default:~0~m)
\ttitem{KS}
The solenoid strength \(B_0 / B \rho\)
(default:~0~rad/m).
For positive {\tt KS} and positive particle charge,
the solenoid field points in the direction of increasing~\(s\).
\end{mylist}
Example:
\myxmp{
SOLO: SOLENOID,L=2.,K=0.001
}
The reference system for a solenoid is shown in Figure~\ref{F-DRF}.
 
\section{Closed Orbit Correctors}
\label{S-KICK}
\index{corrector}
\index{orbit corrector}
Three types of closed orbit correctors are available:
\begin{mylist}
\ttitem{HKICKER}
A corrector for the horizontal plane,
\ttitem{VKICKER}
A corrector for the vertical plane,
\ttitem{KICKER}
A corrector for both planes.
\end{mylist}
\mybox{
label: \=HKICKER, \=TYPE=name,L=real,KICK=real,TILT=real \\
label: \>VKICKER, \>TYPE=name,L=real,KICK=real,TILT=real \\
label: \>KICKER,  \>TYPE=name,L=real,HKICK=real,VKICK=real,TILT=real
}
They have the following attributes:
\begin{mylist}
\ttitem{L}
The length of the closed orbit corrector (default:~0~m).
\ttitem{KICK}
The kick angle for either horizontal or vertical correctors.
(default:~0~rad).
\ttitem{HKICK}
The horizontal kick angle for a corrector in both planes
(default:~0~rad).
\ttitem{VKICK}
The vertical kick angle for a corrector in both planes
(default:~0~rad).
\ttitem{TILT}
The roll angle about the longitudinal axis (default:~0~rad).
A positive angle represents a clockwise rotation of the corrector.
\end{mylist}
A positive kick increases \(p_{x}\) or \(p_{y}\) respectively.
Examples:
\myxmp{
HK1: \=HKICKER, \=KICK=0.001,TILT=RADDEG*10.0 \\
VK3: \>VKICKER, \>KICK=0.0005 \\
KHV: \>KICKER,  \>HKICK=0.001,VKICK=0.0005
}
The first kicker is rotated about the longitudinal axis by 10 degrees.
The reference system for a closed orbit corrector is shown in
Figure~\ref{F-DRF}.
 
\section{RF Cavity}
\label{S-RF}
\index{cavity}
\mybox{
label: RFCAVITY,\=TYPE=name,L=real,VOLT=real,LAG=real,\& \\
                 \>HARMON=integer,BETRF=real,PG=real,\& \\
                 \>SHUNT=real,TFILL=real
}
An \ttindex{RFCAVITY} has seven real attributes and one integer attribute:
\begin{mylist}
\ttitem{L}
The length of the cavity (default:~0~m)
\ttitem{VOLT}
The peak RF voltage (default:~0~MV).
The effect of the cavity is
\[
   \delta E={\rm VOLT} \sin(2\pi({\rm LAG}- h f_{0} \Delta t)).
\]
\ttitem{LAG}
The phase lag in multiples of \(2\pi\) (default:~0).
\ttitem{HARMON}
The harmonic number \(h\) (no default).
Note that the RF frequency is computed from the harmonic number
and the revolution frequency \(f_{0}\).
{\em The frequency attribute {\tt FREQ} must no longer be used.}
\ttitem{BETRF}
RF coupling factor (default:~0).
\ttitem{PG}
The RF power per cavity (default:~0~MW).
\ttitem{SHUNT}
The relative shunt impedance (default:~0~M\(\Omega\)/m).
\ttitem{TFILL}
The filling time of the cavity (default:~0~\(\mu\)s).
\end{mylist}
A cavity requires the particle energy ({\tt ENERGY})
and the particle charge ({\tt CHARGE})
to be set by a {\tt BEAM} command (see Section~\ref{S-BEAM})
before any calculations are performed.
Example:
\myxmp{
xxxxxxxx\= \kill
        \>BEAM,PARTICLE=ELECTRON,ENERGY=50.0 \\
CAVITY: \>RFCAVITY,L=10.0,VOLT=150.0,LAG=0.0,HARMON=31320
}
The reference system for a cavity is shown in Figure~\ref{F-DRF}.
 
\section{Electrostatic Separator}
\label{S-ELSE}
\index{separator}
\index{electrostatic separator}
\mybox{
label: ELSEPARATOR,TYPE=name,L=real,E=real,TILT=real
}
An \ttindex{ELSEPARATOR} (electrostatic separator) has three real
attributes:
\begin{mylist}
\ttitem{L}
The length of the separator (default:~0~m).
\ttitem{E}
The electric field strength (default:~0~MV/m).
\index{electric field}
\index{field!electric}
A positive field increases \(p_{y}\) for positive particles.
\ttitem{TILT}
The roll angle about the longitudinal axis (default:~0~rad).
A positive angle represents a clockwise rotation of the separator.
\end{mylist}
A separator requires the particle energy ({\tt ENERGY})
and the particle charge ({\tt CHARGE})
to be set by a {\tt BEAM} command (see Section~\ref{S-BEAM})
before any calculations are performed.
Example:
\myxmp{
xxxxx\= \kill
     \>BEAM,PARTICLE=POSITRON,ENERGY=50.0 \\
SEP: \>ELSEPARATOR,L=5.0,E=0.5
}
The reference system for a separator is shown in Figure~\ref{F-DRF}.
 
\section{Beam Position Monitor}
\label{S-MONI}
\index{monitor}
\index{beam monitor}
A beam monitor acts on the beam like a drift space.
In addition it serves to record the beam position for closed orbit
corrections.
Four different types of beam position monitors are recognised:
\begin{mylist}
\ttitem{HMONITOR}
Monitor for the horizontal beam position,
\ttitem{VMONITOR}
Monitor for the vertical beam position,
\ttitem{MONITOR}
Monitor for both horizontal and vertical beam position.
\ttitem{INSTRUMENT}
A place holder for any type of beam instrumentation.
Optically it behaves like a drift space;
it returns {\em no beam observation}.
It represent a class of elements
which is completely independent from drifts and monitors.
\end{mylist}
\mybox{
xxxxxxxxxxxxxxxxxxxx\=\kill
label: HMONITOR,    \>TYPE=name,L=real \\
label: VMONITOR,    \>TYPE=name,L=real \\
label: MONITOR,     \>TYPE=name,L=real \\
label: INSTRUMENT,  \>TYPE=name,L=real
}
A beam position monitor has one real attribute:
\begin{mylist}
\ttitem{L}
The length of the monitor (default:~0~m). If the length is
different from zero,
the beam position is recorded in the centre of the monitor.
\end{mylist}
Examples:
\myxmp{
MH: HMONITOR,L=1 \\
MV: VMONITOR
}
The reference system for a monitor is shown in Figure~\ref{F-DRF}.
 
\section{Collimators (Aperture Definitions)}
\label{S-COLL}
\index{collimator}
\index{aperture}
Two types of collimators are defined:
\begin{mylist}
\ttitem{ECOLLIMATOR}
Elliptic aperture,
\ttitem{RCOLLIMATOR}
Rectangular aperture.
\end{mylist}
\mybox{
label: ECOLLIMATOR,TYPE=name,L=real,XSIZE=real,YSIZE=real \\
label: RCOLLIMATOR,TYPE=name,L=real,XSIZE=real,YSIZE=real
}
Either type has three real attributes:
\begin{mylist}
\ttitem{L}
The collimator length (default:~0~m).
\ttitem{XSIZE}
The horizontal half-aperture (default:~unlimited).
\ttitem{YSIZE}
The vertical half-aperture (default:~unlimited).
\end{mylist}
For elliptic apertures,
{\tt XSIZE} and {\tt YSIZE} denote the half-axes respectively,
for rectangular apertures they denote the half-width of the rectangle.
Optically a collimator behaves like a drift space, but during tracking,
it also introduces an aperture limit.
The aperture is checked at the entrance.
If the length is not zero, the aperture is also checked at the exit.
The reference system for a collimator is shown in
Figure~\ref{F-DRF}.
Example:
\myxmp{
COLLIM: ECOLLIMATOR,L=0.5,XSIZE=0.01,YSIZE=0.005
}
 
\section{Coordinate Transformations}
\label{S-ROT}
\index{coordinate transformation}
\index{rotation}
\subsection{Rotation About the Vertical Axis}
\mybox{
label: YROT,TYPE=name,ANGLE=real
}
The element \ttindex{YROT} rotates the reference system
about the vertical (\(y\)) axis.
The reference system is shown in Figure~\ref{F-YROT}.
{\tt YROT} has no effect on the beam,
but it causes the beam to be referred to the new coordinate system
\[
x_{2}=x_{1} \cos\theta - s_{1} \sin\theta,
\qquad
s_{2}=x_{1} \sin\theta + s_{1} \cos\theta.
\]
It has one real attribute:
\begin{mylist}
\ttitem{ANGLE}
The rotation angle \(\theta\) (default:~0~rad).
It must be a {\em small} angle,
i.e. an angle comparable to the transverse angles of the orbit.
\end{mylist}
A positive angle means that the new reference system is rotated
clockwise about the local \(y\)-axis with respect to the old system.
Example:
\myxmp{
KINK: YROT,ANGLE=0.0001
}
 
\begin{figure}[ht]%                                         3.4
\centering
\setlength{\unitlength}{1pt}
\begin{picture}(400,200)
\thinlines
\put(200,100){\circle{8}}\put(200,100){\circle*{2}}
\put(190,90){\makebox(0,0){\(y\)}}
\put(100,100){\line(1,0){96}}
\put(204,100){\vector(1,0){96}}
\put(290,110){\makebox(0,0){\(s_1\)}}
\put(200,0){\line(0,1){96}}
\put(200,104){\vector(0,1){96}}
\put(190,190){\makebox(0,0){\(x_1\)}}
\put(103,124.25){\line(4,-1){93}}
\put(204,99){\vector(4,-1){93}}
\put(287,66){\makebox(0,0){\(s_2\)}}
\put(175.75,3){\line(1,4){23.25}}
\put(201,104){\vector(1,4){23.25}}
\put(234,187){\makebox(0,0){\(x_2\)}}
\bezier{20}(260,100)(260,92.5)(258,85.5)
\put(260,93.5){\vector(-1,-4){2}}
\put(250,93.75){\makebox(0,0){\(\theta\)}}
\thicklines
\put(100,130){\vector(1,0){200}}
\put(290,140){\makebox(0,0){beam}}
\end{picture}
\caption{Reference System for a Rotation Around the y-Axis}
\label{F-YROT}
\end{figure}
 
\subsection{Rotation Around the Longitudinal Axis}
\mybox{
label: SROT,TYPE=name,ANGLE=real
}
The element \ttindex{SROT} rotates the reference
system about the longitudinal (\(s\)) axis.
The reference system is shown in Figure~\ref{F-SROT}.
{\tt SROT} has no effect on the beam,
but it causes the beam to be referred to the new coordinate system
\[
x_{2}=x_{1} \cos\psi + y_{1} \sin\psi,
\qquad
y_{2}=x_{1} \sin\psi + y_{1} \cos\psi.
\]
It has one real attribute:
\begin{mylist}
\ttitem{ANGLE}
The rotation angle \(\psi\) (default:~0~rad)
\end{mylist}
A positive angle means that the new reference system is rotated clockwise
about the \(s\)-axis with respect to the old system.
Example:
\myxmp{
ROLL1: SROT,ANGLE=PI/2. \\
ROLL2: SROT,ANGLE=-PI/2. \\
HBEND: SBEND,L=6.0,ANGLE=0.01 \\
VBEND: LINE=(ROLL1,HBEND,ROLL2)
}
The above is a way to represent a bend down in the vertical plane,
it could be defined more simply by
\myxmp{
VBEND: SBEND,L=6.0,ANGLE=0.01,TILT
}
 
\begin{figure}[ht]%                                         3.5
\centering
\setlength{\unitlength}{1pt}
\begin{picture}(400,200)
\thinlines
\put(200,100){\circle{8}}\put(200,100){\circle*{2}}
\put(190,90){\makebox(0,0){\(s\)}}
\put(100,100){\line(1,0){96}}
\put(204,100){\vector(1,0){96}}
\put(290,90){\makebox(0,0){\(x_1\)}}
\put(200,0){\line(0,1){96}}
\put(200,104){\vector(0,1){96}}
\put(190,210){\makebox(0,0){\(y_1\)}}
\put(103,75.75){\line(4,1){93}}
\put(204,101){\vector(4,1){93}}
\put(287,134){\makebox(0,0){\(x_2\)}}
\put(224.25,3){\line(-1,4){23.25}}
\put(199,104){\vector(-1,4){23.25}}
\put(166,187){\makebox(0,0){\(y_2\)}}
\bezier{20}(260,100)(260,107.5)(258,114.5)
\put(260,106.5){\vector(-1,4){2}}
\put(250,106.25){\makebox(0,0){\(\psi\)}}
\put(220,150){\circle{8}}\put(220,150){\circle*{2}}
\put(220,140){\makebox(0,0){beam}}
\end{picture}
\caption{Reference System for a Rotation Around the s-Axis}
\label{F-SROT}
\end{figure}
 
\section{BEAMBEAM Element}
\label{S-BB}
\index{beam interaction}
\mybox{
label: BEAMBEAM, \=TYPE=name,SIGX=real,SIGY=real,XMA=real,YMA=real,\& \\
                 \>CHARGE=real
}
The command \ttindex{BEAMBEAM} may be inserted in a beam line
to simulate a beam-beam interaction point.%
\footnote{Contributed by J. M. Veuillen (1987).}
It has five real attributes:
\begin{mylist}
\ttitem{SIGX}
The horizontal extent (standard deviation) of the opposite beam
(default:~0~m).
\ttitem{SIGY}
The vertical extent (standard deviation) of the opposite beam
(default:~0~m).
\ttitem{XMA}
The horizontal displacement of the opposite beam with respect to
the ideal orbit (default:~0~m).
\ttitem{YMA}
The vertical displacement of the opposite beam with respect to
the ideal orbit (default:~0~m).
\ttitem{CHARGE}
The charge of particles in the opposite beam in proton charges
(default:~1).
Entering the keyword {\tt CHARGE} alone sets the charge to \(-1\).
\end{mylist}
A beam-beam interaction requires the particle energy ({\tt ENERGY})
and the charge of the particles in the beam considered ({\tt CHARGE}),
as well as the number of particles per bunch ({\tt NPART})
to be set by a {\tt BEAM} command (see Section~\ref{S-BEAM})
before any calculations are performed.
Example:
\myxmp{
BEAM, \=PARTICLE=POSITRON,NPART=1.E12,ENERGY=50.0 \\
BB:   \>BEAMBEAM,SIGX=1.E-3,SIGY=5.E-4,CHARGE=1.
}
 
\section{Arbitrary MATRIX Element}
\label{S-MATRIX}
\mybox{
MATRIX, \=TYPE=name,RM(1,1)=real,...,RM(6,6)=real,\& \\
        \>TM(1,1,1)=real,...,TM(6,6,6)=real
}
The \ttindex{MATRIX} permits definition of an arbitrary transfer matrix.
It has two real array attributes:
\begin{mylist}
\ttitem{RM(i,k)}
Defines the element \((i,k)\) of the linear transfer matrix.
\ttitem{TM(i,k,l)}
Defines the element \((i,k,l)\) of the second-order TRANSPORT matrix.
\end{mylist}
Matrix elements not entered are taken from the identity transformation,
i.e.
\[
\hbox{\tt RM(i,k)} = \left\{ \eqarray{
1,&{\rm if\ } i=k, \\ 0,&{\rm if\ } i \ne k,
} \right.
\qquad \qquad \hbox{\tt TM(i,k,l)} = 0.
\]
 
\section{Editing Element Definitions}
\label{S-EDIT}
\index{edit!elements}\index{element!edit}
\index{redefinition}
\index{element!redefinition}
The {\tt EDIT} command of former MAD versions is no longer available,
but an element definition can be changed in two ways:
\begin{itemize}
\item {\em Entering a new definition:}
The element will be replaced in the main beam line expansion.
\item {\em Entering the element name together with new attributes:}
The element will be updated in place,
and the new attribute values will replace the old ones.
\end{itemize}
This example shows two ways to change the strength of a quadrupole:
\myxmp{
xxxxxxxxxxxxxxxxxxxxxxxxxxxxxxx\=\kill
QF: QUADRUPOLE,L=1,K1=0.01     \>! Original definition of QF \\
QF: QUADRUPOLE,L=1,K1=0.02     \>! Replace whole definition of QF \\
QF,K1=0.02                     \>! Replace value of K1
}
When the type of the element remains the same,
replacement of an attribute is the more efficient way.
 
Element definitions can be edited freely.
The changes do not affect already defined objects which belong to
its class (see Section~\ref{S-CLASS}).
The element to be modified should not be member of a beam line
sequence (see Section~\ref{S-SEQ}).
 
Beam elements or beam lines (see Section~\ref{S-LINDEF})
can be replaced by other beam elements or beam lines.
However, if a beam line is involved in the replacement, and the replaced
item occurs in the working beam line (see Section~\ref{S-USE}),
the working beam line becomes obsolete and is deleted automatically.
\pagebreak[1]
 
\section{Element Classes}
\label{S-CLASS}
\index{class}
\index{element!class}
The concept of element classes solves the problem of addressing
instances of elements in the accelerator in a convenient manner.
It will first be explained by an example.
All the quadrupoles in the accelerator form a class {\tt QUADRUPOLE}.
Let us define three subclasses for the focussing quadrupoles,
the defocussing quadrupoles, and the skewed quadrupoles:
\myxmp{
xxxxxxxxxxxxxxxxxxxxxxxxxxxxxxxxx\=\kill
MQF: QUADRUPOLE,L=LQM,K1=KQD     \>! Focussing quadrupoles \\
MQD: QUADRUPOLE,L=LQM,K1=KQF     \>! Defocussing quadrupoles \\
MQT: QUADRUPOLE,L=LQT,TILT       \>! Skewed quadrupoles
}
These classes can be thought of as new keywords which define
elements with specified default attributes.
We now use theses classes to define the real quadrupoles:
\myxmp{
xxxxxxxxxxxxxxxxxxx\=\kill
QD1: MQD           \>! Defocussing quadrupoles \\
QD2: MQD \\
QD3: MQD \\
... \\
QF1: MQF           \>! Focussing quadrupoles \\
QF2: MQF \\
QF3: MQF \\
... \\
QT1: MQT,K1=KQT1   \>! Skewed quadrupoles \\
QT2: MQT,K1=KQT2 \\
...
}
These quadrupoles inherit all unspecified attributes from their class.
This allows to build up a hierarchy of objects with a rather
economic input structure.
 
The full power of the class concept is revealed when object classes
are used to select instances of elements for printing.
Example:
\myxmp{
xxxxxxxxxxxxxxxxxxxxx\=\kill
PRINT,QUADRUPOLE     \>! Print at all quadrupoles \\
PRINT,MQT[1/2]       \>! Print at the first two skewed quadrupoles
}
More examples appear in Section~\ref{S-PRINT}.
 
More formally, for each element keyword MAD maintains a
class of elements with the same name.
A defined element becomes itself a class which can be used
to define new objects,
which will become members of this class.
A new object inherits all attributes from its class;
but its definition may override some of those values by new ones.
All class and object names can be used in range selections,
providing a powerful mechanism to specify positions
for matching constraints and printing.
\pagebreak[1]
 
\section{LUMP Element}
\label{S-LUMP}
\index{concatenation}
The command \ttindex{LUMP} concatenates transfer maps.
It has the form
\mybox{
label: LUMP,TYPE=name,ORDER=integer,LINE=line
}
It expands the beam line {\tt line},
and evaluates its transfer map to order {\tt ORDER}.
The map is associated with the name {\tt label}.
In other words this command pre-calculates a transfer map and
stores it for future use.
 
The {\tt ORDER} parameter must lie in the range
\(2\leq \hbox{\tt ORDER}\leq 6\).
It refers to the order of the corresponding Hamiltonian;
\(\hbox{\tt ORDER}=2\) denotes a linear transformation,
\(\hbox{\tt ORDER}=3\) includes all terms up to second order
in the TRANSPORT sense.
 
In a {\tt TWISS} command MAD considers terms up to \(\hbox{\tt ORDER}=3\).
To handle chromatic effects properly,
all {\tt LUMP}s used in a beam line should have \(\hbox{\tt ORDER}\geq 3\).
However, in a {\tt TWISS} command MAD ignores any terms with
\(\hbox{\tt ORDER}>3\).
MAD is not capable to find the integral part of the phase advance
over a {\tt LUMP}.
Thus, for \(\hbox{\tt ORDER}\geq 2\) \(\alpha,\beta,D\) are correct,
as well as the fractional parts of \(\mu\).
If \(\hbox{\tt ORDER}\geq 3\), the chromatic functions are also correct.
 
For tracking by the TRANSPORT method MAD limits the order of tracking
to three.
For tracking bye the Lie-algebraic method, it always uses the order
of the {\tt LUMP} definition.
For linear elements, tracking can be speeded up considerably,
if one selects \(\hbox{\tt ORDER}=2\).
 
Some important points must be kept in mind about this element:
\begin{itemize}
\item The transfer map is evaluated when used for the first time,
it is only re-evaluated when parameters of the line change.
\item Any sub-lines must be defined before using the {\tt LUMP}.
MAD may otherwise complain about undefined sub-lines when tracking.
\item The name {\tt label} behaves like an element,
it is not a beam line.
The individual elements building up the {\tt LUMP} cannot be
misaligned nor mispowered.
It is however permissible to misalign (not to mispower) the {\tt LUMP}
as a whole.
A {\tt LUMP} can also be used in a {\tt SURVEY} command.
It must not be used in a {\tt HARMON} run.
\item Concatenation of too many elements causes truncation
of high-order effects.
The number of concatenated elements should therefore not be too large.
\item As from Version 8.2/4 of MAD transfer maps are recalculated automatically
for a lump if one of the elements contained in this lump is changed.
Use of a lump in matching is always safe.
However lumping saves time only if the elements in a lump are not varied.
\end{itemize}
Examples for {\tt LUMP} definitions:
\myxmp{
xxxxxx\=\kill
SEC:  \>LINE=(QF,D,QD,D) \\
LSEC: \>LUMP,ORDER=4,LINE=SEC
}
Putting the line sequence directly in the {\tt LUMP} has the same effect:
\myxmp{
LSEC:LUMP,ORDER=4,LINE=(QF,D,QD,D)
}
In earlier versions of MAD it could save time when bending magnets
were defined as lumps.
From Version 8 MAD treats bending magnets as lumps automatically,
it will be useless if the user attempts to do so.
 
\chapter{Beam Lines, Sequences, Lists, Ranges, Selections}
\label{S-LINDEF}
\index{lines|(}
\index{definition!line}
 
\begin{table}[ht]
\label{T-LINDEF}
\caption{Line Definition Commands}
\vspace{1ex}
\centering
\begin{tabular}{|l|p{0.6\textwidth}|l|}
\hline
Name &Meaning &Section \\
\hline
\ttindex{LINE}&Beam line definition &\ref{S-LINE} \\
\ttindex{SEQUENCE}&Beam line sequence &\ref{S-SEQ} \\
{\tt LIST} &Replacement list definition &\ref{S-LIST} \\
\hline
\end{tabular}
\end{table}
 
\section{Beam Lines}
\label{S-LINE}
\index{beam line}
\index{line}
The accelerator to be studied is known to MAD
as a sequence of physical elements called a {\em beam line}.
A beam line is built from simpler beam lines whose definitions
can be nested to any level.
A powerful syntax allows to repeat, to reflect, or to replace
pieces of beam lines.
Formally a beam line is defined by a \ttindex{LINE} command:
\mybox{
label(arg\{,arg\}): LINE=(member\{,member\})
}
{\tt Label} gives a name to the beam line for later reference.
\index{label!line}
\index{line!label}
The formal argument list {\tt (arg\{,arg\})} is optional (see below).
Each {\tt member} may be one of the following:
\begin{itemize}
\item Element label,
\item Beam line label,
\item Sub-line, enclosed in parentheses,
\item Formal argument name,
\item Replacement list label.
\end{itemize}
Beam lines may be nested to any level.
 
\subsection{Simple Beam Lines}
\index{simple beam line}\index{beam line!simple}
The simplest beam line consists of single elements:
\myxmp{
label: LINE=(member\{,member\})
}
Example:
\myxmp{
xxxxxxxx\= \kill
L:      \>LINE=(A,B,C,D,A,D) \\
        \>USE,L
}
The {\tt USE} command tells MAD to perform all subsequent
calculations on the sequence
\myxmp{
A,B,C,D,A,D
}
 
\subsection{Sub-lines}
\index{sub-line}
Instead of referring to an element,
a beam line member can refer to another beam line
defined in a separate command.
This provides a shorthand notation for sub-lines which occur
several times in a beam line.
Lines and sub-lines can be entered in any order,
but when a line is expanded, all its sub-lines must be known.
Example:
\myxmp{
xxxxxxxx\= \kill
L:      \>LINE=(A,B,S,B,A,S,A,B) \\
S:      \>LINE=(C,D,E) \\
        \>USE,L
}
This example produces the following expansion steps:
\begin{enumerate}
\item Replace sub-line {\tt S}:
\myxmp{
(A,B,(C,D,E),B,A,(C,D,E),A,B)
}
\item Omit parentheses:
\myxmp{
A,B,C,D,E,B,A,C,D,E,A,B
}
\end{enumerate}
 
\subsection{Reflection and Repetition}
\index{reflection}\index{beam line!reflection}
\index{repetition}\index{beam line!repetition}
An unsigned repetition count and an asterisk indicate
repetition of a beam line member.
A minus prefix causes reflection,
i.e. all elements in the subsequence are taken in reverse order.
Sub-lines of reflected lines are also reflected,
but physical elements are not.
If both reflection and repetition are desired,
the minus sign must precede the repetition count.
Example:
\myxmp{
xxxxxxxx\= \kill
R:      \>LINE=(G,H) \\
S:      \>LINE=(C,R,D) \\
T:      \>LINE=(2*S,2*(E,F),-S,-(A,B)) \\
        \>USE,T
}
Proceeding step by step, this example produces
\begin{enumerate}
\item Replace sub-line {\tt S}:
\myxmp{
((C,R,D),(C,R,D),(E,F),(E,F),(D,-R,C),(B,A))
}
\item Replace sub-line {\tt R}:
\myxmp{
((C,(G,H),D),(C,(G,H),D),(E,F),(E,F),(D,(H,G),C),(B,A))
}
\item Omit parentheses:
\myxmp{
C,G,H,D,C,G,H,D,E,F,E,F,D,H,G,C,B,A
}
\end{enumerate}
Note that the inner sub-line {\tt R} is reflected together with
the outer sub-line {\tt S}.
 
\subsection{Replaceable Arguments}
\index{arguments}\index{beam line!arguments}
\index{formal arguments}
A beam line definition may contain a formal argument list,
consisting of labels separated by commas and enclosed in parentheses.
Such a line can be expanded for different values of its arguments.
When it is referred to,
its label must be followed by a list of actual arguments
separated by commas and enclosed in parentheses.
Each top level member of the list represents one argument;
it can contain anything which may occur as a beam line member.
MAD assumes an implicit pair of parentheses around each argument.
The number of actual arguments must agree with the number
of formal arguments.
All occurrences of a formal argument on the right-hand side of the
line definition are replaced by the corresponding actual argument.
Example:
\myxmp{
xxxxxxxx\= \kill
S:      \>LINE=(A,B,C) \\
L(X,Y): \>LINE=(D,X,E,3*Y) \\
        \>USE,L(4*F,-2*S)
}
Proceeding step by step, this example generates the expansion
\begin{enumerate}
\item Replace formal arguments:
\myxmp{
(D,(4*F),E,3*(-2*(A,B,C)))
}
\item Combine repetition counts for last sub-line:
\myxmp{
(D,(4*F),E,-6*(A,B,C))
}
\item Expand repetitions and reflection:
\myxmp{
(D,(F,F,F,F),E,(C,B,A),(C,B,A),(C,B,A),(C,B,A),(C,B,A),(C,B,A))
}
\item Omit parentheses:
\myxmp{
D,F,F,F,F,E,C,B,A,C,B,A,C,B,A,C,B,A,C,B,A,C,B,A
}
\end{enumerate}
Second example:
\myxmp{
xxxxxxxxxxxxxxxx\= \kill
CEL(SF,SD):     \>LINE=(QF,D,SF,D,B,D,QD,D,SD,D,B,D) \\
ARC:            \>LINE=(CEL(SF1,SD1),CEL(SF2,SD2),CEL(SF1,SD1)) \\
                \>USE,ARC
}
This example generates the expansion
\begin{enumerate}
\item Replace the line {\tt CEL} and its formal arguments:
\myxmp{
(\=(QF,D,(SF1),D,B,D,QD,D,(SD1),D,B,D) \\
 \>(QF,D,(SF2),D,B,D,QD,D,(SD2),D,B,D) \\
 \>(QF,D,(SF1),D,B,D,QD,D,(SD1),D,B,D))
}
\item Omit parentheses:
\myxmp{
QF,D,SF1,D,B,D,QD,D,SD1,D,B,D \\
QF,D,SF2,D,B,D,QD,D,SD2,D,B,D \\
QF,D,SF1,D,B,D,QD,D,SD1,D,B,D
}
\end{enumerate}
 
\section{Beam Line Sequences}
\label{S-SEQ}
\index{beam line!sequence}
\ttnindex{SEQUENCE}
\ttnindex{ENDSEQUENCE}
When MAD runs in an accelerator control system,
the sequence of elements is usually generated from a data base.
For this purpose there is a command
\mybox{
label: SEQUENCE,REFER=keyword \\
\ \ \ object: class,AT=real\{,attributes\} \\
\ \ \ ... \\
ENDSEQUENCE
}
It reads a sequence of element definitions,
\index{label!sequence}
\index{sequence!label}
compiles a data bank which resembles a beam line definition,
and gives it the name {\tt label}.
The resulting sequence can be used like a beam line.
For each non-drift element in the sequence one element definition appears
following the {\tt SEQUENCE} command
and preceding the {\tt ENDSEQUENCE} command.
These look like the definitions described
in Chapter~\ref{S-ELMDEF}, with the following differences:
\begin{itemize}
\item The name {\tt object} must be unique;
it must not be defined earlier in the data.
\item The second name {\tt class} on the definition must be a class name.
\item The class name {\em must be followed immediately} by the attribute
{\tt AT=real}.
This specifies the longitudinal position of the
element with respect to the beginning of the sequence.
The attribute \ttindex{REFER} specifies the reference points for the
elements:
\begin{mylist}
\ttitem{CENTRE}
The element centre is placed at the given position (default).
\ttitem{ENTRY}
The element entry is placed at the given position.
\ttitem{EXIT}
The element exit is placed at the given position.
\end{mylist}
The elements must be entered in order of increasing position,
and must not overlap.
\item The elements {\tt object} inherit the attributes from {\tt class}.
If no further attributes appear, MAD uses the data bank of {\tt class}
for the element data.
In this case {\tt class} cannot be modified subsequently.
If further attributes exist,
MAD makes a copy of the data for the class,
as if it had read a normal element definition.
\end{itemize}
When the sequence is expanded in a {\tt USE} command,
MAD generates the drift spaces for proper positioning.
 
For efficiency reasons MAD imposes an {\em important restriction}
for variable lengths.
If the lengths of elements and their positions are defined as constant values,
MAD gives the intervening drifts a constant length as well,
and checks that the drift lengths are all positive.
In this case the drift lengths can never be varied,
and a change in element length affects the total length of the system.
 
Variation of an element position and/or length is possible.
The element length or position to be varied {\em must} then
be defined by an expression.
This causes MAD to adjust the dependent drift length(s)
such as to keep all definitions consistent.
However, it is the responsibility of the program user that drift spaces
do not become negative.
Example:
\myxmp{
xxxxxxx\=\kill
! Define element classes for a simple cell: \\
B:     \>SBEND,L=35.09, ANGLE = 0.011306116 \\
QF:    \>QUADRUPOLE,L=1.6,K1=-0.02268553 \\
QD:    \>QUADRUPOLE,L=1.6,K1=0.022683642 \\
SF:    \>SEXTUPOLE,L=0.4,K2=-0.13129 \\
SD:    \>SEXTUPOLE,L=0.76,K2=0.26328 \\
! Define the cell as a sequence: \\
CELL:  \>SEQUENCE \\
xxx\=xxxxxxx\=xxxxxxxx\=\kill
   \>B1:    \>B,      \>AT=19.115 \\
   \>SF1:   \>SF,     \>AT=37.42 \\
   \>QF1:   \>QF,     \>AT=38.70 \\
   \>B2:    \>B,      \>AT=58.255,ANGLE=B1[ANGLE] \\
   \>SD1:   \>SD,     \>AT=76.74 \\
   \>QD1:   \>QD,     \>AT=78.20 \\
   \>ENDM:  \>MARKER, \>AT=79.0 \\
ENDSEQUENCE
}
In this example all members of the sequence use the storage of their
class except the element {\tt B2} which uses a copy because one of its
attribute has been redefined.
 
\section{Editing a Beam Line Sequence}
A beam line sequence \index{sequence} may be edited {\em before} it is
used in a \ttindex{USE} command.
Seven commands are available for this purpose:
\mybox{
SEQEDIT,SEQUENCE=name \\
INSTALL,ELEMENT=name,AT=value [,FROM=name] \\
REMOVE,ELEMENT=name \\
MOVE,ELEMENT=name [,BY=value] [,TO=value [,FROM=name]] \\
REFLECT \\
CYCLE,START=name \\
ENDEDIT
}
They are explained in detail in the next sub-sections.
 
\subsection{Making a sequence known for editing}
Before modifying a sequence, its name must be set for editing by
the command
\ttnindex{SEQEDIT}\ttnindex{SEQUENCE}
\mybox{
SEQEDIT,SEQUENCE=name
}
MAD will make the sequence {\tt name} the current sequence being edited.
Once editing is finished, one should give the command
\ttnindex{ENDEDIT}
\mybox{
ENDEDIT
}
 
\subsection{Installing an Element}
A new element can be installed in the edited sequence by the command
\ttnindex{INSTALL}
\mybox{
INSTALL,ELEMENT=name,AT=value [,FROM=name]
}
It has the following attributes:
\begin{mylist}
\ttitem{ELEMENT}
The name of the element to be installed.
\ttitem{AT}
The position where to install the element.
\ttitem{FROM}
The name of the element to be taken as the origin.
If the {\tt FROM} attribute does not appear,
{\tt AT} implies an absolute displacement from the beginning of the sequence;
if it does, the distance is relative to the position of that element.
A relative position may be negative.
\end{mylist}
 
\subsection{Removing Elements}
The command
\ttnindex{REMOVE}\ttnindex{ELEMENT}
\mybox{
REMOVE,ELEMENT=name,PATTERN=string
}
removes all elements from the sequence which match at least one
of the two following conditions:
\begin{itemize}
\item The element is member of the class {\tt name},
\item The element name matches the ``wild-card'' pattern {\tt string}.
\item{wild-card}
\end{itemize}
Examples:
\myxmp{
REMOVE,ELEMENT=DRIFT \\
REMOVE,PATTERN="S.*QD.*\.R1"
}
The first command removes all explicit drifts from the sequence,
and the second command removes the elements beginning with
the letter {\tt S}, containing the string {\tt QD},
and ending with the string {\tt .R1}.
Note that the two occurrences of {\tt .*} each stand for an arbitrary
number (including zero) of any character,
and the occurrence {\tt \(\backslash\).} stands for a literal period.
For definitions of patterns refer to Section~\ref{S-WILD}.
 
\subsection{Moving an Element}
The command
\ttnindex{MOVE}
\mybox{
MOVE,ELEMENT=name [,BY=value] [,TO=value [,FROM=name]]
}
moves an element to a new location.
Its attributes are:
\begin{mylist}
\ttitem{ELEMENT}
The name of the element to be moved.
\ttitem{BY}
The amount by which the element is to be moved.
\ttitem{TO}
The new position for the element.
\ttitem{FROM}
The name of the element to be taken as the origin.
If the {\tt FROM} attribute does not appear,
{\tt TO} implies an absolute displacement from the beginning of the sequence;
if it does, the distance is relative to the position of that element.
A relative position may be negative.
The {\tt MOVE} command may change the order of elements in the sequence.
\end{mylist}

\subsection{Reflecting a Sequence}
The command
\ttnindex{REFLECT}
\mybox{REFLECT}
reverses the order of all elements in the sequence currently being
edited.
All further edit commands must refer to the {\em new} positions.

\subsection{Cyclic interchange}
The command
\ttnindex{CYCLE}
\mybox{CYCLE,START=name}
makes a cyclic interchange of all elements in the current edit
sequence so as to start at the element {\tt name}.
This element should preferably be a {\tt MARKER}.
All further edit commands must refer to the {\em new} positions.

\subsection{Example for the Sequence Editor}
\myxmp{
SEQ:\=SEQUENCE \\
    \>B1:  B,  AT=19.115 \\
    \>SF1: SF, AT=37.42 \\
    \>QF1: QF, AT=38.70 \\
    \>B2:  B,  AT=58.255 \\
    \>SD1: SD, AT=76.74 \\
    \>QD1: QD, AT=78.20 \\
    \>ENDM: MARKER, AT=79.0 \\
ENDSEQ \\
\ \\
B2W:B2,ANGLE=0.1*B2[ANGLE] \\
SEQEDIT,SEQUENCE=SEQ \\
MOVE,ELEMENT=SF1,TO=-1.27,FROM=QF1 \\
MOVE,ELEMENT=SD1,BY=0.01 \\
REMOVE,ELEMENT=B2 \\
INSTALL,ELEMENT=B2W,AT=58.255 \\
ENDEDIT
}
This example moves the two sextupoles and replaces the element {\tt B2}
by the element {\tt B2W}.
 
\section{More Examples for Beam Lines}
\subsection{CERN SPS Lattice}
The CERN SPS lattice may be represented using the following
beam elements:
\myxmp{
xxxxxxxx\=xxxxxxxxxxxxxxxxx\= \kill
QF:     \>QUADRUPOLE,\ldots\>! focussing quadrupole \\
QD:     \>QUADRUPOLE,\ldots\>! defocussing quadrupole \\
B1:     \>RBEND,\ldots     \>! bending magnet of type 1 \\
B2:     \>RBEND,\ldots     \>! bending magnet of type 2 \\
DS:     \>DRIFT,\ldots     \>! short drift space \\
DM:     \>DRIFT,\ldots     \>! drift space replacing two bends \\
DL:     \>DRIFT,\ldots     \>! long drift space
}
The SPS machine is represented by the lines
\myxmp{
xxxxxxxx\= \kill
SPS:    \>LINE=(6*SUPER) \\
SUPER:  \>LINE=(7*P44,INSERT,7*P44) \\
INSERT: \>LINE=(P24,2*P00,P42) \\
P00:    \>LINE=(QF,DL,QD,DL) \\
P24:    \>LINE=(QF,DM,2*B2,DS,PD) \\
P42:    \>LINE=(PF,QD,2*B2,DM,DS) \\
P44:    \>LINE=(PF,PD) \\
PD:     \>LINE=(QD,2*B2,2*B1,DS) \\
PF:     \>LINE=(QF,2*B1,2*B2,DS)
}
In order not to overload the example,
small gaps between magnetic elements have been omitted.
 
\subsection{LEP Lattice}
A similar method of defining accelerator structures has been
used in Reference~\cite{B-LEP79} to define the LEP structure.
Translation of those element sequences to the MAD input format gives:
\myxmp{
xxxxxxxx\= \kill
LEP:    \>LINE=(4*SUP) \\
SUP:    \>LINE=(OCT,-OCT) \\
OCT:    \>LINE=(LOBS,RFS,DISS,ARC,DISL,RFL,LOBL) \\
LOBS:   \>LINE=(L1,QS1,L2,QS2,L3,QS3,L4,QS4) \\
RFS:    \>LINE=(L5,QS5,L5,QS6,L5,2*(QS7,L5,QS8,L5)) \\
DISS:   \>LINE=(\=QS11,L25,BW,L22,QS12,L25,B4,L22,QS13,L25,B4, \\
        \>      \>L22,QS14,L25,B4,L31,QS15,L25,B4,L32,SF,L23,QS16) \\
ARC:    \>LINE=(L21,B6,L22,SD,L23,QD, \\
        \>      \>7*(CELL(SF1,SD1),CELL(SF,SD)),CELL(SF1,SD1), \\
        \>      \>L24,B6,L41,QF,L21,B6,L22,SD4,L23,QD, \\
        \>      \>7*(CELL(SF4,SD3),CELL(SF3,SD4)),CELL(SF4,SD3), \\
        \>      \>L24,B6,L22,SF3,L23) \\
DISL:   \>LINE=(QL16,L34,B4,L22,QL15,L33,B4,L22,QL14,L25,B4, \\
        \>      \>L22,QL13,L25,B4,L22,QL12,L25,BW,L22,QL11) \\
RFL:    \>LINE=(2*(L5,QL8,L5,QL7),L5,QL6,L5,QL5,L5) \\
LOBL:   \>LINE=(QL4,L14,QL3,L13,QL2,L12,QL1,L11) \\
BW:     \>LINE=(W,L26,W) \\
B4:     \>LINE=(B,L26,B) \\
B6:     \>LINE=(B,L26,B,L26,B) \\
CELL(SF,SD): LINE=(L24,B6,L22,SF,L23,QF,L21,B6,L22,SD,L23,QD)
}
Here the element definitions have been left open.
 
\section{Beam Line References}
\label{S-LINAT}
\index{beam line!reference}
\index{line!reference}\index{reference!beam line}
Many MAD commands require a reference to a beam line.
Normally one enters the label of a beam line.
If the beam line is defined with formal arguments,
the label is followed by an actual argument list.
MAD also accepts a sequence enclosed by parentheses instead of a name.
Examples:
\myxmp{
USE,CELL1 \\
USE,CELL(SF1,SD1) \\
USE,(QF,D,QD,D)
}
 
\section{Replacement Lists}
\label{S-LIST}
\index{list!replacement}
\index{replacement list}
\index{substitution list}
\index{list!label}
\index{label!list}
The replacement list provides a powerful tool
to describe complex machines.
It replaces a set of identical labels in a sequence in turn
by different items.
A successive refinement of the machine description is made possible
by redefining elements as beam lines or as replacement lists.
A replacement list definition resembles a beam line definition:
\ttnindex{LIST}
\mybox{
label: LIST=(member\{,member\})
}
All items which can occur in a beam line can also occur in a list.
The effect of a replacement list on a beam line is defined as follows:
\begin{enumerate}
\item Expand all replacement lists to the level of single labels.
If a list member is a beam line name, do not expand it.
Mark the first label of each expanded list as current.
\item Expand the given beam line.
Each time the label of a replacement list is encountered,
replace it by the current label in this list
and mark the following label of the list as current.
\item If a substituted label is a beam line, expand it.
\end{enumerate}
To study several arrangements of sextupole families,
one may define one or more replacement lists for each
arrangement.
There is no need to change anything else in the machine structure.
Example:
\myxmp{
xxxxxxxx\= \kill
RING:   \>LINE=(5*(QF,D,SF,D,B,D,QD,D,SD,D,B)) \\
SF:     \>LIST=(2*(SF1,SF2),SF1) \\
SD:     \>LIST=(2*(SD1,SD2),SD1) \\
        \>USE,RING
}
In this example MAD proceeds as follows:
\begin{enumerate}
\item Expand the two lists:
\myxmp{
xxxxxxxx\= \kill
SF:     \>LIST=(SF1,SF2,SF1,SF2,SF1) \\
SD:     \>LIST=(SD1,SD2,SD1,SD2,SD1)
}
\item Expand the line {\tt RING}:
\myxmp{
xxxxxxxx\= \kill
RING:   \>LINE=(\=QF,D,SF,D,B,D,QD,D,SD,D,B,D, \\
        \>      \>QF,D,SF,D,B,D,QD,D,SD,D,B,D, \\
        \>      \>QF,D,SF,D,B,D,QD,D,SD,D,B,D, \\
        \>      \>QF,D,SF,D,B,D,QD,D,SD,D,B,D, \\
        \>      \>QF,D,SF,D,B,D,QD,D,SD,D,B,D)
}
\item Replace labels in {\tt RING}:
\myxmp{
xxxxxxxx\= \kill
RING:   \>LINE=(\=QF,D,SF1,D,B,D,QD,D,SD1,D,B,D, \\
        \>      \>QF,D,SF2,D,B,D,QD,D,SD2,D,B,D, \\
        \>      \>QF,D,SF1,D,B,D,QD,D,SD1,D,B,D, \\
        \>      \>QF,D,SF2,D,B,D,QD,D,SD2,D,B,D, \\
        \>      \>QF,D,SF1,D,B,D,QD,D,SD1,D,B,D) \\
}
\end{enumerate}
In a second example, we replace some drift spaces by special insertions:
\myxmp{
xxxxxxxx\= \kill
FODO:   \>LINE=(4*(QF,D,QD,D)) \\
D:      \>LIST=(DD,DC,3*DD,DC,2*DD) \\
DD:     \>DRIFT,L=2.0 \\
DC:     \>LINE=(D1,PICKUP,D1) \\
D1:     \>DRIFT,L=0.5 \\
PICKUP: \>MONITOR,L=1.0 \\
        \>USE,FODO
}
In this example we have the following steps:
\begin{enumerate}
\item Expand the list {\tt D} (note that {\tt DC} is not expanded yet):
\myxmp{
xxxxxxxx\= \kill
D:      \>LIST=(DD,DC,DD,DD,DD,DC,DD,DD)
}
\item Expand the beam line {\tt FODO}:
\myxmp{
xxxxxxxx\= \kill
FODO:   \>LINE=(QF,D,QD,D,QF,D,QD,D,QF,D,QD,D,QF,D,QD,D)
}
\item Replace labels in the beam line {\tt FODO}:
\myxmp{
xxxxxxxx\= \kill
FODO:   \>LINE=(\=QF,DD,QD,DC,QF,DD,QD,DD, \\ \+
                \>QF,DD,QD,DC,QD,DD,QD,DD)
}
\item Expand the line {\tt DC}:
\myxmp{
xxxxxxxx\= \kill
FODO:   \>LINE=(\=QF,DD,QD,D1,PICKUP,D1,QF,DD,QD,DD, \\ \+
                \>QF,DD,QD,D1,PICKUP,D1,QF,DD,QD,DD)
}
\end{enumerate}
Replacement lists are circular.
When a list is exhausted, and more replacement labels are required,
the current element is reset to the first one, and replacement continues.
Without changing the result,
the lists in the first example could therefore also be written as
\myxmp{
xxxxxxxx\= \kill
SF:     \>LIST=(SF1,SF2) \\
SD:     \>LIST=(SD1,SD2)
}
Replacement lists should {\em not} be used when lines are reflected.
\index{restrictions on lists}
\index{list!restrictions}
If a replacement list occurs within a reflected part of a beam line,
MAD never reflects the replacement list and gives a warning message.
Unless the list happens to be symmetric,
the expansion would be wrong.
Care should also be taken when expanding sub-lines of lines which refer
to replacement lists.
By definition, MAD always begins replacement at the first label
of a replacement list.
Here is an example what could go wrong:
\myxmp{
xxxxxxxx\= \kill
CELL1:  \>LINE=(\ldots,SF,\ldots) \\
CELL2:  \>LINE=(\ldots,SF,\ldots) \\
CELL3:  \>LINE=(\ldots,SF,\ldots) \\
SF:     \>LIST=(SF1,SF2,SF3) \\
ARC:    \>LINE=(CELL1,CELL2,CELL3) \\
        \>USE,CELL3
}
After the command {\tt USE, ARC} the cell {\tt CELL3} will contain
{\tt SF3}.
However, after the command {\tt USE, CELL3} the cell {\tt CELL3}
will contain {\tt SF1}.
In case of doubt, use the {\tt PRINT} command and check the correct
replacement.
 
\section{Ranges and Sets of Observation Points in a Line}
\index{observation point}
\index{selection point}
\index{range}
\index{fit point}
\label{S-OBSAT}
Many MAD commands allow for the possibility to process or display
a range or a subset of the elements occurring in the beam line expanded
by the latest {\tt USE} command.
There are two selection mechanisms:
\begin{itemize}
\item[\(\bullet\)]
A range (an interval) can be defined starting at the exit of a given element
and ending at the exit of another element.
The syntax is
\mybox{element-occurrence 1 / element-occurrence 2}
If the range consists of only one element or,
it may be entered as
\mybox{element-occurrence}
An {\tt element-occurrence} can be defined by
\begin{itemize}
\item[\(\circ\)] A MAD index:
\mybox{\#index}
{\tt index} is the number of physical elements preceding the position
in the expanded beam line.
There are four predefined MAD indices:
\begin{mylist}
\item[\tt \#S]
\index{S@{\tt \#S}}
The start of the {\em full} beam line expanded by {\tt USE},
\item[\tt \#E]
\index{E@{\tt \#E}}
The end of the {\em full} beam line expanded by {\tt USE},
\item[\tt \#F]
\index{F@{\tt \#F}}
The exit of the first element in {\tt RANGE} as read by {\tt USE},
\item[\tt \#L]
\index{L@{\tt \#L}}
The exit of the last element in {\tt RANGE} as read by {\tt USE}.
\end{mylist}
\item[\(\circ\)] A class or element name and its occurrence count:
\mybox{label[occurrence]}
where {\tt label} is a class or element name,
and {\tt occurrence} is the occurrence number of that name in the
expanded beam line.
\item[\(\circ\)] The explicit element name, if there is only one occurrence.
\mybox{label}
Names are always unique in a {\tt SEQUENCE}.
\end{itemize}
 
\item[\(\bullet\)] A set (a selection) of elements belonging to
the same class (e.~g. {\tt QUADRUPOLE}) or beam line is made by
\mybox{label[occurrence 1/occurrence 2]}
If {\tt label} is a class name,
this refers to all elements of that class,
starting at a {\tt occurrence 1} and ending at {\tt occurrence 2}.
If {\tt label} is a beam line name,
this refers to all elements belonging to that beam line,
starting at a {\tt occurrence 1} and ending at {\tt occurrence 2}.
Each {\tt occurrence} is an occurrence number for the class or beam line.
A single occurrence is selected by the syntax
\mybox{label[occurrence]}
If all occurrences are implied, one may simply use
\mybox{label}
 
\item[\(\bullet\)]
A range
\end{itemize}
Occurrences and physical elements are numbered relative to the {\em entire}
beam line expanded by the latest {\tt USE} command.
They are not affected when {\tt RANGE} is set in the {\tt USE} command.
 
\subsection{Examples}
\index{selection examples}
\index{observation points}
\index{range examples}
Assume the following definitions:
\myxmp{
xxx\= \kill
M: \>MARKER \\
S: \>LINE=(C,M,D) \\
L: \>LINE=(A,M,B,2*S,A,M,B) \\
   \>USE,L,RANGE=M[2]/M[3]
}
The line {\tt L} is equivalent to the sequence
\myxmp{
A,M,B,C,M,D,C,M,D,A,M,B
}
 
Examples for ranges:
\begin{mylist}
\item[{\tt \#4/\#6}]
{\tt C[1],M[2],D[1]},
\item[{\tt S[1]}]
The entire first sub-line {\tt S},
that is from entry of first {\tt S} to exit of first {\tt S},
both included,
\item[{\tt S}]
All positions within both occurrences of {\tt S},
including entries and exits of {\tt S},
\item[{\tt A[1]/A[2]}]
Positions {\tt A[1]} to {\tt A[2]} including all intermediate positions,
\item[{\tt M[2]/M[3]}]
From second to third {\tt M}, also known as {\tt \#F/\#L},
\item[{\tt L}]
The entire line {\tt L},
can also be written as {\tt \#S/\#E}.
\end{mylist}
 
Examples for single positions:
\begin{mylist}
\item[{\tt \#0 or \#S}]
At the beginning of the line {\tt L},
\item[{\tt C[1]}]
After the first element {\tt C}, can be written as {\tt \#4},
\item[{\tt M[3]}]
After the third marker {\tt M}, also {\tt \#8} or {\tt \#L},
\item[{\tt \#E}]
After the last element of the line {\tt L}.
\end{mylist}
 
Examples for multiple positions:
\begin{mylist}
\item[{\tt A}]
All {\tt A[i]}, that is {\tt A[1]} and {\tt A[2]},
\item[{\tt A[1/2]}]
{\tt A[1]} and {\tt A[2]} only,
\item[{\tt M}]
All {\tt M[i]}, that is {\tt M[1],M[2],M[3],M[4]},
can also be written as {\tt M[1/4]},
\item[{\tt M[2/3]}]
{\tt M[2]} and {\tt M[3]} only.
\end{mylist}
 
More examples can be found in Section~\ref{S-OPTICS}.
\index{lines|)}
 
\chapter{Action Commands}
\label{S-ACTION}
\index{action commands}\index{commands!actions}
 
\begin{table}[ht]
\label{T-ACTION}
\caption{Environment Setting and Action Commands}
\vspace{1ex}
\centering
\begin{tabular}{|l|p{0.6\textwidth}|l|}
\hline
Name &Meaning &Section \\
\hline
\ttindex{USE}&Select working beam line &\ref{S-USE} \\
\ttindex{BEAM}&
  Beam data: particle energy and charge, emittances etc.&\ref{S-BEAM} \\
\ttindex{PRINT}&Select print positions &\ref{S-PRINT} \\
\ttindex{SELECT}&Select output positions &\ref{S-PRINT} \\
\ttindex{SURVEY}&Print geometry of machine &\ref{S-SURVEY} \\
\ttindex{TWISS}&Print lattice functions &\ref{S-TWISS} \\
\ttindex{IBS}&Intra-beam scattering &\ref{S-IBS} \\
\ttindex{EMIT}&Equilibrium emittances &\ref{S-EMIT} \\
\ttindex{EIGEN}&Eigenvectors for normal modes &\ref{S-EIGEN} \\
\ttindex{ENVELOPE}&Beam envelopes in 3 degrees of freedom &\ref{S-ENVE} \\
\ttindex{TWISS3}&Mais-Ripken lattice functions &\ref{S-TWS3} \\
\ttindex{DYNAMIC}&Dynamic normal form analysis &\ref{S-LIE} \\
\ttindex{STATIC}&Static normal form analysis &\ref{S-LIE} \\
{\tt SPLIT} &Interpolate within element for {\tt OPTICS} command
&\ref{S-SPLIT} \\
{\tt OPTICS} &Output lattice functions and element strengths
for control system &\ref{S-OPTICS} \\
\hline
\end{tabular}
\end{table}
 
\section{USE Statement}
\label{S-USE}
\index{working beam line}
\index{main beam line}
\index{beam line!used}
The \ttindex{USE} statement specifies the beam line and its range
to be used in subsequent commands.
It must be entered before any physics computation is requested.
It has the form
\mybox{
USE,PERIOD=line,RANGE=range,SYMM,SUPER=integer
}
If the {\tt PERIOD} attribute appears,
the {\tt USE} command causes the following actions:
\begin{itemize}
\item Replace formal arguments by the corresponding actual arguments,
\item Insert sub-lines in the proper places,
\item Expand reflections and repetitions of sub-lines,
\item Reset all selection flags (see Section~\ref{S-PRINT}),
\item Clear all error definitions (see Chapter~\ref{S-ERROR}),
\end{itemize}
A data structure is generated containing
\index{element!sequence}
\index{sequence}
one entry for each beam element occurring in the beam line,
and one entry for the entrance and exit of each beam line or sub-line.
This structure remains in memory until another {\tt USE}
creates a new one,
or until the replacement of a beam element or beam line
makes it invalid.
A subsequent {\tt USE} command referring to the same beam line will
recreate the structure,
and effectively clear all selection flags and error definitions.
 
If the {\tt RANGE} attribute appears,
{\tt USE} marks beginning and end of {\tt range} for future use.
All subsequent commands which require a beam line operate on the
{\tt range} selected.
Example:
\myxmp{
xxxxxxxx\=xxxxxxxxxxxxxxxxx\= \kill
A:      \>LINE=(...,B,...) \>! definition of A \\
B:      \>LINE=(...)       \>! definition of B \\
        \>USE,A            \>! expand beam line A \\
        \>...              \>! some actions \\
B:      \>LINE=(...)       \>! redefinition of B kills sequence \\
        \>USE,A            \>! new expansion of beam line A \\
        \>...              \>! more actions \\
}
The {\tt USE} statement has four attributes:
\begin{mylist}
\ttitem{PERIOD}
The beam line to be expanded (see Section~\ref{S-LINAT}).
If omitted, the previous line is used without a new expansion.
\ttitem{RANGE}
The range of the beam line to be used.
If {\tt PERIOD} is given and {\tt RANGE} is omitted,
the range is the complete line.
If {\tt PERIOD} and {\tt RANGE} are both omitted,
the previous line and range are assumed.
If {\tt RANGE} is given, but {\tt PERIOD} omitted,
a new range is selected from the previous line.
\ttitem{SYMM}
A logical flag.
If set, subsequent calculations are made as if the mirror image had
been appended to the range.
\ttitem{SUPER}
\index{super-period}
An integer.
Specifies the number of superperiods desired in the calculations.
Quantities like tunes, chromaticities, and the like which refer to
the machine circumference will be scaled with the value of
{\tt SUPER} (default:~1).
\end{mylist}
Example:
\myxmp{
xxxxx\=xxxxxxxxxxxxxxxxxxxxxxxx\=\kill
OCT: \>LINE=(...)              \>! one octant of the machine \\
     \>USE,OCT,SYMM,SUPER=4
}
Here the beam line {\tt OCT} is expanded.
The effect is the same as if we had expanded the full machine as follows:
\myxmp{
TURN: LINE=(4*(OCT,-OCT))
}
However, MAD needs only to walk through one octant,
instead of through the whole machine.
If the beam line is defined with a formal argument list,
an actual argument list must be provided:
\myxmp{
CELL(SF,SD): \=LINE=(\ldots) \\
             \>USE,PERIOD=CELL(SF1,SD1)
}
In this example, one super-period is used.
The beam line {\tt CELL} is expanded with the formal arguments
{\tt SF} and {\tt SD} replaced by
{\tt SF1} and {\tt SD1} respectively.
It is neither repeated nor made symmetric,
since both {\tt SUPER} and {\tt SYMM} have been omitted.

 
\section{BEAM Command}
\label{S-BEAM}
Many commands in MAD require the setting of various quantities related
to the beam in the machine.
These are entered by a \ttindex{BEAM} command:
\mybox{
BEAM, \=PARTICLE=name,MASS=real,CHARGE=real,\& \\
      \>ENERGY=real,PC=real,GAMMA=real,\& \\
      \>EX=real,EXN=real,EY=real,EYN=real,\& \\
      \>ET=real,SIGT=real,SIGE=real,\& \\
      \>KBUNCH=integer,NPART=real,BCURRENT=real,\& \\
      \>BUNCHED=logical,RADIATE=logical
}
Warning: {\tt BEAM} updates, i.~e. it replaces attributes explicitely
mentioned, but does not return to default values for others!
To reset to defaults, use {\tt RESBEAM} (see below).
The particle mass and charge are defined by:
\begin{mylist}
\ttitem{PARTICLE}
The name of particles in the machine.
MAD knows the mass and the charge for the following particles:
\begin{mylist}
\ttitem{POSITRON}
The particles are positrons (default \hbox{\tt MASS}=\(m_e\),
\hbox{\tt CHARGE}=1),
\ttitem{ELECTRON}
The particles are electrons (\hbox{\tt MASS}=\(m_e\),
\hbox{\tt CHARGE}=-1),
\ttitem{PROTON}
The particles are protons (\hbox{\tt MASS}=\(m_p\),
\hbox{\tt CHARGE}=1),
\ttitem{ANTI-PROTON}
The particles are anti-protons (\hbox{\tt MASS}=\(m_p\),
\hbox{\tt CHARGE}=-1).
\end{mylist}
\end{mylist}
For other particle names one may enter:
\begin{mylist}
\ttitem{MASS}
\index{particle!mass}\index{mass}
The particle mass in GeV.
\ttitem{CHARGE}
\index{particle!charge}\index{charge}
The particle charge  expressed in elementary charges.
\end{mylist}
By default the total particle energy is 1~GeV.
A different value can be defined by one of the following:
\begin{mylist}
\ttitem{ENERGY}
\index{particle!energy}\index{energy}
\index{total energy}
The total energy per particle in GeV.
If given, it must be greater then the particle mass.
\ttitem{PC}
\index{momentum}
\index{particle!momentum}
The momentum per particle in GeV/c.
If given, it must be greater than zero.
\ttitem{GAMMA}
The ratio between total energy and rest energy of the particles
\(\gamma = E / m_{0}\).
If given, it must be greater than one.
If the mass is changed a new value for the energy should be entered.
Otherwise the energy remains unchanged,
and the momentum and \(\gamma\) are recalculated.
The emittances are defined by:
\ttitem{EX}
\index{emittance!horizontal}
The horizontal emittance
\(E_{x}=\sigma_{x}^{2}/\beta_{x}\)
(default:~1~m).
\ttitem{EY}
\index{emittance!vertical}
The vertical emittance
\(E_{y}=\sigma_{y}^{2}/\beta_{y}\)
(default:~1~m).
\ttitem{ET}
\index{emittance!longitudinal}
The longitudinal emittance
\(E_{t}=\sigma_e/(p_0c) \cdot c\sigma_{t}\)
(default:~1~m).
The emittances can be replaced
by the normalised emittances and the energy spread:
\ttitem{EXN}
\index{normalised emittance}\index{emittance!normalised}
The normalised horizontal emittance [m]:
\(E_{xn}=4\beta\gamma E_{x}\)
(ignored if \(E_{x}\) is given).
\ttitem{EYN}
The normalised vertical emittance [m]:
\(E_{yn}=4\beta \gamma E_{y}\)
(ignored if \(E_{y}\) is given).
\ttitem{SIGT}
\index{bunch!length}
The bunch length \(c\sigma_{t}\) [m].
\ttitem{SIGE}
\index{energy!spread}
The {\em relative} energy spread \(\sigma_{e}/p_0 c\) [1].
\end{mylist}
Certain commands compute the synchrotron tune \(Q_{s}\)
from the RF cavities.
If \(Q_{s}\neq 0\),
the relative energy spread \(\sigma_{e}/p_{0}c\)
and the bunch length \(c\sigma_{t}\) are
\[
   \frac{\sigma_e}{p_0c}=\sqrt{\frac{2\pi Q_s E_t}{\eta C}},
   \qquad
   c\sigma_{t}=\sqrt{\frac{\eta C E_t}{2\pi Q_s}},
\]
where \(C\) is the machine circumference, and
\(\eta = (1/\gamma^{2}) - (1/\gamma_{tr}^{2})\).
Finally, the {\tt BEAM} command accepts
\begin{mylist}
\ttitem{KBUNCH}
\index{bunch!number}
The number of particle bunches in the machine (default:~1).
\ttitem{NPART}
\index{bunch}
The number of particles per bunch (default:~0).
\ttitem{BCURRENT}
\index{beam!current}\index{bunch!current}
\index{current}
The bunch current (default:~0~A).
\ttitem{BUNCHED}
\index{bunch}
A logical flag.
If set, the beam is treated as bunched whenever this makes sense.
\ttitem{RADIATE}
\index{radiation}
\index{synchrotron!radiation}
A logical flag.
If set, synchrotron radiation is considered in all bipolar magnets.
\end{mylist}
The {\tt BEAM} command changes only the parameters entered.
The command
\mybox{
RESBEAM
}
resets all beam data to their defaults (listed in Table~\ref{T-BEAM}).
\index{default beam}
\index{beam!defaults}
It is defined in the command dictionary as
\myxmp{
RESBEAM:BEAM, \=PARTICLE="POSITRON",ENERGY=1.0,EX=1.0,EY=1.0,ET=1.0,\& \\
              \>KBUNCH=1.0,NPART=0.0,BCURRENT=0.0,-BUNCHED,-RADIATE,-LIST
}
\begin{table}[ht]
\caption{Default Beam Data}
\vspace{1ex}
\label{T-BEAM}
\centering
\begin{tabular}{|l|c|l||l|c|l|}
\hline
attribute      &value          &unit  &
attribute      &value          &unit \\
\hline
\ttindex{PARTICLE}&{\tt POSITRON} &      &
\ttindex{ENERGY}  &1              &GeV \\
\ttindex{EX}      &1              &rad m &
\ttindex{EY}      &1              &rad m \\
\ttindex{ET}      &1              &GeV s &
\ttindex{KBUNCH}  &1              &1 \\
\ttindex{NPART}   &0              &1     &
\ttindex{BCURRENT}&0              &A \\
\ttindex{BUNCHED} &false          &      &
\ttindex{RADIATE} &false          & \\
\hline
\end{tabular}
\end{table}
Examples:
\myxmp{
xxxxxxxx\= \kill
BEAM,   \>PARTICLE=ELECTRON,ENERGY=50,EX=1.E-6,EY=1.E-8,SIGE=1.E-3 \\
\ldots \\
BEAM,   \>RADIATE \\
\ldots \\
RESBEAM \\
BEAM,   \>EX=2.E-5,EY=3.E-7,SIGE=4.E-3
}
The first command selects electrons, and sets energy and emittances.
The second one turns on synchrotron radiation.
The last two select positrons (by default),
set the energy to 1~GeV (default),
clear the synchrotron radiation flag,
and set the emittances to the values entered.
 
The {\tt BEAM} command stores all specified attributes
in a data bank with the name {\tt BEAM}.
Some program modules of MAD also store data into this bank.
Real expressions may refer to data in the {\tt BEAM} bank
using the notation
\mybox{
BEAM[attribute-name]
}
This notation refers to the value of {\tt attribute-name}
found in the {\tt BEAM} bank {\tt before} the command containing the
reference is executed.
For example, after an {\tt EMIT} command both emittances
are stored in the {\tt BEAM} bank,
but the vertical emittance {\tt EY} may become zero.
For subsequent tracking one may set the {\em new} value of {\tt EY}
to \(1/2\)~times the {\em old} value of {\tt EX} by entering
\myxmp{
BEAM,EY=0.5*BEAM[EX]
}
 
\section{PRINT and SELECT Statements}
\label{S-PRINT}
\ttnindex{PRINT}\ttnindex{SELECT}
Each position in the beam line
carries several associated selection flags.
They are initially cleared by the \ttindex{USE} command
when the beam line is expanded.
Output is selected by setting some of these flags by one of the commands
\mybox{
PRINT,RANGE=range\{,range\},TYPE=type\{,type\},FULL,CLEAR \\
SELECT,FLAG=name,RANGE=range\{,range\},TYPE=type\{,type\},FULL,CLEAR
}
Both commands have the same attributes to select positions:
\begin{mylist}
\ttitem{RANGE}
With a single command up to five selection points or ranges
can be entered.
If more than five ranges are desired,
several commands must be used.
For definitions refer to Section~\ref{S-OBSAT}.
\ttitem{TYPE}
With a single command up to five {\tt TYPE} names can be entered.
All elements carrying one of these type attributes will be selected.
If more than five type names are desired,
several commands must be used.
\ttitem{FULL}
A logical flag.
If set, all flags of a specified type are set,
and no other processing of attributes occurs.
\ttitem{CLEAR}
A logical flag.
If set, all flags of a specified type are cleared
before they are set in selected positions.
\end{mylist}
The limits for the number of ranges or types may be changed by editing
the command dictionary.
The {\tt PRINT} command always affects the print flag for
{\tt SURVEY} (see Section~\ref{S-SURVEY}), or
{\tt TWISS} (see Section~\ref{S-TWISS}).
In {\tt SELECT} the flag type is chosen by the attribute \ttindex{FLAG}:
Three of its possible values affect action commands:
\begin{mylist}
\ttitem{TWISS}
A {\tt SELECT,TWISS} statement is equivalent to {\tt PRINT}.
The two commands
\myxmp{
PRINT,FULL \\
SELECT,FLAG=TWISS,FULL
}
have identical effect.
\ttitem{OPTICS}
Selects output positions for {\tt OPTICS} (see Section~\ref{S-OPTICS}).
\ttitem{TRACK}
Selects print positions for tracking (see Chapter~\ref{S-TRACK}).
Care must be taken in using this option, as it may generate a lot of
output.
\end{mylist}
Four more values, intended for the programmer, specify debugging output:
\begin{mylist}
\ttitem{FIRST}
Selects dumping of first-order transfer matrices for selected
elements during closed orbit search in {\tt TWISS}.
\ttitem{SECOND}
Selects dumping of second-order TRANSPORT maps for selected
elements during their accumulation in {\tt TWISS}.
\ttitem{REFER}
Selects dumping of first-order transfer matrices for selected
elements during accumulation for adjusting RF cavities.
\ttitem{LIE}
Selects dumping of Lie-algebraic maps during their accumulation.
\end{mylist}
{\tt PRINT} and/or {\tt SELECT} command(s)
must be placed after the {\tt USE} command,
and before any action command (e.g.~{\tt TWISS}) to be affected.
Regardless of the setting of print flags,
start and end points of the computation range are always printed
by {\tt NORMAL}, {\tt SURVEY}, and {\tt TWISS}.
Examples:
\myxmp{
xxxxxxxxxxxxxxxxxxxxxxx\= \kill
USE,OCT                \>! print at beginning and end only \\
PRINT,\#35/37          \>! print at positions number 35 to 37 \\
SELECT,TWISS,FULL      \>! set all print flags \\
PRINT,CLEAR            \>! clear all print flags \\
PRINT,OCT              \>! set all print flags \\
PRINT,CELL[3],CLEAR    \>! clear all flags, \\
                       \>! then set flags for all of third CELL
}
More examples can be found near the end of this chapter.
 
\section{SPLIT Command, Request Interpolation for OPTICS}
\label{S-SPLIT}
\ttnindex{SPLIT}
The \ttindex{OPTICS} command writes one table line for each element
selected by {\tt SELECT,OPTICS}.
The output line contains the element parameters
and the lattice functions for the element centre or its exit.
The command
\mybox{
SPLIT,RANGE=range{,range},TYPE=name{,name},FULL,CLEAR,FRACTION=real
}
selects additional positions for output of the lattice functions only.
Element parameters in those positions are output as zero.
The elements are selected as by \ttindex{SELECT}, and \ttindex{FRACTION}
specifies a fraction of the length of the element where output is
desired.
Any number of points can be selected in the same element;
output occurs in order of increasing \(s\).
Example:
\myxmp{
SELECT,B,FRACTION=0.25 \\
SELECT,B,FRACTION=0.5 \\
SELECT,B,FRACTION=0.75
}
Gives three lines for each {\tt B},
at \(1/4\),~\(1/2\), and~\(3/4\) of its length respectively.
 
\section{SURVEY Statement}
\label{S-SURVEY}
\index{geometry}
The \ttindex{SURVEY} command computes the geometry of the machine:
\mybox{
SURVEY, \=X0=real,Y0=real,Z0=real,\& \\
        \>THETA0=real,PHI0=real,PSI0=real,TAPE=file-name
}
It operates on the working beam line entered in the latest
{\tt USE} command (see Section~\ref{S-USE}).
Its parameter list specifies the initial position and orientation
of the reference orbit in the global coordinate system \((X,Y,Z)\).
Omitted attributes assume zero values.
Valid attributes are:
\begin{mylist}
\ttitem{X0}
The initial X coordinate [m].
\ttitem{Y0}
The initial Y coordinate [m].
\ttitem{Z0}
The initial Z coordinate [m].
\ttitem{THETA0}
The initial angle \(\theta\) [rad].
\ttitem{PHI0}
The initial angle \(\phi\) [rad].
\ttitem{PSI0}
The initial angle \(\psi\) [rad].
\ttitem{TAPE}
If {\tt TAPE=file-name} appears
MAD writes a full survey table on a disk file {\tt file-name}.
Appendix~\ref{A-TAPE3} describes the format of the file written.
Appendix~\ref{A-FILES} explains the format for file names.
{\tt TAPE} alone is equivalent to {\tt TAPE="survey"}.
\end{mylist}
{\tt SURVEY} prints one line for either end of the computation range
and a summary.
It also prints one line for each element and
for the entrance and exit of each beam line,
if this position has been selected by {\tt PRINT} or
by {\tt SELECT,FLAG=TWISS} (see Section~\ref{S-PRINT}).
The layout coordinates and angles have been defined in
Section~\ref{S-LAYOUT}.
Example:
\myxmp{
SURVEY, TAPE=LAYOUT
}
This example computes the machine layout with zero initial
conditions and writes the results on a file called {\tt LAYOUT}.
 
\section{TWISS Statement}
\label{S-TWISS}
\index{lattice functions}\index{functions!lattice}
\index{chromatic functions}\index{functions!chromatic}
The \ttindex{TWISS} command causes computation of the linear lattice
functions,
and optionally of the chromatic functions.
It operates on the working beam line defined in the latest
{\tt USE} command (see Section~\ref{S-USE}).
{\tt TWISS} prints one or two lines for either end
of the computation range and a summary.
It also prints one or two lines for each element and
for the entrance and exit of each beam line,
if this position has been selected by {\tt PRINT} or
by {\tt SELECT,FLAG=TWISS} (see Section~\ref{S-PRINT}).
The variables used have been defined in Section~\ref{S-VARIA}.
Three forms are distinguished for the {\tt TWISS} command.
In all three forms initial values for
the phase angles \ttindex{MUX} and \ttindex{MUY} are accepted.
\index{phase}
The relative energy error \ttindex{DELTAP} may be entered in one of the forms
\index{momentum error}\index{error!momentum}
\index{energy error}\index{error!energy}
\mybox{
DELTAP=real{,real} \\
DELTAP=initial:final:step
}
The first form lists several numbers, which may be general expressions,
separated by commas.
The second form specifies an {\tt initial} value, a {\tt final} value,
and a {\tt step}, which must be constant expressions,
separated by colons.
Mixtures of both forms are accepted.
Examples:
\myxmp{
xxxxxxxxxxxxxxxxxxxxxxxxxxxxxxxx\= \kill
DELTAP=0.001                    \>! a single value \\
DELTAP=0.001,0.005              \>! two values \\
DELTAP=0.001:0.007:0.002        \>! four values \\
DELTAP=0.001,0.005:0.03:0.005   \>! seven values
}
If \ttindex{DELTAP} is missing, MAD uses the value 0.0.
Further attributes common to all three forms
of the {\tt TWISS} statements are:
\begin{mylist}
\ttitem{CHROM}
A logical flag.
If set, MAD also computes the chromatic functions defined in~\cite{B-MON79}.
\ttitem{COUPLE}
\index{coupling}
A logical flag.
If set,
MAD computes the coupled linear lattice functions as defined
in~\cite{B-EDW72,B-TEN71}.
In this case {\tt CHROM} is ignored.
\ttitem{TAPE}
If {\tt TAPE=file-name} appears
MAD writes a full Twiss table on a disk file {\tt file-name}.
Appendix~\ref{A-TAPE3} describes the format of the file written.
Appendix~\ref{A-FILES} explains the format for file names.
{\tt TAPE} alone is equivalent to {\tt TAPE="twiss"}.
\ttitem{SAVE}
If {\tt SAVE=table-name} appears on the command,
MAD creates a full Twiss table in memory and gives it the name
{\tt table-name} (an identifier).
Entering {\tt SAVE} alone is equivalent to {\tt SAVE=TWISS}.
\index{lattice functions}\index{functions!lattice}
This table includes linear lattice functions as well as the chromatic
functions for all positions.
\ttitem{TUNES}
If {\tt TUNES=table-name} appears on the command,
MAD creates a table of tunes and chromaticities versus the selected
values of {\tt DELTAP} and gives it the name {\tt table-name}
(an identifier).
Entering {\tt TUNES} alone is equivalent to {\tt TUNES=TUNES}.
\index{tune}
\index{chromaticity}
\end{mylist}
The tables are suited for plotting (see Chapter~\ref{S-PLOT}).
They may be written on disk by the {\tt ARCHIVE command}
(see Section~\ref{S-ARCHRET}).
 
\subsection{Twiss Parameters for a Period}
\index{periodic solution}
The simplest form of the \ttindex{TWISS} command is
\mybox{
TWISS, \=DELTAP=real\{,value\},CHROM,COUPLE,\& \\
       \>TAPE=file-name,SAVE=table-name
}
\ttnindex{CHROM}
\ttnindex{COUPLE}
\ttnindex{TAPE}
\ttnindex{SAVE}
It computes the periodic solution for the specified beam
line for all values of \ttindex{DELTAP} entered
(or for {\tt DELTAP}~=~0, if none is entered).
Example:
\myxmp{
USE,OCT,SYMM,SUPER=4 \\
TWISS,DELTAP=0.001,CHROM,TAPE=OPTICS
}
This example computes the periodic solution for the linear lattice
and chromatic functions for the beam line {\tt OCT},
made symmetric and repeated in four superperiods.
The {\tt DELTAP} value used is 0.001.
Apart from saving computing time,
it is equivalent to the command sequence
\myxmp{
RING: \=LINE=(4*(OCT,-OCT)) \\
      \>USE,RING \\
      \>TWISS,DELTAP=0.001,CHROM,TAPE=OPTICS
}
 
\subsection{Initial Values for Twiss Parameters taken from a Periodic
Line}
\index{initial values, lattice}
It is often useful to track the lattice functions
starting with the periodic solution for another beam line.
If this is desired the \ttindex{TWISS} command takes the form
\mybox{
TWISS, \=DELTAP=real\{,value\},LINE=beam-line,\& \\
       \>MUX=real,MUY=real\& \\
       \>TAPE=file-name,SAVE=table-name
}
\ttnindex{LINE}
\ttnindex{MUX}
\ttnindex{MUY}
\ttnindex{TAPE}
\ttnindex{SAVE}
No other attributes should appear in the command.
For each value of \ttindex{DELTAP} MAD first searches for the periodic
solution for the beam line mentioned in {\tt LINE=beam-line}.
The result is used as an initial condition for the lattice function
tracking.
In this form of the \ttindex{TWISS} command
the \ttindex{SYMM} flag and the value for \ttindex{SUPER} are ignored.
Example:
\myxmp{
xxxxxxxx\= \kill
CELL:   \>LINE=(\ldots) \\
INSERT: \>LINE=(\ldots) \\
        \>USE,INSERT \\
        \>TWISS,LINE=CELL,DELTAP=0.0:0.003:0.001,CHROM,TAPE
}
For four values of \ttindex{DELTAP} the following happens:
First MAD finds the periodic solution for the beam line {\tt CELL}.
Then it uses this solution as initial conditions for tracking
the lattice functions of the beam line {\tt TWISS}.
The initial phases are not specified, they are set to zero.
Output is also written on the file {\tt TWISS}.
 
If any of the beam lines was defined with formal arguments,
actual arguments must be provided:
\myxmp{
CELL(SF,SD): \=LINE=(\ldots) \\
INSERT(X):   \>LINE=(\ldots) \\
             \>USE,INSERT \\
             \>TWISS,LINE=CELL(SF1,SD1)
}
 
\subsection{Twiss Parameters with Numerical Initial Values}
\index{initial values, lattice}
\label{TWISS}
Initial values for linear lattice functions may also be numerical.
Two equivalent methods are available to specify initial values:
\begin{itemize}
\item In the \ttindex{TWISS} command itself, as in
\mybox{
xxxxxxxxx\=\kill
TWISS,   \>BETX=real,ALFX=real,MUX=real,\& \\
         \>BETY=real,ALFY=real,MUY=real,\& \\
         \>DX=real,DPX=real,DY=real,DPY=real,\& \\
         \>X=real,PX=real,Y=real,DPY=real,\& \\
         \>WX=real,PHIX=real,DMUX=real,\& \\
         \>WY=real,PHIY=real,DMUY=real,\& \\
         \>DDX=real,DDY=real,DDPX=real,DDPY=real,\& \\
         \>TAPE=file-name,SAVE=table-name,\& \\
         \>DELTAP=real:real:real
}
\ttnindex{BETX}
\ttnindex{ALFX}
\ttnindex{MUX}
\ttnindex{BETY}
\ttnindex{ALFY}
\ttnindex{MUY}
\ttnindex{DX}
\ttnindex{DPX}
\ttnindex{DY}
\ttnindex{DPY}
\ttnindex{X}
\ttnindex{PX}
\ttnindex{Y}
\ttnindex{PY}
\ttnindex{T}
\ttnindex{PT}
\ttnindex{WX}
\ttnindex{PHIX}
\ttnindex{DMUX}
\ttnindex{WY}
\ttnindex{PHIY}
\ttnindex{DMUY}
\ttnindex{DDX}
\ttnindex{DDY}
\ttnindex{DDPX}
\ttnindex{DDPY}
\ttnindex{TAPE}
\ttnindex{SAVE}
\ttnindex{DELTAP}
\item In a separate command, \ttindex{BETA0}
(see also Section~\ref{S-SVBT}):
\mybox{
INITIAL: \=BETA0,\& \\
         \>BETX=real,ALFX=real,MUX=real,\& \\
         \>BETY=real,ALFY=real,MUY=real,\& \\
         \>DX=real,DPX=real,DY=real,DPY=real,\& \\
         \>X=real,PX=real,Y=real,PY=real,T=real,PT=real,\& \\
         \>WX=real,PHIX=real,DMUX=real,\& \\
         \>WY=real,PHIY=real,DMUY=real,\& \\
         \>DDX=real,DDY=real,DDPX=real,DDPY=real \\
TWISS,   \>BETA0=INITIAL,TAPE=file-name,SAVE=table-name,\& \\
         \>DELTAP=real:real:real
}
\end{itemize}
In the second case the {\tt TWISS} command may override some of the
values set by {\tt BETA0}.
All variables listed in Section~\ref{S-LINLAT} are permitted
as input attributes,
but \ttindex{BETX} and \ttindex{BETY} are required in either case.
The flags \ttindex{SYMM} and \ttindex{COUPLE},
and the value for \ttindex{SUPER} are ignored
in this use of the {\tt TWISS} command.
As entered in this command,
the initial conditions cannot depend on the quantity \ttindex{DELTAP},
and can thus be correct only for one such value.
This should be remembered when using this form
of the \ttindex{TWISS} command.
 
\subsection{SAVEBETA Command, Save Lattice Parameters for Later Use}
\label{S-SVBT}
\index{lattice parameters}
Sometimes it is useful to transfer computed lattice parameters to
later commands.
The command sequence
\mybox{
USE,... \\
SAVEBETA,LABEL=name,PLACE=place \\
TWISS,...
}
may help to do this.
When reaching the place {\tt place} during execution of \ttindex{TWISS}
MAD will create a \ttindex{BETA0} bank with the name {\tt name}.
This bank is filled with the values of all lattice parameters
in {\tt place}.
The \ttindex{SAVEBETA} command has two attributes:
\begin{mylist}
\ttitem{LABEL}
The name to be given to the created \ttindex{BETA0} bank.
\ttitem{PLACE}
A position within the selected range of the working beam line.
\end{mylist}
Example~1:
\myxmp{
USE,CELL \\
SAVEBETA,LABEL=END,PLACE=\#E \\
TWISS \\
USE,INSERT \\
TWISS,BETA0=END
}
This will first calculate the periodic solution of the line {\tt CELL},
and then track lattice parameters through {\tt INSERT}, using all end
conditions (including phases and orbit) in {\tt CELL} to start.
 
Example~2:
\myxmp{
USE,CELL \\
SAVEBETA,LABEL=END,PLACE=\#E \\
TWISS \\
USE,INSERT \\
TWISS,BETX=END[BETY],BETY=END[BETX]
}
This is similar to the first example,
but the beta functions are interchanged (overwritten).
 
\section{IBS Command, Intra-Beam Scattering}
\label{S-IBS}
\index{intra-beam scattering}
\index{scattering (intra-beam)}
The \ttindex{IBS} command takes an internal table (default:~{\tt TWISS})
generated by a previous {\tt TWISS,SAVE} command
and prints out information about intra-beam scattering.
This command uses the method by Bjorken and Mtingwa~\cite{B-BM,B-CM}.
\mybox{
IBS,TABLE=table
}
This single attribute \ttindex{TABLE} specifies the Twiss table to be
used (default:~TWISS).
The \ttindex{IBS} command uses the same print flags as {\tt TWISS}
to select output positions.
Example:
\myxmp{
TWISS,SAVE=LATTICE \\
IBS,TABLE=LATTICE
}
 
\section{Emittances, Eigenvectors, Normal Modes, Beam Envelope}

\subsection{Electron Beam Parameters}
\label{S-EMIT}
The command
\mybox{EMIT,DELTAP=real}
adjusts the RF~frequencies such as to obtain the specified average
energy error.
More precisely,
the revolution frequency \(f_0\) is determined for a fictitious particle
with constant momentum error
\(\delta p / p_0 c = \delta_s = \hbox{\tt DELTAP}\)
which travels along the design orbit.
The RF~frequencies are then set to \(h f_0\).

If the machine contains at least one RF~cavity, and if synchrotron
radiation is on (Section~\ref{S-BEAM}),
the {\tt EMIT} command computes the equilibrium emittances and other
electron beam parameters using the method of A.~Chao~\cite{B-CHAO}.
In this calculation the effects of quadrupoles, sextupoles, and
octupoles along the closed orbit is also considered.
Thin multipoles are ignored,
unless they have a fictitious length {\tt LRAD} different from zero.

If the machine contains no RF~cavity, or if synchrotron radiation is
off, it only computes the parameters which are not related to
radiation.
Example:
\myxmp{
RFC: \=RFCAVITY,HARMON...,VOLT=... \\
     \>BEAM,ENERGY=100.0,RADIATE \\
     \>EMIT,DELTAP=0.01
}

\subsection{Representations of Beam with Full Coupling}
\label{S-EIGEN}
\label{S-ENVE}
\label{S-TWS3}
The three commands
\mybox{
EIGEN,SAVE=name \\
ENVELOPE,SAVE=name,SIGMA0=name,LINE=line \\
TWISS3,SAVE=name,LINE=line
}
track different representations of the beam in three degrees of
freedom.
If the ring contains RF~cavities, the command \ttindex{EMIT} should be run
before any of these commands in order to adjust the RF~frequencies.
Their actions of are the following:

\begin{mylist}
\ttitem{EIGEN}
Tracks the eigenvectors for the three eigen-modes and prints the
principal phase for each of them.
Example:
\myxmp{
RFC: \=RFCAVITY,HARMON...,VOLT=... \\
     \>BEAM,ENERGY=100.0,RADIATE \\
     \>EMIT,DELTAP=0.01\\
     \>EIGEN,SAVE=name \\
}
\ttitem{ENVELOPE}
Tracks the beam envelope in a way similar to the TRANSPORT program.
This uses the eigenvectors and emittances for the three modes to
determine the initial beam envelope,
and tracks this envelope along the ring.
Example:
\myxmp{
RFC: \=RFCAVITY,HARMON...,VOLT=... \\
     \>BEAM,ENERGY=100.0,RADIATE \\
     \>EMIT,DELTAP=0.01 \\
     \>ENVELOPE,SAVE=name,SIGMA0=name,LINE=line
}
\ttitem{TWISS3}
Lists the coupled lattice functions defined in two papers by
H.~Mais~\cite{B-MAI82} and G.~Ripken~\cite{B-RIP70}.
These are essentially the projections of the lattice functions for the
eigen-modes on the three planes.
Example:
\myxmp{
RFC: \=RFCAVITY,HARMON...,VOLT=... \\
     \>BEAM,ENERGY=100.0,RADIATE \\
     \>EMIT,DELTAP=0.01 \\
     \>TWISS3,SAVE=name,LINE=line
}
\end{mylist}

The attributes mean:

\begin{mylist}
\ttitem{SAVE}
If this keyword is present,
it requests saving of the computed functions in a table.
It may be followed by a name to be given to the table.
If it appears without a value the table has the same name as the
command.
If the \ttindex{SAVE} keyword is omitted, no table is generated.
If the table is to be processed outside MAD, it must be written to
disk using the {\tt ARCHIVE} command (Section~\ref{S-ARCHRET}).
The positions appearing in the tables are selected by means of the
command
\mybox{
SELECT,OPTICS,RANGE=...
}
in a way similar to the {\tt OPTICS} command.
\ttitem{SIGMA0}
May contain the name of a {\tt SIGMA0} block,
to be used as the initial beam envelope in \ttindex{ENVELOPE}.
\ttitem{LINE}
May specify a beam line whose periodic solution should be taken for
initial condition in \ttindex{ENVELOPE} or \ttindex{TWISS3}.
\end{mylist}

For all three commands printing and selection for saving is controlled
by the {\tt PRINT} command (\ref{S-PRINT}).
A \ttindex{SIGMA0} block can be generated by two means.
First, it may be entered like any definition:
\mybox{
XXXXXXXXXXXXXX\=\kill
name:SIGMA0,  \>X,PX,Y,PY,T,PT,DX,DPX,DY,DPY,DT,DPT,\& \\
              \>SIGX,R21,R31,R41,R51,R61,SIGPX,R32,\& \\
              \>R42,R52,R62,SIGY,R43,R53,R63,SIGPY,\& \\
              \>R54,R64,SIGT,R65,SIGPT
}
The first group of six values defines the initial orbit.
For a static machine (constant energy) the second group of six values
defines the initial dispersion.
The values \ttindex{SIGxx} are the standard deviations and the
\ttindex{Rij} the correlations defining the beam ellipsoid in
TRANSPORT sense.
\label{S_SAVSIG}

A \ttindex{SIGMA0} block can also be generated and used as follows:
\mybox{
XXXXXXXXXXXXXX\=\kill
              \>USE,RING \\
              \>SAVESIGMA,LABEL=SIGMAHERE,PLACE=HERE \\
              \>EMIT ! required only if RF cavities are present \\
              \>ENVELOPE \\
              \>... \\
              \>USE,RING,RANGE=HERE/THERE \\
              \>ENVELOPE,SIGMA0=SIGMAHERE
}
\ttnindex{SAVESIGMA}
\ttnindex{LABEL}
\ttnindex{PLACE}
The first envelope command computes the periodic beam envelope for the
line {\tt RING}, and fills in the {\tt SIGMA0}-block with the name
{\tt SIGMAHERE}.
The second {\tt ENVELOPE} uses the values in position {\tt HERE} to
track the envelope from {\tt HERE} to {\tt THERE}.
 
\section{DYNAMIC and STATIC, Lie-Algebraic Analysis}
\label{S-LIE}
\ttnindex{DYNAMIC}
\ttnindex{STATIC}
The two commands
\mybox{
DYNAMIC, \=DELTAP=real,MAP,ORBIT,A,N,\& \\
         \>RESONANCE,EXPONENT,HAMILTON,INVARIANT \\
 \\
STATIC,  \>DELTAP=real,MAP,ORBIT,FIXED,T,A,N,\& \\
         \>RESONANCE,EXPONENT,HAMILTON,BETATRON,\& \\
         \>NONLINEAR,CONJUGATE,INVARIANT
}
\ttnindex{RFCAVITY}
\ttnindex{DELTAP}
\ttnindex{MAP}
\ttnindex{ORBIT}
\ttnindex{FIXED}
\ttnindex{T}
\ttnindex{A}
\ttnindex{N}
\ttnindex{RESONANCE}
\ttnindex{EXPONENT}
\ttnindex{HAMILTON}
\ttnindex{BETATRON}
\ttnindex{NONLINEAR}
\ttnindex{CONJUGATE}
\ttnindex{INVARIANT}
both evaluate the fourth-order Lie~transformation for one turn
and print the eigenvectors and lattice parameters at the end
of the system.
Both have various options to print selected maps and functions.
The differences are the following:
\begin{mylist}
\ttitem{DYNAMIC}
The RF~cavities are adjusted as for the command \ttindex{NORMAL}.
The transfer map is interpreted as dynamic, i.e. there is synchrotron
motion with an average momentum error \ttindex{DELTAP}.
\ttitem{STATIC}
The program assumes that there are no cavities,
and it interprets the transfer map as static,
i.e. having constant momentum error {\tt DELTAP}.
\end{mylist}
Both commands have the following logical flags:
\begin{mylist}
\ttitem{MAP}
Print the original map to be analysed.
\ttitem{ORBIT}
Print the map around the closed orbit.
\ttitem{A}
Print the conjugating map
(the transformation from the closed orbit to normal form).
\ttitem{N}
Print the normal form map.
\ttitem{RESONANCE}
Print the resonance coefficients left in the normal form.
\ttitem{EXPONENT}
Print the exponent for the normal form.
\ttitem{HAMILTON}
Print the Hamiltonian for the normal form.
\ttitem{INVARIANT}
Print the linear invariants.
\end{mylist}
The \ttindex{DYNAMIC} command adjusts the RF~frequencies according
to the revolution frequencies.
It uses the following attributes for this purpose:
\begin{mylist}
\ttitem{RFCAVITY}
The name of a class of cavities whose phase lag is to be adjusted
to make the time lag of the closed orbit approximately zero.
\ttitem{DELTAP}
The average momentum error \(\Delta p/p_0 c\) for the closed orbit.
\end{mylist}
The \ttindex{STATIC} command also finds the fixed point of the map,
that is the variation of the closed orbit with momentum.
It has the following additional attributes:
\begin{mylist}
\ttitem{DELTAP}
The constant momentum error \(\Delta p/p_0 c\) for the closed orbit.
\ttitem{FIXED}
Print the map about the fixed point.
\ttitem{T}
Print the transformation to the fixed point.
\ttitem{BETATRON}
Print the betatron map.
\ttitem{NONLINEAR}
Print the non-linear factor of the betatron map.
\ttitem{CONJUGATE}
Print the map conjugate to the betatron factor.
\end{mylist}
 
\section{OPTICS Command}
\label{S-OPTICS}
Control system applications often require tables with
arbitrary selections of element data and/or lattice functions.
Such tables can be generated by the command
\mybox{
OPTICS, \=BETX=real,ALFX=real,MUX=real,\& \\
        \>BETY=real,ALFY=real,MUY=real,\& \\
        \>DX=real,DPX=real,DY=real,DPY=real,\& \\
        \>X=real,PX=real,Y=real,PY=real,T=real,PT=real\& \\
        \>WX=real,PHIX=real,DMUX=real,\& \\
        \>WY=real,PHIY=real,DMUY=real,\& \\
        \>DDX=real,DDY=real,DDPX=real,DDPY=real,\& \\
        \>LINE=beam-line,BETA0=name,CENTRE,\& \\
        \>FILENAME=file-name,DELTAP=real:real:real,\& \\
        \>COLUMNS=name\{,name\}
}
\ttnindex{OPTICS}
\ttnindex{BETX}
\ttnindex{ALFX}
\ttnindex{MUX}
\ttnindex{BETY}
\ttnindex{ALFY}
\ttnindex{MUY}
\ttnindex{DX}
\ttnindex{DPX}
\ttnindex{DY}
\ttnindex{DPY}
\ttnindex{X}
\ttnindex{PX}
\ttnindex{Y}
\ttnindex{PY}
\ttnindex{T}
\ttnindex{PT}
\ttnindex{WX}
\ttnindex{PHIX}
\ttnindex{DMUX}
\ttnindex{WY}
\ttnindex{PHIY}
\ttnindex{DMUY}
\ttnindex{DDX}
\ttnindex{DDY}
\ttnindex{DDPX}
\ttnindex{DDPY}
\ttnindex{LINE}
\ttnindex{BETA0}
\ttnindex{FILENAME}
\ttnindex{DELTAP}
\ttnindex{COLUMNS}
It first creates a table in memory.
It then writes it on a disk file in coded TFS format
(see Appendix~\ref{S-TFS}).
Initial conditions for the optical functions are specified like
for the {\tt TWISS} command.
They are explained in Sections~\ref{S-VARIA} and~\ref{S-TWISS}.
Further attributes are:
\begin{mylist}
\ttitem{CENTRE}
Normally output occurs at the exit of each selected element.
If the {\tt CENTRE} flag is on,
output occurs at the centre of each selected element;
\ttitem{FILENAME}
The output is written on the file {\tt file-name}
(default:~{\tt optics}).
Note that the resulting table is written on this file and erased from
the computer's memory.
In order to plot it, it must be read back by an \ttindex{RETRIEVE}
command.
\ttitem{COLUMNS}
Up to 50 table columns may be selected by name for output.
All optical and chromatic functions listed
in Sections~1.5.1, 1.5.3, and~1.5.4 are accepted.
Further possibilities are:
\begin{mylist}
\ttitem{NAME}
The element name.
\ttitem{CLASS}
The smallest class containing the element as an object.
\ttitem{KEYWORD}
The element keyword.
\ttitem{TYPE}
The {\tt TYPE} attribute of the element.
\ttitem{S}
The longitudinal position.
\ttitem{DP}
The current \(\Delta E / p_s\).
It is not called {\tt DELTAP} to avoid conflicts with the command
attribute of this name which selects a single value for the calculation.
\ttitem{L}
The element length.
\ttitem{RADLOSS}
The systematic part of the relative radiation loss for this element.
\ttitem{K0L}
The dipole component of this element (excluding kicks).
\ttitem{KnL}
The integrated multipole component. The value of \(n\) specifies the
multipole with \(2 (n + 2)\) poles, and \(0 \le n \le 9\).
\ttitem{VOLT}
The cavity voltage.
\ttitem{LAG}
The RF frequency lag.
\ttitem{FREQ}
The RF frequency.
\ttitem{HARMON}
The RF harmonic number.
\ttitem{TILT}
The roll angle of the element, zero for multipoles.
\ttitem{KS}
The integrated solenoid strength.
\ttitem{HKICK}
The horizontal corrector deflection.
\ttitem{VKICK}
The vertical corrector deflection.
\ttitem{E1}
The entrance pole face angle for dipoles.
\ttitem{E2}
The exit pole face angle for dipoles.
\ttitem{H1}
The entrance pole face curvature for dipoles.
\ttitem{H2}
The exit pole face curvature for dipoles.
\ttitem{EFIELD}
The electrostatic field for a separator.
\end{mylist}
\end{mylist}
Example:
\myxmp{
xxxxxxx\=\kill
! Define element classes for a simple cell: \\
B:     \>SBEND,L=35.09, ANGLE = 0.011306116 \\
QF:    \>QUADRUPOLE,L=1.6,K1=-0.02268553 \\
QD:    \>QUADRUPOLE,L=1.6,K1=0.022683642 \\
SF:    \>SEXTUPOLE,L=0.4,K2=-0.13129 \\
SD:    \>SEXTUPOLE,L=0.76,K2=0.26328 \\
! Define the cell as a sequence: \\
CELL:  \>SEQUENCE \\
xxx\=xxxxxxx\=xxxxxxxx\=\kill
   \>B1:    \>B,      \>AT=19.115 \\
   \>SF1:   \>SF,     \>AT=37.42 \\
   \>QF1:   \>QF,     \>AT=38.70 \\
   \>B2:    \>B,      \>AT=58.255,ANGLE=B1[ANGLE] \\
   \>SD1:   \>SD,     \>AT=76.74 \\
   \>QD1:   \>QD,     \>AT=78.20 \\
   \>ENDM:  \>MARKER, \>AT=79.0 \\
ENDSEQUENCE
USE,CELL \\
SELECT,OPTICS,SBEND,QUAD,SEXT \\
OPTICS,FILENAME="cell.optics.f",EXIT,COLUMN=NAME,S,BETX,BETY
}
The resulting table file is listed in Table~\ref{T-OPTICS}.

\begin{table}[hb]
\caption{An OPTICS Output Table Example}
\vspace{1ex}
\label{T-OPTICS}
\begin{verbatim}
@ GAMTR            %f    64.3336
@ ALFA             %f   0.241615E-03
@ XIY              %f   -.455678
@ XIX              %f    2.05279
@ QY               %f   0.250049
@ QX               %f   0.249961
@ CIRCUM           %f    79.0000
@ DELTA            %f   0.000000E+00
@ COMMENT          %20s "DATA FOR TEST CELL"
@ ORIGIN           %24s "MAD 8.01    IBM - VM/CMS"
@ DATE             %08s "19/06/89"
@ TIME             %08s "09.47.40"
* NAME             S              BETX           BETY
\) %16s             %f             %f             %f
  B1                  36.6600        24.8427        126.380
  SF1                 37.6200        23.8830        130.925
  QF1                 39.5000        23.6209        132.268
  B2                  75.8000        124.709        25.2153
  SD1                 77.1200        130.933        23.8718
  QD1                 79.0000        132.277        23.6098
\end{verbatim}
\end{table}
\clearpage
 
\section{Examples for SPLIT and OPTICS Commands}
\index{examples!SPLIT@{\tt SPLIT}}
\index{examples!OPTICS@{\tt OPTICS}}
\index{examples!PRINT@{\tt PRINT}}
\index{examples!EALIGN@{\tt EALIGN}}
\index{range examples}\index{examples!range}
\index{place examples}\index{examples!place}
The following is an excerpt of the LEP description:
\myxmp{
xxxxxxx\=xxxxxxxxxxx\=xxxxxxxxxxx\=\kill
! Bending magnet pairs: \\
! The definitions take into account the different magnetic length \\
! for the inner and outer pairs of a group of six. \\
B2:    \>RBEND,     \>L=11.55,ANGLE=KMB2,K1=KQB,K2=KSB, \& \\
       \>           \>E1=-.25*B2[ANGLE],E2=-.25*B2[ANGLE] \\
B2OUT: \>B2,        \>ANGLE=1.00055745184472*KMB2, \& \\
       \>           \>E1=-.25*B2OUT[ANGLE],E2=-.25*B2OUT[ANGLE] \\
B2MID: \>B2,        \>ANGLE=1.00111490368947*KMB2, \& \\
       \>           \>E1=-.25*B2MID[ANGLE],E2=-.25*B2MID[ANGLE] \\
 
! Quadrupoles: \\
MQ:    \>QUADRUPOLE,\>L=1.6       \>! standard quadrupoles = \\
QD:    \>MQ,        \>K1=KQD      \>! cell quadrupoles, defocussing \\
QF:    \>MQ,        \>K1=KQF      \>!cell quadrupoles, focussing \\
 
! Sextupoles: \\
MSF:   \>SEXTUPOLE, \>L=0.40      \>! F sextupoles \\
MSD:   \>SEXTUPOLE, \>L=0.76      \>! D sextupoles \\
SF1.2: \>MSF,       \>K2=KSF1.2   \>! F family 1, circuit 2 \\
SF2.2: \>MSF,       \>K2=KSF2.2   \>! F family 2, circuit 2 \\
SF3.2: \>MSF,       \>K2=KSF3.2   \>! F family 3, circuit 2 \\
SD1.2: \>MSD,       \>K2=KSD1.2   \>! D family 1, circuit 2 \\
SD2.2: \>MSD,       \>K2=KSD2.2   \>! D family 2, circuit 2 \\
SD3.2: \>MSD,       \>K2=KSD3.2   \>! D family 3, circuit 2 \\
 
! Orbit correctors and monitors: \\
CH:    \>HKICK,     \>L=0.4       \>! Horizontal orbit correctors \\
CV:    \>VKICK,     \>L=0.4       \>! Vertical orbit correctors \\
MHV:   \>MONITOR,   \>L=0         \>! Orbit position monitors \\
 
xxxxxxxxxxxxxxxx\=xxxxxxx\=\kill \\
LEP:SEQUENCE \\
   \ldots \\
QF23.R1:        \>QF,    \>AT=639.180037 \\
   SF2.QF23.R1: \>SF2.2, \>AT=640.460037 \\
   B2L.QF23.R1: \>B2OUT, \>AT=647.257037 \\
   B2M.QD24.R1: \>B2MID, \>AT=659.147037 \\
   B2R.QD24.R1: \>B2OUT, \>AT=671.037037 \\
   CV.QD24.R1:  \>CV,    \>AT=677.392037, KICK=KCV24.R1 \\
   PU.QD24.R1:  \>MHV,   \>AT=677.712037 \\
QD24.R1:        \>QD,    \>AT=678.680037 \\
   SD2.QD24.R1: \>SD2.2, \>AT=680.140037 \\
   B2L.QD24.R1: \>B2OUT, \>AT=686.757037 \\
   B2M.QF25.R1: \>B2MID, \>AT=698.647037 \\
   B2R.QF25.R1: \>B2OUT, \>AT=710.537037 \\
   CH.QF25.R1:  \>CH,    \>AT=716.942037, KICK=KCH25.R1 \\
QF25.R1:        \>QF,    \>AT=718.180037 \\
   SF1.QF25.R1: \>SF1.2, \>AT=719.460037 \\
   B2L.QF25.R1: \>B2OUT, \>AT=726.257037 \\
   B2M.QD26.R1: \>B2MID, \>AT=738.147037 \\
   B2R.QD26.R1: \>B2OUT, \>AT=750.037037 \\
   CV.QD26.R1:  \>CV,    \>AT=756.392037, KICK=KCV26.R1 \\
   PU.QD26.R1:  \>MHV,   \>AT=756.712037 \\
QD26.R1:        \>QD,    \>AT=757.680037 \\
   SD1.QD26.R1: \>SD1.2, \>AT=759.140037 \\
   B2L.QD26.R1: \>B2OUT, \>AT=765.757037 \\
   B2M.QF27.R1: \>B2MID, \>AT=777.647037 \\
   B2R.QF27.R1: \>B2OUT, \>AT=789.537037 \\
QF27.R1:        \>QF,    \>AT=797.180037 \\
   SF3.QF27.R1: \>SF3.2, \>AT=798.460037 \\
   B2L.QF27.R1: \>B2OUT, \>AT=805.257037 \\
   B2M.QD28.R1: \>B2MID, \>AT=817.147037 \\
   B2R.QD28.R1: \>B2OUT, \>AT=829.037037 \\
   PU.QD28.R1:  \>MHV,   \>AT=835.712037 \\
QD28.R1:        \>QD,    \>AT=836.680037 \\
   SD3.QD28.R1: \>SD3.2, \>AT=838.140037 \\
   B2L.QD28.R1: \>B2OUT, \>AT=844.757037 \\
   B2M.QF29.R1: \>B2MID, \>AT=856.647037 \\
   B2R.QF29.R1: \>B2OUT, \>AT=868.537037 \\
   CH.QF29.R1:  \>CH,    \>AT=874.942037, KICK=KCH29.R1 \\
   \ldots \\
ENDSEQUENCE
}
In the above structure it is easy to select many sets of observation
points:
\begin{itemize}
\item Print at all F sextupoles:
\myxmp{
PRINT,MSF
}
\item Split all quadrupoles at \(1/3\) of their length for
{\tt OPTICS} command:
\myxmp{
SPLIT,QUADRUPOLE,FACTOR=1/3
}
\item Misalign two quadrupole {\tt QF25.R1} and {\tt QD26.R1}:
\myxmp{
EALIGN,QF25.R1,QD26.R1,DX=0.001*GAUSS(),DY=0.0005*GAUSS()
}
\item Print first-order matrices for elements {\tt B2L.QD24.R1}
through {\tt CV.QD26.R1}:
\myxmp{
SELECT,FIRST,B2L.QD24.R1[1]/CV.QD24.R1[1]
}
Print lattice functions at all F-sextupoles of the first family,
if connected to the second circuit:
\myxmp{
PRINT,SF1.2
}
\end{itemize}

\chapter{Calculation of Beam Parameters}
\label{S-BEAMPARAM}
\index{beam parameters|(}\index{electron beam parameters|(}
\index{BMPM@{\tt BMPM}|(}
\begin{table}[ht]
\caption{Electron Beam Parameter Command}
\vspace{1ex}
\label{T-BMPMCOM}
\centering
\begin{tabular}{|l|l|l|}
\hline
Name &Function &Section \\
\hline
\ttindex{BMPM}&Calculate and print beam parameters&\ref{S-BMPM} \\
\hline
\end{tabular}
\end{table}
 
\section{Introduction}
The program \ttindex{BEAMPARAM} (see~\cite{B-GK})
which served as the basis for this module,
is valid for \epem\ storage rings only. Inside MAD, however, this
new module will use the correct mass and radius of the proton if this
is chosen by the user. This means that all formulae will remain
correct even in the case of the proton. However, it has not been
checked whether some of them have been derived under the assumption
that the particles are electrons. In the text, only \epem\ storage
rings are mentioned.
 
The utility of the {\tt BMPM} command is achieved at the price of a few
simplifying assumptions and approximations concerning the accelerator
lattice. Although these conditions are quite well satisfied by many
\epem\ storage rings, the user does need to be aware of them.
Accordingly we list the most important among them below:
 
\begin{description}
 
\item[Flat machine] If any element has a non-zero {\tt TILT} parameter
\index{flat machine}
then the command will abort. This
excludes, in particular, skew quadrupoles and vertical bends,
hence the vertical dispersion will be zero everywhere.
Combined-function dipoles are not accepted.
 
\item[Emittance coupling] Although the previous assumption implies
\index{emittance!coupling}\index{coupling}
that the betatron coupling vanishes so that the vertical emittance
will be determined by the small transverse components of the photon
recoil, the vertical emittance is in fact calculated using a non-zero
betatron coupling parameter \(\kappa\) (\ttindex{KAPPA}).
This factor is defined by the equation \(E_y/E_x = \kappa^2 (J_x/J_y)\).
 
\item[Damping partition number variation] In reality, this is achieved
\index{damping}
by small shifts of the equilibrium momentum which place the beam on a
slightly different orbit. {\tt BMPM} computes the values of the damping
partition numbers on the central orbit with \eqref{E-JX}--\eqref{E-JE}.
These are later used in the calculations of damping times, emittances
and energy spread. In reality, their values on other (off-momentum)
orbits involve further synchrotron radiation integrals. However,
changes in the damping partition number are effected by
artificially changing the value of the synchrotron integral \(I_4\). For
convenience, it is the value of the ratio \(I_4/I_2\) that is specified
in the input rather than \(I_4\) itself. It is the user's responsibility
to ensure that the damping partition number variation brought about in
this way does not violate the aperture constraints of his machine.
\[
\eqarray{
    \hbox{Horizontal } J_x  & = & 1- I_4/I_2   \label{E-JX} \\
    \hbox{Vertical }  J_y  & = & 1 \\
    \hbox{Energy }  J_e  & = & 2 + I_4/I_2   \label{E-JE}
}
\]
 
\item[Luminosity] The luminosity value computed assumes that the beam
\index{luminosity}
sizes have their natural values,
and does not allow for any beam-beam blow-up.
 
\end{description}
 
\section{BMPM Command}
\label{S-BMPM}
The {\tt BMPM} command performs the calculation of beam parameters
for an \epem\ storage ring according to the parameters and options
specified below.
It requires the prior execution of the commands BEAM and TWISS
with SAVE option.
This command requires the beam energy to be set,
and the machine to contain an RF cavity
whose shunt impedance and filling time are specified.
It does not handle combined function dipoles.
\mybox{
BMPM, \=NINT=integer,DELQ=real,TAUQ=real,BUCKET=real,\& \\
      \>KAPPA=real,I4I2=real,EXDATA=real,FX=real,FY=real,\& \\
      \>KHM=real,SYNRAD=logical,CLORB=logical,\& \\
      \>TOUSCH=logical,SINGLE=logical,EVARY=logical,\& \\
      \>MIDARC=logical,INTERACT=range,RANGE=range
}
\begin{mylist}
\ttitem{NINT}
Number of crossing points (default:~twice the number of bunches).
\ttitem{DELQ}
Maximum permissible beam-beam tune shift per crossing (default:~0.06).
\ttitem{TAUQ}
Quantum lifetime (default:~600~Minutes).
\ttitem{BUCKET}
\(\sigma_b/\sigma_e\), where \(\sigma_b\) is the bucket half-height,
and \(\sigma _e\) the relative r.m.s.\ energy spread (default:~0).
\ttitem{KAPPA}
Coupling factor between horizontal and vertical betatron
oscillations (default:~0).
\ttitem{I4I2}
Ratio of the two synchrotron integrals \(I_4\) and \(I_2\) (default:~0).
\ttitem{EXDATA}
Horizontal emittance \(E_x\) (default:~0) in \(\mu\)m.
\ttitem{FX}
Horizontal multiplication factor (default:~10).
\ttitem{FY}
Vertical multiplication factor (default:~10).
\ttitem{KHM}
Higher mode loss factor (default:~0) in V/pC.
\ttitem{SYNRAD}
Performs synchrotron radiation calculations (default:~F).
\ttitem{CLORB}
Performs closed orbit calculations (default:~F).
\ttitem{TOUSCH}
Performs Touschek lifetime calculations (default:~F).
\ttitem{SINGLE}
True for one beam, false for two beams (default:~F).
\ttitem{EVARY}
Vary energy between current nominal energy and twice its value
until the calculated power is equal to the power given (default:~F).
\ttitem{MIDARC}
If {\tt MIDARC=T}, then the e\({}^-\) and e\({}^+\) are
assumed to be interleaved with equal spacing; otherwise they are
assumed to be coincident. The small difference in arrival times which
occurs in practice is neglected. For single beams, the {\tt
MIDARC} parameter is irrelevant (default:~F).
\ttitem{INTERACT}
Position (name, number, etc.) of the interaction element. If a range
is entered, the first element of the range is used
(default:~\#E, see Section~\ref{S-OBSAT}).
\ttitem{RANGE}
Element range for which a detailed table is printed
(default:~no range, see Section~\ref{S-OBSAT}).
\end{mylist}
Example:
\myxmp{
XXXXXXX\=\kill
xxx:   \>RFCAVITY,SHUNT=...,TFILL=... \\
BEAM,  \>ENERGY=100.0 \\
TWISS, \>SAVE
BMPM
}
\section{Output of overall machine parameters}
\subsection{Synchrotron integrals}
\index{synchrotron!integrals}
This table gives the synchrotron integrals \(I_1\) to \(I_5\) as defined
in reference~\cite{B-HLMS}. Further integrals, \(I_{6x}\) and \(I_{6y}\),
defined by M.Bassetti,
as well as \(I_8\) are also calculated and printed.
 
\subsection{Machine parameters}
\index{machine parameters}
This table contains the tunes \(Q_x\), \(Q_y\), the betatron functions
\(\beta_x^*\) and \(\beta_y^*\) and the dispersions \(D_x^*\) and
\(D_y^*\) at the crossing point. Since the program does not handle
vertical dispersion, \(D_y^*\) is set to zero. Also printed are the
momentum compaction \(\alpha\), the machine circumference \(C\), the
revolution time \(t_0\), the damping partition numbers \(J_x\), \(J_y\),
\(J_e\), the derivative of \(J_x\) with respect to \(dE/E\), then \(dE/E\)
itself, and finally the frequency change \(df_{RF}\).
 
\subsection{Beam parameters and luminosities}\label{bbsec}
\index{luminosity}
This table gives the beam energy, the coupling, the
larger of the horizontal and vertical tune shifts \(\Delta Q\), the
energy loss per turn due to synchrotron radiation \(U_0\), the r.m.s.\
energy spread \(\sigma_e\), the magnetic rigidity \(B\rho\), and the
damping times for the horizontal and vertical betatron oscillations,
\(\tau_x\) and \(\tau_y\), and for the synchrotron oscillations \(\tau_e\).
 
\(\sigma_x^*\) and \(\sigma_y^*\) are
the r.m.s.\ beam radii at the crossing
point, defined by \ttindex{INTERACT}:
 
\begin{itemize}
 
\item \rm \(\sigma_{x0}^*\) and \(\sigma_{y0}^*\)  are the uncoupled
betatron values.
 
\item \rm\(\sigma_{xc}^*\) and \(\sigma_{yc}^*\)   are the values due to
coupled betatron oscillations.
 
\item \rm \(\sigma_{xT}^*\) and \(\sigma_{yT}^*\)  are the total values,
including coupled betatron oscillations and energy oscillations.
 
\end{itemize}
 
\begin{description}
 
\item[Convention for the coupling \(\kappa\)]:
\index{coupling}
If a {\sl non-zero value\/} of the coupling parameter \(\kappa\) is read
in, it is used for calculating \(\sigma_{xc}^*\), \(\sigma_{yc}^*\),
\(\sigma_{xT}^*\), \(\sigma_{yT}^*\), \(E_{xc}\), \(E_{yc}\) etc. and never
gets changed. In the output \(\kappa\)~is labelled {\tt (D)KAPPA}.
 
If the input value \(\kappa = 0\), {\tt BMPM} computes \(\kappa\) so that
the beam-beam tune shifts \(\Delta Q_x\) and \(\Delta Q_y\) vary with the
current as shown schematically in Figure~\ref{F-BMP}. For currents at the
beam-beam limit \(\Delta Q\) and above, the coupling \(\kappa\) is
adjusted so that \(\Delta Q_x=\Delta Q_y\) by having \(\sigma_{xT}^* /
\sigma_{yT}^* = \beta_x^*/ \beta_y^*\). (This is usually called optimum
coupling.) For currents below the beam-beam limit \(\Delta Q\), the
coupling \(\kappa\) is adjusted so that \(\Delta Q_x < \Delta Q\) but
\(\Delta Q_y = \Delta Q\).
\end{description}
\begin{figure}[ht]
\centering
\setlength{\unitlength}{1pt}
\begin{picture}(400,210)(-100,-10)
\thinlines
\put(0,0){\vector(0,1){200}}
\put(190,-5){\makebox(0,0)[t]{I}}
\put(0,0){\vector(1,0){200}}
\put(-5,190){\makebox(0,0)[r]{\(\Delta Q\)}}
\put(50,105){\makebox(0,0)[b]{\(\Delta Q_y\)}}
\put(0,100){\line(1,0){100}}
\multiput(0,0)(20,20){10}{\line(1,1){12}}
\put(50,50){\makebox(0,0)[br]{\(\Delta Q_x\)}}
\put(100,101){\line(1,1){100}}
\end{picture}
\caption[Variation of the beam-beam tune shifts]%
{Variation of the beam-beam tune shifts $\Delta Q_x$ and $\Delta Q_y$
with the current obtained by varying the coupling $\kappa$}
\label{F-BMP}
\end{figure}
 
\paragraph{Beam current}
\index{beam!current}\index{current}
Three different methods are foreseen for calculating the circulating
current \(I\), the stored number of particles \(N\) in each beam, and the
luminosity \(L\).
 
\begin{enumerate}
 
\item If neither a value for the current \(I_{xy}\) nor a value for the
RF generator power \(P_g\) is given in the data, i.e.\ \(I_{xy} =P_g =0\),
then the currents \(I_x\) and \(I_y\) are calculated so that
\(\Delta Q_x=\Delta Q\) and \(\Delta Q_y=\Delta Q\) respectively.
These currents are then used to calculate the stored number of particles
\(N_x\) and \(N_y\) and the luminosities \(L_x\) and \(L_y\).  For optimum
coupling, \(I_x = I_y\), \(N_x = N_y \) and \(L_x=L_y\).  This option is
typical for storage rings in the energy range where the luminosity is
limited by the beam-beam tune shift.
 
\item If a value for the current \(I_{xy}\) {\sl is\/} specified in the
data, i.e.\ \(I_{xy} \neq 0\), then \(I_x=I_y=I_{xy}\) and these values
are used to calculate the beam-beam tune shifts \(\Delta Q_x\) and
\(\Delta Q_y, N_x\) and \(N_y (=N_x)\) and the luminosities \(L_x\)
and \(L_y (=L_x)\).
This option typically applies to storage rings in the energy
range where the current is limited, and to synchrotrons. The \(\Delta
Q\) printed is the larger of \(\Delta Q_x\) and \(\Delta Q_y\), it may well
exceed the beam-beam limit.
 
\item If an RF generator power \(P_g \neq 0\) is specified in the data,
the current \(I_x\) is adjusted so that the RF generator power necessary
to sustain it is equal to the input value \(P_g\). The current \(I_x=I_y\)
is then used to calculate \(\Delta Q_x\) and \(\Delta Q_y, N_x\) and \(N_y
(=N_x), L_x\) and \(L_y(=L_x)\).
 
\end{enumerate}
 
In the output, the input parameters, \(\Delta Q, I_x, P_g\),
are labelled~{\tt (D)} and the calculated parameters~{\tt (C)}.
 
The beam-beam bremsstrahlung lifetime \(\tau_{\rm bb}\) is calculated
according to~\cite{B-WM}, using the bucket height obtained in the
calculation of the RF parameters, and assuming that there are \(N_{\rm
int}\) identical crossings. For machines with different crossings the
calculated \(\tau_{\rm bb}\) must be scaled appropriately.
 
\subsection{RF related parameters}
\index{RF parameters}\index{parameter!RF}
The RF parameter calculation is done in two different ways, depending
on the value of the RF generator power in the data. If \(P_g=0\) there
are two possible values for the current \(I_x\): {\sl either\/} \(P_g =
I_{xy} = 0\), in which case a value for \(I_x\) is calculated, {\sl or\/}
\(P_g = 0\) and \(I_{xy} \neq 0\) in which case \(I_x=I_{xy}\). Once one has
a value for \(I_x\) then a single-pass calculation is sufficient to
obtain all the output parameters described below. If \(P_g \neq 0\), the
current \(I_x\) is adjusted so that the input value and the calculated
value of \(P_g\) agree, before the rest of the output parameters is
calculated.
 
The output contains the harmonic number \(f_{\rm RF} / f_0\), the peak RF
voltage \(V_{\rm RF}\), the stable phase angle \(\phi_s\), the synchrotron
tune \(Q_s\) and the synchrotron frequency \(f_s\), the half-height of
the bucket \(\sigma_b\) in units of \(\sigma_e\),
the r.m.s.\ bunch length \(\sigma_s\), the
quantum lifetime \(\tau_q \) and the Touschek lifetime \(\tau_{1/2}\)
calculated according to~\cite{B-UV,B-CL,B-RW}.
 
If \(\gamma \sigma_{x'}/\sigma_b > 100\)
 then we use an approximation which is
good to within 2\%~\cite{B-RW}, otherwise we use the exact formula.
Since the approximation involves an integration along the orbit, it is
somewhat time-consuming. The exact formula will take even longer as a
double integral is involved. So, it is preferable not to ask for the
Touschek lifetime unless it is really needed.
 
\begin{description}
 
\item[Convention for RF frequency \(f_{\rm RF}\):] The harmonic number
\(f_{\rm RF} / f_0\) is computed such that it is the multiple
 of \(k_b\) which
gives an RF frequency \(f_{\rm RF}\) closest to the input value.
 
\item[Convention for RF voltage \(V_{\rm RF}\):] The RF voltage \(V_{\rm
RF}\) may be obtained in one of three ways. Firstly it may be read
in as data as an attribute of the RF cavities. Secondly it may
be computed such  that an energy loss
 \((U_0 +  k_{\rm hm} Q_b)\) is
compensated, at the specified quantum lifetime \(\tau_q\), where
 \(Q_b\)  is the bunch charge and \(k_{\rm hm}\) is the higher mode loss
factor. Finally it can be calculated from the condition that the bucket
  half height \(\sigma_b\) is
\ttindex{BUCKET} times the r.m.s.\ energy spread \(\sigma _e\)
\end{description}
 
The computation of the RF generator power \(P_g\) involves a
complete beam-loading calculation according to~\cite{B-PW}. The input
data required are: the length of the active RF system \(L_c\) (sum of
physical lengths of all cavities),
its shunt impedance per unit length \(Z\), and its unloaded filling time
\(T_{\rm fill}\) (all attributes of the RF cavities).
The beam is described by parameters which are already
known---such as the current \(I_x\) and the number of bunches
\(k_b\)---and by the logical variables \ttindex{SINGLE} and
\ttindex{MIDARC}.
If {\tt SINGLE=T}, then the calculation is done for a single beam
circulating in the machine; otherwise the calculation is done for two
counter-rotating beams of e\({}^-\) and e\({}^+\). The arrival times of the
e\({}^-\) and e\({}^+\) bunches at the RF cavities are controlled by the
variable {\tt MIDARC}. If {\tt MIDARC=T}, then the e\({}^-\) and e\({}^+\) are
assumed to be interleaved with equal spacing; otherwise they are
assumed to be coincident. The small differences in arrival times which
occurs in practice~\cite{B-BH} is neglected. For single beams, the {\tt
MIDARC} parameter is irrelevant.
 
The two  cavity attributes \ttindex{BETRF} and \ttindex{LAG},
represent the RF coupling parameter \(\beta_{RF}\) and
the cavity tuning angle \(\psi_{RF}\)
 respectively.
If both  \(\beta_{RF}\) and
\(\psi_{RF}\) are zero, {\tt BMPM} will adjust them such that
the RF generator power, \(P_g\), is minimised. If \(\beta_{RF}\) is
different from zero, \(\psi_{RF}\) will be varied such that the RF generator
power \(P_g\)  is minimised. If both are different from zero, no such
optimisation takes place.
 
The output of the beam loading calculation consists of
 \(\beta_{\rm RF}\),   \(\psi_{\rm RF}\), \(\tau_q\) and \ttindex{BUCKET},
all preceded by {\tt (C)} or {\tt (D)} accordingly.
 
\section{Evaluation of the beam size}
\index{beam size}
If a valid element range is given (\ttindex{RANGE}),
a table of beam sizes is printed for the elements in the range.
For each element, it gives the number,
name, \(\eta\), \(\eta'\), \(\beta_x\),
\(\beta_y\), \(\alpha_x\), \(\alpha_y\), \(F_x\sigma_x\), and \(F_y \sigma_y\),
all taken at the entrance of the element. The values of \(F_x\) and
\(F_y\) are input data (\ttindex{FX} and \ttindex{FY}).
 
\begin{description}
 
\item[Convention for beam radii:] The horizontal beam radius
\(\sigma_x\) is calculated for an
uncoupled beam using the emittance \(E_{x0}\),
the vertical one for a fully coupled beam
using the emittance \(E_{x0}/2\).
 
\end{description}
 
Because of this convention, the actual beam sizes for all amounts of
coupling should be smaller than or equal to the figures printed. It
may be found useful to split elements with strong variations of
\(\beta_x\) and \(\beta_y\), e.g.\ low-\(\beta\) quadrupoles, into several
pieces in the data in order to get better results for the beam size
variation in these elements.
 
 
\section{Synchrotron radiation, Calculated for One Beam}
\index{synchrotron!radiation}
\index{radiation}
If in addition to a valid range, {\tt SYNRAD=T},
the table of beam sizes will contain in addition for each
element the horizontal beam
divergence \(\sigma_{x'}\), the vertical beam divergence \(\sigma_{y'}\),
the bending radius \(\rho\), the critical photon energy \(E_c\),
the synchrotron
radiation power/metre, and the number of photons/m/s/keV at the
critical energy.
 
\begin{description}
 
\item[Convention for divergences:] \(\sigma_{x'}\) is calculated for an
uncoupled beam; \(\sigma_{y'}\) is calculated for a fully coupled beam;
\(\rho\), \(E_c\) etc.\ are only calculated for bending magnets
(\(\phi\neq0\)) or quadrupoles (\(K_1 \neq 0 \)). For all these calculations,
 combined function magnets
are not permitted (\(K_1\phi \neq 0\)). In quadrupoles, \(\rho\), \(E_c\)
etc., as defined in~\cite{B-KEIL}, are calculated for a fully coupled beam
at a distance \(\sigma\) from the quadrupole axis.
 
\end{description}
 
Again, it may be found useful to split elements with strong
variations of \(\beta_x\) and \(\beta_y\), e.g. low-\(\beta\) quadrupoles,
into several pieces in the data in order to obtain better results
for the variation of the synchrotron radiation parameters in
those elements.
 
\section{Closed orbit calculations}
\index{closed orbit}\index{orbit!closed}
The sensitivity of the closed orbit to alignment errors of the
quadrupoles, and excitation errors and tilts of the bending magnets
can be expressed by integrals around the machine circumference of
appropriate powers of the amplitude functions \(\beta_x\) and \(\beta_y\),
and the focussing parameter \(K_1\).
 
If in addition to a valid range, {\tt CLORB=T},
the table of beam sizes will contain in addition for each element the
amplification factors \(P_x\) and \(P_y\) for all quadrupoles and
bending magnets. For quadrupoles, \(P_x\) and \(P_y\) are amplification
factors for horizontal and vertical misalignments. They are the ratio
between the closed orbit distortion in that element, which will not be
exceeded in 98\% of all machines, to the r.m.s.\ displacement of the
quadrupoles, assumed to be the same for all quadrupoles. For bending
magnets, \(P_x\) is the 98\% ratio between closed orbit distortions in
metres and relative r.m.s.\ magnet errors \(\Delta B/B\), and similarly
\(P_y\) is the ratio between closed orbit distortions in metres and the
r.m.s.\ tilt of the magnets.
\index{BMPM@{\tt BMPM}|)}
\index{beam parameters|)}\index{electron beam parameters|)}

\chapter{File and Pool Handling}
\label{S-FILES}
\index{data!streams|(}\index{streams|(}\index{files|(}
 
\begin{table}[ht]
\label{T-FILES}
\caption{File Handling Commands}
\vspace{1ex}
\centering
\begin{tabular}{|l|p{0.6\textwidth}|l|}
\hline
Name &Meaning &Section \\
\hline
\ttindex{ASSIGN}&Assign standard streams to files &\ref{S-ASSIGN} \\
\ttindex{SAVE}&Save machine structure in MAD format &\ref{S-IMBED} \\
\ttindex{CALL}&Read alternate input file &\ref{S-IMBED} \\
\ttindex{RETURN}&Return to calling file &\ref{S-IMBED} \\
\ttindex{EXCITE}&Read element excitations &\ref{S-EXCITE} \\
\ttindex{INCREMENT}&Increment element excitations &\ref{S-EXCITE} \\
\ttindex{ARCHIVE}&Archive an internal table &\ref{S-ARCHRET} \\
\ttindex{RETRIEVE}&Retrieve an internal table &\ref{S-ARCHRET} \\
\ttindex{PACKMEMORY}&Garbage removal &\ref{S-PACKMEM} \\
\ttindex{STATUS}&Show status of files &\ref{S-STATUS} \\
\ttindex{POOLDUMP}&Dump data pool to disk &\ref{S-POOL} \\
\ttindex{POOLLOAD}&Reload data pool from disk &\ref{S-POOL} \\
\hline
\end{tabular}
\end{table}
 
\section{I/O Data Sets (Files)}
In MAD most data streams are referred to by their name
and opened by FORTRAN OPEN statements with a file name.
Only the standard input and output are referred to by number
and must be assigned to devices by external commands
(job control language, execute files or the like).
The standard data streams used in MAD are listed in
Table~\ref{T-FILE}.
The format of file names is described in Appendix~\ref{A-FILES}
for various computer operating systems.
\begin{table}[ht]
\caption{Standard Files Used by MAD}
\index{input}\ttnindex{INPUT}
\index{output}\ttnindex{PRINT}
\index{dictionary}\ttnindex{DICT}\index{command!dictionary}
\ttnindex{ECHO}\index{error messages}
\index{plot}
\index{dynamic tables}\index{tables}
\vspace{1ex}
\label{T-FILE}
\centering
\begin{tabular}{|l|l|l|}
   \hline
   Purpose                         &unit  &file name \\
   \hline
   Command dictionary input        & 4    &dict \\
   Normal input                    & 5    & \\
   Input lines and error messages  & 6    & \\
   Plot output (GKS metafile)      & 8    &mad.metafile \\
   Plot output (HIGZ metafile)     & 8    &mad.ps \\
   Normal output                   &14    &print \\
   Dynamic tables                  &15    &table \\
   \hline
\end{tabular}
\end{table}
 
\section{ASSIGN Statement}
\label{S-ASSIGN}
The \ttindex{ASSIGN} statement is able to reroute three of the standard
I/O streams to different files:
\mybox{
ASSIGN,DATA=file-name,ECHO=file-name,PRINT=file-name
}
For each {\tt stream=file-name} clause,
it connects the stream {\tt stream} to the file {\tt file-name}.
The special file name \ttindex{TERMINAL} denotes the interactive terminal.
Example:
\myxmp{
ASSIGN,ECHO=PRINT,PRINT=TERMINAL
}
This example sends the \ttindex{ECHO} stream
to the standard \ttindex{PRINT} file,
and the \ttindex{PRINT} stream to the terminal.
Standard connections can be reestablished by
\myxmp{
ASSIGN,ECHO=ECHO,DATA=DATA,PRINT=PRINT
}
 
\section{SAVE, CALL, and RETURN Statements}
\label{S-IMBED}
The \ttindex{SAVE} command
\mybox{
SAVE, FILENAME=string, PATTERN=string
}
causes all beam element, beam line, and parameter definitions
to be written on the named file.
The output format is similar to the normal input format.
The file may be read again in the same run.
The attributes are
\begin{mylist}
\ttitem{FILENAME}
The name of the file to be written (default:~{\tt save}).
\ttitem{PATTERN}
A ``wild-card'' pattern permitting selective saving
\index{wild-card}
\end{mylist}
The default is to save all definitions;
this is equivalent to the {\tt PATTERN=".*"}.
For definition of the pattern refer to Section~\ref{S-WILD}.
Examples:
\myxmp{
SAVE,FILENAME="structure" \\
SAVE,FILENAME="quadrupoles",PATTERN="Q.." \\
SAVE,FILENAME="strengths",PATTERN="K.*QD.*\.R1"
}
The first command saves all definitions onto file {\tt structure}.
The second command saves all definitions
whose names begin with the letter {\tt Q} and have exactly three
characters onto file {\tt quadrupoles}.
The third command saves all definitions whose names begin with {\tt K},
contain {\tt QD}, and end with {\tt .R1}.
The two occurrences of {\tt .*} replace arbitrary strings of any length
(including zero);
and {\tt \(\backslash\).} replaces a literal period character.
 
The command
The \ttindex{CALL} command
\mybox{
CALL, FILENAME=string
}
serves to read an alternate input file.
Input continues on that file until a \ttindex{RETURN} statement
or end of file is encountered.
The attribute is
\begin{mylist}
\ttitem{FILENAME}
The name of the file to be read (default:~{\tt save}).
\end{mylist}
Example:
\myxmp{
CALL, FILENAME=STRUCT
}
reads back the file written by the above {\tt SAVE} example.

The command \ttindex{RETURN}
\mybox{
RETURN
}
on an alternate input file causes reading to resume with the next line
following the {\tt CALL} statement on the standard input file.
Both {\tt CALL} and {\tt RETURN} statements cannot be followed by any
other information (except comments) on the same line.
The {\tt CALL/RETURN} structure can be nested up to a level 10;
i.e. an alternate input file can issue another {\tt CALL}.
Circular calls are forbidden.
The following example shows a possible sequence of input:
\myxmp{
xxxxxxxxxxxxxxxxxxxxxxxx\=xxxxxxxxxxxxxxxxxxxxxxxx\= \kill
contents of             \>contents of             \>contents of \\
main input file         \>file "STRUCT"           \>file "ACTION" \\
\vspace{2ex}
TITLE, "page header" \\
any commands \\
CALL, FILENAME=STRUCT \\
-------------------------> \\
                        \>beam element definitions \\
                        \>beam line definitions \\
                        \>parameter definitions \\
                        \>RETURN \\
<------------------------- \\
more definitions \\
CALL, FILENAME=ACTION \\
---------------------------------------------------> \\
                           \>                     \>action commands \\
                           \>                     \>RETURN \\
<--------------------------------------------------- \\
more action commands \\
SAVE, FILENAME=STRUCT \\
STOP
}
 
\section{Element Excitations}
\label{S-EXCITE}
Two commands, \ttindex{EXCITE} and \ttindex{INCREMENT}  allow to enter
complete lists of element excitations:
\mybox{
xxxxxxxxxxx\=\kill
EXCITE,    \>FILENAME=file-name,NAME=name,VALUE=name \\
INCREMENT, \>FILENAME=file-name,NAME=name,VALUE=name
}
Both have the same attributes:
\begin{mylist}
\ttitem{FILENAME}
The name of a TFS file (see Appendix~\ref{S-TFS}) to be read.
\ttitem{NAME}
The name of the table column containing the strength names.
\ttitem{VALUE}
The name of the table column containing the strength values
(for {\tt EXCITE}) or increments (for {\tt INCREMENT}).
\end{mylist}
The names in the {\tt NAME} column must refer to global MAD parameters.
If a name is not known, MAD creates a new parameter of this name.
The TFS table may contain other columns, but they will be ignored.
 
\section{Archiving and Retrieving Dynamic Tables}
\label{S-ARCHRET}
\subsection{ARCHIVE Command}
\ttindex{ARCHIVE} writes a dynamic table on a formatted TFS file
(see Appendix~\ref{S-TFS}):
\mybox{
ARCHIVE,TABLE=name,FILENAME=file-name
}
The command has two attributes:
\begin{mylist}
\ttitem{TABLE}
\index{dynamic table}
The name of a previously created table to be written.
It may reside in memory or (partly) in bulk storage.
\ttitem{FILENAME}
The name of the file to be written (default:~{\tt twiss}).
\end{mylist}
Example:
\myxmp{
ARCHIVE,TABLE=TWISS,FILENAME=LEP.TWISS.F
}
\subsection{RETRIEVE Command}
\ttindex{RETRIEVE} reads a formatted TFS file
(see Appendix~\ref{S-TFS}) into memory
and creates a dynamic table.
\mybox{
RETRIEVE,TABLE=name,FILENAME=file-name
}
The command has two attributes:
\begin{mylist}
\ttitem{TABLE}\index{dynamic table}
The name of a dynamic table to be created.
It need not be the same as when the table was written.
If there is already a table with this name,
it is deleted before reading a new copy.
\ttitem{FILENAME}
The name of the file to be read (default:~{\tt twiss}).
\end{mylist}
Example:
\myxmp{
RETRIEVE,TABLE=TWISS1,FILENAME=LEP.TWISS.F
}
reads file written by the {\tt ARCHIVE} example and creates a new
table {\tt TWISS1}.
 
\section{PACKMEMORY Command, Garbage Removal}
\label{S-PACKMEM}
\index{pool}
\index{data!pool}
\index{garbage removal}
The MAD memory management routines, based on the ZEBRA system,
reclaim unused memory space automatically.
One may however force garbage collection at any time by the
\ttindex{PACKMEMORY} command
\mybox{
PACKMEMORY
}
 
\section{STATUS}
\label{S-STATUS}
The \ttindex{STATUS} command
\mybox{
STATUS
}
causes MAD to list all known files on the message stream.
These include all files which were referred to in the current
program run.
 
\section{Dumping and Reloading the Memory Pool}
\label{S-POOL}
\index{data!pool}
\index{dump pool}
\index{pool!dump}
The \ttindex{POOLDUMP} command
\mybox{
POOLDUMP,FILENAME=file-name
}
dumps the complete memory pool on the named disk file.
Its attribute is:
\begin{mylist}
\ttitem{FILENAME}
The name of the file to receive the dump (default:~{\tt pooldump}).
\end{mylist}
Memory dumped by {\tt POOLDUMP} can be reloaded in a subsequent
MAD run by a {\tt POOLLOAD} command:
\index{reload pool}
\index{pool!reload}
\mybox{
POOLLOAD,FILENAME=file-name
}
This may save considerable time compared to regenerating a complex
data structure from input.
Its attribute is:
\begin{mylist}
\ttitem{FILENAME}
The name of the file to reload (default:~{\tt pooldump}).
\end{mylist}
\index{data!streams|)}\index{streams|)}\index{files|)}

\chapter{Error Definitions}
\label{S-ERROR}
\index{error!definitions|(}
 
\begin{table}[ht]
\caption{Error Definition Commands}
\vspace{1ex}
\label{T-ERROR}
\centering
\begin{tabular}{|l|p{0.6\textwidth}|l|}
\hline
Name         &Function                         &Section \\
\hline
\ttindex{EALIGN}&Specify misalignment(s)          &\ref{S-MISDEF} \\
\ttindex{EFCOMP}&Specify field error(s)           &\ref{S-FIELDEF} \\
\ttindex{EFIELD}&Specify field error(s)           &\ref{S-FIELDEF} \\
\ttindex{EOPT}  &Specify error options            &\ref{S-EOPTCOM} \\
\ttindex{EPRINT}&List errors assigned to elements &\ref{S-EPRICOM} \\
\hline
\end{tabular}
\end{table}
 
This chapter describes the commands which provide error assignment
and output of errors assigned to elements.%
\footnote{contributed by E. Nordmark (1985).}
It is possible to assign alignment errors and field errors
to single beam elements or to ranges of beam elements.
Errors can be specified both with a constant or random values.
\index{error!random}\index{random error}
Error definitions consist of four types of statements listed in
Table~\ref{T-ERROR}.
They may be entered after having selected a beam line
by means of a {\tt USE} command (see Section~\ref{S-USE}).
 
\section{Misalignment Definitions}
\label{S-MISDEF}
\index{alignment errors}\index{error!alignment}
\index{misalignment errors}
Alignment errors are defined by the \ttindex{EALIGN} command.
The misalignments refer to the local MAD reference system
for a perfectly aligned machine (see Section~\ref{S-REFER}).
Possible misalignments are displacements
along the three coordinate axes,
and rotation about the coordinate axes.
Alignment errors can be assigned to all beam elements except drift
spaces.
The effect of misalignments is treated in a linear approximation.
Position monitors
\index{beam monitor errors}
\index{monitor errors}\index{error!monitor}
can be given read errors in both horizontal and vertical planes.
Monitor read errors (\ttindex{MREX} and \ttindex{MREY})
are ignored for all other elements.
Each new \ttindex{EALIGN} statement replaces the misalignment errors
for all elements in its range.
\par Alignment error values are defined by the statement
\mybox{
xxxxxxxx\=\kill
EALIGN, \>RANGE=range,TYPE=name,DX=real,DY=real,DS=real,\& \\
        \>DPHI=real,DTHETA=real,DPSI=real,\& \\
        \>MREX=real,MREY=real
}
with the attributes
\begin{mylist}
\ttitem{RANGE}
A description of the beam elements in which the error occurs.
It may contain up to five different ranges
(see Section~\ref{S-OBSAT}):
\myxmp{
EALIGN,RANGE=QF,QD[2]/QD[4],\#3/\#5,\#13/\#15
}
\ttitem{TYPE}
Name(s) attached to physical element(s) by a {\tt TYPE=name} clause
(see Section~\ref{S-ELMDEF}).
This may contain up to five different types
\myxmp{
EALIGN,TYPE=MQ,MA,MB,MC,MD
}
\ttindex{RANGE} and \ttindex{TYPE} can be given in the same command.
All elements will be affected which have {\tt TYPE=name}
or which are members of {\tt range}\index{range}.
\ttitem{DX}
The misalignment in the \(x\)-direction for the entry of the
beam element (default:~0~m).
\(\hbox{\tt DX} > 0\) displaces the element in the positive \(x\)-direction
(see Figure~\ref{F-XSDISP}).
\ttitem{DY}
The misalignment in the \(y\)-direction for the entry of the
beam element (default:~0~m).
\(\hbox{\tt DY} > 0\) displaces the element in the positive \(y\)-direction
(see Figure~\ref{F-YSDISP}).
\ttitem{DS}
The misalignment in the \(s\)-direction for the entry of the
beam element (default:~0~m).
\(\hbox{\tt DS} > 0\) displaces the element in the positive \(s\)-direction
(see Figure~\ref{F-XSDISP}).
\ttitem{DPHI}
The rotation around the \(x\)-axis.
A positive angle gives a greater \(y\)-coordinate for the exit
than for the entry (default:~0~rad,
see Figure~\ref{F-YSDISP}).
\ttitem{DTHETA}
The rotation around the \(y\)-axis according to the right hand
rule (default:~0~rad, see Figure~\ref{F-XSDISP}).
\ttitem{DPSI}
The rotation around the \(s\)-axis according to the right hand
rule (default:~0~rad, see Figure~\ref{F-XSDISP}).
\ttitem{MREX}
The horizontal read error for a monitor.
This is ignored if the element is not a monitor
(see Figure~\ref{F-ERMONI}).
If \(\hbox{\tt MREX}>0\),
the reading for \(x\) is too high (default:~0~m).
\ttitem{MREY}
The vertical read error for a monitor.
This is ignored if the element is not a monitor
(see Figure~\ref{F-ERMONI}).
If \(\hbox{\tt MREY}>0\),
the reading for \(y\) is too high (default:~0~m).
\end{mylist}
Example:
\myxmp{
EALIGN,QF[2],DX=0.002,DY=0.0004*RANF(),DPHI=0.0002*GAUSS()
}
Refer to Section~\ref{S-DEFAT} for random value formats.
 
\begin{figure}[ht]
\centering
\setlength{\unitlength}{1pt}
\begin{picture}(400,140)(0,30)
\thinlines
\put(50,75){\line(1,0){96}}
\put(154,75){\vector(1,0){246}}
\put(50,75){\makebox(0,0)[r]{\shortstack{original \\beam line}}}
\put(390,65){\makebox(0,0)[r]{s}}
\put(150,75){\circle{8}}
\put(150,75){\circle*{2}}
\put(140,85){\makebox(0,0){y}}
\put(150,30){\line(0,1){41}}
\put(150,79){\vector(0,1){91}}
\put(140,160){\makebox(0,0){x}}
\put(130,55){\vector(1,1){15}}
\put(130,55){\makebox(0,0)[tr]{\shortstack{original entrance \\
of the magnet}}}
\thicklines
\put(150,55){\vector(1,0){50}}
\put(175,45){\makebox(0,0){DS}}
\put(130,75){\vector(0,1){50}}
\put(125,100){\makebox(0,0)[r]{DX}}
\thinlines
\put(125,125){\line(1,0){185}}
\put(200,50){\line(0,1){120}}
\put(200,125){\circle*{4}}
\thicklines
\put(168,117){\vector(4,1){144}}
\put(210,85){\vector(-1,4){20}}
\bezier{30}(300,125)(300,137)(297,149)
\put(299,140){\vector(-1,4){2}}
\put(305,137){\makebox(0,0)[l]{DTHETA}}
\put(282,137){\line(-1,4){4}}
\put(202,117){\line(-1,4){4}}
\put(198,133){\line(4,1){80}}
\put(202,117){\line(4,1){80}}
\put(150,67){\line(1,0){80}}
\put(150,83){\line(1,0){80}}
\put(150,67){\line(0,1){4}}
\put(150,79){\line(0,1){4}}
\put(230,67){\line(0,1){16}}
\end{picture}
\caption{Example of Misplacement in the $(x,s)$-plane}
\label{F-XSDISP}
\end{figure}
 
\begin{figure}[ht]
\centering
\setlength{\unitlength}{1pt}
\begin{picture}(400,150)
\thinlines
\put(204,75){\line(1,0){96}}
\put(196,75){\vector(-1,0){96}}
\put(305,75){\makebox(0,0)[l]{\shortstack{horizontal \\plane}}}
\put(110,65){\makebox(0,0)[l]{x}}
\put(200,75){\circle{8}}
\put(200,75){\makebox(0,0){\(\times\)}}
\put(210,65){\makebox(0,0){s}}
\put(200,0){\line(0,1){71}}
\put(200,79){\vector(0,1){71}}
\put(190,140){\makebox(0,0){y}}
\thicklines
\put(196,76.3){\vector(-3,1){70}}
\put(204,73.7){\line(3,-1){70}}
\put(198.7,72){\line(-1,-3){23.3}}
\put(201.3,78){\vector(1,3){23.3}}
\bezier{20}(260,75)(260,65)(257,56)
\put(259,62){\vector(-1,-3){2}}
\put(265,67){\makebox(0,0)[l]{DPSI}}
\put(160,55){\line(3,-1){60}}
\put(180,115){\line(3,-1){60}}
\put(160,55){\line(1,3){20}}
\put(220,35){\line(1,3){20}}
\put(202.8,77.8){\line(1,1){50}}
\put(197.2,72.2){\line(-1,-1){50}}
\bezier{20}(213,114)(222,111)(229,104)
\put(227,106){\vector(1,-1){2}}
\put(233,125){\makebox(0,0){ROT}}
\end{picture}
\caption{Example of Misplacement in the $(x,y)$-plane}
\label{F-XYDISP}
\end{figure}
 
\begin{figure}[ht]
\centering
\setlength{\unitlength}{1pt}
\begin{picture}(400,140)(0,30)
\thinlines
\put(50,75){\line(1,0){96}}
\put(154,75){\vector(1,0){246}}
\put(50,75){\makebox(0,0)[r]{\shortstack{original \\beam line}}}
\put(390,65){\makebox(0,0)[r]{s}}
\put(150,75){\circle{8}}\put(150,75){\circle*{2}}
\put(140,85){\makebox(0,0){x}}
\put(150,30){\line(0,1){41}}
\put(150,79){\vector(0,1){91}}
\put(140,160){\makebox(0,0){y}}
\put(130,55){\vector(1,1){15}}
\put(130,55){\makebox(0,0)[tr]{\shortstack{original entrance \\
of the magnet}}}
\thicklines
\put(150,55){\vector(1,0){50}}
\put(175,45){\makebox(0,0){DS}}
\put(130,75){\vector(0,1){50}}
\put(125,100){\makebox(0,0)[r]{DY}}
\thinlines
\put(125,125){\line(1,0){185}}
\put(200,50){\line(0,1){120}}
\put(200,125){\circle*{4}}
\thicklines
\put(168,133){\vector(4,-1){144}}
\put(189,81){\vector(1,4){22}}
\bezier{30}(300,125)(300,113)(297,101)
\put(299,109){\vector(-1,-4){2}}
\put(305,113){\makebox(0,0)[l]{DPHI}}
\put(278,97){\line(1,4){4}}
\put(198,117){\line(1,4){4}}
\put(202,133){\line(4,-1){80}}
\put(198,117){\line(4,-1){80}}
\put(150,67){\line(1,0){80}}
\put(150,83){\line(1,0){80}}
\put(150,67){\line(0,1){4}}
\put(150,79){\line(0,1){4}}
\put(230,67){\line(0,1){16}}
\end{picture}
\caption{Example of Misplacement in the $(y,s)$-plane}
\label{F-YSDISP}
\end{figure}
 
\begin{figure}[ht]
\centering
\setlength{\unitlength}{1pt}
\begin{picture}(400,150)(0,20)
\thinlines
\put(204,75){\line(1,0){96}}
\put(196,75){\vector(-1,0){196}}
\put(305,75){\makebox(0,0)[l]{\shortstack{horizontal \\plane}}}
\put(10,65){\makebox(0,0)[l]{x}}
\put(200,75){\circle{8}}\put(200,75){\makebox(0,0){\(\times\)}}
\put(220,55){\vector(-1,1){15}}
\put(220,55){\makebox(0,0)[tl]{\shortstack{true beam \\position}}}
\put(210,85){\makebox(0,0){s}}
\put(200,30){\line(0,1){41}}
\put(200,79){\vector(0,1){101}}
\put(190,170){\makebox(0,0){y}}
\put(100,50){\vector(0,1){125}}
\put(150,125){\vector(-1,0){100}}
\put(100,125){\circle*{4}}
\put(80,145){\vector(1,-1){15}}
\put(80,145){\makebox(0,0)[br]{\shortstack{beam position given \\
by the monitor}}}
\thicklines
\put(200,55){\vector(-1,0){100}}
\put(150,45){\makebox(0,0){MREX}}
\put(80,75){\vector(0,1){50}}
\put(70,100){\makebox(0,0)[r]{MREY}}
\end{picture}
\caption{Example of Read Errors in a monitor}
\label{F-ERMONI}
\end{figure}
 
\section{Field Error Definitions}
\label{S-FIELDEF}
\index{field errors}\index{error!field}
Field errors can be entered as relative or absolute errors.
Different multipole components can be specified with
different kinds of errors (relative or absolute).
If an attempt is made to assign both a relative
and an absolute error to the same multipole component,
the absolute error is used and a warning is given.
Relative errors cannot be assigned to an element of the type
\ttindex{MULTIPOLE}.
Relations between absolute and relative field errors are listed
below.
\par All field errors are specified as the integrated value
\(\int K_{i}ds\) along the magnet axis in \(\mathrm{m}^{-i}\)
(see Section~\ref{S-FIELD}).
At present field errors may only affect field components
allowed as normal components in a magnet.
This means for example that a dipole may have errors of the type dipole,
quadrupole, sextupole, and octupole; but not of the type decapole.
There is no provision to specify a global relative
excitation error affecting all field components in a combined function
magnet.
Such an error may only be entered by defining the same relative error
for all field components.
\par Field errors can be specified for all magnetic elements
by one of the statements
\mybox{
xxxxxxxx\=\kill
EFIELD, \>RANGE=range,TYPE=name,RADIUS=real,DBL=real,DKL(i)=real,\& \\
        \>DBLR=real,DKLR(i)=real,BROT=real,ROT(i)=real \\
EFCOMP, \>RANGE=range,TYPE=name,DBLN=real,DKLN(i)=real,\& \\
        \>DBLS=real,DKLS(i)=real
}
Each new \ttindex{EFIELD} or \ttindex{EFCOMP} statement replaces the
field errors for all elements in its range (s).
Any old field errors present in the range are discarded.
{\tt EFIELD} defines the error in terms of relative or absolute amplitude
and rotation;
while {\tt EFCOMP} defines them in terms of absolute components.
 
The meaning of the attributes is:
\begin{mylist}
\ttitem{RANGE}
A description of the beam elements in which the error occurs
(see Section~\ref{S-OBSAT}).
It can contain up to five different ranges:
\myxmp{
EFIELD,QF,QD[2]/QD[4],\#3/\#5,\#13/\#15
}
\ttitem{TYPE}
Label(s) attached to physical element(s) by a {\tt TYPE=name} clause.
May contain up to five different types like
\myxmp{
EFIELD,TYPE=MQ,MA,MB,MC,MD
}
\ttitem{RADIUS}
Radius \(R\) were \(\hbox{\tt DKLR}(i), 1\leq i\leq 10\) is specified
(default~1~m).
This attribute is required if {\tt DBLR } or any {\tt DKLR(i)} is specified.
\ttitem{DBL}
Absolute error amplitude for the dipole strength (default:~0~rad).
Its value is \(\int K_{0}ds\) over the length of the magnet
(see below).
\ttitem{DKL(i)}
Absolute error amplitude for the multipole strength with \((2*i + 2)\) poles
(default:~0~\(\mathrm{m}^{-n}\))
Its value is \(\int K_{i}ds\) over the length of the magnet
(see below).
\ttitem{DBLR}
Relative error in the dipole strength (default:~0, see below).
\ttitem{DKLR(i)}
Relative error in multipole strength with \((2*i + 2)\) poles
(default:~0, see below).
This attribute requires that {\tt RADIUS} is also given.
\ttitem{BROT}
Rotation angle \(\phi_{0}\) for the dipole error (default:~0,
see Figure~\ref{F-XYDISP}).
\ttitem{ROT(i)}
Rotation angle \(\phi_{i}\) for the multipole with \((2*i + 2)\) poles,
where \(1\leq i\leq 10\) (default:~0~rad, see Figure~\ref{F-XYDISP}).
\ttitem{DBLN}
Absolute error for the horizontal dipole strength (default:~0).
\ttitem{DKLN(i)}
Absolute error for the normal multipole strength with \((2*i + 2)\) poles
(default:~0).
\ttitem{DBLS}
Absolute error for the vertical dipole strength (default:~0).
\ttitem{DKLS(i)}
Absolute error in skewed multipole strength with \((2*i + 2)\) poles
(default:~0).
\end{mylist}
The dimensions for {\tt DKL()}, {\tt DKLR()}, {\tt DKLN()}, {\tt DKLS()},
and {\tt ROT()} are preset to 10 in the command dictionary,
but this can be changed easily.
Examples:
\myxmp{
EFIELD,TYPE=MQ,DBL=5.0E-4,DKL(3)=0.0025*RANF(),DKL(5)=0.0092*GAUSS()\\
EFIELD,TYPE=MQ,DBLN=5.0E-4,DKLN(3)=0.0025*RANF(),DKLN(5)=0.0092*GAUSS()
}
Refer to Section~\ref{S-DEFAT} for random value formats.
\subsection{Field Errors in Bending Magnets}
\index{dipole errors}\index{error!dipole}
\index{bending magnet errors}
Only the normal dipole, quadrupole, and sextupole components
of the error field are considered in a bending magnet.
They are rotated together with the main field.
The error field components have the values:
\[
\eqarray{
\Delta K_{0}L&=&\hbox{\tt DBL}=\hbox{\tt DBLN}=K_{0}L\cdot\hbox{\tt DBLR}, \\
\Delta K_{1}L&=&\hbox{\tt DKL(1)}\cos\phi_{1}=\hbox{\tt DKLN(1)}=
K_{0} L\frac{1!}{R}\cdot\hbox{\tt DKLR(1)}\cdot\cos\phi_{1}, \\
\Delta K_{2}L&=&\hbox{\tt DKL(2)}\cos\phi_{2}=\hbox{\tt DKLN(2)}=
K_{0} L\frac{2!}{R^{2}}\cdot\hbox{\tt DKLR(2)}\cdot\cos\phi_{2}, \\
\Delta K_{3}L&=&\hbox{\tt DKL(3)}\cos\phi_{3}=\hbox{\tt DKLN(3)}=
K_{0} L\frac{3!}{R^{3}}\cdot\hbox{\tt DKLR(3)}\cdot\cos\phi_{3}.
}
\]

\subsection{Field Errors in Quadrupoles}
\index{quadrupole errors}\index{error!quadrupole}
Only the normal quadrupole component of the error field,
that is the excitation error, is considered.
It is rotated together with the main field.
The error field component has the value:
\[
\Delta K_{1}=\hbox{\tt DKL(1)}\cdot\cos\phi_{1}=\hbox{\tt DKLN(1)}=
K_{1}L\cdot\hbox{\tt DKLR(1)}\cdot\cos\phi_{1}.
\]
\subsection{Field Errors in Sextupoles}
\index{sextupole errors}\index{error!sextupole}
Only the normal sextupole component of the error field,
that is the excitation error, is considered.
It is rotated together with the main field.
The error field component has the value:
\[
\Delta K_{2}L=\hbox{\tt DKL(2)}\cdot\cos\phi_{2}=\hbox{\tt DKLN(2)}=
K_{2}L\cdot\hbox{\tt DKLR(2)}\cdot\cos\phi_{2}.
\]
\subsection{Field Errors in Multipoles}
\index{multipole errors}\index{error!multipole}
Error fields and main field are rotated separately and then added.
The error field components are always absolute values:
\[
\eqarray{
(\Delta B_{x0}+i\cdot\Delta B_{y0})L&=&
\hbox{\tt DBL}\cdot\exp(-i\cdot\hbox{\tt BROT})=\hbox{\tt DBLN}+
i\cdot\hbox{\tt DBLS}, \\
(\Delta K_{xk}+i\cdot\Delta B_{yk})L&=&
\hbox{\tt DKL}(k)\cdot\exp(-(k+1)i\cdot\hbox{\tt ROT}(k))=
\hbox{\tt DKLN}(k)+i\cdot\hbox{\tt DKLS}(k).
}
\]

\subsection{Field Errors in Orbit Correctors}
\index{corrector errors}\index{error!corrector}
The dipole error field is added to the main field and then rotated.
A rotation (\ttindex{TILT}) must not be used when closed orbit corrections
are to be computed.
The error field components are always absolute values:
\[
(\Delta B_{0}+i\cdot\Delta B_{y0})L=\hbox{\tt DBL}\exp(-i\cdot\hbox{\tt BROT})=
\hbox{\tt DBLN}+i\cdot\hbox{\tt DBLS}.
\]
 
\section{Error Option Command}
\label{S-EOPTCOM}
\index{error!options}
\ttindex{EOPT}
\index{random!seed}\index{seed (random generator)}
The random generator for MAD is taken from~\cite{B-KNUTH}.
The error option command specifies different seeds for random values:
\mybox{
EOPT,SEED=integer
}
\begin{mylist}
\ttitem{SEED}
Selects a particular sequence of random values.
A {\tt SEED} value is an integer in the range \([0 \cdots 999999999]\)
(default:~123456789).
{\tt SEED} alone continues with the current sequence
(see also Section~\ref{S-FLTAT}).
\ttitem{ADD}
If this logical flag is set,
an {\tt EALIGN}, {\tt EFIELD}, or {\tt EFCOMP}, causes the errors to
be added on top of existing ones.
If it is not set,
new errors overwrite any previous definitions.
\end{mylist}
Example:
\myxmp{
EOPT,SEED=987456321
}
 
\section{Error Print Command}
\label{S-EPRICOM}
\index{error!printing}
This command prints a table of errors assigned to elements.
The range for these elements has to be specified.
Field errors are printed as absolute errors,
because all relative errors are transformed to the corresponding
absolute error at definition time.
An error print is requested by the \ttindex{EPRINT} statement
\mybox{
EPRINT,RANGE=range\{,range\},TYPE=name\{,name\},FULL
}
It has three attribute:
\begin{mylist}
\ttitem{RANGE}
Up to five ranges for printing (see Section~\ref{S-OBSAT}).
\ttitem{TYPE}
Up to five element types for printing.
\ttitem{FULL}
If this logical flag is true, all positions are printed.
\end{mylist}
Example:
\myxmp{
EPRINT,QD[1]/QD[5]
}
The selection of positions is done as for {\tt PRINT}
(see Section~\ref{S-PRINT}).
\index{error!definitions|)}

\chapter{Closed Orbit Commands}
\label{S-CORREC}
\index{orbit correction|(}
 
\begin{table}[ht]
\caption{Commands Related to the Closed Orbit}
\vspace{1ex}
\label{T-CORREC}
\centering
\begin{tabular}{|l|p{0.6\textwidth}|l|}
\hline
Name &Function &Section \\
\hline
\ttindex{CORRECT}&Complete correction algorithm &\ref{S-CORRECT} \\
\ttindex{GETDISP}&Read table of dispersion readings &\ref{S-GETDISP} \\
\ttindex{GETKICK}&Read table of corrector settings &\ref{S-GETKICK} \\
\ttindex{GETORBIT}&Read table of monitor readings &\ref{S-GETORB} \\
\ttindex{MICADO}&Correction by MICADO algorithm &\ref{S-MICADO} \\
\ttindex{PUTDISP}&Write table of dispersion readings &\ref{S-GETDISP} \\
\ttindex{PUTKICK}&Write table of corrector settings &\ref{S-GETKICK} \\
\ttindex{PUTORBIT}&Write table of monitor readings &\ref{S-GETORB} \\
\ttindex{USEKICK}&Activate/deactivate Correctors &\ref{S-USECOM} \\
\ttindex{USEMONITOR}&Activate/deactivate Monitors &\ref{S-USECOM} \\
\hline
\end{tabular}
\end{table}
 
There are six commands related to the closed orbit,
listed in Table~\ref{T-CORREC}.
Before using one of them a beam line must be selected
by means of a \ttindex{USE} command (see Section~\ref{S-USE}).

\section{GETDISP and PUTDISP Statements}
\label{S-GETDISP}
The two commands\ttnindex{GETDISP}\ttnindex{PUTDISP}
\mybox{
GETDISP,FILENAME=file-name \\
PUTDISP,FILENAME=file-name
}
serve the following purposes:
\begin{mylist}
\ttitem{GETDISP}
Reads a TFS file {\tt file-name} containing dispersion readings for the
orbit position monitors
(see Appendix~\ref{S-TFS}).
This file may have been written by a \ttindex{PUTDISP} command
or by an external program.
The readings can be used in a subsequent \ttindex{MICADO} command
to find corrector settings for correcting the vertical dispersion.
\ttitem{PUTDISP}
Finds the closed orbit and writes the dispersion readings at all orbit
position monitors on a TFS file {\tt file-name}.
(see Appendix~\ref{S-TFS}).
This file may be read again by MAD or an external program.
\end{mylist}
The format of {\tt file-name} is described in Appendix~\ref{A-FILES}.
\ttnindex{FILENAME}
The default {\tt file-name} is {\tt dispersion}.
The table must have the following columns:
\begin{mylist}
\ttitem{PUNAME}
Name of the monitor (string).
\ttitem{DX}
Horizontal dispersion (real).
\ttitem{DY}
Vertical dispersion (real).
\ttitem{STATUS}
Integer zero for active, nonzero for inactive.
\end{mylist}
 
\section{GETKICK and PUTKICK Statements}
\label{S-GETKICK}
The two commands
\ttnindex{GETKICK}
\ttnindex{PUTKICK}
\mybox{
GETKICK,FILENAME=file-name [,ADD]\\
PUTKICK,FILENAME=file-name [,PLANE=X|Y]
}
serve the following purposes:
\begin{mylist}
\ttitem{GETKICK}
Reads a TFS file {\tt file-name} containing the settings for the
orbit correctors (see Appendix~\ref{S-TFS}).
\index{TFS table}\index{table!TFS}
This file may have been written by a \ttindex{PUTKICK} command
or by an external program.
The excitations will be used for any subsequent closed orbit search,
unless changed by \ttindex{MICADO} or \ttindex{CORRECT} commands.
If the flag \ttindex{ADD} is true, the setting read are added to any
previous settings.
\ttitem{PUTKICK}
Writes the current setting of the orbit correctors on a TFS file
{\tt file-name}.
(see Appendix~\ref{S-TFS}).
This file may be read again by MAD or an external program.
\end{mylist}
The format of {\tt file-name} is described in Appendix~\ref{A-FILES}.
\ttnindex{FILENAME}
The default {\tt file-name} is {\tt setting}.
For reading the table must have the following columns:
\begin{mylist}
\item[\tt STR\_NAME]\index{STR NAME@{\tt STR\_NAME}}
Name of the corrector (string).
\item[\tt K\_N\_H]\index{K N H@{\tt K\_N\_H}}
Horizontal setting (real).
\item[\tt K\_N\_V]\index{K N V@{\tt K\_N\_V}}
Vertical setting (real).
\ttitem{STATUS}
Integer zero for active, nonzero for inactive.
\end{mylist}
By default the table written has the same format.
If the attribute {\tt PLANE=X} or {\tt PLANE=Y} is entered,
\ttnindex{PLANE}
only the strengths for that plane are written,
and there is only one column for settings, headed by {\tt K\_N}.
Such a table {\em cannot} be read by MAD.
 
\section{GETORBIT and PUTORBIT Statements}
\label{S-GETORB}
The two commands
\ttnindex{GETORBIT}
\ttnindex{PUTORBIT}
\mybox{
GETORBIT,FILENAME=file-name \\
PUTORBIT,FILENAME=file-name
}
serve the following purposes:
\begin{mylist}
\ttitem{GETORBIT}
Reads a TFS file {\tt file-name} containing orbit readings for the
orbit position monitors
(see Appendix~\ref{S-TFS}).
This file may have been written by a \ttindex{PUTORBIT} command
or by an external program.
The readings can be used in a subsequent \ttindex{MICADO} command
to find corrector settings for correcting the orbit positions.
\ttitem{PUTORBIT}
Finds the closed orbit and writes the readings of all orbit
position monitors on a TFS file {\tt file-name}
(see Appendix~\ref{S-TFS}).
This file may be read again by MAD or an external program.
\end{mylist}
The format of {\tt file-name} is described in Appendix~\ref{A-FILES}.
\ttnindex{FILENAME}
The default {\tt file-name} is {\tt orbit}.
The table must have the following columns:
\begin{mylist}
\ttitem{PUNAME}
Name of the monitor (string).
\ttitem{X}
Horizontal position (real).
\ttitem{Y}
Vertical position (real).
\ttitem{STATUS}
Integer zero for active, nonzero for inactive.
\end{mylist}

\section{USEKICK and USEMONITOR, Activate/deactivate elements}
\label{S-USECOM}
\ttnindex{USEKICK}
\ttnindex{USEMONITOR}
To provide more flexibility with orbit and dispersion correction two
commands are provided:
\mybox{
USEMONITOR, \=STATUS=flag,MODE=name,RANGE=\{range\} \\
USEKICK,    \>STATUS=flag,MODE=name,RANGE=\{range\} \\
}
The purpose of the two commands is:
\begin{mylist}
\ttitem{USEMONITOR}
Activates or deactivates a selection of beam position monitors.
This command affects elements of types \ttindex{MONITOR}, \ttindex{HMONITOR},
of \ttindex{VMONITOR}.
\ttitem{USEKICK}
Activates or deactivates a selection of orbit correctors.
This command affects elements of types \ttindex{KICKER}, \ttindex{HKICKER},
of \ttindex{VKICKER}.
\end{mylist}
Both commands have the same attributes:
\begin{mylist}
\ttitem{STATUS}
If this flag is true (on), the selected elements are activated.
Active orbit monitor readings will be considered,
and active correctors will be allowed to vary in subsequent correction
commands.
Inactive elements will be ignored subsequently.
\ttitem{MODE}
This name may take the value \ttindex{ALL}, in which case all correctors
or monitors will be set to active or inactive.
For the \ttindex{USEKICK} command two more values are possible:
\begin{mylist}
\ttitem{USED}
Changes all correctors which have been set to a non-zero value
by a preceding correction.
\ttitem{UNUSED}
Changes all correctors which have never been set by a preceding
correction.
\end{mylist}
\ttitem{RANGE}
This attribute is only considered if \ttindex{MODE} is {\em not} given.
Up to five ranges may be specified to select correctors or monitors to
be affected.
\end{mylist}
One may do two iterations with disjoint sets of correctors as follows:
\myxmp{
XXXXXXXXXXXXXXXXXXXXXX\=\kill
USE,...               \>! set working beam line \\
...                   \>! define imperfections \\
CORRECT,NCORR=32      \>! use 32 correctors \\
USEKICK,OFF,USED      \>! deactivate used correctors \\
CORRECT,NCORR=32      \>! uses different set of correctors
}
Forcing reuse of the same set is achieved thus:
\myxmp{
XXXXXXXXXXXXXXXXXXXXXX\=\kill
USE,...               \>! set working beam line \\
...                   \>! define imperfections \\
CORRECT,NCORR=32      \>! use 32 correctors \\
USEKICK,OFF,UNUSED    \>! deactivate unused correctors \\
CORRECT,NCORR=32      \>! uses same set of correctors
}

\section{MICADO Statement}
\label{S-MICADO}
The command \ttindex{MICADO} assumes that the closed orbit of the machine
is known,
and that the table of monitor readings
(and optionally of dispersion readings) exists.
These can be the result of previous \ttindex{GETORBIT} and
\ttindex{GETDISP} commands.
\mybox{
MICADO, \=ERROR=real,NCORR=integer [,C2LIST] [,M1LIST] \\
        \>[,DXWEIGHT=real] [,DXWEIGHT=real] [PLANE=letter]
}
It causes closed orbit correction by the MICADO method~\cite{B-AUT73}.
The program assumes that all correctors are available for use,
but it limits its choice to the most effective \ttindex{NCORR} correctors.
The monitors are not used for orbit or dispersion readings,
if flagged as inactive in the corresponding TFS files read.
The attributes have the following meaning:
\begin{mylist}
\ttitem{ERROR}
\index{correction, accuracy}
The desired accuracy of the correction
(r.m.s. error of the closed orbit, default:~0~{\rm m}).
\ttitem{NCORR}
\index{corrector number}
The maximum number of correctors to be used
(correctors are selected by the program, default:~all).
\ttitem{DXWEIGHT}
The weight for horizontal dispersion correction.
At present this should be left at zero to omit dispersion correction
in the horizontal plane.
\ttitem{DYWEIGHT}
The weight for vertical dispersion correction.
If this weight is non-zero,
an attempt is made to correct the vertical dispersion.
\ttitem{PLANE}
If this attribute is \ttindex{X}, only the horizontal correction is made;
it it is \ttindex{Y}, only the vertical correction is made.
In all other cases both planes are corrected.
\end{mylist}
Two attributes affect the printing of tables:
\index{corrector tables}
\index{monitor tables}
\begin{mylist}
\ttitem{C2LIST}
Corrector settings after correction,
\ttitem{M1LIST}
Monitor readings before correction,
\end{mylist}
Both have the same possible values:
\begin{mylist}
\ttitem{NONE}
Print summary only for this table (default),
\ttitem{USED}
Print table of all used correctors or monitors,
plus summary,
\ttitem{ALL}
Print table of all correctors or monitors, plus summary.
\end{mylist}
Example:
\myxmp{
MICADO,ERROR=1.E-4,C2LIST,M1LIST
}
 
\section{CORRECT Statement}
\label{S-CORRECT}
The \ttindex{CORRECT} statement makes a complete closed orbit correction
by the MICADO method, using the {\em computed} values at the monitors:
\mybox{
CORRECT, \=NCORR=integer,ERROR=value,ITERATE=integer,\& \\
         \>C1LIST,C2LIST,M1LIST,M2LIST\& \\
         \>[,DXWEIGHT=real] [,DYWEIGHT=real] [,PLANE=letter]
}
The program always assumes that all correctors and monitors
are available for use,
but at each iteration it limits its choice to \ttindex{NCORR} most
effective correctors.
The attributes have the following meaning:
\begin{mylist}
\ttitem{NCORR}
\index{corrector number}
The maximum number of correctors to be used
(correctors are selected by the program, default:~all).
At each iteration the program may select a different set
of \ttindex{NCORR} correctors, thus using more than \ttindex{NCORR} correctors
in total.
\ttitem{ERROR}
\index{correction, accuracy}
The desired accuracy of the correction
(r.m.s. error of the closed orbit, default:~0{\rm m}).
\ttitem{ITERATE}
\index{correction, iteration}
The number of iterations to be made on the non-linear problem
(default:~1).
If {\tt ITERATE=0}, the monitor and corrector tables are printed,
but no correction is made.
\ttitem{DXWEIGHT}
The weight for horizontal dispersion correction.
At present this should be left at zero to omit dispersion correction
in the horizontal plane.
\ttitem{DYWEIGHT}
The weight for vertical dispersion correction.
If this weight is non-zero,
an attempt is made to correct the vertical dispersion.
\ttitem{PLANE}
If this attribute is \ttindex{X}, only the horizontal correction is made;
it it is \ttindex{Y}, only the vertical correction is made.
In all other cases both planes are corrected.
\end{mylist}
Four attributes affect the printing of tables:
\index{corrector tables}
\index{monitor tables}
\begin{mylist}
\ttitem{C1LIST}
Corrector settings before correction,
\ttitem{C2LIST}
Corrector settings after correction,
\ttitem{M1LIST}
Monitor readings before correction,
\ttitem{M2LIST}
Monitor readings after correction.
\end{mylist}
All four have the same possible values:
\begin{mylist}
\ttitem{NONE}
Print summary only for this table (default),
\ttitem{USED}
Print table of all used correctors or monitors,
plus summary,
\ttitem{ALL}
Print table of all correctors or monitors, plus summary.
\end{mylist}
Example:
\myxmp{
CORRECT,ERROR=1.E-4,NCORR=100,ITERATE=3,C2LIST,M1LIST
}
\index{orbit correction|)}

\chapter{Plotting and Tabulating}
\label{S-PLOT}
\index{lattice functions}\index{functions!lattice}
\index{plot|(}
\begin{table}[ht]
\caption{Plotting Commands}
\vspace{1ex}
\label{T-PLOT}
\centering
\begin{tabular}{|l|p{0.6\textwidth}|l|}
\hline
Name &Function &Section \\
\hline
\ttindex{SETPLOT}&Set plot options &\ref{S-SETPLOT} \\
\ttindex{RESPLOT}&Set default plot options &\ref{S-SETPLOT} \\
\ttindex{PLOT}   &Build a plot &\ref{S-PLOTC} \\
\ttindex{STRING} &Define a table expression &\ref{S-TABLE} \\
\ttindex{TABLE}  &Print selected columns of a table &\ref{S-TABLE} \\
\hline
\end{tabular}
\end{table}
 
\section{SETPLOT and RESPLOT Commands}
\label{S-SETPLOT}
The \ttindex{SETPLOT} command allows to specify parameters
common to all subsequent plots:
\mybox{
SETPLOT, \=FONT=integer,LWIDTH=real,XSIZE=real,YSIZE=real,\& \\
         \>ASCALE=real,LSCALE=real,SSCALE=real,TSCALE=real \\
}
The \ttindex{RESPLOT} command serves to reinstall the defaults
for the {\tt SETPLOT} command parameters:
\mybox{
RESPLOT
}
It is defined in the command dictionary as
\myxmp{
RESPLOT: \=SETPLOT,FONT=1,LWIDTH=1.,XSIZE=0.,YSIZE=0.,\& \\
         \>ASCALE=1.,LSCALE=1.,SSCALE=1.,TSCALE=1.
}

To change the plot size, a {\tt SETPLOT} command specifying both
\ttindex{XSIZE} and \ttindex{YSIZE} must be given before the first
\ttindex{PLOT} command.
These parameters must {\em both} be entered to be effective.
They allow to make long plots on the Versatec plotter.
The values (0.,0.) give the default on any device.
See \ttindex{GXPLOT} manual for details~\cite{B-GXPLOT}.
They may be set to almost the page size for A4~format output,
i.~e. to~20 and 28~cm.
\begin{mylist}
\ttitem{XSIZE}
Horizontal size of plot on the hard copy output device [cm],
\ttitem{YSIZE}
Vertical size of plot on the hard copy output device [cm].
\end{mylist}

 
All other \ttindex{SETPLOT} options become effective when set,
and remain so until redefined.
\begin{mylist}
\ttitem{FONT}
Character font, (default:~1, Roman). Other fonts exist, but are
normally specific to the GKS-package used
(see Appendix of the GKSGRAL manual~\cite{B-GKS}),
and device-dependent.
\ttitem{LWIDTH}
A line width scale factor applying to curves (device-dependent).
Accepted values are normally 1, 3, and 5 (default:~1).
\ttitem{ASCALE}
Scale factor for the size of the plotted
characters of the curve annotation,
\ttitem{LSCALE}
Scale factor for the size of the plotted
characters of the axis labels (scale),
\ttitem{SSCALE}
Scale factor for the size of the plotted
symbol at the point positions,
\ttitem{RSCALE}
Scale factor for the size of the plotted
characters of the axis text.
\end{mylist}
The default is~1.0 for all four scale factors.
 
\section{PLOT Command}
\label{S-PLOTC}
The \ttindex{PLOT} command produces one or several frames (pictures) at a time:
\mybox{
PLOT \=VAXIS=name,VAXIS1=name,VAXIS2=name,VAXIS3=name,VAXIS4=name,\& \\
     \>HAXIS=name,BARS=integer,STYLE=integer,SYMBOL=integer,\& \\
     \>MAXPLOT=integer,SORT=logical,SPLINE=logical,MULTIPLE=logical,\& \\
     \>FFT=logical,HMIN=real,HMAX=real,VMIN=real,VMAX=real,\& \\
     \>TABLE=name,TITLE=string,PARAM=name,RANGE=range,DELTAP=real,\& \\
     \>PARTICLE=integer,TURNS=integer
}
Its parameters are only valid for this one command:
\begin{mylist}
\ttitem{VAXIS}
Up to five names of dependent (vertical) variables to be plotted
{\em on the same axis (!)}.
If more than one vertical axis is required, one may enter up
to five variable names per axis via one of the parameters
\ttindex{VAXIS1}, \ttindex{VAXIS2}, \ttindex{VAXIS3}, \ttindex{VAXIS4}.
If \ttindex{VAXIS} is specified, the others are ignored.
Example:
\myxmp{
PLOT,HAXIS=S,VAXIS=BETX,BETY
}
gives a plot with \ttindex{BETX} and \ttindex{BETY}
both referring to the same axis, whereas
\myxmp{
PLOT,HAXIS=S,VAXIS1=BETX,VAXIS2=BETY
}
provides separate axes (with different scales, normally)
for the two variables.
\ttitem{HAXIS}
The name of the independent (horizontal) variable.
\ttitem{BARS}
Connect each point to the horizontal axis by a vertical line.
\ttitem{STYLE}
Line style selection for all curves.
The default value is 1 (solid line for all curves).
Further possible values are: 0 (no connecting line, i.e.
nothing at all if no symbol plotted, see below), 2 (dashed), 3 (dotted),
4 (dot-dashed), and others mentioned in the \ttindex{GKSGRAL} manual.
A special value is 100: when chosen, it provides line styles 1, 2, 3, 4,
1, 2, etc. for successive curves in the same frame.
\ttitem{SYMBOL}
This option will be mainly used for track plots, and
allows to plot symbols at the points which are plotted.
These can be one of the following:
``hardware'' symbols, selected by the values 1 (dot),
2 (+), 3 (*), 4 (o), or 5 (x);
the turn number (curve number for tables TWISS and TUNES)
in each picture, selected by value 100;
an upper case character A...Z (201 to 226), lower case character (301
to 326), or a digit 0...9, selected by 400 to 409. 200 and 300 act
similar to 100: 200 makes turns (curves) loop over A...Z,
300 over a...z.
The \ttindex{SSCALE} factor on the \ttindex{SETPLOT} command
applies to all of these except
the ``hardware'' symbols which are of fixed size.
\ttitem{MAXPLOT}
Defines the maximum number of pictures (frames) to be plotted
with one command (default:~10).
This is a protection against user mistakes mainly
(e.g. plotting one frame per \(s\) value around the whole machine).
\ttitem{SORT}
Logical flag, sorts the points in ascending
values of the abscissa.
\ttitem{SPLINE}
Logical flag, connects
curve points by a third order natural spline rather than
by straight line segments
(for \(\alpha\),~\(\beta\), \(\mu\)~and~\(D\) the correct formulae are
used for interpolation if plotted against \(s\)).
\ttitem{MULTIPLE}
\index{table!track}\index{track table}
Logical flag, only for {\tt TRACK} table plots: put all particles
into one frame.
\ttitem{FFT}
Plot the result of the Fourier analysis of the picture, rather than
the picture itself (not yet implemented).
\ttitem{TABLE}
Name of the table to be plotted.
If omitted, the last generated table will be used.
\ttitem{TITLE}
Text string to be put on top of the picture
(default:~run title as read on \ttindex{TITLE} command).
\ttitem{PARAM}
Frame selection parameter: \ttindex{S} or \ttindex{DELTAP} for the
{\tt TWISS} table,
\index{twiss table}\index{table!twiss}
\ttindex{PARTICLE} or \ttindex{TURNS} for the {\tt TRACK} table.
\ttitem{RANGE}
Machine range to be used (see Section~\ref{S-OBSAT}).
\ttitem{DELTAP}
Values of \ttindex{DELTAP} to be used (as on \ttindex{TWISS} command).
\ttitem{PARTICLE}
Particle number.
\ttitem{TURNS}
Turn number.
\end{mylist}
The axis ranges are specified by lower and upper values.
If neither value is given, scaling is fully automatic,
if both are given, they are used (possibly rounded).
If only the lower limit is given as zero,
and all corresponding curves lie entirely above or below zero,
the scale starts or ends at zero.
If only the upper limit is given as zero, the scale is symmetric to zero.
\begin{mylist}
\ttitem{HMIN}
Lower limit of the abscissa.
\ttitem{HMAX}
Upper limit of the abscissa.
\ttitem{VMIN}
Up to four lower limits for each of the four vertical axes.
\ttitem{VMAX}
Up to four upper limits for each of the four vertical axes.
\end{mylist}
 
\section{Axis and frame selection}
All variables from the {\tt TWISS} table, the {\tt TUNES} table,
and the {\tt TRACK} table can be plotted on the horizontal or vertical axis.
For a list, see Sections~\ref{S-VARIA},~\ref{S-TUNES},
and~\ref{S-TRTAB}.
In addition, the variables \ttindex{RBETX}, \ttindex{RBETY}, \ttindex{RBXMAX},
and \ttindex{RBYMAX} can be requested. They stand for the square root
of the variables behind the 'R'.
 
The variables \ttindex{S} and  \ttindex{DELTAP} in the {\tt TWISS} table, and
the variables \ttindex{PARTICLE} and \ttindex{TURNS} in the {\tt TRACK} table
are used as frame parameters \ttindex{PARAM}, either automatically or
user-driven. This means that for each value of this  variable  a new
picture  is plotted, as described below for the {\tt TWISS} table.

\subsection{TWISS table}
\index{twiss table}\index{table!twiss}
Depending on the user specification when calculating the table this will
normally contain Twiss parameter values for a number of \ttindex{S} values
(around the ring), and for one or more \ttindex{DELTAP} values
(see  Section~\ref{S-VARIA}).
The table may therefore be regarded as a ``matrix''
with \ttindex{S} as row index, \ttindex{DELTAP} as column index,
and each ``element'' consisting of all Twiss values for a given pair
(\ttindex{S}, \ttindex{DELTAP}),
both \ttindex{S} and \ttindex{DELTAP} included in these values.
 
A subset of the table can be chosen by specifying a \ttindex{RANGE}
for \ttindex{S},
and a list of values for \ttindex{DELTAP}.
If no {\tt S} range is given, all {\tt S} values are selected (!),
and similarly for {\tt DELTAP}: if no values given, all values are used.
 
Out of this (sub-)matrix the user can now select variables to be
plotted.
Let us suppose that the user always chooses \ttindex{BETY} as ordinate.
If \ttindex{S} is abscissa,
\ttindex{BETY} will be plotted against \ttindex{S} with
\ttindex{DELTAP} as ``frame parameter" (\ttindex{PARAM}),
i.e. for each selected value of {\tt DELTAP} there will be a new plot
frame.
Similarly for \ttindex{DELTAP} as abscissa: in this case,
\ttindex{BETY} will be plotted against \ttindex{DELTAP} for each valid
\ttindex{S},
giving a new plot frame each time
(this might be quite a lot if the user makes a mistake;
in order to avoid excessive output,
the number of frames per plot command will be limited,
with an option \ttindex{MAXPLOT} to explicitly changing this limit).
 
In the case that neither \ttindex{S} nor \ttindex{DELTAP} is horizontal axis
(e.g. plotting \ttindex{BETY} against \ttindex{BETX}, whatever this may yield),
the user has to specify in addition what his frame parameter is.
For each value of this frame parameter (\ttindex{S} or \ttindex{DELTAP})
the values in the corresponding row or column will be plotted
in a separate frame.
The default is \ttindex{DELTAP}.
The value of \ttindex{PARAM} is ignored if
\ttindex{HAXIS}=\ttindex{S}~or~\ttindex{DELTAP}, i.e. the "other" one is chosen
automatically (if \ttindex{HAXIS}=\ttindex{DELTAP}, then \ttindex{S} will act
as frame parameter).
 
Examples:
\myxmp{
PLOT,HAXIS=S,VAXIS=BETY
}
produces n plots (for the n \ttindex{DELTAP} values in the table)
with one curve each showing the dependence of \ttindex{BETY} on \ttindex{S}.
\myxmp{
PLOT,HAXIS=S,VAXIS=BETY,DELTAP=0.:0.01:0.01
}
produces two plots of this type, for \ttindex{DELTAP}=0. and 0.01
(provided the \ttindex{DELTAP} values exist in the table).
\myxmp{
PLOT,HAXIS=DELTAP,VAXIS=BETY,RANGE=\#E
}
produces one plot \ttindex{BETY} against \ttindex{DELTAP} at the position
of the last element in the table.
\myxmp{
PLOT,HAXIS=DELTAP,VAXIS=BETY
}
will produce 10 plots (default value of MAXPLOT)
for the first 10 \ttindex{S} values in the table.
\myxmp{
PLOT,HAXIS=BETX,VAXIS=BETY,PARAM=DELTAP,DELTAP=0.
}
will produce one plot (at \ttindex{DELTAP}=0),
the curve points being (\ttindex{BETX},\ttindex{BETY}) at all \ttindex{S} values.
Here, \ttindex{PARAM}=\ttindex{DELTAP} may be omitted
since this is the default.
\myxmp{
PLOT,HAXIS=BETX,VAXIS=BETY,PARAM=S,RANGE=\#E
}
will produce one plot at the last \ttindex{S} value,
with (\ttindex{BETX,BETY}) points for all \ttindex{DELTAP} values
in the table.

\subsection{TRACK table}
\index{track table}\index{table!track}
The {\tt TRACK} table (see~\ref{S-TRTAB})
has the same structure as the {\tt TWISS} table, with
\ttindex{PARTICLE} as row index, and \ttindex{TURNS} as column index
each of which can therefore be chosen as frame selection parameter.
The default parameter is \ttindex{PARTICLE}.
The value of \ttindex{PARAM} is ignored if
\ttindex{HAXIS}=\ttindex{PARTICLE}~or~\ttindex{TURNS},
i.e. the "other" one is chosen
automatically (if \ttindex{HAXIS}=\ttindex{TURNS},
then \ttindex{PARTICLE} will act as frame parameter).
The option
\ttindex{MULTIPLE} allows to plot the points of all particles in the
same frame.
\myxmp{
PLOT,HAXIS=X,VAXIS=PX,PARTICLE=1,3,5,MULTIPLE
}
will produce one plot of \ttindex{PX} versus \ttindex{X} for particles
1, 3, and 5.

\subsection{TUNES table}
\index{tunes table}\index{table!tunes}
The {\tt TUNES} table
(see Section~\ref{S-TUNES})
does not have frame-parameters.
If \ttindex{PARAM} is specified, it will be ignored.
\myxmp{
PLOT,TABLE=TUNES,HAXIS=DELTAP,VAXIS=QX
}
will produce a plot of the x-tune versus \ttindex{DELTAP}.

\section{Tune Grid}
MAD permits to plot an tune grid if the following two conditions
are both true:
\begin{itemize}
\item A plot of \(Q_x\) versus \(Q_y\) or of \(Q_y\) versus \(Q_x\)
is requested,
\item At least one selection criterion has been entered on the \ttindex{PLOT}
command.
\end{itemize}
MAD then plots all resonance lines that are inside the plot window,
and which fulfil conditions of the form
\[ k_x Q_x + k_y Q_y + k_s Q_s = p s, \]
where \(k_x\), \(k_y\), \(k_s\), and \(p\) are integers,
and \(s\) is the super-periodicity (integer).
Furthermore, the user must impose at least one condition of the form
\[ F(k_x,k_y,k_s)=n_1:n_2:n_3, \]
where \(F\) is an arithmetic function \(k_x\), \(k_y\), \(k_s\),
and integers only,
and only the operators ``\(+\)'', ``\(-\)'', ``\(*\)'',
and ``\(/\)'' are allowed.
\(n_1\), \(n_2\), and \(n_3\) are literal integers,
and \(n_1:n_2:n_3\) stands for a range.
It is interpreted like the range in a FORTRAN DO loop,
i.~e. it runs through the values \(n_1, n_1 + n_2, \ldots , n_3\).

Examples:
\[
k_s = -2:2:2 \Rightarrow k_s = -2, 0, +2; \qquad
k_x + k_y = 0:0:1 \Rightarrow k_x = - k_y.
\]
Note that integer arithmetic is used, therefore the condition
\[ 2 * (k_x + k_y / 2) * (k_s - 1) / 2 = -3:3:2 \]
will do something weird.

Criteria are specified via the following new attributes of the
\ttindex{PLOT} command:
\mybox{
PLOT,...,NTMAX=integer,QCONDi=string,integer1,integer2,integer3
}
with the meaning:
\begin{mylist}
\ttitem{NTMAX}
Maximum for \(|k_x| + |kz_y| + |k_s|\) (default: 20).
The absolute upper limits are \(|k_x| + |k_y| + |k_s| \le 20\),
and \(|k_s| \le 10\).
\ttitem{QCONDi}
The function \(F\), encoded as a quoted string,
and the three integers \(n_1,n_2,n_3\) for the selection criterion
(defaults: " ", 0, 0, 1).
The digit {\tt i} may run from~1 to~10 for up to 10~conditions.
When several selection criteria are entered,
they must all be true for a line to be plotted.
\end{mylist}
\noindent Example:
\myxmp{
PLOT,\=HAXIS=QX,VAXIS=QY,\&\\
     \>HMIN=21.,HMAX=23.,VMIN=18.,VMAX=20.,\&\\
     \>NTMAX=5,QS=0.2,\&\\
     \>QCOND1="KX",1,1,\&\\
     \>QCOND2="KS",-3,3,\&\\
     \>QCOND3="KX+KY",-10,10,10
}
will plot \(Q_y\) versus \(Q_x\),
and superpose the lines
\[ k_x Q_x = p s, k_x = 1; \qquad
k_s Q_s = p s, k_s = -3, -2, -2, 0, 1, 2, 3; \]
\[ k_x Q_x + k_y Q_y = p s, k_x + k_y = -10, 0, 10. \]

The line style reflects the importance of the resonance.
The thickness varies from \(|k_x| + |k_y| = 1\) (thickest) to
\(|k_x| + |k_y| = 5\) (thinnest) and remains constant from there on.
Lines with \(k_s = 0\) are solid, with \(|k_s| = 1\) dashed,
with \(|k_s| = 2\) dot-dashed, and with \(|k_s| > 2\) dotted.

\section{TABLE and STRING Commands}
\label{S-TABLE}
Any table created by MAD can be tabulated on the PRINT file by the command
\mybox{
TABLE,NAME=name,COLUMNS=name\{,name\},SUM=name\{,name\}
}
It has the attributes
\begin{mylist}
\ttitem{NAME}
Name of the table to be tabulated.
By default the last table created is used.
\ttitem{COLUMNS}
Up to~50 names whose values shall be tabulated.
These may be any of
\begin{itemize}
\item Name of a column in the table to be listed.
\item Name of a descriptor in the table to be listed.
\item Name of a global variable.
\item Name of a global {\tt STRING} expression (see below).
\item Name of an expression predefined for the table.
\end{itemize}
\ttitem{SUM}
Up to~50 name of columns to be summed.
The \ttindex{TABLE} command will compute (but not print) and sum up
each name for each table row,
and print the total after the table.
This may be used to compute approximate values for synchrotron integrals.
\end{mylist}

A \ttindex{STRING} expression is defined as follows:
\mybox{
label:STRING,"string"
}
where {\tt label} is the name given to the expression,
and {\tt string} (enclosed in single or double quotes)
must contain a valid expression.
Operands in such an expression may include:
\begin{itemize}
\item Name of a column in the table to be listed.
\item Name of a descriptor in the table to be listed.
\item Name of a global variable.
\item Name of an attribute of a command or definition.
\end{itemize}
A string may be redefined by repeating the {\tt STRING} command.

Example:
\myxmp{
OPTICS,... \\
SIGX:STRING,"SQRT(BEAM[EX]*BETX)" \\
CHROMX:STRING,"SQRT(BETX*BEAM[EX] + (DX*DELTAP)\(\hat{\ }2)\)\\
TABLE,NAME=OPTICS,COLUMN=K1L,K2L,BETX,SIGX,SUM=CHROMX
}

\subsection{Plotting Composite Values}
The mechanisms from the preceding subsection can also be used for plotting.
One may for example plot the beam size after a \ttindex{NORMAL} command by
\myxmp{
NORMAL,...\\
SIGX:STRING,\&\\
"SQRT(BEAM[EX]*(E11\(\hat{\ }\)2+E21\(\hat{\ }\)2)+
BEAM[EY]*E31\(\hat{\ }\)2+E42\(\hat{\ }\)2)+
BEAM[ET]*(E51\(\hat{\ }\)2+E61\(\hat{\ }\)2))"\\
SIGY:STRING,\&\\
"SQRT(BEAM[EX]*(E13\(\hat{\ }\)2+E23\(\hat{\ }\)2)+
BEAM[EY]*E33\(\hat{\ }\)2+E43\(\hat{\ }\)2)+
BEAM[ET]*(E53\(\hat{\ }\)2+E63\(\hat{\ }\)2))"\\
SIGT:STRING,\&\\
"SQRT(BEAM[EX]*(E15\(\hat{\ }\)2+E25\(\hat{\ }\)2)+
BEAM[EY]*E35\(\hat{\ }\)2+E45\(\hat{\ }\)2)+
BEAM[ET]*(E55\(\hat{\ }\)2+E65\(\hat{\ }\)2))"\\
PLOT,TABLE=EIGEN,HAXIS=S,VAXIS=SIGX,SIGY,SIGT
}
If desired, such expressions can be predefined and attached to a given
table type.
Such predefinitions would be placed in the MAD command dictionary:
\myxmp{
T\_TWISS:\=KEYWORD,PR=3,SP=2,\&\\
   \>GAMX=(S(3),=3\(\backslash\)*""),GAMY=(S(3),=3\(\backslash\)*""),\&\\
   \>SIGX=(S(3),=3\(\backslash\)*""),SIGY=(S(3),=3\(\backslash\)*"")\\
T\_TWISS:\>T\_TWISS,\&\\
   \>GAMX="(1+ALFX\(\hat{\ }\)2)/BETX",GAMY="(1+ALFY\(\hat{\ }\)2)/BETY",\&\\
   \>SIGX="SQRT(BETX*BEAM[EX])",SIGY="SQRT(BETY*BEAM[EY])"
}
The name of the definition must be the table type, prefixed by {\tt T\_}.
Note that the strings in the first definition must be dimensioned at~3,
thus allowing to enter information like dimension and axis label information
to the second definition.
\index{plot|)}

\chapter{HARMON Module}
\label{S-HARMON}
\index{HARMON@{\tt HARMON}|(}
 
\begin{table}[ht]
\caption{HARMON Commands}
\vspace{1ex}
\label{T-HARMON}
\centering
\begin{tabular}{|l|p{0.6\textwidth}|l|}
\hline
Name &Function &Section \\
\hline
\ttindex{HARMON}&Set up {\tt HARMON} tables &\ref{S-HSETUP} \\
\ttindex{HCELL}&Minimise aberrations &\ref{S-HCELL} \\
\ttindex{HCHROMATICITY}&Calculate chromaticity &\ref{S-HCHROM} \\
\ttindex{HFUNCTION}&Calculate harmonic functions&\ref{S-HCELL} \\
\ttindex{HRESONANCE}&Calculate resonance effects &\ref{S-HRESON} \\
\ttindex{HTUNE}&Tune chromaticity &\ref{S-HCHROM} \\
\ttindex{HWEIGHT}&Set matching weights for resonances&\ref{S-HCELL} \\
\ttindex{HVARY}&Make a multipole variable &\ref{S-HCELL} \\
\ttindex{ENDHARM}&Quit {\tt HARMON} program &\ref{S-HSETUP} \\
\hline
\end{tabular}
\end{table}
 
The {\tt HARMON} program by M. H. Donald~\cite{B-DON82}
can be called directly from MAD.%
\footnote{Adapted to MAD by D. Schofield (1983)
and J.M. Veuillen (1988); rewritten extensively 1989.}
Before calling {\tt HARMON}, a beam line must be selected
by means of a {\tt USE} command (see Section~\ref{S-USE}).
The {\tt HARMON} command sets up internal tables and
defines the emittances to be used for the analysis.
The internal tables are released and {\tt HARMON} mode is
terminated by the {\tt ENDHARM} command.
The available commands are listed in Table~\ref{T-HARMON}.

It is important to note that {\tt HARMON} {\em does not} consider the
following:
\begin{itemize}
\item Coupling effects,
\item Excitation errors ({\tt EFIELD, EFCOMP}),
\item Alignment errors ({\tt EALIGN}).
\end{itemize}
The integral formalism used does not allow to take these effects into
account, and users should be aware of this.
 
\section{Activating and Deactivating HARMON}
\label{S-HSETUP}
The {\tt HARMON} module is activated and the relevant tables are initialised
by the command
\mybox{
HARMON,FX=real,FY=real,FE=real
}
with the attributes:
\begin{mylist}
\ttitem{FX}
Number of horizontal standard deviations to be taken,
\ttitem{FY}
Number of vertical standard deviations to be taken,
\ttitem{FE}
Number of longitudinal standard deviations to be taken.
\end{mylist}
If one of these values is omitted or 0, the program uses 1.
The previous attribute {\tt OBSERVE} no longer exists.
If the machine has a symmetry, a second observation point is given
at the symmetry point.
 
The {\tt HARMON} module is deactivated and its tables are released
by the \ttindex{ENDHARM} command
\mybox{
ENDHARM
}
An example is given at the end of this Chapter.
 
\section{Chromaticity Calculation or Tuning}
\label{S-HCHROM}
\index{chromaticity}
\ttnindex{HCHROMATICITY}
The chromaticity is calculated by a command without attributes:
\mybox{
HCHROMATICITY
}
The chromaticity is tuned to a desired value by the \ttindex{HTUNE}
command
\index{chromaticity!tuning}
\mybox{
HTUNE,QX'=real,QY'=real,TOLERANCE=real
}
The desired values are entered as \(dQ/d\delta\) with the attributes
\begin{mylist}
\ttitem{QX'}
Desired horizontal chromaticity (default:~0),
\ttitem{QY'}
Desired vertical chromaticity (default:~0).
\ttitem{TOLERANCE}
Tolerance desired (default:~0, i.e. as good as feasible).
\end{mylist}
The multipoles to be varied are specified with a \ttindex{HVARY} command
(see Section~\ref{S-HCELL}).
Example:
\myxmp{
HVARY,SF[K2],STEP=0.001 \\
HVARY,SD[K2],STEP=0.001 \\
HVARY,SFK2,STEP=0.0001 \\
HTUNE,QX'=0.0,QY'=0.0
}
 
\section{Resonance Calculations}
\label{S-HRESON}
\index{resonances}
The resonance effects are computed by the \ttindex{HRESONANCE} command
\mybox{
HRESONANCE,ORDER=integer,DISP=logical
}
The order of the resonance is selected by the attribute
\begin{mylist}
\ttitem{ORDER}
Order of the resonance.
This causes computation of the resonances
\[
   N_{1}Q_{x}+N_{2}Q_{y}=p_{1},
   \qquad
   N_{1}Q_{x}-N_{2}Q_{y}=p_{2},
\]
with integer \(p_{1},\ p_{2}\), and the condition
\(N_{1}+N_{2} = {\tt ORDER}\),
\ttitem{DISP}
If set, the dispersion is included in the computation.
\end{mylist}
Contrary to previous versions of {\tt HARMON}, all sextupole components
are considered, i.e. all elements of the types \ttindex{RBEND},
\ttindex{SBEND}, \ttindex{SEXTUPOLE}, \ttindex{MULTIPOLE}.
Example:
\myxmp{
HRESONANCE,ORDER=3,DISP
}
 
\section{Distortion Functions}
\label{S-HCELL}
\index{distortion functions}\index{functions!distortion}
\index{resonances}
The \ttindex{HVARY} command makes a multipole component variable:
\mybox{
HVARY,NAME=variable,STEP=real,LOWER=real,UPPER=real
}
The command has the following attributes:
\begin{mylist}
\ttitem{NAME}
The name of a variable to be varied.
It must be the sextupole, octupole or decapole component
of an element having one of the types \ttindex{SBEND},
\ttindex{RBEND},
\ttindex{SEXTUPOLE}, or \ttindex{MULTIPOLE}.
\ttitem{STEP}
An initial step size to be used in matching (required),
\ttitem{LOWER}
Lower limit for the multipole strength,
\ttitem{UPPER}
Upper limit for the multipole strength.
\end{mylist}
The units for {\tt STEP}, {\tt LOWER}, and {\tt UPPER} are the same
as in the corresponding element definition.
Examples:
\myxmp{
HVARY,NAME=SF1[K2],STEP=0.001,LOWER=-1.0,UPPER=0.0 ! Sextupole \\
HVARY,NAME=MF1[K3L],STEP=0.001,LOWER=0.0,UPPER=1.0 ! Multipole
}
The \ttindex{HFUNCTIONS} command
\mybox{
HFUNCTIONS
}
computes the harmonic functions as defined in~\cite{B-DON82}.
For the chromaticities and dispersions, {\em but only for these},
the effects of octupoles and decapoles are also considered and may
be matched by varying those components.
The \ttindex{HCELL} command
\mybox{
xxxxxxxx\=\kill
HCELL, \=TOLERANCE=real,CALLS=integer,QX'=real,QY'=real,\& \\
       \>QX''=real,QY''=real,QX'''=real,QY'''=real,\& \\
       \>DQXDEX=real,DQYDEY=real,DQYDEX=real,\& \\
       \>DX'I=real,DX''I=real,BX'I=real,BY'I=real,RXI=real,RYI=real,\& \\
       \>DX'S=real,DX''S=real,BX'S=real,BY'S=real,RXS=real,RYS=real
}
adjusts the following quantities to the values entered:
\begin{mylist}
\ttitem{Tolerance}
Tolerance for the penalty function constructed.
\ttitem{CALLS}
Limit for the number of calls to the penalty function.
\ttitem{QX'}
First-order horizontal chromaticity.
\ttitem{QY'}
First-order vertical chromaticity.
\ttitem{QX''}
Second-order horizontal chromaticity.
\ttitem{QY''}
Second-order vertical chromaticity.
\ttitem{QX'''}
Third-order horizontal chromaticity.
\ttitem{QY'''}
Third-order vertical chromaticity.
\ttitem{DQXDEX}
Derivative of horizontal tune with horizontal emittance.
\ttitem{DQYDEY}
Derivative of vertical tune with vertical emittance.
\ttitem{DQYDEX}
Derivative of vertical tune with horizontal emittance.
\ttitem{DX'I}
Second-order horizontal dispersion at interaction point.
\ttitem{DX''I}
Third-order horizontal dispersion at interaction point.
\ttitem{BX'I}
Variation of \(\beta_x\) with energy at interaction point.
\ttitem{BY'I}
Variation of \(\beta_y\) with energy at interaction point.
\ttitem{RXI}
Contributions of horizontal resonances at interaction point.
\ttitem{RYI}
Contributions of vertical resonances at interaction point.
\ttitem{DX'S}
Second-order horizontal dispersion at symmetry point.
\ttitem{DX''S}
Third-order horizontal dispersion at symmetry point.
\ttitem{BX'S}
Variation of \(\beta_x\) with energy at symmetry point.
\ttitem{BY'S}
Variation of \(\beta_y\) with energy at symmetry point.
\ttitem{RXS}
Contributions of horizontal resonances at symmetry point.
\ttitem{RYS}
Contributions of vertical resonances at symmetry point.
\end{mylist}
The chromaticities and dispersions of all orders calculated in {\tt HARMON}
contain the effects of multipoles up to the decapole.
These multipole strengths may all be adjusted to minimise the distorsions.
All quantities are fitted in a least-squares sense.
The sextupole components to be varied must be specified in a
\ttindex{HVARY} command.
The default matching weights are one for the first-order chromaticities,
and zero for all other quantities.
Different weights can be set before calling \ttindex{HCELL}
with the \ttindex{HWEIGHT} command
\mybox{
HWEIGHT,\=QX'=real,QY'=real,QX''=real,QY''=real,QX'''=real,QY'''=real,\& \\
        \>DQXDEX=real,DQYDEY=real,DQYDEX=real,\& \\
        \>DX'I=real,DX''I=real,BX'I=real,BY'I=real,RXI=real,RYI=real,\& \\
        \>DX'S=real,DX''S=real,BX'S=real,BY'S=real,RXS=real,RYS=real
}
\ttindex{QX'}
\ttindex{QX''}
\ttindex{QX'''}
\ttindex{QY'}
\ttindex{QY''}
\ttindex{QY'''}
\ttindex{DQXDEX}
\ttindex{DQYDEY}
\ttindex{DQYDEX}
\ttindex{DX'I}
\ttindex{DY'I}
\ttindex{DX'S}
\ttindex{DY'S}
\ttindex{BX'I}
\ttindex{BY'I}
\ttindex{BX'S}
\ttindex{BY'S}
\ttindex{RXI}
\ttindex{RYI}
\ttindex{RXS}
\ttindex{RYS}
 
\section{Example for a HARMON Sequence}
\index{examples!HARMON@{\tt HARMON}}
\myxmp{
xxxxxxxx\=\kill
USE,      \>OCT,SUPER=4,SYMM \\
BEAM,     \>PARTICLE=ELECTRON,ENERGY=50,\& \\
          \>EX=0.0645E-6,EY=0.03225E-6,SIGE=1.25E-3 \\
! \\
HARMON,   \>FX=10,FY=10,FE=7 \\
! \\
HCHROMATICITY \\
HVARY,    \>NAME=SF1,STEP=0.001 ! Chromaticity sextupoles \\
HVARY,    \>NAME=SD1,STEP=0.001 \\
HTUNE,    \>QX'=0,QY'=0 \\
! \\
HRESONANCE,ORDER=3 \\
! \\
HWEIGHT,  \>QX''=1.0,QY''=1.0,QX'''=10.0,QY'''=10.0 \\
HVARY,    \>SF2,STEP=0.001,LOWER=0.,UPPER=1. \\
HVARY,    \>SD2,STEP=0.001,LOWER=-1.,UPPER=0. \\
HVARY,    \>SF3,STEP=0.001,LOWER=0.,UPPER=1. \\
HVARY,    \>SD3,STEP=0.001,LOWER=-1.,UPPER=0. \\
HCELL     \>! Chromaticities minimised to all three orders \\
! \\
HRESONANCE \\
ENDHARM
}
\index{HARMON@{\tt HARMON}|)}

\chapter{Matching Module}
\label{S-MATCH}
\index{matching|(}
 
\begin{table}[ht]
\caption{Matching Commands}
\vspace{1ex}
\label{T-MATCH}
\centering
\begin{tabular}{|l|p{0.6\textwidth}|l|}
\hline
Name &Function &Section \\
\hline
\ttindex{CELL}&Initialise cell matching &\ref{S-MATACT} \\
\ttindex{MATCH}&Initialise insertion matching &\ref{S-MATACT} \\
\ttindex{WEIGHT}&Set matching weights &\ref{S-VARY} \\
\ttindex{CONSTRAINT}&Impose matching constraint &\ref{S-CONSTR} \\
\ttindex{COUPLE}&Impose periodicity between two points&\ref{S-CONSTR} \\
\ttindex{VARY}&Vary parameter &\ref{S-VARY} \\
\ttindex{FIX}&Fix parameter value &\ref{S-VARY} \\
\ttindex{RMATRIX}&Constraint on linear matrix &\ref{S-CONSTR} \\
\ttindex{TMATRIX}&Constraint on second-order terms &\ref{S-CONSTR} \\
\ttindex{LEVEL}&Set print level &\ref{S-LEVCMD} \\
\ttindex{LMDIF}&Minimisation by gradient method &\ref{S-MATMET} \\
\ttindex{MIGRAD}&Minimisation by gradient method &\ref{S-MATMET} \\
\ttindex{SIMPLEX}&Minimisation by simplex method &\ref{S-MATMET} \\
\ttindex{ENDMATCH}&Leave matching mode &\ref{S-MATACT} \\
\hline
\end{tabular}
\end{table}
 
Before a match operation a beam line must be selected
by means of a {\tt USE} command (see Section~\ref{S-USE}).
Matching of this line is then initiated by a
\ttindex{CELL} or \ttindex{MATCH} command.
From either of these commands to the corresponding \ttindex{ENDMATCH}
command MAD recognises the matching commands listed in
Table~\ref{T-MATCH}.
For a mathematical description of the minimisation procedures
see~\cite{B-JAM71}.
In particular one may do the following:
\begin{itemize}
\item Define parameter(s) to be varied,
\item Couple and/or set parameter values,
\item Define constraints,
\item Select desired printout detail,
\item Match by different methods.
\end{itemize}
The matching commands are described in detail below.
Some other commands can also be issued during matching.
 
\section{Activation and Deactivation of the Matching Module}
\label{S-MATACT}
Before matching a beam line must be selected by means
of a {\tt USE} command (see Section~\ref{S-USE}).
The type of match desired is then defined by entering
a \ttindex{CELL} or \ttindex{MATCH} command.
\subsection{Matching a Periodic Cell}
In the first mode, called cell matching,
a periodic cell is adjusted.
The periodicity is enforced exactly,
and constraints are fulfilled in the least squares sense.
Cell matching mode is initiated by the \ttindex{CELL} command:
\mybox{
CELL,DELTAP=real,ORBIT
}
It has two attributes:
\begin{mylist}
\ttitem{DELTAP}
The value of the momentum error
\(\Delta p/p_0 c\) for which the match should be performed
(default:~0).
\ttitem{ORBIT}
If this flag is true, the closed orbit is also matched.
\end{mylist}
Examples:
\myxmp{
! Match a simple periodic cell \\
USE,PERIOD=OCTANT,RANGE=CELL1 \\
CELL,ORBIT
! \\
! Match a symmetric and periodic cell with repetitions \\
USE,HALFCELL,SYMM,SUPER=5 \\
CELL
}

\subsection{Insertion Matching}
\index{insertion matching}
In the second mode, called insertion matching,
a beam line is matched with given initial values
for the optical functions.
Constraints may be imposed in other places,
i.e. intermediate or end values can be requested.
In this case the initial values are assumed as exact,
and constraints are fulfilled in the least squares sense.
The insertion matching mode is initiated by the \ttindex{MATCH} command.
In the simplest form the initial values for the optical
functions are taken from the periodic solution of another beam line.
In this case, all of
\((\beta_{x}, \beta_{y}, \alpha_{x}, \alpha_{y},
D_{x}, D_{y}, D_{p_{x}}, D_{p_{y}})\)
are transmitted to be used as initial values.
If there is a constraint on the orbit,
the values \((x, p_x, y, p_y)\) of the orbit are also transmitted.
Such a condition is entered in the form
\mybox{
MATCH,LINE=beam-line,MUX=real,MUY=real,DELTAP=real,ORBIT
}
The initial phases angles may be specified by:
\begin{mylist}
\ttitem{MUX}
The initial horizontal phase \(\mu_x\),
\ttitem{MUY}
The initial vertical phase \(\mu_y\).
\ttitem{DELTAP}
The value of the momentum error
\(\Delta E/p_0 c\) for which the match should be performed
(default:~0).
\ttitem{ORBIT}
If this flag is true, the closed orbit is also matched.
\end{mylist}
Example:
\myxmp{
xxxxxxxx\=xxxxxxxx\=\kill
CELL1:  \>LINE=(\ldots) \\
INSERT: \>LINE=(\ldots) \\
        \>USE,INSERT \\
        \>MATCH,LINE=CELL1,MUX=9.345,MUY=9.876,ORBIT
}
This matches the beam line {\tt INSERT}.
Initial conditions are given by the periodic solution for
the beam line {\tt CELL1}.
 
It is also possible to enter numerical initial values.
The \ttindex{MATCH} command then has the form
\mybox{
MATCH, \=BETX=real,ALFX=real,MUX=real,\& \\
       \>BETY=real,ALFY=real,MUY=real,\& \\
       \>X=real,PX=real,Y=real,PY=real,\& \\
       \>DX=real,DY=real,DPX=real,DPY=real,\& \\
       \>DELTAP=real,ORBIT
}
It accepts as attributes the linear lattice functions
listed in Section~\ref{S-LINLAT})
and the orbit coordinates listed in Section~\ref{S-CANON}.
Omitted initial values are assumed to be zero.
This implies that at least \ttindex{BETX} and \ttindex{BETY}
are required to obtain physically meaningful results.
Example:
\myxmp{
xxxxxxx\=\kill
USE,   \>INSERT \\
MATCH, \>BETX=1.6,BETY=0.1
}
This matches the beam line {\tt INSERT} with given initial values.
Values not specified are set to zero.
 
The end conditions of a previous beam line can finally be transmitted
via a \ttindex{SAVEBETA} command
(see also section \ref{TWISS}):
\myxmp{
USE,LINE1 \\
SAVEBETA,LABEL=XYZ,PLACE=\#E \\
TWISS \\
USE,LINE2 \\
MATCH,BETA0=XYZ
}
This example transmits all values (including phases) as initial
values.
The orbit is used only if the flag \ttindex{ORBIT} is entered.
Some values may also be changed explicitly:
\myxmp{
USE,LINE1 \\
SAVEBETA,LABEL=XYZ,PLACE=\#E \\
TWISS \\
USE,LINE2 \\
MATCH,BETA0=XYZ,MUX=1.234,MUY=4.321
}
will overwrite the phases taken from the previous line.
\subsection{End of Matching Run}
The \ttindex{ENDMATCH} command terminates the matching section
and deletes all tables related to a tracking run:
\mybox{
ENDMATCH
}
\section{Variable Parameters}
\label{S-VARY}
\index{variable}
A parameter to be varied is specified by the \ttindex{VARY} command:
\mybox{
VARY,NAME=variable,STEP=real,LOWER=real,UPPER=real
}
It has four attributes:
\begin{mylist}
\ttitem{NAME}
The name of the parameter or attribute to be varied
(see Section~\ref{S-VARAT}),
\ttitem{STEP}
\index{step size}
The approximate initial step size for varying the parameter.
If the step is not entered, MAD tries to find a reasonable step,
but this may not always work.
\ttitem{LOWER}
\index{matching!limits}
\index{limits!matching}
Lower limit for the parameter (optional),
\ttitem{UPPER}
Upper limit for the parameter (optional).
\end{mylist}
Examples:
\myxmp{
VARY,PAR1,STEP=1.0E-4       \=! vary global parameter PAR1 \\
VARY,QL11[K1],STEP=1.0E-6   \>! vary attribute K1 of the QL11 \\
VARY,Q15[K1],STEP=0.0001,LOWER=0.0,UPPER=0.08 ! vary with limits
}
If the upper limit is smaller than the lower limit,
the two limits are interchanged.
If the current value is outside the range defined by the limits,
it is brought back to range.
After a matching operation all varied attributes retain their last value.
They are never reset to an old value.
Parameters cannot be varied if they depend on other parameters.
Example:
\myxmp{
P1:=10.0-P2
}
Here {\tt P2} may be varied, but {\tt P1} may not.
The reason is that MAD is not able to compute {\tt P2}
in terms of {\tt P1}.

The command \ttindex{FIX} removes
\ttnindex{NAME}
a parameter or attribute from the table of variable parameters:
\mybox{
FIX,NAME=variable
}
Example:
\myxmp{
FIX,NAME=QF[K1]
}
 
\section{Constraints}
\label{S-CONSTR}
\subsection{Simple Constraints}
\index{constraint!simple}
Simple constraints are imposed by the \ttindex{CONSTRAINT} command.
It can take three forms. The first form is
\mybox{
CONSTRAINT,PLACE=range,LINE=beam-line,MUX=real,MUY=real
}
Here all of
\((\beta_{x}, \beta_{y}, \alpha_{x}, \alpha_{y},
D_{x}, D_{y}, D_{p_{x}}, D_{p_{y}})\)
are constrained in {\tt range} to the corresponding values
of {\tt beam-line}.
If any of the six components of the closed orbit of the beam line
is constrained, the command implies setting of the \ttindex{ORBIT} flag.
The phase advances \((\mu_{x},\mu_{y})\) may also be specified on the
constraint.
If the match succeeds,
the part of the beam line from its beginning to {\tt place} has
the specified phase advance.
The optical functions are such that {\tt beam-line}
could be inserted in {\tt place} without a mismatch.
If {\tt beam-line} was defined with formal arguments,
an actual argument list must be provided.
Normally {\tt place} refers to a single observation point
(see Section~\ref{S-OBSAT}).

The second form of the \ttindex{CONSTRAINT} command is
\mybox{
CONSTRAINT,PLACE=range,BETA0=beta0-name,MUX=real,MUY=real
}
Here all of
\((\beta_{x}, \beta_{y}, \alpha_{x}, \alpha_{y},
D_{x}, D_{y}, D_{p_{x}}, D_{p_{y}})\)
are constrained in {\tt range} to the corresponding values
of a pre-calculated \ttindex{BETA0} module (see~\ref{TWISS}).
If the \ttindex{ORBIT} flag is set, then the six components of the closed
orbit of the beam line are also constrained.
Note that the default weights for \ttindex{T} or \ttindex{PT} are zero.
The phases \((\mu_{x},\mu_{y})\) are {\em not} taken from the \ttindex{BETA0}
block.  If they are to be constrained, they must be explicitly
specified on the \ttindex{CONSTRAINT} command.  However, it is permitted to
specify the phase values from the \ttindex{BETA0} block:
\myxmp{
CONSTRAINT,PLACE=...,BETA0=DATA,MUX=DATA[MUX],MUY=DATA[MUY]
}
If the match succeeds,
the part of the beam line from its beginning to {\tt place} has
the specified phase advance.
Normally {\tt place} refers to a single observation point
(see Section~\ref{S-OBSAT}).

The third form
\mybox{
CONSTRAINT, \=PLACE=place,\& \\
            \>BETX=real,ALFX=real,MUX=real,\& \\
            \>BETY=real,ALFY=real,MUY=real,\& \\
            \>X=real,PX=real,Y=real,PY=real,T=real,PT=real,\& \\
            \>DX=real,DPX=real,DY=real,DPX=real
}
allows one to enter numerical values for any optical function(s)
in {\tt place}.
{\tt Place} may refer to a range,
denoted by two positions or by a beam line name.
The constraint then applies to all positions within the range.
The linear lattice functions listed in Section~\ref{S-LINLAT}
can all be constrained.
The matching weights can be set by the \ttindex{WEIGHT} command (see below).
Upper and lower limits can be requested by replacing the equals sign
by a less than sign (\(<\)) or by a greater than sign (\(>\)).
Both lower and upper limits may be requested in the same command.
If only lower and upper limits are used in a constraint,
the corresponding {\tt place} may be a range like e.g. a beam line.
Example:
\myxmp{
INSERT: \=LINE=(\ldots) \\
CELL1:  \>LINE=(\ldots) \\
        \>USE,INSERT \\
        \>MATCH,LINE=CELL1 \\
        \>CONSTRAINT,INSERT,DX<2.5,DX>0.0
}
Note that 

\subsection{Sub-Period Constraint}
\index{constraint!periodic}
\index{phase advance}
It is sometimes desired to make the linear lattice parameters
equal in two places without imposing their actual values.
This is requested by the \ttindex{COUPLE} command
\ttnindex{RANGE}\ttnindex{MUX}\ttnindex{MUY}
\index{phase}
\mybox{
COUPLE,RANGE=range,MUX=real,MUY=real
}
where {\tt range} refers to an observation range
(see Section~\ref{S-OBSAT}).
MAD tries to adjust the matched beam line such that the functions
\((\beta_{x}, \beta_{y}, \alpha_{x}, \alpha_{y},
D_{x}, D_{y}, D_{p_{x}}, D_{p_{y}})\)
are equal in the two end points of the range.
The orbit cannot be constrained by this command.
Optionally the phase advances over the {\tt range}
may be specified by means of the attributes \ttindex{MUX} and/or \ttindex{MUY}.
The matching weights are taken from the latest \ttindex{WEIGHT} command
(see below).
Example:
\myxmp{
xxxxxxxx\=\kill
M:      \>MARKER \\
INSERT: \>LINE=(\ldots,M,\ldots,M,\ldots) \\
        \>USE,INSERT \\
        \>MATCH,BETX=2,BETY=1 \\
        \>COUPLE,M[1]/M[2],MUX=0.5
}
\subsection{Constraints on Transfer map}
\label{S-RMAT}
\index{transfer map constraints}
\index{constraint!transfer map}
\ttnindex{RMATRIX}
\ttnindex{TMATRIX}
Often the constraints are desired on the transfer map
rather than on the lattice functions.
The two commands
\mybox{
RMATRIX,RANGE=range \{,RM(i,k)=real\} \{,WEIGHT(i,k)=real\} \\
TMATRIX,RANGE=range \{,TM(i,k,l)=real\} \{,WEIGHT(i,k,l)=real\}
}
are provided for this purpose.
They have the following attributes:
\begin{mylist}
\ttitem{RANGE}
Denotes the beginning and end of the range over which the transfer
map is to be fitted.
\ttitem{RM(i,k)}
A real array of dimension {\tt (6,6)}.
As many elements of the linear transfer matrix can be constrained.
\ttitem{TM(i,k,l)}
A real array of dimension {\tt (6,6,6)}.
As many elements of the linear transfer matrix can be constrained.
\ttitem{WEIGHT}
A real array of dimension {\tt (6,6)} or {\tt (6,6,6)} respectively.
If given, this specifies the matching weights for the corresponding
matrix elements (default:~1).
\end{mylist}
\subsection{Matching Weights}
\index{matching!weight}\index{weight (matching)}
The matching procedures try to fulfil the constraints
in a least square sense.
A penalty function is constructed which is the sum of the
squares of all errors,
each multiplied by the specified weight.
The larger the weight, the more important a constraint becomes.
The weights are taken from a table of current values.
These are initially set to default values,
and may be reset at any time to different values.
Values set in this way remain valid until changed again.
The \ttindex{WEIGHT} command
\mybox{
WEIGHT, \=BETX=real,ALFX=real,MUX=real,\& \\
        \>BETY=real,ALFY=real,MUY=real,\& \\
        \>X=real,PX=real,Y=real,PY=real,T=real,PT=real,\& \\
        \>DX=real,DPX=real,DY=real,DPY=real
}
\ttnindex{BETX}
\ttnindex{ALFX}
\ttnindex{MUX}
\ttnindex{BETY}
\ttnindex{ALFY}
\ttnindex{MUY}
\ttnindex{X}
\ttnindex{PX}
\ttnindex{Y}
\ttnindex{PY}
\ttnindex{T}
\ttnindex{PT}
\ttnindex{DX}
\ttnindex{DPX}
\ttnindex{DY}
\ttnindex{DPY}
changes the weights for subsequent constraints.
The weights are entered with the same name as the linear lattice
function to which the weight applies,
as listed in Section~\ref{S-VARIA}.
The default weights are listed in Table~\ref{T-MWEI}.
Note that matching \ttindex{T} or \ttindex{PT} always requires a
\ttindex{WEIGHT} command,
since these quantities have zero default weights.
Frequently the matching weights serve to select a restricted
set of functions to be matched in a {\tt LINE=line} constraint.
One may match the dispersion at a given place to the dispersion of
a line as in the following example:
\myxmp{
CONSTRAINT, \=\kill
WEIGHT,     \>BETX=0.0,BETX=0.0,ALFX=0.0,ALFY=0.0,MUX=0.0,MUY=0.0 \\
CONSTRAINT, \>\#E,LINE=CELL
}
\begin{table}[ht]
\caption{Default Matching Weights}
\vspace{1ex}
\label{T-MWEI}
\centering
\begin{tabular}{|l|c|l|c|l|c|l|c|}
\hline
name &weight &name &weight &name &weight &name &weight \\
\hline
\ttindex{BETX} &1.0 &\ttindex{ALFX} &10.0 &\ttindex{MUX} &10.0 & & \\
\ttindex{BETY} &1.0 &\ttindex{ALFY} &10.0 &\ttindex{MUY} &10.0 & & \\
\ttindex{X}    &10.0&\ttindex{PX}   &100.0&\ttindex{Y}   &10.0 &
  \ttindex{PY} &100.0 \\
\ttindex{T}    &0.0 &\ttindex{PT}   &0.0  & & & & \\
\ttindex{DX}   &10.0&\ttindex{DPX}  &100.0&\ttindex{DY}  &10.0 
  &\ttindex{DPY} &100.0 \\
\hline
\end{tabular}
\end{table}
 
\section{Matching Output Level}
\label{S-LEVCMD}
\index{matching!output}
The \ttindex{LEVEL} command sets the output level for matching:
\mybox{
LEVEL,LEVEL=integer
}
Recognised level numbers are:
\begin{mylist}
\item[0]
Minimum printout: Messages and final values only,
\item[1]
Normal printout: Messages, initial and final values,
\item[2]
Like {\tt LEVEL=1}, plus every tenth new minimum found,
\item[3]
Like {\tt LEVEL=2}, plus every new minimum found,
\item[4]
Like {\tt LEVEL=3}, plus eigenvalues of covariance matrix
(\ttindex{MIGRAD} method only).
\end{mylist}
Example:
\myxmp{
LEVEL,2
}
 
\section{Matching Methods}
\label{S-MATMET}
\index{matching!methods}

\subsection{LMDIF, Gradient Minimisation}
\index{gradient minimisation}
The \ttindex{LMDIF} command minimises the sum of squares of the constraint
functions using their numerical derivatives:
\mybox{
LMDIF,CALLS=integer,TOLERANCE=real
}
It is the fastest minimisation method available in MAD.
The command has two attributes:
\begin{mylist}
\ttitem{CALLS}
The maximum number of calls to the penalty function (default:~1000).
\ttitem{TOLERANCE}
The desired tolerance for the minimum (default:~\(10^{-6}\)).
\end{mylist}
Example:
\myxmp{
LMDIF,CALLS=2000,TOLERANCE=1.0E-8
}

\subsection{MIGRAD, Gradient Minimisation}
\index{gradient minimisation}
The \ttindex{MIGRAD} command minimises the penalty
function using its numerical derivatives of the sum of squares:
\mybox{
MIGRAD,CALLS=integer,TOLERANCE=real,STRATEGY=1
}
The command has two attributes:
\begin{mylist}
\ttitem{CALLS}
The maximum number of calls to the penalty function (default:~1000).
\ttitem{TOLERANCE}
The desired tolerance for the minimum (default:~\(10^{-6}\)).
\ttitem{STRATEGY}
A code for the strategy to be used (default:~1).
The user is referred to the MINUIT manual for explanations~\cite{B-JAM71}.
\end{mylist}
Example:
\myxmp{
MIGRAD,CALLS=2000,TOLERANCE=1.0E-8
}

\subsection{SIMPLEX, Minimisation by Simplex Method}
The \ttindex{SIMPLEX} command minimises the penalty
function by the simplex method:
\mybox{
SIMPLEX,CALLS=integer,TOLERANCE=real
}
Details are given in the description of the MINUIT program~\cite{B-JAM71}.
The command has two attributes:
\begin{mylist}
\ttitem{CALLS}
The maximum number of calls to the penalty function (default:~1000).
\ttitem{TOLERANCE}
The desired tolerance for the minimum (default:~\(10^{-6}\)).
\end{mylist}
Example:
\myxmp{
SIMPLEX,CALLS=2000,TOLERANCE=1.0E-8
}
\index{matching!methods}
 
\section{Matching Examples}
\index{examples!matching|(}
\label{S-MATEX}

\subsection{Simple Periodic Beam Line}
Match a simple cell with given phase advances:
\myxmp{
xxxxxxxx\=\kill
QF:     \>QUADRUPOLE,\ldots \\
QD:     \>QUADRUPOLE,\ldots \\
CELL1:  \>LINE=(\ldots,QF,\ldots,QD,\ldots) \\
        \>USE,CELL1 \\
        \>CELL \\
        \>VARY,NAME=QD[K1],STEP=0.01 \\
        \>VARY,NAME=QF[K1],STEP=0.01 \\
        \>CONSTRAINT,PLACE=\#E,MUX=0.25,MUY=1/6 \\
        \>MIGRAD,CALLS=2000 \\
        \>ENDMATCH
}
\subsection{Insertion Matching}
Match an insertion {\tt INSERT} to go between
two different cells {\tt CELL1} and {\tt CELL2}:
\myxmp{
xxxxxxxx\=\kill
CELL1:  \>LINE=(\ldots) \\
CELL2:  \>LINE=(\ldots) \\
INSERT: \>LINE=(\ldots) \\
        \>USE,INSERT \\
        \>MATCH,LINE=CELL1 \\
        \>VARY,\ldots \\
        \>CONSTRAINT,PLACE=\#E,LINE=CELL2,MUX=\ldots,MUY=\ldots \\
        \>SIMPLEX \\
        \>MIGRAD \\
        \>ENDMATCH
}
Adjust the beam line {\tt INSERT} such that it has specified values
of the optical functions at its beginning and matches the line
{\tt CELL7} at its end:
\myxmp{
xxxxxxxx\=\kill
CELL7:  \>LINE=(\ldots) \\
INSERT: \>LINE=(\ldots) \\
        \>USE,INSERT \\
        \>MATCH,BETX=1.6,BETY=0.1 \\
        \>VARY,\ldots \\
        \>CONSTRAINT,PLACE=\#E,LINE=CELL7 \\
        \>SIMPLEX \\
        \>MIGRAD \\
        \>ENDMATCH
}
\subsection{A Matching Example for LEP, Version 11}
Here the end values are required to be exact.
This may be achieved by inverting the beam line:
\myxmp{
xxxxxxxx\=\kill
! Define beam lines \\
CELL:   \>LINE=(QDH,L23,SD,L22,B6,L21,QF,L23,SF,L22,B6,L24,QDH) \\
MARK:   \>MARKER \\
DISS:   \>LINE=(QS11,L25,BW,L24,QS12,L25,B4,L24,QS13,L25,B4,\& \\
        \>L24,QS14,L25,B4,L31,QS15,L25,B4,L32,SF2,L23,QS16) \\
RFS:    \>LINE=(L5,QS5,L5,QS6,L5,2*(QS7H,MARK,QS7H,L5,QS8,L5)) \\
LOBS:   \>LINE=(L1,QS1,L2,QS2,L3,QS3,L4,QS4) \\
INSS:   \>LINE=(LOBS,RFS,DISS,L21,B6,L22,QDH) \\
        \>USE,(-INSS) \\
! \\
! Call matching module and define initial values \\
MATCH,LINE=CELL \\
! \\
! Variable parameters \\
xxx\=\kill
   \>VARY,QS1[K1],STEP=0.01,LOWER=-0.06,UPPER=0.06 \\
   \>VARY,QS2[K1],STEP=0.01,LOWER=-0.06,UPPER=0.06 \\
   \>VARY,QS3[K1],STEP=0.01,LOWER=-0.06,UPPER=0.06 \\
   \>VARY,QS4[K1],STEP=0.001,LOWER=-0.06,UPPER=0.06 \\
   \>VARY,QS5[K1],STEP=0.01,LOWER=-0.035,UPPER=0.035 \\
   \>VARY,QS6[K1],STEP=0.01,LOWER=-0.035,UPPER=0.035 \\
   \>VARY,QS7[K1],STEP=0.01,LOWER=-0.035,UPPER=0.035 \\
   \>VARY,QS8[K1],STEP=0.01,LOWER=-0.035,UPPER=0.035 \\
   \>VARY,QS11[K1],STEP=0.01,LOWER=-0.035,UPPER=0.035 \\
   \>VARY,QS12[K1],STEP=0.01,LOWER=-0.035,UPPER=0.035 \\
   \>VARY,QS13[K1],STEP=0.01,LOWER=-0.035,UPPER=0.035 \\
   \>VARY,QS14[K1],STEP=0.01,LOWER=-0.035,UPPER=0.035 \\
   \>VARY,QS15[K1],STEP=0.01,LOWER=-0.035,UPPER=0.035 \\
   \>VARY,QS16[K1],STEP=0.01,LOWER=-0.035,UPPER=0.035 \\
! \\
! Constraints at the interaction point \\
   \>WEIGHT,BETX=10.0,BETY=50.0,ALFX=10.0,ALFY=10.0, \\
   \>MUX=10.0,MUY=10.0,DX=10.0,DPX=100.0 \\
   \>CONSTRAINT,\#E,BETX=1.6,BETY=0.1,ALFX=0.0,ALFY=0.0, \\
   \>MUX=1.771875,MUY=2.0125,DX=0.0,DPX=0.0 \\
! \\
! Constraints at two intermediate points (MARK occurs twice) \\
   \>CONSTRAINT,MARK,ALFX=0.0,ALFY=0.0,DX=0.0,DPX=0.0 \\
   \>COUPLE,MARK[1]/MARK[2] \\
! \\
! Constraint for limits of dispersion in dispersion suppressor \\
   \>CONSTRAINT,DISS,DX>0,DX<1.25 \\
! \\
! Perform match \\
   \>SIMPLEX,CALLS=1000 \\
   \>MIGRAD,CALLS=24000,TOLERANCE=1.0E-9 \\
! \\
! End of matching procedure \\
ENDMATCH
}
 
\subsection{Examples for Complex Matching Constraints}
\index{examples!constraint}
\index{examples!transfer map constraint}
\index{examples!spin matching}
All initial conditions and constraint values can depend on variable
parameters.
This feature permits to set up very complex matching conditions.
The first example shows how an insertion can be designed which exchanges
the values of the \(\beta\)-functions for the two transverse planes
in two places {\tt M1} and {\tt M2}:
\myxmp{
xxxxxxxx\=\kill \\
M1:     \>MARKER \\
M2:     \>MARKER \\
INSERT: \>LINE=(\ldots,M1,\ldots,M2,\ldots) \\
        \>CONSTRAINT,M1,BETX=BETA1,ALFX=ALFA1,BETY=BETA2,ALFY=ALFA2 \\
        \>CONSTRAINT,M2,BETX=BETA2,ALFX=ALFA2,BETY=BETA1,ALFY=ALFA1 \\
        \>VARY,BETA1,STEP=0.01 \\
        \>VARY,BETA2,STEP=0.01 \\
        \>VARY,ALFA1,STEP=0.01 \\
        \>VARY,ALFA2,STEP=0.01 \\
}
The trick is to match the lattice functions in {\tt M1} to the four
variable parameters
\myxmp{
BETA1,ALFA1,BETA2,ALFA2
}
and in {\tt M2} to the same parameters in a different order.
 
The second example is taken from a spin matching problem.
\myxmp{
! Match towards interaction point \\
MULT:=(1+SINTHETA)/SINTHETA ! A constant multiplication factor. \\
MATCH,LINE=SRF3M \\
 \\
! Lattice functions imposed at interaction point \\
WEIGHT,BETX=1,BETY=10,DY=30,DPY=300 \\
CONSTRAINT,LORIPS,BETX=1.75,BETY=0.07,ALFX=0,ALFY=0,DY=0 \\
 \\
! ***** Spin matching constraints ***** \\
RMATRIX,DBB[1]/LORIPS,RM(3,4)=0,RM(2,1)=AUX \\
VARY,AUX,STEP=0.01 \\
RMATRIX,B2UP[1]/LORIPS,RM(3,4)=0,RM(2,1)=MULT*AUX \\
 \\
! Quadrupoles to be varied \\
VARY,KQS0MP,STEP=0.01,LOWER=-0.2,UPPER=0.2 \\
VARY,KQS1MP,STEP=0.01,LOWER=-0.1,UPPER=0.1 \\
VARY,KQS3MP,STEP=0.01,LOWER=-0.1,UPPER=0.1 \\
VARY,KQS4MP,STEP=0.01,LOWER=-0.1,UPPER=0.1 \\
VARY,KQS5AP,STEP=0.01,LOWER=-0.1,UPPER=0.1 \\
VARY,KQS6AP,STEP=0.01,LOWER=-0.1,UPPER=0.1 \\
VARY,KQS5BP,STEP=0.01,LOWER=-0.1,UPPER=0.1 \\
VARY,KQS6BP,STEP=0.01,LOWER=-0.1,UPPER=0.1 \\
 \\
! Initiate matching \\
SIMPLEX,CALLS=1000,TOLE=1E-3 \\
LMDIF,CALLS=6000,TOLE=1E-12 \\
ENDMATCH
}
Here the interesting point lies in the spin matching constraints.
The conditions {\tt RM(3,4)=0} on the two {\tt RMATRIX} commands
ensure the proper phase advances for the vertical phase
(0.5 and 1.5 respectively).
The conditions on {\tt RM(2,1)} enforce the condition
\[
A_{21} = \frac{(1 + \sin \theta)}{\sin \theta} C_{21}.
\]
where \(A\) and \(C\) are the transfer matrices from the centre of either
vertical bend to the interaction point.
Since a direct coupling between the values of the transfer matrix
in two different points is not possible,
a freely variable parameter {\tt AUX} is introduced.
The value of \(C_{21}\) is matched to {\tt AUX},
while the value of \(A_{21}\) is matched to an expression in {\tt AUX}.
\index{examples!matching|)}\index{matching|)}

\chapter{Tracking Module}
\label{S-TRACK}
\index{tracking|(}
 
\begin{table}[ht]
\caption{Tracking Commands}
\vspace{1ex}
\label{T-TRACK}
\centering
\begin{tabular}{|l|p{0.6\textwidth}|l|}
\hline
Name     &Function                          &Section \\
\hline
\ttindex{NOISE}   &Define noise on parameters  &\ref{S-NOISE} \\
\ttindex{OBSERVE} &Define observation point    &\ref{S-OBSERVE} \\
\ttindex{START}   &Define initial coordinates  &\ref{S-START} \\
\ttindex{RUN}     &Start tracking              &\ref{S-RUN} \\
\ttindex{TSAVE}   &Save particle positions     &\ref{S-TSAVE} \\
\ttindex{ENDTRACK}&Terminate tracking mode     &\ref{S-TRAACT} \\
\hline
\end{tabular}
\end{table}
 
Before starting to track, the working beam line must be
selected by means of a \ttindex{USE} command (see Section~\ref{S-USE}).
The energy and emittances should be defined by a \ttindex{BEAM}
command (see Section~\ref{S-BEAM}).
Tracking is then initiated by the \ttindex{TRACK} command.
From this command to the corresponding \ttindex{ENDTRACK} command
MAD accepts the tracking statements listed in
Table~\ref{T-TRACK}.
Tracking is done in parallel i.e. the coordinates of all particles
are transformed at each beam element as it is reached.
 
\section{Activation and Deactivation of the Tracking Module}
\label{S-TRAACT}
Tracking through a beam line is initiated by the \ttindex{TRACK} command.
It has the form
\mybox{
TRACK [,ONEPASS] [,DAMP] [,QUANTUM]
}
The emittances and \(\sigma\)'s are evaluated from the latest
\ttindex{BEAM} command.
A {\tt BEAM} command should not occur while the tracking module is
active.
The \ttindex{TRACK} command has three attributes:

\begin{mylist}
\ttitem{ONEPASS}
This flag tells MAD not to compute the normalisation transformations
and to assume that the machine is stable.
\end{mylist}

The two other flags have only effect if synchrotron radiation is
switched on on the \ttindex{BEAM} command:

\begin{mylist}
\ttitem{DAMP}
This flag causes MAD to consider the systematic component
(bias) of energy loss by synchrotron radiation.
\ttitem{QUANTUM}
If this flag is present together with {\tt DAMP},
it causes MAD to simulate the quantum effects of synchrotron radiation.
\end{mylist}

The former attributes {\tt RFCAVITY} and {\tt DELTAP} are ignored in
this version of MAD.
The \ttindex{TRACK} command must be preceded by a \ttindex{EMIT}
command, which allows to adjust the average momentum error.
The RF~frequencies are set by {\tt TRACK} according to this momentum
error set in {\tt EMIT}.

The synchrotron tune \(Q_{s}\) computed must be non-zero,
\ttnindex{QS}
\index{synchrotron tune}
if \(E_{t}\) or \(\sigma_{e}\) is used on the \ttindex{BEAM} command.
Example for a {\tt TRACK} command:
\myxmp{
BEAM, \=PARTICLE=ELECTRON,ENERGY=50,RADIATE,\& \\
      \>EX=0.0645E-6,EY=0.03225E-6,SIGE=1.25E-3 \\
EMIT,DELTAP=0.01
TRACK,DAMP,QUANTUM
}
The \ttindex{ENDTRACK} command terminates the tracking section:
\mybox{
ENDTRACK
}
It deletes all information relevant to the current tracking run.
 
\section{NOISE Statement}
\label{S-NOISE}
\index{noise}
\index{power supply noise}
One may define noise to be applied to magnet excitations
with the \ttindex{NOISE} statement
\mybox{
NOISE,VARIABLE=variable,AMPLITUDE(i)=real,FREQUENCY=real,PHASE=real
}
The {\tt NOISE} statement must be entered after the {\tt TRACK},
but before the {\tt RUN} command.
It has the effect to apply several sinusoidal variations to the
specified parameter.
The command has four attributes.
\begin{mylist}
\ttitem{VARIABLE}
The variable to be affected.
\ttitem{AMPLITUDE(i)}
The value of the \(i^{th}\) noise amplitude \(A_i\).
\ttitem{FREQUENCY(i)}
The value of the \(i^{th}\) noise frequency \(f_i\).
\ttitem{PHASE(i)}
The value of the \(i^{th}\) noise phase \(\phi_i\).
\end{mylist}
Before each turn is tracked, the noise is re-evaluated as
\[
\Delta A = \sum_{i=1}^N A_i \cos ( 2\pi (f_i t + \phi_i)).
\]
where \(t\) is the real time.
 
\section{OBSERVE Statement}
\label{S-OBSERVE}
The statement
\ttindex{OBSERVE}
\mybox{
OBSERVE,PLACE=place,TABLE=table
}
defines a new observation point where a track table is to be
generated.
It has the attributes:
\begin{mylist}
\ttitem{PLACE}
The observation point.
Only the first occurrence in the machine is used.
\ttitem{TABLE}
The name to be given to the table.
\end{mylist}
The {\tt OBSERVE} statement must be entered after the {\tt TRACK},
but before the {\tt RUN} command.

\section{START Statement}
\label{S-START}
\index{initial conditions}
\index{coordinates!initial}
The \ttindex{START} command defines the initial coordinates of
the particles to be tracked.
There may be many {\tt START} statements, one for each particle.
Particles are always started with coordinates relative
to the computed closed orbit for the defined energy error.
The command format is:
\mybox{
START, \=X=real,PX=real,Y=real,PY=real,T=real,DELTAP=real,\& \\
       \>FX=real,PHIX=real,FY=real,PHIY=real,FT=real,PHIT=real
}
The {\tt START} statement must be entered after the {\tt TRACK},
but before the {\tt RUN} command.

For each canonical pair two specifications are possible.
First, they may be specified as displacements with respect to the
closed orbit:
\ttindex{X}, \ttindex{PX}, \ttindex{Y}, \ttindex{PY}, \ttindex{T},
\ttindex{DELTAP}, as defined in Section~\ref{S-CANON}.
In this case the initial conditions are:
\[
\eqarray{
   x&=&\hbox{\tt X},\qquad p_{x}&=&\hbox{\tt PX}, \\
   y&=&\hbox{\tt Y},       p_{y}&=&\hbox{\tt PY}, \\
   \Delta t&=&\hbox{\tt T}, \qquad \Delta E / cp_{0}&=& \hbox{\tt DELTAP}.
}
\]
It is also possible to enter rotations in normalised phase space.
Initial conditions \(Z\)~in normalised phase space are related to the closed
orbit and normalised via the eigenvectors:
\[
\eqarray{
Z = Z_{co}&+& \sqrt{E_x} \hbox{\tt FX}
        (\Re V_k \cos \hbox{\tt PHIX} + \Im V_k \sin \hbox{\tt PHIX}) \\
          &+& \sqrt{E_y} \hbox{\tt FY}
        (\Re V_k \cos \hbox{\tt PHIY} + \Im V_k \sin \hbox{\tt PHIY}) \\
          &+& \sqrt{E_t} \hbox{\tt FT}
        (\Re V_k \cos \hbox{\tt PHIT} + \Im V_k \sin \hbox{\tt PHIT})
}
\]
where \(Z_{co}\)~is the closed orbit vector,
\(\Re V_k\)~and \(\Im V_l\)~are the real and imaginary parts of the
\(k^{th}\) eigenvectors.
The other variables have the meaning:
\begin{mylist}
\ttitem{FX}
The normalised (betatron) amplitude for mode~1,
\ttitem{PHIX}
The (betatron) phase \(\hbox{\tt PHIX}=\phi_{x}/2\pi\) for mode~1,
\ttitem{FY}
The normalised (betatron) amplitude for mode~2,
\ttitem{PHIY}
The (betatron) phase \(\hbox{\tt PHIY}=\phi_{y}/2\pi\) for mode~2,
\ttitem{FT}
The normalised (synchrotron) amplitude for mode~3,
\ttitem{PHIT}
The (synchrotron) phase \(\hbox{\tt PHIT}=\phi_{t}/2\pi\)) for mode~3.
\end{mylist}
The eigenvectors are computed in the \ttindex{TRACK} command.
The emittances are calculated in the latest \ttindex{EMIT} command
preceding it,
or taken from the latest \ttindex{BEAM} command,
whichever comes last.

\section{TSAVE Statement}
\label{S-TSAVE}
The \ttindex{TSAVE} command
\mybox{
TSAVE,FILENAME=string
}
saves the latest particle positions on the file named by {\tt string}.
These will normally be the positions of surviving particles after the
last \ttindex{RUN} command.
If no {\tt RUN} command has been seen yet,
the positions are taken from any \ttindex{START} commands seen.
The positions are written as {\tt START} commands,
which may be read by a subsequent tracking run.
 
\section{RUN Statement}
\label{S-RUN}
This command starts or continues the actual tracking:
\mybox{
RUN,\=METHOD=TRANSPORT|LIE3|LIE4 [,TABLE=name]\& \\
    \>TURNS=integer,FPRINT=integer,FFILE=integer
}
The \ttindex{RUN} command initialises tracking and uses the values in the last
particle bank for initial conditions.
\index{particle!bank}
It optionally builds a table of particle positions, suited for plotting
or further analysis;
it may also write a portable binary file.
Note that the \ttindex{CONTINUE} command has been deleted from
Version~8.2.
{\tt RUN} has the attributes:
\begin{mylist}
\ttitem{METHOD}
\index{tracking!method}\index{method (tracking)}
The name of the tracking method to be used.
It may be one of:
\begin{mylist}
\ttitem{TRANSPORT}
The TRANSPORT method~\cite{B-BRO80},
using transfer matrices of order two (default),
\ttitem{LIE3}
The Lie-algebra method~\cite{B-DRA81}
to order 2 in the canonical variables,
\ttitem{LIE4}
The Lie-algebra method~\cite{B-DRA81}
to order 3 in the canonical variables.
\end{mylist}
\ttitem{TABLE}
\index{track table}\index{table!track}
The name to be given to the tracking table.
If a table with the same name already exists, it is discarded.
If {\tt TABLE} is omitted, no table is built.
The table is not saved automatically;
it can be saved on a file by the {\tt ARCHIVE} command.
\ttitem{TURNS}
The number of turns (integer) to be tracked (default:~1).
\ttitem{FPRINT}
\index{print frequency}
The print frequency, an integer (default:~no printing).
Printing always occurs before the first and after the last turn.
If the value of {\tt FPRINT} is greater than zero,
printing also occurs after every {\tt FPRINT} turns.
This print-out includes normalised values, invariants, and phases
(for their definitions see~\ref{S-START}).
Printing in other positions may be requested by a {\tt SELECT,TRACK}
command.
\ttitem{FFILE}
\index{output frequency}
The output frequency for the binary file, an integer (default:~no printing).
If the value of {\tt FFILE} is greater than zero,
output occurs after every {\tt FFILE} turns.
The binary file is written in EPIO format, and has not been frozen yet.
The output will be used by an off-line analysis program.
\end{mylist}
Example:
\myxmp{
RUN,TURNS=1000,FPRINT=10 \\
}
This {\tt RUN} command tracks 1000~ turns and prints every \(10^{th}\) turn.
 
\section{A Tracking Example}
\index{examples!track}
\myxmp{
L: LINE=(OCT,-OCT) \\
USE,L \\
! Misalignments and closed orbit correction may be done here. \\
BEAM,EX=0.1E-6,EY=0.05E-6,ET=1.0E-3,ENERGY=60.0 \\
TRACK \\
xxx\=\kill
   \>START,X=0.001,Y=0.001,DELTAP=0.001  \=! absolute X and Y \\
   \>START,FX=10.0,FY=10.0,DELTAP=-0.001 \>! particle at 10 sigma \\
   \>RUN,TURNS=1024,FPRINT=1 \\
   \>PLOT,HAXIS=X,VAXIS=PX,MULTIPLE \\
ENDTRACK
}
Plotting can be requested as described in section~\ref{S-TRTAB}.
The position of the {\tt PLOT} command is irrelevant,
as long as it occurs after the {\tt RUN} command,
and before any command which overwrites the track table generated.
\index{tracking|)}
 
\chapter{Subroutines and Procedures}
\label{S-SUBROUT}
\index{subroutines|(}
\index{procedure}
 
\begin{table}[ht]
\label{T-SUBROUT}
\caption{Subroutine Commands}
\vspace{1ex}
\centering
\begin{tabular}{|l|p{0.6\textwidth}|l|}
\hline
Name &Meaning &Section \\
\hline
\ttindex{SUBROUTINE}&Begin subroutine definition &\ref{S-SUB} \\
\ttindex{ENDSUBROUTINE}&End subroutine definition &\ref{S-SUB} \\
\ttindex{CALLSUBROUTINE}&Call predefined subroutine &\ref{S-CALLSUB} \\
\ttindex{DO}&Repeated execution of commands &\ref{S-DO} \\
\ttindex{ENDDO}&End of {\tt DO} group &\ref{S-DO} \\
\hline
\end{tabular}
\end{table}
 
The input language provides features for gathering
strings of statements,
which may be treated as a simple command called a {\em subroutine}.
There are commands to define a subroutine in terms of other commands,
and to execute subroutines.
 
These language features allow the user to assemble
a personal library of commonly used commands.
Individual commands and subroutines may be called from this
library whenever they are wanted,
and simple labels may be substituted for relatively complicated
subroutines or commands.
The personal statement library can be appended to the MAD dictionary,
which makes the statements available automatically when the program
is started.
Alternatively it can be placed in a user file and accessed on demand
via the {\tt CALL} and {\tt RETURN} commands.
Good candidates for a subroutine library are command sequences
for the {\tt HARMON}, Matching, or Tracking Module.
 
\section{SUBROUTINE and ENDSUBROUTINE}
\label{S-SUB}
The \ttindex{SUBROUTINE} and \ttindex{ENDSUBROUTINE}
statements delimit a group of related statements which together
form the body of the subroutine.
A subroutine definition, followed by a call, has the format
\mybox{
xxxxxxxx\=xxxxxxxxxxxxxxxx\=\kill
label:  \>SUBROUTINE      \>! begin of subroutine "label" \\
        \>\ldots          \>! included statements \\
        \>ENDSUBROUTINE   \>! end of subroutine "label" \\
        \>\ldots          \>! other commands \\
        \>label           \>! call to subroutine "label"
}
The subroutine is given the name {\tt label}.
The statements included are checked and stored in memory,
but not executed until the subroutine is called.
Example:
\myxmp{
S: \=SUBROUTINE \=! begin of definition for S \\
   \>USE,SSC \\
   \>SURVEY \\
   \>TWISS \\
ENDSUBROUTINE \>\>! end of definition for S \\
\ldots \\
S             \>\>! here S is executed the first time
}
The size of a subroutine is limited only by the available
memory space.
A labelled command within a subroutine may be called by its label from
anywhere in the program.
A subroutine cannot contain another \ttindex{SUBROUTINE},
\ttindex{DO} or \ttindex{ENDDO} statement.
However, within a subroutine,
nested calls to other subroutines are permitted.
Also, references to \ttindex{DO} statements which have already been
accepted by the program are permitted.
 
\section{CALLSUBROUTINE}
\label{S-CALLSUB}
The \ttindex{CALLSUBROUTINE} command causes execution of a group
of commands previously assembled as a subroutine or procedure.
It is included in the language for formal purposes only,
to make the call explicit.
\mybox{
CALLSUBROUTINE,label
}
A statement consisting of {\tt label} only is
equivalent to {\tt CALLSUBROUTINE,label}.
The following are two equivalent calls to a subroutine {\tt S}:
\myxmp{
S \\
CALLSUBROUTINE,S
}
 
\section{DO and ENDDO}
\label{S-DO}
The \ttindex{DO} statement allows grouping of one or more commands
and offers an option to repeat their execution.
Both subroutines and commands may be included.
The list of commands follows the {\tt DO} statement,
and ends with an \ttindex{ENDDO}:
\myxmp{
DO,TIMES = integer] \\
xxx\=\kill
   \>\{command\} \\
ENDDO
}
The commands \ttindex{SUBROUTINE}, \ttindex{ENDSUBROUTINE},
or \ttindex{DO} must not be placed between \ttindex{DO} and \ttindex{ENDDO}.
Example:
\myxmp{
QUAD3:QUAD,L=1.5,K1=0.01 \\
DO,TIMES=2 \\
xxx\=\kill
   \>TWISS,DELTAP=-0.01:+0.01:0.002 \\
   \>SET, QUAD3[K1], QUAD3[K1]+0.001 \\
ENDDO
}
This example shows how a {\tt SET} statement may change
parameters of the computations after each iteration.
The length of a {\tt DO} list is not limited.
\index{subroutines|)}

\chapter{Known Defects of Version 8}
\index{defects}
 
\section{Definitions}
MAD uses full \(6 \times 6\)~matrices to allow coupling effects to
be treated, and the canonical variable set
\(x, p_x/p_0\), \(y, p_y/p_0\), \(-c \Delta t, \Delta E/p_0 c\),
as opposed to other programs most of which use the set
\(x, x'\), \(y, y'\), \(- \Delta s, \delta\).
Like the program MARYLIE~\cite{B-DRA81},
MAD uses the relative energy error
\(\Delta E / p_0 c\)
which is related to the relative momentum error \(\delta\) by
\(\Delta E / p_0 c = \beta \Delta p / p_0 = \beta \delta\).
 
As from Version~8.13, MAD uses an additional {\em constant} momentum
error \(\delta_s = \Delta p/p_0\) in all optical calculations.
The transfer maps contain the {\em exact} dependence upon this value;
therefore the tunes for large deviations can be computed with high
accuracy as opposed to previous versions.

The choice of canonical variables in MAD still
leads to slightly different definitions of the lattice functions.
In MAD the Courant-Snyder invariants~\cite{B-COU58}
take the form
\(W_x = \gamma_x x^2 - 2 \alpha_x x p_x + \beta_x p_x^2\).
Comparison to the original form
\(W_x = \gamma_x x^2 - 2 \alpha_x x x' + \beta_x x'^2\)
shows that the orbit functions cannot be the same.
A more detailed analysis, using \(x' = p_x / (1 + \delta)\),
shows that all formulas can be made consistent
by defining the MAD orbit functions as
\[
\beta_{xM} = \beta_x (1 + \delta) , \quad
\alpha_{xM} = \alpha_x , \quad
\gamma_{xM} = \gamma_x / (1 + \delta).
\]
For constant \(\delta_s\) along the beam line and \(\delta = 0\),
the lattice functions are the same.
In a machine where \(\delta\) varies along the circumference, e.g. in a
linear accelerator or in an e{}\(^+\)e\({}^-\) storage ring,
the definition of the
Courant-Snyder invariants must be generalised.
The MAD invariants have the advantage that they
remain invariants along the beam line even for variable \(\delta\).

With the new method this problem occurs in {\tt TWISS} only for
non-constant \(\delta\).

\subsection{Treatment of $\partial^2Q/\partial\delta^2$ in TWISS}
\label{S-treat}
It has been noted in~\cite{B-RUGGIERO} that MAD returned tunes which
are too low for non-zero \(\Delta p\).
The difference was found to be quadratic in \(\Delta p\) with a
negative coefficient.
This problem has been eliminated thanks to the new treatment 
of momentum errors from Version~8.13 onwards.

\appendix
\chapter{Format of the SURVEY and TWISS Files}
\label{A-TAPE3}
\index{data!streams|(}\index{streams|(}\index{files|(}
All output generated by the \ttindex{TAPE} options of the commands
\ttindex{SURVEY} or \ttindex{TWISS} has the same general layout.
The output is a coded file with default name {\tt SURVEY} or {\tt TWISS}.
It consists of a header record,
a set of position records,
and a trailer record.
\section{Common Output}
In all cases the header record has two lines.
It has the format
\myxmp{
(5A8,I8,L8,I8/A80)
}
and contains:
\begin{mylist}
\ttitem{PROGVRSN}
The current version number of MAD,
\ttitem{DATAVRSN}
This field may contain the following strings:
\begin{mylist}
\ttitem{SURVEY}
Output of the {\tt SURVEY} command follows,
\ttitem{TWISS}
Output of the {\tt TWISS} command (linear lattice functions),
follows
\ttitem{CHROM}
Output for the {\tt CHROM} option of the {\tt TWISS} command follows,
\end{mylist}
\ttitem{DATE}
The date of the MAD run.
This is the same date as on the printed output.
\ttitem{TIME}
The time of the MAD run.
This is the same time as on the printed output.
\ttitem{JOBNAME}
The job name for the MAD run, if available from the operating system.
\ttitem{SUPER}
The number of superperiods.
\ttitem{SYMM}
The symmetry flag.
This is true,
if only one half of a symmetric super-period was computed.
\ttitem{NPOS}
The number of position records that follow.
This is larger by one than the number of elements.
\ttitem{TITLE}
The page header as defined by the latest {\tt TITLE} command.
\end{mylist}
There are {\tt NPOS} position records.
Each of them contains four or five lines.
The first two lines have the format
\myxmp{
(A4,A16,F12.6,3E16.9/5E16.9)
}
and contain the following information about the preceding element:
\begin{mylist}
\ttitem{KEYWORD}
The element keyword,
\ttitem{NAME}
The element name,
\end{mylist}
Note that the \ttindex{TYPE} parameter no longer occurs in this file.
The remaining values are written according to Table~\ref{T-TTAPE3}.
The first position record refers to the position preceding
the first element.
Its content is
\myxmp{
KEYWORD=blank, NAME='INITIAL'
}
All other values are zero.
The remaining lines of the position records differ for the three cases.
\begin{table}[ht]
\caption{Element Data in Position Records}
\vspace{1ex}
\label{T-TTAPE3}
\index{element!data}
\centering
\begin{tabular}{|l|l|l|l|l|l|l|l|l|l|}
\hline
Keyword & & & & & & & & & \\
\hline
\ttindex{DRIFT}      &\tt L     &0        &0      &0       &0
                     &0         &0        &0      &0       \\
\ttindex{RBEND}      &\tt L     &\tt ANGLE&\tt K1 &\tt K2  &\tt TILT
                     &\tt E1    &\tt E2   &\tt H1 &\tt H2  \\
\ttindex{SBEND}      &\tt L     &\tt ANGLE&\tt K1 &\tt K2  &\tt TILT
                     &\tt E1    &\tt E2   &\tt H1 &\tt H2  \\
\ttindex{QUADRUPOLE} &\tt L     &0        &\tt K1 &0       &\tt TILT
                     &0         &0        &0      &0       \\
\ttindex{SEXTUPOLE}  &\tt L     &0        &0      &\tt K2  &\tt TILT
                     &0         &0        &0      &0       \\
\ttindex{OCTUPOLE}   &\tt L     &0        &0      &0       &\tt TILT
                     &\tt K3    &0        &0      &0       \\
\ttindex{MULTIPOLE}  &0         &\tt K0L  &\tt K1L&\tt K2L &\tt T0
                     &\tt K3L   &\tt T1   &\tt T2 &\tt T3  \\
\ttindex{SOLENOID}   &\tt L     &0        &0      &0       &0
                     &\tt KS    &0        &0      &0       \\
\ttindex{RFCAVITY}   &\tt L     &0        &0      &0       &0
                     &\tt FREQ  &\tt VOLT &\tt LAG&0       \\
\ttindex{ELSEPARATOR}&\tt L     &0        &0      &0       &\tt TILT
                     &\tt EFIELD&0        &0      &0       \\
\ttindex{KICKER}     &\tt L     &0        &0      &0       &\tt TILT
                     &\tt HKICK &0        &0      &0       \\
\ttindex{HKICKER}    &\tt L     &0        &0      &0       &\tt TILT
                     &\tt KICK  &0        &0      &0       \\
\ttindex{VKICKER}    &\tt L     &0        &0      &0       &\tt TILT
                     &\tt KICK  &0        &0      &0       \\
\ttindex{SROT}       &0         &0        &0      &0       &0
                     &\tt ANGLE &0        &0      &0       \\
\ttindex{YROT}       &0         &0        &0      &0       &0
                     &\tt ANGLE &0        &0      &0       \\
\ttindex{MONITOR}    &\tt L     &0        &0      &0       &0
                     &0         &0        &0      &0       \\
\ttindex{HMONITOR}   &\tt L     &0        &0      &0       &0
                     &0         &0        &0      &0       \\
\ttindex{VMONITOR}   &\tt L     &0        &0      &0       &0
                     &0         &0        &0      &0       \\
\ttindex{MARKER}     &0         &0        &0      &0       &0
                     &0         &0        &0      &0       \\
\hline
\end{tabular}
\end{table}

\section{SURVEY Output}
\index{geometry}
\ttnindex{SURVEY}
When {\tt DATAVRSN=SURVEY},
the third and fourth lines of the position records have the format
\myxmp{
(4E16.9/3E16.9)
}
and contain the global coordinates and the cumulative length
\index{coordinates!global}
in the order
\myxmp{
xxxxxxxxxx\=xxxxxxxxxx\=xxxxxxxxxx\=xxxxxxxxxx\=\kill
X         \>Y         \>Z         \>SUML \\
THETA     \>PHI       \>PSI
}
The trailer record has two lines in the format
\myxmp{
(3E16.9/3E16.9)
}
and contains the coordinates of the machine centre,
the minimum and maximum radius and the machine circumference
in the order
\myxmp{
xxxxxxxxxx\=xxxxxxxxxx\=xxxxxxxxxx\=xxxxxxxxxx\=\kill
X         \>Y         \>Z \\
RMIN      \>RMAX      \>C
}
\section{TWISS Output}
\ttnindex{TWISS}
\index{lattice functions}\index{functions!lattice}
When {\tt DATAVRSN=TWISS},
the third to fifth lines of the position records have the format
\myxmp{
(5E16.9/5E16.9/5E16.9)
}
and contain the quantities
\myxmp{
xxxxxxxxxx\=xxxxxxxxxx\=xxxxxxxxxx\=xxxxxxxxxx\=\kill
ALFX      \>BETX      \>MUX       \>DX        \>DPX \\
ALFY      \>BETY      \>MUY       \>DY        \>DPY \\
X         \>PX        \>Y         \>PY        \>SUML
}
in this order.
The trailer record has three lines in the format
\myxmp{
(3E16.9/5E16.9/5E16.9)
}
and contains
\myxmp{
xxxxxxxxxx\=xxxxxxxxxx\=xxxxxxxxxx\=xxxxxxxxxx\=\kill
DELTAP    \>GAMTR     \>C \\
COSMUX    \>QX        \>QX'       \>BXMAX     \>DXMAX \\
COSMUY    \>QY        \>QY'       \>BYMAX     \>DYMAX
}
in this order.
If the COUPLE option was given, the output has the same format,
but the quantities given refer to the two possible modes (1,2) instead
of referring to the two planes \((x,y)\).

\section{CHROM Output}
\ttnindex{CHROM}
\index{chromatic functions}\index{functions!chromatic}
When {\tt DATAVRSN=CHROM},
the third to fifth lines of the position records have the format
\myxmp{
(5E16.9/5E16.9/5E16.9)
}
and contain
\myxmp{
xxxxxxxxxx\=xxxxxxxxxx\=xxxxxxxxxx\=xxxxxxxxxx\=\kill
WX        \>PHIX      \>DMUX      \>DDX       \>DDPX \\
WY        \>PHIY      \>DMUY      \>DDY       \>DDPY \\
X         \>PX        \>Y         \>PY        \>SUML
}
in this order.
There is no trailer record in this case.

\chapter{Format for File Names}
\label{A-FILES}
\index{file names}
The MAD program normally uses alphanumeric names for files.
This Appendix describes the valid formats for file names,
as used in different computer operating systems.
\section{IBM VM/CMS System}
\label{S-IBM}
\index{IBM}
\index{VM/CMS}
\subsection{Standard Streams}
The standard streams are defined within MAD with the names listed in
Table~\ref{T-IBM}.
\begin{table}[ht]
\caption{Standard Files Used by MAD, IBM-VM/CMS Version}
\vspace{1ex}
\label{T-IBM}
\centering
\begin{tabular}{|l|l|l|}
\hline
Purpose                         &unit  &file name \\
\hline
Command dictionary input        & 4    &{\tt DICT} \\
Normal input                    & 5    & \\
Input lines and error messages  & 6    & \\
Plot output (GKS metafile)      & 8    &{\tt MAD METAFILE} \\
Plot output (HIGZ metafile)     & 8    &{\tt MAD PS} \\
Normal output                   &14    &{\tt PRINT} \\
Dynamic tables                  &15    &{\tt TABLE} \\
\hline
\end{tabular}
\end{table}
At CERN, the EXEC file {\tt MAD8 EXEC} stored on the {\tt MAD} disk
performs the following default initialisations:
\myxmp{
FILEDEF DICT     DISK MAD8 DICT *
FILEDEF 5        DISK TERMINAL \\
FILEDEF 6        DISK TERMINAL \\
FILEDEF PRINT    DISK problem LISTING scratch \\
FILEDEF DUMP     DISK problem DUMP scratch \\
FILEDEF METAFILE DISK MADPLOT METAFILE scratch
}
{\tt Problem} is the name of the input file
(or {\tt PROBLEM} in interactive mode).
{\tt Scratch} is the mode of a scratch disk created by the EXEC file.
The behaviour of MAD can be adapted to the user's needs
by changing the EXEC file.

\subsection{User-Defined Files}
\index{files, user-defined}
\index{user files}
All other file names must be chosen according to the VM/CMS conventions.
Within MAD a file name has one to three parts separated by dots
\mybox{
filename.\=filetype.\=filemode \\
filename.\>filetype \\
         \>filetype
}
All letters are converted to uppercase.
If all three parts are present, the file name is used as entered.
Example:
\myxmp{
CALL,FILENAME='LEPTWO.DATA.D'
}
\par If two parts are present,
they are taken as {\tt filename} and {\tt filetype}.
The {\tt filemode} defaults to~{\tt *} for input files,
or to the contents of the REXX variable {\tt SCRATCH}
for output files.
\index{REXX}
\ttindex{SCRATCH}
The CERN EXEC {\tt MAD8 EXEC} sets this variable to the mode of
a scratch disk.
Examples:
\myxmp{
xxxxxxxx\=\kill
CALL,   \>FILENAME='TEST.MAD' \\
TWISS,  \>TAPE='LEP.TWISS'
}
Assume that the mode of the scratch disk is {\tt F}.
{\tt CALL} requires an input file and {\tt TWISS} an output file,
so these commands have the same effect as:
\myxmp{
xxxxxxxx\=\kill
CALL,   \>FILENAME='TEST MAD *' \\
TWISS,  \>TAPE='LEP TWISS F'
}
\par If only one part appears, it is used as {\tt filetype}.
The {\tt filemode} has the same default as above, and {\tt filename}
\index{REXX}
defaults to the contents of the REXX variable {\tt PROBLEM}.
\index{PROBLEM}
The CERN EXEC {\tt MAD8 EXEC} sets this variable to the {\tt filename}
of the {\tt DATA} file, or if MAD runs interactively,
to the value {\tt PROBLEM}.
Example:
\myxmp{
xxxxxxxx\=xxxxxxxxxxxxxxxxxxxxxxxxxxxxxx\=\kill
CALL,   \>FILENAME='DATA' \\
CALL    \>                              \>! FILENAME='SAVE' is default \\
TWISS,  \>TAPE                          \>! FILENAME='TWISS' is default \\
TWISS,  \>TAPE='LATTICE'
}
Assume that the {\tt filename} of the {\tt DATA} file is {\tt LEP}
and the mode of the scratch disk is {\tt F}.
The above examples have then the same effect as:
\myxmp{
xxxxxxxx\=\kill
CALL,   \>FILENAME='LEP DATA *' \\
CALL,   \>FILENAME='LEP SAVE *' \\
TWISS,  \>TAPE='LEP TWISS F' \\
TWISS,  \>TAPE='LEP LATTICE F'
}

\section{UNIX and UNICOS Systems}
\label{S-UNIX}
\index{UNIX}
\index{UNICOS}
\subsection{Standard Stream Names}
The standard streams are defined within MAD with the names listed in
Table~\ref{T-UNIX}.
Note that on the Cray XMP the {\tt table} file is not used;
dynamic tables reside in SSD.
\begin{table}[ht]
\caption{Standard Files Used by MAD, UNIX Version}
\vspace{1ex}
\label{T-UNIX}
\centering
\begin{tabular}{|l|l|l|}
\hline
Purpose                         &unit  &file name \\
\hline
Command dictionary input        & 4    &{\tt dict} \\
Normal input                    & 5    &{\tt stdin} \\
Input lines and error messages  & 6    &{\tt stdout} \\
Plot output (GKS metafile)      & 8    &{\tt mad.metafile} \\
Plot output (HIGZ metafile)     & 8    &{\tt mad.ps} \\
Normal output                   &14    &{\tt print} \\
Dynamic tables                  &15    &{\tt table} \\
\hline
\end{tabular}
\end{table}
File names are converted to upper case, unless they are enclosed in quotes.
Without special action they refer to files in the
user's working directory.
They can be assigned other names by the {\tt ln} command
entered at the keyboard or issued in a shell script.
Such a link is desired in particular for the command dictionary:
\myxmp{
ln /usr/madman/mad.dict dict
}
This link makes the command dictionary available under the assumption
that it resides in the file {\tt /usr/madman/mad.dict}.
It is recommended to create a subdirectory for each problem,
and to store the data file in this subdirectory.
MAD should then be run while this directory is current,
so as to create all output files in the same directory.
This helps keeping related files together.

\subsection{User-Defined Files}
\index{files, user-defined}
\index{user files}
Other files can be assigned any name accepted by the UNIX system,
with the restriction that MAD accepts at most 40~characters,
and that it converts file names to upper case unless enclosed in quotes.
Examples:
\myxmp{
xxxxxxxx\=\kill
CALL,   \>FILENAME='ssc.data' \\
TWISS,  \>TAPE='ssc/twiss.data'
}
The first file is read in the user's working directory,
and the {\tt TWISS} output is written on the file {\tt twiss.data}
in the subdirectory {\tt ssc}.

\section{VAX VMS System}
\label{S-VMS}
\index{VAX VMS}
\subsection{Standard Stream Names}
The standard streams are defined within MAD with the names listed in
Table~\ref{T-VAX}.
\begin{table}[ht]
\caption{Standard Files Used by MAD, VAX-VMS Version}
\vspace{1ex}
\label{T-VAX}
\centering
\begin{tabular}{|l|l|l|}
\hline
Purpose                         &unit  &file name \\
\hline
Command dictionary input        & 4    &{\tt DICT} \\
Normal input                    & 5    & \\
Input lines and error messages  & 6    & \\
Plot output (GKS metafile)      & 8    &{\tt METAFILE} \\
Plot output (HIGZ metafile)     & 8    &{\tt MAD.PS} \\
Normal output                   &14    &{\tt PRINT} \\
Dynamic tables                  &15    &{\tt TABLE} \\
\hline
\end{tabular}
\end{table}
Without special action, these file names refer to files in the
user's working file directory.
They can be assigned other names by an {\tt ASSIGN} command
entered at the keyboard or issued in a DCL file.
This is desirable in particular for the command dictionary:
\myxmp{
ASSIGN DISK\$SI:[PUBSI]MAD8.DICT DICT
}
This command makes the dictionary available on the CERN central
VAX system (VXCERN).
The other standard streams should be assigned by a sequence like
\myxmp{
ASSIGN DISK\$gg:[user]mad.print PRINT \\
ASSIGN DISK\$gg:[user]mad.metafile METAFILE
}

\subsection{User-Defined Files}
\index{files, user-defined}
\index{user files}
Other files can be assigned any name accepted by the VMS system,
with the restriction that MAD accepts at most 40~characters.
All letters are converted to upper case.
Examples:
\mybox{
xxxxxxxx\=\kill
CALL,   \>FILENAME='SSC.DATA' \\
TWISS,  \>TAPE='DISK\$IZ[MADMAN]SSC.TWISS' \\
SURVEY, \>TAPE='DISK\$IZ[MADMAN.SSC]SURVEY.DATA'
}
The first file is read in the user's working directory;
the {\tt TWISS} output is written
on a file {\tt SSC.TWISS} in the current
directory,
and the {\tt SURVEY} output is written on a file {\tt SURVEY.DATA}
in the subdirectory {\tt SSC}.

\chapter{Format of TFS Files}
\label{S-TFS}
\index{TFS files}
\index{files}
 
TFS files (Table File System) are used in the LEP control system.
Their use is documented in Reference~\cite{B-TFS}.
The MAD program knows only coded TFS files.
This Appendix describes the special formats used.
 
\section{Descriptor Lines}
MAD writes the following descriptors on all tables:
\begin{mylist}
\ttitem{COMMENT}
The current title string from the most recent {\tt TITLE} command.
\ttitem{ORIGIN}
The version of MAD used.
\ttitem{DATE}
The date of the MAD run.
\ttitem{TIME}
The wall clock time of the MAD run.
\ttitem{TYPE}
The type of the table:
\begin{mylist}
\ttitem{TWISS}
Lattice function table.
\ttitem{OPTICS}
Output of the {\tt OPTICS} command.
\ttitem{TRACK}
Track table.
\ttitem{TUNES}
Table of tunes and chromaticities versus \(\Delta E/p_0 c\).
\end{mylist}
\end{mylist}
The track tables contain the following additional descriptors:
\begin{mylist}
\ttitem{BETX}
Horizontal \(\beta_x\),
\ttitem{ALFX}
Horizontal \(\alpha_x\),
\ttitem{BETY}
Vertical \(\beta_y\),
\ttitem{ALFY}
Vertical \(\alpha_y\),
\ttitem{X}
Horizontal orbit position,
\ttitem{PX}
Horizontal orbit slope,
\ttitem{Y}
Vertical orbit position,
\ttitem{PY}
Vertical orbit slope,
\ttitem{DX}
Horizontal dispersion,
\ttitem{DPX}
Horizontal dispersion slope,
\ttitem{DY}
Vertical dispersion,
\ttitem{DPY}
Vertical dispersion slope,
\end{mylist}
The format of descriptor lines is
\mybox{
('@ ',A16,' ',A4,...)
}
Where the {\tt A16} format is used for the descriptor name,
the {\tt A4} format for its format,
and the remaining information uses the format listed below for columns.
 
\section{Column Formats}
\index{TFS}
The column formats used are listed in Table~\ref{T-TFS}.
\begin{table}[ht]
\caption{Column Formats used in TFS Tables}
\label{T-TFS}
\vspace{1ex}
\centering
\begin{tabular}{|l|p{0.6\textwidth}|l|}
\hline
C format   &Meaning            &FORTRAN format \\
{\tt \%hd} &Short integer      &{\tt (I8)} \\
{\tt \%lf} &Long float         &{\tt (G20.12)} \\
{\tt \%f}  &Short float        &{\tt (G14.6)} \\
{\tt \%ks} &String of length k &{\tt ('"',A,'"')} \\
           &Undefined value    &{\tt ('\~\ ')} \\
\hline
\end{tabular}
\end{table}
Control lines begin with the TFS control character, followed by a blank.
Data lines begin with two blanks.
Columns are also separated by one blank character.
The column width is chosen such as to accommodate the large of the
column name and the data values of the column.
Thus an integer column uses at least 8 characters to accommodate
the {\tt (I8)} format,
but if its name has more than 8 characters, it becomes wider.
\index{data!streams|)}\index{streams|)}\index{files|)}
 
\clearpage
\begin{thebibliography}{10}
 
\bibitem{B-GKS}
{\sl The Graphical Kernel System (GKS)}.
ISO, Geneva, July 1985.
International Standard ISO 7942.
 
\bibitem{B-AUT73}
B. Autin and Y. Marti.
{\sl Closed Orbit Correction of Alternating Gradient Machines
  using a small Number of Magnets}.
CERN/ISR-MA/73-17, CERN, 1973.
 
\bibitem{B-BM}
J.~D.~Bjorken and S.~K.~Mtingwa.
{\sl Particle Accelerators} {\bf 13}, pg. 115.
 
\bibitem{B-BH}
P. Bramham and H. Henke.
private communication and LEP Note LEP-70/107, CERN.
 
\bibitem{B-BRO72}
Karl~L. Brown.
{\sl A First-and Second-Order Matrix Theory for the Design
  of Beam Transport Systems and Charged Particle Spectrometers}.
SLAC 75, Revision 3, SLAC, 1972.
 
\bibitem{B-BRO80}
Karl~L. Brown, D.~C. Carey, Ch. Iselin, and F. Rothacker.
{\sl TRANSPORT --- A Computer Program for Designing Charged
  Particle Beam Transport Systems}.
CERN 73-16, revised as CERN 80-4, CERN, 1980.
 
\bibitem{B-CHAO}
A. Chao.
{\sl Evaluation of beam distribution parameters in an electron
  storage ring}.
Journal of Applied Physics, 50:595--598, 1979.
 
\bibitem{B-CL}
A.~W. Chao and M.~J. Lee.
{\sl SPEAR II Touschek lifetime}.
SPEAR-181, SLAC, October 1974.
 
\bibitem{B-CM}
M.~Conte and M.~Martini.
{\sl Particle Accelerators} {\bf 17}, 1 (1985).

\bibitem{B-COU58}
E.~D. Courant and H.~S. Snyder.
{\sl Theory of the alternating gradient synchrotron}.
Annals of Physics, 3:1--48, 1958.
 
\bibitem{B-TFS}
Ph. Defert, Ph. Hofmann, and R. Keyser.
{\sl The Table File System, the C Interfaces}.
LAW Note 9, CERN, 1989.
 
\bibitem{B-DON82}
M. Donald and D. Schofield.
{\sl A User's Guide to the {\tt HARMON} Program}.
LEP Note 420, CERN, 1982.
 
\bibitem{B-DRA81}
A. Dragt.
{\sl Lectures on Nonlinear Orbit Dynamics, 1981 Summer School on High
  Energy Particle Accelerators, Fermi National Accelerator Laboratory, July
  1981}.
American Institute of Physics, 1982.
 
\bibitem{B-EDW72}
D.~A. Edwards and L.~C. Teng.
{\sl Parametrisation of linear coupled motion in periodic systems}.
IEEE Trans. on Nucl. Sc., 20:885, 1973.
 
\bibitem{B-GXPLOT}
H. Grote.
{\sl GXPLOT User's Guide and Reference Manual}.
LEP TH Note 57, CERN, 1988.
 
\bibitem{B-LEP79}
LEP~Design Group.
{\sl Design Study of a 22 to 130 GeV e\({}^{+}\)e\({}^{-}\) Colliding Beam
  Machine (LEP)}.
CERN/ISR-LEP/79-33, CERN, 1979.
 
\bibitem{B-GK}
M. Hanney, J.~M. Jowett, and E. Keil.
{\sl BEAMPARAM --- A program for computing beam dynamics and
  performance of \epem\ storage rings}.
CERN/LEP-TH/88-2, CERN, 1988.
 
\bibitem{B-HLMS}
R.~H. Helm, M.~J. Lee, P.~L. Morton, and M. Sands.
{\sl Evaluation of synchrotron radiation integrals}.
IEEE Trans. Nucl. Sc., NS-20, 1973.
 
\bibitem{B-JAM71}
F. James.
{\sl MINUIT, A package of programs to minimise a function of n
  variables, compute the covariance matrix, and find the true errors}.
program library code D507, CERN, 1978.
 
\bibitem{B-KEIL}
E. Keil.
{\sl Synchrotron radiation from a large electron-positron storage ring}.
CERN/ISR-LTD/76-23, CERN, 1976.
 
\bibitem{B-KNUTH}
D.~E. Knuth.
{\sl The Art of Computer Programming}.
Volume~2, Addison-Wesley, second edition, 1981.
Semi-numerical Algorithms.

\bibitem{B-MAI82}
H.~Mais and G.~Ripken,
{\sl Theory of Coupled Synchro-Betatron Oscillations}.
DESY internal Report, DESY M-82-05, 1982.

\bibitem{B-RUGGIERO}
J. Milutinovic and S. Ruggiero.
{\sl Comparison of Accelerator Codes for a RHIC Lattice}.
AD/AP/TN-9, BNL, 1988.
 
\bibitem{B-MON79}
B.~W. Montague.
{\sl Linear Optics for Improved Chromaticity Correction}.
LEP Note 165, CERN, 1979.

\bibitem{B-RIP70}
Gerhard Ripken,
{\sl Untersuchungen zur Strahlf\"uhrung und Stabilit\"at der
Teilchenbewegung in Beschleunigern und Storage-Ringen unter strenger
Ber\"ucksichtigung einer Kopplung der Betatronschwingungen}.
DESY internal Report R1-70/4, 1970.
 
\bibitem{B-TEN71}
L.~C. Teng.
{\sl Concerning n-Dimensional Coupled Motion}.
FN 229, FNAL, 1971.
 
\bibitem{B-UV}
U. V\"olkel.
{\sl Particle loss by Touschek effect in a storage ring}.
DESY 67-5, DESY, 1967.
 
\bibitem{B-RW}
R.~P. Walker.
{\sl Calculation of the Touschek lifetime in electron storage rings}.
1987.
Also SERC Daresbury Laboratory preprint, DL/SCI/P542A.
 
\bibitem{B-PW}
P.~B. Wilson.
{\sl Proc. 8th Int. Conf. on High-Energy Accelerators}.
Stanford, 1974.
 
\bibitem{B-WM}
A. Wr\"ulich and H. Meyer.
{\sl Life time due to the beam-beam bremsstrahlung effect}.
PET-75-2, DESY, 1975.
 
\end{thebibliography}

\printindex

\end{document}
