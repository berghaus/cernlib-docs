
\DEFCMD{M}{MYCMD}{MYMENU}{MYCMD}{}

   \par
This is the first sentence of the first paragraph.  The HELP menus show the 
   first phrase of the first paragraph.  The phrase finishes at the first line 
   ending with a '.' which is only now.  The first paragraph is still 
   continuing here but this sentence does not appear in the HELP menu.  

   \par
Consecutive lines are formatted into paragraphs.  The maximum line width 
   used for terminal output is determined by the SET/COLUMNS command (usually 
   80 columns).  Because blank lines are ignored the start of a new paragraph 
   has to be indicated by a line containing only a '.' in the first column.  

   \par
A '.' at the end of a line indicates a sentence ending period and adds an 
   additional space for improved readability.  Abbreviations like e.g. or etc. 
   should therefore not appear at the end of a line.  For consistency the 
   extra space between sentences belonging to the menu phrase should be added 
   by hand.  
\begin{verbatim}
    Text not starting in the first column is verbatim
                material and will not be reformatted.
\end{verbatim}
\begin{verbatim}
          This should       be used
                       for
             examples  and  tables.
\end{verbatim}
   \par
Again extra space between verbatim lines has to be indicated by '.' lines.  
   For terminal output a verbatim block is always separated by a blank line 
   from the surrounding paragraphs.  

\begin{verbatim}
 This is the user help information for 'MYCMD'.

 Current date and time: 940610 1001
\end{verbatim}
\ENDCMD
