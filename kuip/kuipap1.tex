%%%%%%%%%%%%%%%%%%%%%%%%%%%%%%%%%%%%%%%%%%%%%%%%%%%%%%%%%%%%%%%%%%%
%                                                                 %
%   KUIP  - Reference Manual -- LaTeX Source                      %
%                                                                 %
%   Chapter 6: KUIP Programming example                           %
%                                                                 %
%   External EPS files referenced: none                           %
%                                                                 %
%   Editor: Michel Goossens / CN-AS                               %
%   Last Mod.:  6 Dec  1991   mg                                  %
%                                                                 %
%%%%%%%%%%%%%%%%%%%%%%%%%%%%%%%%%%%%%%%%%%%%%%%%%%%%%%%%%%%%%%%%%%%

\chapter{KUIP Programming example}

As an example of how to implement a user interface with KUIP the code
for a simple Reverse Polish Notation pocket calculator is presented.
%
%---------------------------------------------------------------------------
%
\section{The Command Definition File}
\begin{XMPt}{The CDF for the RPN calculator}
>Name RPNDEF
 
>Menu RPN
>Guidance
Reverse Polish Notation sub-pocket calculator
using KUIP for the user interface.
 
>Command ENTER
>Parameters
+
NUM 'Enter a number' R D=0.
>Guidance
Push the number(s) given as parameter(s) into the stack.
If none push a zero.
>Action RPENT
 
>Command ADD
>Guidance
Push the number(s) given as parameter(s) into the stack.
Add the two upper-most numbers of the stack and shift it up.
>Action RPOPER
 
>Command SUBTRACT
>Guidance
Push the number(s) given as parameter(s) into the stack.
Subtract the two upper-most numbers of the stack and shift it up.
>Action RPOPER
 
>Command MULTIPLY
>Guidance
Push the number(s) given as parameter(s) into the stack.        
Multiply the two upper-most numbers of the stack and shift it up.
>Action RPOPER
\condbreak{5\baselineskip} 
>Command DIVIDE
>Guidance
Push the number(s) given as parameter(s) into the stack.
Divide the two upper-most numbers of the stack and shift it up.
>Action RPOPER
 
>Command PRINT
>Guidance
Print the content of the stack.
>Action RPRINT
 
>Command CLEAR
>Guidance
Clear the stack.
>Action RPCLR
\end{XMPt}
%
%---------------------------------------------------------------------------
%
\section{The application program}

\begin{XMPt}{FORTRAN code for the RPN calculator application program}
      PROGRAM RPN
      COMMON/PAWC/PAW(50000)
      COMMON/RPNSTK/ISTACK,STACK(100)
*---> Initialize ZEBRA and the store /PAWC/
      CALL MZEBRA(-3)
      CALL MZPAW(50000,' ')
*---> Initialize KUIP with 5000 words as minimum division size
      CALL KUINIT(5000)
*---> Create user command structure from definition file (command
*---> definition routine name RPNDEF defined in CDF 
*---> with '>Name RPNDEF')
      CALL RPNDEF
*---> Change the prompt by executing the command SET/PROMPT
      CALL KUEXEC('SET/PROMPT ''RPN >''')
*---> Initialize the stack in /RPNSTK/
      CALL RPCLR
*---> Give control to KUIP (with no 'STYLE G', call KUWHAG 
*---> to get 'STYLE G')
      CALL KUWHAT
*---> Typing 'QUIT' or 'EXIT' we return here
      END
 
      SUBROUTINE RPCLR
      COMMON/RPNSTK/ISTACK,STACK(100)
      DO 10 I=1,100
        STACK(I)=0.
10    CONTINUE
      ISTACK=50
      CALL RPRINT
      END
 
      SUBROUTINE RPENT
      COMMON/RPNSTK/ISTACK,STACK(100)
      CHARACTER*32 CMD
      CALL KUPATL(CMD,NPAR)
      IF(CMD.EQ.'ENTER' .AND. NPAR.EQ.0) NPAR=1
      DO 10 I=1,NPAR
        CALL KUGETR(R)
        ISTACK=ISTACK+1
        IF(ISTACK.GT.100) THEN
          PRINT *,'ERROR: Stack overflow'
          GOTO 20
        ENDIF
        STACK(ISTACK)=R
10    CONTINUE
20    CONTINUE
      IF(CMD.EQ.'ENTER') CALL RPRINT
      END
 
      SUBROUTINE RPOPER
      COMMON/RPNSTK/ISTACK,STACK(100)
      CHARACTER*32 CMD
      CALL KUPATL(CMD,NPAR)
      CALL RPENT
      IF(ISTACK.LT.2) THEN
        PRINT *,'ERROR: Stack underflow'
        GOTO 10
      ENDIF
      IF(CMD.EQ.'ADD') THEN
        STACK(ISTACK-1)=STACK(ISTACK)+STACK(ISTACK-1)
      ELSEIF(CMD.EQ.'SUBTRACT') THEN
        STACK(ISTACK-1)=STACK(ISTACK)-STACK(ISTACK-1)
      ELSEIF(CMD.EQ.'MULTIPLY') THEN
        STACK(ISTACK-1)=STACK(ISTACK)*STACK(ISTACK-1)
      ELSEIF(CMD.EQ.'DIVIDE') THEN
        IF(STACK(ISTACK).EQ.0) THEN
          PRINT *,'ERROR: Divide by zero'
        ELSE
          STACK(ISTACK-1)=STACK(ISTACK-1)/STACK(ISTACK)
        ENDIF
      ENDIF
      ISTACK=ISTACK-1
10    CONTINUE
      CALL RPRINT
      END

      SUBROUTINE RPRINT
      COMMON/RPNSTK/ISTACK,STACK(100)
      PRINT *,(STACK(I),I=ISTACK,ISTACK-3,-1)
      IF(STACK(ISTACK).GT.0) THEN
        PRINT *,'********'
      ELSE\condbreak{4\baselineskip} 
        PRINT *,'*********'
      ENDIF
      END
\end{XMPt}
\Rind[KUPATL]{}
\Rind[KUGETR]{}
%
%---------------------------------------------------------------------------
%
%\newpage
\section{Example of a run}

\begin{XMPt}{Example of a session using the RPN calculator}
$ rpn
 0.0000000 0.0000000 0.0000000 0.0000000
 *********
 RPN > \underline{style an}
 
 From  /...
 
  1:   KUIP          Command Processor commands.
  2:   MACRO         Macro Processor commands.
  3:   VECTOR        Vector Processor commands.
  4:   RPN           Reverse Polish Notation sub-pocket calculator
 
 Enter a number ('Q'=command mode): \underline{4}
 
 From  /RPN/...
 
  1: * ENTER         Push the number(s) given as parameter(s) into the stack.
  2: * ADD           Push the number(s) given as parameter(s) into the stack.
  3: * SUBTRACT      Push the number(s) given as parameter(s) into the stack.
  4: * MULTIPLY      Push the number(s) given as parameter(s) into the stack.
  5: * DIVIDE        Push the number(s) given as parameter(s) into the stack.
  6: * PRINT         Print the content of the stack.
  7: * CLEAR         Clear the stack.
 
 Enter a number ('\'=one level back, 'Q'=command mode): \underline{1}
 
  * /RPN/ENTER  [ NUM ]
 
 Add parameters or just <CR> to the command line
 ('#'=cancel execution, '?'=help) :

 /RPN/ENTER \underline{1 2 3}
 3.000000 2.000000 1.000000 0.0000000
 ********
 
 From  /RPN/...
 
  1: * ENTER         Push the number(s) given as parameter(s) into the stack.
  2: * ADD           Push the number(s) given as parameter(s) into the stack.
  3: * SUBTRACT      Push the number(s) given as parameter(s) into the stack.
  4: * MULTIPLY      Push the number(s) given as parameter(s) into the stack.
  5: * DIVIDE        Push the number(s) given as parameter(s) into the stack.
  6: * PRINT         Print the content of the stack.
  7: * CLEAR         Clear the stack.
 
 Enter a number ('\'=one level back, 'Q'=command mode): \underline{2}

  * /RPN/ADD
 
 Add parameters or just <CR> to the command line
 ('#'=cancel execution, '?'=help) :

 /RPN/ADD
 5.000000 1.000000 0.0000000 0.0000000
 ********
 From  /RPN/...
 
  1: * ENTER         Push the number(s) given as parameter(s) into the stack.
  2: * ADD           Push the number(s) given as parameter(s) into the stack.
  3: * SUBTRACT      Push the number(s) given as parameter(s) into the stack.
  4: * MULTIPLY      Push the number(s) given as parameter(s) into the stack.
  5: * DIVIDE        Push the number(s) given as parameter(s) into the stack.
  6: * PRINT         Print the content of the stack.
  7: * CLEAR         Clear the stack.
 
 Enter a number ('\'=one level back, 'Q'=command mode): \underline{q}\vspace{7pt}
 RPN > \underline{print}
 5.000000 1.000000 0.0000000 0.0000000
 ********
 RPN > \underline{enter 1 2 3}
 3.000000 2.000000 1.000000 5.000000
 ********
 RPN > \underline{add}
 5.000000 1.000000 5.000000 1.000000
 ********
 RPN > \underline{add}
 6.000000 5.000000 1.000000 0.0000000
 ********
 RPN > \underline{add}
 11.00000 1.000000 0.0000000 0.0000000
 ********
 RPN > \underline{add 5}
 16.00000 1.000000 0.0000000 0.0000000
 ********
 RPN > \underline{add 10 20}
 30.00000 16.00000 1.000000 0.0000000
 ********
 RPN > \underline{usage rpn/}
 
  * RPN/ENTER  [ NUM ]
  * RPN/ADD
  * RPN/SUBTRACT
  * RPN/MULTIPLY
  * RPN/DIVIDE
  * RPN/PRINT
  * RPN/CLEAR
 
 RPN > \underline{help clear}
 
  * /RPN/CLEAR
 
    Clear the stack.
 
 RPN > \underline{c}
 *** Ambiguous command. Possible commands are :
 
 /KUIP/ALIAS/CREATE
 /KUIP/SET_SHOW/COMMAND
 /KUIP/SET_SHOW/COLUMNS
 /MACRO/SYNTAX/Branching/CASE
 /VECTOR/CREATE
 /VECTOR/COPY
 /RPN/CLEAR
 
 RPN > \underline{r/c}
 0.0000000 0.0000000 0.0000000 0.0000000
 *********
 RPN > \underline{help rpn/print}
 
  * /RPN/PRINT
 
    Print the content of the stack.

 RPN > \underline{help en}
 
  * /RPN/ENTER  [ NUM ]
 
    NUM        R 'Enter a number' D=0
 
    Push the number(s) given as parameter(s) into the stack.
    If none push a zero.
 
 RPN > \underline{q}
$
\end{XMPt}
