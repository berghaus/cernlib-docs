\documentstyle[12pt]{article}

\begin{document}
\title{Requirements for a \\ HEP Database System}
\author{S. Banerjee/L3 \and
        L. Barone/L3   \and
        N. Colino/L3   \and
        R. Jones/OPAL  \and
        R. Mount/L3    \and
        J. Shiers/CN   \and
        N. Watson/OPAL}
\date{27 January 1991}
\maketitle

\Filename{H2hdbreq-Introduction}
\section{Introduction}
\par
A requirement that is common to all High Energy Physics experiments
is for a database system in which to store information such as
detector geometry and calibration constants. Up until now, experiments
have typically developed their own solutions to this problem, resulting
in a number of packages with similar functionality.
\par
The goal of the HEPDB working group is to provide a single package
that is sufficiently powerful to meet the major requirements of all
experiments whilst remaining fast and simple to use. The package
would be maintained and distributed in the CERN Program Library.
\Filename{H2hdbreq-General-Requirements}
\section{General Requirements}
\par
The following features are considered desirable:
\begin{enumerate}
\item
The system must be machine independant, i.e. it must run on all
types of computers commonly used in HEP.
\item
The system should be modular. This will enable parts of the system
to be migrated as new standards or techniques emerge.
\item
Both interactive and callable interfaces should exist.
\item
The package should be both easy to use, but yet rich in facilities.
\item
It must be flexible enough to meet the needs of different HEP experiments.
\item
It should be supported and developed in conjunction with the HEP community.
\item
It must run in both centralised and distributed modes.
\end{enumerate}
\par
It is interesting to point out that these requirements were written
in 1982! To this list, we would add
\begin{enumerate}
\item
The package should provide a superset of the functionality of existing systems.
\item
Existing software
should be reused whereever possible and appropriate.
\item
Newly entered data must be available on all relevant on-site
computers within 1 hour and on off-site computers within a few
hours.
\item
The system for distribution of updates must work in the wide-area,
i.e. world wide, and over a variety of networks.
\item
The database should be protected from system crashes etc.
Recovery procedures making the database unavailable for a significant
time are less desirable than mechanisms providing quick recovery.
\item
The need for human intervention must be minimised.
\item
The two extremes of bulky data (of whose internal structure the database system
little or nothing), and lower volume data (where every word can be individually
accessed or used as a search criterion) should be supported efficiently.
\item
The system must be both well supported and documented.
\item
The development and maintanance cost, both in financial and manpower
terms, must be kept to a minimum.
\end{enumerate}

\Filename{H2hdbreq-Requirements-for-the-Database-Software-Package}
\section{Requirements for the Database Software Package}
To prevent the requirements from becoming unintelligibly abstract, it is
assumed below that the database system is built upon the RZ package.  Only end-node
directories of the RZ hierarchical directory structure are used to store database
objects.  Each object in an end-node directory contains keys and data which closely
parallel the normal and long fields in a row of a relational database table.

Bulky data are stored in the data part of an RZ object, which is normally disk
resident.  In contrast, simple tabular data are stored entirely as RZ keys which
are normally memory-resident after the first reference to the directory.

In contrast to the original RZ philosophy, the RZ keys can contain both the
information which must be examined to identify an object and data which should
then be returned to the application.

The RZ requirement that every object in a directory have the same number of
keys is acceptable.

\subsection{Requirements: retrieval}
\begin{enumerate}
\item
Retrieval of objects must be as quick and efficient as
possible.
\item
Database objects should be retrieved by specifying
\begin{itemize}
\item
the hierarchically structured name of an end-node directory/table,
\end{itemize}
and by specifying or putting restrictions on
\begin{itemize}
\item
a number (which may be zero) of the user keys,
\item
the {\it instant} of time for which a valid object is required,
\item
the {\it instant} of time after which changes to the database should be ignored.
\end{itemize}
Retrieval operations which return more than one object (e.g. due to loose
requirements on the user keys) should be supported.

\item
Recently accessed data should be retained in memory and the data storage in memo
should be suitable for direct access by application code.  The application should
not need to store its own copies of database objects.

Utilities which allow the application to mark the memory resident data as
`in use'
and `not in use right now' are needed to allow the database package to manage
memory efficiently.
\end{enumerate}

\subsection{Requirements: storage}
\begin{enumerate}
\item
Objects should only be deleted in pathological cases.  Normally,
 insertion of a
new object with a validity time range having some overlap with an old one
should logically delete the old one for the period of overlap.

One might consider adopting
the more rigorous requirement that not one bit in an existing
object may be changed.  Thus objects initially inserted with validity `from now
to eternity' are not changed when a new object of the same type is inserted which
logically truncates the original infinite validity.

\item
Data compression algorithms should be supported.
Two categories can be identified:
\begin{itemize}
\item
standalone compression of single data objects
\item
storing only the differences between a new data object and an existing one
\end{itemize}
Within the second category, algorithms which can only compare data blocks of
identical length and structure have proved adequate in the past.

It should be possible to turn off data compression for reasons of speed or paranoia.

\item
The database package should be able to produce a `journal' of all activities
modifying the database.  The journal can be used by database servers to exchange
or broadcast updates.  The journal can also be used to restore the database to
its up-to-date condition starting from a backup copy.
\end{enumerate}

\subsection{Requirements: usage and availability}
\begin{enumerate}
\item
The package must have FORTRAN and C interfaces.
\item
Multi-user access must be supported.
\item
An interactive interface is needed (for debugging, maintenance and a few user applications).
The interactive interface should assume that it runs on workstations supporting MOTIF.

\item
The `database' may be several independent RZ files, some of which may be private
or test databases.  It must be possible for all needed RZ files to be simultaneously
open and, for example, for memory-resident copies of data from several RZ files to
be managed simultaneously by the database system.
\end{enumerate}

\subsection{Requirements: underlying package}

For clarity, RZ has been assumed as the underlying random access package without
any explicit reason why a commercial package would be unsuitable.  In the past
(1984) L3 examined commercial packages in some detail and decided that it would
impossible to use any of them.  Although many features of commercial products
have now improved we believe that the situation is essentially unchanged.

To provide a focus for the examination of commercial products we summarise the
advantages and disadvantages of using Oracle, which was the strongest contender
and is almost certainly the strongest contender now.

\subsubsection{Advantages of Oracle}
\begin{enumerate}
\item
The performance of Oracle was considered tolerable by L3 in 1984.  It is
almost certainly even more tolerable now.
\item
Oracle provides sophisticated features to ensure data integrity which are
probably superior to the current capabilites of RZ even when augmented by
the journalling facilities of DBL3.
\item
Oracle provides a full interactive user interface.
\item
Oracle Corp. is interested in the HEP market.
\end{enumerate}

\subsubsection{Disadvantages of Oracle}
\begin{enumerate}
\item
Oracle could replace RZ, but many/most of the features of DBL3 or OPCAL
would still have to be added to Oracle.
\item
Although Oracle's interactive interface is good, its FORTRAN interface is
very unpleasant (largely because of the mis-match between FORTRAN and
the relational model + SQL).
\item
Oracle at CERN requires CERN manpower to install and apply updates to the code a
to `administer' the databases.  It is possible to argue that using Oracle
extensively for calibration databases would cost as much or more CERN
manpower as developing and maintaining RZ to provide the same functions.

Oracle would also require more manpower than
RZ at any other laboratory or university at which it was installed.
\item
Oracle costs a significant amount of money and is not sold or maintained by
IBM/DEC/HP etc.  The administrative effort required to get Oracle installed on most
computers in a collaboration can be contrasted to that required to install,
for example, FORTRAN or (to be more modern) MOTIF.
\item
Oracle cannot be installed in many institutes due to Cocom restrictions.
\item
Unlike FORTRAN or MOTIF, Oracle is not an industry standard (although
the SQL Query language is a standard).  How can HEP guarantee Oracle
Corp's health and interest in HEP for 20 years?
\end{enumerate}

\subsubsection{Verdict on Oracle}
The key problem with Oracle is that for various reasons (manpower, cost,
Cocom, administrative inertia, \ldots ) it would be impossible for most
experiments to have Oracle available on all the computers involved
in the analysis.

\end{document}




\end{document}
