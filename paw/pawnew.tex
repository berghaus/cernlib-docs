\documentstyle[11pt,longtable,epsfig,dingbat]{cernman}
\setlongtables
\romanfont{times}
\begin{document}
\chapter{test}
\section{Part 1: Chapter 1.  What is PAW?}
 
PAW (Physics Analysis Workstation) is an interactive
utility for visualizing experimental data on a computer
graphics display.  It may be run in batch mode if desired
for very large data analyses; typically, however, the user
will decide on an analysis procedure interactively before
running a batch job.
 
PAW combines a handful of CERN High Energy Physics Library systems
that may also used individually in software that
processes and displays data.  The purpose of PAW is
to provide many common analysis and display
procedures that would be duplicated needlessly
by individual programmers, to supply a flexible way to invoke these
common procedures, and yet also to allow user customization where
necessary.
 
Thus, PAW's strong point is that it provides quick access to many
facilities in the CERN library.  One if its limitations is that these
libraries were not designed from scratch to work together, so that a
PAW user must eventually become somewhat familiar with many dissimilar
subsystems in order to make effective use of PAW's more complex
capabilities.  As PAW evolves in the direction of more sophisticated
interactive graphics interfaces and object-oriented interaction
styles, the hope is that such limitations will gradually become less
visible to the user.
 
PAW is most effective when it is run on a powerful computer
workstation with substantial memory, rapid access to a large amount of
disk storage, and graphics support such as a large color screen and a
three-button mouse.  If the network traffic can be tolerated, PAW can
be run remotely over the network from a large, multiuser client
machine to more economical servers such as an X-terminal.  In case such
facilities are unavailable, substantial effort has been made to ensure
that PAW can be used also in noninteractive or batch mode from mainframes
or minicomputers  using text terminals.
 
\subsection{What Can You Do with PAW?}
 
PAW can do a wide variety of tasks relevant to analyzing and
understanding physical data, which are typically statistical
distributions of measured events.   Below we list what are
probably the most frequent and best-adapted applications of PAW;
the list is not intended to be exhaustive, for it is obviously
possible to use PAW's flexibility to do a huge number of things,
some more difficult to achieve than others within the given structure.
 
\paragraph{Typical PAW Applications:}
 
\begin{itemize}
\item {\bf Plot a Vector of Data Fields for a List of Events.}
A set of raw data is typically processed by the user's own software to
give a set of physical quantities, such as momenta, energies, particle
identities, and so on, for each event.  When this digested data is
saved on a file as an \verb|ntuple|, it may be read and manipulated
directly from PAW.  Options for plotting
\verb|ntuples| include the following:
 
\begin{itemize}
\item {\it One Variable.\/}  If a plot of a one variable
from the data set is requested, a histogram showing the statistical
distribution of the values from all the events is automatically
created.  Individual events are not plotted, but appear only as a
contribution to the corresponding histogram bin.
\item {\it Two or Three Variables.\/}  If a plot
of two or three variables from the data set is requested,
no histogram is created, but a 2D or 3D
scatter plot showing a point or marker for
each distinct event is produced.
\item {\it Four Variables.\/}  If a plot
of four variables is requested, a 3D scatter plot of the first three
variables is produced, and a color map is assigned to the fourth
variable; the displayed color of the individual data points in the 3D
scatter plot indicates the approximate value of the fourth variable.
\item {\it Vector Functions of Variables.\/}  PAW allows
the user to define more-or-less arbitrary vector functions of the
original variables in an \verb|ntuple|, and to plot those instead of
the the bare variables.  Thus one can easily plot something
like $\sqrt(P_{x}^{2} + P_{y}^{2})$ if $P_{x}$ and $P_{y}$ are
original variables in the data without having to add a new data
field to the \verb|ntuple| at the time of its creation.
\item {\it Selection Functions (Cuts).\/}  PAW does not require you
to use every event in your data set.  Several methods are provided
to define Boolean functions of the variables themselves
that pick out subsets of the events to be included in a plot.
\end{itemize}
 
\item {\bf Histogram of a Vector of Variables for a List of Events.}
Often one is more interested in the statistical distribution
of a vector of variables (or vector functions of the variables)
than in the variables themselves.  PAW provides utilities for
defining the desired limits and
bin characteristics of a histogram
and accumulating the bin counts by scanning through a list of
events.  The following are some of the features available
for the creation of histograms:
 
  \begin{itemize}
  \item {\it One Dimensional Histograms.\/}  Any single variable
can be analyzed using a one-dimensional histogram that shows
how many events lie in each bin.  This is basically equivalent to
the single-variable data plotting application except that it
is easier to specify personalized features of the display format.
A variety of features allow the user to slice and project a 2D
scatter plot and make a 1D histogram from the resulting projection.
  \item {\it  Two-Dimensional Histograms.\/} The distribution
of any pair of variables for a set of events can be accumulated
into a 2D histogram and plotted in a various of ways to show
the resulting surface.
  \item {\it Three-Dimensional Histograms.\/}  Will be supported
soon.
\item {\it Vector Functions of Variables.\/} User-defined functions
of variables in each event can be used to define the histogram,
just as for an \verb|ntuple| plot.
\item {\it Selection Functions (Cuts).\/}  Events may also
be included or excluded  by invoking Boolean selection functions
that are arbitrary functions of the variables of a given event.
  \item {\it Event Weights.\/} PAW allows the user to include a
multiplicative statistical bias for each event which is a scalar
function of the available variables.  This permits the user to
correct for known statistical biases in the data when making
histograms of event distributions.
  \item {\it Histogram Presentation Options.\/}  Virtually every
aspect of the appearance of a histogram can be controlled by the
user.  Axes labels, tick marks, labels, colors, fonts, and so
on, are specified by a large family of options.  A particular
set of options may be thought of as a ``style'' for presenting
the data in a histogram; ``styles'' are in the process of
becoming a formal part of PAW to aid the user in making
graphics that have a standard pleasing appearance.
  \end{itemize}
 
\item {\bf Fit a Function to a Histogram.}  Once a
histogram is defined, the user may fit the resulting shape with one of
a family of standard functions, or with a custom-designed function.
The parameters of the fit are returned in user-accessible form.
Fitted functions of one variable may be attached to a 1D histogram and
plotted with it.  The capability of associating fits to higher
dimensional histograms and overlaying their representations on the
histogram is in the process of being added to PAW.
 
The fitting process in PAW is normally carried out by the MINUIT
library.  To user this package effectively, users must typically
supply data with reasonable numerical ranges and give reasonable
initial conditions for the fit before passing the task to the
automated procedure.
 
\item {\bf Annotate and Print  Graphics.}
A typical objective of a PAW user is to examine, manipulate, and
display the properties of a body of experimental data, and then to
prepare a graph of the results for use in a report, presentation, or
publication.  PAW includes for convenience a family of graphics
primitives and procedures that may be used to annotate and customize
graphics for such purposes.  In addition, any graphics display
presented on the screen can be converted to a PostScript file for
black-and-white or color printing, or for direct inclusion in a
manuscript.
 
\end{itemize}
 
\subsection{A User's View of PAW}
 
In order to take advantage of PAW, the user must first have
an understanding of its basic structure.  Below we explain
the fundamental ways in which PAW and the user interact.
 
\paragraph{Intialization.}  PAW may be invoked in a variety
of ways, depending on the user's specific computer system;
these are listed in an Appendix to the following chapter.
As PAW starts, it prompts the user to select an interaction
mode (or non-interactive mode) and window size and type
(if interactive).  The available window sizes and positions
are specified in the user file \Lit{higz_windows.dat}.
User-specific intializations are specified in the file
{\tt pawlogon.kumac}.
 
\paragraph{Text Interface.}  The most basic
interface is the {\bf KUIP text interface.}  KUIP provides
a basic syntax for commands that are parsed and passed on
to the PAW application routines to perform specific tasks.
Among the basic features of KUIP with which the user
interacts are the following:
 
\begin{itemize}
\item {\it Command Entry.\/}  Any unique partially entered command
is interpreted as a fully entered command.  KUIP
responds to an ambiguous command by listing the possible
alternatives.
On UNIX systems, individual
command lines can be  edited in place using individual
control keystrokes similar to those of the EMACS editor, or the
BASH or TCSH UNIX command shells.  On other systems, a
command line that is in error can only be revised after
it is entered, using an ``ed'' style text line editing
language.
\item {\it Parameters.\/}
Parameters are typed in following the basic command and
are separated by spaces, so algebraic expressions may not
have embedded spaces.  An exclamation point {\tt (!)} can
be used to keep the default parameters in a sequence when
only a later parameter is being changed.  If an underscore
\Lit{(_)} is the last character on a line, the command
may be continued on the next line; no spaces are allowed in
the middle of continued parameter fields.
\item {\it Command History.\/}  A command history is kept
both for interactive inspection and on disk.  The command
history file can be recovered and used to reconstruct
a set of actions carried out interactively.
\item {\it On-Line Assistance.\/}  The \verb|usage| and
\verb|help| commands can be used to get a short or verbose
description of  parameters and features of any command.
\item {\it Aliases.\/} ...
\item {\it Macros.\/} ...
\end{itemize}
 
\paragraph{KUIP/MOTIF Interface.}  If the user's workstation
supports the X-window MOTIF graphics management system, PAW can be
started in the KUIP/MOTIF mode.  A small text panel and a command
history panel keep track of individual actions and permit entry and
recall of typed commands similar to the Text Interface mode.  Other
basic features of this interface include the following:
 
\begin{itemize}
\item {\it Pull-Down Menu Commands.\/}  Each PAW command that
can be typed as a text command has a corresponding item in
a hierarchy of pull-down menus at the top of the MOTIF panel.
Commands that require arguments cause a parameter-entry
dialog box to appear;  when the arguments are entered, the
command is executed as though typed from the text interface.
\item {\it Action Panel.\/}  A user may have a family of
frequently executed macros or commands assigned to specific
buttons on the action panel.
%% \item {\it Browser Panel.\/}
%% \item {\it Direct Graphics Interaction.\/}
\end{itemize}
 
\paragraph{Graphics Output Window.}  The graphics produced
by PAW commands, regardless of the command interface, appears
on a separate graphics output window.  The actual size and
position of this window on the screen is controlled by a list
of numbers of the form {\tt x-upper-left y-upper-left x-width y-height}
in the user file \Lit{higz_windows.dat}.  The width and height
of the drawing area within this window is subject to additional
user control, and the user can specify ``zones,'' which are essentially
ways of dividing the window into panes to allow simultaneous display
of more than one plot.  Some facilities are available in the
current version of PAW to use the mouse to retrieve data such
as the height of a histogram bin.  Applications currently
under development will extend this style of interaction in
other directions.
 
\subsection{Fundamental Objects of PAW}
 
PAW is implicitly based on a family of fundamental objects.
Each PAW command performs an action that either produces
another object or produces a ``side-effect'' such as
a printed message or graphics display that is not saved
anywhere as a data structure.  Some commands do both,
and some may or may not produce a PAW data structure depending
on the settings of global PAW parameters.  In this section,
we describe the basic objects that the user needs to keep
in mind when dealing with PAW;  the reader should perhaps
note that the PAW text commands themselves do not
necessarily reflect the nature of PAW objects as clearly
as they might.
 
\paragraph{Objects:}
\begin{itemize}
\item {\bf Ntuples.}  An \verb|ntuple| is the basic type
of data used in PAW.  It consists of a list of identical
data structures, one for each event. An \verb|ntuple|
made available by opening a file that contains a ZEBRA logical
directory which itself contains one or more \verb|ntuples|.
A storage area for \verb|ntuple| may  be created directly
using \verb|ntuple/create|, and then have data stored in
it using the \verb|ntuple/loop| or \verb|ntuple/read| commands.
A family of commands merge \verb|ntuples| into larger \verb|ntuples|,
project vector functions of the \verb|ntuple| variables into
histograms, and plot selected subsets of events.
\item {\bf Cuts.}  Cuts are functions with boolean values
that are used to select subsets of events in an \verb|ntuple|
to be used in  commands such as those that create histograms
and plot variables.
\item {\bf Masks.}  Masks are identical in form to a logical
variable added on the end of an \verb|ntuple|'s data structure.
Masks are constructed using the results of cuts, and are currently
stored in separate files, although is logical to
associate them directly with the \verb|ntuple| data.  A mask
is useful only for efficiency;  the effect of a mask is identical
to that of the cut that produced it.
\item {\bf 1D Histograms.}
\item {\bf 2D Histograms.}
\item {\bf Styles.}
\item {\bf Pictures.}
\item {\bf ZEBRA(RZ) Logical Directories.}
\item {\bf Operating System File Directories.}
\end{itemize}
 
\subsection*{Appendix to Chapter 1: A PAW Glossary}

\subsubsection*{Data Analysis Terminology}


\begin{DL}{MMMMM}
\item{DST} \mbox{}\\
     A ``Data Summary Tape'' is one basic form of  output from
     as typical physics experiment.  A DST is generally not used
     directly by PAW, but is analyzed by customized user programs
     to produce \verb|ntuple| files, which PAW can read directly.
\item[Ntuple]\mbox{}\\
     A list of identical data structures, each typically corresponding
     to a single experimental event.  The data structures
     themselves frequently consist of a row of numbers, so that
     many \verb|ntuples| may be viewed as
     two-dimensional arrays of data variables, with one
     index of the array describing the position of the data
     structure in the list (i.e., the row or event number),
     and the other index referring to the position of the data
     variable in the row (i.e., the column or variable number).
     A meaningful name is customarily assigned to each column
     that describes the variable contained in that column for each
     event.  However, the underlying utilities dealing with
     \verb|ntuples|  are currently being generalized to allow
     the name of an element of the data structure to refer
     not only to a single number, but also
     to more general data types such as arrays, strings, and so on.
     Thus it is more general to view an \verb|ntuple| as a sequence
     of tree-structured data, with names assigned to the top-level
     roots of the tree for each event.
\item[Event]\mbox{}\\
     A single instance of a set of data  or experimental measurements,
     usually consisting of a sequence of variables or structures of
     variables resulting from a partial analysis of the raw data.
     In PAW applications, one typically examines the statistical
     characteristics of large sequences of similar events.
\index{event}
\item[Variable]\mbox{}\\
     One of a user-defined set of named values associated with a single
     event in an \verb|ntuple|.
     For example,
     the $(x,\,y,\,z)$ values of a momentum vector could each
     be variables for a given event.  Variables are typically
     useful experimental quantities that are stored in an
     \verb|ntuple|;  they are used in algebraic formulas
     to define boolean cut criteria
     or other dependent variables that are
     relevant to the analysis.
\item[Cut]\mbox{}\\
     A  boolean-valued function of the variables of a given event.
     Such functions allow the user to specify that only  events
     meeting certain criteria are to be included in a given distribution.
\index{cut}
\item[Mask]\mbox{}\\
     A set of columns of zeros and ones that is identical in form
     to a new set of \verb|ntuple| variables.  A mask is typically
     used to save the results of applying a set of cuts to a large
     set of events so that
     time-consuming selection computations are not repeated needlessly.
\index{mask}
\end{DL}

\subsubsection*{Statistical Analysis Terminology}

\begin{DL}{MMMMM}
\item[Histogram] A one- or two-dimensional array of data, generated
                 by HBOOK in batch or in a PAW session. Histograms are (implicitly or
                 explicitly) declared (booked), filled by explicit entry of data
                 or can be derived from other histograms. The information stored
                 about a histogram includes a title, binning and packing definitions,
                 bin contents and errors, statistic values, possibly an
                 associated function vector and output attributes.
                 Some of these items are optional.
                 The ensemble of this information constitutes an {\bf histogram}.
\index{histogram}
\index{histogram!booking}
\item[Booking]   The operation of declaring (creating) an histogram.
\index{book histogram}
\item[Filling]   The operation of entering data values into a given histogram.
\index{fill!histogram}
\index{histogram!filling}
\item[Fitting]   Least squares and maximum likelihood fits of
                 parametric functions to histograms and vectors.
\index{fit}
\item[Projection]The operation of projecting two-dimensional
                 distributions onto both axes.
\index{projection}
\item[Band]      A band is a projection onto the X (or Y) axis
                 restricted to an interval
                 along the other Y (or X) axis.
\index{band}
\item[Slice]     A slice is a projection onto the X (or Y) 
                 axis restricted to one bin
                 along the other Y (or X) axis.
                 Hence a slice is a special case of a band, with
                 the interval limited to one bin.
\index{slice}
\item[Ntuple]    Two-dimensional array, characterised by a
                 {\bf fixed} number \Lit{N}, 
                 specifying the number of entries per element,
                 and by a {\bf length}, giving the
                 total number of elements. 
                 An element of a Ntuple can be thought of
                 as a physics event.
                 Selection criteria can be applied to each entry of an element and
                 a complete Ntuple can be analysed in a fast, efficient
                 and interactive way.
\index{Ntuple}
\index{Ntuple!selection criteria}
\index{selection criteria!Ntuple}
\item[DST]       Data Summary Tape, also, Data Summary File. 
                 A collection of experimental data,
                 organized in a sequential way.
\index{DST!Data Summary!Tape}
\index{DST!Data Summary!File}
\item[Function]  Sequence of one or more statements with a FORTRAN-like syntax
                 entered on the command line or via an external file.
\index{function}
\item[Weight]\mbox{}\\
      Events may also
     be included or excluded  by invoking Boolean selection functions
     that are arbitrary functions of the variables of a given event.
\item[Event Weights] \mbox{}\\
     PAW allows the user to include a
     multiplicative statistical bias for each event which is a scalar
     function of the available variables.  This permits the user to
     correct for known statistical biases in the data when making
     histograms of event distributions.
\end{DL}

\subsubsection*{KUIP/ZEBRA User Environment Terminology}

\begin{DL}{MMMMM}
\item[Macro]     A text file containing PAW commands 
                 and logical constructs to control the flow of execution. 
                 Parameters can be supplied to it when calling the macro.
\index{macro}
\item[Vector]    The equivalent of a FORTRAN array supporting 
                 up to three dimensions.
                 The elements of a vector can be stored using a real or an
                 integer representation;
                 they can be entered interactively on a terminal or read
                 from an external file.
\index{vector}
\item[Logical Directory]\mbox{}\\
     The ZEBRA data storage system resembles a file system organized
     as logical directories.  PAW maintains
     a global variable corresponding to the ``current directory'' where
     PAW applications will look for PAW objects such as histograms.
     The ZEBRA directory structure is a tree, and user functions permit
     the ``current directory'' to be set anywhere in the current tree,
     as well as creating new ``directories'' where the results
     of PAW actions can be stored.  A special
     directory called {\tt //PAWC} corresponds to a memory-resident
     branch of this virtual file system.  ZEBRA files may be written
     to the operating system file system, but entire hierarchies of
     ZEBRA directories typically are contained in a single binary operating
     system file.
\end{DL}

\subsubsection*{Graphics Production Terminology}

\begin{DL}{MMMMM}
\item[GKS]       The {\bf G}raphical {\bf K}ernel {\bf S}ystem is
                 ISO standard document ISO~8805. It defines a common
                 interface to interactive computer graphics 
                 for application programs.
\index{GKS!Graphical Kernel System}
\item[Metafile]  A file containing graphical information
                 stored in a device independent format,
                 which can be replayed on various types of output devices.
                 (e.g. the GKS Appendix E metafile and PostScript, 
                 both used at CERN).
\index{metafile}
\item[Picture]   A graphics object composed of graphics primitives 
                 and attributes.
                 Pictures are generated by
                 the HIGZ graphics interface and they can be stored in a picture
                 direct-access database, built with the RZ-package of the
                 data structure manager ZEBRA.
\index{picture}
\index{PostScript}
\item[PostScript]\mbox{}\\
     A high level page description language permitting the description of complex
     text and graphics using only text commands.  Using PostScript
     representations of graphics makes it possible to create graphics
     files that can be exchanged with other users and printed on
     a wide variety of printers without regard to the computer system
     upon which the graphics were produced.  Any graphics display
     produced by PAW can be expressed in terms of PostScript, written
     to a file, and printed.
\end{DL}
 
\chapter{Getting Started with PAW.}
\section*{Appendix to Chapter 2: Accessing PAW from
Specific Computer Systems.}
 
\subsection{SUN SparcStation}
 
\begin{XMP}
 ******************************************************
 *                                                    *
 *            W E L C O M E    to   P A W             *
 *                                                    *
 *           Version 1.13/00  9 March 1992            *
 *                                                    *
 ******************************************************
 Workstation type (?=HELP) <CR>=1 : ?
 
 List of valid workstation types:
       0:  Alphanumeric terminal
    1-10:  Describe in file higz_windows.dat
  n.host:  Open the display on host (1 < n < 10)
       m:  PAW_MOTIF on local host
  m.host:  PAW_MOTIF on specified host
    7878:  FALCO terminal
    7879:  xterm
 
 Metafile workstation types:
    -111:  HIGZ/PostScript (Portrait)
    -112:  HIGZ/PostScript (Landscape)
    -113:  HIGZ/Encapsulated PostScript
  -777/8:  HIGZ/LaTex
 
 Workstation type (?=HELP) <CR>=1 :
Version 1.14/08 of HIGZ started
 
<Now reading pawlogon.kumac>
\end{XMP}
 
\section{Part 1: Chapter 3. Doing More Complicated Tasks with PAW.}
\subsection{A Developer's View of PAW}
\subsection*{Appendix to Chapter 3: Wallet Card Summary of
Useful PAW Commands.}
 
 
\section{Part 2: How to Use PAW Subsystems.}
 
\section{Part 3: Reference Section.}
 
\section*{Appendix: A library of examples of PAW usage.}
 
\end{document}
 
 
