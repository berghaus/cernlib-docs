\chapter{Intended audience}

This manual is aimed primarily at those people responsible for the
installation of the CERN Program Library on systems at CERN or
elsewhere. Some of the material is only relevent for the
CERN Program Library section of the CN/ASD group at CERN.

Having read this manual you should be able to:

\begin{enumerate}
\item
Install all or part of the CERN Program Library on one of the 
supported platforms.
\item
Understand how to port the Program Library to a new platform.
\end{enumerate}

Certain sections of this manual will also be of interest
to software developers, such as Appendix \ref{sect-FLAGS},
where the C pre-processor flags are described and Appendix \ref{sect-PREFIX},
where reserved routine and common block prefixes are listed.

\chapter{Overview of the CERN Program Library and related environment}
The CERN Program  Library is a large collection  of general purpose programs
maintained  and offered  in both  source and  object code  form on  the CERN
central computers.   Most of these programs  were developed at CERN  and are
therefore  oriented towards  the  needs of  a  physics research  laboratory.
Nearly all, however, are of  a general mathematical or data-handling nature,
applicable to a wide range of problems.
 
\section{Contents and Organization of the Library}
The library contains several thousand subroutines and complete programs which are
grouped  together  by  logical  affiliation into several hundred program
packages. 80\%  of the programs are  written in FORTRAN and  the remainder in
assembly code, or C  usually with  a FORTRAN  version also  available.  The
language supported is currently Fortran 77.
 
Each package is assigned a unique code, consisting of a letter
and three or four digits. The  letter is used to indicate the category
in which the program or package resides. A package  consists of one or more related
subprograms with one package name and one or more user-callable entry names,
all described briefly in a {\tt Short write-up}~\cite{bib-CERNLIB}, and if necessary, an additional
{\tt Long  write-up}.  The terms under which the library material and associated
documentation may be distributed are given below.
 
\section{Availability and Charging}
\begin{itemize}
\item
Access to the CERN Program Library is free of charge to all HEP users
worldwide.
\item
Network  access  is the  sole distribution method.  
In order to gain  access, users must register the node
name of  their computer with  the Program  Library Office.  This  service is
free of charge.  Distribution on magnetic tape is not available.
\item
Non HEP academic and not-for-profit organizations are offered
registered network access for one  year for a registration fee of 1000
Swiss Francs. This service is free for physics departments of
institutes in CERN member states.
\item
Commercial users are offered 
registered  network access  for 1 year  for a base  fee of  5000 Swiss Francs.

The above fees are doubled for requests coming from non-member states.
\end{itemize}
 
The  Library  is  not  available  for  any  purpose   related  to  military
applications.
 
Distribution  to  organizations  not  covered  by the  above  rests  at  the
discretion  of  the  Director-General.    Machine  readable  copies  of  the
documentation are provided with the library and recipients are free to make
copies  for  their  users.   Paper  copies of  the  documentation  are  made
available  for  a  charge  which  covers the  printing  and  handling  cost.
Charging is waived for cumulative orders of less than 100 Swiss Francs.
 
If  your organisation is  expected to pay a  fee as
described above,  it will be  billed by CERN after the material
has been shipped. Please include special
billing address and instructions, if any, with the request.
 
As of the  publication date of this  document the member states  of CERN are
Austria, Belgium, Czech Republic, Denmark, Finland, France, Germany, Greece,
Hungary, Italy, The Netherlands, Norway, Poland, Portugal, Slovak Republic, Spain, Sweden,
Switzerland and the United Kingdom.  

States with Observer status are Israel, the Russian Federation, Turkey, Yugoslavia (status
suspended after UN embargo, June 1992), the European Commission and Unesco.  
 
\section{Conditions of Use Outside CERN}
 
Programs  and  documentation are  provided  solely  for  the  use of  the 
organization to which  they are distributed, and may not  be redistributed or
reproduced in  large numbers without the  express agreement of  CERN.  Note
that such agreement may have to be established somewhere else in addition to
or instead of CERN in the case of programs originating from sources outside
CERN.  The  appropriate Short  Write-up gives information  about authorship.
The material cannot be sold.  CERN should be given credit in all references,
library documentation, and publications based on the programs.
 
If the programs are  modified beyond what is necessary to  adapt them to the
local  machine/system  environment,  it  should   be  made  clear  in  local
documentation that they are locally modified versions of the CERN originals.
CERN should be informed of such modifications, and given the possibility of
introducing  the  same modifications  in  the  original version.   If  local
modifications are so  important as to change significantly  the behaviour of
the program, its  name should also be  changed in order to  avoid confusion
with the  original.  CERN welcomes  comments concerning the  Program Library
service, but undertakes  no obligation for the maintenance  of the programs,
nor  responsibility   for  their  correctness,  and   accepts  no  liability
whatsoever resulting from the use of its programs.
 
\chapter{CERNLIB installation environment}

{\bf N.B. from release 96A of the CERN Program Library, the installation
procedures have been completely rewritten to

\begin{itemize}
\item
Use standard tools, rather than HEP specific packages
\item
Permit common procedures across multiple systems (Unix, Windows/NT, VMS)
\end{itemize}

HEP specific packages such as PATCHY and CMZ are no longer used.

}

In most cases, such as on the CERNSP system at CERN, the manual
page for any of the following components can be obtained by
typing, for example, \underline{man cvs}.

\section{CVS}

\htmladdnormallinkfoot{CVS}{http://asdwww.cern.ch/pl/cvs/index.html}, or the
Concurrent Versions System, is a system that lets groups of people work 
simultaneously on groups of files (for instance program sources). 

It works by holding a central `repository' of the most recent version of the files. You may
at any time create a personal copy of these files by `checking out' the files from the
repository into one of your directories. If at a later date newer versions of the files are put
in the repository, you can `update' your copy. 

You may edit your copy of the files freely. If new versions of the files have been put in the
repository in the meantime, doing an update merges the changes in the central copy into
your copy. 

When you are satisfied with the changes you have made in your copy of the files, you can
`commit' them into the central repository. 

When you are finally done with your personal copy of the files, you can `release' them and
then remove them. 

From release 96A of the CERN Program Library, the master sources of the library 
are maintained in CVS. However, you do not need access to CVS to build the
library yourself - this is done using 'checked -out' files - pure Fortran and/or C
files, which include CPP statements to handle conditional compilation (e.g. machine
dependancies) and include files.

\section{Imake}

Imake is a tool that was written by the X consortium and is used to build
the X software. 

Imake is used to generate Makefiles from a template, a set
of  cpp  macro  functions,  and a per-directory input file
called an Imakefile.   This  allows  machine  dependencies
(such  has  compiler options, alternate command names, and
special make rules) to be kept separate from the  descriptions of the various items to be built.

\section{Makedepend}

Makedepend was also written by the X consortium is used to
determine the dependencies that a given soruce file
has on various include files, either directly or indirectly.

Makedepend is normally invoked by having a "depend" target
in the makefile to be updated - typing \underline{make depend}
brings the dependencies in the makefile up to date.

\section{Make}

Make is a tool that allows you to rebuild programs
from their components. The {\it rules} for building {\it targets}
are described in a {\it makefile}. The make program ensures
that only those steps that have to be repeated are executed.
Thus, if none of the source files that make up a module have changed, 
they are not recompiled. 

\section{CPP}

The C preprocessor, or CPP, permits conditional compilation and 
performs file inclusion and macro substitution on
source files. It is automatically invoked by the C compiler, and also by
the Fortran compiler on many Unix systems (normally requiring an extension
.F, rather than .f).

\chapter{Space requirements}

\index{disk space}

The complete CERN library requires approximately 200MB of disk
space. Slightly over 50MB may be saved if the sources are not
kept locally.

\chapter{Description of CERNLIB components}

The CERN Program Library consists of a number of independent
Fortran callable libraries and a collection of complete
programs. The overall structure is described briefly below.
Note that the library is not static and small deviations
from what is described may exist.

\section{KERNLIB}

\KERNLIB{} is a Fortran callable library composed almost entirely
of individual subroutines and functions. All of these are described
in the CERNLIB short writeups manual~\cite{bib-CERNLIB}.

The \KERNLIB source files are organised as follows:
\begin{itemize}
\item
Machine independent routines, mainly Fortran.
\item
Machine dependent routines, often C or Assembler.
\item
Numerical routines, including random number generators.
\end{itemize}

The actual contents of the library varies slightly from
platform to platform, largely due to the machine 
dependent routines.

%A complete list of routines, grouped according to section,
%is given below.
%
%% 24 Feb 1997 mg
\chapter{Catalog of Program Packages and Entries}
\section*{Elementary Functions}
\begin{DLtt}{12345678901}
\item[B002 PRMFCT] Prime Numbers and Prime Factor Decomposition
\item[B100 RBINOM] Binomial Coefficient
\item[B101 ATG] Arc Tangent Function
\item[B102 ASINH] Hyperbolic Arcsine
\item[B105 RPLNML] Value of a Polynomial
\item[B300 RSRTNT] Integral of type $R(x,\sqrt{a+bx+cx^2})$
\end{DLtt}
\section*{Equations and Special Functions}
\begin{DLtt}{12345678901}
\item[C200 RZEROX] Zero of a Function of One Real Variable
\item[C201 RSNLEQ] Numerical Solution of Systems of Nonlinear Equations
\item[C202 RMULLZ] Zeros of a Real Polynomial
\item[C205 RZERO] Zero of a Function of One Real Variable
\item[C207 RRTEQ3] Roots of a Cubic Equation
\item[C208 RRTEQ4] Roots of a Quartic Equation
\item[C209 CPOLYZ] Zeros of a Complex Polynomial
\item[C210 NZERFZ] Number of Zeros of a Complex Function
\item[C300 ERF] Error Function and Complementary Error Function
\item[C301 FREQ] Normal Frequency Function
\item[C302 GAMMA] Gamma Function for Positive Argument
\item[C303 GAMMF] Gamma Function for Real Argument
\item[C304 ALGAMA] Logarithm of the Gamma Function
\item[C305 CGAMMA] Gamma Function for Complex Argument
\item[C306 CLGAMA] Logarithm of the Gamma Function for Complex Argument
\item[C309 CCLBES] Coulomb Wave, Bessel, and Spherical Bessel
Functions for Complex Argument(s) and \\ Order
\item[C312 BESJ0] Bessel Functions J and Y of Orders Zero and One
\item[C313 BESI0] Modified Bessel Functions I and K of Orders
Zero and One
\item[C315 RRIZET] Riemann Zeta Function
\item[C316 RPSIPG] Psi (Digamma) and Polygamma Functions
\item[C317 CPSIPG] Psi (Digamma) and Polygamma Functions for Complex
Argument
\item[C318 RELFUN] Jacobian Elliptic Functions sn, cn, dn
\item[C320 CELFUN] Jacobian Elliptic Functions sn, cn, dn for
Complex Argument
\item[C321 CGPLG] Nielsen's Generalized Polylogarithm
\item[C322 RFRSIN] Fresnel Integrals
\item[C323 RFERDR] Fermi-Dirac Function
\item[C324 RATANI] Arctangent Integral
\item[C326 RCLAUS] Clausen Function
\item[C327 BSIR4] Modified Bessel Functions I and K of
Order 1/4, 1/2 and 3/4
\item[C328 CWHITM] Whittaker Function M of Complex Argument and
Complex Indices
\item[C330 RASLGF] Legendre and Associated Legendre Functions
\item[C331 RFCONC] Conical Functions of the First Kind
\item[C332 RDILOG] Dilogarithm Function
\item[C334 RGAPNC] Incomplete Gamma Functions
\item[C335 CWERF] Complex Error Function
\item[C336 RSININ] Sine and Cosine Integrals
\item[C337 REXPIN] Exponential Integral
\item[C338 CEXPIN] Complex Exponential Integral
\item[C339 RDAWSN] Dawson's Integral
\item[C340 BSIR3] Modified Bessel Functions I and K of Order 1/3 and 2/3
\item[C341 BSKA] Modified Bessel Functions K of Certain Order
\item[C342 RSTRH0] Struve Functions of Orders Zero and One
\item[C343 BSJA] Bessel Functions J and I with Positive
Argument and Non-Integer Order
\item[C344 CBSJA] Bessel Functions J with Complex Argument
and Non-Integer Order
\item[C345 RBZEJY] Zeros of Bessel Functions J and Y
\item[C346 RELI1] Elliptic Integrals of First, Second, and Third Kind
\item[C347 RELI1C] Complete Elliptic Integrals of First,
Second, and Third Kind
\item[C348 CELINT] Elliptic Integral for Complex Argument
\item[C349 RTHETA] Jacobian Theta Functions
\end{DLtt}
\section*{Integration, Minimization, Non-linear Fitting}
\begin{DLtt}{12345678901}
\item[D101 SIMPS] Integration by Simpson's Rule
\item[D102 RADAPT] Adaptive Gaussian Quadrature
\item[D103 GAUSS] Adaptive Gaussian Quadrature
\item[D104 RCAUCH] Cauchy Principal Value Integration
\item[D105 RTRINT] Integration over a Triangle
\item[D106 RGS56P] Gaussian Quadrature with Five- and Six-Point Rules
\item[D107 RGQUAD] N-Point Gaussian Quadrature
\item[D108 TRAPER] Trapezoidal Rule Integration with an Estimated Error
\item[D110 RGMLT] Gaussian Quadrature for Multiple Integrals
\item[D113 CGAUSS] Adaptive Complex Integration Along a Line Segment
\item[D114 RIWIAD] Adaptive Multidimensional Monte-Carlo Integration
{\bf [Obsolete]}
\item[D120 RADMUL] Adaptive Quadrature for Multiple Integrals over
$N$-Dimensional Rectangular Regions
\item[D151 DIVON4] Multidimensional Integration or Random Number
Generation {\bf [Obsolete]}
\item[D200 RRKSTP] First-order Differential Equations (Runge-Kutta)
\item[D201 RDEQBS] First-order Differential Equations
(Gragg--Bulirsch--Stoer)
\item[D202 RDEQMR] First-order Differential Equations
(Runge--Kutta--Merson)
\item[D203 RRKNYS] Second-order Differential Equations
(Runge--Kutta--Nystr\"om)
\item[D300 EPDE1] Elliptic Partial Differential Equation
\item[D302 ELPAHY] Fast Partial Differential Equation Solver
\item[D401 RDERIV] Numerical Differentiation
\item[D501 LEAMAX] Constrained Non-Linear Least Squares and Maximum
                   Likelihood Estimation
\item[D503 RMINFC] Minimum of a Function of One Variable
\item[D506 MINUIT] Function Minimization and Error Analysis
\item[D510 FUMILI] Fitting Chisquare and Likelihood Functions
{\bf [Obsolete]}
\item[D601 RFRDH1] Solution of a Linear Fredholm Integral
Equation of Second Kind
\item[D700 RFT] Real Fast Fourier Transform
\item[D702 CFT] Complex Fast Fourier Transform
\item[D705 RFSTFT] Real Fast Fourier Transform
\item[D706 CFSTFT] Complex Fast Fourier Transform
\end{DLtt}
\section*{Interpolation, Approximations, Linear Fitting}
\begin{DLtt}{12345678901}
\item[E100 POLINT] Polynomial Interpolation
\item[E102 MAXIZE] Maximum and Minimum Elements of Arrays
\item[E103 AMAXMU] Largest Absolute Number in Scattered Vector
\item[E104 FINT] Multidimensional Linear Interpolation
\item[E105 DIVDIF] Function Interpolation
\item[E106 LOCATR] Binary Search for Element in Ordered Array
\item[E201 RLSQPM] Least Squares Polynomial Fit
\item[E208 LSQ] Least Squares Polynomial Fit {\bf [Obsolete]}
\item[E210 NORBAS] Polynomial Splines / Normalized B-Splines
\item[E211 RCSPLN] Cubic Splines and their Integrals
\item[E222 RCHEBN] Solution of Overdetermined Linear System in the
Chebychev Norm
\item[E230 TL] Constrained and Unconstrained Linear Least Squares Fitting
\item[E250 LFIT] Least-Squares Fit to Straight Line
\item[E255 PARLSQ] Least-Squares Fit to Parabola {\bf [Obsolete]}
\item[E406 RCHECF] Chebyshev Series Coefficients of a Function
\item[E407 RCHSUM] Summation of Chebyshev Series
\item[E408 RCHPWS] Conversion of Chebyshev to Power and Power to
Chebyshev Series
\item[E409 RTRGSM] Summation of Trigonometric Series
\end{DLtt}
\section*{Matrices, Vectors and Linear Equations}
\begin{DLtt}{12345678901}
\item[F001 LAPACK] Linear Algebra Package
\item[F002 RVADD] Elementary Vector Processing
\item[F003 RMADD] Elementary Matrix Processing
\item[F004 RMMLT] Matrix Multiplication
\item[F010 RINV] Linear Equations, Matrix Inversion
\item[F011 RFACT] Repeated Solution of Linear Equations,
Matrix Inversion, Determinant
\item[F012 RSINV] Symmetric Positive-Definite Linear Systems
\item[F105 POLROT] Rotate a Three-Dimensional Polar Coordinate System
\item[F110 MXPACK] TC Matrix Manipulation Package {\bf [Obsolete]}
\item[F112 TR] Manipulation of Triangular and Symmetric Matrices
\item[F116 DOTI] Scalar Product of Two Space-Time Vectors
\item[F117 CROSS] Vector Product of Two 3-Vectors
\item[F118 ROT] Rotating a 3-Vector
\item[F121 VECMAN] Vector Algebra
\item[F122 SCATTER] Search Operations on Sparse Vectors
\item[F123 BVSL] Bit Vector Manipulation Package
\item[F150 MXDIPR] Direct or Tensor Matrix Product
\item[F406 RBEQN] Banded Linear Equations
\item[F500 RLHOIN] Linear Homogenous Inequalities
\end{DLtt}
\section*{Statistical Analysis and Probability}
\begin{DLtt}{12345678901}
\item[G100 PROB] Upper Tail Probability of Chi-Squared Distribution
\item[G101 CHISIN] Inverse of Chi-Square Distribution
\item[G102 PROBKL] Kolmogorov Distribution
\item[G103 TKOLMO] Kolmogorov Test
\item[G104 STUDIS] Student's T-Distribution and Its Inverse
\item[G105 GAUSIN] Inverse of Gaussian Distribution
\item[G106 GAMDIS] Gamma Distribution
\item[G110 LANDAU] Landau Distribution
\item[G115 VAVLOV] Approximate Vavilov Distribution and its Inverse
\item[G116 VVILOV] Vavilov Density and Distribution Functions
\item[G900 RANF] Random Number Generator {\bf [Obsolete]}
\end{DLtt}
\section*{Operation Research Techniques and Management Science}
\begin{DLtt}{12345678901}
\item[H101 RSMPLX] Linear Optimization Using the Simplex Algorithm
\item[H301 ASSNDX] Assignment Problem
\end{DLtt}
\section*{Input/Output}
\begin{DLtt}{12345678901}
\item[I101 EPIO] EP Standard Format Input/Output Package
\item[I202 KUIP] KUIP - Kit for a User Interface Package
\item[I302 FFREAD] Format-Free Input Processing {\bf [Obsolete]}
\end{DLtt}
\section*{Output and Graphical Data Presentation}
\begin{DLtt}{12345678901}
\item[J200 VIZPRI] Print Large Characters
\item[J530 BINSIZ] Reasonable Intervals for Histogram Binning
\end{DLtt}
\section*{Executive Routines}
\begin{DLtt}{12345678901}
\item[L210 COMIS] COMIS - Compilation and Interpretation System
\item[L400 PATCHY] Source Code Maintenance
\end{DLtt}
\section*{Data Handling}
\begin{DLtt}{12345678901}
\item[M101 SORTZV] Sort One-Dimensional Array
\item[M103 FLPSOR] Sort One-Dimensional Array into Itself
\item[M104 SORCHA] Sort One-Dimensional Character Array into Itself
\item[M107 SORTR] Sort Rows of a Matrix
\item[M109 SORTRQ] Sort Rows of a Matrix
\item[M215 PSCALE] Find Power-of-Ten Scale for Printing
\item[M220 IE3CONV] Conversion To and From IEEE Number Format
\item[M400 CHTOI] Portable Conversion Between Type CHARACTER
and Type INTEGER
\item[M409 UBUNCH] Concentrate and Disperse Character Strings
{\bf [Partially obsolete]}
\item[M421 BITBYT] Package for Handling Bits and Bytes
\item[M422 PACBYT] Handling Packed Vectors of Bytes
\item[M423 INCBYT] Increment a Byte of a Packed Vector
\item[M426 BLOW] Unpack Full Words into Bytes
\item[M427 PKCHAR] Pack/Unpack Continuous Byte-strings
\item[M428 LOCBYT] Search for Byte-Content
\item[M429 NUMBIT] Number of One-Bits in a Word
\item[M431 IFROMC] Convert Between Character String and Packed
ASCII Form
\item[M432 CHPACK] Utility Routines for Character String Parsing
and Construction
\item[M433 INDEXX] Utility Package for Character Manipulation
\item[M434 VXINV] Fast VAX Byte Inversion
\item[M436 BUNCH] Pack Bytes into Full Words
\item[M437 GETBIT] Set or Retrieve a Bit in a String
\item[M438 BTMOVE] Move Bit String
\item[M439 GETBYT] Set or Retrieve a Bit String
\item[M441 BITPAK] Handling Bits and Bytes, Bit Zero the Least
Significant
\item[M442 NAMEFD] Fortran Emulation of VM/CMS NAMEFIND Command
\item[M501 IUSAME] Locating a String of Same Words
\item[M502 UOPTC] Decoding Options Characters
\item[M503 UBITS] Locate the One-Bits of a Word or an Array
\item[M507 LENOCC] Occupied Length of a Character String
\item[M508 BITPOS] Find One-Bits in a String
\end{DLtt}
\section*{Debugging, Error Handlng}
\begin{DLtt}{12345678901}
\item[N001 KERSET] Error Processing for Sections A-H of KERNLIB
{\bf [Partially obsolete]}
\item[N002 MTLSET] Error Processing for MATHLIB
\item[N100 LOCF] Address of a Variable
\item[N103 IUWEED] Detect Indefinite and Infinite in an Array
\item[N105 TRACEQ] Print Trace-Back
\item[N203 TCDUMP] Memory Dump
\end{DLtt}
\section*{Service or Housekeeping Programming Aids}
\begin{DLtt}{12345678901}
\item[Q100 ZEBRA] Dynamic Data Structure and Memory Manager
\item[Q120 HIGZ] High Level Interface to Graphics and Zebra
\item[Q121 PAW] PAW - Physics Analysis Workstation Package
\item[Q122 SIGMA] SIGMA - System for Interactive Graphical
Mathematical Applications
\item[Q123 FATMEN] Distributed File and Tape Management System
\item[Q124 CSPACK] Client Server Routines and Utilities
\item[Q180 HEPDB] Distributed Database Management System
\item[Q210 ZBOOK] Dynamic Memory Management {\bf [Obsolete]}
\item[Q901 INDENT] Indent Fortran Source
\item[Q902 FLOP] FLOP - Fortran Language Oriented Parser
\item[Q904 CONVERT] Fortran 77 to Fortran 90 source form conversion tool
\item[Q905 WYLBUR] Wylbur Phoenix - a Line Editor for ASCII Text Files
                   \textbf{[Obsolete]}
\end{DLtt}
\section*{Magnet and Beam Design, Electronics}
\begin{DLtt}{12345678901}
\item[T604 POISCR] Solution of Poisson's or Laplace's Equation in
Two-Dimensional Regions
\end{DLtt}
\section*{Quantum Mechanics, Particle Physics}
\begin{DLtt}{12345678901}
\item[U101 LOREN4] Lorentz Transformation
\item[U102 LORENF] Lorentz Transformations
\item[U111 RWIG3J] Wigner 3-j, 6-j, 9-j Symbols; Clebsch-Gordan,
Racah W-, Jahn U-Coefficients
\item[U112 RTCLGN] Clebsch-Gordan Coefficients in Rational Form
\item[U501 RDJMNB] Beta-Term in Wigner's D-Function
\end{DLtt}
\section*{Random Numbers and General Purpose Utilities}
\begin{DLtt}{12345678901}
\item[V104 RNDM] Uniform Random Numbers {\bf [Obsolete]}
\item[V105 NRAN] Arrays of Uniform Random Numbers {\bf [Obsolete]}
\item[V113 RANMAR] Uniform Random Number Generator
\item[V114 RANECU] Uniform Random Number Generator
\item[V115 RANLUX] Uniform Random Numbers of Guaranteed Quality
\item[V116 RM48]   Double Precision Uniform Random Numbers
\item[V120 RNORML] Gaussian-distributed Random Numbers
\item[V122 CORSET] Correlated Gaussian-distributed Random Numbers
\item[V130 RAN3D]  Random Three-Dimensional Vectors {\bf [Obsolete]}
\item[V131 RN3DIM] Random Three-Dimensional Vectors
\item[V135 RNGAMA] Gamma or Chi-Square Random Numbers
\item[V136 RNPSSN] Poisson Random Numbers
\item[V137 RNBNML] Binomial Random Numbers
\item[V138 RNMNML] Multinomial Random Numbers
\item[V149 RNHRAN] Random Numbers According to Any Histogram
\item[V150 HISRAN] Random Numbers According to Any Histogram
{\bf [Obsolete]}
\item[V151 FUNRAN] Random Numbers According to Any Function
{\bf [Obsolete]}
\item[V152 FUNLUX] Random Numbers According to Any Function
\item[V202 PERMU] Permutations and Combinations
\item[V300 UZERO] Preset Parts of an Array
\item[V301 UCOPY] Copy an Array
\item[V302 UCOCOP] Copy a Scattered Vector
\item[V304 IUCOMP] Search a Vector for a Given Element
\item[V306 PROXIM] Adjusting an Angle to Another Angle
\item[V401 GRAPH] Find Compatible Node-Nets in an Incompatibility Graph
\item[V700 RVNSPC] Volume of Intersection of a Circular Cylinder
with a Sphere
\end{DLtt}
\section*{High Energy Physics Simulation, Kinematics, Phase Space}
\begin{DLtt}{12345678901}
\item[W150 TRSPRT] Transport, Second-Order Beam Optics
\item[W151 TURTLE] Beam Transport Simulation, Including Decay
\item[W505 FOWL] General Monte-Carlo Phase-Space
\item[W515 GENBOD] N-Body Monte-Carlo Event Generator
\end{DLtt}
\section*{Statistical Data Analysis and Presentation}
\begin{DLtt}{12345678901}
\item[Y201 IUCHAN] Find Histogram-Channel
\item[Y250 HBOOK] Statistical Analysis and Histogramming
\item[Y251 HPLOT] HPLOT : HBOOK Graphics Interface for Histogram
Plotting
\end{DLtt}
\section*{Miscellaneous System-Dependent Facilities}
\begin{DLtt}{12345678901}
\item[Z001 KERNGT] Print KERNLIB Version Numbers
\item[Z007 DATIME] Job Time and Date
\item[Z009 CALDAT] Calendar Date Conversion
\item[Z020 UMON] Usage Monitor for VAX/VMS
\item[Z035 ABEND] Abnormal Termination of Fortran Programs
\item[Z036 ABUSER] Intercept a Fortran Abend on IBM
\item[Z037 VAXAST] Routines to Handle Control-C Interrupts on Vax
\item[Z041 QNEXTE] Restart of Next Event
\item[Z042 JUMPXN] Calling a Subroutine by its Address {\bf [Obsolete]}
\item[Z044 INTRAC] Identify Job as Interactive
\item[Z045 IFBATCH] Identify Job as Running in Batch Mode
\item[Z203 XINOUT] Short List Reading and Writing
\item[Z264 IARGC] Returns Command Line Arguments
\item[Z265 CINTF] Immediate Interface Routines to the C Library
\item[Z266 WHOAMI] Get the Name of the Executing Module
\item[Z267 FTOVAX] Convert File-name to and from UNIX Syntax
\item[Z301 VAXTIO] VAX Fortran Interface for Reading and
Writing 'Foreign' Tapes
\item[Z303 KAPACK] Random Access I/O Using Keywords {\bf [Obsolete]}
\item[Z310 CFIO] Handle Fixed-length Records on Unix Streams
\item[Z311 CIO] Handle Unix Disk Files
\item[Z313 TMREAD] Terminal Dialog Routines
\end{DLtt}


\section{MATHLIB}

\MATHLIB{} is a Fortran callable library containing routines
of a mathematical nature. It is made up of four components,
namely

\begin{itemize}
\item
BVSL - basic vector subroutine library. These routines
were originally written for the IBM 3090 vector facility.
However, a Fortran version is now available.
\item
GEN - general mathematical routines.
\item
LAPACK - the well-known linear algebra package.
\item
MPA - multiple precision floating point arithmetic routines.
\end{itemize}

Note that LAPACK and MPA may only be distributed in object form
by CERN. Further details can be found in section \ref{sect-LAPACK}
and \ref{sect-MPA}.

\section{PACKLIB}

\PACKLIB{} is a Fortran callable library made up of complete
packages. A few of the packages, such as VMIO, are
machine dependent. The bulk, however, run on a variety
of systems including Unix, MS-DOS, Windows NT, VMS, VM/CMS
and MVS.

Long writeups exist for each of these packages and these
should be consulted for further details.

% Link to appropriate long writeup?

\begin{itemize}
\item
EPIO - EP standard input/output package
\item
FFREAD - processing of Free Format data cards
\item
IOPACK - Input/output package for VM/CMS and MVS systems
\item
KAPACK - Key access package
\item
KUIP - Kit for a User interface package
\item
HBOOK - Histogramming, statistical analysis and Ntuple package
\item
MINUIT - Function minimization package
\item
VMIO - VM/CMS specific I/O package
\item
ZBOOK - Data structure management package
\item
ZEBRA - Data structure management package
\item
CDLIB - The API for the HEPDB detector geometry and calibration system
\item
FATLIB - The API for the FATMEN distributed file management system
\end{itemize}


\section{The graphics libraries}

The graphics libraries are divided into a kernel and
driver specific libraries. Examples are shown below.

\begin{itemize}
\item
GRAFLIB - the graphics kernel
\item
GRAFGKS - the GKS binding
\item
GRAFGDDM - the GDDM binding
\item
GRAFX11 - the X11 binding
\end{itemize}

Some of the above libraries may not be available
on certain platforms.

\section{PAWLIB}

\PAWLIB{} is the library used to build the various
\PAW{} modules. In addition to \PAW{} itself, (apart
from the main program), it also contains
\COMIS{} and \SIGMA{}.

\section{GEANT}

\GEANT{} is provided in a standalone library. The version
number is contained in the library name.

\section{The Monte Carlo libraries}

The Monte Carlo libraries provide numerous event
generators, such as the well-know Lund Monte Carlos,
including those listed below. Again, the long writeup
should be consulted for further details.

% Link?

\begin{itemize}
\item
COJETS - generator for high energy proton-proton and
proton-antiproton collisions
\item
EURODEC - generator for high energy processes
\item
HERWIG - generator for hadron collisions
\item
ISAJET - the BNL generator for high energy proton-proton and
proton-antiproton collisions
\item
JETSET - Lund Monte Carlo for jet fragmentation and
e+e- anihilation
\item
PYTHIA - Lund Monte Carlo for high Pt hadron-hadron scattering
\item
LEPTO - Lund Monte Carlo for deep inelastic lepton-nucleon scattering
\item
PHOTOS - Monte Carlo for generation of radiative
corrections in decays
\item
PDFLIB - Parton Density Functions
\item
TWISTER - Monte Carlo for QCD high Pt scattering
\item
FRITIOF - hadron-hadron, hadron-nucleus, nucleus-nucleus interactions
\item
ARIADNE - Lund Monte Carlo for QCD cascades in the Colour Dipole approximation
\end{itemize}

\section{The CERNLIB modules}

The CERNLIB modules consist of complete standalone programs,
the most famous of which is surely \PAW{}. A brief
description of the various modules for a given system
is given in the system dependent installation section
of this manual.
