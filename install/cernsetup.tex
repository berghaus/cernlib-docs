\chapter{The CERNLIB directory structure}
The CERNLIB directory structure is basically the same
for Unix, VMS and VM/CMS systems. We describe below the
structure for Unix systems and note the differences for
VMS and VM/CMS systems. 

\section{The CERNLIB tree}

The directory tree for CERNLIB files begins at {\bf /cern}.
This may be a filesystem or a link.
Below this top level directory are a number of subdirectories
which fall into one of three categories:

\begin{itemize}
\item
Packages, e.g. {\tt CMZ, PATCHY} or {\tt WWW}
\item
Directories relating to a specific release of the program
library, e.g. {\tt 94a}
\item
Links to specific releases, as shown below.
\end{itemize}

\begin{XMPt}{The {\bf /cern} directory tree}
zfatal:/cern (76) ls /cern
93c     94a     cmz     new     patchy  pro
93d     WWW     mad     old     phigs
#
#
zfatal:/cern (80) ls -l      
total 24
drwxrwxr-x   7 cpl-main asis         512 Jul 05 22:32 93c
drwxrwxr-x   6 cpl-main asis         512 Nov 10 18:13 93d
drwxr-xr-x   4 cernlib  asis         512 Oct 21 03:53 94a
drwxrwxr-x   3 cpl-gean asis         512 Feb 23 1993  WWW
drwxrwxr-x   5 cpl-hist asis         512 Oct 28 09:14 cmz
drwxrwxr-x   4 8120     asis         512 Nov 08 1992  mad
lrwxrwxrwx   1 cpl-main asis           3 Jun 14 13:16 new -> 93d
lrwxrwxrwx   1 cpl-main asis           3 Sep 23 09:02 old -> 93c
drwxrwxr-x   3 8117     asis         512 Nov 26 1992  patchy
drwxrwxr-x   4 cpl-main asis         512 Jun 10 15:24 phigs
lrwxrwxrwx   1 cernlib  asis           3 Oct 28 14:14 pro -> 93d
\end{XMPt}
\section{VMS specific details}
On VMS systems, the logical name {\bf CERN} points to the
root directory for the CERNLIB tree, as shown below.
\begin{XMPt}{The {\bf CERN} logical name}
VXCRNA? sh log cern
   "CERN" = "_$22$DUS206:[CERN.]" (LNM$SYSTEM_TABLE)
\end{XMPt}

This directory contains subdirectories as in the Unix case,
as shown below.
\begin{XMPt}{CERNLIB subdirectory structure}
VXCRNA? dir $22$dus206:[cern]
Directory $22$DUS206:[CERN]
000000.DIR;1        92B.DIR;1           93C.DIR;1           93D.DIR;1
94A.DIR;1           CMZ.DIR;1           DECW.DIR;1          FATMEN.DIR;1
GKS.DIR;1           HEPDB.DIR;1         HISTORIAN.DIR;1     HYDRA.DIR;1
LAPACK.DIR;1        MAD.DIR;1           NAG.DIR;1           NEW.DIR;1
OLD.DIR;1           PATCHY.DIR;1        PHIGS.DIR;1         PRO.DIR;1
RELEASE.LEVEL;8     WWW.DIR;1
\end{XMPt}

\subsection{VXCERN specific details}
\label{sect-VXCERN}

\subsubsection{The {\bf CERN} logical name}

On VXCERN, the logical name {\bf CERN} is a search list. It is composed
of three individual logical names, as follows:

\begin{DLtt}{1234567890}
\item[CERNAXP]The Alpha specific tree.
\item[CERNVAX]The VAX specific tree.
\item[CERNNFS]The {\bf CERN} subdirectory of the asis system 
independant tree {\tt /asis/share}, NFS mounted.
\end{DLtt}

\begin{XMPt}{Defining the CERN logical names on VXCERN}
$!
$! CERN library tree is on DISK$CERNLIB volume set
$!
$ if f$logical("disk$cernlib").nes.""
$ then
$   cernlib_disk = F$getdvi("disk$cernlib","FULLDEVNAM")
$   def/sys/exec/tran=(term,conc) cernaxp 'cernlib_disk'[cernaxp.]
$   def/sys/exec/tran=(term,conc) cernvax 'cernlib_disk'[cern.]
$ endif
$!
$! Now mount the /asis/share file system
$!
$ nfsmount/soft asisnfs:"/asis/share"                 ASIS_SHARE
$ nfsdev=F$GETDVI("ASIS_SHARE","FULLDEVNAM")
$ If nfsdev.nes."" Then -
     assign/sys/exec/trans=(conc,term) 'nfsdev'[cern.] NFSCERN
$ If nfsdev.nes."" Then -
     assign/sys/exec/trans=(conc,term) 'nfsdev'[cern.] CERNNFS
$ Endif
$!
$! Now define the CERNlib logical names as a function of the above
$!
$ @cern:[new.mgr]cern\_logicals

\end{XMPt}

At the time of writing, some directories exist only in the {\bf CERNVAX} tree.
However, it is planned that all files shared between the {\bf VAX} and {\bf Alpha}
architectures be moved progressively to the asis server.

{\bf N.B. the binaries and libraries are {\it incompatible} between VAX and Alpha
systems.} Thus, if you wish to copy the VAX executables over DECnet, use

{\bf COPY VXCERN::CERNVAX:[PRO.EXE]*.*}

Similarly, to get the Alpha versions of the libraries, use

{\bf COPY VXCERN::CERNAXP:[PRO.LIB]*.*}

\subsubsection{The subdirectory structure}

On VXCERN, the \NEW{}, \PRO{} and \OLD{} subdirectories
are {\it pointers} to other subdirectories, created using the
VMS command {\bf SET FILE/ENTER}.

\subsubsection{Symbol definitions for CERNLIB products}

On VXCERN, much of the system wide login procedure is performed using
{\bf CLUSTER\$MANAGER:EXSYLOGIN.EXE}. This is done to
speed up logins. The symbols definitions for the CERNLIB commands
are defined in the program {\bf CERN\_ROOT:[MGR]CERNLOGIN.FOR}.
You may either include this file into your own Fortran file
or use the file {\bf MAIN\_CERNLOGIN.FOR}.

\begin{XMPt}{MAIN\_CERNLOGIN.FOR}

      IMPLICIT INTEGER(A-Z)
      INCLUDE '($LIBCLIDEF)'
      CHARACTER*(*) C_EXE
      PARAMETER(C_EXE='CERN_ROOT:[EXE]')
C
      I=LIB$K_CLI_GLOBAL_SYM
C
      INCLUDE 'CERN_ROOT:[MGR]CERNLOGIN.FOR'
C 
      END

\end{XMPt}

\begin{XMPt}{Symbol definitions on VXCERN using CERNLOGIN.FOR}

C      IMPLICIT INTEGER(A-Z)
C      INCLUDE '($LIBCLIDEF)'
C      CHARACTER*(*) C_EXE
C      PARAMETER(C_EXE='CERN_ROOT:[EXE]')
C
C THIS IS A FAST VERSION OF THE USUAL SYSTEM WIDE LOGIN.COM .
C ALL CERN LIBRARY SYMBOL DEFINITIONS ARE CODED HERE.
C
C ====================================================================
C Customization section: enable if product is available at your site
C
      f_cmz  =1                ! CMZ       - CODEME
      f_garf =1                ! GARFIELD  - CERN
      f_gks  =1                ! GKS       - GTS-GRAL   
      f_his  =1                ! HISTORIAN - OPCODE
      f_mad  =1                ! MAD       - CERN
      f_nag  =1                ! NAG       - NAG Ltd
      f_patc =1                ! PATCHY    - CERN
      f_rzco =1                ! RZCONV    - CERN
      f_umon =1                ! UMON      - CERN
      f_vaxt =1                ! VAXTAP    - CERN
      f_www  =1                ! WWW       - CERN
C
C End of customization
C ====================================================================
C
C     I=LIB$K_CLI_GLOBAL_SYM
C
      RECODE=LIB$SET_SYMBOL('CERNLIB','$'//C_EXE//'CERNLIB' ,I)
      RECODE=LIB$SET_SYMBOL('CERN_LEVEL','PRO'                  ,I)
C
C---  PATCHY Symbols
C
      IF(F_PATC.EQ.1) THEN
        RECODE=
     +  LIB$SET_SYMBOL('FCASPLIT','$CERN:[patchy.4_15.EXE]FCASPLIT',I)
C
        RECODE=
     +  LIB$SET_SYMBOL('YCOM*PAR','$CERN:[patchy.4_15.EXE]YCOMPAR',I)
        RECODE=
     +  LIB$SET_SYMBOL('YEDI*T'  ,'$CERN:[patchy.4_15.EXE]YEDIT'  ,I)
        RECODE=
     +  LIB$SET_SYMBOL('YFRC*ETA','$CERN:[patchy.4_15.EXE]YFRCETA',I)
        RECODE=
     +  LIB$SET_SYMBOL('YIND*EX' ,'$CERN:[patchy.4_15.EXE]YINDEX' ,I)
        RECODE=
     +  LIB$SET_SYMBOL('YLIS*T'  ,'$CERN:[patchy.4_15.EXE]YLIST'  ,I)
C 
        RECODE=
     +  LIB$SET_SYMBOL('YPAT*CHY','$CERN:[patchy.4_15.EXE]YPATCHY',I)
        RECODE=
     +  LIB$SET_SYMBOL('YSEA*RCH','$CERN:[patchy.4_15.EXE]YSEARCH',I)
        RECODE=
     +  LIB$SET_SYMBOL('YSHI*FT' ,'$CERN:[patchy.4_15.EXE]YSHIFT' ,I)
        RECODE=
     +  LIB$SET_SYMBOL('YTOBC*D' ,'$CERN:[patchy.4_15.EXE]YTOBCD' ,I)
        RECODE=
     +  LIB$SET_SYMBOL('YTOBI*N' ,'$CERN:[patchy.4_15.EXE]YTOBIN' ,I)
        RECODE=
     +  LIB$SET_SYMBOL('YTOC*ETA','$CERN:[patchy.4_15.EXE]YTOCETA',I)
C 
        RECODE=LIB$SET_SYMBOL('YEXP*AND','@'//C_EXE//'YEXPAND',I)
C 
      ENDIF
C
C---  CERNlib Tools
C
      RECODE=LIB$SET_SYMBOL('GETOPT','@'//C_EXE//'GETOPT'   ,I)
      RECODE=LIB$SET_SYMBOL('SETENV','@'//C_EXE//'SETENV " "',I)
      RECODE=LIB$SET_SYMBOL('XTERM' ,'@'//C_EXE//'XTERM " "',I)
      RECODE=LIB$SET_SYMBOL('RESIZE','@'//C_EXE//'RESIZE'   ,I)
      RECODE=LIB$SET_SYMBOL('MKCOMP*ILE','@'//C_EXE//'MKCOMPILE',I)
C
C---  CERNlib Products
C
      RECODE=LIB$SET_SYMBOL('DZEDIT','@'//C_EXE//'DZEDIT " "',I)
      RECODE=LIB$SET_SYMBOL('FM','$'//C_EXE//'FATMEN'       ,I)
      RECODE=LIB$SET_SYMBOL('HEPDB','$'//C_EXE//'HEPDB'     ,I)
      RECODE=LIB$SET_SYMBOL('GXINT','@'//C_EXE//'GXINT " "' ,I)
      RECODE=LIB$SET_SYMBOL('HIGZCONV','$'//C_EXE//'HIGZCONV',I)
      RECODE=LIB$SET_SYMBOL('KUIPC','$'//C_EXE//'KUIPC'     ,I)
      RECODE=LIB$SET_SYMBOL('PAW' ,'@'//C_EXE//'PAW.COM " "',I)
      RECODE=LIB$SET_SYMBOL('HTONEW','$'//C_EXE//'HTONEW'   ,I)
      RECODE=LIB$SET_SYMBOL('FATSREQ','$'//C_EXE//'FATSREQ' ,I)
      RECODE=LIB$SET_SYMBOL('SYSREQ','$'//C_EXE//'SYSREQ'   ,I)
      RECODE=
     +LIB$SET_SYMBOL('TELNETG','@'//C_EXE//'TELNETG.COM " "',I)
      RECODE=LIB$SET_SYMBOL('ZFTP','$'//C_EXE//'ZFTP'       ,I)
      RECODE=LIB$SET_SYMBOL('WYLBUR','$'//C_EXE//'WYLBUR'   ,I)
      RECODE=LIB$SET_SYMBOL('USE','$'//C_EXE//'WYLBUR'      ,I)
C
      RECODE=LIB$SET_SYMBOL('TO','$'//C_EXE//'NEWTONET.EXE' ,I)
C
C---  CMZ symbols
C
      IF(F_CMZ.EQ.1) THEN
      RECODE=LIB$SET_SYMBOL('CMZ' ,'@'//C_EXE//'CMZ.COM " "',I)
      ENDIF
C
C---  GARFIELD symbols
C
      IF(F_GARF.EQ.1) THEN
      RECODE=LIB$SET_SYMBOL('GA*RFIELD','$'//C_EXE//'GARFRUN',I)
C     RECODE=LIB$SET_SYMBOL('GH*ELP',
C    +                      'HELP/NOLIBLIST/LIBRARY=HELP$GARFIELD',I)
      RECODE=LIB$SET_SYMBOL('GED*IT','LSE/ENV=LSE$GARFIELD',I)
      ENDIF
C
C---  GKS symbols
C
C     IF(F_GKS.EQ.1) THEN
C
C     ENDIF
C
C---  HISTORIAN symbols
C
      IF(F_HIS.EQ.1) THEN
      RECODE=LIB$SET_SYMBOL('HISTE','$HISTORIAN_ROOT:[EXE]HISTE',I)
      RECODE=LIB$SET_SYMBOL('HISTG','$HISTORIAN_ROOT:[EXE]HISTG',I)
      RECODE=LIB$SET_SYMBOL('HISTO*R','$HISTORIAN_ROOT:[EXE]HISTOR',I)
      ENDIF
C
C---  MAD symbols
C
      IF(F_MAD.EQ.1) THEN
      RECODE=LIB$SET_SYMBOL('MAD8','@'//C_EXE//'MAD8.COM " "',I)
      ENDIF
C
C---  NAG symbols
C
      IF(F_NAG.EQ.1) THEN
C     RECODE=LIB$SET_SYMBOL('NAGH*ELP','$NAG_ROOT:[EXE]NAGHELP'    ,I)
      RECODE=LIB$SET_SYMBOL('NAGT*EST','@NAG_ROOT:[EXE]NAGTEST " "',I)
      ENDIF
C
C---  RZCONV symbols
C
      RECODE=LIB$SET_SYMBOL('RTOA','$'//C_EXE//'RTOA',I)
      RECODE=LIB$SET_SYMBOL('RTOX','$'//C_EXE//'RTOX',I)
      RECODE=LIB$SET_SYMBOL('RFRA','$'//C_EXE//'RFRA',I)
      RECODE=LIB$SET_SYMBOL('RFRX','$'//C_EXE//'RFRX',I)
C
C---  VAXTAP symbols
C
      IF(F_VAXT.EQ.1) THEN
       RECODE=LIB$SET_SYMBOL('EINIT'    ,'$'//C_EXE//'EINIT'    ,I)
       RECODE=LIB$SET_SYMBOL('LABELDUMP','$'//C_EXE//'LABELDUMP',I)
       RECODE=LIB$SET_SYMBOL('SETUP'    ,'$'//C_EXE//'SETUP'    ,I)
       RECODE=LIB$SET_SYMBOL('STAGE'    ,'$'//C_EXE//'STAGE'    ,I)
       RECODE=LIB$SET_SYMBOL('TAPECOPY' ,'@'//C_EXE//'TAPECOPY' ,I)
       RECODE=LIB$SET_SYMBOL('WRTAPE'   ,'$'//C_EXE//'WRTAPE'   ,I)
       RECODE=LIB$SET_SYMBOL('XTAPE'    ,'$'//C_EXE//'XTAPE'    ,I)
      ENDIF 
C
C---  WWW symbols
C
      IF(F_WWW.EQ.1) THEN
      RECODE=LIB$SET_SYMBOL('WWW'     ,'$CERN:[WWW.EXE]WWW'        ,I)
      RECODE=LIB$SET_SYMBOL('WEB'     ,'@CERN_ROOT:[EXE]WEB " "'   ,I)
C     RECODE=LIB$SET_SYMBOL('XMOSAIC' ,'@CERN:[WWW.EXE]XMOSAIC'    ,I)
      ENDIF
C 
\end{XMPt}

\section{VM/CMS specific details}

\subsection{CERNVM specific details}
\label{sect-CERNVM}
\chapter{Initial setup and configuration for Unix systems}
 
The following steps must be followed before installing CERNLIB
software on Unix systems.

\begin{enumerate}
\item
Create a directory for the CERN software
\item
Install the \PATCHY{} modules. For further information,
see chapter \ref{sect-UNIXPATCHY}
\item
Install the installation procedures. For further information,
see chapter \ref{sect-UNIXINSTALL}
\item
Define the appropriate environmental variables. For further
information, see chapter \ref{sect-UNIXINSTALL}.
\end{enumerate}

\chapter{Initial setup and configuration for VMS systems}

\label{sect-VMSSETUP}

The following steps must be followed before installing CERNLIB
software on VMS systems.

\begin{enumerate}
\item
Create a directory for the CERN software
\item
Install the \PATCHY{} modules
\item
Install the installation command files
\item
Define the appropriate symbols
\end{enumerate}

\section{Installing PATCHY}

The \PATCHY{} modules are stored in the {\bf BACKUP} saveset 
{\bf VXCERN::CERN:[PRO.BCK]CRNPAT.BCK}, or on asis in 
the backup directory for a given release, e.g. 
{\bf cernlib/vax\_vms/94a/bck.} We can either retrieve
and unpack this saveset or install \PATCHY{} from scratch
as shown below.

\index{installing PATCHY}
\index{PATCHY, installation of}

\begin{XMPt}{Installing PATCHY from scratch}

VSCLIB? create/directory [.patchy]

VSCLIB? create/directory [.scratch] ! work directory for patchy installation

VSCLIB? set default [.patchy]

VSCLIB? copy vxcern::cernvax:[patchy.src.vaxvms]*.* *

VSCLIB? edit/edt patchy.com  ! modify the source, work and target directories as appropriate

VSCLIB? @patchy

\end{XMPt}

{\bf N.B. Ensure that you modify all lines with <<< MODIFY THIS >>> on
them in the example below.}

\begin{XMPt}{PATCHY.COM command file}

$!******************************************************************   
$!*                                                                *   
$!*  Patchy installation job                                       *   
$!*                                                                *   
$!******************************************************************
$!   
$!--- Customization section ----------------------------------------
$! SDIR="CERN:[PATCHY.SRC.VAXVMS]"     ! Patchy source directory
$! WDIR="DISK$SCRATCH:[PUBCR.WORK]"    ! working directory
$  sdir="disk$user1:[jamie.patchy]"    ! <<< MODIFY THIS >>>
$  wdir="disk$user1:[jamie.scratch]"   ! <<< MODIFY THIS >>>
$!------------------------------------------------------------------
$!
$ ver_imag=F$ENVIRONMENT("verify_image")
$ ver_def =F$ENVIRONMENT("DEFAULT")
$ ver_proc=F$VERIFY(0)
$
$ SET DEFAULT 'WDIR'
$
$ If F$SEARCH("P4INCETA.CET").eqs."" Then COPY 'SDIR'P4INCETA.CET *.*
$!-------------
$ If p1.nes."R" 
$ Then
$   @'SDIR'rceta.sh
$   differ p4boot.com 'SDIR'p4boot.sh0
$   @p4boot 0
$ Else
$   copy 'SDIR'p4boot.sh0 p4boot.com
$   @p4boot 1    
$   @p4boot 2
$ Endif
$!----------------------------------------------------------------------
$ copy/log   Y*.EXE   CERN_ROOT:[EXE] ! <<< MODIFY THIS >>>
$ delete/log Y*.EXE;*,RCETA.*;*
$!
$ dummy=F$VERIFY(ver_proc,ver_imag)
$ SET DEFAULT 'ver_def'
$ Exit

\end{XMPt}

\section{Creating directories for CERNLIB}

The following example shows how one might setup the CERNLIB 
tree on a new system. 

\begin{XMPt}{Example of directory setup for installing the CERNLIB software}

VSCLIB? show default

  DISK$USER1:[CERNLIB]

VSCLIB? create/directory [.cern]

VSCLIB? show logical disk$user1

   "DISK$USER1" = "_UXCNB1$DUA20:" (LNM$SYSTEM_TABLE)

VSCLIB? def/job/tran=(conc,term) cern _uxcnb1$dua20:[cernlib.cern.]

VSCLIB? create/directory cern:[cmz]              ! If you wish to install CMZ

VSCLIB? set default cern:[new]

VSCLIB? create/directory cern:[new.src]          

VSCLIB? create/directory cern:[new.src.car]      ! For the source files

VSCLIB? create/directory cern:[new.mgr]          ! for the installation command files

VSCLIB? create/directory cern:[new.exe]          ! required for GETOPT, SETENV etc.

VSCLIB? create/directory cern:[new.lib]          ! for OLBs and link options files

\end{XMPt}

\section{Retrieving the installation command files}

We now retrieve the installation files as shown below.

\begin{XMPt}{Retrieving the installation files}

VSCLIB? set default cern:[new.mgr]

VSCLIB? copy vxcern::cern:[new.mgr]*.com; *

VSCLIB? copy vxcern::cern:[new.mgr]*.for; *

VSCLIB? copy vxcern::cern:[new.mgr]*.c; *

VSCLIB? set default [-.exe]

VSCLIB? copy vxcern::cern:[new.exe]*.com; *

VSCLIB? set default cern:[new.lib]

VSCLIB? copy vxcern::cern:[new.lib]*.opt; *

VSCLIB? create/directory cern:[decw] ! If you want to link PAW with the old version
                                     ! of DECwindows
VSCLIB? set default cern:[decw]

VSCLIB? copy vxcern::cernvax:[decw]*.exe *

\end{XMPt}

We now compile and link the program {\bf MAIN\_CERNLOGIN.FOR} and customise our {\bf LOGIN.COM}.

\begin{XMPt}{Creating MAIN\_CERNLOGIN.EXE}

VSCLIB? fortran cern:[new.mgr]main_cernlogin

VSCLIB? link main_cernlogin

\end{XMPt}

Before running this program, we must define the logical 
name {\bf CERN\_ROOT}. Note that CERN\_ROOT should be defined
consistantly with the CERN logical name. In a production
environment, CERN\_ROOT is typically the PRO subdirectory
of the CERN tree. When installing, however, it is wise to
point CERN\_ROOT to the NEW area, as shown below.
 
\begin{XMPt}{Defining CERN\_ROOT}

VSCLIB? define/sys/exec/trans=(concealed,terminal) -
cern_root -
disk$user1:[jamie.cernlib.cern.new.]

\end{XMPt}

\begin{XMPt}{Example LOGIN.COM for the CERNLIB account}

$ define cern _uxcnb1$dua20:[cernlib.cern.] /trans=(conc,term)
$ run cern:[new.mgr]main_cernlogin
$!
$! The following symbols are defined by
$! MAIN_CERNLOGIN but we chose to keep
$! these modules in a different place
$!
$ ycompare :== $cern:[new.exe]ycompare
$ yedit    :== $cern:[new.exe]yedit
$ yfrceta  :== $cern:[new.exe]yfrceta
$ yindex   :== $cern:[new.exe]yindex
$ ylist    :== $cern:[new.exe]ylist
$ ypatchy  :== $cern:[new.exe]ypatchy
$ ysearch  :== $cern:[new.exe]ysearch
$ yshift   :== $cern:[new.exe]yshift
$ ytobcd   :== $cern:[new.exe]ytobcd
$ ytobin   :== $cern:[new.exe]ytobin
$ ytoceta  :== $cern:[new.exe]ytoceta
$!
$ yexp*and :== @cern:[new.exe]yexpand
$!
$ fcasplit :== $cern:[new.exe]fcasplit
$!
$ cern_level :== new
$!
$ @cern:[new.mgr]plienv

\end{XMPt}

\section{Retrieving the source files}

We can retrieve the source files as follows.

\begin{XMPt}{Retrieving the source files}

VSCLIB? set default cern:[new.src.car]

VSCLIB? copy vxcern::cern:[new.src.car]*.c%% *  ! .cra and .car files

\end{XMPt}

If we are in DECnet areas 22 or 23, the DECnet areas
reserved for CERN, the installation procedures
will automatically access the sources through VXCERN.

\section{Final configuration}

We are now ready to begin the CERNLIB installation.
If there has been no previous installation of CERNLIB
on this system, we must procede as follows:

\index{gethostname}

\begin{enumerate}
\item
Compile {\bf GETHOSTNAME.C} and place the object file in
{\bf CERN:[NEW.LIB]}.
\item
Prior to building {\bf PACKLIB}, we must build
\begin{DLtt}{1234567890}
\item[CERNLIB]The {\bf CERNLIB} command. Built by typing {\underline {\bf make cernlib}}.
\item[FCASPLIT]A PATCHY auxiliary. Built by typing {\underline {\bf make fcasplit}}.
\item[KUIPC]The KUIP compiler. Built by typing {\underline {\bf make kuipc}}.
\end{DLtt}
\end{enumerate}

For further information, see chapter \ref{sect-VMSINSTALL}.

\chapter{Initial setup and configuration for VMS systems when {\it asisnfs} is accessible}

For systems at CERN, it may be more appropriate to access the source
files over \NFS{} from asisnfs. In this case, we procede as above
{\it except} that we do not need to copy the {\bf .cra} and {\bf .car}
files from VXCERN or asisftp. Instead, we modify our {\bf LOGIN.COM}
as shown below to NFS mount the appropriate asis file system.

\begin{XMPt}{CERNLIB LOGIN.COM to access sources over NFS from asisnfs}
$!
$! Command file to define environment for CERNLIB installation
$! on AXCLIB (Alpha system)
$!
$! N.B. before running this command file, the following must have
$! been performed:
$!
$! 1) The root for the CERNLIB installation must be setup.
$!
$!    On the CNLAVC, the CERNLIB tree is installed on
$!    VSCLIB$USER1.
$!
$!    e.g. CREATE/DIRECTORY VSCLIB$USER1:[CERNLIB]
$!
$!         CREATE/DIRECTORY VSCLIB$USER1:[CERNLIB.CERNAXP]
$!
$!    These directory names are arbitrary. One could equally
$!    well use 
$!
$!         CREATE/DIRECTORY VSCLIB$USER1:[FRODO]
$!
$!         CREATE/DIRECTORY VSCLIB$USER1:[FRODO.BAGGINS]
$!
$! 2) The subdirectory for the appropriate version must be created.
$!
$!    e.g. CREATE/DIRECTORY VSCLIB$USER1:[CERNLIB.CERNAXP.94A]
$!
$!    Again, this directory name is arbitrary. One could have used
$!    NEW, 93D, GLONK etc.
$!
$! 3) The subdirectories [.MGR] and [.EXE] must be created.
$!
$!    e.g. CREATE/DIRECTORY VSCLIB$USER1:[CERNLIB.CERNAXP.94A.MGR]
$!
$!    e.g. CREATE/DIRECTORY VSCLIB$USER1:[CERNLIB.CERNAXP.94A.EXE]
$!
$! 4) The command files in VXCERN::CERN_ROOT:[EXE] and
$!                         VXCERN::CERN_ROOT:[MGR] must be copied
$!    to the appropriate subdirectory.
$!
$! 5) Finally, the files VXCERN::CERN_ROOt:[MGR]*.FOR should 
$!    be retrieved and compiled. MAIN_CERNLOGIN.FOR includes
$!    CERN_ROOT:[MGR]CERNLOGIN.FOR. As CERN_ROOT is not defined
$!    at this stage, simply edit the file and include CERNLOGIN.FOR
$!    directly.
$!
$!  If you cannot access the CERN sources via NFS, the following 
$!  steps are also required.
$!
$!  a) Create the subdirectories [.SRC] and [.SRC.CAR]
$!
$!  b) Copy the .CAR and .CRA files from asisftp or VXCERN
$!     to the [.SRC.CAR] directory.
$!
$!  c) Remove all references to NFS below.
$!
$! ===============================================================
$!    
$! Mount /asis/share tree using NFS. This avoids us having to copy
$! the .CAR and .CRA files locally
$!
$ if f$trnlnm("asis_share") .eqs. "" then nfsmount/soft asisnfs:"/asis/share" -
     asis_share
$ nfsdev     = f$getdvi("ASIS_SHARE","FULLDEVNAM")
$ define/sys/exec nfscern 'nfsdev'[cern.] /trans=(conc,term)
$!
$! Set the CERNLIB level
$!
$ cern_level :== 94a
$!
$! Define the CERN and CERN_ROOT logical names
$!
$ define/SYS/EXEC/TRANS=(CONCEALED,TERMINAL) -
     cern _$177$dka300:[cernlib.cernaxp.],'nfsdev'[cern.]
$!
$ define/SYS/EXEC/TRANS=(CONCEALED,TERMINAL) -
     cern_root _$177$dka300:[cernlib.cernaxp.'cern_level'.],-
     'nfsdev'[cern.'cern_level'.]
$!
$! For Alpha, need also CERNAXP
$!
$ define/SYS/EXEC/TRANS=(CONCEALED,TERMINAL) -
     cernaxp _$177$dka300:[cernlib.cernaxp.]
$!
$! Define various symbols required by the CERNLIB installation
$! (and CERNLIB users)
$!
$ run main_cernlogin
$!
$! Reassert the CERNLIB level (MAIN_CERNLOGIN sets it to PRO)
$!
$ cern_level :== 94a
$!
$! Set the Program Library Installation Environment
$!
$ @cern:['cern_level'.mgr]plienv
$!
$! Override the settings of the following symbols made
$! by MAIN_CERNLOGIN.EXE
$! 
$! We must also install these files...
$!
$ ycompare   :== $cern:['cern_level'.exe]ycompare
$ yedit      :== $cern:['cern_level'.exe]yedit
$ yfrceta    :== $cern:['cern_level'.exe]yfrceta
$ yindex     :== $cern:['cern_level'.exe]yindex
$ ylist      :== $cern:['cern_level'.exe]ylist
$ ypatchy    :== $cern:['cern_level'.exe]ypatchy
$ ysearch    :== $cern:['cern_level'.exe]ysearch
$ yshift     :== $cern:['cern_level'.exe]yshift
$ ytobcd     :== $cern:['cern_level'.exe]ytobcd
$ ytobin     :== $cern:['cern_level'.exe]ytobin
$ ytoceta    :== $cern:['cern_level'.exe]ytoceta
$!
$ yexp*and   :== @cern:['cern_level'.exe]yexpand
$!
$ fcasplit   :== $cern:['cern_level'.exe]fcasplit
$!
$! Private symbols
$!
$ ed*it      :== edit/tpu/section=sys$login:eve
$ t*ype      :== type/page
\end{XMPt}

\chapter{Initial setup and configuration for VM/CMS systems}
