\chapter{Reserved prefixes}

\label{sect-PREFIX}

Most of the routines in the CERN library begin with
one or two letters (three in the case of \HPLOT{})
that identify the package. Collaborations wishing
to avoid clashes may register two character prefixes
with the CERN Program Library office.

{\bf N.B. the letter {\it Z} should be avoided as
second character as this is used by ZEBRA routines}.

The following convention is used. 

\begin{DLtt}{1}
\item[A]

\item[B]

\item[C]

\begin{DLtt}{123}
\item[CD]HEPDB routines
\item[CS]COMIS routines
\item[CZ]Client routines of CSPACK
\end{DLtt}

\item[D]

\begin{DLtt}{123}
\item[DZ]ZEBRA debug and documentation routines
\end{DLtt}

\item[E]

\begin{DLtt}{123}
\item[EP]EPIO routines
\end{DLtt}

\item[F]

\begin{DLtt}{123}
\item[FA]FATMEN internal routines and common blocks
\item[FF]FFREAD routines
\item[FM]FATMEN user callable routines
\item[FM]ZEBRA sequential I/O (FZ) routines
\end{DLtt}

\item[G]

\begin{DLtt}{123}
\item[G]GEANT routines
\end{DLtt}

\item[H]

\begin{DLtt}{123}
\item[H]HBOOK routines
\item[HPL]HPLOT routines
\end{DLtt}

\item[I]

\item[J]

\begin{DLtt}{123}
\item[JZ]The ZEBRA jump package
\end{DLtt}

\item[K]

\begin{DLtt}{123}
\item[K]KUIP routines
\item[KA]KAPACK routines
\end{DLtt}

\item[L]

\begin{DLtt}{123}
\item[LZ]ZEBRA utility functions
\end{DLtt}

\item[M]

\begin{DLtt}{123}
\item[MZ]ZEBRA memory management routines
\end{DLtt}

\item[N]

\item[O]

\item[P]

\item[Q]

\item[R]

\begin{DLtt}{123}
\item[RZ]ZEBRA random access (RZ) I/O routines
\end{DLtt}

\item[S]

\begin{DLtt}{123}
\item[SI]SIGMA 
\item[SZ]CSPACK server routines
\end{DLtt}

\item[T]

\begin{DLtt}{123}
\item[TZ]ZEBRA titles package
\end{DLtt}

\item[U]

\item[V]

\begin{DLtt}{123}
\item[VM]VMIO routines
\end{DLtt}

\item[W]

\item[X]

\begin{DLtt}{123}
\item[XZ]CSPACK remote I/O routines
\end{DLtt}

\item[Y]

\item[Z]

\begin{DLtt}{123}
\item[Z]ZBOOK routines
\item[LZ]ZEBRA utility routines
\end{DLtt}

\end{DLtt}
