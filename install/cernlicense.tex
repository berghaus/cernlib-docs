\chapter{Access to licensed products distributed by CERN}
\section{CMZ}
CMZ  is  the product of a software company (CodeMe S.A.R.L.); and it uses parts
of the CERN Program Library. The following agreement has been  reached  between
CERN and the supplier:
\begin{enumerate} 
\item
CERN users will have the right to use CMZ on all machines on the CERN site.
\item 
In  CERN  member  states all institutions collaborating with CERN, national
research laboratories in nuclear and particle physics and academic  physics
departments will receive CMZ for free.
\item 
Groups  or  individuals  outside  CERN member states collaborating with the
experimental programme of CERN or of one of the institutions  mentioned  in
the previous paragraph will have the right to receive CMZ free of charge and to use it for the
work they are doing in the framework of the above  collaboration.  If  they
want  to  use  CMZ  for any other activity, then they must obtain a license
from the supplier.
\item 
All cases not directly covered by the above rules 1,2 and 3 will have to be
negotiated directly with the supplier. More information about
licence fees may be obtained from CODEME@CERNVM.
\item 
The CERN agreement with the company includes  maintenance  and  development
coverage for users of CMZ in the first three categories listed above
\end{enumerate}
 
Individuals or institutions entitled to receive CMZ free of charge according to
the above conditions, should request the program and the documentation from the
CERN Program Library Office.

\section{GKS}

\section{GPHIGS}

This information should be read in conjunction with the OPGS installation
guide and the information available through http://asdwww.cern.ch/pl/gphigs/index.html.

\subsection{General information}

The GPHIGS software should be installed on the VXCERN cluster
and in AFS. It will be automatically copied from AFS into the asis
tree, accessed via NFS on those systems that do not have AFS.

In AFS, it should be stored in {\bf /afs/cern.ch/asis/packages/licensed/g5g-V3.0/@sys}.
On asis, the corresponding tree is {\bf /asis/packages/licensed/g5g-V3.0/@sys.}

For management purposes, a separate volume is created for each operating system. 

\begin{XMPt}{AFS volumes for the NAG software}

/usr/local/bin/fs lsmount *

'alpha_osf20' is a mount point for volume '#asis.osf.c.g5g-v3.0'
'hp700_ux90' is a mount point for volume '#asis.hpx.c.g5g-v3.0'
'pmax_ul43a' is a mount point for volume '#asis.dec.c.g5g-v3.0'
'rs_aix32' is a mount point for volume '#asis.irs.c.g5g-v3.0'
'sgi_52' is a mount point for volume '#asis.sgi.c.g5g-v3.0'
'sun4c_413' is a mount point for volume '#asis.sun.c.g5g-v3.0'
'sun4c_53' is a mount point for volume '#asis.sol.c.g5g-v3.0'

\end{XMPt}

In addition, the {\it tar} files and {\it backup savesets} are stored
under the CERN tree, as shown below:

\begin{XMPt}{Location of the {\it tar} files}

/afs/cern.ch/asis/alpha_osf20/cern/g5g/V30/tar:
opgs.ALPHA.OSF.2.0.contents
opgs.ALPHA.OSF.2.0.tar

 afs/cern.ch/asis/hp700_ux90/cern/g5g/V30/tar:
opgs.HP700.9.0.contents
opgs.HP700.9.0.tar

/afs/cern.ch/asis/hppa1.1-hp-hpux90/cern/g5g/V30/tar:
opgs.HP700.9.0.contents
opgs.HP700.9.0.tar

/afs/cern.ch/asis/mips-dec-ultrix4.2/cern/g5g/V30/tar:
opgs.DEC-ULTRIX.4.3.contents
opgs.DEC-ULTRIX.4.3.tar

/afs/cern.ch/asis/mips-dec-ultrix4.3/cern/g5g/V30/tar:
opgs.DEC-ULTRIX.4.3.contents
opgs.DEC-ULTRIX.4.3.tar

/afs/cern.ch/asis/mips-sgi-irix4.0.5/cern/g5g/V30/tar:
opgs.SGI.4.0.contents
opgs.SGI.4.0.tar

/afs/cern.ch/asis/pmax_ul4.old/cern/g5g/V30/tar:
opgs.DEC-ULTRIX.4.3.contents
opgs.DEC-ULTRIX.4.3.tar

/afs/cern.ch/asis/pmax_ul4/cern/g5g/V30/tar:
opgs.DEC-ULTRIX.4.3.contents
opgs.DEC-ULTRIX.4.3.tar

/afs/cern.ch/asis/pmax_ul42/cern/g5g/V30/tar:
opgs.DEC-ULTRIX.4.3.contents
opgs.DEC-ULTRIX.4.3.tar

/afs/cern.ch/asis/pmax_ul43/cern/g5g/V30/tar:
opgs.DEC-ULTRIX.4.3.contents
opgs.DEC-ULTRIX.4.3.tar

/afs/cern.ch/asis/pmax_ul43a/cern/g5g/V30/tar:
opgs.DEC-ULTRIX.4.3.contents
opgs.DEC-ULTRIX.4.3.tar

/afs/cern.ch/asis/rs6000-ibm-aix3.2/cern/g5g/V30/tar:
opgs.RS6000.3.2.contents
opgs.RS6000.3.2.tar

/afs/cern.ch/asis/rs_aix32/cern/g5g/V30/tar:
opgs.RS6000.3.2.contents
opgs.RS6000.3.2.tar

/afs/cern.ch/asis/sgi3000_40/cern/g5g/V30/tar:
opgs.SGI.4.0.contents
opgs.SGI.4.0.tar

/afs/cern.ch/asis/sgi3000_52/cern/g5g/V30/tar:
opgs.SGI.5.2.contents
opgs.SGI.5.2.tar

/afs/cern.ch/asis/sparc-sun-solaris2.2/cern/g5g/V30/tar:
opgs.SUN.5.3.contents
opgs.SUN.5.3.tar

/afs/cern.ch/asis/sparc-sun-solaris2.3/cern/g5g/V30/tar:
opgs.SUN.5.3.contents
opgs.SUN.5.3.tar

/afs/cern.ch/asis/sparc-sun-sunos4.1.1/cern/g5g/V30/tar:
opgs.SUN.4.1.contents
opgs.SUN.4.1.tar

/afs/cern.ch/asis/sparc-sun-sunos4.1.2/cern/g5g/V30/tar:
opgs.SUN.4.1.contents
opgs.SUN.4.1.tar

/afs/cern.ch/asis/sparc-sun-sunos4.1.3/cern/g5g/V30/tar:
opgs.SUN.4.1.contents
opgs.SUN.4.1.tar

/afs/cern.ch/asis/sun4_411/cern/g5g/V30/tar:
opgs.SUN.4.1.contents
opgs.SUN.4.1.tar

/afs/cern.ch/asis/sun4_412/cern/g5g/V30/tar:
opgs.SUN.4.1.contents
opgs.SUN.4.1.tar

/afs/cern.ch/asis/sun4_413/cern/g5g/V30/tar:
opgs.SUN.4.1.contents
opgs.SUN.4.1.tar

/afs/cern.ch/asis/sun4_52/cern/g5g/V30/tar:
opgs.SUN.5.3.contents
opgs.SUN.5.3.tar

/afs/cern.ch/asis/sun4_53/cern/g5g/V30/tar:
opgs.SUN.5.3.contents
opgs.SUN.5.3.tar

/afs/cern.ch/asis/sun4c_411/cern/g5g/V30/tar:
opgs.SUN.4.1.contents
opgs.SUN.4.1.tar

/afs/cern.ch/asis/sun4c_412/cern/g5g/V30/tar:
opgs.SUN.4.1.contents
opgs.SUN.4.1.tar

/afs/cern.ch/asis/sun4c_413/cern/g5g/V30/tar:
opgs.SUN.4.1.contents
opgs.SUN.4.1.tar

/afs/cern.ch/asis/sun4c_52/cern/g5g/V30/tar:
opgs.SUN.5.3.contents
opgs.SUN.5.3.tar

/afs/cern.ch/asis/sun4c_53/cern/g5g/V30/tar:
opgs.SUN.5.3.contents
opgs.SUN.5.3.tar

/afs/cern.ch/asis/sun4m_411/cern/g5g/V30/tar:
opgs.SUN.4.1.contents
opgs.SUN.4.1.tar

/afs/cern.ch/asis/sun4m_412/cern/g5g/V30/tar:
opgs.SUN.4.1.contents
opgs.SUN.4.1.tar

/afs/cern.ch/asis/sun4m_413/cern/g5g/V30/tar:
opgs.SUN.4.1.contents
opgs.SUN.4.1.tar

/afs/cern.ch/asis/sun4m_52/cern/g5g/V30/tar:
opgs.SUN.5.3.contents
opgs.SUN.5.3.tar

/afs/cern.ch/asis/sun4m_53/cern/g5g/V30/tar:
opgs.SUN.5.3.contents
opgs.SUN.5.3.tar

\end{XMPt}

\subsection{Unix}

The {\it tar} files should be retrived from the G5G ftp server,
as explained in the OPGS installation guide. These files should
then be stored in the appropriate directories in afs and unpacked
as shown below:

\begin{XMPt}{Unpacking the {\it tar} files}

cd /afs/cern.ch/asis/packages/licensed/g5g-V3.0/rs_aix32

tar xvf /afs/cern.ch/project/ftp/cernlib/rs_aix32/g5g/V3.0/tar/opgs.RS6000.3.2.tar

\end{XMPt}

The ACLs on the target directory should then be set to ensure that access
is only available to CERN nodes.

\begin{XMPt}{ACLs for G5G software}

Access list for /asis/packages/licensed/g5g-V3.0 is
Normal rights:
  asis:nag rlidwka
  asis:licensed rlidwk
  asis:mgr rlidwk
  cern:nodes rl

\end{XMPt}

Finally, a suitable file named {\bf license} should be created
in the directory {\bf /asis/packages/licensed/g5g-V3.0/@sys/OPGS/run\_time}.

One may now procede to test the software, e.g. by building the sample C and
Fortran programs.

\subsection{VM/CMS}

There is no version of GPHIGS for VM/CMS.

\subsection{VMS}

\section{LAPACK}

\label{sect-LAPACK}
\index{LAPACK}
\index{netlib}

LAPACK may be obtained by anonymous ftp from 
netlib2.cs.utk.edu as shown below.

\begin{XMPt}{Obtaining the LAPACK tar file}
zfatal:/home/cp/jamie (4) ftp netlib2.cs.utk.edu
Connected to netlib2.cs.utk.edu.
220 netlib2 FTP server (Version 2.1aWU(1) Thu Jun 3 23:00:04 EDT 1993) ready.
Name (netlib2.cs.utk.edu:jamie): anonymous
331 Guest login ok, send your complete e-mail address as password.
Password:
230 Guest login ok, access restrictions apply.
ftp> bin
200 Type set to I.
ftp> ls *.tar.z
200 PORT command successful.
150 Opening ASCII mode data connection for file list.
lapack.tar.z
manpages.tar.z
revisions1.0a.tar.z
revisions1.0b.tar.z
testing.tar.z
timing.tar.z
226 Transfer complete.
ftp> mget *

...

\end{XMPt}

\begin{verbatim}
==================
LAPACK README FILE
==================

VERSION 1.0  :  February 29, 1992
VERSION 1.0a :  June 30, 1992
VERSION 1.0b :  October 31, 1992

DATE:  October 31, 1992
\end{verbatim}

LAPACK is a library of Fortran 77 subroutines for solving
the most commonly occurring problems in numerical linear algebra.
It is public-domain software, and can be used freely.

The tar tape contains the Fortran source for LAPACK, the testing programs, and
the timing programs.

It also contains Fortran code for the Basic Linear Algebra Subprograms
(the Level 1, 2, and 3 BLAS) needed by LAPACK.
However this code is intended for use only if there is no other implementation
of the BLAS already available on your machine; the efficiency of LAPACK
depends very much on the efficiency of the BLAS.

The complete package, including test code and timing programs in four
different Fortran data types (real, complex, double precision, double
complex), contains some 600,000 lines of Fortran source and comments.
You will need approximately 28 Mbytes to read the complete tape.
We recommend that you run the testing and timing programs.
The total space requirements for the testing and timing for all four data
types, including the object files, is approximately 70 Mbytes.

A README file containing the information in this letter and a
QUICK\_INSTALL file containing a quick reference guide to the
installation process are located in the LAPACK directory.
Postscript and LaTeX versions of the Installation
Guide are in the LAPACK/INSTALL directory, in the files install.tex,
psfig.tex, install.ps, and org2.ps.  Consult the Installation Guide
for further details on installing the package and on what is contained
in each subdirectory.

It is highly recommended that you obtain a copy of the LAPACK
Users' Guide published by SIAM.  This Users' Guide gives a detailed
description of the philosophy behind LAPACK as well as an explanation of
its usage.  The LAPACK Users' Guide can be purchased from:
SIAM; 3600 University City Science Center; Philadelphia, PA 19104-2688;
215-382-9800, FAX 215-386-7999.  It will also be available from booksellers.
The Guide costs \$15.60 for SIAM members, and \$19.50 for non-members.
Please specify order code OT31 when ordering.  To order by email, send
email to service@siam.org.

LAPACK has been thoroughly tested, on many different
types of computers.  The LAPACK project supports the package in the
sense that reports of errors or poor performance will gain immediate
attention from the developers. Such reports, descriptions
of interesting applications, and other comments should be sent by
electronic mail to lapack@cs.utk.edu.

A list of known problems, bugs, and compiler errors for LAPACK is
maintained on netlib.  For a copy of this report, send email to
netlib@ornl.gov with a message of the form: send release\_notes from lapack.

A number of working notes were written during the
development of LAPACK and published as LAPACK Working Notes,
initially by Argonne National Laboratory and later by the University
of Tennessee.  Many of these reports have subsequently appeared as journal
articles.  Most of these working notes are available in postscript form
from netlib.  To receive a list of available reports, send email to
netlib@ornl.gov with a message of the form: send index from lapack/lawns.
Otherwise, requests for copies of these working notes can be sent to
the following address.

\begin{verbatim}
LAPACK Project
c/o J.J. Dongarra
Computer Science Department
University of Tennessee
Knoxville, Tennessee 37996-1301
USA
Email: lapack@cs.utk.edu
\end{verbatim}

\subsection{Installing LAPACK}

\subsubsection{Installing LAPACK on Unix systems}

See the LAPACK provided notes. The QUICK\_INSTALL guide
is included here.

====================================================
Quick Reference Guide for the Installation of LAPACK
====================================================

VERSION 1.0  :  February 29, 1992
VERSION 1.0a :  June 30, 1992
VERSION 1.0b :  October 31, 1992
VERSION 1.1  :  March 31, 1993

DATE:  March 31, 1993 

This Quick Reference Guide to the installation of LAPACK has been
extracted from the Installation Guide contained in LAPACK/INSTALL.
It is only intended as a quick reference.  The Installation Guide should
be consulted for full details.

To install, test, and time LAPACK:

1. Read the tape or uncompress and tar the file.

   tar  xvf  /dev/rst0  (cartridge tape),  or

   tar  xvf  /dev/rmt8  (9-track tape)
  
   uncompress lapack.tar.z (from a file), and

   tar xvf lapack.tar      (from a file)

2. Test and Install the Machine-Dependent Routines
   (WARNING:  You may need to supply a correct version of second.f and
              dsecnd.f for your machine)

   cd LAPACK/INSTALL
   make
   testlsame
   testslamch
   testdlamch
   testsecond
   testdsecnd

3. Create the BLAS Library, if necessary
   (NOTE:  For best performance, it is recommended you use the
           manufacturers' BLAS)

   cp LAPACK/INSTALL/lsame.f LAPACK/BLAS/SRC/
   cd LAPACK/BLAS/SRC
   make

4. Run the Level 2 and 3 BLAS Test Programs

   cd LAPACK/BLAS/TESTING
   make -f makeblat2
   cd LAPACK/BLAS
   xblat2s < sblat2.in
   xblat2d < dblat2.in
   xblat2c < cblat2.in
   xblat2z < zblat2.in
   cd LAPACK/BLAS/TESTING
   make -f makeblat3
   cd LAPACK/BLAS 
   xblat3s < sblat3.in
   xblat3d < dblat3.in
   xblat3c < cblat3.in
   xblat3z < zblat3.in

5. Create the LAPACK Library

   cp LAPACK/INSTALL/lsame.f LAPACK/SRC/
   cp LAPACK/INSTALL/slamch.f LAPACK/SRC/
   cp LAPACK/INSTALL/dlamch.f LAPACK/SRC/
   cp LAPACK/INSTALL/second.f LAPACK/SRC/
   cp LAPACK/INSTALL/dsecnd.f LAPACK/SRC/
   cd LAPACK/SRC
   make

6. Create the Library of Test Matrix Generators

   cd LAPACK/TESTING/MATGEN
   make

7. Run the LAPACK Test Programs

   cd LAPACK/TESTING
   make

8. Run the LAPACK Timing Programs

   cd LAPACK/TIMING
   make
   xlintims < sblasa.in > sblasa.out
   xlintims < sblasb.in > sblasb.out
   xlintims < sblasc.in > sblasc.out

   repeat timing of blas for c, d, and z

\subsubsection{Installing LAPACK on VMS systems}

The following steps can be used to install LAPACK on VMS systems.

\begin{enumerate}
\item
Create directories [.LAPACK.SRC] and [.LAPACK.BLAS.SRC]
\item
ftp the Fortran files from the unpacked tar file into the
appropriate directory
\item
Compile the Fortran files
\item
Append the object files into CERN:[NEW.OBJ]LAPACK.OBJ
\end{enumerate}

The Fortran files can be compiled with a simple command
procedure such as the one shown below.

\begin{XMPt}{Compiling the LAPACK source}

$ loop:
$ a = f$search("*.F")
$ if a .eqs. "" then exit
$ write sys$output "Compiling ''a'"
$ fortran 'a'
$ goto loop

\end{XMPt}

\subsubsection{Installing LAPACK on VM/CMS systems}

LAPACK is installed as two separate text files:
\begin{DLtt}{12345678}
\item[LABLASRC]Basic Linear Algebra component (BLAS)
\item[LAPACK]LAPACK itself
\end{DLtt}

These correspond to the files found in LAPACK/BLAS/SRC and LAPACK/SRC
in the LAPACK tar file respectively.

Having unpacked the tar file, LAPACK can be installed on VM systems using
the following steps:

\begin{enumerate}
\item
Transfer the source files to VM, e.g. using ftp.
\item
Edit the source files so that they will compile. 
\begin{itemize}
\item
Change occurances of {\bf double complex} to {\bf complex*16}.
\item
Change {\bf double complex} functions to {\bf COMPLEX function*16}.
\item
Change calls to the {\bf DBLE} intrinsic funtion. When the
argument is of type {\bf COMPLEX}, change to {\bf REAL(argument)}.
e.g. change DBLE{x} to DBLE(REAL(x)).
\item
Change calls to the {\bf INT} intrinsic funtion. When the
argument is of type {\bf COMPLEX}, change to {\bf REAL(argument)}.
e.g. change INT{x} to INT(REAL(x)).
\item
Change DCMPLX(1) to DCMPLX(1,0.)
\end{itemize}
\end{enumerate}

The above changes, with the exception of the first two
which can be simply performed using an editor, can
be done using {\bf FLOPPY}~\footnote{Thanks to Julian Bunn
for making the necessary changes to a private version of
FLOPPY for this purpose.}.

\begin{XMPt}{Converting LAPACK for VSFORTRAN compatibility}

AXCLIB? SET COMMAND DISK$USER1:[JAMIE.FLOPPY]FLOPPY

AXCLIB? FLOPPY/NOCHECK/TIDY/INDENT=3 LAPACK.FORTRAN ! output to LAPACK.FLOPFOR

\end{XMPt}

\subsubsection{Install LAPACK on Windows/NT systems}
\subsubsection{Install LAPACK on MSDOS systems}

\section{MPA}

\label{sect-MPA}

\index{MPA}

MPA is a package for multiple precision floating point
operations. It is a commercial product that can only
be distributed in object form. It is automatically included
in versions of MATHLIB obtained from asis.

If you intend to rebuilt MATHLIB from scratch, you should
copy the object file from CERN.

\begin{DLtt}{1234567890}
\item[VM/CMS]MPA TEXT on CERNLIB 322
\item[VAX/VMS]Copy VXCERN::CERNVAX:[PRO.OBJ]MPA.OBJ to CERN:[NEW.OBJ]MPA.OBJ
\item[AXP/VMS]Copy VXCERN::CERNAXP:[PRO.OBJ]MPA.OBJ to CERN:[NEW.OBJ]MPA.OBJ
\item[Unix]Copy 
\end{DLtt}

\section{NAG}

The information in this section supplements that found in the NAG installation
notes and concerns only the NAG Fortran double precision library, release
MARK 16. For further details, consult the NAG installation notes for the
platform in question. 

\subsection{General information}

The NAG library should be installed on CERNVM, the VXCERN cluster
and in AFS. It will be automatically copied from AFS into the asis
tree, accessed via NFS on those systems that do not have AFS.

In AFS, it should be stored in {\bf /afs/cern.ch/asis/packages/licensed/nag-mark16/@sys}.
On asis, the corresponding tree is {\bf /asis/packages/licensed/nag-mark16/@sys.}

For management purposes, a separate volume is created for each operating system.

\begin{XMPt}{AFS volumes for the NAG software}

/usr/local/bin/fs lsmount *

'alpha_osf22' is a mount point for volume '#asis.alpha_osf2.c.ng16'
'hp700_ux90' is a mount point for volume '#asis.hp700_ux90.c.ng16'
'pmax_ul43a' is a mount point for volume '#asis.pmax_ul4.c.ng16'
'rs_aix32' is a mount point for volume '#asis.rs_aix32.c.ng16'
'sgi3000_40' is a mount point for volume '#asis.sgi3000_40.c.ng16'
'sgi_52' is a mount point for volume '#asis.sgi_52.c.ng16'
'sun4_413' is a mount point for volume '#asis.sun4_413.c.ng16'
'sun4_53' is a mount point for volume '#asis.sun4_53.c.ng16'

\end{XMPt}

\subsection{Unix}

Unix versions of NAGLIB should be delivered with on QIC or DAT tapes.
There is a DAT drive on {\bf hpcernlib} and a QIC drive on {\bf zfatal}.

Go to the directory where you would like to install the software, e.g.
{\bf /asis/packages/licensed/nag-mark16/rs\_aix32}, mount the tape
in the drive and load the software by typing \underline{tar x}.

On some systems, more than one version of NAGLIB exists. For example,
on RS6000s and HP/UX, versions both with and without the trailing
underscore are provided. 

In general, make sure that the file {\bf libnag.a}

\begin{enumerate}
\item
Contains a trailing underscore
\item
Is "complete", i.e. does not require any external libraries,
such as optimised versions of BLAS, sometimes supplied by the 
vendors.
\item
If there is a shared library {\bf libnag.so}, rename it to
avoid picking up the shared version by default.
\end{enumerate}

\begin{XMPt}{List of all Unix NAG Fortran libraries}
zfatal:/asis/packages/licensed/nag-mark16 (364) ls -l */nagfl16df/lib*
-rw-r--r--   1 cernlib  cr       16850912 Dec  6 14:08 alpha_osf22/nagfl16df/libnag.a
-rw-r--r--   1 cernlib  cr       8582742 Jul 26 16:11 alpha_osf22/nagfl16df/libnag.share
-rw-r--r--   1 cernlib  cr       16645272 Jul 26 16:14 alpha_osf22/nagfl16df/libnagdx.a
-rw-r--r--   1 cernlib  cr       8497254 Jul 26 16:14 alpha_osf22/nagfl16df/libnagdx.so
lrwxr-xr-x   1 cernlib  cr            12 Dec  6 14:40 hp700_ux90/nagfl16df/libnag.a -> libnag_ppu.a
-rw-------   1 cernlib  cr       8800940 Jul  6 12:31 hp700_ux90/nagfl16df/libnag_noppu.a
-rw-------   1 cernlib  cr       8805816 Jul  6 12:32 hp700_ux90/nagfl16df/libnag_ppu.a
-rw-------   1 cernlib  cr       13587936 Dec  6 17:05 pmax_ul43a/nagfl16df/libnag.a
lrwxr-xr-x   1 jamie    cp            10 Dec  6 15:10 rs_aix32/nagfl16df/libnag.a -> libnag_u.a
-rw-r--r--   1 jamie    cp       8584439 Dec  6 14:46 rs_aix32/nagfl16df/libnag_b.a
-rw-r--r--   1 jamie    cp       8603762 Dec  6 12:53 rs_aix32/nagfl16df/libnag_b_noextname.a
-rw-r--r--   1 jamie    cp       8605192 Dec  6 12:52 rs_aix32/nagfl16df/libnag_noextname.a
-rw-r--r--   1 jamie    cp       8600181 Dec  6 12:53 rs_aix32/nagfl16df/libnag_u.a
-rw-r--r--   1 cernlib  cr       11313952 May  8 1994  sgi3000_40/nagfl16df/libnag.a
-rw-r--r--   1 cernlib  cr       11341700 May  8 1994  sgi3000_40/nagfl16df/libnagblas.a
lrwxr-xr-x   1 cernlib  cr            12 Dec  6 15:34 sgi_52/nagfl16df/libnag.a -> libnag3000.a
-rw-r--r--   1 cernlib  cr       13263820 Jun  8 1994  sgi_52/nagfl16df/libnag3000.a
-rw-r--r--   1 cernlib  cr       12357016 Jun  8 1994  sgi_52/nagfl16df/libnag4000.a
-rw-r--r--   1 cernlib  cr       13327488 Jun  8 1994  sgi_52/nagfl16df/libnagblas3000.a
-rw-r--r--   1 cernlib  cr       12417468 Jun  8 1994  sgi_52/nagfl16df/libnagblas4000.a
-rw-r--r--   1 jamie    cp       11648720 Dec  6 15:57 sun4_413/nagfl16df/libnag.a
-rw-r--r--   1 cernlib  cr       4662443 Dec  6 13:11 sun4_413/nagfl16df/libnag_1.4.a.gz
-rw-r--r--   1 cernlib  cr       14132336 Nov  4 1993  sun4_53/nagfl16df/libnag.a
zfatal:/asis/packages/licensed/nag-mark16 (365) 
\end{XMPt}

In the above list, 3 links where created locally, the Alpha/OSF shared library was renamed
and the RS6000 library rebuilt locally. In fact, the NAG supplied {\bf libnag\_u.a} would
have been sufficient.

Test programs can be run in the following manner:

\begin{XMPt}{Running the test programs}

export NAG=/asis/packages/licensed/nag-mark16/rs_aix32/nagfl16df

xlf -qextname a02aafe.f -L$NAG -lnag -o a02aafe

a02aafe < a02aaf.data

\end{XMPt}

\subsection{VM/CMS}

The VM/CMS version of NAGLIB arrives on 9 track tape. As there are no
more units of this type on the CERN central services, the first step
is to copy the tape to a 3480 cartridge. This can be done in the tape
vault, with the help of the DCS staff.

Having copied the tape, it must now be loaded. This is done using
the CMS {\bf TAPE} command.

Procede as follows:

\begin{enumerate}
\item
Choose a CERNLIB disk on which to load the software. You might
wish to reuse one of the following disks:
\begin{DLtt}{12345678}
\item[0A14]NAG Fortran library Mark 14
\item[0014]NAG Fortran library Mark 15
\item[0353]NAG Fortran library Mark 16
\end{DLtt}
\item
Link and access the appropriate disk in R/W mode, e.g. \underline{GIME CERNLIB 353 (MR}.
\item
Ask the operators to attach a manual 3480 tape drive to your virtual machine at address 181.
\item
Go to the tape vault and mount the tape
\item
Use the {\bf TAPE LOAD} command to load all of the software, as shown in the following log file.
\item
Detach the tape drive using \underline{CP DETACH 181} and retrieve the tape from the vault.
\end{enumerate}

\begin{XMPt}{Loading the VM/CMS distribution}
1Ready; T=0.01/0.01 14:19:22
 tape load * * z
  Loading ...
  IN       DOC      Z1
  TAPE     DOC      Z1
  DOCS     DOC      Z1
  End-of-file or end-of-tape
 Ready; T=0.08/0.30 14:20:06
 fd z
 Ready; T=0.02/0.02 14:20:14
 fd z
 Ready; T=0.02/0.02 14:20:39
 tape load * * z
  Loading ...
  NAGLIB1  TXTLIB   Z1
  NAGLIB2  TXTLIB   Z1
  End-of-file or end-of-tape
 Ready; T=0.35/1.64 14:23:33
 QUERY DISK
 LABEL  VDEV M  STAT  CYL TYPE BLKSIZE  FILES  BLKS USED-(%) BLKS LEFT  BLK TOTAL
 LIB191 191  A   R/W    5 3390 4096       178        358-40        542        900
 N/COM  320  C   R/O    1 3390 4096        40        134-74         46        180
 PLTOOL 197  G   R/O    5 3390 4096       258        758-84        142        900
 /P/CRN 199  P   R/O  340 3390 4096      1316      26175-43      35025      61200
 /Q/CRN 198  Q   R/O  250 3390 4096       265      40334-90       4666      45000
 S-DISK 190  S   R/O   82 3390 4096       345       9191-62       5569      14760
 Y-DISK 19E  Y/S R/O  250 3390 4096       865      33082-74      11918      45000
 NAG-16 1C0  Z   R/W  100 3380 4096         5       3918-26      11082      15000
 Ready; T=0.01/0.01 14:23:47
 tape load * * z
  Loading ...
  EXPT     FORTRAN  Z1
  EXPD     DATA     Z1
  EXPR     RESULTS  Z1
  End-of-file or end-of-tape
 Ready; T=0.18/0.65 14:24:21
 tape load * * z
  Loading ...
  NAGA     ICAFILE  Z1
  NAGC     ICAFILE  Z1
  NAGD     ICAFILE  Z1
  NAGE     ICAFILE  Z1
  NAGF     ICAFILE  Z1
  NAGG     ICAFILE  Z1
  NAGH     ICAFILE  Z1
  NAGM     ICAFILE  Z1
  NAGP     ICAFILE  Z1
  NAGS     ICAFILE  Z1
  NAGX     ICAFILE  Z1
  End-of-file or end-of-tape
 Ready; T=0.11/0.45 14:24:43
 tape load * * z
  Loading ...
  TEXTA00  FORTRAN  Z1
  TEXTA02  FORTRAN  Z1
  TEXTC02  FORTRAN  Z1
  TEXTC05  FORTRAN  Z1
  TEXTC06  FORTRAN  Z1
  TEXTD01  FORTRAN  Z1
  TEXTD02  FORTRAN  Z1
1 TEXTD03  FORTRAN  Z1
  TEXTD04  FORTRAN  Z1
  TEXTD05  FORTRAN  Z1
  TEXTE01  FORTRAN  Z1
  TEXTE02  FORTRAN  Z1
  TEXTE04  FORTRAN  Z1
  TEXTF01  FORTRAN  Z1
  TEXTF02  FORTRAN  Z1
  TEXTF03  FORTRAN  Z1
  TEXTF04  FORTRAN  Z1
  TEXTF05  FORTRAN  Z1
  TEXTF06  FORTRAN  Z1
  TEXTF07  FORTRAN  Z1
  TEXTF08  FORTRAN  Z1
  TEXTG01  FORTRAN  Z1
  TEXTG02  FORTRAN  Z1
  TEXTG03  FORTRAN  Z1
  TEXTG04  FORTRAN  Z1
  TEXTG05  FORTRAN  Z1
  TEXTG07  FORTRAN  Z1
  TEXTG08  FORTRAN  Z1
  TEXTG10  FORTRAN  Z1
  TEXTG11  FORTRAN  Z1
  TEXTG12  FORTRAN  Z1
  TEXTG13  FORTRAN  Z1
  TEXTH    FORTRAN  Z1
  TEXTM01  FORTRAN  Z1
  TEXTP01  FORTRAN  Z1
  TEXTS    FORTRAN  Z1
  TEXTX    FORTRAN  Z1
  TEXTY07  FORTRAN  Z1
  TEXTY90  FORTRAN  Z1
  DEPM     ASSEMBLE Z1
  YXXLIB   TXTLIB   Z1
  End-of-file or end-of-tape
 Ready; T=0.61/2.22 14:26:16
 fd z
 Ready; T=0.02/0.03 14:26:28
 tape load * * z
  Loading ...
  End-of-file or end-of-tape
 File * * Z not found
 Ready(00028); T=0.01/0.01 14:26:30
 q disk z
 LABEL  VDEV M  STAT  CYL TYPE BLKSIZE  FILES  BLKS USED-(%) BLKS LEFT  BLK TOTAL
 NAG-16 1C0  Z   R/W  100 3380 4096        60       9652-64       5348      15000
 Ready; T=0.01/0.01 14:27:06
 det 181
 TAPE 0181 DETACHED
 Ready; T=0.01/0.01 14:27:09
 cp log
 CONNECT= 02:14:38 VIRTCPU= 000:02.85 TOTCPU= 000:07.35
 LOGOFF AT 16:26:42 WET MONDAY 12/05/94
\end{XMPt}

The installation documentation is supplied in a single file: {\tt DOCS DOC}.
This can be split using the {\bf COPYFILE} command, e.g. using the following
EXEC.

\begin{XMPt}{Unpacking the documentation file}
&TRACE ALL
COPYFILE DOCS DOC Z UN = Z (FRL *AUN TOL *ZUN
COPYFILE DOCS DOC Z LIST = Z (FRL *ALIST TOL *ZLIST
COPYFILE DOCS DOC Z ESSINT = Z (FRL *AESSINT TOL *ZESSINT
COPYFILE DOCS DOC Z SUMMARY = Z (FRL *ASUMMARY TOL *ZSUMMARY
COPYFILE DOCS DOC Z NEWS = Z (FRL *ANEWS TOL *ZNEWS
COPYFILE DOCS DOC Z REPLACED = Z (FRL *AREPLACED TOL *ZREPLACED
COPYFILE DOCS DOC Z CALLS = Z (FRL *ACALLS TOL *ZCALLS
COPYFILE DOCS DOC Z CALLED = Z (FRL *ACALLED TOL *ZCALLED
\end{XMPt}

One should then run some tests to verify that the installation
is ok. As on other systems, we run three tests, {\bf A02AAF, G05FFF} and
{\bf X03AAF}. These tests can be found in the file {\bf EXPT FORTRAN}.
As was the case with the documentation file, this file can be split using
the {\bf COPYFILE} command, or simply with a text editor. Input data for
those tests that require it can be found in the file {\bf EXPD DATA}.

\begin{XMPt}{A02AAF}
*AA02AAF
*     A02AAF Example Program Text
*     Mark 14 Revised.  NAG Copyright 1989.
*     .. Parameters ..
      INTEGER          NIN, NOUT
      PARAMETER        (NIN=5,NOUT=6)
*     .. Local Scalars ..
      DOUBLE PRECISION XI, XR, YI, YR
*     .. External Subroutines ..
      EXTERNAL         A02AAF
*     .. Executable Statements ..
      WRITE (NOUT,*) 'A02AAF Example Program Results'
*     Skip heading in data file
      READ (NIN,*)
      READ (NIN,*) XR, XI
*
      CALL A02AAF(XR,XI,YR,YI)
*
      WRITE (NOUT,*)
      WRITE (NOUT,*) '   XR    XI      YR       YI'
      WRITE (NOUT,99999) XR, XI, YR, YI
      STOP
*
99999 FORMAT (1X,2F6.1,2F9.4)
      END


A02AAF Example Program Data                                                     
 -1.7 2.6                                                                       

\end{XMPt}

\begin{XMPt}{Running A02AAF}

VFORT A02AAF
CERNLIB NAGLIB1 NAGLIB2
FILEDEF 5 DISK A02AAF DATA Z (PERM
LOAD A02AAF (NOMAP NOAUTO CLEAR START

\end{XMPt}

The remaining two tests can be run in a similar manner, with the exception
that {\bf G05FFF} has no input data. The correct results from these programs
can be found in the file {\bf EXPR RESULTS}.
Finally, one should modify or create the {\tt GIME NOTICE} file, e.g.

\begin{XMPt}{Example GIME NOTICE file}
---------------------------------------------------------------------------
 (c)1994 Numeric Algorithms Group Ltd.
 
         NAG Fortran Library - Mark 16
 
         The NAG software has been purchased ONLY for local usage at CERN.
         Please contact the CERN Program Library Office for information on
         the availability and licensing.
 
See the file UN DOC * (Users' Note) for supplementary information
regarding this version of the NAG Fortran Library.
 
---------------------------------------------------------------------------
\end{XMPt}

The {\bf CERNLIB} exec may also need modification to permit access to the
new versions of NAGLIB. Once this has been done, the software can be announced.

\subsection{VMS}

VMS versions of NAGLIB are supplied on TK50. This is a problem as there
are no TK50 drives on the CNLAVC, nor on VXCERN. For the time being,
drives still exist on VXLDB2 node. This will not be the case for Mark 17
at which point a different method of distribution, e.g. DLT, must have
been found for VMS systems.

On VMS systems, the NAG provided installation instructions should be
followed. It was not possible to read the Fortran source on either
AXP or VAX, but the libraries were successfully loaded.

The files were then copied to {\bf CERN:[NAG.MARK16} on VXCERN
and the CNLAVC.

The tests may then be run as on other systems, e.g.

\begin{XMPt}{Example of running the tests on VMS}

$ FORTRAN A02AAFE
$ LINK A02AAFE,CERN:[NAG.MARK16]NAG$LIBRARY/LIB
$ DEFINE FOR005 A02AAF.DATA
$ RUN A02AAFE

\end{XMPt}
