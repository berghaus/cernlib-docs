%.*****************************************************************>
%.* CSPACK Reference Guide - Tutorial                             *>
%.*  Last Mod.   22 Jan 1992  15:40   mg                          *>
%.*****************************************************************>
\Filename{H1csptutr-A-tutorial-introduction-to-CSPACK}
\chapter{A tutorial introduction to CSPACK}
\Filename{H2csptutr-File-transfer-using-the-ZFTP-program}
\section{File transfer using the ZFTP program}
\par
ZFTP is a file transfer program which supports the transfer of formatted,
unformatted and ZEBRA RZ files (CMZ, HBOOK, etc.).
Formatted files are typically KUIP
macros, command or EXEC files, source code or even FZ ALPHA exchange
format files. Unformatted files may be FZ binary exchange format or EPIO
files, or any other binary file with fixed length records.
ZFTP also provides LS, CD, PWD and RSHELL commands.
It provides a common interface on all systems and the possibility of
macros (via KUIP). It avoids the problems of file format
conversion that occur when transferring binary files to UNIX systems.
\par
PAM files may be transferred in CARD, CETA, CMZ or even compact binary
PAM format. (In the first release this last format is limited to 32 bit
machines.) This is particularly convenient for software distribution
amongst disparate hardware types.
\subsubsection{Advantages of using ZFTP}
\par
The advantages of using ZFTP are best explained via examples. Suppose
one creates an HBOOK file (which is stored in ZEBRA RZ format) on the
IBM and that is required on a VAX. Without ZFTP, the file must
first be converted to sequential format using the RTOX utility.
The output file can then
be transferred to the VAX via Interlink or standard FTP but must
then be reconverted to RZ format using the RFRX utility. This requires
extra disk space on both sides for the sequential file
and a three-step process.
On some UNIX systems the situation is even worse
as the file transferred by FTP cannot be read by RFRX but must
be converted for a second time.
\par
The same operation using ZFTP is much simpler:
\begin{OL}
\item
The user issues the command {\bf\it ZFTP nodename}
and provides the remote username
and password much like with standard FTP.
\item
The command {\bf\it PUTRZ local-file remote-file} is issued.
\end{OL}
\par
Using this single stage operation the file is transferred with automatic
conversion from IBM to VAX format. As the user interface of ZFTP is based
on KUIP, the power of ZFTP may be extended using macros.
\par
Data transfer rates are currently about 2/3 of those obtainable with
standard FTP, but the effective transfer rate achieved by the elimination
of the conversion to sequential format and back is much higher.
\par
\begin{XMPt}{Example of a ZFTP session}
VXST? zftp crnvmc
system 'crnvmc' service 'zserv'
Remote host/port = crnvmc/17303
Name (crnvmc:jamie):
fmsn201
Password (crnvmc:fmsn201):
crnvmc: loading zserv exec (30 sec timeout)...
Connected to crnvmc on TCP port 1305 at Wed Sep 19 12:43:49 1990
ZFTP>
ZFTP> pwd # Show current working directory
Current working directory is FMSN201 191
ZFTP>
ZFTP> cd fat3.192 # Change to FAT3 192
Remote directory changed to FAT3 192
ZFTP>
ZFTP> ls cspack.cards -l # Directory listing
CSPACK   CARDS    A1 F         80      10406        204  9/17/90 14:33:37 ZEBRA
ZFTP>
ZFTP> geta cspack.cards = s # Transfer file displaying statistics
 File transfer completed
 Transferred       261512 bytes, transfer rate =    4.636364     KB/S
 Elapsed time = 00:00:55 CP time =    6.590000     sec.
ZFTP>
ZFTP> cd fmsn201 # Back to home directory
Remote directory changed to FMSN201
ZFTP>
ZFTP> ls
FATSENDR EXEC     A1
FMCHARM2 KUMAC    A1
FXFILE   BIG      A1
FXFILE   DAT      A1
GETZS    EXEC     A1
LASTING  GLOBALV  A0
PROFILE  EXEC     A1
ZSERV    MODULE   A1
ZFTP>
ZFTP> getb fxfile.dat = 32400 s # Transfer a ZEBRA FZ exchange file
 File transfer completed
 Transferred           10 records, transfer rate =    31.60000     KB/S
 Elapsed time = 00:00:10 CP time =   0.9100000     sec.
ZFTP>
ZFTP> rm fmcharm2.kumac # Delete a remote file
ZFTP>
ZFTP>
ZFTP> cd fmcndiv
Remote directory changed to FMCNDIV
ZFTP>
ZFTP> getrz cern.fatrz = s
File transfer completed


     NREC    NWORDS    QUOTA(%)  FILE(%)   DIR. NAME
       2       1187      0.00     0.00   //RZ/CNDIV/CERNLIB/SUN
       3       2211      0.00     0.00   //RZ/CNDIV/CERNLIB
       2       1350      0.00     0.00   //RZ/CNDIV/JUDY
       2       1839      0.00     0.00   //RZ/CNDIV/JAMIE
       8       6424      0.01     0.01   //RZ/CNDIV
      12      10520      0.01     0.01   //RZ
 Transferred           41 KB, rate =    8.200000     KB/S
 Elapsed time = 00:00:05 CP time =   0.6700000     sec.
ZFTP>
ZFTP> q
\end{XMPt}
\newpage
\begin{longtable}{|>{\tt}p{.15\textwidth}|p{.30\textwidth}|p{.44\textwidth}|}
\caption[ZFTP commands]{ZFTP commands\label{tab:ZFTPCOM}}\\
\hline
\rm\bf Command     & \rm\bf Function             &\rm\bf Description          \\
\hline
\endfirsthead
\caption[]{ZFTP commands (continued)}\\
\hline
\rm\bf Command     & \rm\bf Function             &\rm\bf Description          \\
\hline
\endhead
\hline
\endfoot
\tt OPEN         & Open connection             &
   Establish connection to specified host                                     \\
\tt CLOSE        & Close connection            &
   Close connection with current host                                         \\
\tt GETA/PUTA    & Text file transfer          &
   Text file transfer, e.g. scripts, EXECs, CARD pams etc.                    \\
\tt GETB/PUTB    & Binary file transfer        &
   Binary file transfer (fixed length records only)
   e.g. ZEBRA FZ binary exchange format, EPIO, CETA files                     \\
\tt GETD/PUTD    & Direct-access file transfer &
   Direct-access file transfer
   e.g. ZEBRA RZ file between like machines.                                  \\
\tt GETRZ/PUTRZ  & ZEBRA RZ file transfer      &
   RZ file transfer with automatic conversion between different
   data representations,
   e.g. HBOOK histogram or ntuple files, CMZ files.                           \\
\tt GETFZ/PUTFZ  & ZEBRA FZ file transfer      &
   FZ file transfer with automatic conversion between different
   data representations (currently in preparation)                            \\
\tt GETP/PUTP    & Compact binary PAM transfer &
   Transfer a compact binary PAM file (not yet to Cray)                       \\
\tt CD           & Change working directory    &
   Set working directory on remote node                                       \\
\tt LCD          & Change working directory    &
   Set working directory on local  node                                       \\
\tt LS           & Remote LS command           &
   Make remote directory listing                                              \\
\tt LLS          & Local  LS command           &
   Make local  directory listing                                              \\
\tt MPUT         & Put multiple files          &
    Send all files matching the specified pattern to the remote machine.
    The mode of transfer is determined by the file type:
    {\tt.CET, .CETA = PUTB, .CMZ = PUTRZ, other = PUTA}                     \\
\tt MGET         & Get multiple files          &
    Retrieve all files matching the specified pattern from the remote machine.
    The mode of transfer is determined by the file type:
    {\tt.CET, .CETA = GETB, .CMZ = GETRZ, other = GETA}                     \\
\tt MV           & Move (rename) remote file   &                            \\
\tt PWD          & Print remote directory      &
    Display current remote directory                                          \\
\tt LPWD         & Print local  directory      &
    Display current local  directory                                          \\
\tt RM           & Remove remote file          &
    Remote file deletion                                                      \\
\tt LRM          & Remove local  file          &
    Local  file deletion                                                      \\
\tt RSH          & Remote shell                &
    Issue command to the remote shell                                         \\
\tt VERSION      & Version of ZFTP             &
    Print version of the ZFTP program                                         \\
\tt SVERSION     & Version of server           &
Print version of the remote server (if connected)                             \\
\hline
\end{longtable}
\clearpage

\Filename{H2csptutr-Record-transfer-using-the-FORTRAN-interface}
\section{Record transfer using the FORTRAN interface}

The following example shows how individual records of
a FORTRAN direct-access file may be
accessed remotely.
\begin{XMPt}{Example of remote access of records in direct-access file}
      common/pawc/paw(50000)
      parameter (lrecl=4096)
      dimension buff(lrecl)
*
*     Initialise ZEBRA the easy way (get HBOOK to do it for us...)
*
      call hlimit(50000)
*
*     Open the file /fatmen/fmopal/cern.fatrz on node fatcat
*     The record length is 4096 bytes
*
      call xzopen(80,'/fatmen/fmopal/cern.fatrz','fatcat',
     +            lrecl,'D',irc)
      open(81,file='opal.fatrz',access='direct',recl=lrecl)
      nrec = 0
*
*     Now read each record in turn. Error is assumed to be end of file
*
10    continue
      nrec = nrec + 1
      call xzread(80,buff,nrec,lrecl,ngot,' ',irc)
      if(irc.eq.0) then
         write(81,rec=nrec) buff
         goto 10
      endif
*
*     Terminate
*
      call xzclos(80,' ',irc)
      close (81)
      end
\end{XMPt}
\subsection{File transfer using the FORTRAN callable routines}
\par
The following program demonstrates file transfer using the
FORTRAN callable routines. This program is used to transfer
updates to the FATMEN catalogue, which are distributed
as ZEBRA FZ files in ASCII exchange format, between CERNVM
and the FATMEN server.
It performs the following functions:
\begin{OL}
\item
Initialise ZEBRA (via call to HLIMIT)
\item
Initialise XZ    (define logical units, log level)
\item
Open connection to the FATMEN server
\item
Call an EXEC that uses WAKEUP to wakeup upon arrival of new files
in the RDR, or every hour.
\item
If a new file has been received, this is then sent to the
appropriate directory on the FATCAT machine.
\item
If no new file has been received, or after successfully sending
any new files, a search is made in the appropriate directories
on the remote node for pending updates for CERNVM.
\item
Any such files are transferred, and then the call to WAKEUP is reissued.
\item
The program can only exit if a user hits enter on the console
of the virtual machine, or if an appropriate SMSG is received
from a suitably authorised used.
\end{OL}
\begin{XMPt}{Example of file transfer using FORTRAN callable routines}
      PROGRAM FATCAT
*CMZ :          21/02/91  16.24.17  by  Jamie Shiers
*-- Author :    Jamie Shiers   21/02/91
*     Program to move updates between CERNVM and FATCAT
*
      PARAMETER    (NMAX=100)
      CHARACTER*64 FILES(NMAX)
      CHARACTER*8  FATUSR,FATNOD,REMUSR,REMNOD
      CHARACTER*64 REMOTE
      CHARACTER*12 CHTIME
      CHARACTER*8  CHUSER,CHPASS
      CHARACTER*80 CHMAIL,LINE
      COMMON/PAWC/PAW(50000)
      PARAMETER    (IPRINT=6)
      PARAMETER    (IDEBUG=3)
      PARAMETER    (LUNI=1)
      PARAMETER    (LUNO=2)
      COMMON /QUEST/ IQUEST(100)
      COMMON/SLATE/IS(6),IDUMM(34)
*
*     Initialise ZEBRA
*
      CALL HLIMIT(50000)
*
*     Initialise XZ
*
      CALL XZINIT(IPRINT,IDEBUG,LUNI,LUNO)
*
*     Open connection to FATCAT...
*
      CALL CZOPEN('zserv','FATCAT',IRC)

    1 CALL VMCMS('EXEC FATSERV',IRC)
      IF(IRC.EQ.3) GOTO 2
      IF(IRC.NE.0) THEN
         PRINT *,'FATCAT. error ',IRC,' from FATSERV. Stopping...'
         GOTO 99
      ENDIF

*
*     Get the user and node name for this file...
*
      CALL VMCMS('GLOBALV SELECT *EXEC STACK FATADDR',IC)
      CALL VMRTRM(LINE,IEND)
      ISTART = ICFNBL(LINE,1,IEND)
      CALL FMWORD(FATUSR,0,' ',LINE(ISTART:IEND),IC)
      LFAT   = LENOCC(FATUSR)
      CALL FMWORD(FATNOD,1,' ',LINE(ISTART:IEND),IC)
      LNOD   = LENOCC(FATNOD)

      PRINT *,'FATCAT. Update received from ',FATUSR(1:LFAT), ' at ',
     +         FATNOD(1:LNOD)

      CALL DATIME(ID,IT)
      WRITE(CHTIME,'(I6.6,I4.4,I2.2)') ID,IT,IS(6)
*
*    Now put this file...
*    This assumes the FATCAT naming convention: /fatmen/fmgroup,
*                                          e.g. /fatmen/fml3
*
      REMOTE = '/fatmen/'//FATUSR(1:LFAT)//
     +         '/todo/'//FATUSR(1:LFAT)//'_'
     +         //FATNOD(1:LNOD)//'.'//CHTIME
      LREM   = LENOCC(REMOTE)

      CALL XZPUTA('FATMEN.RDRFILE.A',REMOTE(1:LREM),' ',IC)
      IF(IC.NE.0) THEN
         PRINT *,'FATCAT. error ',IC,' sending update from ',
     +            FATUSR,' at ',FATNOD,' to FATCAT'
         CALL VMCMS('#CP LOGOFF',IC)
      ENDIF
      CALL VMCMS('ERASE FATMEN RDRFILE A',IC)

*
*     Are there any files for us to get?
*
    2 CONTINUE
      ICONT  = 0
      NFILES = 0
      CALL XZLS('/fatmen/fm*/tovm/*',FILES,NMAX,NFILES,ICONT,' ',IC)
      IF(ICONT.NE.0) THEN
         PRINT *,'FATSRV. too many files - excess names ',
     +   'will be flushed'
*
   10    CONTINUE
         CALL CZGETA(CHMAIL,ISTAT)
         LCH = LENOCC(CHMAIL)
         IF(CHMAIL(1:1).EQ.'0') THEN
*
*        Nop
*
         ELSEIF(CHMAIL(1:1).EQ.'1') THEN
         ELSEIF(CHMAIL(1:1).EQ.'2') THEN
            GOTO 10
         ELSEIF(CHMAIL(1:1).EQ.'3') THEN
            IQUEST(1) = 1
            IRC = 1
         ELSEIF(CHMAIL(1:1).EQ.'E') THEN
            IQUEST(1) = -1
            IRC = -1
         ELSEIF(CHMAIL(1:1).EQ.'V') THEN
            GOTO 10
         ELSE
            IQUEST(1) = 1
            IRC = 1
         ENDIF
*
      ENDIF


      DO 3 I=1,NFILES
      LF = LENOCC(FILES(I))
      CALL CLTOU(FILES(I))
*
*     Fix for the case when there are no files...
*
      IF((NFILES.EQ.1).AND.
     +   (INDEX(FILES(I)(1:LF),'DOES NOT EXIST').NE.0)) GOTO 1
*
*     Remote file syntax is /fatmen/fm*/tovm
*
      ISLASH = INDEXB(FILES(I)(1:LF),'/')
      IF(INDEX(FILES(I)(ISLASH+1:LF),FATNOD(1:LNOD)).NE.0) THEN
         PRINT *,'FATCAT. skipping update for ',FATNOD(1:LNOD),
     +           '(',FILES(I)(1:LF),')'
         GOTO 3
      ENDIF
*
*     Get the name of the server for whom this update is intended...
*
      ISTART = INDEX(FILES(I)(1:LF),'/FM') + 1
      IEND   = INDEX(FILES(I)(ISTART:LF),'/')
      REMUSR = FILES(I)(ISTART:ISTART+IEND-2)
      LREM   = LENOCC(REMUSR)

      PRINT *,'FATCAT. update found for ',REMUSR(1:LREM),
     +           '(',FILES(I)(1:LF),')'

      CALL XZGETA('FATMEN.UPDATE.B',FILES(I)(1:LF),' ',IC)
      IF(IC.NE.0) THEN
         PRINT *,'FATCAT. error ',IC,' retrieving update'
         GOTO 99
      ENDIF

      CALL VMCMS('EXEC SENDFILE FATMEN UPDATE B TO '
     +           //REMUSR(1:LREM),IC)

      CALL XZRM(FILES(I)(1:LF),IC)
      IF(IC.NE.0) PRINT *,'FATCAT. error ',IC,' deleting file ',
     +           '(',FILES(I)(1:LF),')'

3     CONTINUE
*
*     Wait for some action...
*
      GOTO 1

   99 CALL CZCLOS(ISTAT)
      END
\end{XMPt}
