%.*****************************************************************>
%.* CSPACK Reference Guide - Installation Guide                   *>
%.*  Last Mod.   22 Jun 1993  12.00   mg                          *>
%.*****************************************************************>
\Filename{H1cspinst-Availability-of-CSPACK-at-CERN}
\chapter{Availability of CSPACK at CERN}
\par
The CSPACK PAM file is available on all central CERN systems
in the normal place, e.g. on the CERNPAMS disk on CERNVM,
in /cern/pro/pam on Unix systems and in CERN:<PRO.PAM> on VAX/VMS.
\par
The CSPACK FORTRAN callable interface routines are installed in PACKLIB
on all systems which have PACKLIB corresponding to CERN Computer
News Letter 200 or above. The tools (ZFTP, ZSERV, PAWSERV, SYSREQ, TELNETG)
are installed on all central systems with the following exceptions:
\begin{OL}
\item
TELNETG is available on Unix and VAX/VMS systems only
\item
SYSREQ requires Wollongong or Multinet TCP/IP (VAX/VMS systems).
Note that SYSREQ is currently only used (in the context of CSPACK)
for remote access to the CERN TMS system.
On VM/CMS systems, IUCREQ (SRQSRV) should be used.
\end{OL}
\Filename{H1cspinst-Installing-and-using-the-CSPACK-package}
\chapter{Installing and using the CSPACK package}
\par
All of the routines and programs that compose CSPACK are installed used
the standard CERNLIB installation tools. The general procedure is:
\begin{OL}
\item
Generate PACKLIB (e.g. MAKEPACK -l PACKLIB)
\item
Generate the tools (e.g. MAKEPACK ZFTP, MAKEPACK ZSERV etc.)
\item
Configure the system files
\end{OL}
More information on the CERNLIB installation procedures can be
found on the INSTALL PAM in the deck UGUIDE for the relevant machine.
For example, on Unix systems, the deck is in PATCH=DUNIX,D=UGUIDE,
for VAX systems, the deck is in PATCH=DVAX,D=UGUIDE etc.
\par
The link procedure for the generation of user-developed client-server
programs should be based on that of ZFTP and ZSERV for the relevant
machines. In all cases, the CERNLIB installation procedure
should be followed.
\Filename{H2cspinst-Configuration-for-use-with-TCPAW}
\section{Configuration for use with TCPAW}
\par
To use the CSPACK tools with TCPAW as the network layer, files
on both the client and server side must be correctly configured.
Firstly, the TCP port numbers and associated services must be
defined. Secondly, in the case that incoming connections are
allowed (and possible), the programs to be run for each known
service must be defined. The following table lists the current
values and names used at CERN. Note that these definitions
must be made on both client and server and must match.
On Unix these definitions are made by adding a line to the
file {\tt/etc/services} as follows:
\begin{XMP}
pawserv  345/tcp   # Comment string, e.g. 'For distributed PAW'
\end{XMP}
For VAX/VMS and other systems, see the relevant system specific
details.
\begin{table}[h]
\caption{Service names and TCP ports used by CSPACK}
\begin{tabularx}{\textwidth}{|X|X|X|}
\hline
\bf Service   & \bf TCP port    &  \bf Description  \\
\hline
\bf pawserv   &{\tt 345}        &   Distributed PAW  \\
\bf zserv     &{\tt 346}        & Server for ZFTP, remote Zebra \\
\bf fatserv   &{\tt 347}        & Server for FATMEN  \\
\hline
\end{tabularx}
\end{table}
\par
These ports have been registered with the Internet Assigned Number Authority
at ISI.
\par
To permit incoming connections, a definition in the relevant
services file must be made. This is typically in {\bf\it /etc/inetd.conf}
 for Unix systems, although in the case of the Silicon Graphics the
file is in {\bf\it /usr/etc/inetd.conf.}
The information to be added to this file consists of one line
per service, specifying the service name and program to be run, e.g.
\begin{XMP}
zserv stream tcp nowait root /cern/pro/bin/zserv zserv
\end{XMP}
\subsection{Unix specific details}
\par
After modifying the {\tt /etc/inetd.conf} file, the inetd must
be told to re-read the file. This can be done by rebooting the
system, or by sending a hangup signal.
\begin{table}[h]
\caption{Signalling inetd to reread the /etc/inetd.conf file}
\begin{tabularx}{\textwidth}{|X|X|}
\hline
\bf System   & \bf Command          \\
\hline
\bf AIX      & \bf refresh -s inetd \\
\bf HP/UX    & \bf inetd -l         \\
\bf Others   & \bf kill -1 pid      \\
\hline
\end{tabularx}
\end{table}
\subsubsection{AIX}
On systems running AIX, it is recommended that the following
line be added to the file {\tt /etc/environment}. This sets
the shell variable {\tt xrf\_messages} to {\tt no} and thus
prevents FORTRAN run-time messages being printed.
\begin{XMP}
xrf_messages=no
\end{XMP}

These messages disturb the protocol used between the ZSERV or
PAWSERV servers and the ZFTP or PAW clients.
\subsection{VAX/VMS specific details}
\par
On VAX/VMS systems, three versions of TCP/IP are currently supported.
The CERNLIB installation procedure automatically performs the correct
link procedure but the configuration of the system files must be performed
manually.
\par
The VAX/VMS version of ZFTP and ZSERV also support connections
via DECnet from other VAX/VMS systems. This is activated by
the -d option, e.g.
\begin{XMP}
$ZFTP VXCRNA -D
\end{XMP}
\subsubsection{DECnet}
\par
No configuration is required for ZFTP. However, ZSERV must be
defined as a known DECnet object, e.g.
\begin{XMP}
MCR NCP
NCP>SET OBJECT ZSERV NUMBER 0 FILE CERN:[PRO.EXE]ZSERV
NCP>DEF OBJECT ZSERV NUMBER 0 FILE CERN:[PRO.EXE]ZSERV
NCP>EXIT
\end{XMP}
\par
The ZSERV program automatically detects whether the incoming
request is via DECnet or TCP/IP and acts accordingly.
\subsubsection{DEC UCX}
\par
With the current version of the DEC UCX product,
an Internet Daemon (inetd) is not provided,
hence incoming connections are not possible.
Thus there is no equivalent to the {\tt /etc/inetd.conf} file.
\par
Furthermore, the library routine GETSERVBYNAME
returns -1, indicating 'function not implemented'.
Until this function is included in the UCX library,
the routine GETSERVBYNAME from PATCH TCPAW in the CSPACK
pam file may be used. This code is activated by selecting
+USE,UCX in the PATCHY step of the installation procedure.
An example configuration file for use with this routine
is available in P=CONFIG,D=SERVICES.
\subsubsection{Wollongong}
\par
In the case of Wollong the equivalent of {\tt/etc/services} is
{\tt TWG\$TCP:<NETDIST.ETC>SERVICES.} The file format
is the same for Unix systems, thus an entry such as
\begin{XMP}
zserv 346/tcp
\end{XMP}
should be made.
\par
The equivalent to the {\bf\it /etc/inetd.conf} is the file
{\tt  TWG\$TCP:<NETDIST.ETC>SERVERS.DAT}. An example entry is
shown below.
\begin{XMP}
service-name    Pawserv
program         CERN:<PRO.EXE>PAWSERV.EXE
socket-type     SOCK_STREAM
socket-options  SO_ACCEPTCONN | SO_KEEPALIVE
socket-address  AF_INET , 345
working-set     300
INIT            TCP_Init
LISTEN          TCP_Listen
CONNECTED       TCP_Connected
SERVICE         Run_Program

\end{XMP}
\subsubsection{Multinet}
\par
On systems running MULTINET TCP/IP, the equivalent file to
{\tt /etc/services} is (somewhat confusingly)
MULTINET:HOSTS.LOCAL. Entries should be added to this file in
the format
\begin{XMP}
SERVICE : TCP : 345 : PAWSERV :
\end{XMP}
After making changes to this file, it should be compiled using the
command
\begin{XMP}
MULTINET HOST_TABLE COMPILE
\end{XMP}
To activate these changes without restarting the system, type
\begin{XMP}
@MULTINET:INSTALL_DATABASES
\end{XMP}
\begin{XMP}
@MULTINET:START_SERVER
\end{XMP}
\par
To define servers to Multinet, use the command
\begin{XMP}
MULTINET CONFIGURE /SERVERS
\end{XMP}
An example dialogue is given below:
\begin{XMP}
SERVER-CONFIG>add zserv
Protocol: [TCP] tcp
TCP Port number: 346
Program to run: CERN:[PRO.EXE]ZSERV.EXE
SERVER-CONFIG>RESTART
\end{XMP}
\par
More details on configuration MULTINET may be found in
the Multinet System Administrator's Guide, including how to
restrict access to certain services etc.
\subsection{VM/CMS specific details}
\par
On VM/CMS systems, two versions of TCPAW exist. The recommended
version is the same as used on other systems, but requires the
IBM SAA C compiler and IBM's TCP/IP version 2 or higher.
When using this version, which is activated by selecting the PATCHY
flag TCPSOCK (performed by default in the CERN Program Library
installation cradles), the file ETC SERVICES, which resides on
the TCP/IP installation disk, must be modified to include
definitions of the required services (ZSERV, PAWSERV) as
for Unix systems.
\par
If TCPSOCK is de-selected, e.g. +USE,TCPSOCK,T=INHIBIT, then
the older PASCAL version of TCPAW is activated. This version
has the definitions of zserv and pawserv hard-coded (to ports
346 and 345 respectively).
\par
There are some limitations with the PASCAL version,
the most significant of which is the
fact that connections between two VM systems is not currently possible.
\par
Other limitations that currently exist for VM/CMS systems include:
\begin{OL}
\item
Servers are started using a different technique to other systems.
It is for this reason that the remote system must know that it is
talking to a VM system.
Use the option -V on the open command, e.g. ZFTP vmnode -V in
such cases.
\item
Due to limitations of VM/CMS, the username specified when starting
a server on VM systems must not be currently in use ('logged on').
\end{OL}
