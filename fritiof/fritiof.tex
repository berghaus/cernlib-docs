\documentstyle{article}
%%\textwidth 432pt
\textwidth 440pt
%%\flushbottom
\raggedbottom
\textheight 594pt
\topmargin 72pt
\headheight 0pt
\headsep 0pt
\footskip 54pt
\oddsidemargin 0pt
\parindent 0in
\parskip 3ex
\renewcommand{\textfraction}{0.2}    %float (figures) parameters
\renewcommand{\topfraction}{0.8}
\renewcommand{\bottomfraction}{0.4}
\renewcommand{\floatpagefraction}{0.8}

\newcommand{\ee}{$e^{+}e^{-}$}
\newcommand{\lsim}{\:\raise -4pt\hbox{$\stackrel{\textstyle <} {\sim}$}\:}
\newcommand{\ipt}{ {\cal P}_T }

\begin{document}
\begin{titlepage}

%%%\begin{flushright}
%%% January 1992

%%% LU TP 91-28
%%%\end{flushright}

\vspace{0.2in}
\LARGE
%
\begin{center}
{\bf AN EVENT GENERATOR \\FOR INTERACTIONS BETWEEN HADRONS \\AND NUCLEI 
- FRITIOF VERSION 7 }\\ 
\vspace{.5in}
%
\large
Hong Pi\footnote{pihong@thep.lu.se (internet), thephp@seldc52 (bitnet) }\\
\vspace{0.2in}
Department of Theoretical Physics, University of Lund\\
S\"{o}lvegatan 14A, S-22362 Lund, Sweden\\
%
\vspace{0.5in}
Submitted to {\bf Computer Physics Communications}\\
\vspace{0.2in}
\end{center}


\large 
\vspace{0.4in}
%%%\input frabstr.tex
Abstract:

FRITIOF is a Monte Carlo program that implements the Lund string dynamics model for hadron-hadron, hadron-nucleus and nucleus-nucleus collisions.  
This new version extends from the original low-$P_T$ 
constructions to include the hard scattering effects and a more refined
treatment of the QCD gluon radiation in terms of a soft radiation model in
the colour dipole approximation.  This paper includes a brief presentation
of the model and a description of the program.


\end{titlepage}

\section*{PROGRAM SUMMARY}

\normalsize

{\it Title of Program:} FRITIOF version 7 \\

{\it Catalogue number:} \\

{\it Program obtainable from:} pihong@thep.lu.se (internet),
thephp@seldc52 (bitnet) \\

{\it Computer for which the programme is designed:} DECstation,
VAX, IBM and others with a FORTRAN 77 compiler \\

{\it Computer:} DECstation 3100; installation: Department of
Theoretical Physics, University of Lund, Sweden \\

{\it Operating system:} ULTRIX RISC 4.2 \\

{\it Program language used:} FORTRAN 77 \\

{\it High speed storage required:} $\approx 90$k words\\

{\it No. of bits in a word:} 32 \\

{\it Peripherals used:} terminal for input, terminal or printer
for output \\

{\it No. of lines in combined program and test deck:} 5200 \\

{\it Keywords:} Monte Carlo, hadrons, nuclei, interactions, gluon radiation,
                Rutherford parton scattering, fragmentation. \\

{\it Nature of the physical problem:} In high energy hadron-hadron, 
 hadron-nucleus and nucleus-nucleus collisions multi particle final states are
 produced.  The bulk of the particles produced are
 at low $P_T$.  Strong interaction processes are 
 generally non-perturbative and can not yet be fully understood  
 in terms of quantum chromodynamics (QCD).  As energy gets higher, 
 some high-$P_T$ phenomena begin to appear in accordance with the
 perturbative calculations in QCD.  
 The problem is to consistently connect the high $P_T$ Rutherford scattering
 and the ensuing gluonic bremsstrahlung with a nonperturbative model for the low  
 $P_T$ phenomena. \\
 \\

{\it Method of solution:} 
 A collision between two hadrons is modelled by many momentum transfers.
 The possibility of one of the momentum transfers corresponding to large $P_T$
 scattering is properly treated according to QCD.  
 After the exchange of momenta the
 hadrons are assumed to become two excited string states, which emit
 further gluonic radiations in a colour dipole approach to the QCD
 parton branching.  The final hadronization is performed by using the Lund 
 string fragmentation model.  Collisions with nuclei are assumed to
 involve only independent collisions between the constituent nucleons. \\ 

{\it Restriction of complexity of the problem:}
 The program is not supposed to be applicable at collision 
 centre-of-mass energies ($\sqrt{s}$) below
 10 GeV.  At very high energies ($\sqrt{s}$ in the TeV range), especially
 for heavy ion collisions, certain arrays need to be expanded to
 accommodate the large number of particles produced.   

{\it Typical running time:}
 Depends on the type of collision and energy.  Some examples:
 \begin{tabbing}
 $^{16}$O $+$ $^{197}$ Au \qquad \= $p_{\rm lab} = 200$ A GeV/$c$:\qquad \=
 2 events/min (central collision)      \kill
 $pp$ \> $\sqrt{s}=30$ GeV: \> $\sim$ 90 events/min \\
 $\bar pp$ \> $\sqrt{s}=900$ GeV: \> $\sim$ 70 events/min \\
 $\bar pp$ \> $\sqrt{s}=40$ TeV: \> $\sim$ 30 events/min \\
 $^{16}$O $+$ $^{197}Au$ \> $p_{\rm lab} = 200$ A GeV/$c$: \> 
 $\sim$ 6 events/min     
 \end{tabbing}

\vspace{0.3in}


%%%%\large
\normalsize

\newpage
%%%%\baselineskip 0.8cm

%%%%\input frintro.tex
\section*{LONG WRITE-UP}

\section{Introduction}

Hadronic interactions are complex processes that involve many strong and
electro-weak phenomena.  Although some of the more exotic parts of the 
processes, such as 
the production of high-$P_T$ jets or W, Z particles, involve a large mass
scale and are well described by perturbative QCD and electro-weak theory, the
bulk of the event structure is non-perturbative in nature and can not yet be
calculated from first principles.  Current understanding of hadronic physics
is therefore facilitated to a large extent by phenomenological models.  The Monte Carlo
implementations of such models are particularly useful to test various 
theoretical ideas.  Good Monte Carlo generators are also invaluable to
the implementations at various stages of an actual experiment, from its initial
design to the final data analysis.

FRITIOF is a Monte Carlo implementation \cite{fr16} of a model \cite{bogn} 
\cite{bogns} for hadron-hadron, hadron-nucleus and nucleus-nucleus collisions proposed some five   years ago.  The basic idea of the model is in a simple picture that a hadron behaves like a
relativistic string with a confined colour field similar to the vortex line
in a type II superconductor embedded in a superconducting vacuum.  The field of such a vortex line is equivalent to that of a chain of dipoles lined up along the vortex line.  The dipole links act as partons.  During a soft interaction many
small transverse momenta are exchanged between the dipole links and two
longitudinally excited string states result from the collision.  Disturbance of the colour field will in general initialise gluonic radiation 
according to QCD, which can then be naturally incorporated in this picture
by the colour dipole approximation \cite{dipol}.  The final state particles are obtained by
fragmenting the string states like the usual strings \cite{lstri} in a $e^+e^-$ 
annihilation.

Over the years this model and its extensions to nuclear collisions have been
rather successful in describing a large amount of experimental data.  However,
using the soft interaction picture as outlined above to describe a complete 
hadronic interaction can only be acceptable at collision centre-of-mass energies of no larger than a few tens of GeV.  At higher energies the occurrence of high-$P_T$ jets and the onset of the minijet phenomena \cite{mjets} 
clearly dictate new inputs to the model.
This version of FRITIOF represents an effort to extend the applicability of
the model into the TeV energy regime.  It differs from the previous versions
mainly by its inclusion of hard interaction effects through PYTHIA \cite{pyth} and a more refined treatment of gluonic radiations through ARIADNE \cite{aria}.

The earliest version of FRITIOF was the version 1.6 written
in 1986 by Nilsson-Almqvist and Stenlund.  A program manual was then 
published \cite{fr16}.  However since then the program has been modified 
a number of times under different authorships.  Multiple gluon emission 
was added to the program, and a more refined nuclear collision geometry
\cite{ngeo} was put in later.  Earlier attempt was also made to implement 
the hard interaction physics into the model \cite{ohard}.  FRITIOF 6.0 was
the most recent version upon which this current program is based.  
It should be noted that despite of the numerous evolutions with FRITIOF,
no updated program manual was written since the earliest publication.
Therefore we feel it is a necessity to have a carefully written documentation
for this new version.  
  
A detailed description of the model and its comparison with data can be found in ref.\,\cite{prep1}.  The theoretical construction of the model is briefly presented here in section 2.  The extension of the model to collisions  with nuclei is summarised in section 3.  Descriptions of the program components are given in section 4.  An example on how to use the program is given
in the appendix.


\section{Hadron-hadron interaction}

\subsection{Soft momentum transfer}

In FRITIOF a interacting hadron is looked upon as a string-like object,
with its colour force field stretching like a vortex line in a type II superconductor.  A vortex line has a hard core surrounded by an exponentially
damped field.  Two hadrons interact with each other as their fields overlap,
and momentum transfers take place.  It is assumed that no net colour is
exchanged between the hadrons despite of the momentum transfers.  So two hadrons
with mass $m_1$, $m_2$ and with incoming light-cone momenta
\begin{equation}
\label{pini}
 P_1^i = (p_+, {m_1^2\over p_+}, \vec{0}_T),\;\;
 P_2^i = ({m_2^2\over p_-}, p_-, \vec{0}_T)
\end{equation}
will emerge as two excited (colour singlet) states after the interaction with new momenta
\begin{equation}
\label{pfin}
 P_1^f = (p_++{m_2^2\over p_-}-P_+^f, P_-^f, \vec{Q}_T),\;\;
 P_2^f = (P_+^f, p_-+{m_1^2\over p_+}-P_-^f, -\vec{Q}_T)
\end{equation}
where the final state components $P_-^f$ and $P_+^f$ are related to the
total momentum transfer $Q$ by
\begin{equation}
\label{pfinq}
 P_-^f = {m_1^2\over p_+} + Q_-,\;\;
 P_+^f = {m_2^2\over p_-} + Q_+.
\end{equation}
Assuming incoherence in the collision process, the total momentum transfer $Q$ in FRITIOF is a sum of many momentum exchanges among the quanta of colour field, 
\begin{equation}
\label{qkj}
Q=\sum_j k_j.    
\end{equation}
Furthermore if the quanta  
obey Feynman's wee parton spectrum the total longitudinal momentum
components should follow the distribution
\begin{equation}
\label{pdist}
 dP \propto {dP_-^f\over P_-^f} {dP_+^f\over P_+^f}.
\end{equation}
The transverse components of the $k_j$'s are in general rather small and pointing
at random directions.  Similar to a random walk in the 
transverse plane, $\vec{Q}_T=\sum \vec{k}_{jT}$ is expected to obey a
Gaussian distribution with the average $<Q_T>$ on the order of hadronic mass  scale.  This completes the FRITIOF description of the momentum transfers in soft hadronic
interaction.  There are also situations when one or more of the
$k_j$'s have a large transverse component.  This corresponds to Rutherford  parton-parton scattering.   
One can also expect that the Gaussian spectrum for the
soft transverse momentum exchange $Q_T$ should acquire a tail that joins smoothly
with the high-$P_T$ QCD spectrum.  These effects will be discussed 
in section 2.2. 

The phase space for $P_-^f$ and $P_+^f$ are specified by the kinematical
constraints
\begin{eqnarray}
\label{pphas}
 P_-^f (p_++{m_2^2\over p_-} - P_+^f) &\geq& m_1^{\prime 2} + Q_T^2, \nonumber\\
 P_+^f (p_-+{m_1^2\over p_+} - P_-^f) &\geq& m_2^{\prime 2} + Q_T^2,
\end{eqnarray}
where $m_1^{\prime }$ and $m_2^{\prime }$ are the minimum masses for
the excited hadron states.  $m_1^{\prime }$ and $m_2^{\prime }$ will be assumed
to take the values of the initial hadron masses.
Technically there are some problems in fragmenting strings of small 
invariant masses.  To avoid such problems a mass parameter $m_d$, which is
somewhat larger than $m^\prime$,  
is used in the program.  A string
with mass less than $m_d$ is not treated with string fragmentation,
instead it is simply set back to the original hadron mass shell.  
Such type of events resemble the diffractive events observed experimentally.
For this reason $m_d$ will be called ``diffractive mass''.  Obviously,
only the diffractive and low multiplicity event samples would be
sensitive to this parameter.  For most applications $m_d$ will have little
effect.  

\subsection{Rutherford parton-parton scattering}

When one of the momentum transfers
in eq.\,(\ref{qkj}) has a large transverse component, it corresponds to
Rutherford parton-parton scattering (RPS).  
The differential cross section for
the elastic parton-parton scattering is giving by perturbative QCD 
\begin{eqnarray}
\label{pqcd}
q_1^+q_2^- {{d\sigma}\over{dq_1^+dq_2^-dq_T^2}} &=&
\sum_{ij} x_1f_i(x_1,Q^2)\, x_2f_j(x_2,Q^2) 
\,{d\hat\sigma_{ij}\over d\hat t}(\hat s,\hat t,\hat u) \nonumber \\
& &\cdot\, \delta(q_T^2-{\hat t\hat u\over\hat s}) 
\,\delta(x_1\sqrt{s}-q_1^+-{q_T^2\over q_2^-})
\,\delta(x_2\sqrt{s}-q_2^--{q_T^2\over q_1^+}),
\end{eqnarray}
where 
$\hat s$, $\hat t$ and $\hat u$ are the usual Mandelstam variables describing
the kinematics of the parton process, and $s$ is the total centre-of-mass
energy squared of the hadron-hadron system.  
The $q_1^+$ and $q_2^-$ are the light-cone momentum components of the two outgoing partons.  The partons are taken to be massless and on shell.
This QCD spectrum is dominated
by contributions from the semihard regime where the parton energy fraction 
$x$ is small, and the dominating structure function behave roughly as
$f(x)\sim 1/x$, which is essentially the Feynman's spectrum for wee partons.
This means that the distribution in eq.\,(\ref{pdist}) should be preserved 
even when hard scattering occurs.  A truly hard scattering gives a strong
kick to the string end or interior, and produces gluon kinks on the string.
The momentum for the string as a whole should then satisfy the condition 
\begin{equation}
\label{pphasq}
P_-^f  > q_1^-, \;\;
P_+^f  > q_2^+
\end{equation}
in addition to the restraints in eq.\,(\ref{pphas}).
Furthermore 
the Gaussian behaviour of the transverse component of $\vec{Q}_T$ will be superimposed by a hard $\vec{q}_T$ given by eq.\,(\ref{pqcd}).

The applicability of eq.\,(\ref{pqcd}) is limited under strong kinematical
constraints as $P_T$ gets smaller, and it diverges in the infrared limit.
The way to handle these difficulties is crucial to a model that deals with
hard as well as soft interactions.  The divergence problem can be handled
by a probabilistic approach involving a Sudakov probability factor,
such that the distribution for the hardest RPS is given by
\begin{equation}
\label{probq}
Prob(q_T) = \rho(q_T) \exp\left(-\int_{q_T}^{\sqrt{s}/2} dq_T^\prime 
 \rho(q_T^\prime ) \right),
\end{equation}
where $\rho(q_T) = (d\sigma/{dq_T})/\sigma_{nd}(s)$ with $\sigma_{nd}(s)$
being the hadron-hadron non-diffractive inelastic cross section.
The Sudakov factor in the exponential function is interpreted as 
the probability of having no RPS harder than $q_T$.  Such an approach
can be found in ref.\,\cite{multisj} which has been implemented in
the PYTHIA program \cite{pyth}.  FRITIOF uses PYTHIA to generate the
Rutherford parton-parton scattering.  Since the event structure is
dominated by the hardest momentum transfer, FRITIOF includes only
the hardest RPS generated according to eq.\,(\ref{probq}).     
 
With the above approach we obtain a unitarized RPS distribution.  Still not
all the RPS generated from eq.\,(\ref{probq}) should be accepted.  
During the collision process there is colour excitation in the nucleon
states and consequently there is gluon bremsstrahlung radiation according to QCD.
In the picture of QCD parton model eq.\,(\ref{pqcd}) is applicable only
to the momentum transfer which is hardest compared with the associated
bremsstrahlung gluons.  We therefore adopt a procedure that compares
the ``hardness'' of the Rutherford partons generated from eq.\,(\ref{probq}) 
to that of the bremsstrahlung gluons.  The RPS is accepted only if
it is harder than the associated radiation.  If the RPS is ``drowned'', 
which is to say that it is softer than the radiation, then the RPS is not
acceptable and the collision proceeds as a purely soft collision. 
With this prescription the RPS spectrum is suppressed smoothly at small
to medium $q_T$.  This ``drowning'' mechanism is critical to the merging 
of hard and soft interactions in FRITIOF.    
The treatment of gluon radiation and the prescription
for the RPS drowning will be presented in more detail in the next section.  

To make it clearer to see the difference between a RPS event and a purely
soft event, 
it is useful to write the momentum of the nucleon as a sum of two terms
\begin{equation}
\label{psoft}
P^f = P_{soft} + q,
\end{equation}
where $q$ is the momentum of the hard scattered parton, and  $P_{soft}$ is the momentum of the remnant.  In this paper $q$ will be referred as the
``hard momentum'' and $P_{soft}$ will be termed the ``soft momentum''.  
In terms of $P_{soft}$, the
distribution in eq.\,(\ref{pdist}) can be rewritten as
\begin{equation}
\label{pdists}
 dP \propto {{dP^-_{soft,1}}\over {P^-_{soft,1}+q_1^-}} 
            {{dP^+_{soft,2}}\over {P^+_{soft,2}+q_2^+}}. 
\end{equation}
This means that RPS introduces a bias in the soft momentum distribution.  Collisions having very hard scattering would in general also
have larger than usual components of $P_{soft}$.  This feature in FRITIOF
results in, for example, higher than average multiplicity and transverse
energy flows in events containing RPS.


\subsection{The state of the excited strings and the gluon radiation}

The outcome of the interaction prescribed above is two
excited colour singlet string states.  In the absence of hard scattering or 
gluon radiations such states are just two longitudinal string states
with the valence constituents of the hadrons (quark-antiquark for mesons and 
quark-diquark for baryons) attached to the end points.  The end points
will be given a (primordial) transverse momentum distributed according to
a Gaussian.  The momenta for the two end partons $Q_1$ and $Q_2$ can be easily 
figured out given that they share the total soft momentum of the system.
For instance for a projectile nucleon, the plus component for the diquark
momentum is 
\begin{eqnarray}
\label{q1q2}
Q_2^+ &=&{1\over{2P^-_{soft}}} 
\left\{ ( P^+_{soft}P^-_{soft} + Q^2_{2T} -Q^2_{1T}) \right. \nonumber \\  
  & & +[ (P^+_{soft}P^-_{soft} + Q^2_{2T} -Q^2_{1T})^2           
  - \left. 4P^+_{soft}P^-_{soft}Q^2_{2T} ]^{1/2} \right\} 
\end{eqnarray}
where the transverse momentum $Q_{2T}$ is generated from a
Gaussian distribution.  
The inclusion of
multiple gluon emission or the hard scattering produces ``kinks" on the
string which results in ``bent" strings.  As energy increases such effects
become stronger, and so it is of critical importance to treat them properly.

A high mass string state will have associated gluon bremsstrahlung.  Such 
emission of multiple gluons can be treated conveniently as a parton
branching process. In FRITIOF we use 
the colour dipole approximation \cite{dipol}, 
{\it i.e.}, a string is treated as a colour
dipole, it splits into two dipoles after emitting one gluon, and the two
dipoles go on to emit gluons independently and more dipoles are formed
to result in a cascading process.  Such a scheme has already been 
implemented with success
in the ARIADNE Monte Carlo program \cite{aria}, 
which is then incorporated into FRITIOF
to perform the gluon bremsstrahlung radiation.

Complication arises since a hadron is an extended object. During interaction
it is acted upon by many small momentum transfers over an extended space-time
region, and therefore the source responsible for the emission has certain finite
size rather than being point-like as in $e^+e^-$ annihilation.  
Radiation from an extended source is treated in FRITIOF in terms of
the Soft Radiation Model (SRM) \cite{exten}.  In SRM, it is assumed that,     
similar to an emitting antenna, radiation of wavelengths shorter than the
source size would be damped due to destructive interference.  In a Lorenz
frame where a gluon with transverse momentum $k_T$ is emitted at a right
angle to momenta of the string end points, the gluon wavelength is $\lambda\sim 1/k_T$.  Introducing
a parameter $\mu_0$ to characterise the inverse of the source size, then only
a piece of the source, with energy fraction
\begin{equation}
\label{akt}
a(k_T) = \left({\mu_0\over k_T}\right)^\alpha,
\end{equation}
can radiate coherently, where $\alpha$ is a parameter characterising the
dimensionality of the source.  This mechanism \cite{exten} reduces the available phase space for radiation from an extended source as compared to a true point energy source.   The phase space for the emitted gluon can be
represented approximately in the rest frame of the string by
\begin{equation}
\label{akt1}
k_T e^{\pm y} \lsim a(k_T) M, 
\end{equation}
where $M$ is the invariant mass of the emitting dipole.   
In FRITIOF $\alpha$ is set to unity, consistent with the
assumption that the energy density has one-dimensional string-like shape.
$\mu_0$ is an adjustable parameter in the model, and its value is around 
0.7 GeV corresponding to a transverse extension of the source
$\pi/\mu\sim 0.9 fm$.  

For a string that comes out of a purely soft
collision, the excited string as a whole acts coherently for
gluon emission, and here the inverse of $\mu_0$ represents the transverse 
extension of the whole string.  There are also situations when the two
string ends should be treated separately, with each end having its own
extension.   The two ends are then 
characterised by two numbers $\mu_1$ and $\mu_2$.  The suppression of
phase space becomes $k^2_Te^{-y}\lsim \mu_1 M$, $k^2_Te^{y}\lsim \mu_2 M$.
In FRITIOF such situation arises in a RPS event.  The presence of a Rutherford
gluon effectively breaks down the colour coherence on the string as a whole, 
and the two string ends act independent of each other as coherent emitters.
$mu_1$ and $\mu_2$ are interrelated here as the extensions of the two string
ends are expected to add up to give the extension of the whole string.
If $r$ is the fraction of the total extension at one end, then 
\begin{equation}
\label{mu12}
\mu_1 = \mu_0/r ,  \qquad \mu_2 = \mu_0/(1-r) 
\end{equation}
with $r \in (0,1)$.  
The choice of $r$ is rather arbitrary.  
In a simple scenario one can expect that the two string ends each takes half
of the string transverse extension and therefore $r\approx 1/2$.
One can also imagine that the extensions of the two ends are not exactly
half and half but rather they are distributed in certain way that relates
to the overlap of the two strings.  $r$ may be uniformly distributed 
for example.   
The model is not very sensitive to these choices.  

In a RPS event, hard scattered partons represent well localised energy
density on the string, and the emission from this parton is not subject
to the suppression of the SRM.  The large and localised momentum transfer
would also bend the string, so that gluon kinks are formed 
before the cascade develops.   
The presence of these
kinks gives extra folds of phase space to the gluon radiation, which means
more gluonic activities in the cascade.  
In the following we discuss
the situation with valence quarks and gluons.  
Sea quarks represent a relatively
small contribution in comparison to the gluons and in the 
current FRITIOF they are treated simply as gluons. 

In case the hard scattered parton is a valence quark, which always
corresponds to a string end in FRITIOF, the string end
suffers a violent kick and 
the string is
bent and consequently a gluon kink is formed on the string.  The
string configuration, taken a baryon as an example, is then  
$q_h$-$g_s$-$qq_s$, where the symbol $q_h$ represents a hard quark, $g_s$ a soft gluon kink and $qq_s$ a soft diquark.  
The word ``soft'' here means that
they will be treated in the Soft Radiation Model as extended objects.
The soft momenta are partitioned as in eq.\,(\ref{q1q2}) and so
$g_s$ has momentum $Q_1$ and $qq_s$ has momentum $Q_2$.  

At high energy RPS is dominated by gluon scattering.  A hard scattered gluon
is expected to stick out somewhere in the interior of the extended region 
which was acted upon by many soft momentum transfers.  
This region would then be split into two parts by the presence of the hard gluon.
This results in a kink in the interior of the string which
acts as an extra gluon.   
The string configuration becomes $q_s$-$g_h$-$g_s$-$qq_s$.  Here $q_s$ 
is the trailing quark end,
$g_h$ is the hard gluon,
$g_s$ is the soft gluon kink 
and $qq_s$ is the leading diquark end.  
As usual the soft momentum $P_{soft}$ is partitioned into two terms
$Q_1$ and $Q_2$. The diquark $qq_s$ has the momentum $Q_2$, while the
the kink $g_s$ takes a fraction $r^\prime$ of the momentum $Q_1$, and
the quark $q_s$ gets a momentum $(1-r^\prime) Q_1$.
This energy fraction $r^\prime$ is assumed to take a distribution
\begin{equation}
\label{rkink}
{d r^\prime\over {r^\prime+{\textstyle q_-\over \textstyle P^f_-}} }  
\end{equation}
for the projectile string, and a corresponding expression with the plus
momentum components is used for the target.  
where $q_T$ is the transverse momentum of the Rutherford gluon and 
$M_T$ is the transverse mass of the string system.  
We further assume the extension of the string piece is proportional to
its energy.  The three extended partons are then assigned the following
values to the SRM suppression parameter:
$\mu_0/r(1-r^\prime)$ for $q_s$, $\mu_0/rr^\prime$ for $g_s$ and 
$\mu_0/(1-r)$ for $qq_s$.  

We have specified the possible string configurations as a result of
the hadronic collision.  Such string states will then proceed to
a dipole cascade emission.   As mentioned in section 2.3, the bremsstrahlung
gluons provide a background noise in which the Rutherford partons may
become ``drowned''.  In case this happens the RPS must be rejected
and replaced by a purely soft event.  To quantify this we need to define
a measure for the ``hardness'' of the Rutherford partons and the bremsstrahlung
gluons.  We use an invariant $\ipt$ defined in terms of the invariant masses
of the parton systems as
\begin{equation}
\label{invpt}
\ipt^2 = {{s_{12}s_{23}}\over{s_{123}}}.
\end{equation}
Here 2 is the parton in question, and 1 and 3 are the two partons neighbouring
to 2 in the dipole chain.  This definition of invariant $\ipt$ is also used
as the ordering variable in the ARIADNE dipole cascade program \cite{aria}.
With this definition, the acceptable RPS must satisfy the condition
\begin{equation}
\label{rpsok}
\ipt^R \geq \ipt^b \geq \ipt^{cut},
\end{equation}
where $\ipt^R$ is the invariant $\ipt$ of the Rutherford parton, 
$\ipt^b$ is the invariant $\ipt$ of the hardest bremsstrahlung gluon emitted
from the string system, and $\ipt^{cut} = 0.5\sim 1$ GeV is the ARIADNE 
cut off parameter for the minimum invariant $\ipt$ in a dipole cascade.
The RPS is drowned when the condition above
is not satisfied.  Most of the RPS with small or moderate $q_T$ will become 
drowned.  In this way a feature of infrared stability is realised in FRITIOF.
The QCD spectrum is largely suppressed at small $q_T$ 
as a result of the drowning.
A cut off parameter $q_{Tmin}$ which is normally needed in connection
to eq.\,(\ref{pqcd}) becomes essentially irrelevant in FRITIOF.  
The effect of this ``drowning'' mechanism on the event structure 
is quite substantial.  As an example
at $\sqrt{s}=900$ GeV, roughly 70\% of the RPS generated from eq.\,(\ref{probq})
become drowned and therefore only 30\% or less of the events can be characterised
as RPS events.  

The dipole cascade stops when the invariant $\ipt$ of the possible emission 
becomes lower than the cut off $\ipt^{cut}$.  
The final step is fragmentation.  The string fragmentation scheme \cite{lstri}
as implemented
in the Lund Monte Carlo JETSET 7.3 \cite{j73} is used.  
The fragmentation parameters
there are tuned to fit the $e^+e^-$ data.  These fragmentation properties,
however, should also be applicable to the strings formed in hadronic collisions
due to the assumption of jet universality.


\section{Hadron-nucleus and nucleus-nucleus collisions}

The generalisation of the model to hadron-nucleus and nucleus-nucleus
collisions is rather straightforward.  At high energies a collision between
nuclei can be regarded as incoherent collisions between their nucleons.
Thus a nucleon from the projectile interacts independently with the
encountered target nucleons as it passes through the nucleus. 
Each of these sub-collision can be treated the same way as a usual
hadron-hadron collision.  A crucial assumption made here is that although
the nucleon becomes excited as a result of consecutive collisions with
target nucleons, it remains essentially a nucleonic state as 
it traverses the target.  This is to say that, on the time scale of
the collision process, the excited nucleonic state does not fragment
in the intermediate stage and so there is no intra nuclear cascade.  
This assumption is reasonable at high energy since the time scale associated
with fragmentation is expected to be much longer due to time dilation
as compared to the traverse time for the nucleon passing through the nucleus.
So in this picture a projectile nucleon passing through the nucleus
will exchange momentum according to eq.\,(\ref{pdist}) each time it encounters
a target nucleon.  If it interacts with $\nu$ nucleons in the target,
$\nu+1$ excited string states will be formed as a result.  These string
states will then have associated bremsstrahlung radiations and finally
fragment into hadrons. 

One important aspect of the collision is the nuclear collision geometry.
This part of the program was originally written by Ding and Stenlund
and is basically unchanged in the current version.  The only modifications
made are the implementation of energy-dependent $N$-$N$ cross sections
and some reorganisation of common blocks.  In this article we present
only briefly the parametrizations used in the program.     
For a detailed description on the nuclear collision geometry, users are 
referred to ref.\,\cite{ngeo}. 
  
During a collision it is assumed that the projectile passes 
through the target nucleus 
on a straight line geometry.  If the nucleons are distributed inside the nucleus
according to $\rho(r)$, the impact parameters of the incident nucleon
to each of the target nucleons can be calculated.  The probability of
whether there will be interaction between two nucleons is related to the impact parameter through the nucleon-nucleon overlap function $G(b)$.  A set of
target nucleons that would interact with the incident nucleon can thus be predetermined.  Here the incident nucleon is assumed to impinge on the
target nucleus at a random impact parameter.

Some modifications are needed for the above prescription when RPS is
included.  When a nucleon system contains a hard parton, the system would
have a rather large $P_T$ and the system centre of mass would then not
move on a straight line afterwards.  This does not cause real difficulties
though.  It is assumed that the hard parton would not have further interaction,
and therefore it is only the remnant that moves forward and exchanges
momentum with the next nucleon.  From this viewpoint the collision geometry
is entirely preserved.  

Since the sub-collisions are assumed to be independent, 
it is possible for each of them to have RPS.  Such multiple hard
scattering is a rather complicated process and a proper treatment
should include a usage of multi parton structure functions.
Within the framework of FRITIOF the treatment is simplified by
assuming that the hard interactions are independent.  Upon
each $N$-$N$ sub-collision, RPS is generated by regarding the excited
(remnant) nucleon state as a ground state nucleon. 
The ``drowning'' procedure can be applied at each $N$-$N$ sub-collision. 
That is, the RPS is accepted only if it is harder than the 
hardest bremsstrahlung gluon.    
It is straightforward to use an iterative procedure
to implement this sequence of $N$-$N$ collisions.  
   
The excited string system now includes the possibility of having several 
hard partons.  
The procedure of configuring the string in the hadron-hadron
interaction scenario can be extended easily to cover this situation. 
We requires that the presence of the extra gluons does not inhibit
the possible gluon radiations.  We therefore first pick up the
hardest gluon and configure the string exactly as in the case for 
hadron-hadron collision.  The cascade would then allowed to develop
and the rest of the gluons are one by one 
inserted into the cascading sequence in a $P_T$ and rapidity ordered fashion.
The assumption of ordering in rapidity is to make the string in some sense
as ``short'' as possible, which is related to the infrared stability on Lund strings.  Other aspects of the model remain 
essentially the same as that for hadron-hadron collisions.

In the following we list some of the parametrizations used in connection
to the nuclear collision geometry.  
The nucleon density distribution used in the program is the Wood-Saxon type
for heavier nuclei with atomic number $A>16$,
\begin{equation}
\label{ror}
\rho(\vec{r}) =  {{\rho_0}\over {1+\exp\left({{\textstyle r-r_0A^{1/3}}\over \textstyle C}\right)}},
\end{equation}
where $r_0$ is a radius parameter, $C$ is the diffuseness parameter,
and $\rho_0$ is a normalisation constant.
$r_0$ has been taken to be slightly A-dependent and is parametrized as 
\begin{equation}
\label{wsr0}
r_0 = 1.16 \/ (1 - 1.16 A^{-2/3}) {\ \rm fm} .
\end{equation}
$C$ is also taken to be slightly A-dependent and its values used in the
program ranging from 0.47-0.55 fm.  For lighter nuclei with $A\leq 16$,
a harmonic oscillator shell model density is used,
\begin{eqnarray}
\label{rorh}
\rho(\vec{r}) & = & {4\over{\pi^{3/2}d^3}}\left[1+{{A-4}\over 6}
 \left({r\over d}\right)^2\right]
 e^{-r^2/d^2},\nonumber \\
    d^2 & = & \left({5\over 2}-{4\over A}\right)^{-1}
              \left(<r^2_{ch}>_A-<r^2_{ch}>_p\right), 
\end{eqnarray}
where $<r^2_{ch}>_A$ and $<r^2_{ch}>_p=0.656$ fm$^2$ are the mean squared charge radii of the the nucleus and proton, respectively.  The values used in the
program for these radius parameters are taken from the measurements of
lepton-nucleus scattering experiments \cite{size}.  The nucleons
in a nucleus are not supposed to come infinitely close to each other.
A parameter $R_{min}$ is used in the program to specify the minimum allowable 
distance between two nucleons.   

For the inelastic overlap function three choices exist in the program:  
\begin{enumerate}
\item eikonal-type overlap function
\begin{eqnarray}
\label{geikon}
G(b) &=& 1 - \exp(-2\Omega(b)), \nonumber \\
\Omega(b) &=& \Omega_0 \exp( - \beta b^2);
\end{eqnarray}
\item Gaussian overlap function
\begin{equation}
\label{ggaus}
G(b) = 1 - \left[1-g_0\exp(-\gamma b^2)\right]^2; 
\end{equation}
\item gray disk 
\begin{equation}
\label{gdisk}
 G(b) = \left\{ \begin{array}{ll}
              \alpha & \mbox{$b<R$} \\
                0    & \mbox{$b>R$}
              \end{array} \right. 
\end{equation}
\end{enumerate}
There are two parameters in each of the parametrizations above.  They
are determined by fitting the total and inelastic cross sections,
\begin{eqnarray}
\label{totel}
\int d^2b\,G(b) &=& \sigma_{inel}, \nonumber \\
2\int d^2b\,\Gamma(b) &=& \sigma_{tot},\qquad  
 {\rm with\ } \Gamma(b) = 1-\sqrt{1-G(b)}. 
\end{eqnarray}
As energy gets beyond the ISR range one needs to take the rising of
the cross sections into account.  In the program the cross section fit No. 8 
and the slope parameter fit No.2 of Block and Cahn \cite{block}  
are used as the input.
The Block and Cahn fit includes the CERN collider data and predicts
a constant total cross section at ultra-high energies.  The total and
elastic cross sections from the fits are in good agreement with data
up to Tevatron energy.    

%%%%\input frprog.tex 
\section{Description of the program}

FRITIOF 7 is a subroutine package for simulating events of hadron-hadron,
hadron-nucleus, or nucleus-nucleus collisions at high energies.  A user
is expected to construct his own main program, in which he must set up
a specific collision environment and kinematics through the variables in
the main subroutine FREVENT and also through certain common blocks.  
One event is generated each time FREVENT is called.  
The full information on the event
is stored in common block LUJETS, through which the user may design
specific trigger condition and perform statistical analysis.  
The subroutine FREVENT and a few common blocks are normally all that
a user needs to know.  These are described in more detail in section 4.1.
Some of the lesser used common blocks are described in section 4.2. 
There are also a large number of service subroutines, which are used
only internally and may not be of interest to most users.  These are listed
nevertheless in section 4.3 for completeness.  

The program works in the following manner. After initialisation, first a set of subroutines are called to generate the nuclear collision geometry.  The designated nucleons are then arranged in subroutine FRRINGO to start off 
the collision sequence.  At this stage FRHARDP is called to generate 
hard scattering and the soft momentum transfers.
After momenta for all the wounded nucleons and their partons have been determined, they are given an 
appropriate string configuration
in subroutine FRATLEO, which calls ARIADNE for gluon 
radiation.  Finally the JETSET routine LUEXEC is called for fragmentation.

The program is written in FORTRAN 77, with the exception that subroutine and
common block names of non-standard lengths are used. 
The random number generator is supplied
by the JETSET function RLU \cite{j73}. 
FRITIOF 7 should be loaded together with
JETSET 7.3 \cite{j73}, PYTHIA 5.5 \cite{pyth}, and ARIADNE 3.3r \cite{aria}
(for FRITIOF 7.01) or ARIADNE 4.02r \cite{arian} (for FRITIOF 7.02).  
To ensure a proper execution the correct version numbers of these companion programs must be observed.  Since all these
Lund programs share a common block LUJETS as the event record, users should
also make sure that the same dimension size is specified for LUJETS in all
the relevant programs on sites.

\begin{flushleft}
\underline{Update history for FRITIOF version 7} 
\end{flushleft}

\begin{itemize}
\item{Version 7.01: A new implementation of Rutherford Parton Scattering is the
 main change as compared to earlier versions.  Adaptations made to run with
 JETSET 7.3, PYTHIA 5.5 and ARIADNE 3.3.   
 }
\item{Version 7.02: ARIADNE version 4.02 is adopted to replace the older version
 3.3.  Gluon splitting into $q\bar q$ pairs and photon emission
 in the dipole cascade, preciously
 forbidden for technical reasons, are now allowed.
 }

\end{itemize}

\subsection{User interface}

This section contains the minimum information a user must know about the program,
namely the main subroutine FREVENT, the common block FRPARA1 for 
input parameters and and switch controls, 
and the event record LUJETS.  The collision
environment is set up by specifying the variables in FREVENT, and then events
are generated by calling FREVENT repeatedly.  

 
\begin{itemize}

\item{SUBROUTINE FREVENT(FRAME,BEAM,TARGET,WIN)}

\begin{description}
\item[Purpose:] to administrate the generation of one complete event.
                 
\item[FRAME:] character variable to specify the frame of the experiment. 
                
\begin{description}
      \item[=]'FIXT' : fixed target experiment, with beam particle momentum
          pointing in $+$z direction. 
      \item[=]'CMS' : centre of momentum frame, with beam momentum in the $+$z
          direction and target momentum in the $-$z direction.
\end{description}

\item[BEAM, TARGET:] character variables to specify the beam and target
                     hadrons and/or nuclei.   
\item[          ] For BEAM and TARGET the following selections are available:
\begin{description}
     \item[ 'P' ]   proton
     \item[ 'D' ]   deuteron	(A=2)
     \item['HE' ]   helium	(A=4)
     \item['BE' ]   beryllium	(A=9)
     \item[ 'B' ]   boron	(A=11)
     \item[ 'C' ]   carbon 	(A=12)
     \item[ 'O' ]   oxygen 	(A=16)
     \item['AL' ]   aluminium 	(A=27)
     \item['SI' ]   silicon 	(A=28)
     \item[ 'S' ]   sulphur 	(A=32)
     \item['AR' ]   argon	(A=40)
     \item['CA' ]   calcium	(A=40)
     \item['CU' ]   copper	(A=64)
     \item['AG' ]   silver	(A=108)
     \item['XE' ]   xenon	(A=131)
     \item[ 'W' ]   Tungsten	(A=184)
     \item['AU' ]   gold	(A=197)
     \item['PB' ]   lead	(A=207)
     \item[ 'U' ]   uranium	(A=238)
\end{description}
 
\item[          ] The following particles are allowed for BEAM only:
\begin{description}
     \item[ 'N'  ] neutron
     \item['PBAR'] anti-proton
     \item['PI$+$' ] positive pion
     \item['PI$-$' ] negative pion
     \item[ 'K$+$' ] positive kaon
     \item[ 'K$-$' ] negative kaon
     \item['NEW1'] user-specified hadron or nucleus. 
               The following additional parameters must be specified by the user
               through common block FRCODES. \\
               If `NEW1' is a hadron, enter its KF code \cite{datag}
               in KCD(1) and its charge in NPROT(1).
               For this `NEW1' hadron, its minimum excitation mass $m^\prime$
               and ``diffractive" excitation mass $m_d$ are set by default
               to the hadron mass.  If adjustment is needed these two
               parameters can be set by user in EXMA(1,1) and EXMA(1,2).   
               Furthermore if `NEW1' is to collide with a nucleus, 
               user may wish to assign the
               `NEW1'-nucleon total and elastic cross sections
               in VFR(17-18) through common block FRPARA1. \\
               If `NEW1' is a nucleus, enter the nuclear
               A and Z parameter in NNUC(1) and NPROT(1) respectively.
    	       The nuclear size parameters must also be specified: 
               for A$\leq 16$, enter the root mean squared charge radius
               $<r^2_{ch}>^{1/2}$ in RO1(1,1);  
               for A$> 16$, specify the values of $r_0$ 
               and $C$ in RO1(1,1) and RO1(1,2). $r_0$ and $C$ are
               the parameters for the Wood-Saxon nucleon density. 
\end{description}

\item[          ] For TARGET only:
\begin{description}
     \item['NEW2'] user-specified nucleus. Additional parameters
                   in the following must be specified by the user. \\
                   Enter the nuclear A and Z parameters in NNUC(2) and
                   NPROT(2) respectively. Furthermore,
                   if A$\leq 16$, enter the root mean squared charge radius
                   $<r^2_{ch}>^{1/2}$ in RO1(2,1), otherwise specify 
                   the parameters $r_0$ and $C$ for the 
                   nucleon density function in RO1(2,1) and RO1(2,2). 
\end{description}

\item[WIN:] specifies the collision energy.  More specifically,
     for FRAME='FIXT', WIN is the laboratory momentum of the beam particle 
         (per nucleon) in GeV/c;  
     for FRAME='CMS', WIN is the (nucleon-nucleon) CMS energy in GeV.

\item[Remark:] FRITIOF reinitialises itself whenever the arguments
      in FREVENT is changed.  This feature allows the possibility to mix 
      different types of collisions in one single run.    
      The character variables FRAME, BEAM and TARGET are case insensitive.  

\end{description}
\end{itemize}

The switch controls and adjustable parameters are contained in the common
block FRPARA1.  These input parameters are provided with sensible default
values.  The physics presented in section 2 is installed 
in the program as the default choices.  Numerous options are also 
included to make the program more adaptive and flexible.

\begin{itemize}
\item{PARAMETER (KSZ1=20) \\
      COMMON/FRPARA1/KFR(KSZ1), VFR(KSZ1)}

\begin{description}
\item[KFR(1)] (D=1)
       Fragmentation
      \begin{description}
       \item[=0] Off.
       \item[=1] On.   
      \end{description}
\item[KFR(2)] (D=1) 
       Multiple gluon emission (dipole radiation) 
      \begin{description}
       \item[=0] Off.
       \item[=1] On.
      \end{description}
\item[KFR(3)] (D=0) Event selection for collisions with a nucleus 
      \begin{description}
       \item[=0] Generate minimum bias events (all interactions recorded).
       \item[=1] Generate only events with all projectile nucleons
                  participated.
       \item[=2] Generate only events with impact parameter between
                  $b_{min}$=VFR(1) and $b_{max}$=VFR(2).
       \item[=3] Apply both requirements in 1 and 2.
      \end{description}
\item[KFR(4)] (D=1)
       Fermi motion in nuclei 
      \begin{description}
       \item[=0] Neglected.
       \item[=1] Included.
      \end{description}
\item[KFR(5)] (D=0) 
       Nucleon-nucleon overlap function 
      \begin{description}
       \item[=0] Eikonal.
       \item[=1] Gaussian.
       \item[=2] Gray disc.
      \end{description}
\item[KFR(6)] (D=2)
       Target Nucleus deformation
      \begin{description}
       \item[=0] No deformation.
       \item[=1] Deformed target nucleus.
       \item[=2] Apply deformation only if the target atomic number $A\geq 80$.  
      \end{description}
\item[KFR(7)] (D=1) 
       Rutherford parton scattering processes
      \begin{description}
       \item[=0] Off.
       \item[=1] On. Here only the hardest RPS is used in FRITIOF.    
       \item[=2] On. The full multiple hard scattering scenario of PYTHIA
               is used.
      \end{description}
\item[KFR(8)] (D=1) 
      Hard gluons cause a corner (soft gluon kink) on the string
      \begin{description}
       \item[=0] No kink is formed. 
       \item[=1] Gluon kink is formed.
      \end{description}
\item[KFR(9)] (D=1)
        `Drowning' of Rutherford parton scattering 
      \begin{description}
       \item[=0] Off. Accept all RPS events.    
       \item[=-1] On. Throw away the drowned RPS event completely 
                  and replace it by a purely soft event.
       \item[=1] As in -1, but the 
            transverse momentum transfer of the soft collision is superimposed
            by the $q_T$ of the drowned RPS.  
      \end{description}
\item[KFR(10)] (D=1) 
        SRM parameters in RPS events: $\mu_1=\mu_0/r$, $\mu_2=\mu_0/(1-r)$
      \begin{description}
      \item[=0] $\mu$ remains the same as in a soft event: $\mu_1=\mu_2=\mu_0$.  
      \item[=1] $r$\,=\,VFR(16). 
      \item[=2] $r$ takes a uniform distribution in (0,1).
      \end{description}
\item[KFR(11)] (D=4)
      Write out of a message when the arguments in FREVENT is changed.  
      \begin{description}
      \item[=-1] Write it out every time the change occurs.    
      \item[=$n$ ($n\geq 0$)] The write out is limited to $n$ times.    
      \end{description}
\item[KFR(12)] (D=2) 
        Set up of the dipole cascade and string fragmentation parameters.
      \begin{description}
      \item[=0] No set up.  The default values are used. 
      \item[=1] Set to the values optimised by 
       OPAL collaboration \cite{opal}:
       VAR(1)=0.20, VAR(3)=1.0, PARJ(21)=0.37, PARJ(41)=0.18, PARJ(42)=0.34.  
      \item[=2] Set to the values optimised by DELPHI collaboration:
       VAR(1)=0.22, VAR(3)=0.6, PARJ(21)=0.405, PARJ(41)=0.23, PARJ(42)=0.34.  
       \end{description}
\item[KFR(13)] (D=0) 
       Compresses the event record to save space in LUJETS.  This switch
       is particularly needed for heavy ion collisions at high energy
       where LUJETS must be compressed before it gets overfilled.  
      \begin{description}
      \item[=0] Do not compress LUJETS.  
      \item[=1-3] LUEDIT(KFR(13)) is called and LUJETS is compressed. 
        Specifically, for KFR(13)=1 fragmented jets and 
        decayed particles are removed, for KFR(13)=2 neutrinos and
        unknown particles are also removed, and for KFR(13)=3
        neutral particles are further excluded.
      \item[=4] A dummy subroutine FREDITD() is provided as an interface
        in which a user may write his own special purpose codes to edit
        and compress LUJETS.    
      \end{description}
\item[KFR(14)] (D=0) 
      If set to 1, the outcome of each event will be checked for  
      charge and energy-momentum conservation.
%%
%%%%%%%%%%%%%%%%%%%%%%%%%%%%%%%%%
\vskip 12pt

\item[VFR(1)] (D=0.0 fm)
       Minimum impact parameter for options KFR(3)=2 or 3.
\item[VFR(2)] (D=0.2 fm)
       Maximum impact parameter for options KFR(3)=2 or 3.
\item[VFR(3)] (D=0.8 fm)
       The minimum allowable distance $R_{min}$ between nucleons in a nucleus.  
\item[VFR(4-5)] (D=0.2, 0.1)
       Dipole and quadrupole deformation coefficients for deformed target
       nucleus.
\item[VFR(6)] (D=0.01 GeV$^2/c^2$)
       The $<Q_T^2>$ for the Gaussian distribution of soft transverse 
       momentum transfer.
\item[VFR(7)] (D=0.30 GeV$^2/c^2$)
       The $<Q^2_{2T}>$ for the Gaussian distribution of primordial transverse
       momenta on the string ends.       
\item[VFR(8)] (D=0.75 GeV)
       Soft radiation coherence parameter $\mu_0$ for projectile hadron or
       nucleon. 
\item[VFR(9)] (D=0.75 GeV)
       Soft radiation coherence parameter $\mu_0$ for target hadron or nucleon. 
\item[VFR(10-11)] (D=0.0, 0.0 mb)
       Projectile-target nucleon total and elastic cross 
       sections, respectively.
       By default, they are taken from the parametrization 
       of Block and Cahn \cite{block} (MSTP(31)=5 in PYTHIA).  
       The meson-nucleon cross sections are obtained
       simply by scaling down the Block-Cahn fit. The scale factor is
       $(2/3-a/\sqrt s)$, where $a=1.13$ GeV for pions and $a=3.27$ GeV for kaons
       are chosen to reproduce the low energy experimental data. For
       all the other baryons, it is treated as a pion if it is a meson
       and it is treated as a proton if it is a baryon.
       User may override the default by setting VFR(17-18) to positive values.
       However, the user assigned cross sections will only affect the
       N-N interaction probability in nucleus collisions. The
       probability for Rutherford parton scattering is not affected.  
\item[VFR(12)] (D=1.0 GeV/$c$)
       The $q_{Tmin}$ for Rutherford parton scattering.  
\item[VFR(13-15)] (D=1/6, 1/3, 1/2)
       The probabilities for assigning various spins and flavours to the diquark
       end of the string.  For example in a proton, VFR(13-15) are the
       probabilities of finding a $ud$ diquark of spin 1,
       a $uu$ diquark of spin 1, and a $ud$ diquark of spin 0, respectively. 
\item[VFR(16)] (D=0.5)
       The fraction $r$ in option KFR(10)=1. 

\item[Remark: ] More variations in the physics controls can be accessed 
      by directly setting up PYTHIA parameters.  For example MSTP(33)
      sets up a $K$ factor in Rutherford parton scattering cross sections, 
      MSTP(51) offers various choices of structure functions, and 
      MSTP(82) selects different scenarios for multiple interactions.  
\end{description}
\end{itemize} 




The event generated must be stored and be accessible to users for event  
analysis.  This purpose is served in FRITIOF by adopting the JETSET 
common block LUJETS.  This allows the full advantage of an easy access to
a large number of very useful JETSET event analysis subroutines.  The documentation on JETSET \cite{j73} should
be consulted if the user is not already familiar with LUJETS and the
event analysis routines built around it.

\begin{itemize}
\item{PARAMETER (KSZJ=4000) \\
      COMMON/LUJETS/N,K(KSZJ,5),P(KSZJ,5),V(KSZJ,5)}

\begin{description}
\item[Remark: ] briefly, N gives the number of lines in the K, P and V matrices
            which are filled in the current event.  K matrix contains the
            particle status codes and history information.  P matrix gives
            the momentum, energy and mass of the particle.  V matrix contains
            information about the production vertex.  Refer to 
            JETSET manuel \cite{j73} for detailed explanation. 
            There is one non-standard code in FRITIOF for nuclear spectator, 
            as explained in the following.  
 
\item[Codes for nuclear spectators: ]  Spectator fragmentation is not included
            in the current version.  However the total momentum of the 
            spectators from either the projectile or target is included and
            occupies one line in LUJETS.  This line is given a special code\\
            K(J,2) = $\pm (10000+N_p)$, \\
            where the `$+$' sign denotes for projectile 
            and `$-$' sign for target,
            and $N_p$ is the number of the protons in the spectator.
            The P matrix provides information on the total momentum,
            energy, and mass of the projectile/target spectator nucleons.
            Note that because this is not a standard code in JETSET, when
            an event is listed with JETSET routine LULIST, this line will
            appear without a character name, and certain JETSET functions
            such as PLU(J,6), which supposedly gives the charge of a particle,
            will not work on this line.  Special treatment should be
            applied to this line when the spectator is included in the
            event analysis.   

\item[Remark on the memory size: ] The size of LUJETS is specified in
            FRITIOF by a parameter KSZJ.  Its current value of 4000 may not
            be sufficient for storing the products of heavy ion collisions of
            a few 100's A GeV in energy.  
            One then need to expand LUJETS to a larger size.  The system
            parameters must be consulted to make sure the size does not
            exceed the system limit.  
            When making the change,
            corresponding changes to LUJETS must be made in ALL companion
            program packages (JETSET, PYTHIA), 
            and in addition new value must be given to the
            parameters MSTU(4) in JETSET.  

\end{description}
\end{itemize} 

\subsection{Other common blocks}

A few other common blocks exist in the program which contain some
relatively useful information.  These common blocks are listed in the following.  

\begin{itemize}
\item{COMMON/FRCODES/IPT(2),PACD(27),NNUC(27),NPROT(27),KCD(27),\\
      RO1(27,2),EXMA(9,2)}

\begin{description}
\item[Remark: ] Here it stores a list of particles/nuclei
            available in FRITIOF and their relevant data.  It is possible
            for users to adjust the nuclear density parameters and the
            minimum and diffractive excitation mass parameters through this
            common block.
\item[IPT(L): ] a pointer which gives the location of the selected
                projectile (L=1) and target (L=2) particles on the PACD array.
                For example, if user selects O+Au collision, IPT(L=1,2) will
                then have values 15 and 25.
\item[PACD(I): ] The default values are given below in order of I from 1 to 27:
\begin{verbatim}
'NEW1','NEW2','PI+ ','PI- ','K+  ','K-  ','N   ','P   ','PBAR',
'D   ','HE  ','BE  ','B   ','C   ','O   ','AL  ','SI  ','S   ',
'AR  ','CA  ','CU  ','AG  ','XE  ','W   ','AU  ','PB  ','U   '.
\end{verbatim}
\item[NNUC(I): ] gives A, the number of nucleons in the particle/nucleus
         corresponding to PACD(I).  Note for all the hadrons including
         the mesons, this number is 1.  NNUC(I=3-27) are provided with
         default values.  NNUC(1-2) must be entered by user if `NEW1' or
         `NEW2 'is selected as the BEAM/TARGET in subroutine FREVENT. 
\item[NPROT(I): ] gives Z, the number of protons or the charge of 
         the nucleus/particle corresponding to PACD(I). NPROT(1-2) must
         be entered by user if `NEW1' or `NEW2' is selected in FREVENT.
\item[KCD(I): ] gives the KF code \cite{datag} of the particle in PACD(I). 
         Default values are \\
         0,    0,    211,  -211,   321,   -321, 2112,  2212, -2212,  
         0 (I$\geq$ 10). \\
         KCD(I) is a irrelevant number when corresponding to a nucleus.
         It becomes necessary for users to enter values for 
         KCD(1-2) only when `NEW1' or `NEW2' selected as the BEAM/TARGET is a
         hadron.
\item[RO1(I,L): ] gives the parameters of the nuclear density distribution
         for the corresponding nucleus PACD(I).  
         RO1 is irrelevant when corresponding to hadrons.  For nuclei
         with $A\leq 16$, RO1(I,1) is the root-mean-squared radius of
         the nuclear charge distribution, and RO(I,2) is irrelevant.
         For nuclei with $A> 16$, RO(I,1) and RO(I,2) give the
         radius parameter $r_0$ and the edge thickness parameter $C$ 
         for the Wood-Saxon nucleon density distribution.  The following
         values are used as default: \\
         RO1(I,1) (I=1,27) =  0.,    0.,    0.,    0.,     0.,    0., 
                0.,    0.,
                0., 2.095,  1.74, 2.519,  2.37, 2.446, 2.724,   1.01,  
             1.014, 1.027, 1.045, 1.045, 1.076, 1.101, 1.108,  1.118,  
             1.120, 1.122, 1.125; \\
         RO1(I,2) (I=1,27) = 0.,    0.,    0.,    0.,     0.,    0.,  
                 0.,    0.,
                 0.,    0.,    0.,    0.,     0.,    0.,    0., 0.478, 
             0.480, 0.490, 0.490, 0.490,  0.490, 0.495,  0.52, 0.530, 
             0.540, 0.545,  0.55. \\
         Note the numbers given are in fermi.  
         Again when user selects a new nucleus through NEW1 or NEW2 as the
         BEAM/TARGET, he must enter the corresponding nucleon density
         parameters in RO1.  
\item[EXMA(I,L): ] gives the parameters for the minimum excitation mass (L=1)
         and the diffractive excitation mass (L=2) corresponding to the 
         first nine hadrons on the list of  
         PACD(I).  Default values (in GeV) are as follows. \\
     EXMA(I,1) (I=1,9) = 0., 0., 0.14, 0.14, 0.50, 0.50, 0.94, 0.94, 0.94;\\  
     EXMA(I,2) (I=1,9) = 0., 0., 0.40, 0.40, 0.75, 0.75, 1.20, 1.20, 1.20.\\
         In the program, the minimum excitation mass is enforced to be
         larger than the initial hadron mass.
\end{description}
\end{itemize}

The following common blocks are used for transmitting data internally.
While they can be accessed to read out information, they should never be
used to input data. 

\begin{itemize}
\item{PARAMETER (KSZ1=20) \\
      COMMON/FRINTN0/PLI0(2,4),AOP(KSZ1),IOP(KSZ1),NFR(KSZ1)}

\begin{description}

\item[PLI0(L,J): ] initial momenta of the colliding hadron/nucleon. \\
          L=1,2 stands for projectile and target, respectively;
          J=1,4 give light-cone components of the momentum $P_x$, $P_y$,
                $P_-$ and $P_+$.
\item[AOP(I): ] provides some information for the current event. 
\begin{description}
  \item[I=1: ] the total centre-of-mass energy of initial 
        projectile-nucleon collision system.
  \item[I=2: ] impact parameter (in fermi) of the collision. 
  \item[I=3-4: ] the parameters $\Omega_0$ and $\beta$ for the eikonal overlap function.
  \item[I=5-6: ] the parameters $g_0$ and $\gamma$ for the Gaussian overlap function.
  \item[I=7-8: ] the parameters $R$ and $\alpha$ for the gray disc overlap function.
  \item[I=9-10: ] the minimum excitation masses for projectile and target
                  respectively.
  \item[I=11-12: ] the ``diffractive'' excitation masses for projectile and 
            target respectively.
\end{description}

\item[IOP(I): ] provides some statistics for the current event. 
\begin{description}
  \item[I=1: ]  the current subcollision number. 
  \item[I=2: ]  number of Nucleon-Nucleon subcollisions.
  \item[I=3: ]  number of projectile nucleons. For mesons this is set to 1.
  \item[I=4: ]  number of projectile protons. For hadrons this is its charge. 
  \item[I=5: ]  as IOP(3) but for target.
  \item[I=6: ]  as IOP(4) but for target.
  \item[I=7: ]  the KF code for projectile.  For nuclei IOP(7)=0.
  \item[I=8: ]  as IOP(7) but for target.
  \item[I=9: ]  number of wounded projectile nucleons.
  \item[I=10:]  number of wounded target nucleons.
  \item[I=11:]  number of projectile spectator protons.
  \item[I=12:]  number of target spectator protons.
  \item[I=13:]  number of N-N subcollisions containing at least one hard
                scattering.
  \item[I=14:]  total number hard scatterings in the event.  
                Note IOP(14) would differ from IOP(13) only when
                multiple Rutherford parton scattering is allowed (KFR(7)=2).
  \item[I=15-18:]  used for internal flags.
\end{description}

\item[NFR(I): ] provides some statistics for all the events in aggregate.  
\begin{description}
  \item[I=1: ] total number of calls to FRINGEB (the current event number).
  \item[I=2: ] total number of calls to FRHARDP (the number of attempts to
        generate Rutherford parton scattering).
  \item[I=3: ] total number of N-N collisions.  
  \item[I=4: ] total number of N-N collisions containing at least one
               hard scattering.  
  \item[I=5: ] total number of A-A collisions containing at least one 
               hard scattering. 
\end{description}

\end{description}


\item{PARAMETER (KSZ2=300) \\
      COMMON/FRINTN1/PPS(2,KSZ2,5),PPH(2,KSZ2,5),PPSY(2,KSZ2,5),\\
                     PPA(2,5)}
\begin{description}
\item[Remark: ] what stores here is the energy-momenta of the wounded
         nucleons.  PPSY gives the total momentum of the system, PPS
         gives the soft momentum, and PPH gives the hard momentum 
         (the total momentum of the hard scattered partons).  
\item[PPS(L,I,J):] for the soft remnants, \\ 
      L=1,2 - index to label the projectile (L=1) and target (L=2);\\
      I=1,2,3,... - index to label the wounded nucleon;\\
      J=1,2,3,4,5 - stands for $P_x$, $P_y$, $P_-$, $P_+$ and invariant mass,
                    respectively.
\item[PPH(L,I,J):]  gives the total momenta of the hard scattered partons
      in each nucleon.  Here the index L, I and J have the same meaning
      as in PPS.
\item[PPSY(L,I,J):]  gives the momentum for each of the nucleon systems,
      which is the sum of PPS(L,I,J) and PPH(L,I,J).  The indices have
      the same meaning as in PPS.  
\item[PPA(L,J):] gives momenta of the nuclei spectators.  \\      
      L=1,2 - index to label the projectile spectator (L=1) and target 
              spectator (L=2); \\
      J=1,2,3,4,5 - stands for $P_x$, $P_y$, $P_-$, $P_+$ and mass, respectively.
\end{description}


\item{PARAMETER (KSZ2=300) \\
      COMMON/FRINTN2/NHP(2),IHQP(2,KSZ2),KHP(2,KSZ2,100,5),\\
                     PHP(2,KSZ2,100,5)}
\begin{description}
\item[Remark: ] this common block records the hard scattered partons in one
                event.

\item[NHP(L): ] gives the total number of hard partons in projectile (L=1)
        and target nucleus (L=2).
\item[IHQP(L,I):] the total number of hard partons in the I-th wounded
      nucleon.\\ 
      L=1,2 - index to label the projectile (L=1) and target (L=2); \\
      I=1,2,3,... - index to label the wounded nucleon.
\item[KHP(L,I,K,J): ] with the index J=1,...5, KHP duplicates the parton 
      status codes of K$(I^\prime, J)$ in LUJETS.  The extra index L and I
      have the same meaning as in IHQP.  K=1,2,...IHQP(L,I) labels the
      hard partons. \\
\item[PHP(L,I,K,J): ] records momenta of the hard partons.  With index
      J=1,...5 PHP duplicates the momentum matrix P$(I^\prime, J)$ in LUJETS.
      Indices L, I and K have the same meaning as in KHP.
\end{description}

\item{PARAMETER (KSZ2=300) \\
      COMMON/FRINTN3/IDN(2,KSZ2),FMN(2,KSZ2),NUC(2,3000)}
\begin{description}
\item[IDN(L,I), FMN(L,I): ] IDN stores the KF codes for nucleons, and FMN
      gives the corresponding masses.  As usual, L=1, 2 labels a
      projectile or a target.  I=1,2,3,... labels the nucleons in the nucleus.
\item[NUC(L,IC): ] NUC(1,IC) and NUC(2,IC) give the identification numbers 
      of the two nucleons involved in the IC-th subcollision.
\item[Remark: ] FMN(L,I) can become different from the nucleon rest mass
      when the nucleon Fermi motion is included.    
\end{description}

%%%%\item{PARAMETER (KSZ2=300) \\
%%%%      COMMON/FRINTN4/KFEND(2,KSZ2,2) } \\
%%%%Here it stores the flavor codes for the string ends.

\end{itemize} 

Finally there are a few common blocks taken from JETSET, PYTHIA and ARIADNE.
They are used mainly for setting control parameters and transmitting relevant
data.  Their special usage in FRITIOF is listed below.

\begin{itemize} 
\item{COMMON/LUDAT1/MSTU(200),PARU(200),MSTJ(200),PARJ(200)}\\
%%
FRITIOF shares MSTU(11) as the designated file number to which all 
the write out of the program is directed.  It is the responsibility of 
the user to make this file available when writing his main program. \\ 
%%

\item{COMMON/AROPTN/IAR(10), KAR(10), VAR(10)}\\
This common block is used to set up parameters for ARIADNE.  In particular, 
KAR(5) and KAR(9) have been set to zero in FRITIOF, which means 
gluon splitting into quarks and photon radiation are forbidden 
in the cascading process.  These effects are insignificant and ignoring them
simplifies the treatment in FRITIOF.  
%%
\end{itemize} 



\subsection{Other subroutines}

All the subroutines in FRITIOF are named with seven characters, 
in contrast to the functions which are named with five characters.    
The supporting subroutines are listed here for completeness.
The purpose of each routine is briefly stated.  

\begin{itemize} 
\item  SUBROUTINE FRINITA(CFRAME,CBEAM,CTARG,WIN) \\
  identifies the particles involved and makes initialisations.  It also
  writes out information about the selected collision and the set up
  of control parameters.  

\item  SUBROUTINE FRHILDN \\
  randomly orders the neutrons and protons in a nucleus, and then sets the
  particle codes and masses and fills out the arrays
  IDN and FMN in common block FRINTN3.

\item  SUBROUTINE FRINGEB \\
  administrates the generation of one event. 
  First the routine FRANGAN is called to generate the nuclear 
  geometry. Then FRRINGO is called to generate momentum transfers 
  in the sequence of nucleon-nucleon subcollisions.  After that,   
  FRTORST is called, which prepares parton configurations on a string 
  for dipole radiation.  Finally LUEXEC is called for fragmentation.

\item  SUBROUTINE FRANGAN \\
  administrates the generation of nuclear collision geometry.

\item  SUBROUTINE FRPPCOL \\
 sets certain codes for hadron-hadron collisions.

\item  SUBROUTINE FRPACOL \\
 determines a set of wounded nucleons for hadron-nucleus collision.

\item  SUBROUTINE FRAACOL \\
 works out the collision geometry for nucleus-nucleus collision.

\item  SUBROUTINE FRNUCOR(L,NA,RMIN2,FMM,RMM,COR) \\
 determines the nucleon coordinates inside a nucleus (L=1 for projectile
 and L=2 for target) according to an appropriate nucleon density distribution.

\item  SUBROUTINE FRNUCOD(NA,RMIN2,A0,A2,A4,RMAX3,COR) \\
 determines the nucleon coordinates inside a deformed nucleus.

\item  SUBROUTINE FRNUCDF(NA,A0,A2,A4,RMAX3) \\
 calculates parameters of nucleon density distribution of a deformed nucleus.

\item  SUBROUTINE FRSEARC(L,fmax,xmin) \\
 searches for the maximum value of the nucleon density distribution
 and the range of the density function beyond which the 
 nucleon density can be neglected.

\item  SUBROUTINE FROVLAP \\
 determines the parameters of the nucleon overlap function by fitting
 the total and inelastic cross sections.

\item  SUBROUTINE FRRINGO \\
 administrates the generation of hard partons and soft momentum
 transfers.  It also calculates the new momenta and masses for the 
 nucleons after the collisions.  

\item  SUBROUTINE FRPSOFT(I,PJK,DP,DN,KFEL) \\
 generates the soft longitudinal momentum transfer for a N-N collision.

\item  SUBROUTINE FRPLIMT(DWP,DWM,DA,DB,DPLO3,DPHI3,DPLO4,DPHI4) \\
 returns the upper and lower limits for the phase space of the 
 momentum transfers.

\item SUBROUTINE FRCHEXG(*,I) \\
 administrates the comparison of the RPS with the ensuing bremsstrahlung
 and removes the RPS from the record if it is ``drowned''. 

\item  SUBROUTINE FRFILHD(I,IQ,KFEL) \\
 evaluates the mass of the nucleon system and stores information of the
 hard partons to common block FRINTN2.

\item  SUBROUTINE FRSETDM(I,KFEL,IQ) \\
 resets an excited nucleon state into its ground state if its mass is lower
 than the diffractive mass.
 
\item  SUBROUTINE FRVECTC(I, IQ, DP) \\
 transfers four vectors from one to another.

\item  SUBROUTINE FRCOLPT(PK2M,PX,PY,PX0,PY0) \\
 generates a soft transverse momentum exchange for N-N collision according
 to a Gaussian distribution.

\item  SUBROUTINE FRHELGE \\
 generates Fermi motion to all the nucleons and 
 calculates the total momenta of the spectator nuclei. 

\item  SUBROUTINE FRTORST(L,J) \\
 administrates the set up of string configurations.

\item  SUBROUTINE FRBELEO(IFLA,IFLB,KF) \\
 gives spins and quark flavours to the string ends.

\item  SUBROUTINE FRANGUR(L,J) \\
 adds a diffractive particle to the event record LUJETS.

\item  SUBROUTINE FRATLEO(L,J) \\
 sets up the string configuration and calls ARIADNE for dipole radiation.

\item  SUBROUTINE FRTESTG(L,IQQK,IQGL,IOK,RFA) \\
 compares the invariant $\ipt$ of the hard partons and the bremsstrahlung
 gluons.  

\item  SUBROUTINE FRARIAD  {\it (used in version 7.02 only)} \\
 picks out the relevant strings for dipole emission through ARIADNE.   

\item  SUBROUTINE FRMXGPT(N1,N2,IMX,VRPTNX) {\it (used in version 7.02 only)} \\
 finds the gluons with the largest and the next largest invariant $P_T$. 

\item  SUBROUTINE FRINSET(NA,N1,N2,NOK,IQ) \\
 inserts a hard gluon onto the string configuration.  

\item  SUBROUTINE FRINSKK(XF,NKK) \\
 inserts a soft gluon kink onto the string configuration.

\item  SUBROUTINE FRPPART(L,PPSR,DPV1,DPV2) \\
 partitions the soft momentum into two parts, one for
 the leading quark/diquark and one for the trailing quark.

\item  SUBROUTINE FRSAVEN(N1,IQ) \\
 saves or retracts a string configuration.  

\item  SUBROUTINE FRFILHW \\
 adds the colourless particles produced from parton subprocesses, such as the
 vector bosons or Higgs, to the event record LUJETS.  This routine is
 not in use currently since those exotic production processes 
 have not been included in FRITIOF.  

\item  SUBROUTINE FRORDER(L,NS,NE) \\
 orders the partons according to rapidity or $P_T$.

\item  SUBROUTINE FRORD01(P,II,N,IQ) \\
 orders an array in a ascending or descending manner.  

\item  SUBROUTINE FRQPROB(KFI,KFT,IQ) \\
 evaluates various scattering cross sections.   

\item  SUBROUTINE FRHARDP(KFI,KFT,W,IHAV,IQ) \\
 generates hard scattering by calling PYTHIA.

\item  SUBROUTINE FRSETPY(IQ) \\
 sets control parameters for PYTHIA.

\item  SUBROUTINE FRVECRC(JF,L,IQ) \\
 converts four vectors from one array into another.

\item  SUBROUTINE FREDIPY \\
 picks out the outgoing hard partons from the event record of PYTHIA.

\item  SUBROUTINE FRHPLIS \\
 outputs the PYTHIA event record and the hard partons picked out by FREDIPY.
 This routine is called only when error occurs and there seems to be a need
 to visually examine the hard scattering event.

\item  SUBROUTINE FRGAUSS(P,V,PMAX) \\
 generates a value for P according to a Gaussian distribution with the average
 $<P^2> = V$.  The Gaussian is cut off at PMAX.

\item  SUBROUTINE FRBETAV(ID,DBETA,DP) \\
 calculates the beta factor for boosting a pair of momenta
 into their centre-of-momentum frame.

\item  SUBROUTINE FRTOCMS(ID,IQ,DP,DBETAO) \\
 boosts two momenta into their centre-of-momentum frame, or do the reverse.

\item  SUBROUTINE FRPOLAR(DTHE,DPHI,DP) \\
 returns the polar angles about which a system needs to be rotated in order 
 to put a four vector on the $z$-axis.

\item  SUBROUTINE FRROTAR(DTHE,DPHI,IQ,DP) \\
 rotates a pair of momenta by the polar angles DTHE and DPHI.

\item  SUBROUTINE FRROTAY(DTHE,DPV) \\
 rotates coordinates around the $y$-axis by an angle DTHE.

\item  SUBROUTINE FRROTAZ(DPHI,DPV) \\
 rotates coordinates around the $z$-axis by an angle DPHI.

\item  SUBROUTINE FRBOOT1(ID,DPV,DBETA) \\
 boosts a single four vector DPV(4) by a beta factor DBETA(3).

\item  SUBROUTINE FRMGOUT(ID,ILIST,MESG,A,B,C,D,E) \\
 manages the writing out of error messages.

\item  SUBROUTINE FRVALUE(IQ) \\
 writes out the values of the FRITIOF parameters.  For IQ=0, the output
 is written in file MSTU(11), otherwise the output will be written in
 a file called `Oxchk'.   

\item  SUBROUTINE FRPYINI(FRAME,BEAM,TARGET,WIN) \\
\item  SUBROUTINE FRPYXTO \\
 These two routines are taken from PYTHIA  
 with modifications made to accommodate a different Block and Cahn
 parametrization and some different handling in meson-nucleon cross sections. 

\item  SUBROUTINE FRUPCAS(STR)
 converts a character string to uppercase.  

\end{itemize} 

\section{Acknowledgements}

I would like to thank B. Andersson and G. Gustafson.  
Their insights have been
crucial for the development of the FRITIOF model.
I also wish to thank the people who worked on the previous versions
of the program, especially E. Stenlund, U. Petterson and C. Sj\"ogren,
for answering my questions regarding the earlier versions.  
Special thanks also go to T. Sj\"ostrand and L.\ L\"onnblad,
from whom I have benefited through discussions with respect to 
the FRITIOF implementation of the PYTHIA or ARIADNE program.   


\appendix
\newpage
\renewcommand{\thesection}{Appendix \Alph{section}}
\renewcommand{\theequation}{\mbox{A}\arabic{equation}}
\setcounter{equation}{0}

\section{A Sample Main Program}
\begin{verbatim}
C..This program generates a few sample FRITIOF events, and then does
C..histogram for negatively charged particle multiplicity distribution
C..in O+Au collision at 200 GeV/nucleon lab energy.
C..One does not usually need to mix different types of events, 
C..the purpose here is only for illustrating the possible usages.  
C..This routine, loaded together with (FRITIOF_7.02, ARIADNE_4.02r,
C..PYTHIA_5.5 and JETSET_7.3) can be used to test the installation of Fritiof.  
C---------------------------------------------------------------------

      PARAMETER (KSZJ=4000,KSZ1=20)
C...    **** Be sure to check that all the KSZJ's in MAIN, Fritiof,
C...         Jetset and Pythia are identically set      *****
      COMMON/FRPARA1/KFR(KSZ1),VFR(KSZ1)
      COMMON/FRINTN0/PLI0(2,4),AOP(KSZ1),IOP(KSZ1),NFR(KSZ1)
      COMMON/LUDAT1/MSTU(200),PARU(200),MSTJ(200),PARJ(200)
      COMMON/LUJETS/N,K(KSZJ,5),P(KSZJ,5),V(KSZJ,5)
      COMMON/LUDAT3/MDCY(500,3),MDME(2000,2),BRAT(2000),KFDP(2000,5)
      DIMENSION MP(0:300)

C...Open a file to take the write out of the program:
      MSTU(11) = 20
      OPEN(MSTU(11),FILE='test.out',STATUS='unknown')

C:::::::Multiplicity distribution for O+AU collision at 200 GeV ::::::::

C...Forbid the decays of Lambda and K_S0: 
      MDCY(LUCOMP(3122),1) = 0
      MDCY(LUCOMP(310),1) = 0

C...Book spaces for the histogram (or use a histogram package):
      DO 50 J=0,300
50    MP(J) = 0

C...Test 10 events (of course a lot more events are needed realistically):
      NEVENT=10
      NTRIG = 0
      DO 100 I=1, NEVENT

      CALL FREVENT('FIXT','O','AU',200.)

C...Output the event using JETSET routine LULIST:
      IF(I.LE.3) CALL LULIST(1)

C...Edit the event record, remove partons or decayed particles:
      CALL LUEDIT(1)

C...Assume a trigger requiring that the energy in the forward cone
C...(theta < 0.3 degree) must be less than 60% of the total beam energy.
C...Also find out the number of negatively charged particles:
      IQTRIG = 0
      EFWD = 0.
      N_=0
      DO 70 J=1, N
      THETA = PLU(J,14)
C...  (PLU is a JETSET function.  Please refer to the JETSET manual.)
      IF(THETA.LT.0.3) EFWD = EFWD+PLU(J,4)

C...Count the negative particles.  Spectator nuclei, which have codes
C...ABS(K(J,2))=10000+N_proton, must be excluded:
      IF(ABS(K(J,2)).LT.10000) THEN
        IF(PLU(J,6).LT.0.) N_ = N_+1
      ENDIF
70    CONTINUE

      EBEAM = 200.*IOP(3)
      IF(EFWD.LT.0.6*EBEAM) IQTRIG = 1

C...Do histogram:
      IF(IQTRIG.EQ.1) THEN
      NTRIG = NTRIG+1
      MP(N_) = MP(N_)+1
      ENDIF

100   CONTINUE

C...Output the histogram data:  
      WRITE(MSTU(11),500) NEVENT, NTRIG
      DO 200 J=0,300
200   WRITE(MSTU(11),*) J, FLOAT(MP(J))/FLOAT(NTRIG)
500   FORMAT(X,'Number of events:',I4,2x,'Triggered events:',I4)

C...Write out the values of the parameters and some statistics:
      CALL FRVALUE(0)
 		
      CLOSE (MSTU(11))
      END
\end{verbatim}


%%%%\input frbibl.tex 
\newpage
\begin{thebibliography}{99}

\bibitem{fr16} B. Nilsson-Almqvist and E. Stenlund, 
   {\it Comut. Phys. Commun.} {\bf 43} (1987) 387.

\bibitem{bogn} B. Andersson, G. Gustafson and B. Nilsson-Almqvist,
   {\it Nucl. Phys.} {\bf Vol. 34} (1986) 451.

\bibitem{bogns} B. Andersson, S. Garpman, H.-\AA. Gustavsson, 
   G. Gustafson and B. Nilsson-Almqvist, I. Otterlund and E. Stenlund, 
   {\it Phys. Scripta} {\bf B281} (1987) 289.

\bibitem{dipol} G. Gustafson,
   {\it Phys. Lett.} {\bf 175B} (1986) 453.

\bibitem{lstri} B. Andersson, G. Gustafson, G. Ingelman and T. Sj\"ostrand, 
   {\it Phys. Rep.} {\bf 97} (1983) 31.

\bibitem{mjets} C. Albajar, {\it et al.}, 
   {\it Nucl. Phys.} {\bf B309} (1988) 405.


\bibitem{multisj} T. Sj\"ostrand and M. van Zijl,  
   {\it Phys. Rev.} {\bf D36} (1987) 2019. 

\bibitem{pyth} H.-U. Bengtsson and T. Sj\"ostrand, 
   {\it Comput. Phys. Commun.} {\bf 46} (1987) 43 \\
   An updated version of this paper, ``A Manual to The Lund Monte Carlo
   for Hadronic Processes:  PYTHIA version 5.5'', is available upon request
   to the authors.

\bibitem{aria} L.\ L\"onnblad, 
  ``{\tt ARIADNE-3}, A Monte Carlo for QCD Cascades in the Colour Dipole
     Formulation'', Lund preprint LU TP 89-10.

\bibitem{arian} L.\ L\"onnblad, 
  ``{\tt ARIADNE Version 4}'',  preprint DESY 92-046.

\bibitem{exten} B. Andersson, G. Gustafson, L.\ L\"onnblad and U. Petterson,
   {\it Z. Phys.} {\bf C43} (1989) 625.

\bibitem{ngeo} L. Ding and E. Stenlund, 
  ``A Monte Carlo for Nuclear Collision Geometry'', Lund preprint LU TP 89-6.

\bibitem{ohard} B. Andersson, G. Gustafson and B. Nilsson-Almqvist,
  ``A High Energy String Dynamics Model for Hadronic Interactions'',
  Lund preprint LU TP 87-6.

\bibitem{prep1} B. Andersson, G. Gustafson and H. Pi,
  in preparation.

\bibitem{j73} T. Sj\"ostrand, 
   ``A Manual to The Lund Monte Carlo for Jet Fragmentation and
     $e^+e^-$ Physics: JETSET version 7.3'', available upon request
     to the author.

\bibitem{size} R. C. Barrett and D. F. Jackson, 
   {\it Nuclear Sizes and Structure} (Oxford University, New York, 1977).

\bibitem{block} M. M. Block and R. N. Cahn, 
   in {\it Physics Simulations at High Energy}, edited by V. Barger, 
   T. Gottschalk and F. Halzen (World Scientific, Singapore, 1987), p. 89.

\bibitem{datag} Particle Data Group,\\  
   {\it Review of Particle Properties}, 
   {\it Phys. Lett.} {\bf B239} (1990) 1. \\
   Refer to section III.67 for the particle codes.

\bibitem{opal} M. Z. Akrawy {\it et al}, 
   {\it Z. Phys.} {\bf C47} (1990) 505.
     

\end{thebibliography}


\end{document}
